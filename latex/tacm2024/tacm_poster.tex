%%%%%%%%%%%%%%%%%%%%%%%%%%%%%%%%%%%%%%%%%%%
%
% From a template maintained at https://github.com/jamesrobertlloyd/cbl-tikz-poster
%
%%%%%%%%%%%%%%%%%%%%%%%%%%%%%%%%%%%%%%%%%%%

\documentclass[portrait,a0b,final,a4resizeable]{a0poster}
\setlength{\paperwidth}{36in} % A0 width: 46.8in
\setlength{\paperheight}{48in} % A0 width: 46.8in

\usepackage{atbegshi}% http://ctan.org/pkg/atbegshi
\AtBeginDocument{\AtBeginShipoutNext{\AtBeginShipoutDiscard}}
\usepackage{qrcode}
\usepackage{multicol}
\usepackage{enumitem}
\usepackage{mathtools}
%\usepackage{color}
%\usepackage{morefloats}
%\usepackage[pdftex]{graphicx}
%\usepackage{rotating}
\usepackage{amsmath, amsthm, amssymb, bm}
%\usepackage{array}
%\usepackage{booktabs}
\usepackage{multirow}
%\usepackage{hyperref}
\usepackage{pgf-soroban}
\usepackage{svg}
\usepackage{bussproofs}
\usepackage{nicematrix}
\usetikzlibrary{cd,shapes.geometric,arrows,chains,matrix,positioning,scopes,calc,trees}
\tikzstyle{mybox} = [draw=white, rectangle]
%\definecolor{darkblue}{rgb}{0,0.08,0.45}
%\definecolor{blue}{rgb}{0,0,1}
%\usepackage{dsfont}
\usepackage[margin=0.5in]{geometry}
%\usepackage{fp}

\input{include/jlposter.tex}

\usepackage{include/preamble}


% Custom notation
\newcommand{\fdeep}{\vf^{(1:L)}}
\newcommand{\flast}{\vf^{(L)}}
\newcommand{\Jx}{J_{\vx \rightarrow \vy}}
\newcommand{\Jxx}{J_{\vx \rightarrow \vy}(\vx)}
\newcommand{\Jy}{J_{\vy \rightarrow \vx}}
\newcommand{\Jyy}{J_{\vy \rightarrow \vx}(\vy)}
\newcommand{\detJyy}{ \left| J_{\vy \rightarrow \vx}(\vy) \right|}

\newcommand\transpose{{\textrm{\tiny{\sf{T}}}}}
\newcommand{\note}[1]{}
\newcommand{\hlinespace}{~\vspace*{-0.15cm}~\\\hline\\\vspace*{0.15cm}}
\newcommand{\embeddingletter}{g}
\newcommand{\bo}{{\sc bo}}
\newcommand{\agp}{Arc \gp}

\newcommand{\D}{\mathcal{D}}
\newcommand{\X}{\mathbf{X}}
\newcommand{\y}{y}
\newcommand{\data} {\X, \y}
\newcommand{\x}{\mathbf{x}}
\newcommand{\f}{\mathit{f}}

\newcommand{\fx}{ f(\mathbf{x}) }
\newcommand{\U}{\mathcal{U}}
\newcommand{\E}{\mathbf{E}}


\newcommand{\bardist}[0]{\hspace{-0.2cm}}

\newlength{\arrowsize}
\pgfarrowsdeclare{biggertip}{biggertip}{
\setlength{\arrowsize}{10pt}
\addtolength{\arrowsize}{2\pgflinewidth}
\pgfarrowsrightextend{0}
\pgfarrowsleftextend{-5\arrowsize}
}{
\setlength{\arrowsize}{1pt}
\addtolength{\arrowsize}{\pgflinewidth}
\pgfpathmoveto{\pgfpoint{-5\arrowsize}{4\arrowsize}}
\pgfpathlineto{\pgfpointorigin}
\pgfpathlineto{\pgfpoint{-5\arrowsize}{-4\arrowsize}}
\pgfusepathqstroke
}


% Custom commmands.

\def\jointspacing{\vspace{0.3in}}

\def\boxwidth{0.21\columnwidth}
\newcommand{\gpdrawbox}[1]{
\setlength\fboxsep{0pt}
\hspace{-0.36in}
\fbox{\hspace{-4mm}
%\includegraphics[width=\boxwidth]{../figures/deep_draws/deep_gp_sample_layer_#1}
\hspace{-4mm}}}

\newcommand{\mappic}[1]{
%\hspace{-0.05in}\includegraphics[width=\boxwidth]{../../figures/seed-0-map/latent_coord_map_layer_#1}
}

\newcommand{\mappiccon}[1]{
%\hspace{-0.05in}\includegraphics[width=\boxwidth]{../../figures/seed-0-map-connected/latent_coord_map_layer_#1}
}

\newcommand{\spectrumpic}[1]{
%\includegraphics[trim=4.5mm 0mm 4mm 3mm, clip, width=0.44\columnwidth]{../figures/spectrum/layer-#1}
}

\usepackage{dsfont}

\newcommand{\feat}{\vh}
\newcommand{\bs}{\blacksquare}
\newcommand{\ws}{\square}

\tikzset{
  treenode/.style = {shape=rectangle, rounded corners,
  draw, align=center,
  top color=white, bottom color=blue!20},
  root/.style     = {treenode, font=\tiny, bottom color=red!30},
  env/.style      = {treenode, font=\tiny},
  dummy/.style    = {circle,draw}
}


\begin{document}
  \begin{poster}
    \vspace{-0.3cm}
    %%% Header
    \begin{center}
      \begin{pcolumn}{1.03}
        %%% Title
        \begin{minipage}[c][9cm][c]{0.85\textwidth}
          \begin{center}
          {\Huge \textbf{Let's wrap this up! Incremental structured decoding with resource constraints}}\\[10mm]
          {\huge Breandan Considine}
          \end{center}
        \end{minipage}
      \end{pcolumn}
    \end{center}

    \vspace*{-0.5cm}

    \large


    %%%%%%%%%%%%%%%%%%%%%%%%%%%%%%%%%%%%%%%%%%%%%%%%%%%%%%%%%%%%%%%%%%%%%%
    %%% Beginning of Document
    %%%%%%%%%%%%%%%%%%%%%%%%%%%%%%%%%%%%%%%%%%%%%%%%%%%%%%%%%%%%%%%%%%%%%%

    \Large

    \begin{multicols}{2}


      \mysection{Main Idea}

      \vspace*{-1cm}
      \null\hspace*{2.5cm}\begin{minipage}[c]{0.88\columnwidth}
      \renewcommand\labelitemi{$\vcenter{\hbox{\small\bullet}}$}
      \begin{itemize}
%        \item Sampling with rejection is unnecessary if you can map onto a simpler distribution
        \item \phantom{\small{.}} Language models have trouble with single-shot constraint satisfaction
        \item \phantom{\small{.}} Typically solved via rejection sampling or backtracking style decoders
        \item \phantom{\small{.}} We implement an incremental structured decoder for autoregressive LLMs
        \item \phantom{\small{.}} Guarantees monotonic progress and preservation of resource constraints
        \item \phantom{\small{.}} Ensures all valid words are generable and all generable words are valid
      \end{itemize}
      \end{minipage}

\jointspacing

\mysection{Motivation}
\null\hspace*{3cm}\begin{minipage}[c]{0.85\columnwidth}
Suppose we want to force an autoregressive LLM to generate syntactically valid next tokens $P(x_n \mid x_1, \ldots, x_{n-1})$, under certain resource constraints. Here is a concrete example: ``Generate an arithmetic expression with two or more variables in ten or fewer tokens.''. If we sample the partial trajectory,
\begin{center}\texttt{( x + ( y * }\underline{\texttt{(}}\end{center}\\
then we will spend quite a long time rejecting invalid completions, because this trajectory has passed the point of no return. Even though \texttt{(} is a locally valid continuation, we need to avoid this scenario, because we would like a linear sampling delay and to guarantee this, we must avoid backtracking.
\end{minipage}

      \jointspacing

      \mysection{Semiring Parsing}
      \null\hspace*{3cm}\begin{minipage}[c]{0.85\columnwidth}
          Given a CFG, $G: \mathcal{G} = \langle V, \Sigma, P, S\rangle$, in Chomsky Normal Form (CNF), we may construct a recognizer $R_\mathcal{G}: \Sigma^n \rightarrow \mathbb{B}$ for strings $\sigma: \Sigma^n$ as follows. Let $2^V$ be our domain, where $0$ is $\varnothing$, $\oplus$ is $\cup$, and $\otimes$ be defined as:\vspace{1cm}
      \end{minipage}

      \[
        s_1 \otimes s_2 = \{C \mid \langle A, B\rangle \in s_1 \times s_2, (C\rightarrow AB) \in P\}
      \]

      \null\hspace*{3cm}\begin{minipage}[c]{0.85\columnwidth}
If we define $\hat\sigma_r = \{w \mid (w \rightarrow \sigma_r) \in P\}$, then construct a matrix with unit nonterminals on the superdiagonal, $M_0[r+1=c](G, \sigma) = \;\hat\sigma_r$ the fixpoint $M_{i+1} = M_i + M_i^2$ is fully determined by the first diagonal:\vspace{0.5cm}
\end{minipage}

\begin{align*}
\hspace{-0.5cm}\resizebox{.86\columnwidth}{!}{
$M$_0 =
\begin{pNiceMatrix}[xdots/line-style=loosely dotted]
   \varnothing & \hat\sigma_1 & \varnothing & \Cdots & \varnothing \\
   \Vdots      & \Ddots       & \Ddots      & \Ddots & \Vdots\\
               &              &             &        & \varnothing\\
               &              &             &        & \hat\sigma_n \\
   \varnothing & \Cdots       &             &        & \varnothing
\end{pNiceMatrix} \Rightarrow
\begin{pNiceMatrix}[xdots/line-style=loosely dotted]
  \varnothing & \hat\sigma_1 & \Lambda & \Cdots & \varnothing \\
  \Vdots      & \Ddots       & \Ddots  & \Ddots & \Vdots\\
              &              &         &        & \Lambda\\
              &              &         &        & \hat\sigma_n \\
  \varnothing & \Cdots       &         &        & \varnothing
\end{pNiceMatrix} \Rightarrow \ldots \Rightarrow M_\infty =
\begin{pNiceMatrix}[xdots/line-style=loosely dotted]
   \varnothing & \hat\sigma_1 & \Lambda & \Cdots & \Lambda^*_\sigma\\
   \Vdots      & \Ddots       & \Ddots  & \Ddots & \Vdots\\
               &              &         &        & \Lambda\\
               &              &         &        & \hat\sigma_n \\
   \varnothing & \Cdots       &         &        & \varnothing
\end{pNiceMatrix}
}
\end{align*}

\null\hspace*{3cm}\begin{minipage}[c]{0.85\columnwidth}
CFL membership is recognized by $R(G, \sigma) = [S \in \Lambda^*_\sigma] \Leftrightarrow [\sigma \in \mathcal{L}(G)]$.
\end{minipage}

      \jointspacing

      \mysection{Porous Completion}

      \null\hspace*{3cm}\begin{minipage}[c]{0.85\columnwidth}
      Let us consider an example with two holes, $\sigma = 1$ \underline{\hspace{1cm}} \underline{\hspace{1cm}}, with the grammar being $G=\{S\rightarrow N O N, O \rightarrow + \mid \times, N \rightarrow 0 \mid 1\}$. This can be rewritten into CNF as $G'=\{S \rightarrow N L, N \rightarrow 0 \mid 1, O \rightarrow \times \mid +, L \rightarrow O N\}$.\vspace{0.5cm}
      \end{minipage}

      \null\hspace*{2.7cm}\begin{minipage}[c]{\columnwidth}
                      \resizebox{.85\columnwidth}{!}{
                      {\renewcommand{\arraystretch}{1.2}
\begin{tabular}{|c|c|c|c|}
  \hline
  & $2^V$ & $\mathbb{B}^{|V|}$ & $\mathbb{B}^{|V|}\rightarrow\mathbb{B}^{|V|}$\\\hline
  $M_0$ & \begin{pmatrix}
  \phantom{V} & \tiny{\{N\}} &         &             \\
              &              & \{N,O\} &             \\
              &              &         & \{N,O\} \\
              &              &         &
  \end{pmatrix} & \begin{pmatrix}
  \phantom{V} & \ws\bs\ws\ws &              &              \\
              &              & \ws\bs\bs\ws &              \\
              &              &              & \ws\bs\bs\ws \\
              &              &              &
  \end{pmatrix} & \begin{pmatrix}
     \phantom{V} & V_{0, 1} &          &          \\
                 &          & V_{1, 2} &          \\
                 &          &          & V_{2, 3} \\
                 &          &          &
  \end{pmatrix} \\\hline
  $M_1$ & \begin{pmatrix}
  \phantom{V} & \tiny{\{N\}} & \varnothing &         \\
              &              & \{N,O\}     & \{L\}   \\
              &              &             & \{N,O\} \\
              &              &             &
  \end{pmatrix} & \begin{pmatrix}
  \phantom{V} & \ws\bs\ws\ws & \ws\ws\ws\ws &              \\
              &              & \ws\bs\bs\ws & \bs\ws\ws\ws \\
              &              &              & \ws\bs\bs\ws \\
              &              &              &
  \end{pmatrix} & \begin{pmatrix}
  \phantom{V} & V_{0, 1} & V_{0, 2} &          \\
              &          & V_{1, 2} & V_{1, 3} \\
              &          &          & V_{2, 3} \\
              &          &          &
  \end{pmatrix} \\\hline
  $M_\infty$ & \begin{pmatrix}
  \phantom{V} & \tiny{\{N\}} & \varnothing & \{S\}   \\
              &              & \{N,O\}     & \{L\}   \\
              &              &             & \{N,O\} \\
              &              &             &
  \end{pmatrix} & \begin{pmatrix}
  \phantom{V} & \ws\bs\ws\ws & \ws\ws\ws\ws & \ws\ws\ws\bs \\
              &              & \ws\bs\bs\ws & \bs\ws\ws\ws \\
              &              &              & \ws\bs\bs\ws \\
              &              &              &
  \end{pmatrix} & \begin{pmatrix}
  \phantom{V} & V_{0, 1} & V_{0, 2} & V_{0, 3} \\
              &          & V_{1, 2} & V_{1, 3} \\
              &          &          & V_{2, 3} \\
              &          &          &
  \end{pmatrix}\\\hline
\end{tabular}\\
}
                      }\vspace{0.4cm}
      \end{minipage}

      \null\hspace*{3cm}\begin{minipage}[c]{0.90\columnwidth}
      This procedure decides if $\exists \sigma' \in \mathcal{L}(G) \mid \sigma' \sqsubseteq \sigma$ but forgets provenance.
\vspace{1cm}
      \end{minipage}
      \jointspacing

\pagebreak
      \mysection{Regular Expression Propagation}
      \jointspacing

      \hspace*{2cm}\begin{minipage}[c]{0.90\columnwidth}
Regular expressions that permit union, intersection and concatenation are called generalized regular expressions (GREs). These can be constructed as follows:
\end{minipage}

\vspace{-2cm}
\setlength{\columnseprule}{0pt}
\setlength{\columnsep}{-3cm}
\begin{multicols}{2}
\begin{eqnarray*}
\mathcal{L}(& \varnothing & ) = \varnothing \\
\mathcal{L}(& \varepsilon & ) = \{\varepsilon\} \\
\mathcal{L}(& a           & ) = \{a\}\\
\mathcal{L}(& R\cdot S    & ) = \mathcal{L}(R) \times \mathcal{L}(S)
\end{eqnarray*} \break\vspace{-0.45cm}
\begin{eqnarray*}
\mathcal{L}(& R^*         & ) = \{\varepsilon\} \cup \mathcal{L}(R\cdot R^*)\\
\mathcal{L}(& R\vee S     & ) = \mathcal{L}(R) \cup \mathcal{L}(S)\\
\mathcal{L}(& R\land S    & ) = \mathcal{L}(R) \cap \mathcal{L}(S)\\
\mathcal{L}(& \neg R      & ) = \Sigma^* \setminus \mathcal{L}(R)
\end{eqnarray*}
\end{multicols}

\jointspacing

\hspace*{2cm}\begin{minipage}[c]{0.90\columnwidth}
Finite slices of a CFL are finite and therefore regular a fortiori. Just like sets, bitvectors and other datatypes, we can propagate GREs through a parse chart. Here, the algebra will carry $\text{GRE}^{|V|}$, where $0=[\varepsilon]_{v \in V}$, and $\oplus, \otimes$ are defined:
\end{minipage}

\vspace{-2cm}
\setlength{\columnseprule}{0pt}
\setlength{\columnsep}{-2cm}
\begin{multicols}{2}
\begin{eqnarray*}
\hspace{3cm}s_1\otimes s_2 = \Big[\bigvee_{(v \rightarrow AB) \in P} s_1[A] \cdot s_2[B] \Big]_{v \in V} \\
\end{eqnarray*} \break\vspace{-0.45cm}
\begin{eqnarray*}
\phantom{--}s_1\oplus s_2 = \big[s_1[v] \vee s_2[v]\big]_{v \in V} \\
\end{eqnarray*}
\end{multicols}

\hspace*{2cm}\begin{minipage}[c]{0.90\columnwidth}
Initially, we have $M_0[r+1=c](G, \sigma) = \Sigma$.
Now after computing the fixpoint, when we unpack $\Lambda^*_\sigma[S]$, this will be a GRE recognizing the finite CFL slice.
\end{minipage}

\jointspacing

      \mysection{Brzozowski Differentiation}

      \jointspacing

      \hspace*{2cm}\begin{minipage}[c]{0.90\columnwidth}
Janusz Brzozowski (1964) introduced a derivative operator $\partial_a: \textsc{Reg} \rightarrow \textsc{Reg}$, which slices a given prefix off a language: $\partial_a L = \{b \in \Sigma^* \mid ab \in L\}$. The Brzozowski derivative over a GRE is effectively a normalizing rewrite system:
      \end{minipage}

\vspace{-1cm}
\begin{multicols}{2}
\begin{eqnarray*}
\phantom{--}\partial_a & \varnothing & = \varnothing                                           \\
\phantom{--}\partial_a & \varepsilon & = \varnothing                                           \\
\phantom{--}\partial_a & a           & = \varepsilon                                           \\
\phantom{--}\partial_a & b           & = \varnothing  \text{ for each } a \neq b               \\
\phantom{--}\partial_a & R^*         & = (\partial_x R)\cdot R^*                               \\
\phantom{--}\partial_a & \neg R      & = \neg \partial_a R                                     \\
\phantom{--}\partial_a & R\cdot S    & = (\partial_a R)\cdot S \vee \delta(R)\cdot\partial_a S \\
\phantom{--}\partial_a & R\vee S     & = \partial_a R \vee \partial_a S                        \\
\phantom{--}\partial_a & R\land S    & = \partial_a R \land \partial_a S
\end{eqnarray*} \break\vspace{-0.45cm}
\begin{eqnarray*}
\delta(& \varnothing &)= \varnothing                                      \\
\delta(& \varepsilon &)= \varepsilon                                      \\
\delta(& a           &)= \varnothing                                      \\
\delta(& R^*         &)= \varepsilon                                      \\
\delta(& \neg R      &)= \varepsilon \text{ if } \delta(R) = \varnothing  \\
\delta(& \neg R      &)= \varnothing \text{ if } \delta(R) = \varepsilon  \\
\delta(& R\cdot S    &)= \delta(R) \land \delta(S)                        \\
\delta(& R\vee S     &)= \delta(R) \vee  \delta(S)                        \\
\delta(& R\land S    &)= \delta(R) \land \delta(S)
\end{eqnarray*}
\end{multicols}

    \hspace*{2cm}\begin{minipage}[c]{0.90\columnwidth}
    The key property we care about is, this formulation allows us to sample lazily from language intersections, without first materializing the product automaton.
    \end{minipage}

    \jointspacing

      \mysection{Example: Determinantal Point Processes}

      \jointspacing

      \hspace*{2cm}\begin{minipage}[c]{0.90\columnwidth}
      Consider a time series, $A$, whose points which are not too close nor far apart, and $n \leq \sum_{i=1}^{|A|} \mathbf{1}[A_i = \bs]$. We want to sample the typical set using an LLM.\vspace{0.5cm}
\begin{itemize}[leftmargin=2cm]
\item The words are bitvectors of some length, $T$, i.e., $A = \{\ws, \bs\}^T$
\item Consecutive $\bs$ separated by $\ws^{[a,b]}$, i.e., $B = \ws^*(\bs\ws^{[a, b]})^{[n,\infty)}\{\bs,\epsilon\}\ws^*$
\end{itemize}\vspace{0.5cm}

The DPP language is regular. Let $C$ be an FSA such that $\mathcal{L}(C) = \mathcal{L}(A) \cap \mathcal{L}(B)$. For example, here is the minimal automaton for $T=13, a=3, b=5, n=2$.
      \end{minipage}

\jointspacing

\hspace{5cm}\resizebox{0.7\columnwidth}{!}{
\includegraphics{dpp.png}
}

\jointspacing

      \hspace*{2cm}\begin{minipage}[c]{0.90\columnwidth}
This automaton for $\mathcal{L}(C)$ can grow very large, and we may only need to sample a small sublanguage with distributional support.
Question: Can we incrementally subsample $\mathcal{L}(C)$ while ensuring partial trajectories always lead to acceptance?
\vspace{2cm}
      \end{minipage}

      \jointspacing

    \end{multicols}

    \vspace{-2.2cm}
    \bottombox{
    %% QR code
    %    \hfill\bottomboxlogo{img/kotlin_logo.png}
    % Comment out the line below out to hide logo

    \hspace{1.8cm}
    \begin{minipage}[c][0.1\paperheight][c]{0.18\textwidth}\qrcode[height=2.6in]{https://tidyparse.github.io/} \end{minipage}
    \begin{minipage}[c][0.1\paperheight][c]{0.25\textwidth}\includegraphics[height=2.6in]{../figures/tidyparse_logo.png} \end{minipage}
    \hspace{-4cm}
    \begin{minipage}[c][0.1\paperheight][c]{0.33\textwidth}\includegraphics[height=3in]{../figures/mcgill.png} \end{minipage}
    \hspace{2cm}
    \begin{minipage}[c][0.1\paperheight][c]{0.33\textwidth}\includegraphics[height=3.2in]{../figures/mila.png} \end{minipage}
    %    \hfill\bottomboxlogo{img/mila_mauve.png} % \hfill shifts the logo across so it meets the right hand side margin
    % Note that \bottomboxlogo takes an optional width argument. It defaults to the following:
    % \hfill\bottomboxlogo[width=\textwidth]{<path_to_image_file>}
    % where \textwidth is actually the width of a minipage which is defined in the \bottombox command of
    % betterportaitposter.cls It's a standard \includegraphics command in there, so easy to change if
    % you need to add a border etc.
    }
\end{poster}
\end{document}
