%% For double-blind review submission, w/o CCS and ACM Reference (max submission space)
%\documentclass[sigplan,10pt,review,anonymous]{acmart}
%\settopmatter{printfolios=false,printccs=false,printacmref=false}
%% For double-blind review submission, w/ CCS and ACM Reference
%\documentclass[sigplan,review,anonymous]{acmart}\settopmatter{printfolios=true}
%% For single-blind review submission, w/o CCS and ACM Reference (max submission space)
%\documentclass[sigplan,review]{acmart}\settopmatter{printfolios=true,printccs=false,printacmref=false}
%% For single-blind review submission, w/ CCS and ACM Reference
%\documentclass[sigplan,review]{acmart}\settopmatter{printfolios=true}
%% For final camera-ready submission, w/ required CCS and ACM Reference
\documentclass[sigplan,nonacm]{acmart}\settopmatter{printfolios=false,printccs=false,printacmref=false}

%% Conference information
%% Supplied to authors by publisher for camera-ready submission;
%% use defaults for review submission.
\acmConference[ARRAY'22]{ACM SIGPLAN Conference on Programming Languages}{June 13, 2022}{San Diego, CA, USA}
%\acmYear{2018}
%\acmISBN{} % \acmISBN{978-x-xxxx-xxxx-x/YY/MM}
%\acmDOI{} % \acmDOI{10.1145/nnnnnnn.nnnnnnn}
%\startPage{1}

%% Copyright information
%% Supplied to authors (based on authors' rights management selection;
%% see authors.acm.org) by publisher for camera-ready submission;
%% use 'none' for review submission.
\setcopyright{none}
%\setcopyright{acmcopyright}
%\setcopyright{acmlicensed}
%\setcopyright{rightsretained}
%\copyrightyear{2018}           %% If different from \acmYear

\usepackage{dsfont}
\usepackage{stmaryrd}
\usepackage{colortbl}
\usepackage{hyperref}

%% Bibliography style
\bibliographystyle{acmart}

\usepackage{dsfont}
\usepackage{stmaryrd}
\usepackage{colortbl}
\usepackage{hyperref}

\usepackage{amsmath}
\DeclareMathOperator*{\argmax}{argmax}
\DeclareMathOperator*{\argmin}{argmin}
\usepackage{amssymb}

\usepackage[dvipsnames, table]{xcolor}
\usepackage{textcomp}

% Packages
\usepackage[pdf]{graphviz}
\usepackage{mathrsfs}

\newcommand*\circled[1]{\tikz[baseline=-0.1cm]{
  \node[shape=circle,draw,inner sep=0.48pt] (char) {\fontsize{7}{12}\textsf{#1}};}}

\DeclareMathAlphabet{\mathcal}{OMS}{cmsy}{m}{n}
\usepackage{cancel}
\newcommand\ccancel[2][red]{\renewcommand\CancelColor{\color{#1}}\cancel{#2}}
\newcommand{\nDownarrow}{\ensuremath{\text{ }\cancel{\Downarrow}\text{ }}}
\usepackage{centernot}

\usepackage{pgfplots, pgfplotstable}
\pgfplotsset{compat=1.7}
\usepgfplotslibrary{fillbetween}
\usetikzlibrary{patterns}
\pgfmathdeclarefunction{gauss}{2}{\pgfmathparse{1/(#2*sqrt(2*pi))*exp(-((x-#1)^2)/(2*#2^2))}}
\pgfmathdeclarefunction{nil}{1}{\pgfmathparse{0.001}}

\usepackage{arydshln}
\usepackage{adjustbox}
\usepackage{enumerate}
\usepackage{enumitem}
\usepackage{tikz-cd}
\usetikzlibrary{calc}
\usepackage{amsfonts}
%\usepackage{prooftrees}
\usepackage{bussproofs}
\renewcommand{\sectionautorefname}{\S}
\renewcommand{\subsectionautorefname}{\S}
\usepackage{float}

\usepackage{tikz-3dplot}
\usetikzlibrary{3d}
\usetikzlibrary{calligraphy}
\newif\ifshowcellnumber
\showcellnumbertrue

\usepackage{algorithm}
\usepackage{algpseudocode}
\usepackage{algorithmicx}
\usepackage{sourcecodepro}
\usepackage{tikz-qtree}
\usepackage{amsthm}
\usepackage{bm}
\usetikzlibrary{bayesnet}
\usetikzlibrary{arrows}
\usepackage{subcaption}
\usetikzlibrary{backgrounds}
\usetikzlibrary{tikzmark}

\newcommand{\E}{\mathbb{E}}
\newcommand{\Var}{\mathrm{Var}}
\newcommand{\Cov}{\mathrm{Cov}}

\newcommand{\CompOrder}{\mathcal{O}}
\def\graphspace{\mathbf{G}}
\def\Uniform{\mbox{\rm Uniform}}
\def\Gaussian{\mbox{\rm Gaussian}}
\def\Bernoulli{\mbox{\rm Bernoulli}}
\def\Dirichlet{\mbox{\rm Dirichlet}}

\usepackage{mathtools}% superior to amsmath
\usepackage{tikz}
% Packages
\usepackage{listings}
\DeclareRobustCommand{\hlred}[1]{{\sethlcolor{pink}\hl{#1}}}
\usepackage{fontspec}

\setmonofont[Scale=0.8]{JetBrainsMono}[
  Contextuals={Alternate},
  Path=./font/,
  Extension = .ttf,
  UprightFont=*-Regular,
  BoldFont=*-Bold,
  ItalicFont=*-Italic,
  BoldItalicFont=*-BoldItalic
]

\usepackage[skins,breakable,listings]{tcolorbox}

\lstdefinelanguage{kotlin}{
  comment=[l]{//},
  commentstyle={\color{gray}\ttfamily},
  emph={delegate, filter, firstOrNull, forEach, it, lazy, mapNotNull, println, repeat, assert, with, head, tail, len, return@},
  numberstyle=\noncopyable,
  identifierstyle=\color{black},
  keywords={abstract, actual, as, as?, break, by, class, companion, continue, data, do, dynamic, else, enum, expect, false, final, for, fun, get, if, import, in, infix, interface, internal, is, null, object, open, operator, override, package, private, public, return, sealed, set, super, suspend, this, throw, true, try, catch, typealias, val, var, vararg, when, where, while, tailrec, reified},
  keywordstyle={\bfseries},
  morecomment=[s]{/*}{*/},
  morestring=[b]",
  morestring=[s]{"""*}{*"""},
  ndkeywords={@Deprecated, @JvmField, @JvmName, @JvmOverloads, @JvmStatic, @JvmSynthetic, Array, Byte, Double, Float, Boolean, Int, Integer, Iterable, Long, Runnable, Short, String, int},
  ndkeywordstyle={\bfseries},
  sensitive=true,
  stringstyle={\ttfamily},
  literate={`}{{\char0}}1,
  escapeinside={(*@}{@*)}
}
\lstdefinelanguage{tidy}{
  comment=[l]{//},
  commentstyle={\color{gray}\ttfamily},
  emph={|, ->, ---},
  emphstyle={\color{red}},
  identifierstyle=\color{black},
  keywords={\|, ->, ---},
  otherkeywords={|,->},
  morekeywords={|,->},
  keywordstyle={\color{blue}\bfseries},
  morecomment=[s]{/*}{*/},
  morestring=[b]",
  morestring=[s]{"""*}{*"""},
  ndkeywords={@Deprecated, @JvmField, @JvmName, @JvmOverloads, @JvmStatic, @JvmSynthetic, Array, Byte, Double, Float, Int, Integer, Iterable, Long, Runnable, Short, String},
  ndkeywordstyle={\color{orange}\bfseries},
  sensitive=true,
  stringstyle={\color{green}\ttfamily},
  literate={`}{{\char0}}1
}

%%%%%%%%%%%%%%%%%%%%%%%%%%%%%%%%%%%%%%%%%%%
%
% Color boxes
%
%%%%%%%%%%%%%%%%%%%%%%%%%%%%%%%%%%%%%%%%%%%

\tcbset{
  enhanced jigsaw,
  breakable,
  listing only,
%  boxsep=-1pt,
%  top=-1pt,
  bottom=0.1cm,
  right=0.5cm,
  overlay first={
    \node[black!50] (S) at (frame.south) {\Large\ding{34}};
    \draw[dashed,black!50] (frame.south west) -- (S) -- (frame.south east);
  },
  overlay middle={
    \node[black!50] (S) at (frame.south) {\Large\ding{34}};
    \draw[dashed,black!50] (frame.south west) -- (S) -- (frame.south east);
    \node[black!50] (S) at (frame.north) {\Large\ding{34}};
    \draw[dashed,black!50] (frame.north west) -- (S) -- (frame.north east);
  },
  overlay last={
    \node[black!50] (S) at (frame.north) {\Large\ding{34}};
    \draw[dashed,black!50] (frame.north west) -- (S) -- (frame.north east);
  },
  before={\par\vspace{5pt}},
  after={\par\vspace{\parskip}\noindent}
}

\definecolor{slightgray}{rgb}{0.90, 0.90, 0.90}

\usepackage{soul}
\makeatletter
\def\SOUL@hlpreamble{%
  \setul{}{3.0ex}%
  \let\SOUL@stcolor\SOUL@hlcolor%
  \SOUL@stpreamble%
}
\makeatother

\newcommand{\inline}[1]{%
  \begingroup%
  \sethlcolor{slightgray}%
  \hl{\ttfamily\footnotesize #1}%
  \endgroup
}

\newcommand{\tinline}[1]{%
  \begingroup%
  \sethlcolor{slightgray}%
  \hl{\ttfamily\tiny #1}%
  \endgroup
}

\newtcblisting{halftidyinput}[1][]{%
  left skip=0.7cm,
  width=6cm,
%  left=-0.01cm,
  top=-0.1cm,
  bottom=-0.35cm,
  listing options={
    language=tidy,
    basicstyle=\ttfamily\small,
%numberstyle=\footnotesize,
    showstringspaces=false,
    tabsize=2,
    breaklines=true,
    numbers=none,
    inputencoding=utf8,
    escapeinside={(*@}{@*)},
    #1
  },
  underlay unbroken and first={%
    \path[draw=none] (interior.north west) rectangle node[white]{\includegraphics[width=4mm]{../figures/tidyparse_logo.png}} ([xshift=-10mm,yshift=-7mm]interior.north west);
  }
}

\newtcblisting{wholetidyinput}[1][]{%
  left skip=0.7cm,
  top=0.1cm,
  middle=0mm,
  boxsep=0mm,
  listing options={
    language=tidy,
    basicstyle=\ttfamily\small,
%numberstyle=\footnotesize,
    showstringspaces=false,
    tabsize=2,
    breaklines=true,
    numbers=none,
    inputencoding=utf8,
    escapeinside={(*@}{@*)},
    #1
  },
  underlay unbroken and first={%
      \path[draw=none] (interior.north west) rectangle node[white]{\includegraphics[width=4mm]{../figures/tidyparse_logo.png}} ([xshift=-10mm,yshift=-9mm]interior.north west);
  }
}

\definecolor{A}{RGB}{6,150,104}
\definecolor{B}{RGB}{196,74,137}
\definecolor{C}{RGB}{117,237,133}
\definecolor{D}{RGB}{246,46,243}
\definecolor{E}{RGB}{89,162,12}
\definecolor{F}{RGB}{113,12,158}
\definecolor{G}{RGB}{191,205,142}
\definecolor{H}{RGB}{51,58,158}
\definecolor{I}{RGB}{244,212,3}
\definecolor{J}{RGB}{37,36,249}
\definecolor{K}{RGB}{253,165,71}
\definecolor{L}{RGB}{27,81,29}
\colorlet{LA}{A!30}
\colorlet{LB}{B!30}
\colorlet{LC}{C!30}
\colorlet{LD}{D!30}
\colorlet{LE}{E!30}
\colorlet{LF}{F!30}
\colorlet{LG}{G!30}
\colorlet{LH}{H!30}
\colorlet{LI}{I!30}
\colorlet{LJ}{J!30}
\colorlet{LK}{K!30}
\colorlet{LL}{L!30}
\newcommand{\hiliA}[1]{%
  \colorbox{LA}{$#1$}}
\newcommand{\hiliB}[1]{%
  \colorbox{LB}{$#1$}}
\newcommand{\hiliC}[1]{%
  \colorbox{LC}{$#1$}}
\newcommand{\hiliD}[1]{%
  \colorbox{LD}{$#1$}}
\newcommand{\hiliE}[1]{%
  \colorbox{LE}{$#1$}}
\newcommand{\hiliF}[1]{%
  \colorbox{LF}{$#1$}}
\newcommand{\hiliG}[1]{%
  \colorbox{LG}{$#1$}}
\newcommand{\hiliH}[1]{%
  \colorbox{LH}{$#1$}}
\newcommand{\hiliI}[1]{%
  \colorbox{LI}{$#1$}}
\newcommand{\hiliJ}[1]{%
  \colorbox{LJ}{$#1$}}
\newcommand{\hiliK}[1]{%
  \colorbox{LK}{$#1$}}
\newcommand{\hiliL}[1]{%
  \colorbox{LL}{$#1$}}
\newcommand{\highlight}[1]{%
  \colorbox{lgray}{$#1$}}
\colorlet{lred}{red!30}
\colorlet{lorange}{orange!30}
\colorlet{lgreen}{green!30}
\colorlet{lgray}{black!15}
\colorlet{dgray}{black!75}
\DeclareRobustCommand{\hlred}[1]{{\sethlcolor{lred}\hl{#1}}}
\DeclareRobustCommand{\hlorange}[1]{{\sethlcolor{lorange}\hl{#1}}}
\DeclareRobustCommand{\hlgreen}[1]{{\sethlcolor{lgreen}\hl{#1}}}
\DeclareRobustCommand{\hlgray}[1]{{\sethlcolor{lgray}\hl{#1}}}
\DeclareRobustCommand{\caret}[1]{{\sethlcolor{dgray}\textcolor{white}{\hl{#1}}}}

\usepackage{url}
\usepackage{qtree}

\usepackage{filecontents}
\usepackage{pstricks-add}
\usepackage{emoji}
\usepackage{alltt}
\usepackage{nicematrix}
\usepackage{graphicx}
\usepackage{ulem}
\usepackage{upquote}
\tikzstyle{every picture}+=[remember picture]
\usepackage{menukeys}
\pgfplotstableread[col sep=comma,]{timings_loc.csv}\loctimings
\pgfplotstableread[col sep=comma,]{timings_unloc.csv}\unloctimings

\makeatletter
\DeclareRobustCommand{\cev}[1]{%
  {\mathpalette\do@cev{#1}}%
}
\newcommand{\do@cev}[2]{%
  \vbox{\offinterlineskip
  \sbox\z@{$\m@th#1 x$}%
  \ialign{##\cr
  \hidewidth\reflectbox{$\m@th#1\vec{}\mkern4mu$}\hidewidth\cr
  \noalign{\kern-\ht\z@}
    $\m@th#1#2$\cr
  }%
  }%
}
\makeatother

\makeatletter
\DeclareRobustCommand{\pder}[1]{%
  \@ifnextchar\bgroup{\@pder{#1}}{\@pder{}{#1}}}
\newcommand{\@pder}[2]{\frac{\partial#1}{\partial#2}}
\makeatother

\newcommand{\shup}{\shortuparrow}
\newcommand{\shri}{\shortrightarrow}
\newcommand{\shur}{\shup\hspace{-5pt}\shri}

\makeatletter
\def\squiggly{\bgroup \markoverwith{\textcolor{red}{\lower3\p@\hbox{\sixly \char58}}}\ULon}
\makeatother

\newcommand{\err}[1]{\smash{\squiggly{#1}{}}}
\newcommand{\stirlingii}{\genfrac{\{}{\}}{0pt}{}}

%======== Arrows =========
\newcommand{\knightarrow}{
  \tikz{
    \fill (0pt,0pt) circle [radius = 1pt];
    \fill (0pt,6pt) circle [radius = 1pt];
    \fill (6pt,0pt) circle [radius = 1pt];
    \fill (6pt,6pt) circle [radius = 1pt];
    \fill (12pt,0pt) circle [radius = 1pt];
    \fill (12pt,6pt) circle [radius = 1pt];
    \fill (6pt,0pt) circle [radius = 1pt];
    \fill (12pt,0pt) circle [radius = 1pt];
    \draw [-to] (0pt,0pt) -- (12pt,6pt);
  }
}

\newcommand{\kingarrow}{
  \tikz{
    \fill (0pt,0pt) circle [radius = 1pt];
    \fill (6pt,0pt) circle [radius = 1pt];
    \fill (0pt,6pt) circle [radius = 1pt];
    \fill (6pt,6pt) circle [radius = 1pt];
    \draw [-to] (0pt,0pt) -- (6pt,6pt);
    \draw [-to] (0pt,0pt) -- (0pt,6pt);
    \draw [-to] (0pt,0pt) -- (6pt,0pt);
  }
}

\newcommand{\knightkingarrow}{
  \tikz{
    \fill (0pt,0pt) circle [radius = 1pt];
    \fill (0pt,6pt) circle [radius = 1pt];
    \fill (6pt,0pt) circle [radius = 1pt];
    \fill (6pt,6pt) circle [radius = 1pt];
    \fill (12pt,0pt) circle [radius = 1pt];
    \fill (12pt,6pt) circle [radius = 1pt];
    \draw [-to] (0pt,0pt) -- (6pt,6pt);
    \draw [-to] (0pt,0pt) -- (0pt,6pt);
    \draw [-to] (0pt,0pt) -- (6pt,0pt);
    \draw [-to] (0pt,0pt) -- (12pt,6pt);
  }
}

%======== Arrows =========

\usetikzlibrary{decorations.pathreplacing,automata,calc,positioning,matrix,fit}
\usepackage{wrapfig}

\newcommand{\mkTrellis}[1]{
  \begin{tikzpicture}
    \def\dx{20pt}
    \def\dy{30pt}
    \newcounter{i}
    \stepcounter{i}
    \node[circle, draw, fill=black!30] (\arabic{i}) at (0,0){};
    \foreach [count=\i] \x in {2,...,#1}{
      \pgfmathsetmacro{\lox}{\x-1}%
      \pgfmathsetmacro{\loxt}{\x-3}%
      \foreach [count=\j] \xx in {-\lox,-\loxt,...,\lox}{
        \pgfmathsetmacro{\jj}{\j-1}%
        \stepcounter{i}
        \pgfmathsetmacro{\kk}{\xx-2}%
        \pgfmathsetmacro{\lbl}{\lox!/(\jj!*(\lox-\jj)!)}
        \ifnum\x<\kk
        \pgfmath\node[circle, draw]  (\arabic{i}) at (\xx*\dx, -\lox*\dy) {};
        \else
        \pgfmath\node[circle, draw, fill=black!30]  (\arabic{i}) at (\xx*\dx, -\lox*\dy) {};
        \fi
      }
    }
    \newcounter{z}
    \newcounter{xn}
    \newcounter{xnn}
    \pgfmathsetmacro{\maxx}{#1 - 1}
    \foreach \x in {1,...,\maxx}{
      \foreach \xx in {1,...,\x}{
        \stepcounter{z}
        \setcounter{xn}{\arabic{z}}
        \addtocounter{xn}{\x}
        \setcounter{xnn}{\arabic{xn}}
        \stepcounter{xnn}
        \draw [<-] (\arabic{z}) -- (\arabic{xn});
        \draw [<-] (\arabic{z}) -- (\arabic{xnn});
      }
    }
  \end{tikzpicture}
}

\newcommand{\dx}{20pt}
\newcommand{\dy}{30pt}
\newcounter{i}
\newcounter{z}
\newcounter{xn}
\newcounter{xnn}
\newcommand{\mkTrellisAppend}[1]{
  \begin{tikzpicture}
    \setcounter{i}{0}
    \setcounter{z}{0}
    \setcounter{xn}{0}
    \setcounter{xnn}{0}
    \stepcounter{i}
    \node[circle, draw] (\arabic{i}) at (0,0){};
    \foreach [count=\i] \x in {2,...,#1}{
      \pgfmathsetmacro{\lox}{\x-1}%
      \pgfmathsetmacro{\loxt}{\x-3}%
      \foreach [count=\j] \xx in {-\lox,-\loxt,...,\lox}{
        \pgfmathsetmacro{\jj}{\j-1}%
        \stepcounter{i}
        \pgfmathsetmacro{\kk}{\xx+2}%
        \pgfmathsetmacro{\lbl}{\lox!/(\jj!*(\lox-\jj)!)}
        \ifnum\x>\kk
        \pgfmath\node[circle, draw, fill=black!30]  (\arabic{i}) at (\xx*\dx, -\lox*\dy) {};
        \else
        \pgfmath\node[circle, draw]  (\arabic{i}) at (\xx*\dx, -\lox*\dy) {};
        \fi
      }
    }
    \pgfmathsetmacro{\maxx}{#1 - 1}
    \foreach \x in {1,...,\maxx}{
      \foreach \xx in {1,...,\x}{
        \stepcounter{z}
        \setcounter{xn}{\arabic{z}}
        \addtocounter{xn}{\x}
        \setcounter{xnn}{\arabic{xn}}
        \stepcounter{xnn}
        \draw [<-] (\arabic{z}) -- (\arabic{xn});
        \draw [<-] (\arabic{z}) -- (\arabic{xnn});
      }
    }
  \end{tikzpicture}
}

\newcommand{\mkTrellisInsert}[1]{
  \begin{tikzpicture}
    \setcounter{i}{0}
    \setcounter{z}{0}
    \setcounter{xn}{0}
    \setcounter{xnn}{0}
    \stepcounter{i}
    \node[circle, draw] (\arabic{i}) at (0,0){};
    \foreach [count=\i] \x in {2,...,#1}{
      \pgfmathsetmacro{\lox}{\x-1}%
      \pgfmathsetmacro{\loxt}{\x-3}%
      \foreach [count=\j] \xx in {-\lox,-\loxt,...,\lox}{
        \pgfmathsetmacro{\jj}{\j-1}%
        \stepcounter{i}
        \pgfmathsetmacro{\mp}{\xx+#1}%
        \pgfmathsetmacro{\mq}{\xx+\x}%
        \pgfmathsetmacro{\lbl}{\lox!/(\jj!*(\lox-\jj)!)}
        \ifnum\x>\mp
        \pgfmath\node[circle, draw, fill=black!30]  (\arabic{i}) at (\xx*\dx, -\lox*\dy) {};
        \else
        \ifnum#1<\mq
        \pgfmath\node[circle, draw, fill=black!30]  (\arabic{i}) at (\xx*\dx, -\lox*\dy) {};
        \else
        \pgfmath\node[circle, draw]  (\arabic{i}) at (\xx*\dx, -\lox*\dy) {};
        \fi
        \fi

      }
    }
    \pgfmathsetmacro{\maxx}{#1 - 1}
    \foreach \x in {1,...,\maxx}{
      \foreach \xx in {1,...,\x}{
        \stepcounter{z}
        \setcounter{xn}{\arabic{z}}
        \addtocounter{xn}{\x}
        \setcounter{xnn}{\arabic{xn}}
        \stepcounter{xnn}
        \draw [<-] (\arabic{z}) -- (\arabic{xn});
        \draw [<-] (\arabic{z}) -- (\arabic{xnn});
      }
    }
  \end{tikzpicture}
}

\usetikzlibrary{automata, positioning, arrows}

\newcommand{\nobarfrac}{\genfrac{}{}{0pt}{}}
\pgfplotstableread[col sep=comma,]{timings_loc.csv}\loctimings
\pgfplotstableread[col sep=comma,]{timings_unloc.csv}\unloctimings
\pgfplotstableread[col sep=comma,]{natural_errors.csv}\naturalerrors
\pgfplotstableread[col sep=comma,]{synthetic_errors.csv}\syntheticerrors

\begin{document}

%% Title information
\title{Tidyparse: Real-Time Context Free Error Correction}
\begin{abstract}
Tidyparse is a program synthesizer that performs real-time error correction for context free languages.
Given both an arbitrary context free grammar (CFG) and an invalid string, the tool lazily generates admissible repairs while the author is typing, ranked by Levenshtein edit distance.
 Repairs are guaranteed to be complete, grammatically consistent and minimal.
 Tidyparse is the first system of its kind offering these guarantees in a real-time editor. To accelerate code completion, we design and implement a novel incremental parser-synthesizer that transforms CFGs onto a dynamical system over finite field arithmetic, enabling us to suggest syntax repairs in-between keystrokes. We have released an IDE plugin demonsrating the system described.\footnote{https://plugins.jetbrains.com/plugin/19570-tidyparse}
\end{abstract}

\author{Breandan Mark Considine}
\affiliation{
  \institution{McGill University}
}
\email{bre@ndan.co}

\author{Jin Guo}
\affiliation{
  \institution{McGill University}
}
\email{jguo@cs.mcgill.ca}

\author{Xujie Si}
\affiliation{
  \institution{McGill University}
}
\email{xsi@cs.mcgill.ca}

\maketitle

\section{Introduction}

Modern research on error correction can be traced back to the early days of coding theory, when researchers designed \textit{error-correcting codes} (ECCs) to denoise transmission errors induced by external interference, whether due to collision with a high-energy proton, manipulation by an adversary or some typographical mistake. In this context, \textit{code} can be any logical representation for communicating information between two parties (such as a human and a computer), and an ECC is a carefully-designed code which ensures that even if some portion of the message should be corrupted through accidental or intentional means, one can still recover the original message by solving a linear system of equations. In particular, we frame our work inside the context of errors arising from human factors in computer programming.

In programming, most such errors initially manifest as syntax errors, and though often cosmetic, manual repair can present a significant challenge for novice programmers. The ECC problem may be refined by introducing a language, $\mathcal{L} \subset \Sigma^*$ and considering admissible edits transforming an arbitrary string, $s \in \Sigma^*$ into a string, $s'\in\mathcal{L}$. Known as \textit{error-correcting parsing} (ECP), this problem was well-studied in the early parsing literature, cf. Aho and Peterson~\cite{aho1972minimum}, but fell out of favor for many years, perhaps due to its perceived complexity. By considering only minimal-length edits, ECP can be reduced to the so-called \textit{language edit distance} (LED) problem, recently shown to be subcubic~\cite{bringmann2019truly}, suggesting its possible tractability. Previous results on ECP and LED were primarily of a theorietical nature, but now, thanks to our contributions, we have finally realized a practical prototype.

%String constraints
%
%Word equations

\section{Toy Example}

Suppose we are given the following context free grammar:

\begin{tidyinput}
S -> S and S | S or S | ( S ) | true | false | ! S
\end{tidyinput}

\noindent For reasons that will become clear in the following section, this is automatically rewritten into the equivalent grammar:

\begin{verbatim}
 F.! → !  ε+ → ε      S → false    F.and → and
 F.( → (  ε+ → ε+ ε+  S → F.! S      S.) → S F.)
 F.) → )   S → <S>    S → S or.S    or.S → F.or S
 F.ε → ε   S → true   S → S and.S  and.S → F.and S
F.or → or  S → S ε+   S → F.( S.)
\end{verbatim}

%\noindent We can visualize the CFG as either a graph or a matrix:
%
%\begin{figure}[H]
%    \includegraphics[width=3.5cm]{../figures/bool_arith_cfg_graph.png}
%    \hspace{20pt}
%    \includegraphics[width=3.5cm]{../figures/bool_arith_cfg_mat.bmp}
%\end{figure}

\noindent Given a string containing holes such as the one below, Tidyparse will return several completions in a few milliseconds:

\begin{tidyinput}
true _ _ _ ( false _ ( _ _ _ _ ! _ _ ) _ _ _ _
\end{tidyinput}

\begin{tidyoutput}
true or ! ( false or ( <S> ) or ! <S> ) or <S>
true or ! ( false and ( <S> ) or ! <S> ) or <S>
true or ! ( false and ( <S> ) and ! <S> ) or <S>
true or ! ( false and ( <S> ) and ! <S> ) and <S>
true and ( false and ( <S> ) and ! ! <S> ) and <S>
...
\end{tidyoutput}

\noindent Similarly, if provided with a string containing various errors, Tidyparse will return several suggestions how to fix it, where \hlgreen{green} is insertion, \hlorange{orange} is substitution and \hlred{red} is deletion.

\begin{tidyinput}
true and ( false or and true false
\end{tidyinput}

\begin{tidyoutput}
1.) true and ( false or (*@\hlorange{!}@*) true (*@\hlorange{)}@*)
2.) true and ( false or (*@\hlgreen{<S>}@*) and true (*@\hlorange{)}@*)
3.) true and ( false or (*@\hlorange{(}@*) true (*@\hlorange{)}@*) (*@\hlgreen{)}@*)
...
9.) true and ( false or (*@\hlgreen{!}@*) (*@\hlgreen{<S>}@*) (*@\hlgreen{)}@*) and true (*@\hlred{false} @*)
\end{tidyoutput}

\noindent In the following paper, we will describe how we built it.

\section{Matrix Theory}

We recall that a CFG is a quadruple consisting of terminals, $\Sigma$, nonterminals, $V$, productions, $P: V \rightarrow (V \mid \Sigma)^*$, and the start symbol, $S$. It is a well-known fact that every CFG can be reduced to \textit{Chomsky Normal Form} (CNF), $P^*: V \rightarrow (V^2 \mid \Sigma)$, in which every production takes one of two forms, either $v_0 \rightarrow v_1 v_2$, or $v_0 \rightarrow \sigma$, where $v_{0, 1, 2}: V$ and $\sigma: \Sigma$. For example, we can rewrite the CFG $\{S \rightarrow S S \mid ( S ) \mid ()\}$, into CNF as:\vspace{-10pt}

\[
\{S\rightarrow XR \mid SS \mid LR,\; L \rightarrow (,\; R \rightarrow ),\; X\rightarrow LS\}
\]

\noindent Given a CFG, $\mathcal{G} : \Sigma, \langle V, P, S\rangle$ in CNF, we can construct a recognizer $R_\mathcal{G}: \Sigma^n \rightarrow \mathbb{B}$ for strings $\sigma: \Sigma^n$ as follows. Let $\mathcal P(V)$ be our domain, $0$ be $\varnothing$, $\oplus$ be $\cup$, and $\otimes$ be defined as:\vspace{-10pt}

\begin{align}
a \otimes b := \{C \mid \langle A, B\rangle \in a \times b, (C\rightarrow AB) \in P\}
\end{align}

\noindent We initialize $\mathbf{M}^0_{r,c}(\mathcal{G}, \sigma) \coloneqq \{V \mid c = r + 1, (V \rightarrow \sigma_r) \in P\}$ and search for a matrix $\mathbf{M}^*$ via fixpoint iteration,\vspace{-10pt}

\begin{align}
\mathbf{M}^* = \begin{pNiceMatrix}[nullify-dots,xdots/line-style=loosely dotted]
   \varnothing & \{V\}_{\sigma_1} & \Cdots                  &                            & \mathcal{T} \\
   \Vdots      & \Ddots           & \Ddots[shorten=-0.1cm]  & \phantom{\{V\}_{\sigma_2}} & \Vdots \\
               &                  &                         &                            & \\
               &                  &                         &                            & \{V\}_{\sigma_n} \\
   \varnothing & \Cdots           &                         &                            & \varnothing
\end{pNiceMatrix}\label{eq:fpm}
\end{align}

\noindent where $\mathbf{M}^*$ is the least solution to $\mathbf{M} = \mathbf{M} + \mathbf{M}^2$. We can then define the recognizer as: $S \in \mathcal{T}? \iff \sigma \in \mathcal{L}(\mathcal{G})?$ %\footnote{Strictly speaking, $\mathbf{M} = \mathbf{M} + \mathbf{M}^2$ is only necessary if we need to do fixedpoint iteration. Since we solve for $\mathbf{M}^*$ and $S \in \mathcal{T}$ directly, then unlike Valiant, we can solve for the computationally more efficient fixpoint $\mathbf{M} = \mathbf{M}^2$.}

%While theoretically elegant, this decision procedure can be optimized by lowering onto a rank-3 binary tensor.
Note that $\bigoplus_{k = 1}^n \mathbf{M}_{ik} \otimes \mathbf{M}_{kj}$ has cardinality bounded by $|V|$ and is thus representable as a fixed-length vector using the characteristic function, $\mathds{1}$. In particular, $\oplus, \otimes$ are defined as $\boxplus, \boxtimes$, so that the following diagram commutes:\vspace{-10pt}

\[\begin{tikzcd}[row sep=huge, column sep=huge]
  V \times V \arrow[r, "\oplus/\otimes"] \arrow[d, "\mathds{1}^2"]
  & V \arrow[d, "\mathds{1}\phantom{^{-1}}"] \\
  \mathbb{B}^{|V|} \times \mathbb{B}^{|V|} \arrow[r, "\boxplus/\boxtimes", labels=below] \arrow[u, "\mathds{1}^{-2}"]
  & \mathbb{B}^{|V|} \arrow[u, "\mathds{1}^{-1}"]
\end{tikzcd}\]

%\noindent The compactness of this representation can be improved via a combinatorial number system without loss of generality, although $\mathds{1}$ is a convenient encoding for SAT.

\noindent Full details of the bisimilarity between parsing and matrix multiplication can be found in Valiant~\cite{valiant1975general}, who shows its time complexity to be $\mathcal{O}(n^\omega)$ where $\omega$ is the matrix multiplication bound, and Lee~\cite{lee2002fast}, showing that speedups to Boolean matrix multiplication are realizable by CFL parsers. %Assuming sparsity, this technique is typically linearithmic.%, and is believed to be the most efficient procedure for CFL recognition to date.

\subsection{Sampling k-combinations without replacement}

\noindent Let $\textbf{M}: \text{GF}(2^{n\times n})$ be a matrix whose struture is depicted in Eq.~\ref{eq:lfsr}, where $P$ is a feedback polynomial over $GF(2^n)$ with coefficients $P_{1\ldots n}$ and semiring operators $\oplus := \veebar, \otimes := \land$. Selecting any $V \neq \mathbf{0}$ and coefficients $P_{1\ldots n}$ from a known \textit{primitive polynomial} then powering the matrix $\mathbf{M}$ generates an ergodic sequence over GF$(2^n)$, as shown in Eq.~\ref{eq:ergo}.\vspace{-10pt}

\begin{align}
    \mathbf{M}^tV = \begin{pNiceMatrix}[nullify-dots,xdots/line-style=loosely dotted]
        P_1    & \Cdots &        &       &        & P_n \\
        \top   & \circ  & \Cdots &       &        & \circ \\
        \circ  & \Ddots & \Ddots &       &        & \Vdots \\
        \Vdots & \Ddots &        &       &        & \\
               &        &        &       &        & \\
        \circ  & \Cdots &        & \circ & \top   & \circ
    \end{pNiceMatrix}^t
    \begin{pNiceMatrix}[nullify-dots,xdots/line-style=loosely dotted]
        V_1 \\
        \Vdots \\
        \\
        \\
        \\
       V_n
    \end{pNiceMatrix}\label{eq:lfsr}\\
    \mathbf{S} = \begin{pmatrix}V & \mathbf{M}V & \mathbf{M}^{2}V & \mathbf{M}^{3}V & \cdots & \mathbf{M}^{2^n-1}V \end{pmatrix}\label{eq:ergo}
\end{align}

\noindent This sequence has \textit{full periodicity}, in other words, for all $i, j \in [0, 2^n), \mathbf{S}_i = \mathbf{S}_j \Rightarrow i = j$. To uniformly sample $\bm\sigma \sim \Sigma^n$ without replacement, we form an injection $GF(2^n)\rightharpoonup\Sigma^d$, cycle through $\mathbf{S}$, then discard samples that do not identify an element in any indexed dimension. This procedure rejects $(1 - |\Sigma|2^{-\lceil\log_2|\Sigma|\rceil})^d$ samples on average and requires $\sim\mathcal{O}(1)$ per sample and $\mathcal{O}(2^n)$ to exhaustively search the space.

\noindent For example, in order to sample $\mathbf\sigma \sim \Sigma^2 = \{A, B, C\}^2$, we could use the primitive polynomial $x^4 + x^3 +1$ shown below:\vspace{-10pt}


\[\begin{array}{rccccccccc}{
  i & 0  &  1  &  2 &  3  &  4  &  5  &  6  &  7 \\
  \mathbf{S}_i & 1000 & 0100 & 0010 & 1001 & 1100 & 0110 & 1011 & 0101 \\
  \mathbf\sigma & $C\:A$  & $B\:A$  & $A\:C$  & $C\:B$  &      & $B\:C$  &      & $B\:B$\\
}\]

\noindent We will use this technique to lazily sample from the space of hole configurations without replacement as described in \S\ref{sec:holes}.

\subsection{Encoding CFG parsing as SAT solving}\label{sec:sat}

By allowing the matrix $\mathbf{M}^*$ in Eq.~\ref{eq:fpm} to contain bitvector variables representing holes in the string and nonterminal sets, we obtain a set of multilinear SAT equations whose solutions exactly correspond to the set of admissible repairs and their corresponding parse forests. Specifically, the repairs coincide with holes in the superdiagonal $\mathbf{M}^*_{r+1 = c}$, whose parse forests occur along the upper-triangular entries $\mathbf{M}^*_{r + 1 < c}$.

%We precompute the shadow of fully-resolved substrings before feeding it to the SAT solver. If the substring is known, we can simply compute this directly outside the SAT solver. Shaded regions are bitvector literals and light regions correspond to bitvector variables.

%We illustrate this fact in \S\ref{sec:error}:
%
%\begin{figure}[H]
%    \includegraphics[width=2cm]{../figures/parse1.png}
%    \includegraphics[width=2cm]{../figures/parse2.png}
%    \includegraphics[width=2cm]{../figures/parse3.png}
%    \includegraphics[width=2cm]{../figures/parse4.png}
%\end{figure}

\newcommand\ddd{\Ddots}
\newcommand\vdd{\Vdots}
\newcommand\cdd{\Cdots}
\newcommand\lds{\ldots}
\newcommand\vno{\varnothing}
\newcommand{\ts}[1]{\textsuperscript{#1}}
\newcommand\non{1\ts{st}}
\newcommand\ntw{2\ts{nd}}
\newcommand\nth{3\ts{rd}}
\newcommand\nfo{4\ts{th}}
\newcommand\nfi{5\ts{th}}
\newcommand\nsi{6\ts{th}}
\newcommand\nse{7\ts{th}}
\newcommand{\vs}[1]{\{V\}_{\sigma_{#1}}}
\newcommand\rcr{\rowcolor{black!15}}
\newcommand\rcw{\rowcolor{white}}
\newcommand\pcd{\cdot}
\newcommand\pcp{\phantom\cdot}
\newcommand\ppp{\phantom{\nse}}

\begin{figure}[H]
\[
  \mathbf{M}^* = \begin{pNiceArray}{>{\strut}ccccccc}[margin, extra-margin=2pt,colortbl-like, xdots/line-style=loosely dotted]
    \vno & \rcr \vs{1} &  \mathcal{L}_{1,3} & \mathcal{L}_{1,3} & \rcw \mathcal{V}_{1,4} & \cdd & \mathcal{V}_{1,n} \\
    \vdd & \ddd        &  \rcr\vs{2}        & \mathcal{L}_{2,3} & \rcw\vdd               &      & \vdd \\
         &             &                    & \rcr\vs{3}        & \rcw                   &      & \\
         &             &                    &                   & \mathcal{V}_{4,4}      &      & \\
         &             &                    &                   &                        & \ddd & \\
         &             &                    &                   &                        &      & \mathcal{V}_{n,n} \\
    \vno & \cdd        &                    &                   &                        &      & \vno
  \end{pNiceArray}
\]
\end{figure}

\noindent Depicted above is a SAT tensor representing \hlgray{$\sigma_1\:\sigma_2\:\sigma_3$}$\:\_\:\ldots\:\_$ where shaded regions demarcate known bitvector literals $\mathcal{L}_{r,c}$ (i.e., representing established nonterminal forests) and unshaded regions correspond to bitvector variables $\mathcal{V}_{r,c}$ (i.e., representing seeded nonterminal forests to be grown). Since $\mathcal{L}_{r,c}$ are fixed, we precompute them outside the SAT solver.

\subsection{Deletion, substitution, and insertion}\label{sec:dsi}

Deletion, substitution and insertion can be simulated by first adding a left- and right- $\varepsilon$-production to each unit production:\vspace{5pt}

\begin{prooftree}
    \AxiomC{$\Gamma \vdash \varepsilon \in \Sigma$}
    \RightLabel{$\varepsilon\textsc{-dup}$}
    \UnaryInfC{$\Gamma \vdash (\varepsilon^+ \rightarrow \varepsilon \mid \varepsilon^+\:\varepsilon^+) \in P$}
\end{prooftree}

\begin{prooftree}
    \AxiomC{$\Gamma \vdash (A \rightarrow B) \in P$}
    \RightLabel{$\varepsilon^+\textsc{-int}$}
    \UnaryInfC{$\Gamma \vdash (A \rightarrow B\:\varepsilon^+ \mid \varepsilon^+\:B \mid B) \in P$}
\end{prooftree}

\vspace{5pt}
\noindent To generate the sketch templates, we substitute two holes at each index to be replaced, $H(\sigma, i) = \sigma_{1\ldots i-1}\:\texttt{\_ \_}\:\sigma_{i+1\ldots n}$, and invoke the SAT solver. Five outcomes are then possible:

\begin{align}
&\sigma_{1}\ldots\sigma_{i-1}\:\text{\hlred{$\gamma_1$}\hlred{$\gamma_2$}}\:\sigma_{i+1}\ldots\sigma_{n}, \gamma_{1, 2} = \varepsilon\label{eq:del}\\
&\sigma_{1}\ldots \sigma_{i-1}\:\text{\hlorange{$\gamma_1$}\hlred{$\gamma_2$}}\:\sigma_{i+1}\ldots \sigma_{n}, \gamma_1 \neq \sigma_i, \gamma_2 = \varepsilon\label{eq:sub1}\\
&\sigma_{1}\ldots \sigma_{i-1}\:\text{\hlred{$\gamma_1$}\hlorange{$\gamma_2$}}\:\sigma_{i+1}\ldots \sigma_{n}, \gamma_1 = \varepsilon, \gamma_2 \neq \sigma_i\label{eq:sub2}\\
&\sigma_{1}\ldots\sigma_{i-1}\: \text{\hlorange{$\gamma_1$}\hlgreen{$\gamma_2$}}\:\sigma_{i+1}\ldots\sigma_{n}, \gamma_1 = \sigma_i, \gamma_2 \neq \varepsilon\label{eq:ins1}\\
&\sigma_{1}\ldots\sigma_{i-1}\: \text{\hlgreen{$\gamma_1$}\hlorange{$\gamma_2$}}\:\sigma_{i+1}\ldots\sigma_{n}, \gamma_1 \notin \{\varepsilon, \sigma_i\}, \gamma_2 = \sigma_i\label{eq:ins2}
\end{align}

\noindent Eq.~(\ref{eq:del}) corresponds to deletion, Eqs.~(\ref{eq:sub1},~\ref{eq:sub2}) correspond to substitution, and Eqs.~(\ref{eq:ins1},~\ref{eq:ins2}) correspond to insertion. This procedure is repeated for all indices in the replacement set. The solutions returned by the SAT solver will be strictly equivalent to handling each edit operation as separate cases.

\section{Error Recovery}\label{sec:error}

Unlike classical parsers which need special care to recover from errors, if the input string does not parse, Tidyparse can return partial subtrees. If no solution exists, the upper triangular entries will appear as a jagged-shaped ridge whose peaks represent the roots of parsable ASTs. These provide a natural debugging environment to aid the repair process.

\begin{tidyinput}
true and true ! and false or true ! true and false
\end{tidyinput}

\begin{verbatim}
 Parseable subtrees (3 leaves / 2 branches):
\end{verbatim}
\noindent\hspace{0.64cm}\emoji{herb}\hspace{1.70cm}\emoji{herb}\hspace{1.98cm}\emoji{herb}\vspace{-5pt}
\begin{verbatim}
    └── ! [3]   └── and [4]   └── or [6]
\end{verbatim}
\hspace{0.63cm}\emoji{herb}\hspace{3.4cm}\emoji{herb}\vspace{-5pt}
\begin{verbatim}
    └── S [0..2]          └── S [8..11]
        ├── true [0]          ├── S [8..9]
        ├── and [1]           │   ├── ! [8]
        └── true [2]          │   └── true [9]
                              ├── and [10]
                              └── false [11]
\end{verbatim}

\noindent These branches correspond to peaks on the upper triangular (UT) matrix ridge. As depicted in Fig.~\ref{fig:peaks}, we traverse the peaks by decreasing elevation to collect partial AST branches.

\begin{figure}[H]
\[
  \setcounter{MaxMatrixCols}{30}
  \begin{NiceMatrix}
                 & \nse & \nsi & \nfi & \nfo & \nth & \ntw & \non & \ppp & \ppp \\
                 &      & \ddd & \ddd & \ddd & \ddd & \ddd & \ddd & \ddd & \ppp \\
\{V\}_{\sigma_1} & \cdd &      & A    &      &      &      &      &      & \ppp \\
\varnothing      & \ddd &  T_A & \vdd &      &      &      &      &      & \ppp \\
\vdd             & \ddd &      & \pcd & \cdd &      & B    &      &      & \ppp \\
                 &      &      &      &      & T_B  & \vdd &      &      & \ppp \\
                 &      &      &      &      &      & \pcd & \cdd &      & C    \\
                 &      &      &      &      &      &      &      & T_C  & \vdd \\
                 &      &      &      &      &      &      & \text{\emoji{cross-mark}} &      & \\
                 &      &      &      &      &      &      &      &      & \\
                 &      &      &      &      &      &      &      &      & \\
                 &      &      &      &      &      &      &      & \ppp & \{V\}_{\sigma_n} \\
\varnothing      & \cdd &      &      &      &      &      &      & \varnothing &
  \end{NiceMatrix}
\]\caption{Peaks along the UT matrix ridge correspond to maximally parseable substrings. By recursing over upper diagonals of decreasing elevation and discarding all subtrees that fall under the shadow of another's canopy, we can recover the partial subtrees. The example depicted above contains three such branches, rooted at nonterminals $C, B, A$.}\label{fig:peaks}
\end{figure}

\section{Tree Denormalization}

%fun denormalize(): Tree {
%    fun Tree.removeSynthetic(
%    refactoredChildren: List<Tree> = children.map { it.removeSynthetic() }.flatten(),
%    isSynthetic: (Tree) -> Boolean = { 2 <= root.split('.').size }
%    ): List<Tree> =
%    if (children.isEmpty()) listOf(Tree(root, terminal, span = span))
%    else if (isSynthetic(this)) refactoredChildren
%    else listOf(Tree(root, children = refactoredChildren.toTypedArray(), span = span))
%
%    return removeSynthetic().first()
%}
% https://www.ling.upenn.edu/advice/latex/qtree/qtreenotes.pdf
% https://cpb-us-w2.wpmucdn.com/campuspress.yale.edu/dist/0/119/files/2014/12/sec-10.17-drawing-arrows-2ek9r2b.pdf

Our parser emits a binary forest consisting of parse trees for the candidate string according to the CNF grammar, however this forest contains many so-called \textit{Krummholz}, or \textit{flag trees}, often found clinging to windy ridges and mountainsides.

\begin{figure}[H]
\resizebox{.50\textwidth}{!}{
\begin{tabular}{ll}
    \Tree [.\texttt{S} \tikz\node(v1){\texttt{true}} [.$\ccancel{\texttt{and.S}}$ \tikz\node(v3){\texttt{and}} [.\texttt{S} \tikz\node(v5){\texttt{(}} [.$\ccancel{\texttt{S.)}}$ [.\texttt{S} \tikz\node(v9){\texttt{false}} [.$\ccancel{\texttt{or.S}}$ \tikz\node(v11){\texttt{or}} [.\texttt{S} \texttt{!} \texttt{true} ] ] ] \tikz\node(v7){\texttt{)}} ] ] ] ]
%    \Tree [.S [.NP John ] [.VP [.\tikz\node(v1){V}; sleeps ] ] ]
    \hspace{-2cm}
    &
    \Tree [.\texttt{S} \tikz\node(v2){\texttt{true}} \tikz\node(v4){\texttt{and}} [.\texttt{S} \tikz\node(v6){\texttt{(}} [.\texttt{S} \tikz\node(v10){\texttt{false}} \tikz\node(v12){\texttt{or}} [.\texttt{S} \texttt{!} \texttt{true} ] ] \tikz\node(v8){\texttt{)}} ] ]\\\\
%    \Tree [.\tikz\node(v2){V}; [.\tikz\node(v3){V}; ] [.Adv {a lot} ] ]
    \hspace{1cm}\huge{Pre-Denormalization} & \hspace{3cm}\huge{Post-Denormalization}
\end{tabular}

\begin{tikzpicture}[overlay]
%    \draw [red,dashed,-stealth] (v1) to[bend left] (v2);
    \draw [red,dashed,-stealth] (v3) to[bend left] (v4);
%    \draw [red,dashed,-stealth] (v5) to[bend left] (v6);
    \draw [red,dashed,-stealth] (v7) to[bend left] (v8);
%    \draw [red,dashed,-stealth] (v9) to[bend right] (v10);
    \draw [red,dashed,-stealth] (v11) to[bend right] (v12);
\end{tikzpicture}
}
%    \caption{Result of applying Algorithm~\ref{alg:cap} to the tree obtained by parsing the string: \texttt{true and ( false or ! true )}.}
\end{figure}\vspace{-10pt}
\begin{algorithm}
    \caption{Tree denormalization procedure.}\label{alg:cap}
    \begin{algorithmic}
        \Procedure{Denormalize}{\texttt{t: Tree}}
            \State $\texttt{stems} \leftarrow \{\:\textsc{Denormalize}(\texttt{c}) \mid \texttt{c} \in \texttt{t.children}\:\}$
            \If{$\texttt{t.root} \text{ is a synthetic } V$}
                \State \textbf{return } \texttt{stems} \Comment{Drop synthetic nonterminals.}
            \Else\Comment{Graft the denormalized children on root.}
                \State \textbf{return } $\{\:\texttt{Tree(root, stems)}\:\}$
            \EndIf
        \EndProcedure
    \end{algorithmic}
\end{algorithm}

\noindent To recover a parse tree congruent with the user-specified grammar, we prune all synthetic nodes and graft their stems onto the grandparent using Algorithm~\ref{alg:cap}, which is used to denormalize both complete and partial ASTs (cf. \S~\ref{sec:error}) alike.

\section{Realtime Error Correction}\label{sec:holes}

Now that we have a method, $\mathcal{G} \times (\Sigma\cup\{\varepsilon, \texttt{\_}\})^n \rightarrow \{\Sigma^n\}\subseteq \mathcal{L}_\mathcal{G}$, given a string, $\Sigma^n$, where do we put the holes? Assuming $k$ holes, there are ${n \choose k}$ possible HCs, each with $(|\Sigma| + 1)^{2k}$ possible repairs (before parsing, cf. Eqs.~\ref{eq:del}-\ref{eq:ins2}). In practice, depending on $n$ and $k$, this space can be intractable to search through exhaustively. To provide real-time assistance, we generate repairs lazily, according to an eight-step procedure:

\begin{enumerate}
  \item Fetch the most recent CFG and string from the editor.
  \item Exclude parsable substrings from hollowing.
  \item Enumerate hole configurations of increasing length.
  \item Sample HCs without replacement using Eq.~\ref{eq:ergo}.
  \item Prioritize HCs first by distance to caret index, then by Earthmover's distance to a set of suspicious indices.$^*$
  \item Translate HCs to sketch templates using~\S\ref{sec:dsi}.
  \item Feed sketch templates to an incremental SAT solver.
  \item Decode and rerank models by Levenshtein distance.
\end{enumerate}

\noindent $^*$ E.g., supplied by edit history or a neural language model. As soon as a new repair is discovered by the SAT solver, it is immediately rendered to the screen. Incoming keystrokes interrupt the sequence and reset the process back to Step 1.

\pagebreak
\section{Example Workflow}

Suppose we have the slightly more complicated grammar:

\begin{tidyinput}
S -> X
X -> A | V | ( X , X ) | X X | ( X )
A -> FUN | F | L | L in X
FUN -> fun V (*@`@*)->(*@`@*) X
F -> if X then X else X
L -> let V = X | let rec V = X

V -> Vexp | ( Vexp ) | List | Vexp Vexp
Vexp -> VarName | FunName | Vexp VO Vexp | B
Vexp -> ( VarName , VarName ) | Vexp Vexp | I
List -> [] | V :: V
VarName -> a | b | c | d | e | ... | z
FunName -> foldright | map | filter
VO ->  + | - | * | / | >
VO -> = | < | (*@`@*)||(*@`@*) | &&
I -> 0 | 1 | 2 | 3 | 4 | 5 | 6 | 7 | 8 | 9
B ->  true | false
\end{tidyinput}

\subsection{Interactive Nonterminal Expansion}

Tidyparse can be used to interactively build up an expression by targeting a nonterminal to expand and asking for a completion.

\begin{tidyinput}
if <Vexp> X then <Vexp> else <Vexp>
\end{tidyinput}
Press Ctrl+Space and the tool will output the following suggestions, from which the users can expand:

\begin{tidyoutput}
if map X then <Vexp> else <Vexp>
if true X then <Vexp> else <Vexp>
if false X then <Vexp> else <Vexp>
if filter X then <Vexp> else <Vexp>
if uncurry X then <Vexp> else <Vexp>
if foldright X then <Vexp> else <Vexp>
if <Vexp> X then <Vexp> else <Vexp>
if <Vexp> <B> X then <Vexp> else <Vexp>
\end{tidyoutput}

There are some more examples too.

The line between parsing and computation is blurry.

\section{Shortcomings}

The synthesis results are not very natural. This could be improved with a neural guided search procedure.

\section{Conclusion}

Tidyparse accepts a CFG and a string to parse and returns a set of candidate strings, ordered by their Levenshtein edit distance to the original string. Our method lowers the CFG and candidate string onto a matrix dynamical system using an extended version of Valiant's construction and solves for the fixpoint matrix using an incremental SAT solver. Our approach to parsing has many advantages.

\begin{itemize}
\item Error correction.
\item Program repair.
\item Program synthesis.
\item Parsing with holes.
\item Naturally integrates with masked language model (MLM)-based neural program repair.
\item Parsing with natural error recovery.
\item Helps to facilitate language learning.
\item GPU acceleration.
\end{itemize}

\section{Acknowledgements}
The first author would like to thank his co-advisor Xujie Si for providing many helpful suggestions during the development of this project, including the optimized fixpoint, test cases, and tree denomalization procedure, Zhixin Xiong for contributing the OCaml CFG and collaborator Nghi Bui for early feedback on the plugin.
\bibliography{../bib/acmart}
\end{document}