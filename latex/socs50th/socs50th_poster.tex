%%%%%%%%%%%%%%%%%%%%%%%%%%%%%%%%%%%%%%%%%%%
%
% From a template maintained at https://github.com/jamesrobertlloyd/cbl-tikz-poster
%
%%%%%%%%%%%%%%%%%%%%%%%%%%%%%%%%%%%%%%%%%%%

\documentclass[portrait,a0b,final,a4resizeable]{a0poster}
\setlength{\paperwidth}{48in}
\setlength{\paperheight}{48in}

\usepackage{atbegshi}% http://ctan.org/pkg/atbegshi
\AtBeginDocument{\AtBeginShipoutNext{\AtBeginShipoutDiscard}}
\usepackage{qrcode}
\usepackage{multicol}
\usepackage{enumitem}
\setlist[itemize]{labelsep=1cm}
\usepackage{mathtools}
%\usepackage{color}
%\usepackage{morefloats}
%\usepackage[pdftex]{graphicx}
%\usepackage{rotating}
\usepackage{amsmath, amsthm, amssymb, bm}
\usepackage{nicematrix}
%\usepackage{array}
\usepackage{xcolor,colortbl}
%\usepackage{booktabs}
\usepackage{multirow}
%\usepackage{hyperref}
\usepackage{pgf-soroban}
\usepackage{bussproofs}
\usepackage{adjustbox}
\usepackage{tikz-3dplot}
\usetikzlibrary{3d}
\usepackage{dsfont}
\usetikzlibrary{decorations.pathreplacing,calligraphy}
\newif\ifshowcellnumber
\showcellnumbertrue
\usetikzlibrary{cd,shapes.geometric,arrows,chains,matrix,positioning,scopes,calc}

\tikzstyle{mybox} = [draw=white, rectangle]
%\definecolor{darkblue}{rgb}{0,0.08,0.45}
%\definecolor{blue}{rgb}{0,0,1}
%\usepackage{dsfont}
\usepackage[margin=0.5in]{geometry}
%\usepackage{fp}

\input{include/jlposter.tex}

\usepackage{include/preamble}

% Custom notation
\newcommand{\fdeep}{\vf^{(1:L)}}
\newcommand{\flast}{\vf^{(L)}}
\newcommand{\Jx}{J_{\vx \rightarrow \vy}}
\newcommand{\Jxx}{J_{\vx \rightarrow \vy}(\vx)}
\newcommand{\Jy}{J_{\vy \rightarrow \vx}}
\newcommand{\Jyy}{J_{\vy \rightarrow \vx}(\vy)}
\newcommand{\detJyy}{ \left| J_{\vy \rightarrow \vx}(\vy) \right|}

\newcommand\transpose{{\textrm{\tiny{\sf{T}}}}}
\newcommand{\note}[1]{}
\newcommand{\hlinespace}{~\vspace*{-0.15cm}~\\\hline\\\vspace*{0.15cm}}
\newcommand{\embeddingletter}{g}
\newcommand{\bo}{{\sc bo}}
\newcommand{\agp}{Arc \gp}

\newcommand{\D}{\mathcal{D}}
\newcommand{\X}{\mathbf{X}}
\newcommand{\y}{y}
\newcommand{\data} {\X, \y}
\newcommand{\x}{\mathbf{x}}
\newcommand{\f}{\mathit{f}}

\newcommand{\fx}{ f(\mathbf{x}) }
\newcommand{\U}{\mathcal{U}}
\newcommand{\E}{\mathbf{E}}


\newcommand{\bardist}[0]{\hspace{-0.2cm}}

\newlength{\arrowsize}
\pgfarrowsdeclare{biggertip}{biggertip}{
\setlength{\arrowsize}{10pt}
\addtolength{\arrowsize}{2\pgflinewidth}
\pgfarrowsrightextend{0}
\pgfarrowsleftextend{-5\arrowsize}
}{
\setlength{\arrowsize}{1pt}
\addtolength{\arrowsize}{\pgflinewidth}
\pgfpathmoveto{\pgfpoint{-5\arrowsize}{4\arrowsize}}
\pgfpathlineto{\pgfpointorigin}
\pgfpathlineto{\pgfpoint{-5\arrowsize}{-4\arrowsize}}
\pgfusepathqstroke
}


% Custom commmands.

\def\jointspacing{\vspace{0.3in}}

\def\boxwidth{0.21\columnwidth}
\newcommand{\gpdrawbox}[1]{
\setlength\fboxsep{0pt}
\hspace{-0.36in}
\fbox{\hspace{-4mm}
%\includegraphics[width=\boxwidth]{../figures/deep_draws/deep_gp_sample_layer_#1}
\hspace{-4mm}}}

\newcommand{\mappic}[1]{
%\hspace{-0.05in}\includegraphics[width=\boxwidth]{../../figures/seed-0-map/latent_coord_map_layer_#1}
}

\newcommand{\mappiccon}[1]{
%\hspace{-0.05in}\includegraphics[width=\boxwidth]{../../figures/seed-0-map-connected/latent_coord_map_layer_#1}
}

\newcommand{\spectrumpic}[1]{
%\includegraphics[trim=4.5mm 0mm 4mm 3mm, clip, width=0.44\columnwidth]{../figures/spectrum/layer-#1}
}

\newcommand{\feat}{\vh}

\makeatletter
\DeclareRobustCommand{\cev}[1]{%
    {\mathpalette\do@cev{#1}}%
}
\newcommand{\do@cev}[2]{%
  \vbox{\offinterlineskip
  \sbox\z@{$\m@th#1 x$}%
  \ialign{##\cr
  \hidewidth\reflectbox{$\m@th#1\vec{}\mkern4mu$}\hidewidth\cr
  \noalign{\kern-\ht\z@}
    $\m@th#1#2$\cr
  }%
  }%
}
\makeatother

\makeatletter
\DeclareRobustCommand{\pder}[1]{%
  \@ifnextchar\bgroup{\@pder{#1}}{\@pder{}{#1}}}
\newcommand{\@pder}[2]{\frac{\partial#1}{\partial#2}}
\makeatother
\usepackage{stmaryrd}
\newcommand{\shup}{\shortuparrow}

\definecolor{A}{RGB}{6,150,104}
\definecolor{B}{RGB}{196,74,137}
\definecolor{C}{RGB}{117,237,133}
\definecolor{D}{RGB}{246,46,243}
\definecolor{E}{RGB}{89,162,12}
\definecolor{F}{RGB}{113,12,158}
\definecolor{G}{RGB}{191,205,142}
\definecolor{H}{RGB}{51,58,158}
\definecolor{I}{RGB}{244,212,3}
\definecolor{J}{RGB}{37,36,249}
\definecolor{K}{RGB}{253,165,71}
\definecolor{L}{RGB}{27,81,29}
\colorlet{LA}{A!30}
\colorlet{LB}{B!30}
\colorlet{LC}{C!30}
\colorlet{LD}{D!30}
\colorlet{LE}{E!30}
\colorlet{LF}{F!30}
\colorlet{LG}{G!30}
\colorlet{LH}{H!30}
\colorlet{LI}{I!30}
\colorlet{LJ}{J!30}
\colorlet{LK}{K!30}
\colorlet{LL}{L!30}
\newcommand{\hiliA}[1]{%
  \colorbox{LA}{$#1$}}
\newcommand{\hiliB}[1]{%
  \colorbox{LB}{$#1$}}
\newcommand{\hiliC}[1]{%
  \colorbox{LC}{$#1$}}
\newcommand{\hiliD}[1]{%
  \colorbox{LD}{$#1$}}
\newcommand{\hiliE}[1]{%
  \colorbox{LE}{$#1$}}
\newcommand{\hiliF}[1]{%
  \colorbox{LF}{$#1$}}
\newcommand{\hiliG}[1]{%
  \colorbox{LG}{$#1$}}
\newcommand{\hiliH}[1]{%
  \colorbox{LH}{$#1$}}
\newcommand{\hiliI}[1]{%
  \colorbox{LI}{$#1$}}
\newcommand{\hiliJ}[1]{%
  \colorbox{LJ}{$#1$}}
\newcommand{\hiliK}[1]{%
  \colorbox{LK}{$#1$}}
\newcommand{\hiliL}[1]{%
  \colorbox{LL}{$#1$}}
\newcommand{\highlight}[1]{%
  \colorbox{lgray}{$#1$}}
\colorlet{lred}{red!30}
\colorlet{lorange}{orange!30}
\colorlet{lgreen}{green!30}
\colorlet{lgray}{black!15}
\colorlet{dgray}{black!75}
\DeclareRobustCommand{\hlred}[1]{{\sethlcolor{lred}\hl{#1}}}
\DeclareRobustCommand{\hlorange}[1]{{\sethlcolor{lorange}\hl{#1}}}
\DeclareRobustCommand{\hlgreen}[1]{{\sethlcolor{lgreen}\hl{#1}}}
\DeclareRobustCommand{\hlgray}[1]{{\sethlcolor{lgray}\hl{#1}}}
\DeclareRobustCommand{\caret}[1]{{\sethlcolor{dgray}\textcolor{white}{\hl{#1}}}}


\begin{document}
  \begin{poster}
    \vspace{1\baselineskip}   % Add some space at the top of the poster


    %%% Header
    \begin{center}
      \begin{pcolumn}{1.03}
        %%% Title
        \begin{minipage}[c][9cm][c]{\textwidth}
          \begin{center}
          {\VERYHuge\hspace{-3.7cm}\textbf{Backpropagation of Syntax Errors in Context-Sensitive Languages}}\\[10mm]
          {\Huge Breandan Considine, Jin Guo, Xujie Si\\[7.5mm]
          }
          \end{center}
        \end{minipage}
      \end{pcolumn}
    \end{center}

    \vspace*{1.5cm}

    \large


    %%%%%%%%%%%%%%%%%%%%%%%%%%%%%%%%%%%%%%%%%%%%%%%%%%%%%%%%%%%%%%%%%%%%%%
    %%% Beginning of Document
    %%%%%%%%%%%%%%%%%%%%%%%%%%%%%%%%%%%%%%%%%%%%%%%%%%%%%%%%%%%%%%%%%%%%%%

    \Large

    \begin{multicols}{2}

      \mysection{Main Idea}

      \vspace*{-1cm}
      \null\hspace*{3cm}{\huge\begin{minipage}[c]{0.95\columnwidth}
      \begin{itemize}
        \item Matrices over $\mathbb{Z}_2^n$ are useful structures for studying finite state machines
        \item The operators $\{\text{XOR}, \land, \top$\} are \textit{functionally complete} logical primitives
        \item We use them to implement probabilistic context-sensitive program repair
      \end{itemize}
      \end{minipage}}

      \jointspacing

      \mysection{Algebraic Parsing}
      {\huge
      \null\hspace*{3cm}\begin{minipage}[c]{0.90\columnwidth}
          Given a CFG, $\mathcal{G}' : \langle \Sigma, V, P, S\rangle$ in Chomsky Normal Form (CNF), we can define a \textit{recognizer}, $R: \mathcal{G}' \rightarrow \Sigma^n \rightarrow \mathbb{B}$ for bounded strings $\sigma: \highlight{\Sigma}^n$ using the following construction. Let $2^V$ be our domain, $0$ be $\varnothing$, $\oplus$ be $\cup$, and $\otimes$:\\
      \end{minipage}

        \[
          X \otimes Z := \big\{\;w \mid \langle x, z\rangle \in X \times Z, (w\rightarrow xz) \in P\;\big\}
        \]

      \null\hspace*{3cm}\begin{minipage}[c]{0.90\columnwidth}
        Valiant (1975) shows that if we let $\sigma_r^{\shup} \coloneqq \{V \mid (V \rightarrow \sigma_r^{\shup}) \in P\}$, initialize the matrix $M^0_{r+1=c}(\mathcal{G}', e) := \;\sigma_r^\shup$ and solve for its fixpoint $M^* = M + M^2$,\\
      \end{minipage}

        \[
          M^0 := \begin{pNiceMatrix}
              \varnothing & \sigma_1^\shup & \varnothing & \Cdots & \varnothing \\
              \Vdots      & \Ddots         & \Ddots      & \Ddots & \Vdots\\
                          &                &             &        & \varnothing\\
                          &                &             &        & \sigma_n^\shup \\
              \varnothing & \Cdots         &             &        & \varnothing
          \end{pNiceMatrix} \Rightarrow
          M^* = \begin{pNiceMatrix}
                  \varnothing & \sigma_1^\shup & V           & \Cdots & V^*_\sigma \\
                  \Vdots      & \Ddots         & \Ddots      & \Ddots & \Vdots\\
                              &                &             &        & V\\
                              &                &             &        & \sigma_n^\shup \\
                  \varnothing & \Cdots         &             &        & \varnothing
        \end{pNiceMatrix}
        \]

      \vspace{1cm}\null\hspace*{3cm}\begin{minipage}[c]{0.90\columnwidth}
        the recognizer is then defined as: $R(\mathcal{G}', \sigma) := S \in V^*_\sigma? \Longleftrightarrow \sigma \in \mathcal{L}(\mathcal{G})?$
      \end{minipage}
      }

      \jointspacing

%      \mysection{Parsing Dynamics}
%      \null\hspace*{3cm}\begin{minipage}[c]{0.90\columnwidth}
%      \begin{itemize}
%        \item The matrix $\mathbf M_0$ is strictly upper triangular, i.e., nilpotent of degree $n$
%        \item The recognizer can be translated into a parser by storing \textit{backpointers}\\\\
%      \end{itemize}\vspace{-3cm}
%      \begin{tabular}{ c c c }
%        \small{$\mathbf{M}_1 = \mathbf{M}_0 + \mathbf{M}_0^2$} & \small{$\mathbf{M}_2 = \mathbf{M}_1 + \mathbf{M}_1^2$} & \small{$\mathbf{M}_3 = \mathbf{M}_2 + \mathbf{M}_2^2 = \mathbf{M}_4$} \\
%        \includegraphics[trim=420 288 0 0,clip, width=12.24cm]{../figures/parse2.png} &
%        \includegraphics[trim=420 285 0 0,clip, width=12.24cm]{../figures/parse3.png} &
%        \includegraphics[trim=420 287 0 0,clip, width=12.34cm]{../figures/parse4.png}
%      \end{tabular}
%      \begin{itemize}
%        \item If we had a way to solve for $\mathbf{M = M + M}^2$ directly, power iteration would be unnecessary and we could solve for $\mathbf{M = M}^2$ above the superdiagonal\ldots
%      \end{itemize}
%      \end{minipage}
%      \jointspacing

      \mysection{Galois Connection}
      {\huge
      \null\hspace*{3cm}\begin{minipage}[c]{0.90\columnwidth}
      \begin{itemize}
        \item CYK parsers can be lowered onto $\mathbb{Z}_2^{|V|\times n\times n}$ or $\mathcal{M}: (\mathbb{Z}_2^{|V|}\rightarrow\mathbb{Z}_2)^{|V|\times n\times n}$
        \item $\mathcal{M}^*$ can be solved for directly using Gaussian elimination or XOR-SAT
        \item Enables sketch-based synthesis in $\sigma$ or $\mathcal G$: just use variables for holes!
        \item We can encode using the characteristic function, i.e., $\mathds{1}_{\subseteq V}: V\rightarrow \mathbb{Z}_2^{|V|}$
        \item $\oplus, \otimes$ are defined as $\boxplus, \boxtimes$, so that the following diagram commutes:
      \end{itemize}
      \vspace{2cm}
        \adjustbox{scale=0.62,center}{%
          \[\begin{tikzcd}[row sep=large]
            \langle\mathcal{G}', \highlight{\Sigma}^{n-1}\rangle \arrow[leftrightarrow, drrr, shorten=-1mm] & & [-135pt] & \vspace{20pt}\textbf{Set} \arrow[d, phantom] & [-70pt] & \phantom{\langle\mathcal{G}', \Sigma^{n-1}\rangle} & [-70pt] & \textbf{Bit} \arrow[d, phantom] & [-90pt] & \langle\mathcal{G}', \Sigma^{n-1}\rangle \arrow[drr, shorten=-2mm] & [-90pt] & \textbf{SAT} \arrow[d, phantom]\\[-30pt]
            \textbf{Rubix}  \arrow[rr, phantom] & & [-135pt] & M \times M \arrow[rrrr, "\mathds{1}^{2^{n\times n}}", labels=above] \arrow[d, "\hspace{-17.5mm}+\:\:\:*"]& [-70pt] & & [-70pt] & \mathbb{Z}_2^{|V|^{n\times n}} \times \mathbb{Z}_2^{|V|^{n\times n}} \arrow[d, "\hspace{-17.5mm}+\:\:\:*"] \arrow[llll, "\mathds{1}^{-2^{n\times n}}", labels=below] \arrow[rrrr, rightarrowtail, "\varphi^{2^{n\times n}}", labels=above] & [-90pt] & & [-90pt] & \mathcal{M} \times \mathcal{M} \arrow[llll, rightharpoonup, shorten=3.5mm, "\varphi^{-2^{n\times n}}", labels=below] \arrow[d, "\hspace{-17.5mm}+\:\:\:*"] \\
            \textbf{Matrix} \arrow[rr, phantom] & & [-135pt] & 2^V \times 2^V \arrow[rrrr, "\mathds{1}^{2}", labels=above] \arrow[d, "\hspace{-15.5mm}\oplus\:\otimes"] & [-70pt] & & [-70pt] & \mathbb{Z}_2^{|V|} \times \mathbb{Z}_2^{|V|} \arrow[d, "\hspace{-15.5mm}\boxplus\:\boxtimes"] \arrow[llll, "\mathds{1}^{-2}", labels=below] \arrow[rrrr, rightarrowtail, "\varphi^2", labels=above] & [-90pt] & & [-90pt] & \mathcal{V} \times \mathcal{V} \arrow[llll, rightharpoonup, shorten=3.5mm, "\varphi^{-2}", labels=below] \arrow[d, "\hspace{-15.5mm}\boxplus\:\boxtimes"] \arrow[u] \\
            \textbf{Vector} \arrow[rr, phantom] & & [-135pt] & 2^V \arrow[rrrr, "\mathds{1}", labels=above] & [-90pt] & & [-90pt] & \mathbb{Z}_2^{|V|} \arrow[llll, "\mathds{1}^{-1}", labels=below] \arrow[rrrr, rightarrowtail, "\varphi", labels=above] & [-90pt] & & [-90pt] & \mathcal{V} \arrow[llll, rightharpoonup, shorten=3.5mm, "\varphi^{-1}", labels=below] \arrow[u]
        \end{tikzcd}\]
        }
%        \item These operators can be lifted into matrices and tensors in the usual way
%        \item In most cases, only a few nonterminals will be active at any given time
%        \item More sophisticated representations are known for $\binom{n}{0 \leq k}$ subsets
%        \item If density is desired, possible to use the Maculay representation
%        \item If you know of a more efficient encoding, please let us know!
      \end{minipage}
      }
      \pagebreak

      \jointspacing

%      \mysection{Probabilistic Programming}
%
%      {\huge
%      \hspace*{3cm}\begin{minipage}[c]{0.90\columnwidth}
%      \vspace{1cm}Let $\text{R}: \mathbb{Z}_2^{n\times n}$ be a matrix $\text{R}_{0, c} = P_c + \text{R}_{r=c+1} = \top$, where $P$ is a feedback polynomial with coefficients $P_{1\ldots n}$ and $\oplus := \veebar, \otimes := \land$:\\
%      \end{minipage}
%
%      \[
%        \text{R}^tV = \begin{pNiceMatrix}
%          P_1    & \Cdots &        &       & P_n \\
%          \top   & \circ  & \Cdots &       & \circ \\
%          \circ  & \Ddots & \Ddots &       & \Vdots \\
%          \Vdots & \Ddots &        &       &       \\
%          \circ  & \Cdots & \circ  & \top  & \circ
%        \end{pNiceMatrix}^t
%        \begin{pNiceMatrix}
%          V_1 \\
%          \Vdots \\
%              \\
%              \\
%          V_n
%        \end{pNiceMatrix}
%      \]\\
%
%      \hspace*{2cm}\begin{minipage}[c]{0.90\columnwidth}
%      Selecting any $V \neq \mathbf{0}$ and coefficients $P_j$ from a \textit{primitive polynomial}, then powering the matrix $\text{R}$ generates an ergodic sequence over GF$(2^n)$, which
%      has \textit{full periodicity}, i.e., for all $i, j \in [0, 2^n), \text{R}^{i}V = \text{R}^{j}V \Rightarrow i = j$.
%      \end{minipage}
%      }
%      \jointspacing

      \mysection{Brozozowski's Derivative}
      {\huge
      \hspace*{2.5cm}\begin{minipage}[c]{0.90\columnwidth}
         \vspace{1cm}Valiant's $\otimes$ operator, which unifies known factors in a binary CFG, implies a left- and right-quotient, which yield the set of nonterminal forests that may appear to either side of a known factor and its corresponding root.\\
       \end{minipage}
      }

      {\LARGE
      \begin{center}
        \begin{tabular}{ccccccc}
          Valiant's $\otimes$ &&& Left Quotient &&& Right Quotient \\\\
          $x \otimes y = \big\{\;w \mid (w \rightarrow x z)\;\big\}$ &&&
          $\frac{\partial f}{\partial \cev{x}} = \big\{\;z \mid (w \rightarrow x z)\;\big\}$ &&&
          $\frac{\partial f}{\partial \vec{z}} = \big\{\;x \mid (w \rightarrow x z)\;\big\}$ \\\\
          \begin{tabular}{|c|c|}
            \hline
            \cellcolor{black!15}\,\,x\,\, & \:w\: \\ \hline
            \multicolumn{1}{c|}{~} & \cellcolor{black!15}\:z\: \\
            \cline{2-2}
          \end{tabular} &&&
          \begin{tabular}{|c|c|}
            \hline
            \cellcolor{black!15}\,\,x\,\, & \cellcolor{black!15}\:w\: \\ \hline
            \multicolumn{1}{c|}{~} & \:z\: \\
            \cline{2-2}
          \end{tabular} &&&
          \begin{tabular}{|c|c|}
            \hline
            \,\,x\,\, & \cellcolor{black!15}\:w\: \\ \hline
            \multicolumn{1}{c|}{~} & \cellcolor{black!15}\:z\: \\
            \cline{2-2}
          \end{tabular}
        \end{tabular}
      \end{center}
      }
\vspace{1cm}
      {\huge
      \hspace*{2.5cm}\begin{minipage}[c]{0.90\columnwidth}
           The left quotient coincides with Brzozowski's derivative (1964) over regular languages, here lifted into the context-sensitive setting (our work).\\
      \end{minipage}
      }

%      \noindent
%
%      \definecolor{R}{RGB}{202,65,55}
%      \definecolor{G}{RGB}{151,216,56}
%      \definecolor{W}{RGB}{255,255,255}
%      \definecolor{X}{RGB}{65,65,65}
%
%      \newcommand{\TikZRubikFaceLeft}[9]{\def\myarrayL{#1,#2,#3,#4,#5,#6,#7,#8,#9}}
%      \newcommand{\TikZRubikFaceRight}[9]{\def\myarrayR{#1,#2,#3,#4,#5,#6,#7,#8,#9}}
%      \newcommand{\TikZRubikFaceTop}[9]{\def\myarrayT{#1,#2,#3,#4,#5,#6,#7,#8,#9}}
%      \newcommand{\BuildArray}{\foreach \X [count=\Y] in \myarrayL%
%      {\ifnum\Y=1%
%      \xdef\myarray{"\X"}%
%      \else%
%      \xdef\myarray{\myarray,"\X"}%
%      \fi}%
%      \foreach \X in \myarrayR%
%      {\xdef\myarray{\myarray,"\X"}}%
%      \foreach \X in \myarrayT%
%      {\xdef\myarray{\myarray,"\X"}}%
%      \xdef\myarray{{\myarray}}%
%      }
%      \TikZRubikFaceLeft
%      {LA}{W}{W}
%      {W}{LB}{LC}
%      {LD}{W}{W}
%      \TikZRubikFaceRight
%      {W}{LK}{W}
%      {LC}{W}{LG}
%      {W}{LH}{W}
%      \TikZRubikFaceTop
%      {LA}{W}{LI}
%      {W}{W}{LJ}
%      {W}{LK}{W}
%      \BuildArray
%      \pgfmathsetmacro\radius{0.1}
%      \tdplotsetmaincoords{55}{135}
%
%      \showcellnumberfalse
%
%      \bgroup
%      \newcommand\ddd{\Ddots}
%      \newcommand\vdd{\Vdots}
%      \newcommand\cdd{\Cdots}
%      \newcommand\lds{\ldots}
%      \newcommand\vno{\varnothing}
%      \newcommand{\ts}[1]{\textsuperscript{#1}}
%      \newcommand\non{1\ts{st}}
%      \newcommand\ntw{2\ts{nd}}
%      \newcommand\nth{3\ts{rd}}
%      \newcommand\nfo{4\ts{th}}
%      \newcommand\nfi{5\ts{th}}
%      \newcommand\nsi{6\ts{th}}
%      \newcommand\nse{7\ts{th}}
%      \newcommand\rcr{\rowcolor{black!15}}
%      \newcommand\rcw{\rowcolor{white}}
%      \newcommand\pcd{\cdot}
%      \newcommand\pcp{\phantom\cdot}
%      \newcommand\ppp{\phantom{\nse}}
%
%        \hspace{2.5cm}\begin{minipage}[l]{6cm}
%        \begin{tikzpicture}
%               \clip (-3,-2.5) rectangle (3,2.5);
%               \begin{scope}[tdplot_main_coords]
%                 \filldraw [canvas is yz plane at x=1.5] (-1.5,-1.5) rectangle (1.5,1.5);
%                 \filldraw [canvas is xz plane at y=1.5] (-1.5,-1.5) rectangle (1.5,1.5);
%                 \filldraw [canvas is yx plane at z=1.5] (-1.5,-1.5) rectangle (1.5,1.5);
%                 \foreach \X [count=\XX starting from 0] in {-1.5,-0.5,0.5}{
%                   \foreach \Y [count=\YY starting from 0] in {-1.5,-0.5,0.5}{
%                     \pgfmathtruncatemacro{\Z}{\XX+3*(2-\YY)}
%                     \pgfmathsetmacro{\mycolor}{\myarray[\Z]}
%                     \draw [thick,canvas is yz plane at x=1.5,shift={(\X,\Y)},fill=\mycolor] (0.5,0) -- ({1-\radius},0) arc (-90:0:\radius) -- (1,{1-\radius}) arc (0:90:\radius) -- (\radius,1) arc (90:180:\radius) -- (0,\radius) arc (180:270:\radius) -- cycle;
%                     \ifshowcellnumber
%                     \node[canvas is yz plane at x=1.5,shift={(\X+0.5,\Y+0.5)}] {\Z};
%                     \fi
%                     \pgfmathtruncatemacro{\Z}{2-\XX+3*(2-\YY)+9}
%                     \pgfmathsetmacro{\mycolor}{\myarray[\Z]}
%                     \draw [thick,canvas is xz plane at y=1.5,shift={(\X,\Y)},fill=\mycolor] (0.5,0) -- ({1-\radius},0) arc (-90:0:\radius) -- (1,{1-\radius}) arc (0:90:\radius) -- (\radius,1) arc (90:180:\radius) -- (0,\radius) arc (180:270:\radius) -- cycle;
%                     \ifshowcellnumber
%                     \node[canvas is xz plane at y=1.5,shift={(\X+0.5,\Y+0.5)},xscale=-1] {\Z};
%                     \fi
%                     \pgfmathtruncatemacro{\Z}{2-\YY+3*\XX+18}
%                     \pgfmathsetmacro{\mycolor}{\myarray[\Z]}
%                     \draw [thick,canvas is yx plane at z=1.5,shift={(\X,\Y)},fill=\mycolor] (0.5,0) -- ({1-\radius},0) arc (-90:0:\radius) -- (1,{1-\radius}) arc (0:90:\radius) -- (\radius,1) arc (90:180:\radius) -- (0,\radius) arc (180:270:\radius) -- cycle;
%                     \ifshowcellnumber
%                     \node[canvas is yx plane at z=1.5,shift={(\X+0.5,\Y+0.5)},xscale=-1,rotate=-90] {\Z};
%                     \fi
%                   }
%                 }
%                 \draw [decorate,decoration={calligraphic brace,amplitude=10pt,mirror},yshift=0pt, line width=1.25pt]
%                 (3,0) -- (3,3) node [black,midway,xshift=-8pt, yshift=-14pt] {\footnotesize $V_x$};
%                 \draw [decorate,decoration={calligraphic brace,amplitude=10pt},yshift=0pt, line width=1.25pt]
%                 (3,0) -- (0,-3) node [black,midway,xshift=-16pt, yshift=0pt] {\footnotesize $V_y$};
%                 \draw [decorate,decoration={calligraphic brace,amplitude=10pt},yshift=0pt, line width=1.25pt]
%                 (0,-3) -- (-3,-3) node [black,midway,xshift=-8pt, yshift=14pt] {\footnotesize $V_w$};
%               \end{scope}
%        \end{tikzpicture}
%        \end{minipage}
%        \begin{minipage}[c]{10cm}
%            \begin{align*}
%              o &\rightarrow \hiliD{so} \mid \hiliC{rs} \mid \hiliB{rr}\hspace{0.5pt} \mid \hiliA{oo}\\
%              r &\rightarrow \hiliE{so} \mid \hiliH{ss}\hspace{0.4pt}\mid \hiliF{rr}\hspace{0.5pt} \mid \hiliK{os}\\
%              s &\rightarrow \hiliL{so} \mid \hiliG{rs} \mid \hiliJ{or} \mid \hiliI{oo}
%            \end{align*}
%        \end{minipage}
%        \begin{minipage}[c]{25.5cm}
%            \begin{align*}
%            \mathcal{H} = \left[\begin{pmatrix}
%                \hiliA{\pder{^2 o}{\cev{o}\partial\vec{o}}} & \pder{^2 o}{\cev{o}\partial\vec{r}} & \pder{^2 o}{\cev{o}\partial\vec{s}}\\
%                \pder{^2 o}{\cev{r}\partial\vec{o}} & \hiliB{\pder{^2 o}{\cev{r}\partial\vec{r}}} & \hiliC{\pder{^2 o}{\cev{r}\partial\vec{s}}}\\
%                \hiliD{\pder{^2 o}{\cev{s}\partial\vec{o}}} & \pder{^2 o}{\cev{s}\partial\vec{r}} & \pder{^2 o}{\cev{s}\partial\vec{s}}
%            \end{pmatrix},
%            \begin{pmatrix}
%                \pder{^2 r}{\cev{o}\partial\vec{o}} & \pder{^2 r}{\cev{o}\partial\vec{r}} & \hiliK{\pder{^2 r}{\cev{o}\partial\vec{s}}}\\
%                \pder{^2 r}{\cev{r}\partial\vec{o}} & \hiliF{\pder{^2 r}{\cev{r}\partial\vec{r}}} & \pder{^2 r}{\cev{r}\partial\vec{s}}\\
%                \hiliE{\pder{^2 r}{\cev{s}\partial\vec{o}}} & \pder{^2 r}{\cev{s}\partial\vec{r}} & \hiliH{\pder{^2 r}{\cev{s}\partial\vec{s}}}
%            \end{pmatrix},
%            \begin{pmatrix}
%                \hiliI{\pder{^2 s}{\cev{o}\partial\vec{o}}} & \hiliJ{\pder{^2 s}{\cev{o}\partial\vec{r}}} & \pder{^2 s}{\cev{o}\partial\vec{s}}\\
%                \pder{^2 s}{\cev{r}\partial\vec{o}} & \pder{^2 s}{\cev{r}\partial\vec{r}} & \hiliG{\pder{^2 s}{\cev{r}\partial\vec{s}}}\\
%                \hiliL{\pder{^2 s}{\cev{s}\partial\vec{o}}} & \pder{^2 s}{\cev{s}\partial\vec{r}} & \pder{^2 s}{\cev{s}\partial\vec{s}}
%            \end{pmatrix}\right]
%          \end{align*}
%        \end{minipage}

      \mysection{Context Sensitivity}

      {\huge
      \hspace*{2.5cm}\begin{minipage}[c]{0.90\columnwidth}
       \vspace{1cm}It is well-known that the family of CFLs is not closed under intersection. For example, consider $\mathcal{L}_\cap := \mathcal{L}(\mathcal{G}_1) \cap \mathcal{L}(\mathcal{G}_1)$ defined in the following way:\\

       \begin{center}
        \begin{tabular}{llll}
          $P_1 := \big\{\;S \rightarrow L R,$ & $L \rightarrow a b \mid a L b,$ & $R \rightarrow c \mid c R\;\big\}$\vspace{5pt}\\
          $P_2 := \big\{\;S \rightarrow L R,$ & $R \rightarrow b c \mid b R c,$ & $L \rightarrow a \mid a L\;\big\}$\\\\
        \end{tabular}
        \end{center}

         $\mathcal{L}_\cap$ is equivalent to the language $\big\{\;a^d b^d c^d \mid d > 0\;\big\}$, which is not a CFL.

       We can encode $\bigcap_{i=1}^c \mathcal{L}(\mathcal{G}_i)$ as a polygonal prism with upper-triangular matrices adjoined to each rectangular face. Specifically, we intersect all terminals $\Sigma_\cap := \bigcap_{i=1}^c \Sigma_i$, then for each $t \in \Sigma_\cap$, construct an equivalence class $E(t, \mathcal{G}_i) = \{ w_i \mid (w_i \rightarrow t) \in P_i\}$ and glue them together at each $\sigma_i$:

       \[
         \bigwedge_{t\in\Sigma_\cap}\bigwedge_{j = 1}^{c-1}\bigwedge_{i=1}^{|\sigma|} E(t, \mathcal{G}_j) \equiv_{\sigma_i} E(t, \mathcal{G}_{j+1})
         \]

       % Generated by cfl4_intersection.vox, open with https://voxelator.com/
         \includegraphics[height=0.127\textwidth]{../figures/angle1.png}\hspace{-0.7cm}
         \includegraphics[height=0.127\textwidth]{../figures/angle2.png}\hspace{-0.7cm}
         \includegraphics[height=0.127\textwidth]{../figures/angle5.png}\hspace{-0.7cm}
         \includegraphics[height=0.127\textwidth]{../figures/angle3.png}\hspace{-0.7cm}
         \includegraphics[height=0.127\textwidth]{../figures/angle4.png}
         \begin{center}Orientations of a $\bigcap_{i = 1}^4 \mathcal{L}(\mathcal{G}_i) \cap \Sigma ^6$ configuration, reprojected into 2-space.\end{center}\\\vspace{2cm}

       \noindent As $c \rightarrow \infty$, this shape approximates a circular cone whose symmetric axis intersects orthonormal CNF unit productions $w_i \rightarrow t$, with $S_i \in V^*_{\sigma}?$ encoded by bitvectors on the base perimeter. Equations of this form are equiexpressive with the family of CSLs realizable by finite CFL intersection.
      \end{minipage}
      }

    \end{multicols}


    \bottombox{
    %% QR code
    %    \hfill\bottomboxlogo{img/kotlin_logo.png}
    % Comment out the line below out to hide logo
    \hspace{4cm}
    \begin{minipage}[c][0.1\paperheight][c]{0.18\textwidth}\qrcode[height=2.2in]{ssnp.ndan.co} \end{minipage}
    \begin{minipage}[c][0.1\paperheight][c]{0.25\textwidth}\includegraphics[height=2.6in]{../figures/fpt_logo.png} \end{minipage}
    \begin{minipage}[c][0.1\paperheight][c]{0.33\textwidth}\includegraphics[height=2.6in]{../figures/mcgill.png} \end{minipage}
    \begin{minipage}[c][0.1\paperheight][c]{0.33\textwidth}\includegraphics[height=3.2in]{../figures/mila.png} \end{minipage}
    %    \hfill\bottomboxlogo{img/mila_mauve.png} % \hfill shifts the logo across so it meets the right hand side margin
    % Note that \bottomboxlogo takes an optional width argument. It defaults to the following:
    % \hfill\bottomboxlogo[width=\textwidth]{<path_to_image_file>}
    % where \textwidth is actually the width of a minipage which is defined in the \bottombox command of
    % betterportaitposter.cls It's a standard \includegraphics command in there, so easy to change if
    % you need to add a border etc.
    }
\end{poster}
\end{document}
