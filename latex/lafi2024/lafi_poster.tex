%%%%%%%%%%%%%%%%%%%%%%%%%%%%%%%%%%%%%%%%%%%
%
% From a template maintained at https://github.com/jamesrobertlloyd/cbl-tikz-poster
%
%%%%%%%%%%%%%%%%%%%%%%%%%%%%%%%%%%%%%%%%%%%

\documentclass[portrait,a0b,final,a4resizeable]{a0poster}
\setlength{\paperwidth}{36in} % A0 width: 46.8in
\setlength{\paperheight}{48in} % A0 width: 46.8in

\usepackage{atbegshi}% http://ctan.org/pkg/atbegshi
\AtBeginDocument{\AtBeginShipoutNext{\AtBeginShipoutDiscard}}
\usepackage{qrcode}
\usepackage{multicol}
\usepackage{enumitem}
\usepackage{mathtools}
%\usepackage{color}
%\usepackage{morefloats}
%\usepackage[pdftex]{graphicx}
%\usepackage{rotating}
\usepackage{amsmath, amsthm, amssymb, bm}
%\usepackage{array}
%\usepackage{booktabs}
\usepackage{multirow}
%\usepackage{hyperref}
\usepackage{pgf-soroban}
\usepackage{bussproofs}
\usepackage{nicematrix}
\usetikzlibrary{cd,shapes.geometric,arrows,chains,matrix,positioning,scopes,calc,trees}
\tikzstyle{mybox} = [draw=white, rectangle]
%\definecolor{darkblue}{rgb}{0,0.08,0.45}
%\definecolor{blue}{rgb}{0,0,1}
%\usepackage{dsfont}
\usepackage[margin=0.5in]{geometry}
%\usepackage{fp}

\input{include/jlposter.tex}

\usepackage{include/preamble}


% Custom notation
\newcommand{\fdeep}{\vf^{(1:L)}}
\newcommand{\flast}{\vf^{(L)}}
\newcommand{\Jx}{J_{\vx \rightarrow \vy}}
\newcommand{\Jxx}{J_{\vx \rightarrow \vy}(\vx)}
\newcommand{\Jy}{J_{\vy \rightarrow \vx}}
\newcommand{\Jyy}{J_{\vy \rightarrow \vx}(\vy)}
\newcommand{\detJyy}{ \left| J_{\vy \rightarrow \vx}(\vy) \right|}

\newcommand\transpose{{\textrm{\tiny{\sf{T}}}}}
\newcommand{\note}[1]{}
\newcommand{\hlinespace}{~\vspace*{-0.15cm}~\\\hline\\\vspace*{0.15cm}}
\newcommand{\embeddingletter}{g}
\newcommand{\bo}{{\sc bo}}
\newcommand{\agp}{Arc \gp}

\newcommand{\D}{\mathcal{D}}
\newcommand{\X}{\mathbf{X}}
\newcommand{\y}{y}
\newcommand{\data} {\X, \y}
\newcommand{\x}{\mathbf{x}}
\newcommand{\f}{\mathit{f}}

\newcommand{\fx}{ f(\mathbf{x}) }
\newcommand{\U}{\mathcal{U}}
\newcommand{\E}{\mathbf{E}}


\newcommand{\bardist}[0]{\hspace{-0.2cm}}

\newlength{\arrowsize}
\pgfarrowsdeclare{biggertip}{biggertip}{
\setlength{\arrowsize}{10pt}
\addtolength{\arrowsize}{2\pgflinewidth}
\pgfarrowsrightextend{0}
\pgfarrowsleftextend{-5\arrowsize}
}{
\setlength{\arrowsize}{1pt}
\addtolength{\arrowsize}{\pgflinewidth}
\pgfpathmoveto{\pgfpoint{-5\arrowsize}{4\arrowsize}}
\pgfpathlineto{\pgfpointorigin}
\pgfpathlineto{\pgfpoint{-5\arrowsize}{-4\arrowsize}}
\pgfusepathqstroke
}


% Custom commmands.

\def\jointspacing{\vspace{0.3in}}

\def\boxwidth{0.21\columnwidth}
\newcommand{\gpdrawbox}[1]{
\setlength\fboxsep{0pt}
\hspace{-0.36in}
\fbox{\hspace{-4mm}
%\includegraphics[width=\boxwidth]{../figures/deep_draws/deep_gp_sample_layer_#1}
\hspace{-4mm}}}

\newcommand{\mappic}[1]{
%\hspace{-0.05in}\includegraphics[width=\boxwidth]{../../figures/seed-0-map/latent_coord_map_layer_#1}
}

\newcommand{\mappiccon}[1]{
%\hspace{-0.05in}\includegraphics[width=\boxwidth]{../../figures/seed-0-map-connected/latent_coord_map_layer_#1}
}

\newcommand{\spectrumpic}[1]{
%\includegraphics[trim=4.5mm 0mm 4mm 3mm, clip, width=0.44\columnwidth]{../figures/spectrum/layer-#1}
}

\usepackage{dsfont}

\newcommand{\feat}{\vh}
\newcommand{\bs}{\blacksquare}
\newcommand{\ws}{\square}

\tikzset{
  treenode/.style = {shape=rectangle, rounded corners,
  draw, align=center,
  top color=white, bottom color=blue!20},
  root/.style     = {treenode, font=\tiny, bottom color=red!30},
  env/.style      = {treenode, font=\tiny},
  dummy/.style    = {circle,draw}
}


\begin{document}
  \begin{poster}
    \vspace{-0.3cm}
    %%% Header
    \begin{center}
      \begin{pcolumn}{1.03}
        %%% Title
        \begin{minipage}[c][9cm][c]{0.85\textwidth}
          \begin{center}
          {\veryHuge \textbf{A Tree Sampler for Bounded Context-Free Languages}}\\[10mm]
          {\huge Breandan Considine}
          \end{center}
        \end{minipage}
      \end{pcolumn}
    \end{center}

    \vspace*{-0.5cm}

    \large


    %%%%%%%%%%%%%%%%%%%%%%%%%%%%%%%%%%%%%%%%%%%%%%%%%%%%%%%%%%%%%%%%%%%%%%
    %%% Beginning of Document
    %%%%%%%%%%%%%%%%%%%%%%%%%%%%%%%%%%%%%%%%%%%%%%%%%%%%%%%%%%%%%%%%%%%%%%

    \Large

    \begin{multicols}{2}


      \mysection{Main Idea}

      \vspace*{-1cm}
      \null\hspace*{2.5cm}\begin{minipage}[c]{0.88\columnwidth}
      \renewcommand\labelitemi{$\vcenter{\hbox{\small$\bullet$}}$}
      \begin{itemize}
%        \item Sampling with rejection is unnecessary if you can map onto a simpler distribution
        \item \phantom{\small{.}} Analytic combinatorics: if you can count it, then you can sample from it!
        \item \phantom{\small{.}} We implement a bijection between labeled binary trees in BCFLs and $\mathbb{Z}_{|T|}$
        \item \phantom{\small{.}} Allows for communication-free parallel no-replacement sampling in $\widetilde{\mathcal{O}}(1)$
      \end{itemize}
      \end{minipage}

      \jointspacing

      \mysection{Semiring Parsing}
      \null\hspace*{3cm}\begin{minipage}[c]{0.85\columnwidth}
          Given a CFG $\mathcal{G} = \langle V, \Sigma, P, S\rangle$ in Chomsky Normal Form (CNF), we may construct a recognizer $R_\mathcal{G}: \Sigma^n \rightarrow \mathbb{B}$ for strings $\sigma: \Sigma^n$ as follows. Let $2^V$ be our domain, where $0$ is $\varnothing$, $\oplus$ is $\cup$, and $\otimes$ be defined as:\vspace{0.5cm}
      \end{minipage}

      \[
        s_1 \otimes s_2 = \{C \mid \langle A, B\rangle \in s_1 \times s_2, (C\rightarrow AB) \in P\}
      \]

      \null\hspace*{3cm}\begin{minipage}[c]{0.85\columnwidth}
If we define $\hat\sigma_r = \{w \mid (w \rightarrow \sigma_r) \in P\}$, then construct a matrix with unit nonterminals on the superdiagonal, $M_0[r+1=c](G', \sigma) = \;\hat\sigma_r$ the fixpoint $M_{i+1} = M_i + M_i^2$ is fully determined by the first diagonal:\vspace{0.5cm}
\end{minipage}

\begin{align*}
M_0=
\begin{pNiceMatrix}[nullify-dots,xdots/line-style=loosely dotted]
   \varnothing & \hat\sigma_1 & \varnothing & \Cdots & \varnothing \\
   \Vdots      & \Ddots   & \Ddots      & \Ddots & \Vdots\\
               &          &             &        & \varnothing\\
               &          &             &        & \hat\sigma_n \\
   \varnothing & \Cdots   &             &        & \varnothing
\end{pNiceMatrix} \Rightarrow
\begin{pNiceMatrix}[nullify-dots,xdots/line-style=loosely dotted]
  \varnothing & \hat\sigma_1 & \Lambda & \Cdots & \varnothing \\
  \Vdots      & \Ddots   & \Ddots  & \Ddots & \Vdots\\
              &          &         &        & \Lambda\\
              &          &         &        & \hat\sigma_n \\
  \varnothing & \Cdots   &         &        & \varnothing
\end{pNiceMatrix} \Rightarrow \ldots \Rightarrow M_\infty =
\begin{pNiceMatrix}[nullify-dots,xdots/line-style=loosely dotted]
   \varnothing & \hat\sigma_1 & \Lambda & \Cdots & \Lambda^*_\sigma\\
   \Vdots      & \Ddots   & \Ddots  & \Ddots & \Vdots\\
               &          &         &        & \Lambda\\
               &          &         &        & \hat\sigma_n \\
   \varnothing & \Cdots   &         &        & \varnothing
\end{pNiceMatrix}
\end{align*}

\null\hspace*{3cm}\begin{minipage}[c]{0.85\columnwidth}
CFL membership is recognized by $R(G', \sigma) = [S \in \Lambda^*_\sigma] \Leftrightarrow [\sigma \in \mathcal{L}(G)]$.
\end{minipage}

      \jointspacing

      \mysection{Parsing Dynamics}

      \null\hspace*{3cm}\begin{minipage}[c]{0.85\columnwidth}
      Let us consider an example with two holes, $\sigma = 1$ \underline{\hspace{1cm}} \underline{\hspace{1cm}}, and the grammar being $G=\{S\rightarrow N O N, O \rightarrow + \mid \times, N \rightarrow 0 \mid 1\}$. This can be rewritten into CNF as $G'=\{S \rightarrow N L, N \rightarrow 0 \mid 1, O \rightarrow \times \mid +, L \rightarrow O N\}$.\vspace{0.5cm}
      \end{minipage}

      \null\hspace*{2.7cm}\begin{minipage}[c]{\columnwidth}
                      \resizebox{.85\columnwidth}{!}{
                      {\renewcommand{\arraystretch}{1.2}
\begin{tabular}{|c|c|c|c|}
  \hline
  & $2^V$ & $\mathbb{B}^{|V|}$ & $\mathbb{B}^{|V|}\rightarrow\mathbb{B}^{|V|}$\\\hline
  $M_0$ & \begin{pmatrix}
  \phantom{V} & \tiny{\{N\}} &         &             \\
              &              & \{N,O\} &             \\
              &              &         & \{N,O\} \\
              &              &         &
  \end{pmatrix} & \begin{pmatrix}
  \phantom{V} & \ws\bs\ws\ws &              &              \\
              &              & \ws\bs\bs\ws &              \\
              &              &              & \ws\bs\bs\ws \\
              &              &              &
  \end{pmatrix} & \begin{pmatrix}
     \phantom{V} & V_{0, 1} &          &          \\
                 &          & V_{1, 2} &          \\
                 &          &          & V_{2, 3} \\
                 &          &          &
  \end{pmatrix} \\\hline
  $M_1$ & \begin{pmatrix}
  \phantom{V} & \tiny{\{N\}} & \varnothing &         \\
              &              & \{N,O\}     & \{L\}   \\
              &              &             & \{N,O\} \\
              &              &             &
  \end{pmatrix} & \begin{pmatrix}
  \phantom{V} & \ws\bs\ws\ws & \ws\ws\ws\ws &              \\
              &              & \ws\bs\bs\ws & \bs\ws\ws\ws \\
              &              &              & \ws\bs\bs\ws \\
              &              &              &
  \end{pmatrix} & \begin{pmatrix}
  \phantom{V} & V_{0, 1} & V_{0, 2} &          \\
              &          & V_{1, 2} & V_{1, 3} \\
              &          &          & V_{2, 3} \\
              &          &          &
  \end{pmatrix} \\\hline
  $M_\infty$ & \begin{pmatrix}
  \phantom{V} & \tiny{\{N\}} & \varnothing & \{S\}   \\
              &              & \{N,O\}     & \{L\}   \\
              &              &             & \{N,O\} \\
              &              &             &
  \end{pmatrix} & \begin{pmatrix}
  \phantom{V} & \ws\bs\ws\ws & \ws\ws\ws\ws & \ws\ws\ws\bs \\
              &              & \ws\bs\bs\ws & \bs\ws\ws\ws \\
              &              &              & \ws\bs\bs\ws \\
              &              &              &
  \end{pmatrix} & \begin{pmatrix}
  \phantom{V} & V_{0, 1} & V_{0, 2} & V_{0, 3} \\
              &          & V_{1, 2} & V_{1, 3} \\
              &          &          & V_{2, 3} \\
              &          &          &
  \end{pmatrix}\\\hline
\end{tabular}\\
}
                      }\vspace{0.4cm}
      \end{minipage}

      \null\hspace*{3cm}\begin{minipage}[c]{0.90\columnwidth}
      This procedure decides if $\exists \sigma' \in \mathcal{L}(G) \mid \sigma' \sqsubseteq \sigma$ but forgets provenance.
      \end{minipage}
      \jointspacing

      \mysection{Encoding CFL Sketching into SAT}
      \null\hspace*{2.5cm}\begin{minipage}[c]{0.90\columnwidth}
      \renewcommand\labelitemi{$\vcenter{\hbox{\small$\bullet$}}$}
      \begin{itemize}
        \item \phantom{.} CYK parser can be lowered onto a Boolean tensor $\mathbb{B}^{n\times n \times |V|}$ (Valiant, 1975)
        \item \phantom{.} Binarized CYK parser can be compiled to SAT to solve for $\mathbf{M}^*$ directly
%       \item \phantom{.} Enables sketch-based synthesis in either $\sigma$ or $\mathcal G$: just use variables for holes!
        \item \phantom{.} We simply encode the characteristic function, i.e., $\mathds{1}_{\subseteq V}: 2^V\rightarrow \mathbb{B}^{|V|}$
        \item \phantom{.} $\oplus, \otimes$ are defined as $\boxplus, \boxtimes$, so that the following diagram commutes:
        \[\begin{tikzcd}
            2^V \times 2^V \arrow[r, "\oplus/\otimes"] \arrow[d, "\mathds{1}^2"]
            & 2^V \arrow[d, "\mathds{1}\phantom{^{-1}}"] \\
            \mathbb{B}^{|V|} \times \mathbb{B}^{|V|} \arrow[r, "\boxplus/\boxtimes", labels=below] \arrow[u, "\mathds{1}^{-2}"]
            & \mathbb{B}^{|V|} \arrow[u, "\mathds{1}^{-1}"]
        \end{tikzcd}\]
        \item \phantom{.} These operators can be lifted into matrices and tensors in the usual manner
%        \item In most cases, only a few nonterminals will be active at any given time
%        \item If density is desired, possible to use the Maculay representation
%        \item If you know of a more efficient encoding, please let us know!
      \end{itemize}
      \vspace{1.8cm}
      \end{minipage}

      \jointspacing

      \pagebreak      \mysection{A Nested Datatype for BCFLs}

      \jointspacing

      \hspace*{2cm}\begin{minipage}[c]{0.90\columnwidth}
      We define a map from nonterminals $\mathbb{T}_3 = (V \cup \Sigma) \rightharpoonup \mathbb{T}_2$ onto the datatype $\mathbb{T}_2 = (V \cup \Sigma) \times (\mathbb{N} \rightharpoonup \mathbb{T}_2\times\mathbb{T}_2)$, whose inhabitants satisfy the recurrence:
      \end{minipage}

      \begin{equation*}
        L(p) = 1 + p L(p) \phantom{addspace} P(a) = V + a L\big(V^2P(a)^2\big)
      \end{equation*}

      \hspace*{2cm}\begin{minipage}[c]{0.90\columnwidth}
        Each $\mathbb{T}_2$ consists of a root nonterminal, and a list of distinct products, e.g.,
      \end{minipage}

      \jointspacing

      \hspace*{10cm}\resizebox{0.5\columnwidth}{!}{
        \begin{tikzpicture}
        [
          grow                    = right,
          sibling distance        = 3em,
          level distance          = 5em,
          edge from parent/.style = {draw, -latex},
          every node/.style       = {font=\footnotesize},
          sloped
        ]
          \node [root] {S}
          child { node [env] {BC}
          child { node [root] {B}
          child { node [env] {RD}
          child { node [root] {R} edge from parent node [above] }
          child { node [root] {D} edge from parent node [above] }
          edge from parent node [above] }
          edge from parent node [below] }
          child { node [root] {C}
          child { node [env] {\ldots\vphantom{BB}} edge from parent node [below] }
%  child { edge from parent node [above] {\ldots} }
          edge from parent node [below] }
          edge from parent node [above] }
          child { node [env] {\ldots\vphantom{BB}} edge from parent node [below] }
          child { node [env] {AB}
          child { node [root] {A}
          child {
            node [env] {QC}
            child { node [root] {Q} edge from parent node [above] }
            child { node [root] {C} edge from parent node [above] }
            edge from parent node [above]
          }
%    child { node [env] {ZQ} edge from parent node [above] }
          child { node [env] {\ldots\vphantom{BB}} edge from parent node [below] }
          edge from parent node [below] }
          child { node [root] {B}
          child { node [env] {RD}
          child { node [root] {R} edge from parent node [above] }
          child { node [root] {D} edge from parent node [above] }
          edge from parent node [above] }
          edge from parent node [below] }
          edge from parent node [above] };
        \end{tikzpicture}
      }

      \vspace{1.5cm}

      \hspace*{2cm}\begin{minipage}[c]{0.90\columnwidth}
      Morally, $\mathbb{T}_2$ denotes an implicit set of possible trees sharing the same root, where $\mathbb{T}_3$ is a dictionary of possible $\mathbb{T}_2$ values indexed by possible roots, given by a specific CFG and porous string. Instead of $\hat\sigma_r$, we initialize using $\Lambda(\sigma_r)$:
      \end{minipage}

      \[
           \Lambda(s: \underline\Sigma) \mapsto \begin{cases}
               \bigoplus_{s\in \Sigma} \Lambda(s) & \text{if $s$ is a hole,} \vspace{5pt}\\
               \big\{\mathbb{T}_2\big(w, \big[\langle\mathbb{T}_2(s), \mathbb{T}_2(\varepsilon)\rangle\big]\big) \mid (w \rightarrow s)\in P\big\} & \text{otherwise.}
           \end{cases}
      \]
%
      \hspace*{2cm}\begin{minipage}[c]{0.90\columnwidth}
      The operations $\oplus, \otimes: \mathbb{T}_3 \times \mathbb{T}_3 \rightarrow \mathbb{T}_3$ are then redefined over trees as follows:
      \end{minipage}

      \[
          X \oplus Z \mapsto \bigcup_{\mathclap{k\in \pi_1(X \cup Z)}}\Big\{k \Rightarrow \mathbb{T}_2(k, x \cup z) \mid x \in \pi_2(X\circ k), z \in \pi_2(Z\circ k)\Big\}
      \]

      \[
        X \otimes Z \mapsto \bigoplus_{\mathclap{(w\rightarrow xz) \in P}}\Big\{\mathbb{T}_2\Big(w, \big[\langle X\circ x, Z\circ z\rangle\big]\Big) \mid x \in \pi_1(X), z \in \pi_1(Z)\Big\}
      \]
%
%      \hspace*{2cm}\begin{minipage}[c]{0.90\columnwidth}
%      These operators group subtrees by their root nonterminal, then aggregate their children. Each $\Lambda$ becomes a dictionary indexed by the root nonterminal, which can be sampled by obtaining $(\Lambda_\sigma^* \circ S): \mathbb{T}_2$, then recursively choosing twins.
%      \end{minipage}

      \vspace{-0.5cm}
      \jointspacing

      \mysection{Sampling with Replacement}

      \jointspacing

      \hspace*{2cm}\begin{minipage}[c]{0.90\columnwidth}
      Given a PCFG whose productions indexed by each nonterminal are decorated with a probability vector $\mathbf{p}$, we define a tree sampler $\Gamma: (\mathbb{T}_2 \mid \mathbb{T}_2\times\mathbb{T}_2) \rightsquigarrow \mathbb{T}$:
      \end{minipage}

      \begin{equation*}
        \Gamma(T) \mapsto \begin{cases}
          \Gamma\Big(\text{Multi} \big(\texttt{children}(T), \mathbf{p}\big)\Big) & \text{ if $T: \mathbb{T}_2$ } \\
          \Big\langle \Gamma\big(\pi_1(T)\big), \Gamma\big(\pi_2(T)\big) \Big\rangle & \text{ if $T: \mathbb{T}_2\times \mathbb{T}_2$ }
        \end{cases}
      \end{equation*}


      \hspace*{2cm}\begin{minipage}[c]{0.90\columnwidth}
      This relates to the generating function for the ordinary Boltzmann sampler,
      \end{minipage}

      \begin{equation*}
        \Gamma C(x) \mapsto \begin{cases}
                              \text{Bern} \left(\frac{A(x)}{A(x) + B(x)}\right) \rightarrow \Gamma A(x) \mid \Gamma B(x) & \text{ if } \mathcal{C}=\mathcal{A}+\mathcal{B} \\
                              \big\langle \Gamma A(x), \Gamma B(x)\big\rangle & \text{ if } \mathcal{C}=\mathcal{A} \times \mathcal{B}
        \end{cases}
      \end{equation*}


      \hspace*{2cm}\begin{minipage}[c]{0.90\columnwidth}
      however unlike Duchon et al. (2004), our work does require rejection to ensure exact-size sampling, as all trees contained in $\mathbb{T}_2$ are necessarily the same width.
      \end{minipage}

      \jointspacing

      \mysection{Sampling without Replacement}

      \jointspacing

      \hspace*{2cm}\begin{minipage}[c]{0.90\columnwidth}
      To sample all trees in a given $T: \mathbb{T}_2$ uniformly without replacement, we define a modular pairing function $\varphi: \mathbb{T}_2 \rightarrow \mathbb{Z}_{|T|} \rightarrow \texttt{BTree}$ using the construction:
      \end{minipage}

      \begin{equation*}
        \varphi(T: \mathbb{T}_2, i: \mathbb{Z}_{|T|}) \mapsto \begin{cases}
          \big\langle\texttt{BTree}\big(\texttt{root}(T)\big), i\big\rangle & \text{if $T$ is a leaf,} \vspace{5pt}\\
          \text{Let } b = |\texttt{children}(T)|,\\
          \phantom{\text{Let }} q_1, r=\big\langle\lfloor\frac{i}{b}\rfloor, i \pmod{b}\big\rangle,\\
          \phantom{\text{Let }} lb, rb = \texttt{children}[r],\\
          \phantom{\text{Let }} T_1, q_2 = \varphi(lb, q_1),\\
          \phantom{\text{Let }} T_2, q_3 = \varphi(rb, q_2) \text{ in } \\
          \big\langle\texttt{BTree}\big(\texttt{root}(T), T_1, T_2\big), q_3\big\rangle & \text{otherwise.} \\
        \end{cases}
      \end{equation*}

      \hspace*{2cm}\begin{minipage}[c]{0.90\columnwidth}
      Then instead of sampling trees, we can simply sample integers WoR from $\mathbb{Z}_{|T|}$.\vspace{1.8cm}
      \end{minipage}

      \jointspacing

    \end{multicols}

    \vspace{-2.2cm}
    \bottombox{
    %% QR code
    %    \hfill\bottomboxlogo{img/kotlin_logo.png}
    % Comment out the line below out to hide logo

    \hspace{1.8cm}
    \begin{minipage}[c][0.1\paperheight][c]{0.18\textwidth}\qrcode[height=2.6in]{https://tidyparse.github.io/} \end{minipage}
    \begin{minipage}[c][0.1\paperheight][c]{0.25\textwidth}\includegraphics[height=2.6in]{../figures/tidyparse_logo.png} \end{minipage}
    \hspace{-4cm}
    \begin{minipage}[c][0.1\paperheight][c]{0.33\textwidth}\includegraphics[height=3in]{../figures/mcgill.png} \end{minipage}
    \hspace{2cm}
    \begin{minipage}[c][0.1\paperheight][c]{0.33\textwidth}\includegraphics[height=3.2in]{../figures/mila.png} \end{minipage}
    %    \hfill\bottomboxlogo{img/mila_mauve.png} % \hfill shifts the logo across so it meets the right hand side margin
    % Note that \bottomboxlogo takes an optional width argument. It defaults to the following:
    % \hfill\bottomboxlogo[width=\textwidth]{<path_to_image_file>}
    % where \textwidth is actually the width of a minipage which is defined in the \bottombox command of
    % betterportaitposter.cls It's a standard \includegraphics command in there, so easy to change if
    % you need to add a border etc.
    }
\end{poster}
\end{document}
