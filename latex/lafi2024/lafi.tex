\pdfoutput=1
%% For double-blind review submission, w/o CCS and ACM Reference (max submission space)
%\documentclass[sigplan,10pt,review,anonymous]{acmart}
%\settopmatter{printfolios=false,printccs=false,printacmref=false}
%% For double-blind review submission, w/ CCS and ACM Reference
%\documentclass[sigplan,review,anonymous]{acmart}\settopmatter{printfolios=true}
%% For single-blind review submission, w/o CCS and ACM Reference (max submission space)
%\documentclass[sigplan,review]{acmart}\settopmatter{printfolios=true,printccs=false,printacmref=false}
%% For single-blind review submission, w/ CCS and ACM Reference
%\documentclass[sigplan,review]{acmart}\settopmatter{printfolios=true}
%% For final camera-ready submission, w/ required CCS and ACM Reference
\documentclass[sigplan,nonacm]{acmart}\settopmatter{printfolios=false,printccs=false,printacmref=false}
\makeatletter
\def\@ACM@checkaffil{% Only warnings
    \if@ACM@instpresent\else
    \ClassWarningNoLine{\@classname}{No institution present for an affiliation}%
    \fi
    \if@ACM@citypresent\else
    \ClassWarningNoLine{\@classname}{No city present for an affiliation}%
    \fi
    \if@ACM@countrypresent\else
        \ClassWarningNoLine{\@classname}{No country present for an affiliation}%
    \fi
}
\makeatother

%% Conference information
%% Supplied to authors by publisher for camera-ready submission;
%% use defaults for review submission.
%\acmConference[ARRAY'22]{ACM SIGPLAN Conference on Programming Languages}{June 13, 2022}{San Diego, CA, USA}
%\acmYear{2018}
%\acmISBN{} % \acmISBN{978-x-xxxx-xxxx-x/YY/MM}
%\acmDOI{} % \acmDOI{10.1145/nnnnnnn.nnnnnnn}
%\startPage{1}

%% Copyright information
%% Supplied to authors (based on authors' rights management selection;
%% see authors.acm.org) by publisher for camera-ready submission;
%% use 'none' for review submission.
\setcopyright{none}
%\setcopyright{acmcopyright}
%\setcopyright{acmlicensed}
%\setcopyright{rightsretained}
%\copyrightyear{2018}           %% If different from \acmYear

\newcommand\hmmax{0}
\newcommand\bmmax{0}

%% Bibliography style
\bibliographystyle{acmart}

\usepackage{dsfont}
\usepackage{stmaryrd}
\usepackage{colortbl}
\usepackage{hyperref}

\usepackage{amsmath}
\DeclareMathOperator*{\argmax}{argmax}
\DeclareMathOperator*{\argmin}{argmin}
\usepackage{amssymb}

\usepackage[dvipsnames, table]{xcolor}
\usepackage{textcomp}

% Packages
\usepackage[pdf]{graphviz}
\usepackage{mathrsfs}

\newcommand*\circled[1]{\tikz[baseline=-0.1cm]{
  \node[shape=circle,draw,inner sep=0.48pt] (char) {\fontsize{7}{12}\textsf{#1}};}}

\DeclareMathAlphabet{\mathcal}{OMS}{cmsy}{m}{n}
\usepackage{cancel}
\newcommand\ccancel[2][red]{\renewcommand\CancelColor{\color{#1}}\cancel{#2}}
\newcommand{\nDownarrow}{\ensuremath{\text{ }\cancel{\Downarrow}\text{ }}}
\usepackage{centernot}

\usepackage{pgfplots, pgfplotstable}
\pgfplotsset{compat=1.7}
\usepgfplotslibrary{fillbetween}
\usetikzlibrary{patterns}
\pgfmathdeclarefunction{gauss}{2}{\pgfmathparse{1/(#2*sqrt(2*pi))*exp(-((x-#1)^2)/(2*#2^2))}}
\pgfmathdeclarefunction{nil}{1}{\pgfmathparse{0.001}}

\usepackage{arydshln}
\usepackage{adjustbox}
\usepackage{enumerate}
\usepackage{enumitem}
\usepackage{tikz-cd}
\usetikzlibrary{calc}
\usepackage{amsfonts}
%\usepackage{prooftrees}
\usepackage{bussproofs}
\renewcommand{\sectionautorefname}{\S}
\renewcommand{\subsectionautorefname}{\S}
\usepackage{float}

\usepackage{tikz-3dplot}
\usetikzlibrary{3d}
\usetikzlibrary{calligraphy}
\newif\ifshowcellnumber
\showcellnumbertrue

\usepackage{algorithm}
\usepackage{algpseudocode}
\usepackage{algorithmicx}
\usepackage{sourcecodepro}
\usepackage{tikz-qtree}
\usepackage{amsthm}
\usepackage{bm}
\usetikzlibrary{bayesnet}
\usetikzlibrary{arrows}
\usepackage{subcaption}
\usetikzlibrary{backgrounds}
\usetikzlibrary{tikzmark}

\newcommand{\E}{\mathbb{E}}
\newcommand{\Var}{\mathrm{Var}}
\newcommand{\Cov}{\mathrm{Cov}}

\newcommand{\CompOrder}{\mathcal{O}}
\def\graphspace{\mathbf{G}}
\def\Uniform{\mbox{\rm Uniform}}
\def\Gaussian{\mbox{\rm Gaussian}}
\def\Bernoulli{\mbox{\rm Bernoulli}}
\def\Dirichlet{\mbox{\rm Dirichlet}}

\usepackage{mathtools}% superior to amsmath
\usepackage{tikz}
% Packages
\usepackage{listings}
\DeclareRobustCommand{\hlred}[1]{{\sethlcolor{pink}\hl{#1}}}
\usepackage{fontspec}

\setmonofont[Scale=0.8]{JetBrainsMono}[
  Contextuals={Alternate},
  Path=./font/,
  Extension = .ttf,
  UprightFont=*-Regular,
  BoldFont=*-Bold,
  ItalicFont=*-Italic,
  BoldItalicFont=*-BoldItalic
]

\usepackage[skins,breakable,listings]{tcolorbox}

\lstdefinelanguage{kotlin}{
  comment=[l]{//},
  commentstyle={\color{gray}\ttfamily},
  emph={delegate, filter, firstOrNull, forEach, it, lazy, mapNotNull, println, repeat, assert, with, head, tail, len, return@},
  numberstyle=\noncopyable,
  identifierstyle=\color{black},
  keywords={abstract, actual, as, as?, break, by, class, companion, continue, data, do, dynamic, else, enum, expect, false, final, for, fun, get, if, import, in, infix, interface, internal, is, null, object, open, operator, override, package, private, public, return, sealed, set, super, suspend, this, throw, true, try, catch, typealias, val, var, vararg, when, where, while, tailrec, reified},
  keywordstyle={\bfseries},
  morecomment=[s]{/*}{*/},
  morestring=[b]",
  morestring=[s]{"""*}{*"""},
  ndkeywords={@Deprecated, @JvmField, @JvmName, @JvmOverloads, @JvmStatic, @JvmSynthetic, Array, Byte, Double, Float, Boolean, Int, Integer, Iterable, Long, Runnable, Short, String, int},
  ndkeywordstyle={\bfseries},
  sensitive=true,
  stringstyle={\ttfamily},
  literate={`}{{\char0}}1,
  escapeinside={(*@}{@*)}
}
\lstdefinelanguage{tidy}{
  comment=[l]{//},
  commentstyle={\color{gray}\ttfamily},
  emph={|, ->, ---},
  emphstyle={\color{red}},
  identifierstyle=\color{black},
  keywords={\|, ->, ---},
  otherkeywords={|,->},
  morekeywords={|,->},
  keywordstyle={\color{blue}\bfseries},
  morecomment=[s]{/*}{*/},
  morestring=[b]",
  morestring=[s]{"""*}{*"""},
  ndkeywords={@Deprecated, @JvmField, @JvmName, @JvmOverloads, @JvmStatic, @JvmSynthetic, Array, Byte, Double, Float, Int, Integer, Iterable, Long, Runnable, Short, String},
  ndkeywordstyle={\color{orange}\bfseries},
  sensitive=true,
  stringstyle={\color{green}\ttfamily},
  literate={`}{{\char0}}1
}

%%%%%%%%%%%%%%%%%%%%%%%%%%%%%%%%%%%%%%%%%%%
%
% Color boxes
%
%%%%%%%%%%%%%%%%%%%%%%%%%%%%%%%%%%%%%%%%%%%

\tcbset{
  enhanced jigsaw,
  breakable,
  listing only,
%  boxsep=-1pt,
%  top=-1pt,
  bottom=0.1cm,
  right=0.5cm,
  overlay first={
    \node[black!50] (S) at (frame.south) {\Large\ding{34}};
    \draw[dashed,black!50] (frame.south west) -- (S) -- (frame.south east);
  },
  overlay middle={
    \node[black!50] (S) at (frame.south) {\Large\ding{34}};
    \draw[dashed,black!50] (frame.south west) -- (S) -- (frame.south east);
    \node[black!50] (S) at (frame.north) {\Large\ding{34}};
    \draw[dashed,black!50] (frame.north west) -- (S) -- (frame.north east);
  },
  overlay last={
    \node[black!50] (S) at (frame.north) {\Large\ding{34}};
    \draw[dashed,black!50] (frame.north west) -- (S) -- (frame.north east);
  },
  before={\par\vspace{5pt}},
  after={\par\vspace{\parskip}\noindent}
}

\definecolor{slightgray}{rgb}{0.90, 0.90, 0.90}

\usepackage{soul}
\makeatletter
\def\SOUL@hlpreamble{%
  \setul{}{3.0ex}%
  \let\SOUL@stcolor\SOUL@hlcolor%
  \SOUL@stpreamble%
}
\makeatother

\newcommand{\inline}[1]{%
  \begingroup%
  \sethlcolor{slightgray}%
  \hl{\ttfamily\footnotesize #1}%
  \endgroup
}

\newcommand{\tinline}[1]{%
  \begingroup%
  \sethlcolor{slightgray}%
  \hl{\ttfamily\tiny #1}%
  \endgroup
}

\newtcblisting{halftidyinput}[1][]{%
  left skip=0.7cm,
  width=6cm,
%  left=-0.01cm,
  top=-0.1cm,
  bottom=-0.35cm,
  listing options={
    language=tidy,
    basicstyle=\ttfamily\small,
%numberstyle=\footnotesize,
    showstringspaces=false,
    tabsize=2,
    breaklines=true,
    numbers=none,
    inputencoding=utf8,
    escapeinside={(*@}{@*)},
    #1
  },
  underlay unbroken and first={%
    \path[draw=none] (interior.north west) rectangle node[white]{\includegraphics[width=4mm]{../figures/tidyparse_logo.png}} ([xshift=-10mm,yshift=-7mm]interior.north west);
  }
}

\newtcblisting{wholetidyinput}[1][]{%
  left skip=0.7cm,
  top=0.1cm,
  middle=0mm,
  boxsep=0mm,
  listing options={
    language=tidy,
    basicstyle=\ttfamily\small,
%numberstyle=\footnotesize,
    showstringspaces=false,
    tabsize=2,
    breaklines=true,
    numbers=none,
    inputencoding=utf8,
    escapeinside={(*@}{@*)},
    #1
  },
  underlay unbroken and first={%
      \path[draw=none] (interior.north west) rectangle node[white]{\includegraphics[width=4mm]{../figures/tidyparse_logo.png}} ([xshift=-10mm,yshift=-9mm]interior.north west);
  }
}

\definecolor{A}{RGB}{6,150,104}
\definecolor{B}{RGB}{196,74,137}
\definecolor{C}{RGB}{117,237,133}
\definecolor{D}{RGB}{246,46,243}
\definecolor{E}{RGB}{89,162,12}
\definecolor{F}{RGB}{113,12,158}
\definecolor{G}{RGB}{191,205,142}
\definecolor{H}{RGB}{51,58,158}
\definecolor{I}{RGB}{244,212,3}
\definecolor{J}{RGB}{37,36,249}
\definecolor{K}{RGB}{253,165,71}
\definecolor{L}{RGB}{27,81,29}
\colorlet{LA}{A!30}
\colorlet{LB}{B!30}
\colorlet{LC}{C!30}
\colorlet{LD}{D!30}
\colorlet{LE}{E!30}
\colorlet{LF}{F!30}
\colorlet{LG}{G!30}
\colorlet{LH}{H!30}
\colorlet{LI}{I!30}
\colorlet{LJ}{J!30}
\colorlet{LK}{K!30}
\colorlet{LL}{L!30}
\newcommand{\hiliA}[1]{%
  \colorbox{LA}{$#1$}}
\newcommand{\hiliB}[1]{%
  \colorbox{LB}{$#1$}}
\newcommand{\hiliC}[1]{%
  \colorbox{LC}{$#1$}}
\newcommand{\hiliD}[1]{%
  \colorbox{LD}{$#1$}}
\newcommand{\hiliE}[1]{%
  \colorbox{LE}{$#1$}}
\newcommand{\hiliF}[1]{%
  \colorbox{LF}{$#1$}}
\newcommand{\hiliG}[1]{%
  \colorbox{LG}{$#1$}}
\newcommand{\hiliH}[1]{%
  \colorbox{LH}{$#1$}}
\newcommand{\hiliI}[1]{%
  \colorbox{LI}{$#1$}}
\newcommand{\hiliJ}[1]{%
  \colorbox{LJ}{$#1$}}
\newcommand{\hiliK}[1]{%
  \colorbox{LK}{$#1$}}
\newcommand{\hiliL}[1]{%
  \colorbox{LL}{$#1$}}
\newcommand{\highlight}[1]{%
  \colorbox{lgray}{$#1$}}
\colorlet{lred}{red!30}
\colorlet{lorange}{orange!30}
\colorlet{lgreen}{green!30}
\colorlet{lgray}{black!15}
\colorlet{dgray}{black!75}
\DeclareRobustCommand{\hlred}[1]{{\sethlcolor{lred}\hl{#1}}}
\DeclareRobustCommand{\hlorange}[1]{{\sethlcolor{lorange}\hl{#1}}}
\DeclareRobustCommand{\hlgreen}[1]{{\sethlcolor{lgreen}\hl{#1}}}
\DeclareRobustCommand{\hlgray}[1]{{\sethlcolor{lgray}\hl{#1}}}
\DeclareRobustCommand{\caret}[1]{{\sethlcolor{dgray}\textcolor{white}{\hl{#1}}}}

\usepackage{url}
\usepackage{qtree}

\usepackage{filecontents}
\usepackage{pstricks-add}
\usepackage{emoji}
\usepackage{alltt}
\usepackage{nicematrix}
\usepackage{graphicx}
\usepackage{ulem}
\usepackage{upquote}
\tikzstyle{every picture}+=[remember picture]
\usepackage{menukeys}
\pgfplotstableread[col sep=comma,]{timings_loc.csv}\loctimings
\pgfplotstableread[col sep=comma,]{timings_unloc.csv}\unloctimings

\makeatletter
\DeclareRobustCommand{\cev}[1]{%
  {\mathpalette\do@cev{#1}}%
}
\newcommand{\do@cev}[2]{%
  \vbox{\offinterlineskip
  \sbox\z@{$\m@th#1 x$}%
  \ialign{##\cr
  \hidewidth\reflectbox{$\m@th#1\vec{}\mkern4mu$}\hidewidth\cr
  \noalign{\kern-\ht\z@}
    $\m@th#1#2$\cr
  }%
  }%
}
\makeatother

\makeatletter
\DeclareRobustCommand{\pder}[1]{%
  \@ifnextchar\bgroup{\@pder{#1}}{\@pder{}{#1}}}
\newcommand{\@pder}[2]{\frac{\partial#1}{\partial#2}}
\makeatother

\newcommand{\shup}{\shortuparrow}
\newcommand{\shri}{\shortrightarrow}
\newcommand{\shur}{\shup\hspace{-5pt}\shri}

\makeatletter
\def\squiggly{\bgroup \markoverwith{\textcolor{red}{\lower3\p@\hbox{\sixly \char58}}}\ULon}
\makeatother

\newcommand{\err}[1]{\smash{\squiggly{#1}{}}}
\newcommand{\stirlingii}{\genfrac{\{}{\}}{0pt}{}}

%======== Arrows =========
\newcommand{\knightarrow}{
  \tikz{
    \fill (0pt,0pt) circle [radius = 1pt];
    \fill (0pt,6pt) circle [radius = 1pt];
    \fill (6pt,0pt) circle [radius = 1pt];
    \fill (6pt,6pt) circle [radius = 1pt];
    \fill (12pt,0pt) circle [radius = 1pt];
    \fill (12pt,6pt) circle [radius = 1pt];
    \fill (6pt,0pt) circle [radius = 1pt];
    \fill (12pt,0pt) circle [radius = 1pt];
    \draw [-to] (0pt,0pt) -- (12pt,6pt);
  }
}

\newcommand{\kingarrow}{
  \tikz{
    \fill (0pt,0pt) circle [radius = 1pt];
    \fill (6pt,0pt) circle [radius = 1pt];
    \fill (0pt,6pt) circle [radius = 1pt];
    \fill (6pt,6pt) circle [radius = 1pt];
    \draw [-to] (0pt,0pt) -- (6pt,6pt);
    \draw [-to] (0pt,0pt) -- (0pt,6pt);
    \draw [-to] (0pt,0pt) -- (6pt,0pt);
  }
}

\newcommand{\knightkingarrow}{
  \tikz{
    \fill (0pt,0pt) circle [radius = 1pt];
    \fill (0pt,6pt) circle [radius = 1pt];
    \fill (6pt,0pt) circle [radius = 1pt];
    \fill (6pt,6pt) circle [radius = 1pt];
    \fill (12pt,0pt) circle [radius = 1pt];
    \fill (12pt,6pt) circle [radius = 1pt];
    \draw [-to] (0pt,0pt) -- (6pt,6pt);
    \draw [-to] (0pt,0pt) -- (0pt,6pt);
    \draw [-to] (0pt,0pt) -- (6pt,0pt);
    \draw [-to] (0pt,0pt) -- (12pt,6pt);
  }
}

%======== Arrows =========

\usetikzlibrary{decorations.pathreplacing,automata,calc,positioning,matrix,fit}
\usepackage{wrapfig}

\newcommand{\mkTrellis}[1]{
  \begin{tikzpicture}
    \def\dx{20pt}
    \def\dy{30pt}
    \newcounter{i}
    \stepcounter{i}
    \node[circle, draw, fill=black!30] (\arabic{i}) at (0,0){};
    \foreach [count=\i] \x in {2,...,#1}{
      \pgfmathsetmacro{\lox}{\x-1}%
      \pgfmathsetmacro{\loxt}{\x-3}%
      \foreach [count=\j] \xx in {-\lox,-\loxt,...,\lox}{
        \pgfmathsetmacro{\jj}{\j-1}%
        \stepcounter{i}
        \pgfmathsetmacro{\kk}{\xx-2}%
        \pgfmathsetmacro{\lbl}{\lox!/(\jj!*(\lox-\jj)!)}
        \ifnum\x<\kk
        \pgfmath\node[circle, draw]  (\arabic{i}) at (\xx*\dx, -\lox*\dy) {};
        \else
        \pgfmath\node[circle, draw, fill=black!30]  (\arabic{i}) at (\xx*\dx, -\lox*\dy) {};
        \fi
      }
    }
    \newcounter{z}
    \newcounter{xn}
    \newcounter{xnn}
    \pgfmathsetmacro{\maxx}{#1 - 1}
    \foreach \x in {1,...,\maxx}{
      \foreach \xx in {1,...,\x}{
        \stepcounter{z}
        \setcounter{xn}{\arabic{z}}
        \addtocounter{xn}{\x}
        \setcounter{xnn}{\arabic{xn}}
        \stepcounter{xnn}
        \draw [<-] (\arabic{z}) -- (\arabic{xn});
        \draw [<-] (\arabic{z}) -- (\arabic{xnn});
      }
    }
  \end{tikzpicture}
}

\newcommand{\dx}{20pt}
\newcommand{\dy}{30pt}
\newcounter{i}
\newcounter{z}
\newcounter{xn}
\newcounter{xnn}
\newcommand{\mkTrellisAppend}[1]{
  \begin{tikzpicture}
    \setcounter{i}{0}
    \setcounter{z}{0}
    \setcounter{xn}{0}
    \setcounter{xnn}{0}
    \stepcounter{i}
    \node[circle, draw] (\arabic{i}) at (0,0){};
    \foreach [count=\i] \x in {2,...,#1}{
      \pgfmathsetmacro{\lox}{\x-1}%
      \pgfmathsetmacro{\loxt}{\x-3}%
      \foreach [count=\j] \xx in {-\lox,-\loxt,...,\lox}{
        \pgfmathsetmacro{\jj}{\j-1}%
        \stepcounter{i}
        \pgfmathsetmacro{\kk}{\xx+2}%
        \pgfmathsetmacro{\lbl}{\lox!/(\jj!*(\lox-\jj)!)}
        \ifnum\x>\kk
        \pgfmath\node[circle, draw, fill=black!30]  (\arabic{i}) at (\xx*\dx, -\lox*\dy) {};
        \else
        \pgfmath\node[circle, draw]  (\arabic{i}) at (\xx*\dx, -\lox*\dy) {};
        \fi
      }
    }
    \pgfmathsetmacro{\maxx}{#1 - 1}
    \foreach \x in {1,...,\maxx}{
      \foreach \xx in {1,...,\x}{
        \stepcounter{z}
        \setcounter{xn}{\arabic{z}}
        \addtocounter{xn}{\x}
        \setcounter{xnn}{\arabic{xn}}
        \stepcounter{xnn}
        \draw [<-] (\arabic{z}) -- (\arabic{xn});
        \draw [<-] (\arabic{z}) -- (\arabic{xnn});
      }
    }
  \end{tikzpicture}
}

\newcommand{\mkTrellisInsert}[1]{
  \begin{tikzpicture}
    \setcounter{i}{0}
    \setcounter{z}{0}
    \setcounter{xn}{0}
    \setcounter{xnn}{0}
    \stepcounter{i}
    \node[circle, draw] (\arabic{i}) at (0,0){};
    \foreach [count=\i] \x in {2,...,#1}{
      \pgfmathsetmacro{\lox}{\x-1}%
      \pgfmathsetmacro{\loxt}{\x-3}%
      \foreach [count=\j] \xx in {-\lox,-\loxt,...,\lox}{
        \pgfmathsetmacro{\jj}{\j-1}%
        \stepcounter{i}
        \pgfmathsetmacro{\mp}{\xx+#1}%
        \pgfmathsetmacro{\mq}{\xx+\x}%
        \pgfmathsetmacro{\lbl}{\lox!/(\jj!*(\lox-\jj)!)}
        \ifnum\x>\mp
        \pgfmath\node[circle, draw, fill=black!30]  (\arabic{i}) at (\xx*\dx, -\lox*\dy) {};
        \else
        \ifnum#1<\mq
        \pgfmath\node[circle, draw, fill=black!30]  (\arabic{i}) at (\xx*\dx, -\lox*\dy) {};
        \else
        \pgfmath\node[circle, draw]  (\arabic{i}) at (\xx*\dx, -\lox*\dy) {};
        \fi
        \fi

      }
    }
    \pgfmathsetmacro{\maxx}{#1 - 1}
    \foreach \x in {1,...,\maxx}{
      \foreach \xx in {1,...,\x}{
        \stepcounter{z}
        \setcounter{xn}{\arabic{z}}
        \addtocounter{xn}{\x}
        \setcounter{xnn}{\arabic{xn}}
        \stepcounter{xnn}
        \draw [<-] (\arabic{z}) -- (\arabic{xn});
        \draw [<-] (\arabic{z}) -- (\arabic{xnn});
      }
    }
  \end{tikzpicture}
}

\usetikzlibrary{automata, positioning, arrows}

\newcommand{\nobarfrac}{\genfrac{}{}{0pt}{}}
\pgfplotstableread[col sep=comma,]{timings_loc.csv}\loctimings
\pgfplotstableread[col sep=comma,]{timings_unloc.csv}\unloctimings
\pgfplotstableread[col sep=comma,]{natural_errors.csv}\naturalerrors
\pgfplotstableread[col sep=comma,]{synthetic_errors.csv}\syntheticerrors

\begin{document}

  \title{A Tree Sampler for Bounded Context-Free Languages}
  \begin{abstract}
    In the following paper, we present a simple method for sampling trees with or without replacement from BCFLs. A BCFL is a context-free language (CFL) corresponding to an incomplete string with holes, which can be completed by valid terminals. To solve this problem, we introduce an algebraic datatype that compactly represents candidate parse forests for porous strings. Once constructed, sampling trees is a straightforward matter of sampling integers uniformly without replacement, then lazily decoding them into trees.
  \end{abstract}

  \author{Breandan Considine}
  \affiliation{\institution{McGill University}}
  \email{bre@ndan.co}

  \maketitle

  \section{Introduction}

  A CFG is a quadruple consisting of terminals $(\Sigma)$, nonterminals $(V)$, productions $(P\colon V \rightarrow (V \mid \Sigma)^*)$, and a start symbol, $(S)$. It is a well-known fact that every CFG is reducible to \textit{Chomsky Normal Form}, $P'\colon V \rightarrow (V^2 \mid \Sigma)$, in which every production takes one of two forms, either $w \rightarrow xz$, or $w \rightarrow t$, where $w, x, z: V$ and $t: \Sigma$. For example, the CFG, $P=\{S \rightarrow S S \mid ( S ) \mid ()\}$, corresponds to the CNF:\vspace{-3pt}

  \begin{table}[H]
    \begin{tabular}{llll}
      $P'=\big\{\;S\rightarrow QR \mid SS \mid LR,$ & $L \rightarrow (,$ & $R \rightarrow ),$ & $Q\rightarrow LS\;\big\}$
    \end{tabular}
  \end{table}\vspace{-8pt}

  \noindent Given a CFG, $\mathcal{G}' : \langle \Sigma, V, P, S\rangle$ in CNF, we can construct a recognizer $R: \mathcal{G}' \rightarrow \Sigma^n \rightarrow \mathbb{B}$ for strings $\sigma: \Sigma^n$ as follows. Let $2^V$ be our domain, $0$ be $\varnothing$, $\oplus$ be $\cup$, and $\otimes$ be defined as:\vspace{-10pt}

  \begin{align}
    X \otimes Z = \big\{\;w \mid \langle x, z\rangle \in X \times Z, (w\rightarrow xz) \in P\;\big\}
  \end{align}

  \noindent If we define $\hat\sigma_r = \{w \mid (w \rightarrow \sigma_r) \in P\}$, then initialize $M^0_{r+1=c}(\mathcal{G}', e) = \;\hat\sigma_r$ and solve for $M = M + M^2$, the fixedpoint $M_\infty$ is fully determined by the superdiagonal entries:\vspace{-10pt}

  \begin{align*}
      M^0=\begin{pNiceMatrix}[nullify-dots,xdots/line-style=loosely dotted]
        \varnothing & \hat\sigma_1   & \varnothing & \Cdots & \varnothing \\
        \Vdots      & \Ddots         & \Ddots      & \Ddots & \Vdots\\
                    &                &             &        & \varnothing\\
                    &                &             &        & \hat\sigma_n \\
        \varnothing & \Cdots         &             &        & \varnothing
      \end{pNiceMatrix} &\Rightarrow M_\infty =
      \begin{pNiceMatrix}[nullify-dots,xdots/line-style=loosely dotted]
        \varnothing & \hat\sigma_1   & \Lambda & \Cdots & \Lambda^*_\sigma\\
        \Vdots      & \Ddots         & \Ddots  & \Ddots & \Vdots\\
                    &                &         &        & \Lambda\\
                    &                &         &        & \hat\sigma_n \\
        \varnothing & \Cdots         &         &        & \varnothing
      \end{pNiceMatrix}
  \end{align*}

  \noindent Once obtained, the proposition $[S \in \Lambda^*_\sigma]$ decides language membership, i.e., $[\sigma \in \mathcal{L}(\mathcal{G})]$. This procedure is essentially the textbook CYK algorithm in a linear algebraic notation~\cite{goodman1999semiring}. We are now ready to define the sampling problem as follows:

  \begin{definition}[Completion]
    Let $\underline\Sigma = \Sigma \cup \{\_\}$, where $\_$ represents a hole. We denote $\sqsubseteq: \Sigma^n \times \underline\Sigma^n$ as the relation $\{\langle\sigma', \sigma\rangle \mid \sigma_i: \Sigma \implies \sigma_i' = \sigma_i\}$ and the set $\{\sigma': \Sigma \mid \sigma' \sqsubseteq \sigma\}$ as $\text{H}(\sigma)$. Given a porous string $\sigma: \underline\Sigma$, we want to sample parse trees generated by $\mathcal{G}$ corresponding to $\sigma': \text{H}(\sigma)\cap\ell$.
  \end{definition}

  $\text{H}(\sigma)\cap\ell$ is often a large-cardinality set, so we want a procedure which samples trees uniformly without replacement, without enumerating the whole set, parsing and shuffling it.

  \pagebreak\section{Method}\label{sec:method}

   We define an algebraic data type $\mathbb{T}_3 = (V \cup \Sigma) \rightharpoonup \mathbb{T}_2$ where $\mathbb{T}_2 = (V \cup \Sigma) \times (\mathbb{N} \rightharpoonup \mathbb{T}_2\times\mathbb{T}_2)$\footnote{Given a $T:\mathbb{T}_2$, we may also refer to $\pi_1(T), \pi_2(T)$ as $\texttt{root}(T)$ and $\texttt{children}(T)$ respectively, where children are pairs of conjoined twins.}. Morally, we can think of $\mathbb{T}_2$ as an implicit set of possible trees sharing the same root, and $\mathbb{T}_3$ as a dictionary of possible $\mathbb{T}_2$ values indexed by possible roots, given by a specific CFG under a finite-length porous string. We construct $\hat\sigma_r = \Lambda(\sigma_r)$ as follows:

\vspace{-10pt}\begin{equation*}
  \begin{footnotesize}
\Lambda(s: \underline\Sigma) \mapsto \begin{cases}
\bigoplus_{s\in \Sigma} \Lambda(s) & \text{if $s$ is a hole,} \vspace{5pt}\\
\big\{\mathbb{T}_2\big(w, \big[\langle\mathbb{T}_2(s), \mathbb{T}_2(\varepsilon)\rangle\big]\big) \mid (w \rightarrow s)\in P\big\} & \text{otherwise.}
\end{cases}
  \end{footnotesize}
\end{equation*}

\noindent This initializes the superdiagonal, enabling us to compute the fixpoint $M_\infty$ by redefining $\oplus, \otimes: \mathbb{T}_3 \times \mathbb{T}_3 \rightarrow \mathbb{T}_3$ as:

\begin{equation*}
  \begin{footnotesize}
  X \oplus Z \mapsto \bigcup_{\mathclap{k\in \pi_1(X \cup Z)}}\Big\{k \Rightarrow \mathbb{T}_2(k, x \cup z) \mid x \in \pi_2(X\circ k), z \in \pi_2(Z\circ k)\Big\}
  \end{footnotesize}
\end{equation*}

\begin{equation*}
  \begin{footnotesize}
  X \otimes Z \mapsto \bigoplus_{\mathclap{(w\rightarrow xz) \in P}}\Big\{\mathbb{T}_2\Big(w, \big[\langle X\circ x, Z\circ z\rangle\big]\Big) \mid x \in \pi_1(X), z \in \pi_1(Z)\Big\}
\end{footnotesize}
\end{equation*}

  These operators group subtrees by their root nonterminal, then aggregate their children. Instead of tracking sets, each $\Lambda$ now becomes a dictionary indexed by the root nonterminal, which can be sampled by obtaining $(\Lambda_\sigma^* \circ S): \mathbb{T}_2$, then recursively choosing twins as we describe in \S~\ref{sec:replacement}, or without replacement via enumeration as described in \S~\ref{sec:pairing}.

%\begin{equation*}
%  \mathcal{C}(T: \mathbb{T}_2) \mapsto \begin{cases}
%    \texttt{root}(T) & \text{if $T$ is a leaf,} \\
%    \big\{x z \mid \langle X, Z\rangle \in \texttt{children}(T), x \in \mathcal{C}(X), z \in \mathcal{C}(Z)\big\} & \text{otherwise.}%\text{if $d \leq \max(|\err{\sigma}|, \min_{\sigma \in \mathcal{L}(G')}|\sigma|)$}.
%  \end{cases}
%\end{equation*}

\subsection{Sampling trees with replacement}\label{sec:replacement}

Given a probabilistic CFG whose productions indexed by each nonterminal are decorated with a probability vector $\mathbf{p}$ (this may be uniform in the non-probabilistic case), we define a tree sampler $\Gamma: (\mathbb{T}_2 \mid \mathbb{T}_2^2) \rightsquigarrow \mathbb{T}$ which recursively samples children according to a Multinoulli distribution:

\begin{equation*}
  \Gamma(T) \mapsto \begin{cases}
        \Gamma\big(\text{Multi} \big(\texttt{children}(T), \mathbf{p}\big)\big) & \text{ if $T: \mathbb{T}_2$ } \\
        \big\langle \Gamma\big(\pi_1(T)\big), \Gamma\big(\pi_2(T)\big) \big\rangle & \text{ if $T: \mathbb{T}_2\times\mathbb{T}_2$ }
  \end{cases}
\end{equation*}

This is closely related to the generating function for the ordinary Boltzmann sampler from analytic combinatorics,

\begin{equation*}
  \Gamma C(x) \mapsto \begin{cases}
  \text{Bern} \left(\frac{A(x)}{A(x) + B(x)}\right) \rightarrow \Gamma A(x) \mid \Gamma B(x) & \text{ if } \mathcal{C}=\mathcal{A}+\mathcal{B} \\
  \big\langle \Gamma A(x), \Gamma B(x)\big\rangle & \text{ if } \mathcal{C}=\mathcal{A} \times \mathcal{B}
  \end{cases}
\end{equation*}

\noindent however unlike Duchon et al.~\cite{duchon2004boltzmann}, our work does not depend on rejection to guarantee exact-size sampling, as all trees contained in $\mathbb{T}_2$ will necessarily be the same width.

\pagebreak\subsection{Sampling without replacement}\label{sec:pairing}

The type $\mathbb{T}_2$ of all possible trees that can be generated by a CFG in Chomksy Normal Form corresponds to the fixpoints of the following recurrence, which tells us that each $\mathbb{T}_2$ can be a terminal, or a nonterminal and a (possibly empty) sequence of nonterminal pairs and their two children:\vspace{-10pt}

\begin{equation*}
  L(p) = 1 + p L(p) \phantom{addspace} P(a) = \Sigma + V L\big(V^2P(a)^2\big)
\end{equation*}

Given a $\sigma: \underline\Sigma$, we construct $\mathbb{T}_2$ from the bottom-up, and sample from the top-down. Depicted below is a partial $\mathbb{T}_2$, where red nodes are \texttt{root}s and blue nodes are \texttt{children}:

\begin{figure}[H]
\resizebox{0.9\columnwidth}{!}{
  \begin{tikzpicture}
  [
    grow                    = right,
    sibling distance        = 3em,
    level distance          = 5em,
    edge from parent/.style = {draw, -latex},
    every node/.style       = {font=\footnotesize},
    sloped
  ]
  \node [root] {S}
  child { node [env] {BC}
    child { node [root] {B}
      child { node [env] {RD}
        child { node [root] {R} }
        child { node [root] {D} }
      }
    }
    child { node [root] {C}
      child { node [env] {$\ldots\vphantom{BB}$} }
    }
  }
  child { node [env] {$\ldots\vphantom{BB}$} }
  child { node [env] {AB}
    child { node [root] {A}
      child {
        node [env] {QC}
        child { node [root] {Q} }
        child { node [root] {C} }
      }
      child { node [env] {$\ldots\vphantom{BB}$} }
    }
    child { node [root] {B}
      child { node [env] {RD}
        child { node [root] {R}  }
        child { node [root] {D}  }
      }
    }
  };
  \end{tikzpicture}
}
\caption{A partial $\mathbb{T}_2$ for the grammar with productions $P=\{S \rightarrow BC \mid \ldots \mid AB, B\rightarrow RD \mid \ldots, A\rightarrow QC \mid \ldots\}$.}
\end{figure}

To obtain the total number of trees with breadth $n$, we abstractly parse the porous string using the algebra defined in \S~\ref{sec:method}, and compute the total number of trees using:

\begin{equation*}
  |T: \mathbb{T}_2| \mapsto \begin{cases}
%    \big|\{s \mid \big(\texttt{root}(T) \rightarrow s\big) \in P^\cap\}\big| & \text{if $T$ is a leaf,} \\
    1  & \text{if $T$ is a leaf,} \\
    \sum_{\langle T_1, T_2\rangle \in \texttt{children}(T)} |T_1| \cdot |T_2| & \text{otherwise.}
  \end{cases}
\end{equation*}

To sample all trees in a given $T: \mathbb{T}_2$ uniformly without replacement, we then construct a modular pairing function $\varphi: \mathbb{T}_2 \rightarrow \mathbb{Z}_{|T|} \rightarrow \texttt{BTree}$, which is defined as follows:

\begin{small}
\begin{equation*}\label{eq:pairing}
  \varphi(T: \mathbb{T}_2, i: \mathbb{Z}_{|T|}) \mapsto \begin{cases}
  \texttt{BTree}\big(\texttt{root}(T)\big) \text{ if $T$ is a leaf,} \vspace{5pt}\\
  \textbf{let } r = |\texttt{children}(T)|,\\
  \phantom{\textbf{Let }} F(n) = \sum_{\langle l, r\rangle \in \texttt{children}[0 \ldots n]}|l|\cdot|r|,\\
  \phantom{\textbf{Let }} F^{-1}(u)=\inf \big\{x \mid u \leq F(x)\big\},\\
  \phantom{\textbf{Let }} q=i - F\big(F^{-1}(i)\big),\\
  \phantom{\textbf{Let }} l, r = \texttt{children}[t],\\
  \phantom{\textbf{Let }} q_1, q_2 =\big\langle\lfloor\frac{q}{|r|}\rfloor, q \pmod{|r|}\big\rangle,\\
  \phantom{\textbf{Let }} T_1, T_2 = \big\langle\varphi(l, q_1), \varphi(r, q_2)\big\rangle \textbf{ in } \\
  \texttt{BTree}\big(\texttt{root}(T), T_1, T_2\big) \text{ otherwise.} \\
  \end{cases}
\end{equation*}
\end{small}

%\begin{small}
%\begin{equation*}
%  \varphi(T: \mathbb{T}_2, i: \mathbb{Z}_{|T|}) \mapsto \begin{cases}
%  \Big\langle\texttt{BTree}\big(\texttt{root}(T)\big), i\Big\rangle & \text{if $T$ is a leaf,} \vspace{5pt}\\
%  \text{Let } b = |\texttt{children}(T)|,\\
%  \phantom{\text{Let }} q_1, r=\big\langle\lfloor\frac{i}{b}\rfloor, i \pmod{b}\big\rangle,\\
%  \phantom{\text{Let }} lb, rb = \texttt{children}[r],\\
%  \phantom{\text{Let }} T_1, q_2 = \varphi(lb, q_1),\\
%  \phantom{\text{Let }} T_2, q_3 = \varphi(rb, q_2) \text{ in } \\
%  \Big\langle\texttt{BTree}\big(\texttt{root}(T), T_1, T_2\big), q_3\Big\rangle & \text{otherwise.} \\
%  \end{cases}
%\end{equation*}
%\end{small}

Then, instead of sampling trees, we can simply sample integers uniformly without replacement from $\mathbb{Z}_{|T|}$ using the construction defined in \S~\ref{sec:method}, and lazily decode them into trees.

%  \section{Benchmark}
%
%  Here we plot the number of distinct results found in Python code snippets.\footnote{Implementation found here: \url{https://github.com/breadnan/galoisenne}}
%
%  \begin{figure}[H]
%    \resizebox{0.5\columnwidth}{!}{%% Creator: Matplotlib, PGF backend
%%
%% To include the figure in your LaTeX document, write
%%   \input{<filename>.pgf}
%%
%% Make sure the required packages are loaded in your preamble
%%   \usepackage{pgf}
%%
%% Also ensure that all the required font packages are loaded; for instance,
%% the lmodern package is sometimes necessary when using math font.
%%   \usepackage{lmodern}
%%
%% Figures using additional raster images can only be included by \input if
%% they are in the same directory as the main LaTeX file. For loading figures
%% from other directories you can use the `import` package
%%   \usepackage{import}
%%
%% and then include the figures with
%%   \import{<path to file>}{<filename>.pgf}
%%
%% Matplotlib used the following preamble
%%   \usepackage{fontspec}
%%   \setmainfont{DejaVuSerif.ttf}[Path=\detokenize{/Users/breandan/miniconda3/envs/gym-fs/lib/python3.10/site-packages/matplotlib/mpl-data/fonts/ttf/}]
%%   \setsansfont{DejaVuSans.ttf}[Path=\detokenize{/Users/breandan/miniconda3/envs/gym-fs/lib/python3.10/site-packages/matplotlib/mpl-data/fonts/ttf/}]
%%   \setmonofont{DejaVuSansMono.ttf}[Path=\detokenize{/Users/breandan/miniconda3/envs/gym-fs/lib/python3.10/site-packages/matplotlib/mpl-data/fonts/ttf/}]
%%
\begingroup%
\makeatletter%
\begin{pgfpicture}%
\pgfpathrectangle{\pgfpointorigin}{\pgfqpoint{6.400000in}{4.800000in}}%
\pgfusepath{use as bounding box, clip}%
\begin{pgfscope}%
\pgfsetbuttcap%
\pgfsetmiterjoin%
\definecolor{currentfill}{rgb}{1.000000,1.000000,1.000000}%
\pgfsetfillcolor{currentfill}%
\pgfsetlinewidth{0.000000pt}%
\definecolor{currentstroke}{rgb}{1.000000,1.000000,1.000000}%
\pgfsetstrokecolor{currentstroke}%
\pgfsetdash{}{0pt}%
\pgfpathmoveto{\pgfqpoint{0.000000in}{0.000000in}}%
\pgfpathlineto{\pgfqpoint{6.400000in}{0.000000in}}%
\pgfpathlineto{\pgfqpoint{6.400000in}{4.800000in}}%
\pgfpathlineto{\pgfqpoint{0.000000in}{4.800000in}}%
\pgfpathlineto{\pgfqpoint{0.000000in}{0.000000in}}%
\pgfpathclose%
\pgfusepath{fill}%
\end{pgfscope}%
\begin{pgfscope}%
\pgfsetbuttcap%
\pgfsetmiterjoin%
\definecolor{currentfill}{rgb}{1.000000,1.000000,1.000000}%
\pgfsetfillcolor{currentfill}%
\pgfsetlinewidth{0.000000pt}%
\definecolor{currentstroke}{rgb}{0.000000,0.000000,0.000000}%
\pgfsetstrokecolor{currentstroke}%
\pgfsetstrokeopacity{0.000000}%
\pgfsetdash{}{0pt}%
\pgfpathmoveto{\pgfqpoint{0.800000in}{0.528000in}}%
\pgfpathlineto{\pgfqpoint{5.760000in}{0.528000in}}%
\pgfpathlineto{\pgfqpoint{5.760000in}{4.224000in}}%
\pgfpathlineto{\pgfqpoint{0.800000in}{4.224000in}}%
\pgfpathlineto{\pgfqpoint{0.800000in}{0.528000in}}%
\pgfpathclose%
\pgfusepath{fill}%
\end{pgfscope}%
\begin{pgfscope}%
\pgfpathrectangle{\pgfqpoint{0.800000in}{0.528000in}}{\pgfqpoint{4.960000in}{3.696000in}}%
\pgfusepath{clip}%
\pgfsetbuttcap%
\pgfsetmiterjoin%
\definecolor{currentfill}{rgb}{0.500000,0.000000,1.000000}%
\pgfsetfillcolor{currentfill}%
\pgfsetlinewidth{0.000000pt}%
\definecolor{currentstroke}{rgb}{0.000000,0.000000,0.000000}%
\pgfsetstrokecolor{currentstroke}%
\pgfsetstrokeopacity{0.000000}%
\pgfsetdash{}{0pt}%
\pgfpathmoveto{\pgfqpoint{1.025455in}{0.528000in}}%
\pgfpathlineto{\pgfqpoint{1.164501in}{0.528000in}}%
\pgfpathlineto{\pgfqpoint{1.164501in}{4.048000in}}%
\pgfpathlineto{\pgfqpoint{1.025455in}{4.048000in}}%
\pgfpathlineto{\pgfqpoint{1.025455in}{0.528000in}}%
\pgfpathclose%
\pgfusepath{fill}%
\end{pgfscope}%
\begin{pgfscope}%
\pgfpathrectangle{\pgfqpoint{0.800000in}{0.528000in}}{\pgfqpoint{4.960000in}{3.696000in}}%
\pgfusepath{clip}%
\pgfsetbuttcap%
\pgfsetmiterjoin%
\definecolor{currentfill}{rgb}{0.500000,0.000000,1.000000}%
\pgfsetfillcolor{currentfill}%
\pgfsetlinewidth{0.000000pt}%
\definecolor{currentstroke}{rgb}{0.000000,0.000000,0.000000}%
\pgfsetstrokecolor{currentstroke}%
\pgfsetstrokeopacity{0.000000}%
\pgfsetdash{}{0pt}%
\pgfpathmoveto{\pgfqpoint{1.422731in}{0.528000in}}%
\pgfpathlineto{\pgfqpoint{1.561778in}{0.528000in}}%
\pgfpathlineto{\pgfqpoint{1.561778in}{3.860147in}}%
\pgfpathlineto{\pgfqpoint{1.422731in}{3.860147in}}%
\pgfpathlineto{\pgfqpoint{1.422731in}{0.528000in}}%
\pgfpathclose%
\pgfusepath{fill}%
\end{pgfscope}%
\begin{pgfscope}%
\pgfpathrectangle{\pgfqpoint{0.800000in}{0.528000in}}{\pgfqpoint{4.960000in}{3.696000in}}%
\pgfusepath{clip}%
\pgfsetbuttcap%
\pgfsetmiterjoin%
\definecolor{currentfill}{rgb}{0.500000,0.000000,1.000000}%
\pgfsetfillcolor{currentfill}%
\pgfsetlinewidth{0.000000pt}%
\definecolor{currentstroke}{rgb}{0.000000,0.000000,0.000000}%
\pgfsetstrokecolor{currentstroke}%
\pgfsetstrokeopacity{0.000000}%
\pgfsetdash{}{0pt}%
\pgfpathmoveto{\pgfqpoint{1.820008in}{0.528000in}}%
\pgfpathlineto{\pgfqpoint{1.959055in}{0.528000in}}%
\pgfpathlineto{\pgfqpoint{1.959055in}{3.649931in}}%
\pgfpathlineto{\pgfqpoint{1.820008in}{3.649931in}}%
\pgfpathlineto{\pgfqpoint{1.820008in}{0.528000in}}%
\pgfpathclose%
\pgfusepath{fill}%
\end{pgfscope}%
\begin{pgfscope}%
\pgfpathrectangle{\pgfqpoint{0.800000in}{0.528000in}}{\pgfqpoint{4.960000in}{3.696000in}}%
\pgfusepath{clip}%
\pgfsetbuttcap%
\pgfsetmiterjoin%
\definecolor{currentfill}{rgb}{0.500000,0.000000,1.000000}%
\pgfsetfillcolor{currentfill}%
\pgfsetlinewidth{0.000000pt}%
\definecolor{currentstroke}{rgb}{0.000000,0.000000,0.000000}%
\pgfsetstrokecolor{currentstroke}%
\pgfsetstrokeopacity{0.000000}%
\pgfsetdash{}{0pt}%
\pgfpathmoveto{\pgfqpoint{2.217285in}{0.528000in}}%
\pgfpathlineto{\pgfqpoint{2.356332in}{0.528000in}}%
\pgfpathlineto{\pgfqpoint{2.356332in}{3.296590in}}%
\pgfpathlineto{\pgfqpoint{2.217285in}{3.296590in}}%
\pgfpathlineto{\pgfqpoint{2.217285in}{0.528000in}}%
\pgfpathclose%
\pgfusepath{fill}%
\end{pgfscope}%
\begin{pgfscope}%
\pgfpathrectangle{\pgfqpoint{0.800000in}{0.528000in}}{\pgfqpoint{4.960000in}{3.696000in}}%
\pgfusepath{clip}%
\pgfsetbuttcap%
\pgfsetmiterjoin%
\definecolor{currentfill}{rgb}{0.500000,0.000000,1.000000}%
\pgfsetfillcolor{currentfill}%
\pgfsetlinewidth{0.000000pt}%
\definecolor{currentstroke}{rgb}{0.000000,0.000000,0.000000}%
\pgfsetstrokecolor{currentstroke}%
\pgfsetstrokeopacity{0.000000}%
\pgfsetdash{}{0pt}%
\pgfpathmoveto{\pgfqpoint{2.614561in}{0.528000in}}%
\pgfpathlineto{\pgfqpoint{2.753608in}{0.528000in}}%
\pgfpathlineto{\pgfqpoint{2.753608in}{3.131100in}}%
\pgfpathlineto{\pgfqpoint{2.614561in}{3.131100in}}%
\pgfpathlineto{\pgfqpoint{2.614561in}{0.528000in}}%
\pgfpathclose%
\pgfusepath{fill}%
\end{pgfscope}%
\begin{pgfscope}%
\pgfpathrectangle{\pgfqpoint{0.800000in}{0.528000in}}{\pgfqpoint{4.960000in}{3.696000in}}%
\pgfusepath{clip}%
\pgfsetbuttcap%
\pgfsetmiterjoin%
\definecolor{currentfill}{rgb}{0.500000,0.000000,1.000000}%
\pgfsetfillcolor{currentfill}%
\pgfsetlinewidth{0.000000pt}%
\definecolor{currentstroke}{rgb}{0.000000,0.000000,0.000000}%
\pgfsetstrokecolor{currentstroke}%
\pgfsetstrokeopacity{0.000000}%
\pgfsetdash{}{0pt}%
\pgfpathmoveto{\pgfqpoint{3.011838in}{0.528000in}}%
\pgfpathlineto{\pgfqpoint{3.150885in}{0.528000in}}%
\pgfpathlineto{\pgfqpoint{3.150885in}{2.876158in}}%
\pgfpathlineto{\pgfqpoint{3.011838in}{2.876158in}}%
\pgfpathlineto{\pgfqpoint{3.011838in}{0.528000in}}%
\pgfpathclose%
\pgfusepath{fill}%
\end{pgfscope}%
\begin{pgfscope}%
\pgfpathrectangle{\pgfqpoint{0.800000in}{0.528000in}}{\pgfqpoint{4.960000in}{3.696000in}}%
\pgfusepath{clip}%
\pgfsetbuttcap%
\pgfsetmiterjoin%
\definecolor{currentfill}{rgb}{0.500000,0.000000,1.000000}%
\pgfsetfillcolor{currentfill}%
\pgfsetlinewidth{0.000000pt}%
\definecolor{currentstroke}{rgb}{0.000000,0.000000,0.000000}%
\pgfsetstrokecolor{currentstroke}%
\pgfsetstrokeopacity{0.000000}%
\pgfsetdash{}{0pt}%
\pgfpathmoveto{\pgfqpoint{3.409115in}{0.528000in}}%
\pgfpathlineto{\pgfqpoint{3.548162in}{0.528000in}}%
\pgfpathlineto{\pgfqpoint{3.548162in}{2.652524in}}%
\pgfpathlineto{\pgfqpoint{3.409115in}{2.652524in}}%
\pgfpathlineto{\pgfqpoint{3.409115in}{0.528000in}}%
\pgfpathclose%
\pgfusepath{fill}%
\end{pgfscope}%
\begin{pgfscope}%
\pgfpathrectangle{\pgfqpoint{0.800000in}{0.528000in}}{\pgfqpoint{4.960000in}{3.696000in}}%
\pgfusepath{clip}%
\pgfsetbuttcap%
\pgfsetmiterjoin%
\definecolor{currentfill}{rgb}{0.500000,0.000000,1.000000}%
\pgfsetfillcolor{currentfill}%
\pgfsetlinewidth{0.000000pt}%
\definecolor{currentstroke}{rgb}{0.000000,0.000000,0.000000}%
\pgfsetstrokecolor{currentstroke}%
\pgfsetstrokeopacity{0.000000}%
\pgfsetdash{}{0pt}%
\pgfpathmoveto{\pgfqpoint{3.806392in}{0.528000in}}%
\pgfpathlineto{\pgfqpoint{3.945439in}{0.528000in}}%
\pgfpathlineto{\pgfqpoint{3.945439in}{2.321545in}}%
\pgfpathlineto{\pgfqpoint{3.806392in}{2.321545in}}%
\pgfpathlineto{\pgfqpoint{3.806392in}{0.528000in}}%
\pgfpathclose%
\pgfusepath{fill}%
\end{pgfscope}%
\begin{pgfscope}%
\pgfpathrectangle{\pgfqpoint{0.800000in}{0.528000in}}{\pgfqpoint{4.960000in}{3.696000in}}%
\pgfusepath{clip}%
\pgfsetbuttcap%
\pgfsetmiterjoin%
\definecolor{currentfill}{rgb}{0.500000,0.000000,1.000000}%
\pgfsetfillcolor{currentfill}%
\pgfsetlinewidth{0.000000pt}%
\definecolor{currentstroke}{rgb}{0.000000,0.000000,0.000000}%
\pgfsetstrokecolor{currentstroke}%
\pgfsetstrokeopacity{0.000000}%
\pgfsetdash{}{0pt}%
\pgfpathmoveto{\pgfqpoint{4.203668in}{0.528000in}}%
\pgfpathlineto{\pgfqpoint{4.342715in}{0.528000in}}%
\pgfpathlineto{\pgfqpoint{4.342715in}{1.999512in}}%
\pgfpathlineto{\pgfqpoint{4.203668in}{1.999512in}}%
\pgfpathlineto{\pgfqpoint{4.203668in}{0.528000in}}%
\pgfpathclose%
\pgfusepath{fill}%
\end{pgfscope}%
\begin{pgfscope}%
\pgfpathrectangle{\pgfqpoint{0.800000in}{0.528000in}}{\pgfqpoint{4.960000in}{3.696000in}}%
\pgfusepath{clip}%
\pgfsetbuttcap%
\pgfsetmiterjoin%
\definecolor{currentfill}{rgb}{0.500000,0.000000,1.000000}%
\pgfsetfillcolor{currentfill}%
\pgfsetlinewidth{0.000000pt}%
\definecolor{currentstroke}{rgb}{0.000000,0.000000,0.000000}%
\pgfsetstrokecolor{currentstroke}%
\pgfsetstrokeopacity{0.000000}%
\pgfsetdash{}{0pt}%
\pgfpathmoveto{\pgfqpoint{4.600945in}{0.528000in}}%
\pgfpathlineto{\pgfqpoint{4.739992in}{0.528000in}}%
\pgfpathlineto{\pgfqpoint{4.739992in}{1.699842in}}%
\pgfpathlineto{\pgfqpoint{4.600945in}{1.699842in}}%
\pgfpathlineto{\pgfqpoint{4.600945in}{0.528000in}}%
\pgfpathclose%
\pgfusepath{fill}%
\end{pgfscope}%
\begin{pgfscope}%
\pgfpathrectangle{\pgfqpoint{0.800000in}{0.528000in}}{\pgfqpoint{4.960000in}{3.696000in}}%
\pgfusepath{clip}%
\pgfsetbuttcap%
\pgfsetmiterjoin%
\definecolor{currentfill}{rgb}{0.500000,0.000000,1.000000}%
\pgfsetfillcolor{currentfill}%
\pgfsetlinewidth{0.000000pt}%
\definecolor{currentstroke}{rgb}{0.000000,0.000000,0.000000}%
\pgfsetstrokecolor{currentstroke}%
\pgfsetstrokeopacity{0.000000}%
\pgfsetdash{}{0pt}%
\pgfpathmoveto{\pgfqpoint{4.998222in}{0.528000in}}%
\pgfpathlineto{\pgfqpoint{5.137269in}{0.528000in}}%
\pgfpathlineto{\pgfqpoint{5.137269in}{1.279410in}}%
\pgfpathlineto{\pgfqpoint{4.998222in}{1.279410in}}%
\pgfpathlineto{\pgfqpoint{4.998222in}{0.528000in}}%
\pgfpathclose%
\pgfusepath{fill}%
\end{pgfscope}%
\begin{pgfscope}%
\pgfpathrectangle{\pgfqpoint{0.800000in}{0.528000in}}{\pgfqpoint{4.960000in}{3.696000in}}%
\pgfusepath{clip}%
\pgfsetbuttcap%
\pgfsetmiterjoin%
\definecolor{currentfill}{rgb}{0.500000,0.000000,1.000000}%
\pgfsetfillcolor{currentfill}%
\pgfsetlinewidth{0.000000pt}%
\definecolor{currentstroke}{rgb}{0.000000,0.000000,0.000000}%
\pgfsetstrokecolor{currentstroke}%
\pgfsetstrokeopacity{0.000000}%
\pgfsetdash{}{0pt}%
\pgfpathmoveto{\pgfqpoint{5.395499in}{0.528000in}}%
\pgfpathlineto{\pgfqpoint{5.534545in}{0.528000in}}%
\pgfpathlineto{\pgfqpoint{5.534545in}{0.961850in}}%
\pgfpathlineto{\pgfqpoint{5.395499in}{0.961850in}}%
\pgfpathlineto{\pgfqpoint{5.395499in}{0.528000in}}%
\pgfpathclose%
\pgfusepath{fill}%
\end{pgfscope}%
\begin{pgfscope}%
\pgfpathrectangle{\pgfqpoint{0.800000in}{0.528000in}}{\pgfqpoint{4.960000in}{3.696000in}}%
\pgfusepath{clip}%
\pgfsetbuttcap%
\pgfsetmiterjoin%
\definecolor{currentfill}{rgb}{0.001961,0.709281,0.923289}%
\pgfsetfillcolor{currentfill}%
\pgfsetlinewidth{0.000000pt}%
\definecolor{currentstroke}{rgb}{0.000000,0.000000,0.000000}%
\pgfsetstrokecolor{currentstroke}%
\pgfsetstrokeopacity{0.000000}%
\pgfsetdash{}{0pt}%
\pgfpathmoveto{\pgfqpoint{1.025455in}{0.528000in}}%
\pgfpathlineto{\pgfqpoint{1.164501in}{0.528000in}}%
\pgfpathlineto{\pgfqpoint{1.164501in}{3.882511in}}%
\pgfpathlineto{\pgfqpoint{1.025455in}{3.882511in}}%
\pgfpathlineto{\pgfqpoint{1.025455in}{0.528000in}}%
\pgfpathclose%
\pgfusepath{fill}%
\end{pgfscope}%
\begin{pgfscope}%
\pgfpathrectangle{\pgfqpoint{0.800000in}{0.528000in}}{\pgfqpoint{4.960000in}{3.696000in}}%
\pgfusepath{clip}%
\pgfsetbuttcap%
\pgfsetmiterjoin%
\definecolor{currentfill}{rgb}{0.001961,0.709281,0.923289}%
\pgfsetfillcolor{currentfill}%
\pgfsetlinewidth{0.000000pt}%
\definecolor{currentstroke}{rgb}{0.000000,0.000000,0.000000}%
\pgfsetstrokecolor{currentstroke}%
\pgfsetstrokeopacity{0.000000}%
\pgfsetdash{}{0pt}%
\pgfpathmoveto{\pgfqpoint{1.422731in}{0.528000in}}%
\pgfpathlineto{\pgfqpoint{1.561778in}{0.528000in}}%
\pgfpathlineto{\pgfqpoint{1.561778in}{3.717022in}}%
\pgfpathlineto{\pgfqpoint{1.422731in}{3.717022in}}%
\pgfpathlineto{\pgfqpoint{1.422731in}{0.528000in}}%
\pgfpathclose%
\pgfusepath{fill}%
\end{pgfscope}%
\begin{pgfscope}%
\pgfpathrectangle{\pgfqpoint{0.800000in}{0.528000in}}{\pgfqpoint{4.960000in}{3.696000in}}%
\pgfusepath{clip}%
\pgfsetbuttcap%
\pgfsetmiterjoin%
\definecolor{currentfill}{rgb}{0.001961,0.709281,0.923289}%
\pgfsetfillcolor{currentfill}%
\pgfsetlinewidth{0.000000pt}%
\definecolor{currentstroke}{rgb}{0.000000,0.000000,0.000000}%
\pgfsetstrokecolor{currentstroke}%
\pgfsetstrokeopacity{0.000000}%
\pgfsetdash{}{0pt}%
\pgfpathmoveto{\pgfqpoint{1.820008in}{0.528000in}}%
\pgfpathlineto{\pgfqpoint{1.959055in}{0.528000in}}%
\pgfpathlineto{\pgfqpoint{1.959055in}{3.515751in}}%
\pgfpathlineto{\pgfqpoint{1.820008in}{3.515751in}}%
\pgfpathlineto{\pgfqpoint{1.820008in}{0.528000in}}%
\pgfpathclose%
\pgfusepath{fill}%
\end{pgfscope}%
\begin{pgfscope}%
\pgfpathrectangle{\pgfqpoint{0.800000in}{0.528000in}}{\pgfqpoint{4.960000in}{3.696000in}}%
\pgfusepath{clip}%
\pgfsetbuttcap%
\pgfsetmiterjoin%
\definecolor{currentfill}{rgb}{0.001961,0.709281,0.923289}%
\pgfsetfillcolor{currentfill}%
\pgfsetlinewidth{0.000000pt}%
\definecolor{currentstroke}{rgb}{0.000000,0.000000,0.000000}%
\pgfsetstrokecolor{currentstroke}%
\pgfsetstrokeopacity{0.000000}%
\pgfsetdash{}{0pt}%
\pgfpathmoveto{\pgfqpoint{2.217285in}{0.528000in}}%
\pgfpathlineto{\pgfqpoint{2.356332in}{0.528000in}}%
\pgfpathlineto{\pgfqpoint{2.356332in}{3.184773in}}%
\pgfpathlineto{\pgfqpoint{2.217285in}{3.184773in}}%
\pgfpathlineto{\pgfqpoint{2.217285in}{0.528000in}}%
\pgfpathclose%
\pgfusepath{fill}%
\end{pgfscope}%
\begin{pgfscope}%
\pgfpathrectangle{\pgfqpoint{0.800000in}{0.528000in}}{\pgfqpoint{4.960000in}{3.696000in}}%
\pgfusepath{clip}%
\pgfsetbuttcap%
\pgfsetmiterjoin%
\definecolor{currentfill}{rgb}{0.001961,0.709281,0.923289}%
\pgfsetfillcolor{currentfill}%
\pgfsetlinewidth{0.000000pt}%
\definecolor{currentstroke}{rgb}{0.000000,0.000000,0.000000}%
\pgfsetstrokecolor{currentstroke}%
\pgfsetstrokeopacity{0.000000}%
\pgfsetdash{}{0pt}%
\pgfpathmoveto{\pgfqpoint{2.614561in}{0.528000in}}%
\pgfpathlineto{\pgfqpoint{2.753608in}{0.528000in}}%
\pgfpathlineto{\pgfqpoint{2.753608in}{3.028229in}}%
\pgfpathlineto{\pgfqpoint{2.614561in}{3.028229in}}%
\pgfpathlineto{\pgfqpoint{2.614561in}{0.528000in}}%
\pgfpathclose%
\pgfusepath{fill}%
\end{pgfscope}%
\begin{pgfscope}%
\pgfpathrectangle{\pgfqpoint{0.800000in}{0.528000in}}{\pgfqpoint{4.960000in}{3.696000in}}%
\pgfusepath{clip}%
\pgfsetbuttcap%
\pgfsetmiterjoin%
\definecolor{currentfill}{rgb}{0.001961,0.709281,0.923289}%
\pgfsetfillcolor{currentfill}%
\pgfsetlinewidth{0.000000pt}%
\definecolor{currentstroke}{rgb}{0.000000,0.000000,0.000000}%
\pgfsetstrokecolor{currentstroke}%
\pgfsetstrokeopacity{0.000000}%
\pgfsetdash{}{0pt}%
\pgfpathmoveto{\pgfqpoint{3.011838in}{0.528000in}}%
\pgfpathlineto{\pgfqpoint{3.150885in}{0.528000in}}%
\pgfpathlineto{\pgfqpoint{3.150885in}{2.809067in}}%
\pgfpathlineto{\pgfqpoint{3.011838in}{2.809067in}}%
\pgfpathlineto{\pgfqpoint{3.011838in}{0.528000in}}%
\pgfpathclose%
\pgfusepath{fill}%
\end{pgfscope}%
\begin{pgfscope}%
\pgfpathrectangle{\pgfqpoint{0.800000in}{0.528000in}}{\pgfqpoint{4.960000in}{3.696000in}}%
\pgfusepath{clip}%
\pgfsetbuttcap%
\pgfsetmiterjoin%
\definecolor{currentfill}{rgb}{0.001961,0.709281,0.923289}%
\pgfsetfillcolor{currentfill}%
\pgfsetlinewidth{0.000000pt}%
\definecolor{currentstroke}{rgb}{0.000000,0.000000,0.000000}%
\pgfsetstrokecolor{currentstroke}%
\pgfsetstrokeopacity{0.000000}%
\pgfsetdash{}{0pt}%
\pgfpathmoveto{\pgfqpoint{3.409115in}{0.528000in}}%
\pgfpathlineto{\pgfqpoint{3.548162in}{0.528000in}}%
\pgfpathlineto{\pgfqpoint{3.548162in}{2.598851in}}%
\pgfpathlineto{\pgfqpoint{3.409115in}{2.598851in}}%
\pgfpathlineto{\pgfqpoint{3.409115in}{0.528000in}}%
\pgfpathclose%
\pgfusepath{fill}%
\end{pgfscope}%
\begin{pgfscope}%
\pgfpathrectangle{\pgfqpoint{0.800000in}{0.528000in}}{\pgfqpoint{4.960000in}{3.696000in}}%
\pgfusepath{clip}%
\pgfsetbuttcap%
\pgfsetmiterjoin%
\definecolor{currentfill}{rgb}{0.001961,0.709281,0.923289}%
\pgfsetfillcolor{currentfill}%
\pgfsetlinewidth{0.000000pt}%
\definecolor{currentstroke}{rgb}{0.000000,0.000000,0.000000}%
\pgfsetstrokecolor{currentstroke}%
\pgfsetstrokeopacity{0.000000}%
\pgfsetdash{}{0pt}%
\pgfpathmoveto{\pgfqpoint{3.806392in}{0.528000in}}%
\pgfpathlineto{\pgfqpoint{3.945439in}{0.528000in}}%
\pgfpathlineto{\pgfqpoint{3.945439in}{2.267873in}}%
\pgfpathlineto{\pgfqpoint{3.806392in}{2.267873in}}%
\pgfpathlineto{\pgfqpoint{3.806392in}{0.528000in}}%
\pgfpathclose%
\pgfusepath{fill}%
\end{pgfscope}%
\begin{pgfscope}%
\pgfpathrectangle{\pgfqpoint{0.800000in}{0.528000in}}{\pgfqpoint{4.960000in}{3.696000in}}%
\pgfusepath{clip}%
\pgfsetbuttcap%
\pgfsetmiterjoin%
\definecolor{currentfill}{rgb}{0.001961,0.709281,0.923289}%
\pgfsetfillcolor{currentfill}%
\pgfsetlinewidth{0.000000pt}%
\definecolor{currentstroke}{rgb}{0.000000,0.000000,0.000000}%
\pgfsetstrokecolor{currentstroke}%
\pgfsetstrokeopacity{0.000000}%
\pgfsetdash{}{0pt}%
\pgfpathmoveto{\pgfqpoint{4.203668in}{0.528000in}}%
\pgfpathlineto{\pgfqpoint{4.342715in}{0.528000in}}%
\pgfpathlineto{\pgfqpoint{4.342715in}{1.954785in}}%
\pgfpathlineto{\pgfqpoint{4.203668in}{1.954785in}}%
\pgfpathlineto{\pgfqpoint{4.203668in}{0.528000in}}%
\pgfpathclose%
\pgfusepath{fill}%
\end{pgfscope}%
\begin{pgfscope}%
\pgfpathrectangle{\pgfqpoint{0.800000in}{0.528000in}}{\pgfqpoint{4.960000in}{3.696000in}}%
\pgfusepath{clip}%
\pgfsetbuttcap%
\pgfsetmiterjoin%
\definecolor{currentfill}{rgb}{0.001961,0.709281,0.923289}%
\pgfsetfillcolor{currentfill}%
\pgfsetlinewidth{0.000000pt}%
\definecolor{currentstroke}{rgb}{0.000000,0.000000,0.000000}%
\pgfsetstrokecolor{currentstroke}%
\pgfsetstrokeopacity{0.000000}%
\pgfsetdash{}{0pt}%
\pgfpathmoveto{\pgfqpoint{4.600945in}{0.528000in}}%
\pgfpathlineto{\pgfqpoint{4.739992in}{0.528000in}}%
\pgfpathlineto{\pgfqpoint{4.739992in}{1.677479in}}%
\pgfpathlineto{\pgfqpoint{4.600945in}{1.677479in}}%
\pgfpathlineto{\pgfqpoint{4.600945in}{0.528000in}}%
\pgfpathclose%
\pgfusepath{fill}%
\end{pgfscope}%
\begin{pgfscope}%
\pgfpathrectangle{\pgfqpoint{0.800000in}{0.528000in}}{\pgfqpoint{4.960000in}{3.696000in}}%
\pgfusepath{clip}%
\pgfsetbuttcap%
\pgfsetmiterjoin%
\definecolor{currentfill}{rgb}{0.001961,0.709281,0.923289}%
\pgfsetfillcolor{currentfill}%
\pgfsetlinewidth{0.000000pt}%
\definecolor{currentstroke}{rgb}{0.000000,0.000000,0.000000}%
\pgfsetstrokecolor{currentstroke}%
\pgfsetstrokeopacity{0.000000}%
\pgfsetdash{}{0pt}%
\pgfpathmoveto{\pgfqpoint{4.998222in}{0.528000in}}%
\pgfpathlineto{\pgfqpoint{5.137269in}{0.528000in}}%
\pgfpathlineto{\pgfqpoint{5.137269in}{1.270465in}}%
\pgfpathlineto{\pgfqpoint{4.998222in}{1.270465in}}%
\pgfpathlineto{\pgfqpoint{4.998222in}{0.528000in}}%
\pgfpathclose%
\pgfusepath{fill}%
\end{pgfscope}%
\begin{pgfscope}%
\pgfpathrectangle{\pgfqpoint{0.800000in}{0.528000in}}{\pgfqpoint{4.960000in}{3.696000in}}%
\pgfusepath{clip}%
\pgfsetbuttcap%
\pgfsetmiterjoin%
\definecolor{currentfill}{rgb}{0.001961,0.709281,0.923289}%
\pgfsetfillcolor{currentfill}%
\pgfsetlinewidth{0.000000pt}%
\definecolor{currentstroke}{rgb}{0.000000,0.000000,0.000000}%
\pgfsetstrokecolor{currentstroke}%
\pgfsetstrokeopacity{0.000000}%
\pgfsetdash{}{0pt}%
\pgfpathmoveto{\pgfqpoint{5.395499in}{0.528000in}}%
\pgfpathlineto{\pgfqpoint{5.534545in}{0.528000in}}%
\pgfpathlineto{\pgfqpoint{5.534545in}{0.961850in}}%
\pgfpathlineto{\pgfqpoint{5.395499in}{0.961850in}}%
\pgfpathlineto{\pgfqpoint{5.395499in}{0.528000in}}%
\pgfpathclose%
\pgfusepath{fill}%
\end{pgfscope}%
\begin{pgfscope}%
\pgfpathrectangle{\pgfqpoint{0.800000in}{0.528000in}}{\pgfqpoint{4.960000in}{3.696000in}}%
\pgfusepath{clip}%
\pgfsetbuttcap%
\pgfsetmiterjoin%
\definecolor{currentfill}{rgb}{0.503922,0.999981,0.704926}%
\pgfsetfillcolor{currentfill}%
\pgfsetlinewidth{0.000000pt}%
\definecolor{currentstroke}{rgb}{0.000000,0.000000,0.000000}%
\pgfsetstrokecolor{currentstroke}%
\pgfsetstrokeopacity{0.000000}%
\pgfsetdash{}{0pt}%
\pgfpathmoveto{\pgfqpoint{1.025455in}{0.528000in}}%
\pgfpathlineto{\pgfqpoint{1.164501in}{0.528000in}}%
\pgfpathlineto{\pgfqpoint{1.164501in}{3.725967in}}%
\pgfpathlineto{\pgfqpoint{1.025455in}{3.725967in}}%
\pgfpathlineto{\pgfqpoint{1.025455in}{0.528000in}}%
\pgfpathclose%
\pgfusepath{fill}%
\end{pgfscope}%
\begin{pgfscope}%
\pgfpathrectangle{\pgfqpoint{0.800000in}{0.528000in}}{\pgfqpoint{4.960000in}{3.696000in}}%
\pgfusepath{clip}%
\pgfsetbuttcap%
\pgfsetmiterjoin%
\definecolor{currentfill}{rgb}{0.503922,0.999981,0.704926}%
\pgfsetfillcolor{currentfill}%
\pgfsetlinewidth{0.000000pt}%
\definecolor{currentstroke}{rgb}{0.000000,0.000000,0.000000}%
\pgfsetstrokecolor{currentstroke}%
\pgfsetstrokeopacity{0.000000}%
\pgfsetdash{}{0pt}%
\pgfpathmoveto{\pgfqpoint{1.422731in}{0.528000in}}%
\pgfpathlineto{\pgfqpoint{1.561778in}{0.528000in}}%
\pgfpathlineto{\pgfqpoint{1.561778in}{3.582841in}}%
\pgfpathlineto{\pgfqpoint{1.422731in}{3.582841in}}%
\pgfpathlineto{\pgfqpoint{1.422731in}{0.528000in}}%
\pgfpathclose%
\pgfusepath{fill}%
\end{pgfscope}%
\begin{pgfscope}%
\pgfpathrectangle{\pgfqpoint{0.800000in}{0.528000in}}{\pgfqpoint{4.960000in}{3.696000in}}%
\pgfusepath{clip}%
\pgfsetbuttcap%
\pgfsetmiterjoin%
\definecolor{currentfill}{rgb}{0.503922,0.999981,0.704926}%
\pgfsetfillcolor{currentfill}%
\pgfsetlinewidth{0.000000pt}%
\definecolor{currentstroke}{rgb}{0.000000,0.000000,0.000000}%
\pgfsetstrokecolor{currentstroke}%
\pgfsetstrokeopacity{0.000000}%
\pgfsetdash{}{0pt}%
\pgfpathmoveto{\pgfqpoint{1.820008in}{0.528000in}}%
\pgfpathlineto{\pgfqpoint{1.959055in}{0.528000in}}%
\pgfpathlineto{\pgfqpoint{1.959055in}{3.417352in}}%
\pgfpathlineto{\pgfqpoint{1.820008in}{3.417352in}}%
\pgfpathlineto{\pgfqpoint{1.820008in}{0.528000in}}%
\pgfpathclose%
\pgfusepath{fill}%
\end{pgfscope}%
\begin{pgfscope}%
\pgfpathrectangle{\pgfqpoint{0.800000in}{0.528000in}}{\pgfqpoint{4.960000in}{3.696000in}}%
\pgfusepath{clip}%
\pgfsetbuttcap%
\pgfsetmiterjoin%
\definecolor{currentfill}{rgb}{0.503922,0.999981,0.704926}%
\pgfsetfillcolor{currentfill}%
\pgfsetlinewidth{0.000000pt}%
\definecolor{currentstroke}{rgb}{0.000000,0.000000,0.000000}%
\pgfsetstrokecolor{currentstroke}%
\pgfsetstrokeopacity{0.000000}%
\pgfsetdash{}{0pt}%
\pgfpathmoveto{\pgfqpoint{2.217285in}{0.528000in}}%
\pgfpathlineto{\pgfqpoint{2.356332in}{0.528000in}}%
\pgfpathlineto{\pgfqpoint{2.356332in}{3.095319in}}%
\pgfpathlineto{\pgfqpoint{2.217285in}{3.095319in}}%
\pgfpathlineto{\pgfqpoint{2.217285in}{0.528000in}}%
\pgfpathclose%
\pgfusepath{fill}%
\end{pgfscope}%
\begin{pgfscope}%
\pgfpathrectangle{\pgfqpoint{0.800000in}{0.528000in}}{\pgfqpoint{4.960000in}{3.696000in}}%
\pgfusepath{clip}%
\pgfsetbuttcap%
\pgfsetmiterjoin%
\definecolor{currentfill}{rgb}{0.503922,0.999981,0.704926}%
\pgfsetfillcolor{currentfill}%
\pgfsetlinewidth{0.000000pt}%
\definecolor{currentstroke}{rgb}{0.000000,0.000000,0.000000}%
\pgfsetstrokecolor{currentstroke}%
\pgfsetstrokeopacity{0.000000}%
\pgfsetdash{}{0pt}%
\pgfpathmoveto{\pgfqpoint{2.614561in}{0.528000in}}%
\pgfpathlineto{\pgfqpoint{2.753608in}{0.528000in}}%
\pgfpathlineto{\pgfqpoint{2.753608in}{2.952193in}}%
\pgfpathlineto{\pgfqpoint{2.614561in}{2.952193in}}%
\pgfpathlineto{\pgfqpoint{2.614561in}{0.528000in}}%
\pgfpathclose%
\pgfusepath{fill}%
\end{pgfscope}%
\begin{pgfscope}%
\pgfpathrectangle{\pgfqpoint{0.800000in}{0.528000in}}{\pgfqpoint{4.960000in}{3.696000in}}%
\pgfusepath{clip}%
\pgfsetbuttcap%
\pgfsetmiterjoin%
\definecolor{currentfill}{rgb}{0.503922,0.999981,0.704926}%
\pgfsetfillcolor{currentfill}%
\pgfsetlinewidth{0.000000pt}%
\definecolor{currentstroke}{rgb}{0.000000,0.000000,0.000000}%
\pgfsetstrokecolor{currentstroke}%
\pgfsetstrokeopacity{0.000000}%
\pgfsetdash{}{0pt}%
\pgfpathmoveto{\pgfqpoint{3.011838in}{0.528000in}}%
\pgfpathlineto{\pgfqpoint{3.150885in}{0.528000in}}%
\pgfpathlineto{\pgfqpoint{3.150885in}{2.764341in}}%
\pgfpathlineto{\pgfqpoint{3.011838in}{2.764341in}}%
\pgfpathlineto{\pgfqpoint{3.011838in}{0.528000in}}%
\pgfpathclose%
\pgfusepath{fill}%
\end{pgfscope}%
\begin{pgfscope}%
\pgfpathrectangle{\pgfqpoint{0.800000in}{0.528000in}}{\pgfqpoint{4.960000in}{3.696000in}}%
\pgfusepath{clip}%
\pgfsetbuttcap%
\pgfsetmiterjoin%
\definecolor{currentfill}{rgb}{0.503922,0.999981,0.704926}%
\pgfsetfillcolor{currentfill}%
\pgfsetlinewidth{0.000000pt}%
\definecolor{currentstroke}{rgb}{0.000000,0.000000,0.000000}%
\pgfsetstrokecolor{currentstroke}%
\pgfsetstrokeopacity{0.000000}%
\pgfsetdash{}{0pt}%
\pgfpathmoveto{\pgfqpoint{3.409115in}{0.528000in}}%
\pgfpathlineto{\pgfqpoint{3.548162in}{0.528000in}}%
\pgfpathlineto{\pgfqpoint{3.548162in}{2.576488in}}%
\pgfpathlineto{\pgfqpoint{3.409115in}{2.576488in}}%
\pgfpathlineto{\pgfqpoint{3.409115in}{0.528000in}}%
\pgfpathclose%
\pgfusepath{fill}%
\end{pgfscope}%
\begin{pgfscope}%
\pgfpathrectangle{\pgfqpoint{0.800000in}{0.528000in}}{\pgfqpoint{4.960000in}{3.696000in}}%
\pgfusepath{clip}%
\pgfsetbuttcap%
\pgfsetmiterjoin%
\definecolor{currentfill}{rgb}{0.503922,0.999981,0.704926}%
\pgfsetfillcolor{currentfill}%
\pgfsetlinewidth{0.000000pt}%
\definecolor{currentstroke}{rgb}{0.000000,0.000000,0.000000}%
\pgfsetstrokecolor{currentstroke}%
\pgfsetstrokeopacity{0.000000}%
\pgfsetdash{}{0pt}%
\pgfpathmoveto{\pgfqpoint{3.806392in}{0.528000in}}%
\pgfpathlineto{\pgfqpoint{3.945439in}{0.528000in}}%
\pgfpathlineto{\pgfqpoint{3.945439in}{2.245510in}}%
\pgfpathlineto{\pgfqpoint{3.806392in}{2.245510in}}%
\pgfpathlineto{\pgfqpoint{3.806392in}{0.528000in}}%
\pgfpathclose%
\pgfusepath{fill}%
\end{pgfscope}%
\begin{pgfscope}%
\pgfpathrectangle{\pgfqpoint{0.800000in}{0.528000in}}{\pgfqpoint{4.960000in}{3.696000in}}%
\pgfusepath{clip}%
\pgfsetbuttcap%
\pgfsetmiterjoin%
\definecolor{currentfill}{rgb}{0.503922,0.999981,0.704926}%
\pgfsetfillcolor{currentfill}%
\pgfsetlinewidth{0.000000pt}%
\definecolor{currentstroke}{rgb}{0.000000,0.000000,0.000000}%
\pgfsetstrokecolor{currentstroke}%
\pgfsetstrokeopacity{0.000000}%
\pgfsetdash{}{0pt}%
\pgfpathmoveto{\pgfqpoint{4.203668in}{0.528000in}}%
\pgfpathlineto{\pgfqpoint{4.342715in}{0.528000in}}%
\pgfpathlineto{\pgfqpoint{4.342715in}{1.954785in}}%
\pgfpathlineto{\pgfqpoint{4.203668in}{1.954785in}}%
\pgfpathlineto{\pgfqpoint{4.203668in}{0.528000in}}%
\pgfpathclose%
\pgfusepath{fill}%
\end{pgfscope}%
\begin{pgfscope}%
\pgfpathrectangle{\pgfqpoint{0.800000in}{0.528000in}}{\pgfqpoint{4.960000in}{3.696000in}}%
\pgfusepath{clip}%
\pgfsetbuttcap%
\pgfsetmiterjoin%
\definecolor{currentfill}{rgb}{0.503922,0.999981,0.704926}%
\pgfsetfillcolor{currentfill}%
\pgfsetlinewidth{0.000000pt}%
\definecolor{currentstroke}{rgb}{0.000000,0.000000,0.000000}%
\pgfsetstrokecolor{currentstroke}%
\pgfsetstrokeopacity{0.000000}%
\pgfsetdash{}{0pt}%
\pgfpathmoveto{\pgfqpoint{4.600945in}{0.528000in}}%
\pgfpathlineto{\pgfqpoint{4.739992in}{0.528000in}}%
\pgfpathlineto{\pgfqpoint{4.739992in}{1.655116in}}%
\pgfpathlineto{\pgfqpoint{4.600945in}{1.655116in}}%
\pgfpathlineto{\pgfqpoint{4.600945in}{0.528000in}}%
\pgfpathclose%
\pgfusepath{fill}%
\end{pgfscope}%
\begin{pgfscope}%
\pgfpathrectangle{\pgfqpoint{0.800000in}{0.528000in}}{\pgfqpoint{4.960000in}{3.696000in}}%
\pgfusepath{clip}%
\pgfsetbuttcap%
\pgfsetmiterjoin%
\definecolor{currentfill}{rgb}{0.503922,0.999981,0.704926}%
\pgfsetfillcolor{currentfill}%
\pgfsetlinewidth{0.000000pt}%
\definecolor{currentstroke}{rgb}{0.000000,0.000000,0.000000}%
\pgfsetstrokecolor{currentstroke}%
\pgfsetstrokeopacity{0.000000}%
\pgfsetdash{}{0pt}%
\pgfpathmoveto{\pgfqpoint{4.998222in}{0.528000in}}%
\pgfpathlineto{\pgfqpoint{5.137269in}{0.528000in}}%
\pgfpathlineto{\pgfqpoint{5.137269in}{1.248102in}}%
\pgfpathlineto{\pgfqpoint{4.998222in}{1.248102in}}%
\pgfpathlineto{\pgfqpoint{4.998222in}{0.528000in}}%
\pgfpathclose%
\pgfusepath{fill}%
\end{pgfscope}%
\begin{pgfscope}%
\pgfpathrectangle{\pgfqpoint{0.800000in}{0.528000in}}{\pgfqpoint{4.960000in}{3.696000in}}%
\pgfusepath{clip}%
\pgfsetbuttcap%
\pgfsetmiterjoin%
\definecolor{currentfill}{rgb}{0.503922,0.999981,0.704926}%
\pgfsetfillcolor{currentfill}%
\pgfsetlinewidth{0.000000pt}%
\definecolor{currentstroke}{rgb}{0.000000,0.000000,0.000000}%
\pgfsetstrokecolor{currentstroke}%
\pgfsetstrokeopacity{0.000000}%
\pgfsetdash{}{0pt}%
\pgfpathmoveto{\pgfqpoint{5.395499in}{0.528000in}}%
\pgfpathlineto{\pgfqpoint{5.534545in}{0.528000in}}%
\pgfpathlineto{\pgfqpoint{5.534545in}{0.948432in}}%
\pgfpathlineto{\pgfqpoint{5.395499in}{0.948432in}}%
\pgfpathlineto{\pgfqpoint{5.395499in}{0.528000in}}%
\pgfpathclose%
\pgfusepath{fill}%
\end{pgfscope}%
\begin{pgfscope}%
\pgfpathrectangle{\pgfqpoint{0.800000in}{0.528000in}}{\pgfqpoint{4.960000in}{3.696000in}}%
\pgfusepath{clip}%
\pgfsetbuttcap%
\pgfsetmiterjoin%
\definecolor{currentfill}{rgb}{1.000000,0.700543,0.378411}%
\pgfsetfillcolor{currentfill}%
\pgfsetlinewidth{0.000000pt}%
\definecolor{currentstroke}{rgb}{0.000000,0.000000,0.000000}%
\pgfsetstrokecolor{currentstroke}%
\pgfsetstrokeopacity{0.000000}%
\pgfsetdash{}{0pt}%
\pgfpathmoveto{\pgfqpoint{1.025455in}{0.528000in}}%
\pgfpathlineto{\pgfqpoint{1.164501in}{0.528000in}}%
\pgfpathlineto{\pgfqpoint{1.164501in}{3.430770in}}%
\pgfpathlineto{\pgfqpoint{1.025455in}{3.430770in}}%
\pgfpathlineto{\pgfqpoint{1.025455in}{0.528000in}}%
\pgfpathclose%
\pgfusepath{fill}%
\end{pgfscope}%
\begin{pgfscope}%
\pgfpathrectangle{\pgfqpoint{0.800000in}{0.528000in}}{\pgfqpoint{4.960000in}{3.696000in}}%
\pgfusepath{clip}%
\pgfsetbuttcap%
\pgfsetmiterjoin%
\definecolor{currentfill}{rgb}{1.000000,0.700543,0.378411}%
\pgfsetfillcolor{currentfill}%
\pgfsetlinewidth{0.000000pt}%
\definecolor{currentstroke}{rgb}{0.000000,0.000000,0.000000}%
\pgfsetstrokecolor{currentstroke}%
\pgfsetstrokeopacity{0.000000}%
\pgfsetdash{}{0pt}%
\pgfpathmoveto{\pgfqpoint{1.422731in}{0.528000in}}%
\pgfpathlineto{\pgfqpoint{1.561778in}{0.528000in}}%
\pgfpathlineto{\pgfqpoint{1.561778in}{3.327898in}}%
\pgfpathlineto{\pgfqpoint{1.422731in}{3.327898in}}%
\pgfpathlineto{\pgfqpoint{1.422731in}{0.528000in}}%
\pgfpathclose%
\pgfusepath{fill}%
\end{pgfscope}%
\begin{pgfscope}%
\pgfpathrectangle{\pgfqpoint{0.800000in}{0.528000in}}{\pgfqpoint{4.960000in}{3.696000in}}%
\pgfusepath{clip}%
\pgfsetbuttcap%
\pgfsetmiterjoin%
\definecolor{currentfill}{rgb}{1.000000,0.700543,0.378411}%
\pgfsetfillcolor{currentfill}%
\pgfsetlinewidth{0.000000pt}%
\definecolor{currentstroke}{rgb}{0.000000,0.000000,0.000000}%
\pgfsetstrokecolor{currentstroke}%
\pgfsetstrokeopacity{0.000000}%
\pgfsetdash{}{0pt}%
\pgfpathmoveto{\pgfqpoint{1.820008in}{0.528000in}}%
\pgfpathlineto{\pgfqpoint{1.959055in}{0.528000in}}%
\pgfpathlineto{\pgfqpoint{1.959055in}{3.207136in}}%
\pgfpathlineto{\pgfqpoint{1.820008in}{3.207136in}}%
\pgfpathlineto{\pgfqpoint{1.820008in}{0.528000in}}%
\pgfpathclose%
\pgfusepath{fill}%
\end{pgfscope}%
\begin{pgfscope}%
\pgfpathrectangle{\pgfqpoint{0.800000in}{0.528000in}}{\pgfqpoint{4.960000in}{3.696000in}}%
\pgfusepath{clip}%
\pgfsetbuttcap%
\pgfsetmiterjoin%
\definecolor{currentfill}{rgb}{1.000000,0.700543,0.378411}%
\pgfsetfillcolor{currentfill}%
\pgfsetlinewidth{0.000000pt}%
\definecolor{currentstroke}{rgb}{0.000000,0.000000,0.000000}%
\pgfsetstrokecolor{currentstroke}%
\pgfsetstrokeopacity{0.000000}%
\pgfsetdash{}{0pt}%
\pgfpathmoveto{\pgfqpoint{2.217285in}{0.528000in}}%
\pgfpathlineto{\pgfqpoint{2.356332in}{0.528000in}}%
\pgfpathlineto{\pgfqpoint{2.356332in}{2.907466in}}%
\pgfpathlineto{\pgfqpoint{2.217285in}{2.907466in}}%
\pgfpathlineto{\pgfqpoint{2.217285in}{0.528000in}}%
\pgfpathclose%
\pgfusepath{fill}%
\end{pgfscope}%
\begin{pgfscope}%
\pgfpathrectangle{\pgfqpoint{0.800000in}{0.528000in}}{\pgfqpoint{4.960000in}{3.696000in}}%
\pgfusepath{clip}%
\pgfsetbuttcap%
\pgfsetmiterjoin%
\definecolor{currentfill}{rgb}{1.000000,0.700543,0.378411}%
\pgfsetfillcolor{currentfill}%
\pgfsetlinewidth{0.000000pt}%
\definecolor{currentstroke}{rgb}{0.000000,0.000000,0.000000}%
\pgfsetstrokecolor{currentstroke}%
\pgfsetstrokeopacity{0.000000}%
\pgfsetdash{}{0pt}%
\pgfpathmoveto{\pgfqpoint{2.614561in}{0.528000in}}%
\pgfpathlineto{\pgfqpoint{2.753608in}{0.528000in}}%
\pgfpathlineto{\pgfqpoint{2.753608in}{2.786704in}}%
\pgfpathlineto{\pgfqpoint{2.614561in}{2.786704in}}%
\pgfpathlineto{\pgfqpoint{2.614561in}{0.528000in}}%
\pgfpathclose%
\pgfusepath{fill}%
\end{pgfscope}%
\begin{pgfscope}%
\pgfpathrectangle{\pgfqpoint{0.800000in}{0.528000in}}{\pgfqpoint{4.960000in}{3.696000in}}%
\pgfusepath{clip}%
\pgfsetbuttcap%
\pgfsetmiterjoin%
\definecolor{currentfill}{rgb}{1.000000,0.700543,0.378411}%
\pgfsetfillcolor{currentfill}%
\pgfsetlinewidth{0.000000pt}%
\definecolor{currentstroke}{rgb}{0.000000,0.000000,0.000000}%
\pgfsetstrokecolor{currentstroke}%
\pgfsetstrokeopacity{0.000000}%
\pgfsetdash{}{0pt}%
\pgfpathmoveto{\pgfqpoint{3.011838in}{0.528000in}}%
\pgfpathlineto{\pgfqpoint{3.150885in}{0.528000in}}%
\pgfpathlineto{\pgfqpoint{3.150885in}{2.652524in}}%
\pgfpathlineto{\pgfqpoint{3.011838in}{2.652524in}}%
\pgfpathlineto{\pgfqpoint{3.011838in}{0.528000in}}%
\pgfpathclose%
\pgfusepath{fill}%
\end{pgfscope}%
\begin{pgfscope}%
\pgfpathrectangle{\pgfqpoint{0.800000in}{0.528000in}}{\pgfqpoint{4.960000in}{3.696000in}}%
\pgfusepath{clip}%
\pgfsetbuttcap%
\pgfsetmiterjoin%
\definecolor{currentfill}{rgb}{1.000000,0.700543,0.378411}%
\pgfsetfillcolor{currentfill}%
\pgfsetlinewidth{0.000000pt}%
\definecolor{currentstroke}{rgb}{0.000000,0.000000,0.000000}%
\pgfsetstrokecolor{currentstroke}%
\pgfsetstrokeopacity{0.000000}%
\pgfsetdash{}{0pt}%
\pgfpathmoveto{\pgfqpoint{3.409115in}{0.528000in}}%
\pgfpathlineto{\pgfqpoint{3.548162in}{0.528000in}}%
\pgfpathlineto{\pgfqpoint{3.548162in}{2.478089in}}%
\pgfpathlineto{\pgfqpoint{3.409115in}{2.478089in}}%
\pgfpathlineto{\pgfqpoint{3.409115in}{0.528000in}}%
\pgfpathclose%
\pgfusepath{fill}%
\end{pgfscope}%
\begin{pgfscope}%
\pgfpathrectangle{\pgfqpoint{0.800000in}{0.528000in}}{\pgfqpoint{4.960000in}{3.696000in}}%
\pgfusepath{clip}%
\pgfsetbuttcap%
\pgfsetmiterjoin%
\definecolor{currentfill}{rgb}{1.000000,0.700543,0.378411}%
\pgfsetfillcolor{currentfill}%
\pgfsetlinewidth{0.000000pt}%
\definecolor{currentstroke}{rgb}{0.000000,0.000000,0.000000}%
\pgfsetstrokecolor{currentstroke}%
\pgfsetstrokeopacity{0.000000}%
\pgfsetdash{}{0pt}%
\pgfpathmoveto{\pgfqpoint{3.806392in}{0.528000in}}%
\pgfpathlineto{\pgfqpoint{3.945439in}{0.528000in}}%
\pgfpathlineto{\pgfqpoint{3.945439in}{2.165001in}}%
\pgfpathlineto{\pgfqpoint{3.806392in}{2.165001in}}%
\pgfpathlineto{\pgfqpoint{3.806392in}{0.528000in}}%
\pgfpathclose%
\pgfusepath{fill}%
\end{pgfscope}%
\begin{pgfscope}%
\pgfpathrectangle{\pgfqpoint{0.800000in}{0.528000in}}{\pgfqpoint{4.960000in}{3.696000in}}%
\pgfusepath{clip}%
\pgfsetbuttcap%
\pgfsetmiterjoin%
\definecolor{currentfill}{rgb}{1.000000,0.700543,0.378411}%
\pgfsetfillcolor{currentfill}%
\pgfsetlinewidth{0.000000pt}%
\definecolor{currentstroke}{rgb}{0.000000,0.000000,0.000000}%
\pgfsetstrokecolor{currentstroke}%
\pgfsetstrokeopacity{0.000000}%
\pgfsetdash{}{0pt}%
\pgfpathmoveto{\pgfqpoint{4.203668in}{0.528000in}}%
\pgfpathlineto{\pgfqpoint{4.342715in}{0.528000in}}%
\pgfpathlineto{\pgfqpoint{4.342715in}{1.923476in}}%
\pgfpathlineto{\pgfqpoint{4.203668in}{1.923476in}}%
\pgfpathlineto{\pgfqpoint{4.203668in}{0.528000in}}%
\pgfpathclose%
\pgfusepath{fill}%
\end{pgfscope}%
\begin{pgfscope}%
\pgfpathrectangle{\pgfqpoint{0.800000in}{0.528000in}}{\pgfqpoint{4.960000in}{3.696000in}}%
\pgfusepath{clip}%
\pgfsetbuttcap%
\pgfsetmiterjoin%
\definecolor{currentfill}{rgb}{1.000000,0.700543,0.378411}%
\pgfsetfillcolor{currentfill}%
\pgfsetlinewidth{0.000000pt}%
\definecolor{currentstroke}{rgb}{0.000000,0.000000,0.000000}%
\pgfsetstrokecolor{currentstroke}%
\pgfsetstrokeopacity{0.000000}%
\pgfsetdash{}{0pt}%
\pgfpathmoveto{\pgfqpoint{4.600945in}{0.528000in}}%
\pgfpathlineto{\pgfqpoint{4.739992in}{0.528000in}}%
\pgfpathlineto{\pgfqpoint{4.739992in}{1.637225in}}%
\pgfpathlineto{\pgfqpoint{4.600945in}{1.637225in}}%
\pgfpathlineto{\pgfqpoint{4.600945in}{0.528000in}}%
\pgfpathclose%
\pgfusepath{fill}%
\end{pgfscope}%
\begin{pgfscope}%
\pgfpathrectangle{\pgfqpoint{0.800000in}{0.528000in}}{\pgfqpoint{4.960000in}{3.696000in}}%
\pgfusepath{clip}%
\pgfsetbuttcap%
\pgfsetmiterjoin%
\definecolor{currentfill}{rgb}{1.000000,0.700543,0.378411}%
\pgfsetfillcolor{currentfill}%
\pgfsetlinewidth{0.000000pt}%
\definecolor{currentstroke}{rgb}{0.000000,0.000000,0.000000}%
\pgfsetstrokecolor{currentstroke}%
\pgfsetstrokeopacity{0.000000}%
\pgfsetdash{}{0pt}%
\pgfpathmoveto{\pgfqpoint{4.998222in}{0.528000in}}%
\pgfpathlineto{\pgfqpoint{5.137269in}{0.528000in}}%
\pgfpathlineto{\pgfqpoint{5.137269in}{1.248102in}}%
\pgfpathlineto{\pgfqpoint{4.998222in}{1.248102in}}%
\pgfpathlineto{\pgfqpoint{4.998222in}{0.528000in}}%
\pgfpathclose%
\pgfusepath{fill}%
\end{pgfscope}%
\begin{pgfscope}%
\pgfpathrectangle{\pgfqpoint{0.800000in}{0.528000in}}{\pgfqpoint{4.960000in}{3.696000in}}%
\pgfusepath{clip}%
\pgfsetbuttcap%
\pgfsetmiterjoin%
\definecolor{currentfill}{rgb}{1.000000,0.700543,0.378411}%
\pgfsetfillcolor{currentfill}%
\pgfsetlinewidth{0.000000pt}%
\definecolor{currentstroke}{rgb}{0.000000,0.000000,0.000000}%
\pgfsetstrokecolor{currentstroke}%
\pgfsetstrokeopacity{0.000000}%
\pgfsetdash{}{0pt}%
\pgfpathmoveto{\pgfqpoint{5.395499in}{0.528000in}}%
\pgfpathlineto{\pgfqpoint{5.534545in}{0.528000in}}%
\pgfpathlineto{\pgfqpoint{5.534545in}{0.939487in}}%
\pgfpathlineto{\pgfqpoint{5.395499in}{0.939487in}}%
\pgfpathlineto{\pgfqpoint{5.395499in}{0.528000in}}%
\pgfpathclose%
\pgfusepath{fill}%
\end{pgfscope}%
\begin{pgfscope}%
\pgfpathrectangle{\pgfqpoint{0.800000in}{0.528000in}}{\pgfqpoint{4.960000in}{3.696000in}}%
\pgfusepath{clip}%
\pgfsetbuttcap%
\pgfsetmiterjoin%
\definecolor{currentfill}{rgb}{1.000000,0.000000,0.000000}%
\pgfsetfillcolor{currentfill}%
\pgfsetlinewidth{0.000000pt}%
\definecolor{currentstroke}{rgb}{0.000000,0.000000,0.000000}%
\pgfsetstrokecolor{currentstroke}%
\pgfsetstrokeopacity{0.000000}%
\pgfsetdash{}{0pt}%
\pgfpathmoveto{\pgfqpoint{1.025455in}{0.528000in}}%
\pgfpathlineto{\pgfqpoint{1.164501in}{0.528000in}}%
\pgfpathlineto{\pgfqpoint{1.164501in}{2.522816in}}%
\pgfpathlineto{\pgfqpoint{1.025455in}{2.522816in}}%
\pgfpathlineto{\pgfqpoint{1.025455in}{0.528000in}}%
\pgfpathclose%
\pgfusepath{fill}%
\end{pgfscope}%
\begin{pgfscope}%
\pgfpathrectangle{\pgfqpoint{0.800000in}{0.528000in}}{\pgfqpoint{4.960000in}{3.696000in}}%
\pgfusepath{clip}%
\pgfsetbuttcap%
\pgfsetmiterjoin%
\definecolor{currentfill}{rgb}{1.000000,0.000000,0.000000}%
\pgfsetfillcolor{currentfill}%
\pgfsetlinewidth{0.000000pt}%
\definecolor{currentstroke}{rgb}{0.000000,0.000000,0.000000}%
\pgfsetstrokecolor{currentstroke}%
\pgfsetstrokeopacity{0.000000}%
\pgfsetdash{}{0pt}%
\pgfpathmoveto{\pgfqpoint{1.422731in}{0.528000in}}%
\pgfpathlineto{\pgfqpoint{1.561778in}{0.528000in}}%
\pgfpathlineto{\pgfqpoint{1.561778in}{2.442307in}}%
\pgfpathlineto{\pgfqpoint{1.422731in}{2.442307in}}%
\pgfpathlineto{\pgfqpoint{1.422731in}{0.528000in}}%
\pgfpathclose%
\pgfusepath{fill}%
\end{pgfscope}%
\begin{pgfscope}%
\pgfpathrectangle{\pgfqpoint{0.800000in}{0.528000in}}{\pgfqpoint{4.960000in}{3.696000in}}%
\pgfusepath{clip}%
\pgfsetbuttcap%
\pgfsetmiterjoin%
\definecolor{currentfill}{rgb}{1.000000,0.000000,0.000000}%
\pgfsetfillcolor{currentfill}%
\pgfsetlinewidth{0.000000pt}%
\definecolor{currentstroke}{rgb}{0.000000,0.000000,0.000000}%
\pgfsetstrokecolor{currentstroke}%
\pgfsetstrokeopacity{0.000000}%
\pgfsetdash{}{0pt}%
\pgfpathmoveto{\pgfqpoint{1.820008in}{0.528000in}}%
\pgfpathlineto{\pgfqpoint{1.959055in}{0.528000in}}%
\pgfpathlineto{\pgfqpoint{1.959055in}{2.343909in}}%
\pgfpathlineto{\pgfqpoint{1.820008in}{2.343909in}}%
\pgfpathlineto{\pgfqpoint{1.820008in}{0.528000in}}%
\pgfpathclose%
\pgfusepath{fill}%
\end{pgfscope}%
\begin{pgfscope}%
\pgfpathrectangle{\pgfqpoint{0.800000in}{0.528000in}}{\pgfqpoint{4.960000in}{3.696000in}}%
\pgfusepath{clip}%
\pgfsetbuttcap%
\pgfsetmiterjoin%
\definecolor{currentfill}{rgb}{1.000000,0.000000,0.000000}%
\pgfsetfillcolor{currentfill}%
\pgfsetlinewidth{0.000000pt}%
\definecolor{currentstroke}{rgb}{0.000000,0.000000,0.000000}%
\pgfsetstrokecolor{currentstroke}%
\pgfsetstrokeopacity{0.000000}%
\pgfsetdash{}{0pt}%
\pgfpathmoveto{\pgfqpoint{2.217285in}{0.528000in}}%
\pgfpathlineto{\pgfqpoint{2.356332in}{0.528000in}}%
\pgfpathlineto{\pgfqpoint{2.356332in}{2.156056in}}%
\pgfpathlineto{\pgfqpoint{2.217285in}{2.156056in}}%
\pgfpathlineto{\pgfqpoint{2.217285in}{0.528000in}}%
\pgfpathclose%
\pgfusepath{fill}%
\end{pgfscope}%
\begin{pgfscope}%
\pgfpathrectangle{\pgfqpoint{0.800000in}{0.528000in}}{\pgfqpoint{4.960000in}{3.696000in}}%
\pgfusepath{clip}%
\pgfsetbuttcap%
\pgfsetmiterjoin%
\definecolor{currentfill}{rgb}{1.000000,0.000000,0.000000}%
\pgfsetfillcolor{currentfill}%
\pgfsetlinewidth{0.000000pt}%
\definecolor{currentstroke}{rgb}{0.000000,0.000000,0.000000}%
\pgfsetstrokecolor{currentstroke}%
\pgfsetstrokeopacity{0.000000}%
\pgfsetdash{}{0pt}%
\pgfpathmoveto{\pgfqpoint{2.614561in}{0.528000in}}%
\pgfpathlineto{\pgfqpoint{2.753608in}{0.528000in}}%
\pgfpathlineto{\pgfqpoint{2.753608in}{2.057657in}}%
\pgfpathlineto{\pgfqpoint{2.614561in}{2.057657in}}%
\pgfpathlineto{\pgfqpoint{2.614561in}{0.528000in}}%
\pgfpathclose%
\pgfusepath{fill}%
\end{pgfscope}%
\begin{pgfscope}%
\pgfpathrectangle{\pgfqpoint{0.800000in}{0.528000in}}{\pgfqpoint{4.960000in}{3.696000in}}%
\pgfusepath{clip}%
\pgfsetbuttcap%
\pgfsetmiterjoin%
\definecolor{currentfill}{rgb}{1.000000,0.000000,0.000000}%
\pgfsetfillcolor{currentfill}%
\pgfsetlinewidth{0.000000pt}%
\definecolor{currentstroke}{rgb}{0.000000,0.000000,0.000000}%
\pgfsetstrokecolor{currentstroke}%
\pgfsetstrokeopacity{0.000000}%
\pgfsetdash{}{0pt}%
\pgfpathmoveto{\pgfqpoint{3.011838in}{0.528000in}}%
\pgfpathlineto{\pgfqpoint{3.150885in}{0.528000in}}%
\pgfpathlineto{\pgfqpoint{3.150885in}{1.990567in}}%
\pgfpathlineto{\pgfqpoint{3.011838in}{1.990567in}}%
\pgfpathlineto{\pgfqpoint{3.011838in}{0.528000in}}%
\pgfpathclose%
\pgfusepath{fill}%
\end{pgfscope}%
\begin{pgfscope}%
\pgfpathrectangle{\pgfqpoint{0.800000in}{0.528000in}}{\pgfqpoint{4.960000in}{3.696000in}}%
\pgfusepath{clip}%
\pgfsetbuttcap%
\pgfsetmiterjoin%
\definecolor{currentfill}{rgb}{1.000000,0.000000,0.000000}%
\pgfsetfillcolor{currentfill}%
\pgfsetlinewidth{0.000000pt}%
\definecolor{currentstroke}{rgb}{0.000000,0.000000,0.000000}%
\pgfsetstrokecolor{currentstroke}%
\pgfsetstrokeopacity{0.000000}%
\pgfsetdash{}{0pt}%
\pgfpathmoveto{\pgfqpoint{3.409115in}{0.528000in}}%
\pgfpathlineto{\pgfqpoint{3.548162in}{0.528000in}}%
\pgfpathlineto{\pgfqpoint{3.548162in}{1.878750in}}%
\pgfpathlineto{\pgfqpoint{3.409115in}{1.878750in}}%
\pgfpathlineto{\pgfqpoint{3.409115in}{0.528000in}}%
\pgfpathclose%
\pgfusepath{fill}%
\end{pgfscope}%
\begin{pgfscope}%
\pgfpathrectangle{\pgfqpoint{0.800000in}{0.528000in}}{\pgfqpoint{4.960000in}{3.696000in}}%
\pgfusepath{clip}%
\pgfsetbuttcap%
\pgfsetmiterjoin%
\definecolor{currentfill}{rgb}{1.000000,0.000000,0.000000}%
\pgfsetfillcolor{currentfill}%
\pgfsetlinewidth{0.000000pt}%
\definecolor{currentstroke}{rgb}{0.000000,0.000000,0.000000}%
\pgfsetstrokecolor{currentstroke}%
\pgfsetstrokeopacity{0.000000}%
\pgfsetdash{}{0pt}%
\pgfpathmoveto{\pgfqpoint{3.806392in}{0.528000in}}%
\pgfpathlineto{\pgfqpoint{3.945439in}{0.528000in}}%
\pgfpathlineto{\pgfqpoint{3.945439in}{1.713260in}}%
\pgfpathlineto{\pgfqpoint{3.806392in}{1.713260in}}%
\pgfpathlineto{\pgfqpoint{3.806392in}{0.528000in}}%
\pgfpathclose%
\pgfusepath{fill}%
\end{pgfscope}%
\begin{pgfscope}%
\pgfpathrectangle{\pgfqpoint{0.800000in}{0.528000in}}{\pgfqpoint{4.960000in}{3.696000in}}%
\pgfusepath{clip}%
\pgfsetbuttcap%
\pgfsetmiterjoin%
\definecolor{currentfill}{rgb}{1.000000,0.000000,0.000000}%
\pgfsetfillcolor{currentfill}%
\pgfsetlinewidth{0.000000pt}%
\definecolor{currentstroke}{rgb}{0.000000,0.000000,0.000000}%
\pgfsetstrokecolor{currentstroke}%
\pgfsetstrokeopacity{0.000000}%
\pgfsetdash{}{0pt}%
\pgfpathmoveto{\pgfqpoint{4.203668in}{0.528000in}}%
\pgfpathlineto{\pgfqpoint{4.342715in}{0.528000in}}%
\pgfpathlineto{\pgfqpoint{4.342715in}{1.503044in}}%
\pgfpathlineto{\pgfqpoint{4.203668in}{1.503044in}}%
\pgfpathlineto{\pgfqpoint{4.203668in}{0.528000in}}%
\pgfpathclose%
\pgfusepath{fill}%
\end{pgfscope}%
\begin{pgfscope}%
\pgfpathrectangle{\pgfqpoint{0.800000in}{0.528000in}}{\pgfqpoint{4.960000in}{3.696000in}}%
\pgfusepath{clip}%
\pgfsetbuttcap%
\pgfsetmiterjoin%
\definecolor{currentfill}{rgb}{1.000000,0.000000,0.000000}%
\pgfsetfillcolor{currentfill}%
\pgfsetlinewidth{0.000000pt}%
\definecolor{currentstroke}{rgb}{0.000000,0.000000,0.000000}%
\pgfsetstrokecolor{currentstroke}%
\pgfsetstrokeopacity{0.000000}%
\pgfsetdash{}{0pt}%
\pgfpathmoveto{\pgfqpoint{4.600945in}{0.528000in}}%
\pgfpathlineto{\pgfqpoint{4.739992in}{0.528000in}}%
\pgfpathlineto{\pgfqpoint{4.739992in}{1.301774in}}%
\pgfpathlineto{\pgfqpoint{4.600945in}{1.301774in}}%
\pgfpathlineto{\pgfqpoint{4.600945in}{0.528000in}}%
\pgfpathclose%
\pgfusepath{fill}%
\end{pgfscope}%
\begin{pgfscope}%
\pgfpathrectangle{\pgfqpoint{0.800000in}{0.528000in}}{\pgfqpoint{4.960000in}{3.696000in}}%
\pgfusepath{clip}%
\pgfsetbuttcap%
\pgfsetmiterjoin%
\definecolor{currentfill}{rgb}{1.000000,0.000000,0.000000}%
\pgfsetfillcolor{currentfill}%
\pgfsetlinewidth{0.000000pt}%
\definecolor{currentstroke}{rgb}{0.000000,0.000000,0.000000}%
\pgfsetstrokecolor{currentstroke}%
\pgfsetstrokeopacity{0.000000}%
\pgfsetdash{}{0pt}%
\pgfpathmoveto{\pgfqpoint{4.998222in}{0.528000in}}%
\pgfpathlineto{\pgfqpoint{5.137269in}{0.528000in}}%
\pgfpathlineto{\pgfqpoint{5.137269in}{1.060249in}}%
\pgfpathlineto{\pgfqpoint{4.998222in}{1.060249in}}%
\pgfpathlineto{\pgfqpoint{4.998222in}{0.528000in}}%
\pgfpathclose%
\pgfusepath{fill}%
\end{pgfscope}%
\begin{pgfscope}%
\pgfpathrectangle{\pgfqpoint{0.800000in}{0.528000in}}{\pgfqpoint{4.960000in}{3.696000in}}%
\pgfusepath{clip}%
\pgfsetbuttcap%
\pgfsetmiterjoin%
\definecolor{currentfill}{rgb}{1.000000,0.000000,0.000000}%
\pgfsetfillcolor{currentfill}%
\pgfsetlinewidth{0.000000pt}%
\definecolor{currentstroke}{rgb}{0.000000,0.000000,0.000000}%
\pgfsetstrokecolor{currentstroke}%
\pgfsetstrokeopacity{0.000000}%
\pgfsetdash{}{0pt}%
\pgfpathmoveto{\pgfqpoint{5.395499in}{0.528000in}}%
\pgfpathlineto{\pgfqpoint{5.534545in}{0.528000in}}%
\pgfpathlineto{\pgfqpoint{5.534545in}{0.881342in}}%
\pgfpathlineto{\pgfqpoint{5.395499in}{0.881342in}}%
\pgfpathlineto{\pgfqpoint{5.395499in}{0.528000in}}%
\pgfpathclose%
\pgfusepath{fill}%
\end{pgfscope}%
\begin{pgfscope}%
\pgfsetbuttcap%
\pgfsetroundjoin%
\definecolor{currentfill}{rgb}{0.000000,0.000000,0.000000}%
\pgfsetfillcolor{currentfill}%
\pgfsetlinewidth{0.803000pt}%
\definecolor{currentstroke}{rgb}{0.000000,0.000000,0.000000}%
\pgfsetstrokecolor{currentstroke}%
\pgfsetdash{}{0pt}%
\pgfsys@defobject{currentmarker}{\pgfqpoint{0.000000in}{-0.048611in}}{\pgfqpoint{0.000000in}{0.000000in}}{%
\pgfpathmoveto{\pgfqpoint{0.000000in}{0.000000in}}%
\pgfpathlineto{\pgfqpoint{0.000000in}{-0.048611in}}%
\pgfusepath{stroke,fill}%
}%
\begin{pgfscope}%
\pgfsys@transformshift{1.094978in}{0.528000in}%
\pgfsys@useobject{currentmarker}{}%
\end{pgfscope}%
\end{pgfscope}%
\begin{pgfscope}%
\definecolor{textcolor}{rgb}{0.000000,0.000000,0.000000}%
\pgfsetstrokecolor{textcolor}%
\pgfsetfillcolor{textcolor}%
\pgftext[x=1.094978in,y=0.430778in,,top]{\color{textcolor}\sffamily\fontsize{10.000000}{12.000000}\selectfont 60}%
\end{pgfscope}%
\begin{pgfscope}%
\pgfsetbuttcap%
\pgfsetroundjoin%
\definecolor{currentfill}{rgb}{0.000000,0.000000,0.000000}%
\pgfsetfillcolor{currentfill}%
\pgfsetlinewidth{0.803000pt}%
\definecolor{currentstroke}{rgb}{0.000000,0.000000,0.000000}%
\pgfsetstrokecolor{currentstroke}%
\pgfsetdash{}{0pt}%
\pgfsys@defobject{currentmarker}{\pgfqpoint{0.000000in}{-0.048611in}}{\pgfqpoint{0.000000in}{0.000000in}}{%
\pgfpathmoveto{\pgfqpoint{0.000000in}{0.000000in}}%
\pgfpathlineto{\pgfqpoint{0.000000in}{-0.048611in}}%
\pgfusepath{stroke,fill}%
}%
\begin{pgfscope}%
\pgfsys@transformshift{1.492255in}{0.528000in}%
\pgfsys@useobject{currentmarker}{}%
\end{pgfscope}%
\end{pgfscope}%
\begin{pgfscope}%
\definecolor{textcolor}{rgb}{0.000000,0.000000,0.000000}%
\pgfsetstrokecolor{textcolor}%
\pgfsetfillcolor{textcolor}%
\pgftext[x=1.492255in,y=0.430778in,,top]{\color{textcolor}\sffamily\fontsize{10.000000}{12.000000}\selectfont 55}%
\end{pgfscope}%
\begin{pgfscope}%
\pgfsetbuttcap%
\pgfsetroundjoin%
\definecolor{currentfill}{rgb}{0.000000,0.000000,0.000000}%
\pgfsetfillcolor{currentfill}%
\pgfsetlinewidth{0.803000pt}%
\definecolor{currentstroke}{rgb}{0.000000,0.000000,0.000000}%
\pgfsetstrokecolor{currentstroke}%
\pgfsetdash{}{0pt}%
\pgfsys@defobject{currentmarker}{\pgfqpoint{0.000000in}{-0.048611in}}{\pgfqpoint{0.000000in}{0.000000in}}{%
\pgfpathmoveto{\pgfqpoint{0.000000in}{0.000000in}}%
\pgfpathlineto{\pgfqpoint{0.000000in}{-0.048611in}}%
\pgfusepath{stroke,fill}%
}%
\begin{pgfscope}%
\pgfsys@transformshift{1.889531in}{0.528000in}%
\pgfsys@useobject{currentmarker}{}%
\end{pgfscope}%
\end{pgfscope}%
\begin{pgfscope}%
\definecolor{textcolor}{rgb}{0.000000,0.000000,0.000000}%
\pgfsetstrokecolor{textcolor}%
\pgfsetfillcolor{textcolor}%
\pgftext[x=1.889531in,y=0.430778in,,top]{\color{textcolor}\sffamily\fontsize{10.000000}{12.000000}\selectfont 50}%
\end{pgfscope}%
\begin{pgfscope}%
\pgfsetbuttcap%
\pgfsetroundjoin%
\definecolor{currentfill}{rgb}{0.000000,0.000000,0.000000}%
\pgfsetfillcolor{currentfill}%
\pgfsetlinewidth{0.803000pt}%
\definecolor{currentstroke}{rgb}{0.000000,0.000000,0.000000}%
\pgfsetstrokecolor{currentstroke}%
\pgfsetdash{}{0pt}%
\pgfsys@defobject{currentmarker}{\pgfqpoint{0.000000in}{-0.048611in}}{\pgfqpoint{0.000000in}{0.000000in}}{%
\pgfpathmoveto{\pgfqpoint{0.000000in}{0.000000in}}%
\pgfpathlineto{\pgfqpoint{0.000000in}{-0.048611in}}%
\pgfusepath{stroke,fill}%
}%
\begin{pgfscope}%
\pgfsys@transformshift{2.286808in}{0.528000in}%
\pgfsys@useobject{currentmarker}{}%
\end{pgfscope}%
\end{pgfscope}%
\begin{pgfscope}%
\definecolor{textcolor}{rgb}{0.000000,0.000000,0.000000}%
\pgfsetstrokecolor{textcolor}%
\pgfsetfillcolor{textcolor}%
\pgftext[x=2.286808in,y=0.430778in,,top]{\color{textcolor}\sffamily\fontsize{10.000000}{12.000000}\selectfont 45}%
\end{pgfscope}%
\begin{pgfscope}%
\pgfsetbuttcap%
\pgfsetroundjoin%
\definecolor{currentfill}{rgb}{0.000000,0.000000,0.000000}%
\pgfsetfillcolor{currentfill}%
\pgfsetlinewidth{0.803000pt}%
\definecolor{currentstroke}{rgb}{0.000000,0.000000,0.000000}%
\pgfsetstrokecolor{currentstroke}%
\pgfsetdash{}{0pt}%
\pgfsys@defobject{currentmarker}{\pgfqpoint{0.000000in}{-0.048611in}}{\pgfqpoint{0.000000in}{0.000000in}}{%
\pgfpathmoveto{\pgfqpoint{0.000000in}{0.000000in}}%
\pgfpathlineto{\pgfqpoint{0.000000in}{-0.048611in}}%
\pgfusepath{stroke,fill}%
}%
\begin{pgfscope}%
\pgfsys@transformshift{2.684085in}{0.528000in}%
\pgfsys@useobject{currentmarker}{}%
\end{pgfscope}%
\end{pgfscope}%
\begin{pgfscope}%
\definecolor{textcolor}{rgb}{0.000000,0.000000,0.000000}%
\pgfsetstrokecolor{textcolor}%
\pgfsetfillcolor{textcolor}%
\pgftext[x=2.684085in,y=0.430778in,,top]{\color{textcolor}\sffamily\fontsize{10.000000}{12.000000}\selectfont 40}%
\end{pgfscope}%
\begin{pgfscope}%
\pgfsetbuttcap%
\pgfsetroundjoin%
\definecolor{currentfill}{rgb}{0.000000,0.000000,0.000000}%
\pgfsetfillcolor{currentfill}%
\pgfsetlinewidth{0.803000pt}%
\definecolor{currentstroke}{rgb}{0.000000,0.000000,0.000000}%
\pgfsetstrokecolor{currentstroke}%
\pgfsetdash{}{0pt}%
\pgfsys@defobject{currentmarker}{\pgfqpoint{0.000000in}{-0.048611in}}{\pgfqpoint{0.000000in}{0.000000in}}{%
\pgfpathmoveto{\pgfqpoint{0.000000in}{0.000000in}}%
\pgfpathlineto{\pgfqpoint{0.000000in}{-0.048611in}}%
\pgfusepath{stroke,fill}%
}%
\begin{pgfscope}%
\pgfsys@transformshift{3.081362in}{0.528000in}%
\pgfsys@useobject{currentmarker}{}%
\end{pgfscope}%
\end{pgfscope}%
\begin{pgfscope}%
\definecolor{textcolor}{rgb}{0.000000,0.000000,0.000000}%
\pgfsetstrokecolor{textcolor}%
\pgfsetfillcolor{textcolor}%
\pgftext[x=3.081362in,y=0.430778in,,top]{\color{textcolor}\sffamily\fontsize{10.000000}{12.000000}\selectfont 35}%
\end{pgfscope}%
\begin{pgfscope}%
\pgfsetbuttcap%
\pgfsetroundjoin%
\definecolor{currentfill}{rgb}{0.000000,0.000000,0.000000}%
\pgfsetfillcolor{currentfill}%
\pgfsetlinewidth{0.803000pt}%
\definecolor{currentstroke}{rgb}{0.000000,0.000000,0.000000}%
\pgfsetstrokecolor{currentstroke}%
\pgfsetdash{}{0pt}%
\pgfsys@defobject{currentmarker}{\pgfqpoint{0.000000in}{-0.048611in}}{\pgfqpoint{0.000000in}{0.000000in}}{%
\pgfpathmoveto{\pgfqpoint{0.000000in}{0.000000in}}%
\pgfpathlineto{\pgfqpoint{0.000000in}{-0.048611in}}%
\pgfusepath{stroke,fill}%
}%
\begin{pgfscope}%
\pgfsys@transformshift{3.478638in}{0.528000in}%
\pgfsys@useobject{currentmarker}{}%
\end{pgfscope}%
\end{pgfscope}%
\begin{pgfscope}%
\definecolor{textcolor}{rgb}{0.000000,0.000000,0.000000}%
\pgfsetstrokecolor{textcolor}%
\pgfsetfillcolor{textcolor}%
\pgftext[x=3.478638in,y=0.430778in,,top]{\color{textcolor}\sffamily\fontsize{10.000000}{12.000000}\selectfont 30}%
\end{pgfscope}%
\begin{pgfscope}%
\pgfsetbuttcap%
\pgfsetroundjoin%
\definecolor{currentfill}{rgb}{0.000000,0.000000,0.000000}%
\pgfsetfillcolor{currentfill}%
\pgfsetlinewidth{0.803000pt}%
\definecolor{currentstroke}{rgb}{0.000000,0.000000,0.000000}%
\pgfsetstrokecolor{currentstroke}%
\pgfsetdash{}{0pt}%
\pgfsys@defobject{currentmarker}{\pgfqpoint{0.000000in}{-0.048611in}}{\pgfqpoint{0.000000in}{0.000000in}}{%
\pgfpathmoveto{\pgfqpoint{0.000000in}{0.000000in}}%
\pgfpathlineto{\pgfqpoint{0.000000in}{-0.048611in}}%
\pgfusepath{stroke,fill}%
}%
\begin{pgfscope}%
\pgfsys@transformshift{3.875915in}{0.528000in}%
\pgfsys@useobject{currentmarker}{}%
\end{pgfscope}%
\end{pgfscope}%
\begin{pgfscope}%
\definecolor{textcolor}{rgb}{0.000000,0.000000,0.000000}%
\pgfsetstrokecolor{textcolor}%
\pgfsetfillcolor{textcolor}%
\pgftext[x=3.875915in,y=0.430778in,,top]{\color{textcolor}\sffamily\fontsize{10.000000}{12.000000}\selectfont 25}%
\end{pgfscope}%
\begin{pgfscope}%
\pgfsetbuttcap%
\pgfsetroundjoin%
\definecolor{currentfill}{rgb}{0.000000,0.000000,0.000000}%
\pgfsetfillcolor{currentfill}%
\pgfsetlinewidth{0.803000pt}%
\definecolor{currentstroke}{rgb}{0.000000,0.000000,0.000000}%
\pgfsetstrokecolor{currentstroke}%
\pgfsetdash{}{0pt}%
\pgfsys@defobject{currentmarker}{\pgfqpoint{0.000000in}{-0.048611in}}{\pgfqpoint{0.000000in}{0.000000in}}{%
\pgfpathmoveto{\pgfqpoint{0.000000in}{0.000000in}}%
\pgfpathlineto{\pgfqpoint{0.000000in}{-0.048611in}}%
\pgfusepath{stroke,fill}%
}%
\begin{pgfscope}%
\pgfsys@transformshift{4.273192in}{0.528000in}%
\pgfsys@useobject{currentmarker}{}%
\end{pgfscope}%
\end{pgfscope}%
\begin{pgfscope}%
\definecolor{textcolor}{rgb}{0.000000,0.000000,0.000000}%
\pgfsetstrokecolor{textcolor}%
\pgfsetfillcolor{textcolor}%
\pgftext[x=4.273192in,y=0.430778in,,top]{\color{textcolor}\sffamily\fontsize{10.000000}{12.000000}\selectfont 20}%
\end{pgfscope}%
\begin{pgfscope}%
\pgfsetbuttcap%
\pgfsetroundjoin%
\definecolor{currentfill}{rgb}{0.000000,0.000000,0.000000}%
\pgfsetfillcolor{currentfill}%
\pgfsetlinewidth{0.803000pt}%
\definecolor{currentstroke}{rgb}{0.000000,0.000000,0.000000}%
\pgfsetstrokecolor{currentstroke}%
\pgfsetdash{}{0pt}%
\pgfsys@defobject{currentmarker}{\pgfqpoint{0.000000in}{-0.048611in}}{\pgfqpoint{0.000000in}{0.000000in}}{%
\pgfpathmoveto{\pgfqpoint{0.000000in}{0.000000in}}%
\pgfpathlineto{\pgfqpoint{0.000000in}{-0.048611in}}%
\pgfusepath{stroke,fill}%
}%
\begin{pgfscope}%
\pgfsys@transformshift{4.670469in}{0.528000in}%
\pgfsys@useobject{currentmarker}{}%
\end{pgfscope}%
\end{pgfscope}%
\begin{pgfscope}%
\definecolor{textcolor}{rgb}{0.000000,0.000000,0.000000}%
\pgfsetstrokecolor{textcolor}%
\pgfsetfillcolor{textcolor}%
\pgftext[x=4.670469in,y=0.430778in,,top]{\color{textcolor}\sffamily\fontsize{10.000000}{12.000000}\selectfont 15}%
\end{pgfscope}%
\begin{pgfscope}%
\pgfsetbuttcap%
\pgfsetroundjoin%
\definecolor{currentfill}{rgb}{0.000000,0.000000,0.000000}%
\pgfsetfillcolor{currentfill}%
\pgfsetlinewidth{0.803000pt}%
\definecolor{currentstroke}{rgb}{0.000000,0.000000,0.000000}%
\pgfsetstrokecolor{currentstroke}%
\pgfsetdash{}{0pt}%
\pgfsys@defobject{currentmarker}{\pgfqpoint{0.000000in}{-0.048611in}}{\pgfqpoint{0.000000in}{0.000000in}}{%
\pgfpathmoveto{\pgfqpoint{0.000000in}{0.000000in}}%
\pgfpathlineto{\pgfqpoint{0.000000in}{-0.048611in}}%
\pgfusepath{stroke,fill}%
}%
\begin{pgfscope}%
\pgfsys@transformshift{5.067745in}{0.528000in}%
\pgfsys@useobject{currentmarker}{}%
\end{pgfscope}%
\end{pgfscope}%
\begin{pgfscope}%
\definecolor{textcolor}{rgb}{0.000000,0.000000,0.000000}%
\pgfsetstrokecolor{textcolor}%
\pgfsetfillcolor{textcolor}%
\pgftext[x=5.067745in,y=0.430778in,,top]{\color{textcolor}\sffamily\fontsize{10.000000}{12.000000}\selectfont 10}%
\end{pgfscope}%
\begin{pgfscope}%
\pgfsetbuttcap%
\pgfsetroundjoin%
\definecolor{currentfill}{rgb}{0.000000,0.000000,0.000000}%
\pgfsetfillcolor{currentfill}%
\pgfsetlinewidth{0.803000pt}%
\definecolor{currentstroke}{rgb}{0.000000,0.000000,0.000000}%
\pgfsetstrokecolor{currentstroke}%
\pgfsetdash{}{0pt}%
\pgfsys@defobject{currentmarker}{\pgfqpoint{0.000000in}{-0.048611in}}{\pgfqpoint{0.000000in}{0.000000in}}{%
\pgfpathmoveto{\pgfqpoint{0.000000in}{0.000000in}}%
\pgfpathlineto{\pgfqpoint{0.000000in}{-0.048611in}}%
\pgfusepath{stroke,fill}%
}%
\begin{pgfscope}%
\pgfsys@transformshift{5.465022in}{0.528000in}%
\pgfsys@useobject{currentmarker}{}%
\end{pgfscope}%
\end{pgfscope}%
\begin{pgfscope}%
\definecolor{textcolor}{rgb}{0.000000,0.000000,0.000000}%
\pgfsetstrokecolor{textcolor}%
\pgfsetfillcolor{textcolor}%
\pgftext[x=5.465022in,y=0.430778in,,top]{\color{textcolor}\sffamily\fontsize{10.000000}{12.000000}\selectfont 5}%
\end{pgfscope}%
\begin{pgfscope}%
\definecolor{textcolor}{rgb}{0.000000,0.000000,0.000000}%
\pgfsetstrokecolor{textcolor}%
\pgfsetfillcolor{textcolor}%
\pgftext[x=3.280000in,y=0.240809in,,top]{\color{textcolor}\sffamily\fontsize{10.000000}{12.000000}\selectfont Seconds}%
\end{pgfscope}%
\begin{pgfscope}%
\pgfsetbuttcap%
\pgfsetroundjoin%
\definecolor{currentfill}{rgb}{0.000000,0.000000,0.000000}%
\pgfsetfillcolor{currentfill}%
\pgfsetlinewidth{0.803000pt}%
\definecolor{currentstroke}{rgb}{0.000000,0.000000,0.000000}%
\pgfsetstrokecolor{currentstroke}%
\pgfsetdash{}{0pt}%
\pgfsys@defobject{currentmarker}{\pgfqpoint{-0.048611in}{0.000000in}}{\pgfqpoint{-0.000000in}{0.000000in}}{%
\pgfpathmoveto{\pgfqpoint{-0.000000in}{0.000000in}}%
\pgfpathlineto{\pgfqpoint{-0.048611in}{0.000000in}}%
\pgfusepath{stroke,fill}%
}%
\begin{pgfscope}%
\pgfsys@transformshift{0.800000in}{0.528000in}%
\pgfsys@useobject{currentmarker}{}%
\end{pgfscope}%
\end{pgfscope}%
\begin{pgfscope}%
\definecolor{textcolor}{rgb}{0.000000,0.000000,0.000000}%
\pgfsetstrokecolor{textcolor}%
\pgfsetfillcolor{textcolor}%
\pgftext[x=0.481898in, y=0.475238in, left, base]{\color{textcolor}\sffamily\fontsize{10.000000}{12.000000}\selectfont 0.0}%
\end{pgfscope}%
\begin{pgfscope}%
\pgfsetbuttcap%
\pgfsetroundjoin%
\definecolor{currentfill}{rgb}{0.000000,0.000000,0.000000}%
\pgfsetfillcolor{currentfill}%
\pgfsetlinewidth{0.803000pt}%
\definecolor{currentstroke}{rgb}{0.000000,0.000000,0.000000}%
\pgfsetstrokecolor{currentstroke}%
\pgfsetdash{}{0pt}%
\pgfsys@defobject{currentmarker}{\pgfqpoint{-0.048611in}{0.000000in}}{\pgfqpoint{-0.000000in}{0.000000in}}{%
\pgfpathmoveto{\pgfqpoint{-0.000000in}{0.000000in}}%
\pgfpathlineto{\pgfqpoint{-0.048611in}{0.000000in}}%
\pgfusepath{stroke,fill}%
}%
\begin{pgfscope}%
\pgfsys@transformshift{0.800000in}{0.975268in}%
\pgfsys@useobject{currentmarker}{}%
\end{pgfscope}%
\end{pgfscope}%
\begin{pgfscope}%
\definecolor{textcolor}{rgb}{0.000000,0.000000,0.000000}%
\pgfsetstrokecolor{textcolor}%
\pgfsetfillcolor{textcolor}%
\pgftext[x=0.481898in, y=0.922507in, left, base]{\color{textcolor}\sffamily\fontsize{10.000000}{12.000000}\selectfont 0.1}%
\end{pgfscope}%
\begin{pgfscope}%
\pgfsetbuttcap%
\pgfsetroundjoin%
\definecolor{currentfill}{rgb}{0.000000,0.000000,0.000000}%
\pgfsetfillcolor{currentfill}%
\pgfsetlinewidth{0.803000pt}%
\definecolor{currentstroke}{rgb}{0.000000,0.000000,0.000000}%
\pgfsetstrokecolor{currentstroke}%
\pgfsetdash{}{0pt}%
\pgfsys@defobject{currentmarker}{\pgfqpoint{-0.048611in}{0.000000in}}{\pgfqpoint{-0.000000in}{0.000000in}}{%
\pgfpathmoveto{\pgfqpoint{-0.000000in}{0.000000in}}%
\pgfpathlineto{\pgfqpoint{-0.048611in}{0.000000in}}%
\pgfusepath{stroke,fill}%
}%
\begin{pgfscope}%
\pgfsys@transformshift{0.800000in}{1.422536in}%
\pgfsys@useobject{currentmarker}{}%
\end{pgfscope}%
\end{pgfscope}%
\begin{pgfscope}%
\definecolor{textcolor}{rgb}{0.000000,0.000000,0.000000}%
\pgfsetstrokecolor{textcolor}%
\pgfsetfillcolor{textcolor}%
\pgftext[x=0.481898in, y=1.369775in, left, base]{\color{textcolor}\sffamily\fontsize{10.000000}{12.000000}\selectfont 0.2}%
\end{pgfscope}%
\begin{pgfscope}%
\pgfsetbuttcap%
\pgfsetroundjoin%
\definecolor{currentfill}{rgb}{0.000000,0.000000,0.000000}%
\pgfsetfillcolor{currentfill}%
\pgfsetlinewidth{0.803000pt}%
\definecolor{currentstroke}{rgb}{0.000000,0.000000,0.000000}%
\pgfsetstrokecolor{currentstroke}%
\pgfsetdash{}{0pt}%
\pgfsys@defobject{currentmarker}{\pgfqpoint{-0.048611in}{0.000000in}}{\pgfqpoint{-0.000000in}{0.000000in}}{%
\pgfpathmoveto{\pgfqpoint{-0.000000in}{0.000000in}}%
\pgfpathlineto{\pgfqpoint{-0.048611in}{0.000000in}}%
\pgfusepath{stroke,fill}%
}%
\begin{pgfscope}%
\pgfsys@transformshift{0.800000in}{1.869804in}%
\pgfsys@useobject{currentmarker}{}%
\end{pgfscope}%
\end{pgfscope}%
\begin{pgfscope}%
\definecolor{textcolor}{rgb}{0.000000,0.000000,0.000000}%
\pgfsetstrokecolor{textcolor}%
\pgfsetfillcolor{textcolor}%
\pgftext[x=0.481898in, y=1.817043in, left, base]{\color{textcolor}\sffamily\fontsize{10.000000}{12.000000}\selectfont 0.3}%
\end{pgfscope}%
\begin{pgfscope}%
\pgfsetbuttcap%
\pgfsetroundjoin%
\definecolor{currentfill}{rgb}{0.000000,0.000000,0.000000}%
\pgfsetfillcolor{currentfill}%
\pgfsetlinewidth{0.803000pt}%
\definecolor{currentstroke}{rgb}{0.000000,0.000000,0.000000}%
\pgfsetstrokecolor{currentstroke}%
\pgfsetdash{}{0pt}%
\pgfsys@defobject{currentmarker}{\pgfqpoint{-0.048611in}{0.000000in}}{\pgfqpoint{-0.000000in}{0.000000in}}{%
\pgfpathmoveto{\pgfqpoint{-0.000000in}{0.000000in}}%
\pgfpathlineto{\pgfqpoint{-0.048611in}{0.000000in}}%
\pgfusepath{stroke,fill}%
}%
\begin{pgfscope}%
\pgfsys@transformshift{0.800000in}{2.317072in}%
\pgfsys@useobject{currentmarker}{}%
\end{pgfscope}%
\end{pgfscope}%
\begin{pgfscope}%
\definecolor{textcolor}{rgb}{0.000000,0.000000,0.000000}%
\pgfsetstrokecolor{textcolor}%
\pgfsetfillcolor{textcolor}%
\pgftext[x=0.481898in, y=2.264311in, left, base]{\color{textcolor}\sffamily\fontsize{10.000000}{12.000000}\selectfont 0.4}%
\end{pgfscope}%
\begin{pgfscope}%
\pgfsetbuttcap%
\pgfsetroundjoin%
\definecolor{currentfill}{rgb}{0.000000,0.000000,0.000000}%
\pgfsetfillcolor{currentfill}%
\pgfsetlinewidth{0.803000pt}%
\definecolor{currentstroke}{rgb}{0.000000,0.000000,0.000000}%
\pgfsetstrokecolor{currentstroke}%
\pgfsetdash{}{0pt}%
\pgfsys@defobject{currentmarker}{\pgfqpoint{-0.048611in}{0.000000in}}{\pgfqpoint{-0.000000in}{0.000000in}}{%
\pgfpathmoveto{\pgfqpoint{-0.000000in}{0.000000in}}%
\pgfpathlineto{\pgfqpoint{-0.048611in}{0.000000in}}%
\pgfusepath{stroke,fill}%
}%
\begin{pgfscope}%
\pgfsys@transformshift{0.800000in}{2.764341in}%
\pgfsys@useobject{currentmarker}{}%
\end{pgfscope}%
\end{pgfscope}%
\begin{pgfscope}%
\definecolor{textcolor}{rgb}{0.000000,0.000000,0.000000}%
\pgfsetstrokecolor{textcolor}%
\pgfsetfillcolor{textcolor}%
\pgftext[x=0.481898in, y=2.711579in, left, base]{\color{textcolor}\sffamily\fontsize{10.000000}{12.000000}\selectfont 0.5}%
\end{pgfscope}%
\begin{pgfscope}%
\pgfsetbuttcap%
\pgfsetroundjoin%
\definecolor{currentfill}{rgb}{0.000000,0.000000,0.000000}%
\pgfsetfillcolor{currentfill}%
\pgfsetlinewidth{0.803000pt}%
\definecolor{currentstroke}{rgb}{0.000000,0.000000,0.000000}%
\pgfsetstrokecolor{currentstroke}%
\pgfsetdash{}{0pt}%
\pgfsys@defobject{currentmarker}{\pgfqpoint{-0.048611in}{0.000000in}}{\pgfqpoint{-0.000000in}{0.000000in}}{%
\pgfpathmoveto{\pgfqpoint{-0.000000in}{0.000000in}}%
\pgfpathlineto{\pgfqpoint{-0.048611in}{0.000000in}}%
\pgfusepath{stroke,fill}%
}%
\begin{pgfscope}%
\pgfsys@transformshift{0.800000in}{3.211609in}%
\pgfsys@useobject{currentmarker}{}%
\end{pgfscope}%
\end{pgfscope}%
\begin{pgfscope}%
\definecolor{textcolor}{rgb}{0.000000,0.000000,0.000000}%
\pgfsetstrokecolor{textcolor}%
\pgfsetfillcolor{textcolor}%
\pgftext[x=0.481898in, y=3.158847in, left, base]{\color{textcolor}\sffamily\fontsize{10.000000}{12.000000}\selectfont 0.6}%
\end{pgfscope}%
\begin{pgfscope}%
\pgfsetbuttcap%
\pgfsetroundjoin%
\definecolor{currentfill}{rgb}{0.000000,0.000000,0.000000}%
\pgfsetfillcolor{currentfill}%
\pgfsetlinewidth{0.803000pt}%
\definecolor{currentstroke}{rgb}{0.000000,0.000000,0.000000}%
\pgfsetstrokecolor{currentstroke}%
\pgfsetdash{}{0pt}%
\pgfsys@defobject{currentmarker}{\pgfqpoint{-0.048611in}{0.000000in}}{\pgfqpoint{-0.000000in}{0.000000in}}{%
\pgfpathmoveto{\pgfqpoint{-0.000000in}{0.000000in}}%
\pgfpathlineto{\pgfqpoint{-0.048611in}{0.000000in}}%
\pgfusepath{stroke,fill}%
}%
\begin{pgfscope}%
\pgfsys@transformshift{0.800000in}{3.658877in}%
\pgfsys@useobject{currentmarker}{}%
\end{pgfscope}%
\end{pgfscope}%
\begin{pgfscope}%
\definecolor{textcolor}{rgb}{0.000000,0.000000,0.000000}%
\pgfsetstrokecolor{textcolor}%
\pgfsetfillcolor{textcolor}%
\pgftext[x=0.481898in, y=3.606115in, left, base]{\color{textcolor}\sffamily\fontsize{10.000000}{12.000000}\selectfont 0.7}%
\end{pgfscope}%
\begin{pgfscope}%
\pgfsetbuttcap%
\pgfsetroundjoin%
\definecolor{currentfill}{rgb}{0.000000,0.000000,0.000000}%
\pgfsetfillcolor{currentfill}%
\pgfsetlinewidth{0.803000pt}%
\definecolor{currentstroke}{rgb}{0.000000,0.000000,0.000000}%
\pgfsetstrokecolor{currentstroke}%
\pgfsetdash{}{0pt}%
\pgfsys@defobject{currentmarker}{\pgfqpoint{-0.048611in}{0.000000in}}{\pgfqpoint{-0.000000in}{0.000000in}}{%
\pgfpathmoveto{\pgfqpoint{-0.000000in}{0.000000in}}%
\pgfpathlineto{\pgfqpoint{-0.048611in}{0.000000in}}%
\pgfusepath{stroke,fill}%
}%
\begin{pgfscope}%
\pgfsys@transformshift{0.800000in}{4.106145in}%
\pgfsys@useobject{currentmarker}{}%
\end{pgfscope}%
\end{pgfscope}%
\begin{pgfscope}%
\definecolor{textcolor}{rgb}{0.000000,0.000000,0.000000}%
\pgfsetstrokecolor{textcolor}%
\pgfsetfillcolor{textcolor}%
\pgftext[x=0.481898in, y=4.053383in, left, base]{\color{textcolor}\sffamily\fontsize{10.000000}{12.000000}\selectfont 0.8}%
\end{pgfscope}%
\begin{pgfscope}%
\definecolor{textcolor}{rgb}{0.000000,0.000000,0.000000}%
\pgfsetstrokecolor{textcolor}%
\pgfsetfillcolor{textcolor}%
\pgftext[x=0.426343in,y=2.376000in,,bottom,rotate=90.000000]{\color{textcolor}\sffamily\fontsize{10.000000}{12.000000}\selectfont Precision@k}%
\end{pgfscope}%
\begin{pgfscope}%
\pgfsetrectcap%
\pgfsetmiterjoin%
\pgfsetlinewidth{0.803000pt}%
\definecolor{currentstroke}{rgb}{0.000000,0.000000,0.000000}%
\pgfsetstrokecolor{currentstroke}%
\pgfsetdash{}{0pt}%
\pgfpathmoveto{\pgfqpoint{0.800000in}{0.528000in}}%
\pgfpathlineto{\pgfqpoint{0.800000in}{4.224000in}}%
\pgfusepath{stroke}%
\end{pgfscope}%
\begin{pgfscope}%
\pgfsetrectcap%
\pgfsetmiterjoin%
\pgfsetlinewidth{0.803000pt}%
\definecolor{currentstroke}{rgb}{0.000000,0.000000,0.000000}%
\pgfsetstrokecolor{currentstroke}%
\pgfsetdash{}{0pt}%
\pgfpathmoveto{\pgfqpoint{5.760000in}{0.528000in}}%
\pgfpathlineto{\pgfqpoint{5.760000in}{4.224000in}}%
\pgfusepath{stroke}%
\end{pgfscope}%
\begin{pgfscope}%
\pgfsetrectcap%
\pgfsetmiterjoin%
\pgfsetlinewidth{0.803000pt}%
\definecolor{currentstroke}{rgb}{0.000000,0.000000,0.000000}%
\pgfsetstrokecolor{currentstroke}%
\pgfsetdash{}{0pt}%
\pgfpathmoveto{\pgfqpoint{0.800000in}{0.528000in}}%
\pgfpathlineto{\pgfqpoint{5.760000in}{0.528000in}}%
\pgfusepath{stroke}%
\end{pgfscope}%
\begin{pgfscope}%
\pgfsetrectcap%
\pgfsetmiterjoin%
\pgfsetlinewidth{0.803000pt}%
\definecolor{currentstroke}{rgb}{0.000000,0.000000,0.000000}%
\pgfsetstrokecolor{currentstroke}%
\pgfsetdash{}{0pt}%
\pgfpathmoveto{\pgfqpoint{0.800000in}{4.224000in}}%
\pgfpathlineto{\pgfqpoint{5.760000in}{4.224000in}}%
\pgfusepath{stroke}%
\end{pgfscope}%
\begin{pgfscope}%
\definecolor{textcolor}{rgb}{0.000000,0.000000,0.000000}%
\pgfsetstrokecolor{textcolor}%
\pgfsetfillcolor{textcolor}%
\pgftext[x=3.280000in,y=4.307333in,,base]{\color{textcolor}\sffamily\fontsize{12.000000}{14.400000}\selectfont Precision at rank k under varying latency cutoffs}%
\end{pgfscope}%
\begin{pgfscope}%
\pgfsetbuttcap%
\pgfsetmiterjoin%
\definecolor{currentfill}{rgb}{1.000000,1.000000,1.000000}%
\pgfsetfillcolor{currentfill}%
\pgfsetfillopacity{0.800000}%
\pgfsetlinewidth{1.003750pt}%
\definecolor{currentstroke}{rgb}{0.800000,0.800000,0.800000}%
\pgfsetstrokecolor{currentstroke}%
\pgfsetstrokeopacity{0.800000}%
\pgfsetdash{}{0pt}%
\pgfpathmoveto{\pgfqpoint{4.818960in}{3.093603in}}%
\pgfpathlineto{\pgfqpoint{5.662778in}{3.093603in}}%
\pgfpathquadraticcurveto{\pgfqpoint{5.690556in}{3.093603in}}{\pgfqpoint{5.690556in}{3.121381in}}%
\pgfpathlineto{\pgfqpoint{5.690556in}{4.126778in}}%
\pgfpathquadraticcurveto{\pgfqpoint{5.690556in}{4.154556in}}{\pgfqpoint{5.662778in}{4.154556in}}%
\pgfpathlineto{\pgfqpoint{4.818960in}{4.154556in}}%
\pgfpathquadraticcurveto{\pgfqpoint{4.791182in}{4.154556in}}{\pgfqpoint{4.791182in}{4.126778in}}%
\pgfpathlineto{\pgfqpoint{4.791182in}{3.121381in}}%
\pgfpathquadraticcurveto{\pgfqpoint{4.791182in}{3.093603in}}{\pgfqpoint{4.818960in}{3.093603in}}%
\pgfpathlineto{\pgfqpoint{4.818960in}{3.093603in}}%
\pgfpathclose%
\pgfusepath{stroke,fill}%
\end{pgfscope}%
\begin{pgfscope}%
\pgfsetbuttcap%
\pgfsetmiterjoin%
\definecolor{currentfill}{rgb}{0.500000,0.000000,1.000000}%
\pgfsetfillcolor{currentfill}%
\pgfsetlinewidth{0.000000pt}%
\definecolor{currentstroke}{rgb}{0.000000,0.000000,0.000000}%
\pgfsetstrokecolor{currentstroke}%
\pgfsetstrokeopacity{0.000000}%
\pgfsetdash{}{0pt}%
\pgfpathmoveto{\pgfqpoint{4.846738in}{3.993477in}}%
\pgfpathlineto{\pgfqpoint{5.124516in}{3.993477in}}%
\pgfpathlineto{\pgfqpoint{5.124516in}{4.090699in}}%
\pgfpathlineto{\pgfqpoint{4.846738in}{4.090699in}}%
\pgfpathlineto{\pgfqpoint{4.846738in}{3.993477in}}%
\pgfpathclose%
\pgfusepath{fill}%
\end{pgfscope}%
\begin{pgfscope}%
\definecolor{textcolor}{rgb}{0.000000,0.000000,0.000000}%
\pgfsetstrokecolor{textcolor}%
\pgfsetfillcolor{textcolor}%
\pgftext[x=5.235627in,y=3.993477in,left,base]{\color{textcolor}\sffamily\fontsize{10.000000}{12.000000}\selectfont P@1k}%
\end{pgfscope}%
\begin{pgfscope}%
\pgfsetbuttcap%
\pgfsetmiterjoin%
\definecolor{currentfill}{rgb}{0.001961,0.709281,0.923289}%
\pgfsetfillcolor{currentfill}%
\pgfsetlinewidth{0.000000pt}%
\definecolor{currentstroke}{rgb}{0.000000,0.000000,0.000000}%
\pgfsetstrokecolor{currentstroke}%
\pgfsetstrokeopacity{0.000000}%
\pgfsetdash{}{0pt}%
\pgfpathmoveto{\pgfqpoint{4.846738in}{3.789620in}}%
\pgfpathlineto{\pgfqpoint{5.124516in}{3.789620in}}%
\pgfpathlineto{\pgfqpoint{5.124516in}{3.886842in}}%
\pgfpathlineto{\pgfqpoint{4.846738in}{3.886842in}}%
\pgfpathlineto{\pgfqpoint{4.846738in}{3.789620in}}%
\pgfpathclose%
\pgfusepath{fill}%
\end{pgfscope}%
\begin{pgfscope}%
\definecolor{textcolor}{rgb}{0.000000,0.000000,0.000000}%
\pgfsetstrokecolor{textcolor}%
\pgfsetfillcolor{textcolor}%
\pgftext[x=5.235627in,y=3.789620in,left,base]{\color{textcolor}\sffamily\fontsize{10.000000}{12.000000}\selectfont P@20}%
\end{pgfscope}%
\begin{pgfscope}%
\pgfsetbuttcap%
\pgfsetmiterjoin%
\definecolor{currentfill}{rgb}{0.503922,0.999981,0.704926}%
\pgfsetfillcolor{currentfill}%
\pgfsetlinewidth{0.000000pt}%
\definecolor{currentstroke}{rgb}{0.000000,0.000000,0.000000}%
\pgfsetstrokecolor{currentstroke}%
\pgfsetstrokeopacity{0.000000}%
\pgfsetdash{}{0pt}%
\pgfpathmoveto{\pgfqpoint{4.846738in}{3.585763in}}%
\pgfpathlineto{\pgfqpoint{5.124516in}{3.585763in}}%
\pgfpathlineto{\pgfqpoint{5.124516in}{3.682985in}}%
\pgfpathlineto{\pgfqpoint{4.846738in}{3.682985in}}%
\pgfpathlineto{\pgfqpoint{4.846738in}{3.585763in}}%
\pgfpathclose%
\pgfusepath{fill}%
\end{pgfscope}%
\begin{pgfscope}%
\definecolor{textcolor}{rgb}{0.000000,0.000000,0.000000}%
\pgfsetstrokecolor{textcolor}%
\pgfsetfillcolor{textcolor}%
\pgftext[x=5.235627in,y=3.585762in,left,base]{\color{textcolor}\sffamily\fontsize{10.000000}{12.000000}\selectfont P@10}%
\end{pgfscope}%
\begin{pgfscope}%
\pgfsetbuttcap%
\pgfsetmiterjoin%
\definecolor{currentfill}{rgb}{1.000000,0.700543,0.378411}%
\pgfsetfillcolor{currentfill}%
\pgfsetlinewidth{0.000000pt}%
\definecolor{currentstroke}{rgb}{0.000000,0.000000,0.000000}%
\pgfsetstrokecolor{currentstroke}%
\pgfsetstrokeopacity{0.000000}%
\pgfsetdash{}{0pt}%
\pgfpathmoveto{\pgfqpoint{4.846738in}{3.381905in}}%
\pgfpathlineto{\pgfqpoint{5.124516in}{3.381905in}}%
\pgfpathlineto{\pgfqpoint{5.124516in}{3.479128in}}%
\pgfpathlineto{\pgfqpoint{4.846738in}{3.479128in}}%
\pgfpathlineto{\pgfqpoint{4.846738in}{3.381905in}}%
\pgfpathclose%
\pgfusepath{fill}%
\end{pgfscope}%
\begin{pgfscope}%
\definecolor{textcolor}{rgb}{0.000000,0.000000,0.000000}%
\pgfsetstrokecolor{textcolor}%
\pgfsetfillcolor{textcolor}%
\pgftext[x=5.235627in,y=3.381905in,left,base]{\color{textcolor}\sffamily\fontsize{10.000000}{12.000000}\selectfont P@5}%
\end{pgfscope}%
\begin{pgfscope}%
\pgfsetbuttcap%
\pgfsetmiterjoin%
\definecolor{currentfill}{rgb}{1.000000,0.000000,0.000000}%
\pgfsetfillcolor{currentfill}%
\pgfsetlinewidth{0.000000pt}%
\definecolor{currentstroke}{rgb}{0.000000,0.000000,0.000000}%
\pgfsetstrokecolor{currentstroke}%
\pgfsetstrokeopacity{0.000000}%
\pgfsetdash{}{0pt}%
\pgfpathmoveto{\pgfqpoint{4.846738in}{3.178048in}}%
\pgfpathlineto{\pgfqpoint{5.124516in}{3.178048in}}%
\pgfpathlineto{\pgfqpoint{5.124516in}{3.275270in}}%
\pgfpathlineto{\pgfqpoint{4.846738in}{3.275270in}}%
\pgfpathlineto{\pgfqpoint{4.846738in}{3.178048in}}%
\pgfpathclose%
\pgfusepath{fill}%
\end{pgfscope}%
\begin{pgfscope}%
\definecolor{textcolor}{rgb}{0.000000,0.000000,0.000000}%
\pgfsetstrokecolor{textcolor}%
\pgfsetfillcolor{textcolor}%
\pgftext[x=5.235627in,y=3.178048in,left,base]{\color{textcolor}\sffamily\fontsize{10.000000}{12.000000}\selectfont P@1}%
\end{pgfscope}%
\end{pgfpicture}%
\makeatother%
\endgroup%
}
%    \caption{Average number of distinct found across n-holes.}
%  \end{figure}

  \section{Prior work}

  Our work is closely related to Boltzmann sampling~\cite{duchon2004boltzmann} in the case of sampling with replacement, but does not use rejection. Piantodosi~\cite{piantadosi2023enumerate} defines a similar construction in the case of sampling without replacement, but it assumes the CFG generates an infinite language and its productions have a certain ordering. In the setting where the template contains only holes, BCFL completion coincides with the Chomsky-Sch\"utzenberger enumeration theorem~\cite{panholzer2005grobner}, which provides a constructive method for counting finite-length words in unambiguous CFLs (i.e., $\ell\cap\Sigma^n$). Our construction is more general, being designed to handle any CFG and template, regardless of ambiguity, finitude, or production ordering.

  Loosely adapted from semiring parsing~\cite{goodman1999semiring} and Valiant's algorithm~\cite{valiant1975general}, our parser also supports bounded generation. We construct a nested datatype~\cite{bird1998nested} that compactly represents candidate parse forests and which can be used to sample trees with or without replacement by sampling a finite integer range, enabling communication-free parallelization. %Sampling parse trees in CFGs can be viewed as sampling proofs of $S$ in a weak kind of proof system~\cite{opedal2023efficient}.

  \section{Conclusion}

  We have presented a novel sound and complete algorithm for sampling trees in bounded context-free languages with and without replacement. This technique has applications to code completion and program repair. In future work, we intend to provide a proof of correctness and extend our technique to handle sampling from Boolean and conjunctive languages. A reference implementation for the $\mathbb{T}_2$ datatype is provided in Kotlin and may be found at the \href{https://github.com/breandan/galoisenne/blob/adcb90ac775f17582c5f9fbc4da041b0cf4bf3dc/src/commonMain/kotlin/ai/hypergraph/kaliningraph/parsing/SeqValiant.kt}{URL} linked below.\footnote{\url{https://github.com/breandan/galoisenne/blob/3a2e811e8652ba29891aa21789ef0836ed19d257/src/commonMain/kotlin/ai/hypergraph/kaliningraph/parsing/SeqValiant.kt}}

  \section{Acknowledgements}

  The author wishes to thank David Bieber for mentioning analytic combinatorics during a sunrise hike and David Yu-Tung Hui for sharing his thoughts on Boltzmann sampling.

  \bibliography{../bib/acmart}
\end{document}