%% For double-blind review submission, w/o CCS and ACM Reference (max submission space)
%\documentclass[sigplan,10pt,review,anonymous]{acmart}
%\settopmatter{printfolios=false,printccs=false,printacmref=false}
%% For double-blind review submission, w/ CCS and ACM Reference
%\documentclass[sigplan,review,anonymous]{acmart}\settopmatter{printfolios=true}
%% For single-blind review submission, w/o CCS and ACM Reference (max submission space)
%\documentclass[sigplan,review]{acmart}\settopmatter{printfolios=true,printccs=false,printacmref=false}
%% For single-blind review submission, w/ CCS and ACM Reference
%\documentclass[sigplan,review]{acmart}\settopmatter{printfolios=true}
%% For final camera-ready submission, w/ required CCS and ACM Reference
%\documentclass[sigplan,nonacm]{acmart}
\documentclass[sigplan,review,anonymous,acmsmall]{acmart}\settopmatter{printfolios=false,printccs=false,printacmref=false}


%% Conference information
%% Supplied to authors by publisher for camera-ready submission;
%% use defaults for review submission.
\acmConference[SPLASH'24]{ACM SIGPLAN conference on Systems, Programming, Languages, and Applications: Software for Humanity}{October 22-27, 2024}{Pasadena, California, United States}
%\acmYear{2018}
%\acmISBN{} % \acmISBN{978-x-xxxx-xxxx-x/YY/MM}
%\acmDOI{} % \acmDOI{10.1145/nnnnnnn.nnnnnnn}
%\startPage{1}

%% Copyright information
%% Supplied to authors (based on authors' rights management selection;
%% see authors.acm.org) by publisher for camera-ready submission;
%% use 'none' for review submission.
\setcopyright{none}
%\setcopyright{acmcopyright}
%\setcopyright{acmlicensed}
%\setcopyright{rightsretained}
%\copyrightyear{2018}           %% If different from \acmYear

%% Bibliography style
\bibliographystyle{acmart}

\usepackage{dsfont}
\usepackage{stmaryrd}
\usepackage{colortbl}
\usepackage{hyperref}

\usepackage{amsmath}
\DeclareMathOperator*{\argmax}{argmax}
\DeclareMathOperator*{\argmin}{argmin}
\usepackage{amssymb}

\usepackage[dvipsnames, table]{xcolor}
\usepackage{textcomp}

% Packages
\usepackage[pdf]{graphviz}
\usepackage{mathrsfs}

\newcommand*\circled[1]{\tikz[baseline=-0.1cm]{
  \node[shape=circle,draw,inner sep=0.48pt] (char) {\fontsize{7}{12}\textsf{#1}};}}

\DeclareMathAlphabet{\mathcal}{OMS}{cmsy}{m}{n}
\usepackage{cancel}
\newcommand\ccancel[2][red]{\renewcommand\CancelColor{\color{#1}}\cancel{#2}}
\newcommand{\nDownarrow}{\ensuremath{\text{ }\cancel{\Downarrow}\text{ }}}
\usepackage{centernot}

\usepackage{pgfplots, pgfplotstable}
\pgfplotsset{compat=1.7}
\usepgfplotslibrary{fillbetween}
\usetikzlibrary{patterns}
\pgfmathdeclarefunction{gauss}{2}{\pgfmathparse{1/(#2*sqrt(2*pi))*exp(-((x-#1)^2)/(2*#2^2))}}
\pgfmathdeclarefunction{nil}{1}{\pgfmathparse{0.001}}

\usepackage{arydshln}
\usepackage{adjustbox}
\usepackage{enumerate}
\usepackage{enumitem}
\usepackage{tikz-cd}
\usetikzlibrary{calc}
\usepackage{amsfonts}
%\usepackage{prooftrees}
\usepackage{bussproofs}
\renewcommand{\sectionautorefname}{\S}
\renewcommand{\subsectionautorefname}{\S}
\usepackage{float}

\usepackage{tikz-3dplot}
\usetikzlibrary{3d}
\usetikzlibrary{calligraphy}
\newif\ifshowcellnumber
\showcellnumbertrue

\usepackage{algorithm}
\usepackage{algpseudocode}
\usepackage{algorithmicx}
\usepackage{sourcecodepro}
\usepackage{tikz-qtree}
\usepackage{amsthm}
\usepackage{bm}
\usetikzlibrary{bayesnet}
\usetikzlibrary{arrows}
\usepackage{subcaption}
\usetikzlibrary{backgrounds}
\usetikzlibrary{tikzmark}

\newcommand{\E}{\mathbb{E}}
\newcommand{\Var}{\mathrm{Var}}
\newcommand{\Cov}{\mathrm{Cov}}

\newcommand{\CompOrder}{\mathcal{O}}
\def\graphspace{\mathbf{G}}
\def\Uniform{\mbox{\rm Uniform}}
\def\Gaussian{\mbox{\rm Gaussian}}
\def\Bernoulli{\mbox{\rm Bernoulli}}
\def\Dirichlet{\mbox{\rm Dirichlet}}

\usepackage{mathtools}% superior to amsmath
\usepackage{tikz}
% Packages
\usepackage{listings}
\DeclareRobustCommand{\hlred}[1]{{\sethlcolor{pink}\hl{#1}}}
\usepackage{fontspec}

\setmonofont[Scale=0.8]{JetBrainsMono}[
  Contextuals={Alternate},
  Path=./font/,
  Extension = .ttf,
  UprightFont=*-Regular,
  BoldFont=*-Bold,
  ItalicFont=*-Italic,
  BoldItalicFont=*-BoldItalic
]

\usepackage[skins,breakable,listings]{tcolorbox}

\lstdefinelanguage{kotlin}{
  comment=[l]{//},
  commentstyle={\color{gray}\ttfamily},
  emph={delegate, filter, firstOrNull, forEach, it, lazy, mapNotNull, println, repeat, assert, with, head, tail, len, return@},
  numberstyle=\noncopyable,
  identifierstyle=\color{black},
  keywords={abstract, actual, as, as?, break, by, class, companion, continue, data, do, dynamic, else, enum, expect, false, final, for, fun, get, if, import, in, infix, interface, internal, is, null, object, open, operator, override, package, private, public, return, sealed, set, super, suspend, this, throw, true, try, catch, typealias, val, var, vararg, when, where, while, tailrec, reified},
  keywordstyle={\bfseries},
  morecomment=[s]{/*}{*/},
  morestring=[b]",
  morestring=[s]{"""*}{*"""},
  ndkeywords={@Deprecated, @JvmField, @JvmName, @JvmOverloads, @JvmStatic, @JvmSynthetic, Array, Byte, Double, Float, Boolean, Int, Integer, Iterable, Long, Runnable, Short, String, int},
  ndkeywordstyle={\bfseries},
  sensitive=true,
  stringstyle={\ttfamily},
  literate={`}{{\char0}}1,
  escapeinside={(*@}{@*)}
}
\lstdefinelanguage{tidy}{
  comment=[l]{//},
  commentstyle={\color{gray}\ttfamily},
  emph={|, ->, ---},
  emphstyle={\color{red}},
  identifierstyle=\color{black},
  keywords={\|, ->, ---},
  otherkeywords={|,->},
  morekeywords={|,->},
  keywordstyle={\color{blue}\bfseries},
  morecomment=[s]{/*}{*/},
  morestring=[b]",
  morestring=[s]{"""*}{*"""},
  ndkeywords={@Deprecated, @JvmField, @JvmName, @JvmOverloads, @JvmStatic, @JvmSynthetic, Array, Byte, Double, Float, Int, Integer, Iterable, Long, Runnable, Short, String},
  ndkeywordstyle={\color{orange}\bfseries},
  sensitive=true,
  stringstyle={\color{green}\ttfamily},
  literate={`}{{\char0}}1
}

%%%%%%%%%%%%%%%%%%%%%%%%%%%%%%%%%%%%%%%%%%%
%
% Color boxes
%
%%%%%%%%%%%%%%%%%%%%%%%%%%%%%%%%%%%%%%%%%%%

\tcbset{
  enhanced jigsaw,
  breakable,
  listing only,
%  boxsep=-1pt,
%  top=-1pt,
  bottom=0.1cm,
  right=0.5cm,
  overlay first={
    \node[black!50] (S) at (frame.south) {\Large\ding{34}};
    \draw[dashed,black!50] (frame.south west) -- (S) -- (frame.south east);
  },
  overlay middle={
    \node[black!50] (S) at (frame.south) {\Large\ding{34}};
    \draw[dashed,black!50] (frame.south west) -- (S) -- (frame.south east);
    \node[black!50] (S) at (frame.north) {\Large\ding{34}};
    \draw[dashed,black!50] (frame.north west) -- (S) -- (frame.north east);
  },
  overlay last={
    \node[black!50] (S) at (frame.north) {\Large\ding{34}};
    \draw[dashed,black!50] (frame.north west) -- (S) -- (frame.north east);
  },
  before={\par\vspace{5pt}},
  after={\par\vspace{\parskip}\noindent}
}

\definecolor{slightgray}{rgb}{0.90, 0.90, 0.90}

\usepackage{soul}
\makeatletter
\def\SOUL@hlpreamble{%
  \setul{}{3.0ex}%
  \let\SOUL@stcolor\SOUL@hlcolor%
  \SOUL@stpreamble%
}
\makeatother

\newcommand{\inline}[1]{%
  \begingroup%
  \sethlcolor{slightgray}%
  \hl{\ttfamily\footnotesize #1}%
  \endgroup
}

\newcommand{\tinline}[1]{%
  \begingroup%
  \sethlcolor{slightgray}%
  \hl{\ttfamily\tiny #1}%
  \endgroup
}

\newtcblisting{halftidyinput}[1][]{%
  left skip=0.7cm,
  width=6cm,
%  left=-0.01cm,
  top=-0.1cm,
  bottom=-0.35cm,
  listing options={
    language=tidy,
    basicstyle=\ttfamily\small,
%numberstyle=\footnotesize,
    showstringspaces=false,
    tabsize=2,
    breaklines=true,
    numbers=none,
    inputencoding=utf8,
    escapeinside={(*@}{@*)},
    #1
  },
  underlay unbroken and first={%
    \path[draw=none] (interior.north west) rectangle node[white]{\includegraphics[width=4mm]{../figures/tidyparse_logo.png}} ([xshift=-10mm,yshift=-7mm]interior.north west);
  }
}

\newtcblisting{wholetidyinput}[1][]{%
  left skip=0.7cm,
  top=0.1cm,
  middle=0mm,
  boxsep=0mm,
  listing options={
    language=tidy,
    basicstyle=\ttfamily\small,
%numberstyle=\footnotesize,
    showstringspaces=false,
    tabsize=2,
    breaklines=true,
    numbers=none,
    inputencoding=utf8,
    escapeinside={(*@}{@*)},
    #1
  },
  underlay unbroken and first={%
      \path[draw=none] (interior.north west) rectangle node[white]{\includegraphics[width=4mm]{../figures/tidyparse_logo.png}} ([xshift=-10mm,yshift=-9mm]interior.north west);
  }
}

\definecolor{A}{RGB}{6,150,104}
\definecolor{B}{RGB}{196,74,137}
\definecolor{C}{RGB}{117,237,133}
\definecolor{D}{RGB}{246,46,243}
\definecolor{E}{RGB}{89,162,12}
\definecolor{F}{RGB}{113,12,158}
\definecolor{G}{RGB}{191,205,142}
\definecolor{H}{RGB}{51,58,158}
\definecolor{I}{RGB}{244,212,3}
\definecolor{J}{RGB}{37,36,249}
\definecolor{K}{RGB}{253,165,71}
\definecolor{L}{RGB}{27,81,29}
\colorlet{LA}{A!30}
\colorlet{LB}{B!30}
\colorlet{LC}{C!30}
\colorlet{LD}{D!30}
\colorlet{LE}{E!30}
\colorlet{LF}{F!30}
\colorlet{LG}{G!30}
\colorlet{LH}{H!30}
\colorlet{LI}{I!30}
\colorlet{LJ}{J!30}
\colorlet{LK}{K!30}
\colorlet{LL}{L!30}
\newcommand{\hiliA}[1]{%
  \colorbox{LA}{$#1$}}
\newcommand{\hiliB}[1]{%
  \colorbox{LB}{$#1$}}
\newcommand{\hiliC}[1]{%
  \colorbox{LC}{$#1$}}
\newcommand{\hiliD}[1]{%
  \colorbox{LD}{$#1$}}
\newcommand{\hiliE}[1]{%
  \colorbox{LE}{$#1$}}
\newcommand{\hiliF}[1]{%
  \colorbox{LF}{$#1$}}
\newcommand{\hiliG}[1]{%
  \colorbox{LG}{$#1$}}
\newcommand{\hiliH}[1]{%
  \colorbox{LH}{$#1$}}
\newcommand{\hiliI}[1]{%
  \colorbox{LI}{$#1$}}
\newcommand{\hiliJ}[1]{%
  \colorbox{LJ}{$#1$}}
\newcommand{\hiliK}[1]{%
  \colorbox{LK}{$#1$}}
\newcommand{\hiliL}[1]{%
  \colorbox{LL}{$#1$}}
\newcommand{\highlight}[1]{%
  \colorbox{lgray}{$#1$}}
\colorlet{lred}{red!30}
\colorlet{lorange}{orange!30}
\colorlet{lgreen}{green!30}
\colorlet{lgray}{black!15}
\colorlet{dgray}{black!75}
\DeclareRobustCommand{\hlred}[1]{{\sethlcolor{lred}\hl{#1}}}
\DeclareRobustCommand{\hlorange}[1]{{\sethlcolor{lorange}\hl{#1}}}
\DeclareRobustCommand{\hlgreen}[1]{{\sethlcolor{lgreen}\hl{#1}}}
\DeclareRobustCommand{\hlgray}[1]{{\sethlcolor{lgray}\hl{#1}}}
\DeclareRobustCommand{\caret}[1]{{\sethlcolor{dgray}\textcolor{white}{\hl{#1}}}}

\usepackage{url}
\usepackage{qtree}

\usepackage{filecontents}
\usepackage{pstricks-add}
\usepackage{emoji}
\usepackage{alltt}
\usepackage{nicematrix}
\usepackage{graphicx}
\usepackage{ulem}
\usepackage{upquote}
\tikzstyle{every picture}+=[remember picture]
\usepackage{menukeys}
\pgfplotstableread[col sep=comma,]{timings_loc.csv}\loctimings
\pgfplotstableread[col sep=comma,]{timings_unloc.csv}\unloctimings

\makeatletter
\DeclareRobustCommand{\cev}[1]{%
  {\mathpalette\do@cev{#1}}%
}
\newcommand{\do@cev}[2]{%
  \vbox{\offinterlineskip
  \sbox\z@{$\m@th#1 x$}%
  \ialign{##\cr
  \hidewidth\reflectbox{$\m@th#1\vec{}\mkern4mu$}\hidewidth\cr
  \noalign{\kern-\ht\z@}
    $\m@th#1#2$\cr
  }%
  }%
}
\makeatother

\makeatletter
\DeclareRobustCommand{\pder}[1]{%
  \@ifnextchar\bgroup{\@pder{#1}}{\@pder{}{#1}}}
\newcommand{\@pder}[2]{\frac{\partial#1}{\partial#2}}
\makeatother

\newcommand{\shup}{\shortuparrow}
\newcommand{\shri}{\shortrightarrow}
\newcommand{\shur}{\shup\hspace{-5pt}\shri}

\makeatletter
\def\squiggly{\bgroup \markoverwith{\textcolor{red}{\lower3\p@\hbox{\sixly \char58}}}\ULon}
\makeatother

\newcommand{\err}[1]{\smash{\squiggly{#1}{}}}
\newcommand{\stirlingii}{\genfrac{\{}{\}}{0pt}{}}

%======== Arrows =========
\newcommand{\knightarrow}{
  \tikz{
    \fill (0pt,0pt) circle [radius = 1pt];
    \fill (0pt,6pt) circle [radius = 1pt];
    \fill (6pt,0pt) circle [radius = 1pt];
    \fill (6pt,6pt) circle [radius = 1pt];
    \fill (12pt,0pt) circle [radius = 1pt];
    \fill (12pt,6pt) circle [radius = 1pt];
    \fill (6pt,0pt) circle [radius = 1pt];
    \fill (12pt,0pt) circle [radius = 1pt];
    \draw [-to] (0pt,0pt) -- (12pt,6pt);
  }
}

\newcommand{\kingarrow}{
  \tikz{
    \fill (0pt,0pt) circle [radius = 1pt];
    \fill (6pt,0pt) circle [radius = 1pt];
    \fill (0pt,6pt) circle [radius = 1pt];
    \fill (6pt,6pt) circle [radius = 1pt];
    \draw [-to] (0pt,0pt) -- (6pt,6pt);
    \draw [-to] (0pt,0pt) -- (0pt,6pt);
    \draw [-to] (0pt,0pt) -- (6pt,0pt);
  }
}

\newcommand{\knightkingarrow}{
  \tikz{
    \fill (0pt,0pt) circle [radius = 1pt];
    \fill (0pt,6pt) circle [radius = 1pt];
    \fill (6pt,0pt) circle [radius = 1pt];
    \fill (6pt,6pt) circle [radius = 1pt];
    \fill (12pt,0pt) circle [radius = 1pt];
    \fill (12pt,6pt) circle [radius = 1pt];
    \draw [-to] (0pt,0pt) -- (6pt,6pt);
    \draw [-to] (0pt,0pt) -- (0pt,6pt);
    \draw [-to] (0pt,0pt) -- (6pt,0pt);
    \draw [-to] (0pt,0pt) -- (12pt,6pt);
  }
}

%======== Arrows =========

\usetikzlibrary{decorations.pathreplacing,automata,calc,positioning,matrix,fit}
\usepackage{wrapfig}

\newcommand{\mkTrellis}[1]{
  \begin{tikzpicture}
    \def\dx{20pt}
    \def\dy{30pt}
    \newcounter{i}
    \stepcounter{i}
    \node[circle, draw, fill=black!30] (\arabic{i}) at (0,0){};
    \foreach [count=\i] \x in {2,...,#1}{
      \pgfmathsetmacro{\lox}{\x-1}%
      \pgfmathsetmacro{\loxt}{\x-3}%
      \foreach [count=\j] \xx in {-\lox,-\loxt,...,\lox}{
        \pgfmathsetmacro{\jj}{\j-1}%
        \stepcounter{i}
        \pgfmathsetmacro{\kk}{\xx-2}%
        \pgfmathsetmacro{\lbl}{\lox!/(\jj!*(\lox-\jj)!)}
        \ifnum\x<\kk
        \pgfmath\node[circle, draw]  (\arabic{i}) at (\xx*\dx, -\lox*\dy) {};
        \else
        \pgfmath\node[circle, draw, fill=black!30]  (\arabic{i}) at (\xx*\dx, -\lox*\dy) {};
        \fi
      }
    }
    \newcounter{z}
    \newcounter{xn}
    \newcounter{xnn}
    \pgfmathsetmacro{\maxx}{#1 - 1}
    \foreach \x in {1,...,\maxx}{
      \foreach \xx in {1,...,\x}{
        \stepcounter{z}
        \setcounter{xn}{\arabic{z}}
        \addtocounter{xn}{\x}
        \setcounter{xnn}{\arabic{xn}}
        \stepcounter{xnn}
        \draw [<-] (\arabic{z}) -- (\arabic{xn});
        \draw [<-] (\arabic{z}) -- (\arabic{xnn});
      }
    }
  \end{tikzpicture}
}

\newcommand{\dx}{20pt}
\newcommand{\dy}{30pt}
\newcounter{i}
\newcounter{z}
\newcounter{xn}
\newcounter{xnn}
\newcommand{\mkTrellisAppend}[1]{
  \begin{tikzpicture}
    \setcounter{i}{0}
    \setcounter{z}{0}
    \setcounter{xn}{0}
    \setcounter{xnn}{0}
    \stepcounter{i}
    \node[circle, draw] (\arabic{i}) at (0,0){};
    \foreach [count=\i] \x in {2,...,#1}{
      \pgfmathsetmacro{\lox}{\x-1}%
      \pgfmathsetmacro{\loxt}{\x-3}%
      \foreach [count=\j] \xx in {-\lox,-\loxt,...,\lox}{
        \pgfmathsetmacro{\jj}{\j-1}%
        \stepcounter{i}
        \pgfmathsetmacro{\kk}{\xx+2}%
        \pgfmathsetmacro{\lbl}{\lox!/(\jj!*(\lox-\jj)!)}
        \ifnum\x>\kk
        \pgfmath\node[circle, draw, fill=black!30]  (\arabic{i}) at (\xx*\dx, -\lox*\dy) {};
        \else
        \pgfmath\node[circle, draw]  (\arabic{i}) at (\xx*\dx, -\lox*\dy) {};
        \fi
      }
    }
    \pgfmathsetmacro{\maxx}{#1 - 1}
    \foreach \x in {1,...,\maxx}{
      \foreach \xx in {1,...,\x}{
        \stepcounter{z}
        \setcounter{xn}{\arabic{z}}
        \addtocounter{xn}{\x}
        \setcounter{xnn}{\arabic{xn}}
        \stepcounter{xnn}
        \draw [<-] (\arabic{z}) -- (\arabic{xn});
        \draw [<-] (\arabic{z}) -- (\arabic{xnn});
      }
    }
  \end{tikzpicture}
}

\newcommand{\mkTrellisInsert}[1]{
  \begin{tikzpicture}
    \setcounter{i}{0}
    \setcounter{z}{0}
    \setcounter{xn}{0}
    \setcounter{xnn}{0}
    \stepcounter{i}
    \node[circle, draw] (\arabic{i}) at (0,0){};
    \foreach [count=\i] \x in {2,...,#1}{
      \pgfmathsetmacro{\lox}{\x-1}%
      \pgfmathsetmacro{\loxt}{\x-3}%
      \foreach [count=\j] \xx in {-\lox,-\loxt,...,\lox}{
        \pgfmathsetmacro{\jj}{\j-1}%
        \stepcounter{i}
        \pgfmathsetmacro{\mp}{\xx+#1}%
        \pgfmathsetmacro{\mq}{\xx+\x}%
        \pgfmathsetmacro{\lbl}{\lox!/(\jj!*(\lox-\jj)!)}
        \ifnum\x>\mp
        \pgfmath\node[circle, draw, fill=black!30]  (\arabic{i}) at (\xx*\dx, -\lox*\dy) {};
        \else
        \ifnum#1<\mq
        \pgfmath\node[circle, draw, fill=black!30]  (\arabic{i}) at (\xx*\dx, -\lox*\dy) {};
        \else
        \pgfmath\node[circle, draw]  (\arabic{i}) at (\xx*\dx, -\lox*\dy) {};
        \fi
        \fi

      }
    }
    \pgfmathsetmacro{\maxx}{#1 - 1}
    \foreach \x in {1,...,\maxx}{
      \foreach \xx in {1,...,\x}{
        \stepcounter{z}
        \setcounter{xn}{\arabic{z}}
        \addtocounter{xn}{\x}
        \setcounter{xnn}{\arabic{xn}}
        \stepcounter{xnn}
        \draw [<-] (\arabic{z}) -- (\arabic{xn});
        \draw [<-] (\arabic{z}) -- (\arabic{xnn});
      }
    }
  \end{tikzpicture}
}

\usetikzlibrary{automata, positioning, arrows}

\newcommand{\nobarfrac}{\genfrac{}{}{0pt}{}}
\pgfplotstableread[col sep=comma,]{timings_loc.csv}\loctimings
\pgfplotstableread[col sep=comma,]{timings_unloc.csv}\unloctimings
\pgfplotstableread[col sep=comma,]{natural_errors.csv}\naturalerrors
\pgfplotstableread[col sep=comma,]{synthetic_errors.csv}\syntheticerrors

\begin{document}
%
  \title{Syntax Repair as Language Intersection}
  %
  \begin{abstract}
    We introduce a new technique for correcting syntax errors in arbitrary context-free languages. Our work comes from the observation that syntax errors with a small repair typically have very few unique small repairs, which can usually be enumerated up to a small edit distance then ranked within a short amount of time. Furthermore, we place a heavy emphasis on precision: the enumerated set must contain every possible repair within a few edits and no invalid repairs. To do so, we reduce CFL recognition onto Boolean tensor completion, then model error correction as a language intersection problem between a Levenshtein automaton and a context-free grammar. To decode the solutions, we then sample trees without replacement from the intersection grammar, which yields valid repairs within a certain Levenshtein distance. Finally, we rank all repairs discovered within 60 seconds by a Markov chain.
    \keywords{Error correction \and CFL reachability \and Langauge games.}
  \end{abstract}
%
%\titlerunning{Abbreviated paper title}
% If the paper title is too long for the running head, you can set
% an abbreviated paper title here
%
%  \author{Breandan Considine\inst{1} \and
%  Jin Guo\inst{1}\and
%  Xujie Si\inst{2}}
%
%  \authorrunning{Considine et al.}
% First names are abbreviated in the running head.
% If there are more than two authors, 'et al.' is used.
%
%  \institute{McGill University, Montr\'eal, QC H2R 2Z4, Canada\\
%  \email{\{breandan.considine@mail, jguo@cs\}.mcgill.ca}\and
%  University of Toronto, Toronto, ON, M5S 1A1 Canada\\
%  \email{six@utoronto.ca}}
%
  \maketitle              % typeset the header of the contribution


  \section{Introduction}

  Syntax repair is the problem of modifying an invalid sentence so it conforms to some grammar. Prior work has been devoted to fixing syntax errors using handcrafted heuristics. This work features a variety of approaches including rule-based systems and statistical language models. However, these techniques are often brittle, and are susceptible to misgeneralization. In a prior paper published in SPLASH, the authors sample production rules from an error correcting grammar. While theoretically sound, this technique is incomplete, i.e., not guaranteed to sample all edits within a certain Levenshtein distance, and no more. In this paper, we demonsrate it is possible to attain a significant advantage by synthesizing and scoring all repairs within a certain Levenshtein distance. Not only does this technique guarantee perfect generalization, but also helps with precision.

  We take a first-principles approach making no assumptions about the sentence or grammar and focuses on correctness and end-to-end latency. Our technique is simple:

  \begin{enumerate}
    \item We first reduce the problem of CFL recognition to Boolean tensor completion, then use that to compute the Parikh image of the CFL. This follows from a straightforward extension of the Chomsky-Sch\"utzenberger enumeration theorem.
    \item We then model syntax correction as a language intersection problem between a Levenshtein automaton and a context-free grammar, which we explicitly materialize using a specialized version of the Bar-Hillel construction to Levenshtein intersections. This greatly reduces the number of generated productions.
    \item To decode the members from the intersection grammar, we sample trees without replacement by constructing a bijection between syntax trees and the integers, then sampling integers uniformly without replacement from a finite range. This yields concrete repairs within a certain Levenshtein distance.
    \item Finally, we rerank all repairs found before a fixed timeout by n-gram perplexity.
  \end{enumerate}

  Though simple, this technique outperforms SoTA syntax repair techniques. Its efficacy owes to the fact it does not sample edits or nor productions, but unique, fully formed repairs within a certain Levenshtein distance. It is sound and complete up to a Levenshtein bound - i.e., it will find all repairs within an arbitrary Levenshtein distance, and no more. Often the language of small repairs is surprisingly small compared with the language of all possible edits, enabling us to efficiently synthesize and score every possible solution. This offers a significant advantage over memoryless sampling techniques.

  \pagebreak

  \section{Example}

  Consider the following Python snippet, which contains a small syntax error:\\

  \noindent\texttt{def prepend(i, k, L=[]) n and [prepend(i - 1, k, [b] + L) for b in range(k)]}\\

  We can fix it by inserting a colon after the function definition, yielding:\\

  \noindent\texttt{def prepend(i, k, L=[])\hlgreen{:} n and [prepend(i - 1, k, [b] + L) for b in range(k)]}\\

  A careful observer will note that there is only one way to repair this Python snippet by making a single edit. In fact, many programming languages share this curious property: syntax errors with a small repair have few uniquely small repairs. Valid sentences corrupted by a few small errors rarely have many small corrections. We call such sentences \textit{metastable}, since they are relatively stable to small perturbations, as likely to be incurred by a careless typist or novice programmer.
%  Consider the following Kotlin snippet:\\
%
%  \texttt{fun main() = try \{ fetch() \} except(e: Exception) \{ handle(e) \}}\\
%
%  \noindent Again, there are thousands of possible single-token edits, only one of which is a valid repair:\\
%
%  \texttt{fun main() = try \{ fetch() \} \hlorange{catch}(e: Exception) \{ handle(e) \}}\\

  Let us consider a slightly more ambiguous error: \texttt{v = df.iloc(5:, 2:)}. Assuming an alphabet of just one hundred lexical tokens, this tiny statement has millions of possible two-token edits, yet only six of those possibilities are accepted by the Python parser:

%\setlength{\columnsep}{-10pt}
%\setlength{\columnseprule}{-10pt}
%\noindent\begin{multicols}{3}
%  \begin{enumerate}
%    \item\texttt{v = df.iloc(5\hlred{:}, 2\hlorange{,})}\\
%    \item\texttt{v = df.iloc(5\hlgreen{[}:, 2:\hlgreen{]})}\\
%    \item\texttt{v = df.iloc\hlorange{[}5:, 2:\hlorange{]}}\\
%    \item\texttt{v = df.iloc(5\hlorange{)}, 2\hlorange{(})}\\
%    \item\texttt{v = df.iloc(5\hlred{:}, 2\hlred{:})}\\
%    \item\texttt{v = df.iloc(5\hlgreen{[}:, 2\hlorange{]})}
%  \end{enumerate}
%\end{multicols}
  \begin{figure}[h!]
    \noindent\begin{tabular}{@{}l@{\hspace{10pt}}l@{\hspace{10pt}}l@{}}
    (1) \texttt{v = df.iloc(5\hlred{:}, 2\hlorange{,})} & (3) \texttt{v = df.iloc(5\hlgreen{[}:, 2:\hlgreen{]})} & (5) \texttt{v = df.iloc\hlorange{[}5:, 2:\hlorange{]}} \\\\
    (2) \texttt{v = df.iloc(5\hlorange{)}, 2\hlorange{(})} & (4) \texttt{v = df.iloc(5\hlred{:}, 2\hlred{:})} & (6) \texttt{v = df.iloc(5\hlgreen{[}:, 2\hlorange{]})} \\
    \end{tabular}
  \end{figure}

  With some typing information, we could easily narrow the results, but even in the absence of semantic constraints, one can probably rule out (2, 3, 6) given that \texttt{5[} and \texttt{2(} are rare bigrams in Python, and knowing \texttt{df.iloc} is often followed by \texttt{[}, determine (5) is most natural. This is the key insight behind our approach: we can usually locate the intended fix by exhaustively searching small repairs. As the set of small repairs is itself often small, if only we had some procedure to distinguish valid from invalid patches, the resulting solutions could be simply ranked by naturalness.

  The trouble is that any such procedure must be highly sample-efficient. We cannot afford to sample the universe of possible d-token edits, then reject invalid ones -- assuming it takes just 10ms to generate and check each sample, (1-6) could take 24+ hours to find. The hardness of brute-force search grows superpolynomially with edit distance, sentence length and alphabet size. We need a more efficient procedure for sampling all and only small valid repairs.

  \section{Problem}

  We can model syntax repair as a language intersection problem between a context-free language (CFL) and a regular language.

  \begin{definition}[Bounded Levenshtein-CFL reachability]
    Given a CFL $\ell$ and an invalid string $\err{\sigma}: \ell^\complement$, the BCFLR problem is to find every valid string reachable within $d$ edits of $\err{\sigma}$, i.e., letting $\Delta$ be the Levenshtein metric and $L(\err\sigma, d) \coloneqq \{\sigma \mid \Delta(\err{\sigma}, \sigma) \leq d\}$, we seek to find $A = L(\err\sigma, d) \cap \ell$.
  \end{definition}

%  To solve this problem, it is convenient to first consider intersections with a finite-length string with holes, then turn our attention back to BCFLR.
%

  As the admissible set $A$ is typically under-determined, we want a procedure which prioritizes natural and valid repairs over unnatural but valid repairs:

  \begin{definition}[Ranked repair]\label{def:ranked-repair}
    Given a finite language $A = L(\err\sigma, d) \cap \ell$ and a probabilistic language model $\text{P}_\theta: \Sigma^* \rightarrow [0, 1] \subset \mathbb{R}$, the ranked repair problem is to find the top-$k$ maximum likelihood repairs under the language model. That is,
    \begin{equation}
      R(A, P_\theta) \coloneqq \argmax_{\{\bm{\sigma} \mid \bm{\sigma} \subseteq A, |\bm{\sigma}| \leq k\}} \sum_{\sigma \in \bm{\sigma}}\text{P}_\theta(\sigma\mid\err\sigma)
    \end{equation}
    % On average, across all $G, \sigma$ $\hat{R}$ should approximate $R$.
%    We want a procedure $\hat{R}$, minimizing $\mathbb{E}_{G, \sigma}\big[D_{\text{KL}}(\hat{R} \parallel R)\big]$ and wallclock runtime.
  \end{definition}

  The problem of ranked repair is typically solved by learning a distribution over strings, however this approach can be sample-inefficient and generalize poorly to new languages. Representing the problem as a distribution over $\Sigma^*$ forces the language model to jointly learn syntax and stylometry. As our work demonstrates, this problem can be factorized into a bilevel objective: first retrieval, then ranking. By ensuring retrieval is sufficiently precise and exhaustive, maximizing likelihood over the admissible set can be achieved with a much simpler, syntax-oblivious language model.

  Even with an extremely efficient approximate sampler for $\sigma \sim \ell_\cap$, due to the size of $\ell$ and $L(\err\sigma, d)$, it would be intractable to sample either $\ell$ or $L(\err\sigma, d)$, then reject invalid ($\sigma \notin \ell$) or unreachable ($\sigma \notin L(\err\sigma, d)$) edits, and completely out of the question to sample $\sigma \sim \Sigma^*$ as do many large language models. Instead, we will explicitly construct a grammar generating $\ell \cap L(\err\sigma, d)$, sample from it, then rerank all repairs after a fixed timeout. So long as $|\ell_\cap|$ is sufficiently small and recognizes all and only small repairs, our sampler is sure to retrieve the most natural repair and terminate quickly.

  \section{Method}

  \begin{wrapfigure}{r}{0.45\textwidth}
%\begin{figure}[h!]
    \vspace{-1.2cm}
    \begin{center}
      \resizebox{0.45\textwidth}{!}{
        \begin{tikzpicture}[node distance=2cm]
          \node (start) [startstop] {Start};
          \node (pro1) [process, below of=start, yshift=-0.5cm] {$G_\cap \leftarrow G\cap\Delta(\err\sigma, d)$};
          \node (pcfg) [io, left of=pro1, xshift=-3cm] {[P]CFG};
          \node (lnfa) [io, right of=pro1, xshift=3cm] {L-WFA};

          \node (code) [io, right of=start,xshift=3cm] {Code};
          \node (synt) [io, left of=start,xshift=-3cm] {Syntax};

          \node (dec1) [decision, below of=pro1, yshift=-0.5cm] {$[G_{\cap} = \varnothing]$};

          \node (pro2b) [process, right of=dec1, xshift=3cm] {Increase radius, $d$};

          \node (dec2) [decision, below of=dec1, yshift=-1cm] {$|\mathcal{L}(G_\cap)|$};

          \node (samp1) [process, left of=dec2, xshift=-3cm] {Enumerate $\sigma' \in \mathcal{L}(G_\cap)$};
          \node (samp2) [process, right of=dec2, xshift=3cm] {Sample $\sigma' \sim P(G_\cap)$};

          \node (rank) [process, below of=dec2, yshift=-2.5cm] {Rerank by $L_\theta(\sigma')$};
          \node (vlmc) [io, above of=rank, yshift=-0.1cm] {VLMC};

%  \node (out1) [io, below of=pro2a] {Output};
          \node (stop) [startstop, below of=rank] {Done};

%  \draw [arrow] (dec0) -- node[anchor=east] {no} (pro1);

          \draw [->,thick] (0, 1.3) -- (start);

          \draw [arrow] (start) -- (code);
          \draw [arrow] (start) -- (synt);
          \draw [arrow] (code) -- (lnfa);
          \draw [arrow] (synt) -- (pcfg);
          \draw [arrow] (lnfa) -- (pro1);
          \draw [arrow] (pcfg) -- (pro1);

%  \draw [arrow] (in1) -- (pro1);
          \draw [arrow] (pro1) -- (dec1);
          \draw [arrow] (dec1) -- node[anchor=south] {yes} (pro2b);
          \draw [arrow] (dec1) -- node[anchor=east] {no} (dec2);
          \draw [arrow] (pro2b) -- (lnfa);
          \draw [arrow] (dec2) -- node[anchor=south] {small} (samp1);
          \draw [arrow] (dec2) -- node[anchor=south] {large} (samp2);

          \draw [arrow] (vlmc) -- (rank);
          \draw [arrow] (samp1) |- (rank);
          \draw [arrow] (samp2) |- (rank);
%  \draw [arrow] (pro2a) -- (out1);
          \draw [arrow] (rank) -- (stop);
%          \draw [arrow] (dec2) -- node[anchor=east] {1} (stop);

        \end{tikzpicture}
      }
    \end{center}
    \vspace{-0.1cm}
    \caption{Flowchart of our proposed method.}\label{fig:flowchart}
    \vspace{-1cm}
  \end{wrapfigure}

  The syntax of most programming languages is context-free. Our proposed method is simple. We construct a context-free grammar representing the intersection between the language syntax and an automaton recognizing the Levenshtein ball of a given radius. Since CFLs are closed under intersection with regular languages, this is admissible. Three outcomes are possible:

  \begin{enumerate}
    \item $G_\cap$ is empty, in which case there is no repair within the given radius. In this case, we simply increase the radius and try again.
    \item $\mathcal{L}(G_\cap)$ is small, in which case we enumerate all possible repairs. Enumeration is tractable for $\sim 80\%$ of the dataset in $\leq 90$s.
    \item $\mathcal{L}(G_\cap)$ is too large to enumerate, so we sample from the intersection grammar $G_\cap$. Sampling is necessary for $\sim20\%$ of the dataset.
  \end{enumerate}

  As long as we have done our job correctly, the intersection language should contain every repair within a certain Levenshtein distance, and no invalid repairs. This procedure is depicted in Fig.~\ref{fig:flowchart}. In the following section, we will describe how we construct the intersecection grammar (\S~\ref{sec:lev_nfa}, \ref{sec:lev_bh}), then, provide an explicit technique for extracting all repairs contained within it (\S~\ref{sec:ptree}). Finally, we use an n-gram model to rank and return the top-k results by likelihood (\S~\ref{sec:ranking}).

  \subsection{Preliminaries}

  Recall that a CFG, $\mathcal{G} = \langle \Sigma, V, P, S\rangle$, is a quadruple consisting of terminals $(\Sigma)$, nonterminals $(V)$, productions $(P\colon V \rightarrow (V \mid \Sigma)^*)$, and a start symbol, $(S)$. Every CFG is reducible to \textit{Chomsky Normal Form}, $P'\colon V \rightarrow (V^2 \mid \Sigma)$, in which every production takes one of two forms, either $w \rightarrow xz$, or $w \rightarrow t$, where $w, x, z: V$ and $t: \Sigma$. For example:\vspace{-3pt}

  \begin{table}[H]
    \begin{tabular}{llll}
      $G\coloneqq\big\{\;S \rightarrow S\:S \mid (\:S\:) \mid (\:)\;\big\} \Longrightarrow \big\{\;S\rightarrow Q\:R \mid S\:S \mid L\:R,$ & $R \rightarrow\:),$ & $L \rightarrow (,$ & $Q\rightarrow L\:S\;\big\}$
    \end{tabular}
  \end{table}\vspace{-8pt}

  Likewise, a finite state automaton is a quintuple $\mathcal{A} = \langle Q, \Sigma, \delta, I, F\rangle$, where $Q$ is a finite set of states, $\Sigma$ is a finite alphabet, $\delta \subseteq Q \times \Sigma \times Q$ is the transition function, and $I, F \subseteq Q$ are the set of initial and final states, respectively. We will use this notation in the following sections.

  \pagebreak\subsection{The Nominal Levenshtein Automaton}\label{sec:lev_nfa}

  \begin{wrapfigure}{r}{0.5\textwidth}
    \vspace{-0.3cm}
    \begin{center}
      \begin{minipage}[c]{0.45\textwidth}
  \centering
  \underline{NIA}\vspace{10pt}
  \resizebox{\textwidth}{!}{
  \begin{tikzpicture}[
%->, % makes the edges directed
  >=stealth',
  node distance=2.5cm, % specifies the minimum distance between two nodes. Change if necessary.
%  every state/.style={thick, fill=gray!10}, % sets the properties for each ’state’ node
  initial text=$ $, % sets the text that appears on the start arrow
  ]
  \node[state, initial]                (00) {$q_{0,0}$};
  \node[state, right of=00]            (10) {$q_{1,0}$};
  \node[state, right of=10]            (20) {$q_{2,0}$};
  \node[state, right of=20]            (30) {$q_{3,0}$};
  \node[right of=30]                   (40) {$\vphantom{\vdots}\cdots$};
  \node[accepting, state, right of=40] (n0) {$q_{n,0}$};

  \node[state, above of=00]            (01) {$q_{0,1}$};
  \node[state, right of=01]            (11) {$q_{1,1}$};
  \node[state, right of=11]            (21) {$q_{2,1}$};
  \node[state, right of=21]            (31) {$q_{3,1}$};
  \node[right of=31]                   (41) {$\vphantom{\vdots}\cdots$};
  \node[accepting, state, right of=41] (n1) {$q_{n,1}$};

  \node[above of=01]                   (0j) {$\mathmakebox[\widthof{$\cdots$}]{\vdots}$};
\node[right of=0j]                   (1j) {$\mathmakebox[\widthof{$\cdots$}]{\vdots}$};
\node[right of=1j]                   (2j) {$\mathmakebox[\widthof{$\cdots$}]{\vdots}$};
\node[right of=2j]                   (3j) {$\mathmakebox[\widthof{$\cdots$}]{\vdots}$};
\node[right of=3j]                   (4j) {$\iddots$};
\node[accepting, right of=4j]        (nj) {$\mathmakebox[\widthof{$\cdots$}]{\vdots}$};

\node[state, above of=0j]            (0k) {$q_{0,k}$};
\node[state, right of=0k]            (1k) {$q_{1,k}$};
\node[state, right of=1k]            (2k) {$q_{2,k}$};
\node[state, right of=2k]            (3k) {$q_{3,k}$};
\node[right of=3k]                   (4k) {$\vphantom{\vdots}\cdots$};
\node[accepting, state, right of=4k] (nk) {$q_{n,k}$};

\draw [->] (00) edge[below] node{$\sigma_1$} (10);
\draw [->] (10) edge[below] node{$\sigma_2$} (20);
\draw [->] (20) edge[below] node{$\sigma_3$} (30);
\draw [->] (30) edge[below] node{$\sigma_4$} (40);
\draw [->] (40) edge[below] node{$\sigma_n$} (n0);

\draw [->] (01) edge[below] node{$\sigma_1$} (11);
\draw [->] (11) edge[below] node{$\sigma_2$} (21);
\draw [->] (21) edge[below] node{$\sigma_3$} (31);
\draw [->] (31) edge[below] node{$\sigma_4$} (41);
\draw [->] (41) edge[below] node{$\sigma_n$} (n1);

\draw [->] (0j) edge[below] node{$\sigma_1$} (1j);
\draw [->] (1j) edge[below] node{$\sigma_2$} (2j);
\draw [->] (2j) edge[below] node{$\sigma_3$} (3j);
\draw [->] (3j) edge[below] node{$\sigma_4$} (4j);
\draw [->] (4j) edge[below] node{$\sigma_n$} (nj);

\draw [->] (0k) edge[below] node{$\sigma_1$} (1k);
\draw [->] (1k) edge[below] node{$\sigma_2$} (2k);
\draw [->] (2k) edge[below] node{$\sigma_3$} (3k);
\draw [->] (3k) edge[below] node{$\sigma_4$} (4k);
\draw [->] (4k) edge[below] node{$\sigma_n$} (nk);

\draw [->] (00) edge[left] node{$*$}         (11);
\draw [->] (10) edge[left] node{$*$}         (21);
\draw [->] (20) edge[left] node{$*$}         (31);
\draw [->] (30) edge[left] node{$*$}         (41);
\draw [->] (30) edge[bend right, below] node{$\sigma_5$} (41);
\draw [->] (40) edge[           right] node{$\sigma_n$}  (n1);
\draw [->] (40) edge[bend right, left] node{$*$}         (n1);

\draw [->] (01) edge[left] node{$*$}                     (1j);
\draw [->] (11) edge[left] node{$*$}                     (2j);
\draw [->] (21) edge[left] node{$*$}                     (3j);
\draw [->] (31) edge[left] node{$*$}                     (4j);
\draw [->] (31) edge[bend right, below] node{$\sigma_5$} (4j);
\draw [->] (41) edge[           right] node{$\sigma_n$}  (nj);
\draw [->] (41) edge[bend right, left] node{$*$}         (nj);

\draw [->] (0j) edge[left] node{$*$}                     (1k);
\draw [->] (1j) edge[left] node{$*$}                     (2k);
\draw [->] (2j) edge[left] node{$*$}                     (3k);
\draw [->] (3j) edge[left] node{$*$}                     (4k);
\draw [->] (3j) edge[bend right, below] node{$\sigma_5$} (4k);
\draw [->] (4j) edge[           right] node{$\sigma_n$}  (nk);
\draw [->] (4j) edge[bend right, left] node{$*$}         (nk);

\draw [->] (00) edge[bend left, left] node{$*$}   (01);
\draw [->] (10) edge[bend left, left] node{$*$}   (11);
\draw [->] (20) edge[bend left, left] node{$*$}   (21);
\draw [->] (30) edge[bend left, left] node{$*$}   (31);
\draw [->] (40) edge[right] node{$*$}             (41);
\draw [->] (n0) edge[bend right, right] node{$*$} (n1);

\draw [->] (01) edge[bend left, left] node{$*$}   (0j);
\draw [->] (11) edge[bend left, left] node{$*$}   (1j);
\draw [->] (21) edge[bend left, left] node{$*$}   (2j);
\draw [->] (31) edge[bend left, left] node{$*$}   (3j);
\draw [->] (41) edge[right] node{$*$}             (4j);
\draw [->] (n1) edge[bend right, right] node{$*$} (nj);

\draw [->] (0j) edge[bend left, left] node{$*$}   (0k);
\draw [->] (1j) edge[bend left, left] node{$*$}   (1k);
\draw [->] (2j) edge[bend left, left] node{$*$}   (2k);
\draw [->] (3j) edge[bend left, left] node{$*$}   (3k);
\draw [->] (4j) edge[right] node{$*$}             (4k);
\draw [->] (nj) edge[bend right, right] node{$*$} (nk);

\draw [->] (00) edge[below] node{$\sigma_2$}    (21);
\draw [->] (10) edge[below] node{$\sigma_3$}    (31);
\draw [->] (20) edge[below] node{$\sigma_4$}    (41);

\draw [->] (01) edge[below] node{$\sigma_2$}    (2j);
\draw [->] (11) edge[below] node{$\sigma_3$}    (3j);
\draw [->] (21) edge[below] node{$\sigma_4$}    (4j);

\draw [->] (0j) edge[below] node{$\sigma_2$}    (2k);
\draw [->] (1j) edge[below] node{$\sigma_3$}    (3k);
\draw [->] (2j) edge[below] node{$\sigma_4$}    (4k);

%https://tex.stackexchange.com/a/20986/139648
\draw [decorate,decoration={brace,amplitude=10pt,raise=10pt,mirror}] (00.south west) -- (n0.south east) node[midway,yshift=-3em]{\textbf{String length}};
\draw [decorate,decoration={brace,amplitude=10pt,raise=20pt}] (00.south west) -- (0k.north west) node[midway,xshift=-40pt,rotate=90]{\textbf{Levenshtein edit distance}};
\end{tikzpicture}
}
\end{minipage}
\hfill
\begin{minipage}[l]{5 cm}
\centering
\underline{CFG}\vspace{7pt}
\begin{align*}
S &\Rightarrow \{\cdot \in Q \mid \delta(\cdot, q_{n,0}) \leq k\}\\
* &\Rightarrow \{\cdot \in \Sigma\}\\
\big\{q_{i, j} &\Rightarrow \{q_{i, j-1}*\} \mid i, j \in [1, n]\times[1, k]\big\}\\
\big\{q_{i, j} &\Rightarrow \{q_{i-1, j-1}*\}\mid i, j\in[1, n]\times [1, k]\big\}\\
\big\{q_{i, j} &\Rightarrow \{q_{i-1, j} \sigma_i \}\mid i, j \in [1, n]\times[0, k]\big\} \\
\big\{q_{i, j} &\Rightarrow \{q_{i-2, j-1} \sigma_i\} \mid i, j \in [2, n]\times[1, k] \big\}\\
\end{align*}
\end{minipage}
    \end{center}
    \caption{NFA recognizing Levenshtein $\Delta(\sigma: \Sigma^5, 3)$.}\label{fig:lev_nfa}
    \vspace{-0.5cm}
  \end{wrapfigure}

  Levenshtein edits are recognized by an automaton known as the Levenshtein automaton. As the original construction defined by Schultz and Mihov~\cite{schulz2002fast} contains cycles and $\varepsilon$-transitions, we propose a variant which is $\varepsilon$-free and acyclic. Furthermore, we adpot a nominal form which supports infinite alphabets and considerably simplifies the language intersection. Illustrated in Fig.~\ref{fig:lev_nfa} is an example of a small Levenshtein automaton recognizing $\Delta(\sigma: \Sigma^5, 3)$. Unlabeled arcs accept any terminal from the alphabet, $\Sigma$.

  \noindent Alternatively, this transition system can be viewed as a kind of proof system in an unlabeled lattice. This is equivalent to Schultz and Mihov's automaton, but more amenable to our purposes as it does not any contain $\varepsilon$-arcs, and uses skip connections to represent consecutive deletions.

  \begin{prooftree}
    \AxiomC{$s\in\Sigma \phantom{\land} i \in [0, n] \phantom{\land} j \in [1, k]$}
    \RightLabel{$\duparrow$}
    \UnaryInfC{$(q_{i, j-1} \overset{s}{\rightarrow} q_{i,j}) \in \delta$}
    \DisplayProof
    \hskip 1.5em
    \AxiomC{$s\in\Sigma \phantom{\land} i \in [1, n] \phantom{\land} j \in [1, k]$}
    \RightLabel{$\ddiagarrow$}
    \UnaryInfC{$(q_{i-1, j-1} \overset{s}{\rightarrow} q_{i,j}) \in \delta$}
  \end{prooftree}
  \begin{prooftree}
    \AxiomC{$i \in [1, n] \phantom{\land} j \in [0, k]$}
    \RightLabel{$\drightarrow$}
    \UnaryInfC{$(q_{i-1, j} \overset{\sigma_i}{\rightarrow} q_{i,j}) \in \delta$}
    \DisplayProof
    \hskip 1.5em
    \AxiomC{$d \in [1, d_{\max}] \phantom{\land} i \in [d + 1, n] \phantom{\land} j \in [d, k]$}
    \RightLabel{$\knightarrow$}
    \UnaryInfC{$(q_{i-d-1, j-d} \overset{\sigma_i}{\rightarrow} q_{i,j}) \in \delta$}
  \end{prooftree}
  \begin{prooftree}
    \AxiomC{$\vphantom{|}$}
    \RightLabel{$\textsc{Init}$}
    \UnaryInfC{$q_{0,0} \in I$}
    \DisplayProof
    \hskip 1.5em
    \AxiomC{$q_{i, j}$}
    \AxiomC{$|n-i+j| \leq k$}
    \RightLabel{$\textsc{Done}$}
    \BinaryInfC{$q_{i, j}\in F$}
  \end{prooftree}

  \newcommand{\substitutionExample}{
    \tikz{
      \foreach \x in {0,8,16,24,32,40}{
        \fill (\x pt,0pt) circle [radius = 1pt];
        \fill (\x pt,8pt) circle [radius = 1pt];
      }
      \phantom{\fill (0pt,-8pt) circle [radius = 1pt];}
      \draw [-to] (0pt,0pt) -- (8pt,0pt);
      \draw [-to] (8pt,0pt) -- (16pt,0pt);
      \draw [-to] (16pt,0pt) -- (24pt,8pt);
      \draw [-to] (24pt,8pt) -- (32pt,8pt);
      \draw [-to] (32pt,8pt) -- (40pt,8pt);
    }
  }

  \newcommand{\insertionExample}{
    \tikz{
      \foreach \x in {0,8,16,24,32,40}{
        \fill (\x pt,0pt) circle [radius = 1pt];
        \fill (\x pt,8pt) circle [radius = 1pt];
      }
      \phantom{\fill (0pt,-8pt) circle [radius = 1pt];}
      \fill[white] (16pt,0pt) circle [radius = 1.2pt];
      \fill[white] (24pt,8pt) circle [radius = 1.2pt];
      \draw [-to] (0pt,0pt) -- (8pt,0pt);
      \draw [-to] (8pt,0pt) -- (24pt,0pt);
      \draw [-to] (24pt,0pt) -- (16pt,8pt);
      \draw [-to] (16pt,8pt) -- (32pt,8pt);
      \draw [-to] (32pt,8pt) -- (40pt,8pt);
    }
  }

  \newcommand{\deletionExample}{
    \tikz{
      \foreach \x in {0,8,16,24,32,40}{
        \fill (\x pt,0pt) circle [radius = 1pt];
        \fill (\x pt,8pt) circle [radius = 1pt];
      }
      \phantom{\fill (0pt,-8pt) circle [radius = 1pt];}
      \draw [-to] (0pt,0pt) -- (8pt,0pt);
      \draw [-to] (8pt,0pt) -- (16pt,0pt);
      \draw [-to] (16pt,0pt) -- (24pt,0pt);
      \draw [-to] (24pt,0pt) -- (40pt,8pt);
    }
  }

  \newcommand{\doubleDeletionExample}{
    \tikz{
      \foreach \x in {0,8,16,24,32,40}{
        \fill (\x pt,0pt) circle [radius = 1pt];
        \fill (\x pt,8pt) circle [radius = 1pt];
        \fill (\x pt,16pt) circle [radius = 1pt];
      }
      \draw [-to] (0pt,0pt) -- (24pt,16pt);
      \draw [-to] (24pt,16pt) -- (32pt,16pt);
      \draw [-to] (32pt,16pt) -- (40pt,16pt);
    }
  }

  \newcommand{\subDelExample}{
    \tikz{
      \foreach \x in {0,8,16,24,32,40}{
        \fill (\x pt,0pt) circle [radius = 1pt];
        \fill (\x pt,8pt) circle [radius = 1pt];
        \fill (\x pt,16pt) circle [radius = 1pt];
      }
      \draw [-to] (0pt,0pt) -- (8pt,0pt);
      \draw [-to] (8pt,0pt) -- (16pt,8pt);
      \draw [-to] (16pt,8pt) -- (32pt,16pt);
      \draw [-to] (32pt,16pt) -- (40pt,16pt);
    }
  }

  \newcommand{\subSubExample}{
    \tikz{
      \foreach \x in {0,8,16,24,32,40}{
        \fill (\x pt,0pt) circle [radius = 1pt];
        \fill (\x pt,8pt) circle [radius = 1pt];
        \fill (\x pt,16pt) circle [radius = 1pt];
      }
      \draw [-to] (0pt,0pt) -- (8pt,0pt);
      \draw [-to] (8pt,0pt) -- (16pt,8pt);
      \draw [-to] (16pt,8pt) -- (24pt,16pt);
      \draw [-to] (24pt,16pt) -- (32pt,16pt);
      \draw [-to] (32pt,16pt) -- (40pt,16pt);
    }
  }

  \newcommand{\insertDeleteExample}{
    \tikz{
      \foreach \x in {0,8,16,24,32,40,48}{
        \fill (\x pt,0pt) circle [radius = 1pt];
        \fill (\x pt,8pt) circle [radius = 1pt];
        \fill (\x pt,16pt) circle [radius = 1pt];
      }
      \fill[white] (16pt,16pt) circle [radius = 1.2pt];
      \fill[white] (8pt,0pt) circle [radius = 1.2pt];
      \fill[white] (16pt,8pt) circle [radius = 1.2pt];
      \draw [-to] (0pt,0pt) -- (16pt,0pt);
      \draw [-to] (16pt,0pt) -- (8pt,8pt);
      \draw [-to] (8pt,8pt) -- (24pt,8pt);
      \draw [-to] (24pt,8pt) -- (40pt,16pt);
      \draw [-to] (40pt,16pt) -- (48pt,16pt);
    }
  }

  Each arc plays a specific role. $\duparrow$ handles insertions, $\ddiagarrow$ handles substitutions, $\duparrow$ handles insertions and $\knightarrow$ handles [consecutive] deletions of various lengths. Let us consider some illustrative cases.

  \begin{table}[h!]
    \begin{tabular}{ccccccc}

      \texttt{f\hspace{3pt}.\hspace{3pt}\hlorange{[}\hspace{3pt}x\hspace{3pt})} &
      \texttt{f\hspace{3pt}.\hspace{3pt}\phantom{(}\hspace{3pt}x\hspace{3pt})} &
      \texttt{f\hspace{3pt}.\hspace{3pt}(\hspace{3pt}\hlred{x}\hspace{3pt})} &
      \texttt{\hlred{.}\hspace{3pt}\hlred{+}\hspace{3pt}(\hspace{3pt}x\hspace{3pt})} &
      \texttt{f\hspace{3pt}\hlorange{.}\hspace{3pt}\hlred{(}\hspace{3pt}x\hspace{3pt};} &
      \texttt{[\hspace{3pt}\hlorange{,}\hspace{3pt}\hlorange{x}\hspace{3pt}y\hspace{3pt}]} &
      \texttt{[\hspace{3pt}\phantom{,}\hspace{3pt},\hspace{3pt}\hlred{x}\hspace{3pt}y\hspace{3pt}]} \\

      \texttt{f\hspace{3pt}.\hspace{3pt}\hlorange{(}\hspace{3pt}x\hspace{3pt})} &
      \texttt{f\hspace{3pt}.\hspace{3pt}\hlgreen{(}\hspace{3pt}x\hspace{3pt})} &
      \texttt{f\hspace{3pt}.\hspace{3pt}(\hspace{3pt}\phantom{x}\hspace{3pt})} &
      \texttt{\phantom{f}\hspace{3pt}\phantom{.}\hspace{3pt}(\hspace{3pt}x\hspace{3pt})} &
      \texttt{f\hspace{3pt}\hlorange{*}\hspace{3pt}\phantom{(}\hspace{3pt}x\hspace{3pt};} &
      \texttt{[\hspace{3pt}\hlorange{x}\hspace{3pt}\hlorange{,}\hspace{3pt}y\hspace{3pt}]} &
      \texttt{[\hspace{3pt}\hlgreen{x}\hspace{3pt},\hspace{3pt}\phantom{x}\hspace{3pt}y\hspace{3pt}]} \\

      \substitutionExample & \insertionExample & \deletionExample & \doubleDeletionExample & \subDelExample & \subSubExample & \insertDeleteExample
    \end{tabular}
  \end{table}

  Note that the same patch can have multiple Levenshtein alignments. $\textsc{Done}$ constructs the final states, which are all states accepting strings $\sigma'$ such that Levenshtein distance of $\Delta(\sigma, \sigma') \leq d_\max$.

  To avoid creating a parallel bundle of arcs for each insertion and substutition point, we instead decorate each arc with a nominal predicate, accepting or rejecting $\sigma_i$. To distinguish this nominal variant from the original construction, we highlight the modified rules in orange below.

  \begin{prooftree}
    \AxiomC{$i \in [0, n] \phantom{\land} j \in [1, k]$}
    \RightLabel{$\duparrow$}
    \UnaryInfC{$(q_{i, j-1} \overset{{\color{orange}[\neq \sigma_i]}}{\rightarrow} q_{i,j}) \in \delta$}
    \DisplayProof
    \hskip 1.5em
    \AxiomC{$i \in [1, n] \phantom{\land} j \in [1, k]$}
    \RightLabel{$\ddiagarrow$}
    \UnaryInfC{$(q_{i-1, j-1} \overset{{\color{orange}[\neq \sigma_i]}}{\rightarrow} q_{i,j}) \in \delta$}
  \end{prooftree}
  \begin{prooftree}
    \AxiomC{$i \in [1, n] \phantom{\land} j \in [0, k]$}
    \RightLabel{$\drightarrow$}
    \UnaryInfC{$(q_{i-1, j} \overset{{\color{orange}[=\sigma_i]}}{\rightarrow} q_{i,j}) \in \delta$}
    \DisplayProof
    \hskip 1.5em
    \AxiomC{$d \in [1, d_{\max}] \phantom{\land} i \in [d + 1, n] \phantom{\land} j \in [d, k]$}
    \RightLabel{$\knightarrow$}
    \UnaryInfC{$(q_{i-d-1, j-d} \overset{{\color{orange}[=\sigma_i]}}{\rightarrow} q_{i,j}) \in \delta$}
  \end{prooftree}

  Nominalizing the NFA eliminates the creation of $e=|\Sigma - 1|\cdot|\sigma|\cdot2d_\max$ unnecessary arcs over the entire Levenshtein automaton and drastically reduces the size of the construction to follow, but does not affect the underlying semantics. Thus, it is essential to first nominalize the automaton before proceeding to avoid a large blowup in the intermediate grammar.

  \subsection{Levenshtein-Bar-Hillel Construction}\label{sec:lev_bh}

  We now describe the Bar-Hillel construction, which generates a grammar recognizing the intersection between a regular and a context-free language, then specialize it to Levenshtein intersections.

  \begin{lemma}\label{lemma:bar-hillel}
  For any context-free language $\ell$ and finite state automaton $\alpha$, there exists a context-free grammar $G_\cap$ such that $\mathcal{L}(G_\cap) = \ell \cap \mathcal{L}(\alpha)$. See Bar-Hillel~\cite{bar1961formal}.
  \end{lemma}

  \noindent Although Bar-Hillel~\cite{bar1961formal} lacks an explicit construction, Beigel and Gasarch~\cite{beigelproof} construct $G_\cap$ like so:

  \begin{prooftree}
    \AxiomC{$q \in I \phantom{\land} r \in F\vphantom{\overset{a}{\rightarrow}}$}
%  \RightLabel{$\varepsilon^+\textsc{-int}$}
    \UnaryInfC{$\big(S\rightarrow q S r\big) \in P_\cap$}
    \DisplayProof
    \hskip 1em
    \AxiomC{$(A \rightarrow a) \in P$}
    \AxiomC{$(q\overset{a}{\rightarrow}r) \in \delta$}
%  \RightLabel{$\varepsilon^+\textsc{-int}$}
    \BinaryInfC{$\big(qAr\rightarrow a\big)\in P_\cap$}
    \DisplayProof
    \hskip 1em
%\end{prooftree}
%\begin{prooftree}
    \AxiomC{$(w \rightarrow xz) \in P\vphantom{\overset{a}{\rightarrow}}$}
    \AxiomC{$p,q,r \in Q$}
    \RightLabel{\Join}
    \BinaryInfC{$\big(pwr\rightarrow (pxq)(qzr)\big) \in P_\cap$}
  \end{prooftree}

  This, now standard, Bar-Hillel construction applies to any CFL and REG language intersection, but generates a grammar whose cardinality is approximately $|P_\cap|=|I|\cdot|F| + |P|\cdot|\Sigma|\cdot|\sigma|\cdot2d_{\max} + |P|\cdot|Q|^3$. Applying the BH construction directly to practical languages and code snippets can generate hundreds of trillions of productions for even moderately sized grammars and Levenshtein automata. Instead, we will describe a kind of reachability analysis that elides many unreachable productions in the case of Levenshtein intersection.

  Consider $\Join$, the most expensive rule. What $\Join$ tells us is each nonterminal in the intersection grammar, $P_\cap$, matches a substring simultaneously recognized by (1) a pair of states in the original NFA and (2) a nonterminal in the original CFG. A key observation is that $\Join$ considers the intersection between every such triple, but this is a gross overapproximation for most NFAs and CFGs, as the vast majority of all state pairs and nonterminals recognize no strings in common.

  To identify these triples, we define an interval domain that soundly overapproximates the Parikh image, encoding the minimum and maximum number of terminals each nonterminal can generate. Since some intervals may be right-unbounded, we write $\mathbb{N}^*=\mathbb{N} \cup \{\infty\}$ to denote the upper bound, and $\Pi = \{[a, b] \in \mathbb{N} \times \mathbb{N}^* \mid a \leq b\}^{|\Sigma|}$ to denote the Parikh map of all terminals.

  \begin{definition}[Parikh mapping of a nonterminal]
    Let $p: \Sigma^*\rightarrow\mathbb{N}^{|\Sigma|}$ be the Parikh operator~\cite{parikh1966context}, which counts the frequency of terminals in a string. Let $\pi: V \rightarrow \Pi$ be a function returning the smallest interval such that $\forall \sigma: \Sigma^*, \forall v: V$, $v \Rightarrow^* \sigma \vdash p(s) \in \pi(v)$.
  \end{definition}

  In other words, the Parikh mapping computes the greatest lower and least upper bound of the Parikh image over all strings in the language of a nonterminal. The infimum of a nonterminal's Parikh interval tells us how many of each terminal a nonterminal \textit{must} generate, and the supremum tells us how many it \textit{can} generate. Likewise, we define a similar relation over NFA state pairs:

  \begin{definition}[Parikh mapping of NFA states]
    We define $\pi: Q\times Q \rightarrow \Pi$ as returning the smallest interval such that $\forall \sigma: \Sigma^*, \forall q, q': Q$, $q \overset{\sigma}{\Longrightarrow} q' \vdash p(\sigma) \in \pi(q, q')$.
  \end{definition}

  Next, we will define a measure on Parikh intervals, which will represent the minimum total edits required to transform a string in one Parikh interval to a string in another.

  \begin{definition}[Parikh divergence]
    Given two Parikh intervals $\pi, \pi': \Pi$, we define the divergence between them as $\pi \parallel \pi' = \sum_{n=1}^{|\Sigma|} \min_{(i, i') \in \pi[n]\times \pi'[n]} |i - i'|$.
  \end{definition}

  Now, we know that if the Parikh divergence between two intervals exceeds the Levenshtein margin between two states in a Lev-NFA, those intervals must be incompatible as no two strings, one from each Parikh interval, can be transformed into the other with fewer than $\pi \parallel \pi'$ edits.

  \begin{definition}[Levenshtein-Parikh compatibility]
    Let $q = q_{h,i}, q'=q_{j,k}$ be two states in a Lev-NFA and V be a CFG nonterminal. We say that $(q, v, q'): Q\times V\times Q$ are compatible iff the Parikh divergence is bounded by the Levenshtein margin $k-i$, i.e., $v \lhd qq' \iff (\pi(v) \parallel \pi(q, q')) \leq k-i$.
  \end{definition}

  We define the modified Bar-Hillel construction for Levenshtein intersections as follows:

\begin{prooftree}
  \def\defaultHypSeparation{\hskip 0.15cm}
  \AxiomC{$(A \rightarrow a) \in P$}
  \AxiomC{$\color{orange}S.a$}
  \AxiomC{$(q\overset{{\color{orange}S}}{\rightarrow}r) \in \delta$}
%  \RightLabel{$\varepsilon^+\textsc{-int}$}
  \TrinaryInfC{$\big(qAr\rightarrow a\big)\in P_\cap$}
  \DisplayProof
  \AxiomC{$\vphantom{\overset{S}{\rightarrow}}\color{orange} w \lhd pr \phantom{\land} x \lhd pq \phantom{\land} z \lhd qr$}
  \AxiomC{$(w \rightarrow xz) \in P\vphantom{\overset{a}{\rightarrow}}$}
  \AxiomC{$p,q,r \in Q$}
  \RightLabel{\Join}
  \TrinaryInfC{$\big(pwr\rightarrow (pxq)(qzr)\big) \in P_\cap$}
\end{prooftree}

  Finally, once $G_\cap$ is constructed, it is renormalized by removing all unreachable and non-generating productions following~\cite{firsov2015certified} to obtain $G_\cap^*$, which is usually several orders of magnitude smaller. When this grammar is sufficiently small, we can extract all relevant repairs from it, otherwise we sample from it to obtain a subset of candidate repairs.

%Specifically, we compute Parikh intervals generated by every path though the Levenshtein automaton, then intersect the Parikh intervals for the candidate nonterminals in question. For example, suppose we have a $p, q, r: Q$ and $w \rightarrow x z$. Let's check...

%To generate edits from it, we can use the same procedure as before, but instead of interleaving $\err\sigma$ with $\varepsilon$ and introducing holes, we simply use $A\big((\_)^{|\err{\sigma}| + d}\big, G_\cap)$.


  \subsection{Repair as idempotent matrix completion}\label{sec:matrix_completion}

  Now that we have a language that recognizes local repairs, we need a method to extract the repairs themselves. We impose specific criteria on such a procedure: it must generate (1) only valid repairs and (2) all repairs in the language. We also require the sampler to have (3) parallel decomposability and generate samples (4) uniformly, both (5) with and (6) without replacement in (7) constant time and space. These requirements motivate the need for a specialized generating function.

  In this section, we will introduce the porous completion problem and show how it can be translated to a kind of idempotent matrix completion, whose roots are valid strings in a context-free language. This technique is convenient for its geometric interpretability, amenability to parallelization, and generalizability to any CFG, regardless of finitude or ambiguity. We will see how, by redefining the algebraic operations $\oplus, \otimes$ over different carrier sets, one can obtain a recognizer, parser, generator, Parikh map and other convenient structures for working with CFLs.

  Given a CFG, $G' : \mathcal{G}$ in Chomsky Normal Form (CNF), we can construct a recognizer $R: \mathcal{G} \rightarrow \Sigma^n \rightarrow \mathbb{B}$ for strings $\sigma: \Sigma^n$ as follows. Let $2^V$ be our domain, $0$ be $\varnothing$, $\oplus$ be $\cup$, and $\otimes$ be defined as:\vspace{-10pt}

  \begin{align}
    X \otimes Z \coloneqq \big\{\;w \mid \langle x, z\rangle \in X \times Z, (w\rightarrow xz) \in P\;\big\}
  \end{align}

  \noindent If we define $\hat\sigma_r \coloneqq \{w \mid (w \rightarrow \sigma_r) \in P\}$, then construct a matrix with nonterminals on the superdiagonal representing each token, $M_0[r+1=c](G', \sigma) \coloneqq \;\hat\sigma_r$, the fixpoint $M_{i+1} = M_i + M_i^2$ is uniquely determined by the superdiagonal entries, which are computed as follows:\vspace{-10pt}

  \begin{align*}
    M_0\coloneqq
    \begin{pNiceMatrix}[nullify-dots,xdots/line-style=loosely dotted]
      \varnothing & \hat\sigma_1 & \varnothing & \Cdots & \varnothing \\
      \Vdots      & \Ddots       & \Ddots      & \Ddots & \Vdots\\
                  &              &             &        & \varnothing\\
                  &              &             &        & \hat\sigma_n \\
      \varnothing & \Cdots       &             &        & \varnothing
    \end{pNiceMatrix} &\Rightarrow
    \begin{pNiceMatrix}[nullify-dots,xdots/line-style=loosely dotted]
      \varnothing & \hat\sigma_1 & \Lambda & \Cdots & \varnothing \\
      \Vdots      & \Ddots       & \Ddots  & \Ddots & \Vdots\\
                  &              &         &        & \Lambda\\
                  &              &         &        & \hat\sigma_n \\
      \varnothing & \Cdots       &         &        & \varnothing
    \end{pNiceMatrix} &\Rightarrow \ldots \Rightarrow M_\infty =
    \begin{pNiceMatrix}[nullify-dots,xdots/line-style=loosely dotted]
      \varnothing & \hat\sigma_1 & \Lambda & \Cdots & \Lambda^*_\sigma\\
      \Vdots      & \Ddots       & \Ddots  & \Ddots & \Vdots\\
                  &              &         &        & \Lambda\\
                  &              &         &        & \hat\sigma_n \\
      \varnothing & \Cdots       &         &        & \varnothing
    \end{pNiceMatrix}
  \end{align*}

  \noindent The northeasternmost entry $\Lambda^*_\sigma$ gives us the recognizer $R(G', \sigma) \coloneqq [S \in \Lambda^*_\sigma] \Leftrightarrow [\sigma \in \mathcal{L}(G)]$~\footnote{Hereinafter, we use Iverson brackets to denote the indicator function of a predicate with free variables, i.e., $[P] \Leftrightarrow \mathds{1}(P)$.}.

  This procedure can be lifted to the domain of strings containing free variables, which we call the \textit{porous completion problem}, whose solutions are the set of all syntactically valid strings consistent with the input string.

  \begin{definition}[Porous completion]
    Let $\underline\Sigma \coloneqq \Sigma \cup \{\_\}$, where $\_$ denotes a hole. We denote $\sqsubseteq: \Sigma^n \times \underline\Sigma^n$ as the relation $\{\langle\sigma', \sigma\rangle \mid \sigma_i \in \Sigma \implies \sigma_i' = \sigma_i\}$ and the set of all inhabitants $\{\sigma': \Sigma^+ \mid \sigma' \sqsubseteq \sigma\}$ as $\text{H}(\sigma)$. Given a \textit{porous string}, $\sigma: \underline\Sigma^*$ we seek all syntactically valid inhabitants, i.e., $A(\sigma)\coloneqq\text{H}(\sigma)\cap\ell$.
  \end{definition}

  Let us consider an example with two holes, $\sigma = 1$ \_ \_, and the grammar being $G\coloneqq\{S\rightarrow N O N, O \rightarrow + \mid \times, N \rightarrow 0 \mid 1\}$. This can be rewritten into CNF as $G'\coloneqq \{S \rightarrow N L, N \rightarrow 0 \mid 1, O \rightarrow \times \mid +, L \rightarrow O N\}$. Using the algebra where $\oplus=\cup$, $X \otimes Z = \big\{\;w \mid \langle x, z\rangle \in X \times Z, (w\rightarrow xz) \in P\;\big\}$, the fixpoint $M' = M + M^2$ can be computed as follows, shown in the leftmost column:\\

  \begin{small}
  {\renewcommand{\arraystretch}{1.2}
  \noindent\phantom{...}\begin{tabular}{|c|c|c|c|}
                          \hline
                          & $2^V$ & $\mathbb{Z}_2^{|V|}$ & $\mathbb{Z}_2^{|V|}\rightarrow\mathbb{Z}_2^{|V|}$\\\hline
                          $M_0$ & \begin{pmatrix}
                                    \phantom{V} & \tiny{\{N\}} &         &             \\
                                    &              & \{N,O\} &             \\
                                    &              &         & \{N,O\} \\
                                    &              &         &
                          \end{pmatrix} & \begin{pmatrix}
                                            \phantom{V} & \ws\bs\ws\ws &              &              \\
                                            &              & \ws\bs\bs\ws &              \\
                                            &              &              & \ws\bs\bs\ws \\
                                            &              &              &
                          \end{pmatrix} & \begin{pmatrix}
                                            \phantom{V} & V_{0, 1} &          &          \\
                                            &          & V_{1, 2} &          \\
                                            &          &          & V_{2, 3} \\
                                            &          &          &
                          \end{pmatrix} \\\hline
                          $M_1$ & \begin{pmatrix}
                                    \phantom{V} & \tiny{\{N\}} & \varnothing &         \\
                                    &              & \{N,O\}     & \{L\}   \\
                                    &              &             & \{N,O\} \\
                                    &              &             &
                          \end{pmatrix} & \begin{pmatrix}
                                            \phantom{V} & \ws\bs\ws\ws & \ws\ws\ws\ws &              \\
                                            &              & \ws\bs\bs\ws & \bs\ws\ws\ws \\
                                            &              &              & \ws\bs\bs\ws \\
                                            &              &              &
                          \end{pmatrix} & \begin{pmatrix}
                                            \phantom{V} & V_{0, 1} & V_{0, 2} &          \\
                                            &          & V_{1, 2} & V_{1, 3} \\
                                            &          &          & V_{2, 3} \\
                                            &          &          &
                          \end{pmatrix} \\\hline
                          $M_\infty$ & \begin{pmatrix}
                                         \phantom{V} & \tiny{\{N\}} & \varnothing & \{S\}   \\
                                         &              & \{N,O\}     & \{L\}   \\
                                         &              &             & \{N,O\} \\
                                         &              &             &
                          \end{pmatrix} & \begin{pmatrix}
                                            \phantom{V} & \ws\bs\ws\ws & \ws\ws\ws\ws & \ws\ws\ws\bs \\
                                            &              & \ws\bs\bs\ws & \bs\ws\ws\ws \\
                                            &              &              & \ws\bs\bs\ws \\
                                            &              &              &
                          \end{pmatrix} & \begin{pmatrix}
                                            \phantom{V} & V_{0, 1} & V_{0, 2} & V_{0, 3} \\
                                            &          & V_{1, 2} & V_{1, 3} \\
                                            &          &          & V_{2, 3} \\
                                            &          &          &
                          \end{pmatrix}\\\hline
  \end{tabular}\\
  }
  \end{small}

  The same procedure can be translated, without loss of generality, into the bit domain ($\mathbb{Z}_2^{|V|}$) using a lexicographic ordering, however $M_\infty$ in both $2^V$ and $\mathbb{Z}_2^{|V|}$ represents a decision procedure, i.e., $[S\in V_{0, 3}]\Leftrightarrow [V_{0, 3, 3}=\bs] \Leftrightarrow [A(\sigma) \neq \varnothing]$. Since $V_{0, 3} = \{S\}$, we know there exists at least one $\sigma' \in A$, but $M_\infty$ does not reveal its identity.

%$\{\text{xor}, \land, \top\}$ is a functionally complete set is equivalent to $\mathbb{Z}_2$ $\top := 1, \land := \times, \text{xor} := +$. We can define $=$ as $(a = b) \Leftrightarrow (a \text{ xor } b) \text{ xor } \top \Leftrightarrow (a + b) + \top$.

  In order to extract the inhabitants, we can translate the bitwise procedure into an equation with free variables. Here, we can encode the idempotency constraint directly as $M = M^2$. We first define $X \boxtimes Z = [X_2 \land Z_1, \bot, \bot, X_1 \land Z_0]$ and $X \boxplus Z = [X_i \lor Z_i]_{i \in [0, |V|)}$. Since the unit nonterminals $O, N$ can only occur on the superdiagonal, they may be safely ignored by $\boxtimes$. To solve for $M_\infty$, we proceed by first computing $V_{0, 2}, V_{1, 3}$ as follows:

  \begin{small}
  \begin{align*}
    V_{0, 2} &= V_{0, j} \cdot V_{j, 2} = V_{0, 1} \boxtimes V_{1, 2}                         &  V_{1, 3} &= V_{1, j} \cdot V_{j, 3} = V_{1, 2} \boxtimes V_{2, 3}\\
    &= [L \in V_{0, 2}, \bot, \bot, S \in V_{0, 2}]                                           &  &= [L \in V_{1, 3}, \bot, \bot, S \in V_{1, 3}]\\
    &= [O \in V_{0, 1} \land N \in V_{1, 2}, \bot, \bot, N \in V_{0, 1} \land L \in V_{1, 2}] &  &= [O \in V_{1, 2} \land N \in V_{2, 3}, \bot, \bot, N \in V_{1, 2} \land L \in V_{2, 3}]\\
    &= [V_{0, 1, 2} \land V_{1, 2, 1}, \bot, \bot, V_{0, 1, 1} \land V_{1, 2, 0}]             &  &= [V_{1, 2, 2} \land V_{2, 3, 1}, \bot, \bot, V_{1, 2, 1} \land V_{2, 3, 0}]
  \end{align*}
  \end{small}

  Now we solve for the corner entry $V_{0, 3}$ by taking the bitwise dot product between the first row and last column, yielding:

  \begin{align*}
    V_{0, 3} &= V_{0, j} \cdot V_{j, 3} = V_{0, 1} \boxtimes V_{1, 3} \boxplus V_{0, 2} \boxtimes V_{2, 3}\\
%  &= [V_{0, 1, 2} \land V_{1, 3, 1}, \bot, \bot, V_{0, 1, 1} \land V_{1, 3, 0}] + [V_{0, 2, 2} \land V_{2, 3, 1}, \bot, \bot, V_{0, 2, 1} \land V_{2, 3, 0}]\\
    &= [V_{0, 1, 2} \land V_{1, 3, 1} \lor V_{0, 2, 2} \land V_{2, 3, 1}, \bot, \bot, V_{0, 1, 1} \land V_{1, 3, 0} \lor V_{0, 2, 1} \land V_{2, 3, 0}]
  \end{align*}

  \noindent Since we only care about $V_{0, 3, 3} \Leftrightarrow [S \in V_{0, 3}]$, so we can ignore the first three entries and solve for:

  \begin{align*}
    V_{0, 3, 3} &= V_{0, 1, 1} \land V_{1, 3, 0} \lor V_{0, 2, 1} \land V_{2, 3, 0}\\
    &= V_{0, 1, 1} \land (V_{1, 2, 2} \land V_{2, 3, 1}) \lor V_{0, 2, 1} \land \bot\\
    &= V_{0, 1, 1} \land V_{1, 2, 2} \land V_{2, 3, 1}\\
    &= [N \in V_{0, 1}] \land [O \in V_{1, 2}] \land [N \in V_{2, 3}]
  \end{align*}

  Now we know that $\sigma =$ 1 \underline{O} \underline{N} is a valid solution, and we can take the product $\{1\}\times \hat\sigma_2^{-1}(O) \times \hat\sigma_3^{-1}(N)$ to recover the admissible set, yielding $A=\{1+0, 1+1, 1\times 0, 1\times 1\}$. In this case, since $G$ is unambiguous, there is only one parse tree satisfying $V_{0, |\sigma|, 3}$, but in general, there can be multiple valid parse trees.

  \subsection{An algebraic datatype for context-free parse forests}\label{sec:ptree}

  This procedure generates solutions satisfying the matrix fixpoint, but forgets their provenance. The question naturally arises, is there a way to solve for the parse trees directly? This would allow us to handle ambiguous grammars, whilst preserving the natural structure of the parsing domain.

  \begin{wrapfigure}{r}{0.47\textwidth}
    \vspace{-5pt}
    \resizebox{0.47\textwidth}{!}{
      \begin{tikzpicture}
      [
        grow                    = right,
        sibling distance        = 3em,
        level distance          = 5em,
        edge from parent/.style = {draw, -latex},
        every node/.style       = {font=\footnotesize},
        sloped,
        treenode/.style = {shape=rectangle, rounded corners,
        draw, align=center,
        top color=white, bottom color=blue!20},
        root/.style     = {treenode, font=\tiny, bottom color=red!30},
        env/.style      = {treenode, font=\tiny},
        dummy/.style    = {circle,draw}
      ]
        \node [root] {S}
        child { node [env] {BC}
        child { node [root] {B}
        child { node [env] {RD}
        child { node [root] {R} edge from parent node [above] }
        child { node [root] {D} edge from parent node [above] }
        edge from parent node [above] }
        edge from parent node [below] }
        child { node [root] {C}
        child { node [env] {\ldots\vphantom{BB}} edge from parent node [below] }
%  child { edge from parent node [above] {\ldots} }
        edge from parent node [below] }
        edge from parent node [above] }
        child { node [env] {\ldots\vphantom{BB}} edge from parent node [below] }
        child { node [env] {AB}
        child { node [root] {A}
        child {
          node [env] {QC}
          child { node [root] {Q} edge from parent node [above] }
          child { node [root] {C} edge from parent node [above] }
          edge from parent node [above]
        }
%    child { node [env] {ZQ} edge from parent node [above] }
        child { node [env] {\ldots\vphantom{BB}} edge from parent node [below] }
        edge from parent node [below] }
        child { node [root] {B}
        child { node [env] {RD}
        child { node [root] {R} edge from parent node [above] }
        child { node [root] {D} edge from parent node [above] }
        edge from parent node [above] }
        edge from parent node [below] }
        edge from parent node [above] };
      \end{tikzpicture}
    }
    \caption{A partial $\mathbb{T}_2$ corresponding to the grammar $\{S \rightarrow BC \mid \ldots \mid AB, B\rightarrow RD \mid \ldots, A\rightarrow QC \mid \ldots\}$.}
    \label{fig:ptree}
    \vspace{-10pt}
  \end{wrapfigure}

  We will now describe a datatype for compactly representing CFL parse forests, then redefine the semiring algebra over this domain. This construction is especially convenient for tracking provenance under ambiguity, Parikh mapping, counting the size of a finite CFL, and sampling trees either with replacement using a PCFG, or without replacement using an integer pairing function.

  We first define a datatype $\mathbb{T}_3 = (V \cup \Sigma) \rightharpoonup \mathbb{T}_2$ where $\mathbb{T}_2 = (V \cup \Sigma) \times (\mathbb{N} \rightharpoonup \mathbb{T}_2\times\mathbb{T}_2)$\footnote{Given a $T:\mathbb{T}_2$, we may also refer to $\pi_1(T), \pi_2(T)$ as $\texttt{root}(T)$ and $\texttt{children}(T)$ respectively.}. Morally, we can think of $\mathbb{T}_2$ as an implicit set of possible trees that can be generated by a CFG in CNF, consistent with a finite-length porous string. Structurally, we may interpret $\mathbb{T}_2$ as an algebraic data type corresponding to the fixpoints of the following recurrence. This tells us each $\mathbb{T}_2$ can be a terminal, nonterminal, or a nonterminal and a sequence of nonterminal pairs and their two children:\vspace{-10pt}

  \begin{equation}
    L(p) = 1 + p L(p) \phantom{addspace} P(a) = \Sigma + V + V L\big(V^2P(a)^2\big)
  \end{equation}

  Given a $\sigma: \underline\Sigma$, we construct $\mathbb{T}_2$ from the bottom-up, and sample from the top-down. Depicted in Fig.~\ref{fig:ptree} is a partial $\mathbb{T}_2$, where red nodes are \texttt{root}s and blue nodes are \texttt{children}.

  We construct the first upper diagonal $\hat\sigma_r = \Lambda(\sigma_r)$ as follows:

\vspace{-10pt}\begin{equation}
  \begin{footnotesize}
\Lambda(s: \underline\Sigma) \mapsto \begin{cases}
\bigoplus_{s'\in \Sigma} \Lambda(s') & \text{if $s$ is a hole,} \vspace{5pt}\\
\big\{\mathbb{T}_2\big(w, \big[\langle\mathbb{T}_2(s), \mathbb{T}_2(\varepsilon)\rangle\big]\big) \mid (w \rightarrow s)\in P\big\} & \text{otherwise.}
\end{cases}
  \end{footnotesize}
\end{equation}

\noindent This initializes the superdiagonal entries, enabling us to compute the fixpoint $M_\infty$ in the same manner described in \S~\ref{sec:matrix_completion} by redefining $\oplus, \otimes: \mathbb{T}_3 \times \mathbb{T}_3 \rightarrow \mathbb{T}_3$ as:

\begin{align}
  X \oplus Z &\mapsto \bigcup_{\mathclap{k\in \pi_1(X \cup Z)}}\Big\{k \Rightarrow \mathbb{T}_2(k, x \cup z) \mid x \in \pi_2(X\circ k), z \in \pi_2(Z\circ k)\Big\}\\
  X \otimes Z &\mapsto \bigoplus_{\mathclap{(w\rightarrow xz) \in P}}\Big\{\mathbb{T}_2\Big(w, \big[\langle X\circ x, Z\circ z\rangle\big]\Big) \mid x \in \pi_1(X), z \in \pi_1(Z)\Big\}
\end{align}

These operators group subtrees by their root nonterminal, then aggregate their children. Instead of tracking sets, each $\Lambda$ now becomes a dictionary of $\mathbb{T}_2$, indexed by their root nonterminals.

  $\mathbb{T}_2$ is a convenient datatype for many operations involving CFGs. We can use it to approximate the Parikh image, compute the size of a finite CFG, and sample parse trees with or without replacement. For example, to obtain the Parikh map, we may use the following recurrence,

\begin{equation}
  \pi(T: \mathbb{T}_2) \mapsto \begin{cases}
%    \big|\{s \mid \big(\texttt{root}(T) \rightarrow s\big) \in P^\cap\}\big| & \text{if $T$ is a leaf,} \\
  \big[[1, 1] \text{ if } \texttt{root}(T) = s \text{ else } [0, 0]\big]_{s\in \Sigma}  & \text{if $T$ is a leaf,} \\
  \bigoplus_{\langle T_1, T_2\rangle \in \texttt{children}(T)} \pi(T_1) \otimes \pi(T_2) & \text{otherwise.}
  \end{cases}
\end{equation}

  %infix fun IntRange.merge(other: IntRange) =
  %  minOf(start, other.first)..maxOf(last, other.last)
  %
  %operator fun ParikhBounds.plus(other: ParikhBounds) =
  %  (keys + other.keys).associateWith {
  %    (get(it) ?: 0..0) merge (other[it] ?: 0..0)
  %  }
  %
  %operator fun ParikhBounds.times(other: ParikhBounds) =
  %  (keys + other.keys).associateWith {
  %    (get(it) ?: 0..0) join (other[it] ?: 0..0)
  %  }
  %
  %infix fun IntRange.join(other: IntRange) =
  %  (first + other.first)..(last + other.last)

  \noindent where the operations over Parikh maps $\oplus, \otimes: \Pi \times \Pi \rightarrow \Pi$ are defined respectively as follows:

  \begin{align}
      X \oplus Z &\mapsto \big[[\min(X_s \cup Z_s), \max(X_s \cup Z_s)]\big]_{s \in \Sigma}\\ X \otimes Z &\mapsto \big[[\min(X_s) + \min(Z_s), \max(X_s) + \max(Z_s)]\big]_{s \in \Sigma}
  \end{align}

  To obtain the Parikh interval for a CFG up to finite length bounds, we abstractly parse the string $(\_)^n$ for all $n \in [1, l_{\max}]$ and take the union of the intervals across $[|\err\sigma| - d_{\max}, |\err\sigma| + d_{\max}]$, which subsumes every valid repair in the Levenshtein ball.

  $\mathbb{T}_2$ also allows us to sample whole parse trees by obtaining $(\Lambda_\sigma^* \circ S): \mathbb{T}_2$. Given a probabilistic CFG whose productions indexed by each nonterminal are decorated with a probability vector $\mathbf{p}$ (this may be uniform in the non-probabilistic case), we define a tree sampler $\Gamma: (\mathbb{T}_2 \mid \mathbb{T}_2^2) \rightsquigarrow \mathbb{T}$ which recursively samples children according to a Multinoulli distribution:

\begin{equation}
  \Gamma(T) \mapsto \begin{cases}
        \texttt{BTree}\Big(\texttt{root}(T), \Gamma\big(\text{Multi}(\texttt{children}(T), \mathbf{p})\big)\Big) & \text{ if $T: \mathbb{T}_2$ } \\
        \big\langle \Gamma\big(\pi_1(T)\big), \Gamma\big(\pi_2(T)\big) \big\rangle & \text{ if $T: \mathbb{T}_2\times\mathbb{T}_2$ }
  \end{cases}
\end{equation}

This is closely related to the generating function for the ordinary Boltzmann sampler from analytic combinatorics,

\begin{equation}
  \Gamma C(x) \mapsto \begin{cases}
  \text{Bern} \left(\frac{A(x)}{A(x) + B(x)}\right) \rightarrow \Gamma A(x) \mid \Gamma B(x) & \text{ if } \mathcal{C}=\mathcal{A}+\mathcal{B} \\
  \big\langle \Gamma A(x), \Gamma B(x)\big\rangle & \text{ if } \mathcal{C}=\mathcal{A} \times \mathcal{B}
  \end{cases}
\end{equation}

\noindent however unlike Duchon et al.~\cite{duchon2004boltzmann}, our work does not depend on rejection to guarantee exact-size sampling, as all trees contained in $\mathbb{T}_2$ will necessarily be the same width.


The number of binary trees inhabiting a single instance of $\mathbb{T}_2$ is sensititive to the number of nonterminals and rule expansions in the grammar. To obtain the total number of trees with breadth $n$, we abstractly parse the porous string $\sigma = (\_)^n$, letting $T=\Lambda_{\sigma}^* \circ S$, then use the recurrence below to compute the total number of trees:

\begin{equation}
  |T: \mathbb{T}_2| \mapsto \begin{cases}
%    \big|\{s \mid \big(\texttt{root}(T) \rightarrow s\big) \in P^\cap\}\big| & \text{if $T$ is a leaf,} \\
    1 & \text{if $T$ is a leaf,} \\
    \sum_{\langle T_1, T_2\rangle \in \texttt{children}(T)} |T_1| \cdot |T_2| & \text{otherwise.}
  \end{cases}
\end{equation}

To sample all trees in a given $T: \mathbb{T}_2$ uniformly without replacement, we then construct a modular pairing function $\varphi: \mathbb{T}_2 \rightarrow \mathbb{Z}_{|T|} \rightarrow \texttt{BTree}$, that we define as follows:

\begin{equation}
  \varphi(T: \mathbb{T}_2, i: \mathbb{Z}_{|T|}) \mapsto \begin{cases}
  \Big\langle\texttt{BTree}\big(\texttt{root}(T)\big), i\Big\rangle & \text{if $T$ is a leaf,} \vspace{5pt}\\
  \text{Let } b = |\texttt{children}(T)|,\\
  \phantom{\text{Let }} q_1, r=\big\langle\lfloor\frac{i}{b}\rfloor, i \pmod{b}\big\rangle,\\
  \phantom{\text{Let }} lb, rb = \texttt{children}[r],\\
  \phantom{\text{Let }} T_1, q_2 = \varphi(lb, q_1),\\
  \phantom{\text{Let }} T_2, q_3 = \varphi(rb, q_2) \text{ in } \\
  \Big\langle\texttt{BTree}\big(\texttt{root}(T), T_1, T_2\big), q_3\Big\rangle & \text{otherwise.} \\
  \end{cases}
\end{equation}

If the language is suffiently small, instead of sampling trees, we can sample integers uniformly without replacement from $\mathbb{Z}_{|T|}$ then decode them into trees using $\varphi$. This procedure is the basis for our sampling algorithm and the method we use to decode repairs from the intersection grammar.

  \subsection{Ranked Repair}\label{sec:ranking}

  Returning to the ranked repair problem (Def.~\ref{def:ranked-repair}), the above procedure returns a set of strings, and we need an ordering over them. We note that any metric is sufficient, such as the perplexity of the string under a large language model, or the probability of the string under a PCFG. We the simplest possible solution: the negative log likelihood of the string computed by a variable-order Markov chain. This is a simple, fast, and effective metric for ranking strings, and as we will show, already yields competitive results in practice.

  \section{Evaluation}

  We consider the following research questions for our evaluation:

  \begin{itemize}
    \item \textbf{RQ 1}: What properties do natural repairs exhibit? (e.g., error frequency, edit distance)
    \item \textbf{RQ 2}: How precise is our technique at fixing syntax errors? (Precision@k vs. Seq2Parse)
    \item \textbf{RQ 3}: Which design choices are most significant? (e.g., search vs. sampling, n-gram order)
  \end{itemize}

  \subsection{Dataset}

  For our evaluation, we use the StackOverflow dataset from Hindle et al.~\cite{hindle2012naturalness}. We preprocess the dataset to lexicalize both the broken and fixed code snippets, then filter the dataset by length and edit distance, in which all Python snippets whose broken form is fewer than 80 lexical tokens and whose human fix is under four Levenshtein edits is retained.

  In the following experiments, we use two datasets of Python snippets. The first is a set of 5,600 pairwise-aligned (broken, fixed) Python code snippets from Wong et al.'s StackOverflow dataset~\cite{wong2019syntax} shorter than 40 lexical tokens, whose patch sizes are less than five lexical tokens ($|\Sigma| = 50, |\err{\sigma}| \leq 40, \Delta(\err{\sigma}, \ell) \leq 4$).

  The StackOverflow dataset is comprised of 500k Python code snippets, each of which has been annotated with a human repair. We depict the normalized edit loations relative to the snippet length below.

  \begin{figure}[h!]
    \begin{tikzpicture}
      \begin{axis}[
      ybar,
      bar width=5pt,
      xlabel={Length},
      ylabel={Frequency},
      title={Snippet length},
      axis x line*=bottom,
      axis y line*=left,
      ymin=0,
      ymax=65,
      xtick=data,
      xticklabels={,\leq 20,,\leq 40,,\leq 60,,\leq 80,,\leq 100},
      ymajorgrids=true,
      grid style=dashed,
      width=0.45\textwidth,
      height=0.3\textwidth
      ]

      \addplot table {
        X Y
        1 7.60
        2 14.52
        3 22.01
        4 30.54
        5 37.82
        6 44.30
        7 49.68
        8 55.21
        9 59.75
        10 63.59
      };
      \draw[red, dashed] (axis cs:8.5,0) -- (axis cs:8.5,65);
      \end{axis}
    \end{tikzpicture}
    \begin{tikzpicture}
      \begin{axis}[
        ybar,
        bar width=5pt,
        title={Human repair distance},
        xlabel={$\Delta(\err\sigma, \sigma)$},
        ylabel={Frequency},
        axis x line*=bottom,
        axis y line*=left,
        xtick=data,
        ymajorgrids=true,
        grid style=dashed,
        xticklabels={,\leq 2,,\leq 4,,\leq 6,,\leq 8,,\leq 10},
        width=0.45\textwidth,
        height=0.3\textwidth
      ]
        \addplot table {
          X Y
          1  31.48
          2  47.52
          3  54.89
          4  60.44
          5  63.88
          6  66.38
          7  68.02
          8  70.04
          9  71.49
          10 72.22
        };
      \draw[red, dashed] (axis cs:4.5,0) -- (axis cs:4.5,80);
      \end{axis}
    \end{tikzpicture}
    \begin{tikzpicture}
      \begin{axis}[
      ybar,
      bar width=5pt,
      xlabel={Beginning $\longleftrightarrow$ End},
      ylabel={Frequency},
      title={Normalized edit locations},
      axis x line*=bottom,
      axis y line*=left,
      ymin=0,
      ymax=35,
      xtick=data,
      xticklabels={,20\%,,40\%,,60\%,,80\%,,100\%},
      ymajorgrids=true,
      grid style=dashed,
      width=0.45\textwidth,
      height=0.3\textwidth
      ]

      \addplot table {
        X Y
        10 11.6539
        20 5.7252
        30 6.2087
        40 5.9542
        50 5.5980
        60 7.9389
        70 7.0738
        80 6.9466
        90 12.4173
        100 30.4835
      };
      \end{axis}
    \end{tikzpicture}
    \begin{tikzpicture}
      \begin{axis}[
        ybar,
        bar width=5pt,
        title={Intra-patch edit distance},
        xlabel={Caret distance},
        ylabel={Frequency},
        axis x line*=bottom,
        axis y line*=left,
        xtick=data,
        ymajorgrids=true,
        grid style=dashed,
        xticklabels={1,2,3,4,5,6,7,8,9,10+},
        width=0.45\textwidth,
        height=0.3\textwidth
      ]

        \addplot table {
          X Y
          1 40.66
          2 15.00
          3 5.80
          4 4.86
          5 4.26
          6 2.98
          7 2.05
          8 2.73
          9 1.62
          10 13.64
        };
      \end{axis}
    \end{tikzpicture}
    \caption{Repair statistics across the StackOverflow dataset, of which our method can handle about half in $\sim$30s and $\sim$140GB. Larger repairs and Levenshtein radii are possible, albeit requiring additional time and memory.}\label{fig:patch_stats}
  \end{figure}

  In the second set of experiments, we use a dataset of valid Python snippets from BIFI~\cite{yasunaga2021break}, then synthetically corrupt them by introducing syntax errors sampled from a distribution trained from the StackOverflow dataset. We then measure the precision@k of our repair procedure at recovering the original, uncorrupted snippet.

  \subsection{Experimental Setup}

  We measure the precision@k of our repair procedure at recovering human repairs of varying edit distances and latency cutoffs, where both the broken and fixed code are written by a human.

  To train the scoring function, we use a length-5 variable-order Markov (VOM) chain implemented using a count-min sketch based on Apache Datasketches~\cite{apache2022datasketches}. Training on 55 million StackOverflow tokens took roughly 10 minutes, after which calculating perplexity is nearly instantaneous. Sequences are scored using negative log likelihood with Laplace smoothing and our evaluation measures the precision@\{1, 5, 10, All\} for samples at varying latency cutoffs.

  Both sets of experiments were conducted on 40 Intel Skylake cores running at 2.4 GHz, with 16 GB of RAM, running bytecode compiled for JVM 17.0.2. We measure the precision using abstract lexical matching, following the Seq2Parse~\cite{sakkas2022seq2parse} evaluation, and give a baseline for their approach on the same dataset.

  \subsection{StackOverflow Evaluation}

  For our first experiment, we run the sampler until the human repair is detected, then measure the number of samples required to draw the exact human repair across varying Levenshtein radii.

  \begin{figure}[h!]
  % This file was created with tikzplotlib v0.10.1.
\begin{tikzpicture}[scale=0.75]
\begin{axis}[
legend cell align={left},
legend style={fill opacity=0.8, draw opacity=1, text opacity=1, draw=lightgray204, legend columns=-1, legend pos=north west},
width=\textwidth,
height=.4\textwidth,
log basis x={10},
tick align=outside,
tick pos=left,
axis lines*=left,
title={\textbf{Uniform Sampling Efficiency}},
x grid style={darkgray176},
xlabel={Samples drawn (log scale)},
xmin=0.34892211545542, xmax=1001,
xmode=log,
%xtick style={color=black},
xtick={0.01,0.1,1,10,100,1000,10000},
xticklabels={
  \(\displaystyle 10^-2\),
  \(\displaystyle 10^-1\),
  \(\displaystyle 10^2\),
  \(\displaystyle 10^3\),
  \(\displaystyle 10^4\),
  \(\displaystyle 10^5\),
  \(\displaystyle 10^6\)
},
y grid style={darkgray176},
ylabel={Precision@All},
ymin=0, ymax=101,
%ytick style={color=black}
ytick={0,20,40,60,80,100}
]
\draw[draw=none,fill=green!80] (axis cs:-0.5,0) rectangle (axis cs:0.9,100);
\addlegendimage{empty legend}
\addlegendentry{$\Delta(\err\sigma, \sigma')=$}
\addlegendimage{ybar,ybar legend,draw=none,fill=green!80}
\addlegendentry{\footnotesize{$1,$}}
\addlegendimage{ybar,ybar legend,draw=none,fill=blue!80}
\addlegendentry{\footnotesize{$2,$}}
\addlegendimage{ybar,ybar legend,draw=none,fill=orange!80}
\addlegendentry{\footnotesize{$3$}}

\draw[draw=none,fill=green!80] (axis cs:0.1,0) rectangle (axis cs:1.9,100);
\draw[draw=none,fill=green!80] (axis cs:1.1,0) rectangle (axis cs:2.9,100);
\draw[draw=none,fill=green!80] (axis cs:2.1,0) rectangle (axis cs:3.9,100);
\draw[draw=none,fill=green!80] (axis cs:3.1,0) rectangle (axis cs:4.9,100);
\draw[draw=none,fill=green!80] (axis cs:4.1,0) rectangle (axis cs:5.9,100);
\draw[draw=none,fill=green!80] (axis cs:5.1,0) rectangle (axis cs:6.9,100);
\draw[draw=none,fill=green!80] (axis cs:6.1,0) rectangle (axis cs:7.9,100);
\draw[draw=none,fill=green!80] (axis cs:7.1,0) rectangle (axis cs:8.9,100);
\draw[draw=none,fill=green!80] (axis cs:8.1,0) rectangle (axis cs:9.9,100);
\draw[draw=none,fill=green!80] (axis cs:9.1,0) rectangle (axis cs:10.9,100);
\draw[draw=none,fill=green!80] (axis cs:10.1,0) rectangle (axis cs:11.9,100);
\draw[draw=none,fill=green!80] (axis cs:11.1,0) rectangle (axis cs:12.9,100);
\draw[draw=none,fill=green!80] (axis cs:12.1,0) rectangle (axis cs:13.9,100);
\draw[draw=none,fill=green!80] (axis cs:13.1,0) rectangle (axis cs:14.9,100);
\draw[draw=none,fill=green!80] (axis cs:14.1,0) rectangle (axis cs:15.9,100);
\draw[draw=none,fill=green!80] (axis cs:15.1,0) rectangle (axis cs:16.9,100);
\draw[draw=none,fill=green!80] (axis cs:16.1,0) rectangle (axis cs:17.9,100);
\draw[draw=none,fill=green!80] (axis cs:17.1,0) rectangle (axis cs:18.9,100);
\draw[draw=none,fill=green!80] (axis cs:18.1,0) rectangle (axis cs:19.9,100);
\draw[draw=none,fill=green!80] (axis cs:19.1,0) rectangle (axis cs:20.9,100);
\draw[draw=none,fill=green!80] (axis cs:20.1,0) rectangle (axis cs:21.9,100);
\draw[draw=none,fill=green!80] (axis cs:21.1,0) rectangle (axis cs:22.9,100);
\draw[draw=none,fill=green!80] (axis cs:22.1,0) rectangle (axis cs:23.9,100);
\draw[draw=none,fill=green!80] (axis cs:23.1,0) rectangle (axis cs:24.9,100);
\draw[draw=none,fill=green!80] (axis cs:24.1,0) rectangle (axis cs:25.9,100);
\draw[draw=none,fill=green!80] (axis cs:25.1,0) rectangle (axis cs:26.9,100);
\draw[draw=none,fill=green!80] (axis cs:26.1,0) rectangle (axis cs:27.9,100);
\draw[draw=none,fill=green!80] (axis cs:27.1,0) rectangle (axis cs:28.9,100);
\draw[draw=none,fill=green!80] (axis cs:28.1,0) rectangle (axis cs:29.9,100);
\draw[draw=none,fill=green!80] (axis cs:29.1,0) rectangle (axis cs:30.9,100);
\draw[draw=none,fill=green!80] (axis cs:30.1,0) rectangle (axis cs:31.9,100);
\draw[draw=none,fill=green!80] (axis cs:31.1,0) rectangle (axis cs:32.9,100);
\draw[draw=none,fill=green!80] (axis cs:32.1,0) rectangle (axis cs:33.9,100);
\draw[draw=none,fill=green!80] (axis cs:33.1,0) rectangle (axis cs:34.9,100);
\draw[draw=none,fill=green!80] (axis cs:34.1,0) rectangle (axis cs:35.9,100);
\draw[draw=none,fill=green!80] (axis cs:35.1,0) rectangle (axis cs:36.9,100);
\draw[draw=none,fill=green!80] (axis cs:36.1,0) rectangle (axis cs:37.9,100);
\draw[draw=none,fill=green!80] (axis cs:37.1,0) rectangle (axis cs:38.9,100);
\draw[draw=none,fill=green!80] (axis cs:38.1,0) rectangle (axis cs:39.9,100);
\draw[draw=none,fill=green!80] (axis cs:39.1,0) rectangle (axis cs:40.9,100);
\draw[draw=none,fill=green!80] (axis cs:40.1,0) rectangle (axis cs:41.9,100);
\draw[draw=none,fill=green!80] (axis cs:41.1,0) rectangle (axis cs:42.9,100);
\draw[draw=none,fill=green!80] (axis cs:42.1,0) rectangle (axis cs:43.9,100);
\draw[draw=none,fill=green!80] (axis cs:43.1,0) rectangle (axis cs:44.9,100);
\draw[draw=none,fill=green!80] (axis cs:44.1,0) rectangle (axis cs:45.9,100);
\draw[draw=none,fill=green!80] (axis cs:45.1,0) rectangle (axis cs:46.9,100);
\draw[draw=none,fill=green!80] (axis cs:46.1,0) rectangle (axis cs:47.9,100);
\draw[draw=none,fill=green!80] (axis cs:47.1,0) rectangle (axis cs:48.9,100);
\draw[draw=none,fill=green!80] (axis cs:48.1,0) rectangle (axis cs:49.9,100);
\draw[draw=none,fill=green!80] (axis cs:49.1,0) rectangle (axis cs:50.9,100);
\draw[draw=none,fill=green!80] (axis cs:50.1,0) rectangle (axis cs:51.9,100);
\draw[draw=none,fill=green!80] (axis cs:51.1,0) rectangle (axis cs:52.9,100);
\draw[draw=none,fill=green!80] (axis cs:52.1,0) rectangle (axis cs:53.9,100);
\draw[draw=none,fill=green!80] (axis cs:53.1,0) rectangle (axis cs:54.9,100);
\draw[draw=none,fill=green!80] (axis cs:54.1,0) rectangle (axis cs:55.9,100);
\draw[draw=none,fill=green!80] (axis cs:55.1,0) rectangle (axis cs:56.9,100);
\draw[draw=none,fill=green!80] (axis cs:56.1,0) rectangle (axis cs:57.9,100);
\draw[draw=none,fill=green!80] (axis cs:57.1,0) rectangle (axis cs:58.9,100);
\draw[draw=none,fill=green!80] (axis cs:58.1,0) rectangle (axis cs:59.9,100);
\draw[draw=none,fill=green!80] (axis cs:59.1,0) rectangle (axis cs:60.9,100);
\draw[draw=none,fill=green!80] (axis cs:60.1,0) rectangle (axis cs:61.9,100);
\draw[draw=none,fill=green!80] (axis cs:61.1,0) rectangle (axis cs:62.9,100);
\draw[draw=none,fill=green!80] (axis cs:62.1,0) rectangle (axis cs:63.9,100);
\draw[draw=none,fill=green!80] (axis cs:63.1,0) rectangle (axis cs:64.9,100);
\draw[draw=none,fill=green!80] (axis cs:64.1,0) rectangle (axis cs:65.9,100);
\draw[draw=none,fill=green!80] (axis cs:65.1,0) rectangle (axis cs:66.9,100);
\draw[draw=none,fill=green!80] (axis cs:66.1,0) rectangle (axis cs:67.9,100);
\draw[draw=none,fill=green!80] (axis cs:67.1,0) rectangle (axis cs:68.9,100);
\draw[draw=none,fill=green!80] (axis cs:68.1,0) rectangle (axis cs:69.9,100);
\draw[draw=none,fill=green!80] (axis cs:69.1,0) rectangle (axis cs:70.9,100);
\draw[draw=none,fill=green!80] (axis cs:70.1,0) rectangle (axis cs:71.9,100);
\draw[draw=none,fill=green!80] (axis cs:71.1,0) rectangle (axis cs:72.9,100);
\draw[draw=none,fill=green!80] (axis cs:72.1,0) rectangle (axis cs:73.9,100);
\draw[draw=none,fill=green!80] (axis cs:73.1,0) rectangle (axis cs:74.9,100);
\draw[draw=none,fill=green!80] (axis cs:74.1,0) rectangle (axis cs:75.9,100);
\draw[draw=none,fill=green!80] (axis cs:75.1,0) rectangle (axis cs:76.9,100);
\draw[draw=none,fill=green!80] (axis cs:76.1,0) rectangle (axis cs:77.9,100);
\draw[draw=none,fill=green!80] (axis cs:77.1,0) rectangle (axis cs:78.9,100);
\draw[draw=none,fill=green!80] (axis cs:78.1,0) rectangle (axis cs:79.9,100);
\draw[draw=none,fill=green!80] (axis cs:79.1,0) rectangle (axis cs:80.9,100);
\draw[draw=none,fill=green!80] (axis cs:80.1,0) rectangle (axis cs:81.9,100);
\draw[draw=none,fill=green!80] (axis cs:81.1,0) rectangle (axis cs:82.9,100);
\draw[draw=none,fill=green!80] (axis cs:82.1,0) rectangle (axis cs:83.9,100);

\draw[draw=none,fill=blue!80] (axis cs:0.1,0) rectangle (axis cs:1.9,19.7452229299363);
\draw[draw=none,fill=blue!80] (axis cs:1.1,0) rectangle (axis cs:2.9,26.7515923566879);
\draw[draw=none,fill=blue!80] (axis cs:2.1,0) rectangle (axis cs:3.9,31.5286624203822);
\draw[draw=none,fill=blue!80] (axis cs:3.1,0) rectangle (axis cs:4.9,35.3503184713376);
\draw[draw=none,fill=blue!80] (axis cs:4.1,0) rectangle (axis cs:5.9,38.2165605095541);
\draw[draw=none,fill=blue!80] (axis cs:5.1,0) rectangle (axis cs:6.9,40.4458598726115);
\draw[draw=none,fill=blue!80] (axis cs:6.1,0) rectangle (axis cs:7.9,44.9044585987261);
\draw[draw=none,fill=blue!80] (axis cs:7.1,0) rectangle (axis cs:8.9,48.0891719745223);
\draw[draw=none,fill=blue!80] (axis cs:8.1,0) rectangle (axis cs:9.9,51.5923566878981);
\draw[draw=none,fill=blue!80] (axis cs:9.1,0) rectangle (axis cs:10.9,54.7770700636943);
\draw[draw=none,fill=blue!80] (axis cs:10.1,0) rectangle (axis cs:11.9,57.6433121019108);
\draw[draw=none,fill=blue!80] (axis cs:11.1,0) rectangle (axis cs:12.9,58.5987261146497);
\draw[draw=none,fill=blue!80] (axis cs:12.1,2) rectangle (axis cs:13.9,61.1464968152866);
\draw[draw=none,fill=blue!80] (axis cs:13.1,2) rectangle (axis cs:14.9,63.3757961783439);
\draw[draw=none,fill=blue!80] (axis cs:14.1,4) rectangle (axis cs:15.9,66.8789808917197);
\draw[draw=none,fill=blue!80] (axis cs:15.1,4) rectangle (axis cs:16.9,69.7452229299363);
\draw[draw=none,fill=blue!80] (axis cs:16.1,6) rectangle (axis cs:17.9,70.3821656050955);
\draw[draw=none,fill=blue!80] (axis cs:17.1,6) rectangle (axis cs:18.9,72.9299363057325);
\draw[draw=none,fill=blue!80] (axis cs:18.1,6) rectangle (axis cs:19.9,75.1592356687898);
\draw[draw=none,fill=blue!80] (axis cs:19.1,6) rectangle (axis cs:20.9,76.1146496815287);
\draw[draw=none,fill=blue!80] (axis cs:20.1,8) rectangle (axis cs:21.9,78.343949044586);
\draw[draw=none,fill=blue!80] (axis cs:21.1,10) rectangle (axis cs:22.9,79.2993630573248);
\draw[draw=none,fill=blue!80] (axis cs:22.1,10) rectangle (axis cs:23.9,80.2547770700637);
\draw[draw=none,fill=blue!80] (axis cs:23.1,12) rectangle (axis cs:24.9,80.5732484076433);
\draw[draw=none,fill=blue!80] (axis cs:24.1,12) rectangle (axis cs:25.9,82.1656050955414);
\draw[draw=none,fill=blue!80] (axis cs:25.1,12) rectangle (axis cs:26.9,82.484076433121);
\draw[draw=none,fill=blue!80] (axis cs:26.1,12) rectangle (axis cs:27.9,82.8025477707006);
\draw[draw=none,fill=blue!80] (axis cs:27.1,14) rectangle (axis cs:28.9,84.0764331210191);
\draw[draw=none,fill=blue!80] (axis cs:28.1,14) rectangle (axis cs:29.9,85.6687898089172);
\draw[draw=none,fill=blue!80] (axis cs:29.1,14) rectangle (axis cs:30.9,86.3057324840764);
\draw[draw=none,fill=blue!80] (axis cs:30.1,16) rectangle (axis cs:31.9,86.3057324840764);
\draw[draw=none,fill=blue!80] (axis cs:31.1,16) rectangle (axis cs:32.9,87.5796178343949);
\draw[draw=none,fill=blue!80] (axis cs:32.1,16) rectangle (axis cs:33.9,89.171974522293);
\draw[draw=none,fill=blue!80] (axis cs:33.1,16) rectangle (axis cs:34.9,90.1273885350319);
\draw[draw=none,fill=blue!80] (axis cs:34.1,18) rectangle (axis cs:35.9,90.4458598726115);
\draw[draw=none,fill=blue!80] (axis cs:35.1,18) rectangle (axis cs:36.9,90.7643312101911);
\draw[draw=none,fill=blue!80] (axis cs:36.1,18) rectangle (axis cs:37.9,91.7197452229299);
\draw[draw=none,fill=blue!80] (axis cs:37.1,18) rectangle (axis cs:38.9,91.7197452229299);
\draw[draw=none,fill=blue!80] (axis cs:38.1,18) rectangle (axis cs:39.9,92.0382165605096);
\draw[draw=none,fill=blue!80] (axis cs:39.1,18) rectangle (axis cs:40.9,92.9936305732484);
\draw[draw=none,fill=blue!80] (axis cs:40.1,18) rectangle (axis cs:41.9,93.312101910828);
\draw[draw=none,fill=blue!80] (axis cs:41.1,18) rectangle (axis cs:42.9,93.312101910828);
\draw[draw=none,fill=blue!80] (axis cs:42.1,18) rectangle (axis cs:43.9,93.312101910828);
\draw[draw=none,fill=blue!80] (axis cs:43.1,20) rectangle (axis cs:44.9,93.6305732484076);
\draw[draw=none,fill=blue!80] (axis cs:44.1,20) rectangle (axis cs:45.9,95.2229299363057);
\draw[draw=none,fill=blue!80] (axis cs:45.1,20) rectangle (axis cs:46.9,95.2229299363057);
\draw[draw=none,fill=blue!80] (axis cs:46.1,22) rectangle (axis cs:47.9,95.2229299363057);
\draw[draw=none,fill=blue!80] (axis cs:47.1,22) rectangle (axis cs:48.9,95.2229299363057);
\draw[draw=none,fill=blue!80] (axis cs:48.1,22) rectangle (axis cs:49.9,95.2229299363057);
\draw[draw=none,fill=blue!80] (axis cs:49.1,22) rectangle (axis cs:50.9,95.5414012738854);
\draw[draw=none,fill=blue!80] (axis cs:50.1,22) rectangle (axis cs:51.9,95.5414012738854);
\draw[draw=none,fill=blue!80] (axis cs:51.1,24) rectangle (axis cs:52.9,96.1783439490446);
\draw[draw=none,fill=blue!80] (axis cs:52.1,26) rectangle (axis cs:53.9,96.1783439490446);
\draw[draw=none,fill=blue!80] (axis cs:53.1,26) rectangle (axis cs:54.9,96.1783439490446);
\draw[draw=none,fill=blue!80] (axis cs:54.1,26) rectangle (axis cs:55.9,96.1783439490446);
\draw[draw=none,fill=blue!80] (axis cs:55.1,26) rectangle (axis cs:56.9,96.1783439490446);
\draw[draw=none,fill=blue!80] (axis cs:56.1,26) rectangle (axis cs:57.9,96.8152866242038);
\draw[draw=none,fill=blue!80] (axis cs:57.1,26) rectangle (axis cs:58.9,97.1337579617834);
\draw[draw=none,fill=blue!80] (axis cs:58.1,26) rectangle (axis cs:59.9,97.1337579617834);
\draw[draw=none,fill=blue!80] (axis cs:59.1,26) rectangle (axis cs:60.9,97.1337579617834);
\draw[draw=none,fill=blue!80] (axis cs:60.1,26) rectangle (axis cs:61.9,97.1337579617834);
\draw[draw=none,fill=blue!80] (axis cs:61.1,26) rectangle (axis cs:62.9,97.4522292993631);
\draw[draw=none,fill=blue!80] (axis cs:62.1,26) rectangle (axis cs:63.9,98.0891719745223);
\draw[draw=none,fill=blue!80] (axis cs:63.1,26) rectangle (axis cs:64.9,98.0891719745223);
\draw[draw=none,fill=blue!80] (axis cs:64.1,26) rectangle (axis cs:65.9,98.4076433121019);
\draw[draw=none,fill=blue!80] (axis cs:65.1,26) rectangle (axis cs:66.9,98.7261146496815);
\draw[draw=none,fill=blue!80] (axis cs:66.1,26) rectangle (axis cs:67.9,98.7261146496815);
\draw[draw=none,fill=blue!80] (axis cs:67.1,26) rectangle (axis cs:68.9,98.7261146496815);
\draw[draw=none,fill=blue!80] (axis cs:68.1,26) rectangle (axis cs:69.9,99.0445859872611);
\draw[draw=none,fill=blue!80] (axis cs:69.1,26) rectangle (axis cs:70.9,99.0445859872611);
\draw[draw=none,fill=blue!80] (axis cs:70.1,26) rectangle (axis cs:71.9,99.0445859872611);
\draw[draw=none,fill=blue!80] (axis cs:71.1,26) rectangle (axis cs:72.9,99.0445859872611);
\draw[draw=none,fill=blue!80] (axis cs:72.1,26) rectangle (axis cs:73.9,99.3630573248408);
\draw[draw=none,fill=blue!80] (axis cs:73.1,26) rectangle (axis cs:74.9,99.3630573248408);
\draw[draw=none,fill=blue!80] (axis cs:74.1,26) rectangle (axis cs:75.9,99.3630573248408);
\draw[draw=none,fill=blue!80] (axis cs:75.1,28) rectangle (axis cs:76.9,99.3630573248408);
\draw[draw=none,fill=blue!80] (axis cs:76.1,28) rectangle (axis cs:77.9,99.3630573248408);
\draw[draw=none,fill=blue!80] (axis cs:77.1,28) rectangle (axis cs:78.9,99.6815286624204);
\draw[draw=none,fill=blue!80] (axis cs:78.1,30) rectangle (axis cs:786.9,100);

\draw[draw=none,fill=orange!80] (axis cs:12.0,0) rectangle (axis cs:14.99,2);
\draw[draw=none,fill=orange!80] (axis cs:14.0,0) rectangle (axis cs:15.99,4);
\draw[draw=none,fill=orange!80] (axis cs:15.0,0) rectangle (axis cs:16.99,4);
\draw[draw=none,fill=orange!80] (axis cs:16.0,0) rectangle (axis cs:17.99,6);
\draw[draw=none,fill=orange!80] (axis cs:17.0,0) rectangle (axis cs:18.99,6);
\draw[draw=none,fill=orange!80] (axis cs:18.0,0) rectangle (axis cs:19.99,6);
\draw[draw=none,fill=orange!80] (axis cs:19.0,0) rectangle (axis cs:20.99,6);
\draw[draw=none,fill=orange!80] (axis cs:20.0,0) rectangle (axis cs:21.99,8);
\draw[draw=none,fill=orange!80] (axis cs:21.0,0) rectangle (axis cs:22.99,10);
\draw[draw=none,fill=orange!80] (axis cs:22.0,0) rectangle (axis cs:23.99,10);
\draw[draw=none,fill=orange!80] (axis cs:23.0,0) rectangle (axis cs:24.99,12);
\draw[draw=none,fill=orange!80] (axis cs:24.0,0) rectangle (axis cs:25.99,12);
\draw[draw=none,fill=orange!80] (axis cs:25.0,0) rectangle (axis cs:26.99,12);
\draw[draw=none,fill=orange!80] (axis cs:26.0,0) rectangle (axis cs:27.99,12);
\draw[draw=none,fill=orange!80] (axis cs:27.0,0) rectangle (axis cs:28.99,14);
\draw[draw=none,fill=orange!80] (axis cs:28.0,0) rectangle (axis cs:29.99,14);
\draw[draw=none,fill=orange!80] (axis cs:29.0,0) rectangle (axis cs:30.99,14);
\draw[draw=none,fill=orange!80] (axis cs:30.0,0) rectangle (axis cs:31.99,16);
\draw[draw=none,fill=orange!80] (axis cs:31.0,0) rectangle (axis cs:32.99,16);
\draw[draw=none,fill=orange!80] (axis cs:32.0,0) rectangle (axis cs:33.99,16);
\draw[draw=none,fill=orange!80] (axis cs:33.0,0) rectangle (axis cs:34.99,16);
\draw[draw=none,fill=orange!80] (axis cs:34.0,0) rectangle (axis cs:43.99,18);
\draw[draw=none,fill=orange!80] (axis cs:43.0,0) rectangle (axis cs:44.99,20);
\draw[draw=none,fill=orange!80] (axis cs:44.0,0) rectangle (axis cs:45.99,20);
\draw[draw=none,fill=orange!80] (axis cs:45.0,0) rectangle (axis cs:46.99,20);
\draw[draw=none,fill=orange!80] (axis cs:46.0,0) rectangle (axis cs:47.99,22);
\draw[draw=none,fill=orange!80] (axis cs:47.0,0) rectangle (axis cs:48.99,22);
\draw[draw=none,fill=orange!80] (axis cs:48.0,0) rectangle (axis cs:49.99,22);
\draw[draw=none,fill=orange!80] (axis cs:49.0,0) rectangle (axis cs:50.99,22);
\draw[draw=none,fill=orange!80] (axis cs:50.0,0) rectangle (axis cs:51.99,22);
\draw[draw=none,fill=orange!80] (axis cs:51.0,0) rectangle (axis cs:52.99,24);
\draw[draw=none,fill=orange!80] (axis cs:52.0,0) rectangle (axis cs:75.99,26);
\draw[draw=none,fill=orange!80] (axis cs:75.0,0) rectangle (axis cs:76.99,28);
\draw[draw=none,fill=orange!80] (axis cs:76.0,0) rectangle (axis cs:77.99,28);
\draw[draw=none,fill=orange!80] (axis cs:77.0,0) rectangle (axis cs:78.99,28);
\draw[draw=none,fill=orange!80] (axis cs:78.0,0) rectangle (axis cs:79.99,30);
\draw[draw=none,fill=orange!80] (axis cs:79.0,0) rectangle (axis cs:80.99,30);
\draw[draw=none,fill=orange!80] (axis cs:80.0,0) rectangle (axis cs:81.99,30);
\draw[draw=none,fill=orange!80] (axis cs:81.0,0) rectangle (axis cs:82.99,30);
\draw[draw=none,fill=orange!80] (axis cs:82.0,0) rectangle (axis cs:83.99,30);
\draw[draw=none,fill=orange!80] (axis cs:83.0,0) rectangle (axis cs:84.99,30);
\draw[draw=none,fill=orange!80] (axis cs:84.0,0) rectangle (axis cs:85.99,30);
\draw[draw=none,fill=orange!80] (axis cs:85.0,0) rectangle (axis cs:86.99,32);
\draw[draw=none,fill=orange!80] (axis cs:86.0,0) rectangle (axis cs:87.99,32);
\draw[draw=none,fill=orange!80] (axis cs:87.0,0) rectangle (axis cs:88.99,32);
\draw[draw=none,fill=orange!80] (axis cs:88.0,0) rectangle (axis cs:89.99,32);
\draw[draw=none,fill=orange!80] (axis cs:89.0,0) rectangle (axis cs:90.99,32);
\draw[draw=none,fill=orange!80] (axis cs:90.0,0) rectangle (axis cs:91.99,32);
\draw[draw=none,fill=orange!80] (axis cs:91.0,0) rectangle (axis cs:116.99,36);
\draw[draw=none,fill=orange!80] (axis cs:116.0,0) rectangle (axis cs:130.99,38);
\draw[draw=none,fill=orange!80] (axis cs:130.0,0) rectangle (axis cs:138.99,40);
\draw[draw=none,fill=orange!80] (axis cs:138.0,0) rectangle (axis cs:139.99,42);
\draw[draw=none,fill=orange!80] (axis cs:139.0,0) rectangle (axis cs:140.99,42);
\draw[draw=none,fill=orange!80] (axis cs:140.0,0) rectangle (axis cs:141.99,42);
\draw[draw=none,fill=orange!80] (axis cs:141.0,0) rectangle (axis cs:142.99,42);
\draw[draw=none,fill=orange!80] (axis cs:142.0,0) rectangle (axis cs:143.99,42);
\draw[draw=none,fill=orange!80] (axis cs:143.0,0) rectangle (axis cs:144.99,42);
\draw[draw=none,fill=orange!80] (axis cs:144.0,0) rectangle (axis cs:145.99,42);
\draw[draw=none,fill=orange!80] (axis cs:145.0,0) rectangle (axis cs:170.99,44);
\draw[draw=none,fill=orange!80] (axis cs:170.0,0) rectangle (axis cs:191.99,46);
\draw[draw=none,fill=orange!80] (axis cs:191.0,0) rectangle (axis cs:192.99,48);
\draw[draw=none,fill=orange!80] (axis cs:192.0,0) rectangle (axis cs:193.99,48);
\draw[draw=none,fill=orange!80] (axis cs:193.0,0) rectangle (axis cs:194.99,48);
\draw[draw=none,fill=orange!80] (axis cs:194.0,0) rectangle (axis cs:195.99,48);
\draw[draw=none,fill=orange!80] (axis cs:195.0,0) rectangle (axis cs:196.99,48);
\draw[draw=none,fill=orange!80] (axis cs:196.0,0) rectangle (axis cs:197.99,48);
\draw[draw=none,fill=orange!80] (axis cs:197.0,0) rectangle (axis cs:211.99,50);
\draw[draw=none,fill=orange!80] (axis cs:211.0,0) rectangle (axis cs:212.99,52);
\draw[draw=none,fill=orange!80] (axis cs:212.0,0) rectangle (axis cs:213.99,52);
\draw[draw=none,fill=orange!80] (axis cs:213.0,0) rectangle (axis cs:214.99,52);
\draw[draw=none,fill=orange!80] (axis cs:214.0,0) rectangle (axis cs:215.99,56);
\draw[draw=none,fill=orange!80] (axis cs:215.0,0) rectangle (axis cs:216.99,56);
\draw[draw=none,fill=orange!80] (axis cs:216.0,0) rectangle (axis cs:217.99,56);
\draw[draw=none,fill=orange!80] (axis cs:217.0,0) rectangle (axis cs:218.99,56);
\draw[draw=none,fill=orange!80] (axis cs:218.0,0) rectangle (axis cs:219.99,56);
\draw[draw=none,fill=orange!80] (axis cs:219.0,0) rectangle (axis cs:220.99,56);
\draw[draw=none,fill=orange!80] (axis cs:220.0,0) rectangle (axis cs:221.99,56);
\draw[draw=none,fill=orange!80] (axis cs:221.0,0) rectangle (axis cs:222.99,56);
\draw[draw=none,fill=orange!80] (axis cs:222.0,0) rectangle (axis cs:223.99,56);
\draw[draw=none,fill=orange!80] (axis cs:223.0,0) rectangle (axis cs:224.99,56);
\draw[draw=none,fill=orange!80] (axis cs:224.0,0) rectangle (axis cs:236.99,58);
\draw[draw=none,fill=orange!80] (axis cs:236.0,0) rectangle (axis cs:247.99,60);
\draw[draw=none,fill=orange!80] (axis cs:247.0,0) rectangle (axis cs:288.99,64);
\draw[draw=none,fill=orange!80] (axis cs:288.0,0) rectangle (axis cs:289.99,68);
\draw[draw=none,fill=orange!80] (axis cs:289.0,0) rectangle (axis cs:290.99,68);
\draw[draw=none,fill=orange!80] (axis cs:290.0,0) rectangle (axis cs:304.99,70);
\draw[draw=none,fill=orange!80] (axis cs:304.0,0) rectangle (axis cs:305.99,72);
\draw[draw=none,fill=orange!80] (axis cs:305.0,0) rectangle (axis cs:306.99,72);
\draw[draw=none,fill=orange!80] (axis cs:306.0,0) rectangle (axis cs:307.99,72);
\draw[draw=none,fill=orange!80] (axis cs:307.0,0) rectangle (axis cs:308.99,72);
\draw[draw=none,fill=orange!80] (axis cs:308.0,0) rectangle (axis cs:309.99,74);
\draw[draw=none,fill=orange!80] (axis cs:309.0,0) rectangle (axis cs:310.99,74);
\draw[draw=none,fill=orange!80] (axis cs:310.0,0) rectangle (axis cs:311.99,74);
\draw[draw=none,fill=orange!80] (axis cs:311.0,0) rectangle (axis cs:366.99,76);
\draw[draw=none,fill=orange!80] (axis cs:366.0,0) rectangle (axis cs:394.99,78);
\draw[draw=none,fill=orange!80] (axis cs:394.0,0) rectangle (axis cs:415.99,80);
\draw[draw=none,fill=orange!80] (axis cs:415.0,0) rectangle (axis cs:416.99,82);
\draw[draw=none,fill=orange!80] (axis cs:416.0,0) rectangle (axis cs:440.99,84);
\draw[draw=none,fill=orange!80] (axis cs:440.0,0) rectangle (axis cs:490.99,86);
\draw[draw=none,fill=orange!80] (axis cs:490.0,0) rectangle (axis cs:513.99,88);
\draw[draw=none,fill=orange!80] (axis cs:513.0,0) rectangle (axis cs:522.99,90);
\draw[draw=none,fill=orange!80] (axis cs:522.0,0) rectangle (axis cs:554.99,92);
\draw[draw=none,fill=orange!80] (axis cs:554.0,0) rectangle (axis cs:583.99,94);
\draw[draw=none,fill=orange!80] (axis cs:583.0,0) rectangle (axis cs:597.99,96);
\draw[draw=none,fill=orange!80] (axis cs:597.0,0) rectangle (axis cs:786.99,98);
\end{axis}

\end{tikzpicture}
    \caption{Sample efficiency of LBH sampler at varying Levenshtein radii. After drawing up to $\sim10^5$ samples without replacement we can usually saturate the admissible set for almost all repairs fewer than four edits.}\label{fig:sample_efficiency}
  \end{figure}

  Next, measure the precision at various ranking cutoffs for varying wall-clock timeouts. Here, P@\{k=1, 5, 10, All\} indicates the percentage of syntax errors with a human repair of $\Delta=\{1, 2, 3, 4\}$ edits found in $\leq p$ seconds that were matched within the top-k results, based on n-gram likelihood.

  \begin{figure}[h!]
%    \resizebox{.19\textwidth}{!}{% This file was created with tikzplotlib v0.10.1.
\begin{tikzpicture}

\definecolor{darkgray176}{RGB}{176,176,176}
\definecolor{darkviolet1270255}{RGB}{127,0,255}
\definecolor{deepskyblue3176236}{RGB}{3,176,236}
\definecolor{dodgerblue45123246}{RGB}{45,123,246}
\definecolor{royalblue8762253}{RGB}{87,62,253}

\begin{axis}[
tick align=outside,
tick pos=left,
axis lines=left,
title={\(\displaystyle \Delta\in[1,3]\) Repair Precision},
legend style={fill opacity=0.8, draw opacity=1, text opacity=1, legend columns=1, legend pos=north west},
x grid style={darkgray176},
xlabel={Seconds},
xmin=-0.5925, xmax=8.5925,
xtick style={color=black},
xtick={0,1,2,3,4,5,6,7,8},
xticklabels={1,2,3,4,5,6,7,8,9},
y grid style={darkgray176},
ylabel={Precision@k},
ymin=0, ymax=0.6993,
ytick style={color=black}
]
\draw[draw=none,fill=darkviolet1270255] (axis cs:-0.175,0) rectangle (axis cs:0.175,0.149);
\addlegendimage{ybar,ybar legend,draw=none,fill=darkviolet1270255}
\addlegendentry{P@All}

\draw[draw=none,fill=darkviolet1270255] (axis cs:0.825,0) rectangle (axis cs:1.175,0.303);
\draw[draw=none,fill=darkviolet1270255] (axis cs:1.825,0) rectangle (axis cs:2.175,0.412);
\draw[draw=none,fill=darkviolet1270255] (axis cs:2.825,0) rectangle (axis cs:3.175,0.501);
\draw[draw=none,fill=darkviolet1270255] (axis cs:3.825,0) rectangle (axis cs:4.175,0.569);
\draw[draw=none,fill=darkviolet1270255] (axis cs:4.825,0) rectangle (axis cs:5.175,0.605);
\draw[draw=none,fill=darkviolet1270255] (axis cs:5.825,0) rectangle (axis cs:6.175,0.63);
\draw[draw=none,fill=darkviolet1270255] (axis cs:6.825,0) rectangle (axis cs:7.175,0.645);
\draw[draw=none,fill=darkviolet1270255] (axis cs:7.825,0) rectangle (axis cs:8.175,0.666);
\draw[draw=none,fill=royalblue8762253] (axis cs:-0.175,0) rectangle (axis cs:0.175,0.128);
\addlegendimage{ybar,ybar legend,draw=none,fill=royalblue8762253}
\addlegendentry{P@10}

\draw[draw=none,fill=royalblue8762253] (axis cs:0.825,0) rectangle (axis cs:1.175,0.239);
\draw[draw=none,fill=royalblue8762253] (axis cs:1.825,0) rectangle (axis cs:2.175,0.323);
\draw[draw=none,fill=royalblue8762253] (axis cs:2.825,0) rectangle (axis cs:3.175,0.374);
\draw[draw=none,fill=royalblue8762253] (axis cs:3.825,0) rectangle (axis cs:4.175,0.421);
\draw[draw=none,fill=royalblue8762253] (axis cs:4.825,0) rectangle (axis cs:5.175,0.446);
\draw[draw=none,fill=royalblue8762253] (axis cs:5.825,0) rectangle (axis cs:6.175,0.459);
\draw[draw=none,fill=royalblue8762253] (axis cs:6.825,0) rectangle (axis cs:7.175,0.469);
\draw[draw=none,fill=royalblue8762253] (axis cs:7.825,0) rectangle (axis cs:8.175,0.477);
\draw[draw=none,fill=dodgerblue45123246] (axis cs:-0.175,0) rectangle (axis cs:0.175,0.11);
\addlegendimage{ybar,ybar legend,draw=none,fill=dodgerblue45123246}
\addlegendentry{P@5}

\draw[draw=none,fill=dodgerblue45123246] (axis cs:0.825,0) rectangle (axis cs:1.175,0.206);
\draw[draw=none,fill=dodgerblue45123246] (axis cs:1.825,0) rectangle (axis cs:2.175,0.282);
\draw[draw=none,fill=dodgerblue45123246] (axis cs:2.825,0) rectangle (axis cs:3.175,0.325);
\draw[draw=none,fill=dodgerblue45123246] (axis cs:3.825,0) rectangle (axis cs:4.175,0.369);
\draw[draw=none,fill=dodgerblue45123246] (axis cs:4.825,0) rectangle (axis cs:5.175,0.392);
\draw[draw=none,fill=dodgerblue45123246] (axis cs:5.825,0) rectangle (axis cs:6.175,0.405);
\draw[draw=none,fill=dodgerblue45123246] (axis cs:6.825,0) rectangle (axis cs:7.175,0.414);
\draw[draw=none,fill=dodgerblue45123246] (axis cs:7.825,0) rectangle (axis cs:8.175,0.422);
\draw[draw=none,fill=deepskyblue3176236] (axis cs:-0.175,0) rectangle (axis cs:0.175,0.092);
\addlegendimage{ybar,ybar legend,draw=none,fill=deepskyblue3176236}
\addlegendentry{P@1}

\draw[draw=none,fill=deepskyblue3176236] (axis cs:0.825,0) rectangle (axis cs:1.175,0.18);
\draw[draw=none,fill=deepskyblue3176236] (axis cs:1.825,0) rectangle (axis cs:2.175,0.244);
\draw[draw=none,fill=deepskyblue3176236] (axis cs:2.825,0) rectangle (axis cs:3.175,0.285);
\draw[draw=none,fill=deepskyblue3176236] (axis cs:3.825,0) rectangle (axis cs:4.175,0.324);
\draw[draw=none,fill=deepskyblue3176236] (axis cs:4.825,0) rectangle (axis cs:5.175,0.343);
\draw[draw=none,fill=deepskyblue3176236] (axis cs:5.825,0) rectangle (axis cs:6.175,0.356);
\draw[draw=none,fill=deepskyblue3176236] (axis cs:6.825,0) rectangle (axis cs:7.175,0.365);
\draw[draw=none,fill=deepskyblue3176236] (axis cs:7.825,0) rectangle (axis cs:8.175,0.37);
\addplot [red, thick] coordinates {(-0.8925,0.29) (14.8925,0.29)};
\end{axis}

\end{tikzpicture}
}
    \resizebox{.24\textwidth}{!}{% This file was created with tikzplotlib v0.10.1.
\begin{tikzpicture}

\definecolor{darkgray176}{RGB}{176,176,176}
\definecolor{darkviolet1270255}{RGB}{127,0,255}
\definecolor{deepskyblue3176236}{RGB}{3,176,236}
\definecolor{dodgerblue45123246}{RGB}{45,123,246}
\definecolor{royalblue8762253}{RGB}{87,62,253}

\begin{axis}[
tick align=outside,
tick pos=left,
axis lines*=left,
title={{\Large\(\displaystyle \Delta=1\) Repair Precision}},
legend style={fill opacity=0.8, draw opacity=1, text opacity=1, legend columns=1, legend pos=north west},
x grid style={darkgray176},
xlabel={Seconds},
xmin=-0.5925, xmax=8.5925,
xtick style={color=black},
xtick={0,1,2,3,4,5,6,7,8},
xticklabels={1,2,3,4,5,6,7,8,9},
y grid style={darkgray176},
ylabel={Precision@k},
ymin=0, ymax=1.0311,
ytick style={color=black}
]
\draw[draw=none,fill=darkviolet1270255] (axis cs:-0.175,0) rectangle (axis cs:0.175,0.546);
\addlegendimage{ybar,ybar legend,draw=none,fill=darkviolet1270255}
\addlegendentry{P@All}

\draw[draw=none,fill=darkviolet1270255] (axis cs:0.825,0) rectangle (axis cs:1.175,0.752);
\draw[draw=none,fill=darkviolet1270255] (axis cs:1.825,0) rectangle (axis cs:2.175,0.88);
\draw[draw=none,fill=darkviolet1270255] (axis cs:2.825,0) rectangle (axis cs:3.175,0.908);
\draw[draw=none,fill=darkviolet1270255] (axis cs:3.825,0) rectangle (axis cs:4.175,0.933);
\draw[draw=none,fill=darkviolet1270255] (axis cs:4.825,0) rectangle (axis cs:5.175,0.954);
\draw[draw=none,fill=darkviolet1270255] (axis cs:5.825,0) rectangle (axis cs:6.175,0.966);
\draw[draw=none,fill=darkviolet1270255] (axis cs:6.825,0) rectangle (axis cs:7.175,0.975);
\draw[draw=none,fill=darkviolet1270255] (axis cs:7.825,0) rectangle (axis cs:8.175,0.982);
\draw[draw=none,fill=royalblue8762253] (axis cs:-0.175,0) rectangle (axis cs:0.175,0.546);
\addlegendimage{ybar,ybar legend,draw=none,fill=royalblue8762253}
\addlegendentry{P@10}

\draw[draw=none,fill=royalblue8762253] (axis cs:0.825,0) rectangle (axis cs:1.175,0.752);
\draw[draw=none,fill=royalblue8762253] (axis cs:1.825,0) rectangle (axis cs:2.175,0.88);
\draw[draw=none,fill=royalblue8762253] (axis cs:2.825,0) rectangle (axis cs:3.175,0.908);
\draw[draw=none,fill=royalblue8762253] (axis cs:3.825,0) rectangle (axis cs:4.175,0.933);
\draw[draw=none,fill=royalblue8762253] (axis cs:4.825,0) rectangle (axis cs:5.175,0.954);
\draw[draw=none,fill=royalblue8762253] (axis cs:5.825,0) rectangle (axis cs:6.175,0.966);
\draw[draw=none,fill=royalblue8762253] (axis cs:6.825,0) rectangle (axis cs:7.175,0.975);
\draw[draw=none,fill=royalblue8762253] (axis cs:7.825,0) rectangle (axis cs:8.175,0.982);
\draw[draw=none,fill=dodgerblue45123246] (axis cs:-0.175,0) rectangle (axis cs:0.175,0.546);
\addlegendimage{ybar,ybar legend,draw=none,fill=dodgerblue45123246}
\addlegendentry{P@5}

\draw[draw=none,fill=dodgerblue45123246] (axis cs:0.825,0) rectangle (axis cs:1.175,0.752);
\draw[draw=none,fill=dodgerblue45123246] (axis cs:1.825,0) rectangle (axis cs:2.175,0.88);
\draw[draw=none,fill=dodgerblue45123246] (axis cs:2.825,0) rectangle (axis cs:3.175,0.908);
\draw[draw=none,fill=dodgerblue45123246] (axis cs:3.825,0) rectangle (axis cs:4.175,0.933);
\draw[draw=none,fill=dodgerblue45123246] (axis cs:4.825,0) rectangle (axis cs:5.175,0.954);
\draw[draw=none,fill=dodgerblue45123246] (axis cs:5.825,0) rectangle (axis cs:6.175,0.966);
\draw[draw=none,fill=dodgerblue45123246] (axis cs:6.825,0) rectangle (axis cs:7.175,0.975);
\draw[draw=none,fill=dodgerblue45123246] (axis cs:7.825,0) rectangle (axis cs:8.175,0.982);
\draw[draw=none,fill=deepskyblue3176236] (axis cs:-0.175,0) rectangle (axis cs:0.175,0.546);
\addlegendimage{ybar,ybar legend,draw=none,fill=deepskyblue3176236}
\addlegendentry{P@1}

\draw[draw=none,fill=deepskyblue3176236] (axis cs:0.825,0) rectangle (axis cs:1.175,0.752);
\draw[draw=none,fill=deepskyblue3176236] (axis cs:1.825,0) rectangle (axis cs:2.175,0.88);
\draw[draw=none,fill=deepskyblue3176236] (axis cs:2.825,0) rectangle (axis cs:3.175,0.908);
\draw[draw=none,fill=deepskyblue3176236] (axis cs:3.825,0) rectangle (axis cs:4.175,0.933);
\draw[draw=none,fill=deepskyblue3176236] (axis cs:4.825,0) rectangle (axis cs:5.175,0.954);
\draw[draw=none,fill=deepskyblue3176236] (axis cs:5.825,0) rectangle (axis cs:6.175,0.966);
\draw[draw=none,fill=deepskyblue3176236] (axis cs:6.825,0) rectangle (axis cs:7.175,0.975);
\draw[draw=none,fill=deepskyblue3176236] (axis cs:7.825,0) rectangle (axis cs:8.175,0.982);
\addplot [red, thick] coordinates {(-0.8925,0.39) (14.8925,0.39)};
\end{axis}

\end{tikzpicture}
}
    \resizebox{.24\textwidth}{!}{% This file was created with tikzplotlib v0.10.1.
\begin{tikzpicture}

\definecolor{darkgray176}{RGB}{176,176,176}
\definecolor{darkviolet1270255}{RGB}{127,0,255}
\definecolor{deepskyblue3176236}{RGB}{3,176,236}
\definecolor{dodgerblue45123246}{RGB}{45,123,246}
\definecolor{royalblue8762253}{RGB}{87,62,253}

\begin{axis}[
tick align=outside,
tick pos=left,
title={\(\displaystyle \Delta=2\) Repair Precision},
legend style={fill opacity=0.8, draw opacity=1, text opacity=1, legend columns=1, legend pos=north west},
x grid style={darkgray176},
xlabel={Seconds},
xmin=-0.6425, xmax=9.6425,
xtick style={color=black},
xtick={0,1,2,3,4,5,6,7,8,9},
xticklabels={4,8,12,16,20,24,28,32,36,40},
y grid style={darkgray176},
ylabel={Precision@k},
ymin=0, ymax=0.74025,
ytick style={color=black}
]
\draw[draw=none,fill=darkviolet1270255] (axis cs:-0.175,0) rectangle (axis cs:0.175,0.149);
\addlegendimage{ybar,ybar legend,draw=none,fill=darkviolet1270255}
\addlegendentry{P@All}

\draw[draw=none,fill=darkviolet1270255] (axis cs:0.825,0) rectangle (axis cs:1.175,0.305);
\draw[draw=none,fill=darkviolet1270255] (axis cs:1.825,0) rectangle (axis cs:2.175,0.464);
\draw[draw=none,fill=darkviolet1270255] (axis cs:2.825,0) rectangle (axis cs:3.175,0.569);
\draw[draw=none,fill=darkviolet1270255] (axis cs:3.825,0) rectangle (axis cs:4.175,0.631);
\draw[draw=none,fill=darkviolet1270255] (axis cs:4.825,0) rectangle (axis cs:5.175,0.672);
\draw[draw=none,fill=darkviolet1270255] (axis cs:5.825,0) rectangle (axis cs:6.175,0.682);
\draw[draw=none,fill=darkviolet1270255] (axis cs:6.825,0) rectangle (axis cs:7.175,0.697);
\draw[draw=none,fill=darkviolet1270255] (axis cs:7.825,0) rectangle (axis cs:8.175,0.7);
\draw[draw=none,fill=darkviolet1270255] (axis cs:8.825,0) rectangle (axis cs:9.175,0.705);
\draw[draw=none,fill=royalblue8762253] (axis cs:-0.175,0) rectangle (axis cs:0.175,0.072);
\addlegendimage{ybar,ybar legend,draw=none,fill=royalblue8762253}
\addlegendentry{P@10}

\draw[draw=none,fill=royalblue8762253] (axis cs:0.825,0) rectangle (axis cs:1.175,0.162);
\draw[draw=none,fill=royalblue8762253] (axis cs:1.825,0) rectangle (axis cs:2.175,0.231);
\draw[draw=none,fill=royalblue8762253] (axis cs:2.825,0) rectangle (axis cs:3.175,0.282);
\draw[draw=none,fill=royalblue8762253] (axis cs:3.825,0) rectangle (axis cs:4.175,0.297);
\draw[draw=none,fill=royalblue8762253] (axis cs:4.825,0) rectangle (axis cs:5.175,0.318);
\draw[draw=none,fill=royalblue8762253] (axis cs:5.825,0) rectangle (axis cs:6.175,0.321);
\draw[draw=none,fill=royalblue8762253] (axis cs:6.825,0) rectangle (axis cs:7.175,0.331);
\draw[draw=none,fill=royalblue8762253] (axis cs:7.825,0) rectangle (axis cs:8.175,0.333);
\draw[draw=none,fill=royalblue8762253] (axis cs:8.825,0) rectangle (axis cs:9.175,0.336);
\draw[draw=none,fill=dodgerblue45123246] (axis cs:-0.175,0) rectangle (axis cs:0.175,0.072);
\addlegendimage{ybar,ybar legend,draw=none,fill=dodgerblue45123246}
\addlegendentry{P@5}

\draw[draw=none,fill=dodgerblue45123246] (axis cs:0.825,0) rectangle (axis cs:1.175,0.154);
\draw[draw=none,fill=dodgerblue45123246] (axis cs:1.825,0) rectangle (axis cs:2.175,0.213);
\draw[draw=none,fill=dodgerblue45123246] (axis cs:2.825,0) rectangle (axis cs:3.175,0.264);
\draw[draw=none,fill=dodgerblue45123246] (axis cs:3.825,0) rectangle (axis cs:4.175,0.279);
\draw[draw=none,fill=dodgerblue45123246] (axis cs:4.825,0) rectangle (axis cs:5.175,0.295);
\draw[draw=none,fill=dodgerblue45123246] (axis cs:5.825,0) rectangle (axis cs:6.175,0.297);
\draw[draw=none,fill=dodgerblue45123246] (axis cs:6.825,0) rectangle (axis cs:7.175,0.308);
\draw[draw=none,fill=dodgerblue45123246] (axis cs:7.825,0) rectangle (axis cs:8.175,0.31);
\draw[draw=none,fill=dodgerblue45123246] (axis cs:8.825,0) rectangle (axis cs:9.175,0.313);
\draw[draw=none,fill=deepskyblue3176236] (axis cs:-0.175,0) rectangle (axis cs:0.175,0.054);
\addlegendimage{ybar,ybar legend,draw=none,fill=deepskyblue3176236}
\addlegendentry{P@1}

\draw[draw=none,fill=deepskyblue3176236] (axis cs:0.825,0) rectangle (axis cs:1.175,0.126);
\draw[draw=none,fill=deepskyblue3176236] (axis cs:1.825,0) rectangle (axis cs:2.175,0.174);
\draw[draw=none,fill=deepskyblue3176236] (axis cs:2.825,0) rectangle (axis cs:3.175,0.218);
\draw[draw=none,fill=deepskyblue3176236] (axis cs:3.825,0) rectangle (axis cs:4.175,0.233);
\draw[draw=none,fill=deepskyblue3176236] (axis cs:4.825,0) rectangle (axis cs:5.175,0.244);
\draw[draw=none,fill=deepskyblue3176236] (axis cs:5.825,0) rectangle (axis cs:6.175,0.246);
\draw[draw=none,fill=deepskyblue3176236] (axis cs:6.825,0) rectangle (axis cs:7.175,0.256);
\draw[draw=none,fill=deepskyblue3176236] (axis cs:7.825,0) rectangle (axis cs:8.175,0.259);
\draw[draw=none,fill=deepskyblue3176236] (axis cs:8.825,0) rectangle (axis cs:9.175,0.262);
\addplot [red, thick] coordinates {(-0.8925,0.15) (14.8925,0.15)};
\end{axis}

\end{tikzpicture}
}
    \resizebox{.24\textwidth}{!}{% This file was created with tikzplotlib v0.10.1.
\begin{tikzpicture}

\definecolor{darkgray176}{RGB}{176,176,176}
\definecolor{darkviolet1270255}{RGB}{127,0,255}
\definecolor{deepskyblue3176236}{RGB}{3,176,236}
\definecolor{dodgerblue45123246}{RGB}{45,123,246}
\definecolor{royalblue8762253}{RGB}{87,62,253}

\begin{axis}[
tick align=outside,
tick pos=left,
title={\(\displaystyle \Delta=3\) Repair Precision},
legend style={fill opacity=0.8, draw opacity=1, text opacity=1, legend columns=1, legend pos=north west},
x grid style={darkgray176},
xlabel={Seconds},
xmin=-0.6425, xmax=9.6425,
xtick style={color=black},
xtick={0,1,2,3,4,5,6,7,8,9},
xticklabels={6,12,18,24,30,36,42,48,54,60},
y grid style={darkgray176},
ylabel={Precision@k},
ymin=0, ymax=0.5313,
ytick style={color=black}
]
\draw[draw=none,fill=darkviolet1270255] (axis cs:-0.175,0) rectangle (axis cs:0.175,0.013);
\addlegendimage{ybar,ybar legend,draw=none,fill=darkviolet1270255}
\addlegendentry{P@All}

\draw[draw=none,fill=darkviolet1270255] (axis cs:0.825,0) rectangle (axis cs:1.175,0.065);
\draw[draw=none,fill=darkviolet1270255] (axis cs:1.825,0) rectangle (axis cs:2.175,0.156);
\draw[draw=none,fill=darkviolet1270255] (axis cs:2.825,0) rectangle (axis cs:3.175,0.221);
\draw[draw=none,fill=darkviolet1270255] (axis cs:3.825,0) rectangle (axis cs:4.175,0.325);
\draw[draw=none,fill=darkviolet1270255] (axis cs:4.825,0) rectangle (axis cs:5.175,0.377);
\draw[draw=none,fill=darkviolet1270255] (axis cs:5.825,0) rectangle (axis cs:6.175,0.403);
\draw[draw=none,fill=darkviolet1270255] (axis cs:6.825,0) rectangle (axis cs:7.175,0.455);
\draw[draw=none,fill=darkviolet1270255] (axis cs:7.825,0) rectangle (axis cs:8.175,0.468);
\draw[draw=none,fill=darkviolet1270255] (axis cs:8.825,0) rectangle (axis cs:9.175,0.506);
\draw[draw=none,fill=royalblue8762253] (axis cs:-0.175,0) rectangle (axis cs:0.175,0);
\addlegendimage{ybar,ybar legend,draw=none,fill=royalblue8762253}
\addlegendentry{P@10}

\draw[draw=none,fill=royalblue8762253] (axis cs:0.825,0) rectangle (axis cs:1.175,0.013);
\draw[draw=none,fill=royalblue8762253] (axis cs:1.825,0) rectangle (axis cs:2.175,0.052);
\draw[draw=none,fill=royalblue8762253] (axis cs:2.825,0) rectangle (axis cs:3.175,0.065);
\draw[draw=none,fill=royalblue8762253] (axis cs:3.825,0) rectangle (axis cs:4.175,0.104);
\draw[draw=none,fill=royalblue8762253] (axis cs:4.825,0) rectangle (axis cs:5.175,0.117);
\draw[draw=none,fill=royalblue8762253] (axis cs:5.825,0) rectangle (axis cs:6.175,0.13);
\draw[draw=none,fill=royalblue8762253] (axis cs:6.825,0) rectangle (axis cs:7.175,0.143);
\draw[draw=none,fill=royalblue8762253] (axis cs:7.825,0) rectangle (axis cs:8.175,0.156);
\draw[draw=none,fill=royalblue8762253] (axis cs:8.825,0) rectangle (axis cs:9.175,0.156);
\draw[draw=none,fill=dodgerblue45123246] (axis cs:-0.175,0) rectangle (axis cs:0.175,0);
\addlegendimage{ybar,ybar legend,draw=none,fill=dodgerblue45123246}
\addlegendentry{P@5}

\draw[draw=none,fill=dodgerblue45123246] (axis cs:0.825,0) rectangle (axis cs:1.175,0.013);
\draw[draw=none,fill=dodgerblue45123246] (axis cs:1.825,0) rectangle (axis cs:2.175,0.052);
\draw[draw=none,fill=dodgerblue45123246] (axis cs:2.825,0) rectangle (axis cs:3.175,0.065);
\draw[draw=none,fill=dodgerblue45123246] (axis cs:3.825,0) rectangle (axis cs:4.175,0.104);
\draw[draw=none,fill=dodgerblue45123246] (axis cs:4.825,0) rectangle (axis cs:5.175,0.104);
\draw[draw=none,fill=dodgerblue45123246] (axis cs:5.825,0) rectangle (axis cs:6.175,0.117);
\draw[draw=none,fill=dodgerblue45123246] (axis cs:6.825,0) rectangle (axis cs:7.175,0.13);
\draw[draw=none,fill=dodgerblue45123246] (axis cs:7.825,0) rectangle (axis cs:8.175,0.143);
\draw[draw=none,fill=dodgerblue45123246] (axis cs:8.825,0) rectangle (axis cs:9.175,0.143);
\draw[draw=none,fill=deepskyblue3176236] (axis cs:-0.175,0) rectangle (axis cs:0.175,0);
\addlegendimage{ybar,ybar legend,draw=none,fill=deepskyblue3176236}
\addlegendentry{P@1}

\draw[draw=none,fill=deepskyblue3176236] (axis cs:0.825,0) rectangle (axis cs:1.175,0.013);
\draw[draw=none,fill=deepskyblue3176236] (axis cs:1.825,0) rectangle (axis cs:2.175,0.052);
\draw[draw=none,fill=deepskyblue3176236] (axis cs:2.825,0) rectangle (axis cs:3.175,0.065);
\draw[draw=none,fill=deepskyblue3176236] (axis cs:3.825,0) rectangle (axis cs:4.175,0.091);
\draw[draw=none,fill=deepskyblue3176236] (axis cs:4.825,0) rectangle (axis cs:5.175,0.091);
\draw[draw=none,fill=deepskyblue3176236] (axis cs:5.825,0) rectangle (axis cs:6.175,0.104);
\draw[draw=none,fill=deepskyblue3176236] (axis cs:6.825,0) rectangle (axis cs:7.175,0.117);
\draw[draw=none,fill=deepskyblue3176236] (axis cs:7.825,0) rectangle (axis cs:8.175,0.13);
\draw[draw=none,fill=deepskyblue3176236] (axis cs:8.825,0) rectangle (axis cs:9.175,0.13);
\addplot [red, thick] coordinates {(-0.8925,0.07) (14.8925,0.07)};
\end{axis}

\end{tikzpicture}
}
    \resizebox{.24\textwidth}{!}{% This file was created with tikzplotlib v0.10.1.
\begin{tikzpicture}
\begin{axis}[
tick align=outside,
tick pos=left,
axis lines=left,
title={\textbf{\Large Repair Precision \(\mathbf{(\bm\Delta=4)}\)}},
legend style={fill opacity=0.8, draw opacity=1, text opacity=1, legend columns=1, legend pos=north west},
x grid style={darkgray176},
xlabel={Seconds},
xmin=-0.5925, xmax=8.5925,
xtick style={color=black},
xtick={0,1,2,3,4,5,6,7,8},
xticklabels={10,20,30,40,50,60,70,80,90},
y grid style={darkgray176},
ylabel={Precision@k},
ymin=0, ymax=0.54915,
ytick style={color=black}
]
\draw[draw=none,fill=darkviolet1270255] (axis cs:-0.175,0) rectangle (axis cs:0.175,0.123);

\draw[draw=none,fill=darkviolet1270255] (axis cs:0.825,0) rectangle (axis cs:1.175,0.323);
\draw[draw=none,fill=darkviolet1270255] (axis cs:1.825,0) rectangle (axis cs:2.175,0.415);
\draw[draw=none,fill=darkviolet1270255] (axis cs:2.825,0) rectangle (axis cs:3.175,0.431);
\draw[draw=none,fill=darkviolet1270255] (axis cs:3.825,0) rectangle (axis cs:4.175,0.477);
\draw[draw=none,fill=darkviolet1270255] (axis cs:4.825,0) rectangle (axis cs:5.175,0.508);
\draw[draw=none,fill=darkviolet1270255] (axis cs:5.825,0) rectangle (axis cs:6.175,0.523);
\draw[draw=none,fill=darkviolet1270255] (axis cs:6.825,0) rectangle (axis cs:7.175,0.523);
\draw[draw=none,fill=darkviolet1270255] (axis cs:7.825,0) rectangle (axis cs:8.175,0.523);
\draw[draw=none,fill=royalblue8762253] (axis cs:-0.175,0) rectangle (axis cs:0.175,0.092);

\draw[draw=none,fill=royalblue8762253] (axis cs:0.825,0) rectangle (axis cs:1.175,0.169);
\draw[draw=none,fill=royalblue8762253] (axis cs:1.825,0) rectangle (axis cs:2.175,0.215);
\draw[draw=none,fill=royalblue8762253] (axis cs:2.825,0) rectangle (axis cs:3.175,0.231);
\draw[draw=none,fill=royalblue8762253] (axis cs:3.825,0) rectangle (axis cs:4.175,0.277);
\draw[draw=none,fill=royalblue8762253] (axis cs:4.825,0) rectangle (axis cs:5.175,0.308);
\draw[draw=none,fill=royalblue8762253] (axis cs:5.825,0) rectangle (axis cs:6.175,0.323);
\draw[draw=none,fill=royalblue8762253] (axis cs:6.825,0) rectangle (axis cs:7.175,0.323);
\draw[draw=none,fill=royalblue8762253] (axis cs:7.825,0) rectangle (axis cs:8.175,0.323);
\draw[draw=none,fill=dodgerblue45123246] (axis cs:-0.175,0) rectangle (axis cs:0.175,0.092);

\draw[draw=none,fill=dodgerblue45123246] (axis cs:0.825,0) rectangle (axis cs:1.175,0.169);
\draw[draw=none,fill=dodgerblue45123246] (axis cs:1.825,0) rectangle (axis cs:2.175,0.215);
\draw[draw=none,fill=dodgerblue45123246] (axis cs:2.825,0) rectangle (axis cs:3.175,0.231);
\draw[draw=none,fill=dodgerblue45123246] (axis cs:3.825,0) rectangle (axis cs:4.175,0.277);
\draw[draw=none,fill=dodgerblue45123246] (axis cs:4.825,0) rectangle (axis cs:5.175,0.308);
\draw[draw=none,fill=dodgerblue45123246] (axis cs:5.825,0) rectangle (axis cs:6.175,0.323);
\draw[draw=none,fill=dodgerblue45123246] (axis cs:6.825,0) rectangle (axis cs:7.175,0.323);
\draw[draw=none,fill=dodgerblue45123246] (axis cs:7.825,0) rectangle (axis cs:8.175,0.323);
\draw[draw=none,fill=deepskyblue3176236] (axis cs:-0.175,0) rectangle (axis cs:0.175,0.077);

\draw[draw=none,fill=deepskyblue3176236] (axis cs:0.825,0) rectangle (axis cs:1.175,0.138);
\draw[draw=none,fill=deepskyblue3176236] (axis cs:1.825,0) rectangle (axis cs:2.175,0.185);
\draw[draw=none,fill=deepskyblue3176236] (axis cs:2.825,0) rectangle (axis cs:3.175,0.2);
\draw[draw=none,fill=deepskyblue3176236] (axis cs:3.825,0) rectangle (axis cs:4.175,0.246);
\draw[draw=none,fill=deepskyblue3176236] (axis cs:4.825,0) rectangle (axis cs:5.175,0.277);
\draw[draw=none,fill=deepskyblue3176236] (axis cs:5.825,0) rectangle (axis cs:6.175,0.292);
\draw[draw=none,fill=deepskyblue3176236] (axis cs:6.825,0) rectangle (axis cs:7.175,0.292);
\draw[draw=none,fill=deepskyblue3176236] (axis cs:7.825,0) rectangle (axis cs:8.175,0.292);
\addplot [red, thick] coordinates {(-0.8925,0.10) (14.8925,0.10)};
\addplot [orange, thick] coordinates {(-0.8925,0.01) (14.8925,0.01)};
\end{axis}

\end{tikzpicture}
}
%    \resizebox{.24\textwidth}{!}{% This file was created with tikzplotlib v0.10.1.
\begin{tikzpicture}
\begin{axis}[
tick align=outside,
tick pos=left,
axis lines=left,
title={\Large\(\displaystyle \Delta=5\) Repair Precision},
legend style={fill opacity=0.8, draw opacity=1, text opacity=1, legend columns=1, legend pos=north west},
x grid style={darkgray176},
xlabel={Seconds},
xmin=-0.5925, xmax=8.5925,
xtick={0,1,2,3,4,5,6,7,8},
xticklabels={10,20,30,40,50,60,70,80,90},
y grid style={darkgray176},
ylabel={Precision@k},
ymin=0, ymax=0.54915,
]
\draw[draw=none,fill=darkviolet1270255] (axis cs:-0.175,0) rectangle (axis cs:0.175,0.123);
\addlegendimage{ybar,ybar legend,draw=none,fill=darkviolet1270255}
\addlegendentry{P@All}

\draw[draw=none,fill=darkviolet1270255] (axis cs:0.825,0) rectangle (axis cs:1.175,0.323);
\draw[draw=none,fill=darkviolet1270255] (axis cs:1.825,0) rectangle (axis cs:2.175,0.415);
\draw[draw=none,fill=darkviolet1270255] (axis cs:2.825,0) rectangle (axis cs:3.175,0.431);
\draw[draw=none,fill=darkviolet1270255] (axis cs:3.825,0) rectangle (axis cs:4.175,0.477);
\draw[draw=none,fill=darkviolet1270255] (axis cs:4.825,0) rectangle (axis cs:5.175,0.508);
\draw[draw=none,fill=darkviolet1270255] (axis cs:5.825,0) rectangle (axis cs:6.175,0.523);
\draw[draw=none,fill=darkviolet1270255] (axis cs:6.825,0) rectangle (axis cs:7.175,0.523);
\draw[draw=none,fill=darkviolet1270255] (axis cs:7.825,0) rectangle (axis cs:8.175,0.523);
\draw[draw=none,fill=royalblue8762253] (axis cs:-0.175,0) rectangle (axis cs:0.175,0.092);
\addlegendimage{ybar,ybar legend,draw=none,fill=royalblue8762253}
\addlegendentry{P@10}

\draw[draw=none,fill=royalblue8762253] (axis cs:0.825,0) rectangle (axis cs:1.175,0.169);
\draw[draw=none,fill=royalblue8762253] (axis cs:1.825,0) rectangle (axis cs:2.175,0.215);
\draw[draw=none,fill=royalblue8762253] (axis cs:2.825,0) rectangle (axis cs:3.175,0.231);
\draw[draw=none,fill=royalblue8762253] (axis cs:3.825,0) rectangle (axis cs:4.175,0.277);
\draw[draw=none,fill=royalblue8762253] (axis cs:4.825,0) rectangle (axis cs:5.175,0.308);
\draw[draw=none,fill=royalblue8762253] (axis cs:5.825,0) rectangle (axis cs:6.175,0.323);
\draw[draw=none,fill=royalblue8762253] (axis cs:6.825,0) rectangle (axis cs:7.175,0.323);
\draw[draw=none,fill=royalblue8762253] (axis cs:7.825,0) rectangle (axis cs:8.175,0.323);
\draw[draw=none,fill=dodgerblue45123246] (axis cs:-0.175,0) rectangle (axis cs:0.175,0.092);
\addlegendimage{ybar,ybar legend,draw=none,fill=dodgerblue45123246}
\addlegendentry{P@5}

\draw[draw=none,fill=dodgerblue45123246] (axis cs:0.825,0) rectangle (axis cs:1.175,0.169);
\draw[draw=none,fill=dodgerblue45123246] (axis cs:1.825,0) rectangle (axis cs:2.175,0.215);
\draw[draw=none,fill=dodgerblue45123246] (axis cs:2.825,0) rectangle (axis cs:3.175,0.231);
\draw[draw=none,fill=dodgerblue45123246] (axis cs:3.825,0) rectangle (axis cs:4.175,0.277);
\draw[draw=none,fill=dodgerblue45123246] (axis cs:4.825,0) rectangle (axis cs:5.175,0.308);
\draw[draw=none,fill=dodgerblue45123246] (axis cs:5.825,0) rectangle (axis cs:6.175,0.323);
\draw[draw=none,fill=dodgerblue45123246] (axis cs:6.825,0) rectangle (axis cs:7.175,0.323);
\draw[draw=none,fill=dodgerblue45123246] (axis cs:7.825,0) rectangle (axis cs:8.175,0.323);
\draw[draw=none,fill=deepskyblue3176236] (axis cs:-0.175,0) rectangle (axis cs:0.175,0.077);
\addlegendimage{ybar,ybar legend,draw=none,fill=deepskyblue3176236}
\addlegendentry{P@1}

\draw[draw=none,fill=deepskyblue3176236] (axis cs:0.825,0) rectangle (axis cs:1.175,0.138);
\draw[draw=none,fill=deepskyblue3176236] (axis cs:1.825,0) rectangle (axis cs:2.175,0.185);
\draw[draw=none,fill=deepskyblue3176236] (axis cs:2.825,0) rectangle (axis cs:3.175,0.2);
\draw[draw=none,fill=deepskyblue3176236] (axis cs:3.825,0) rectangle (axis cs:4.175,0.246);
\draw[draw=none,fill=deepskyblue3176236] (axis cs:4.825,0) rectangle (axis cs:5.175,0.277);
\draw[draw=none,fill=deepskyblue3176236] (axis cs:5.825,0) rectangle (axis cs:6.175,0.292);
\draw[draw=none,fill=deepskyblue3176236] (axis cs:6.825,0) rectangle (axis cs:7.175,0.292);
\draw[draw=none,fill=deepskyblue3176236] (axis cs:7.825,0) rectangle (axis cs:8.175,0.292);
\addplot [red, thick] coordinates {(-0.8925,0.10) (14.8925,0.10)};
\end{axis}

\end{tikzpicture}
}
%\resizebox{.3\textwidth}{!}{% This file was created with tikzplotlib v0.10.1.
\begin{tikzpicture}

\definecolor{darkgray176}{RGB}{176,176,176}
\definecolor{darkviolet1270255}{RGB}{127,0,255}
\definecolor{deepskyblue3176236}{RGB}{3,176,236}
\definecolor{dodgerblue45123246}{RGB}{45,123,246}
\definecolor{lightgray204}{RGB}{204,204,204}
\definecolor{royalblue8762253}{RGB}{87,62,253}

\begin{axis}[
legend cell align={left},
legend style={fill opacity=0.8, draw opacity=1, text opacity=1, draw=lightgray204, legend columns=-1, legend pos=north west},
tick align=outside,
tick pos=left,
title={$\Delta=1$ Repair Precision},
x grid style={darkgray176},
xlabel={Seconds},
xmin=-0.4925, xmax=6.4925,
xtick style={color=black},
xtick={0,1,2,3,4,5,6},
xticklabels={20,60,100,140,180,220,260},
y grid style={darkgray176},
ylabel={\phantom{Precision@k}},
ymin=0, ymax=0.77595,
ytick style={color=black}
]
\addlegendimage{empty legend}
\addlegendentry{P@}
\draw[draw=none,fill=darkviolet1270255] (axis cs:-0.175,0) rectangle (axis cs:0.175,0.145);
\addlegendimage{ybar,ybar legend,draw=none,fill=darkviolet1270255}
\addlegendentry{All}

\draw[draw=none,fill=darkviolet1270255] (axis cs:0.825,0) rectangle (axis cs:1.175,0.304);
\draw[draw=none,fill=darkviolet1270255] (axis cs:1.825,0) rectangle (axis cs:2.175,0.42);
\draw[draw=none,fill=darkviolet1270255] (axis cs:2.825,0) rectangle (axis cs:3.175,0.507);
\draw[draw=none,fill=darkviolet1270255] (axis cs:3.825,0) rectangle (axis cs:4.175,0.594);
\draw[draw=none,fill=darkviolet1270255] (axis cs:4.825,0) rectangle (axis cs:5.175,0.71);
\draw[draw=none,fill=darkviolet1270255] (axis cs:5.825,0) rectangle (axis cs:6.175,0.739);
\draw[draw=none,fill=royalblue8762253] (axis cs:-0.175,0) rectangle (axis cs:0.175,0.145);
\addlegendimage{ybar,ybar legend,draw=none,fill=royalblue8762253}
\addlegendentry{10}

\draw[draw=none,fill=royalblue8762253] (axis cs:0.825,0) rectangle (axis cs:1.175,0.304);
\draw[draw=none,fill=royalblue8762253] (axis cs:1.825,0) rectangle (axis cs:2.175,0.42);
\draw[draw=none,fill=royalblue8762253] (axis cs:2.825,0) rectangle (axis cs:3.175,0.507);
\draw[draw=none,fill=royalblue8762253] (axis cs:3.825,0) rectangle (axis cs:4.175,0.594);
\draw[draw=none,fill=royalblue8762253] (axis cs:4.825,0) rectangle (axis cs:5.175,0.703);
\draw[draw=none,fill=royalblue8762253] (axis cs:5.825,0) rectangle (axis cs:6.175,0.732);
\draw[draw=none,fill=dodgerblue45123246] (axis cs:-0.175,0) rectangle (axis cs:0.175,0.145);
\addlegendimage{ybar,ybar legend,draw=none,fill=dodgerblue45123246}
\addlegendentry{5}

\draw[draw=none,fill=dodgerblue45123246] (axis cs:0.825,0) rectangle (axis cs:1.175,0.304);
\draw[draw=none,fill=dodgerblue45123246] (axis cs:1.825,0) rectangle (axis cs:2.175,0.42);
\draw[draw=none,fill=dodgerblue45123246] (axis cs:2.825,0) rectangle (axis cs:3.175,0.493);
\draw[draw=none,fill=dodgerblue45123246] (axis cs:3.825,0) rectangle (axis cs:4.175,0.565);
\draw[draw=none,fill=dodgerblue45123246] (axis cs:4.825,0) rectangle (axis cs:5.175,0.638);
\draw[draw=none,fill=dodgerblue45123246] (axis cs:5.825,0) rectangle (axis cs:6.175,0.667);
\draw[draw=none,fill=deepskyblue3176236] (axis cs:-0.175,0) rectangle (axis cs:0.175,0.116);
\addlegendimage{ybar,ybar legend,draw=none,fill=deepskyblue3176236}
\addlegendentry{1}

\draw[draw=none,fill=deepskyblue3176236] (axis cs:0.825,0) rectangle (axis cs:1.175,0.203);
\draw[draw=none,fill=deepskyblue3176236] (axis cs:1.825,0) rectangle (axis cs:2.175,0.268);
\draw[draw=none,fill=deepskyblue3176236] (axis cs:2.825,0) rectangle (axis cs:3.175,0.312);
\draw[draw=none,fill=deepskyblue3176236] (axis cs:3.825,0) rectangle (axis cs:4.175,0.355);
\draw[draw=none,fill=deepskyblue3176236] (axis cs:4.825,0) rectangle (axis cs:5.175,0.428);
\draw[draw=none,fill=deepskyblue3176236] (axis cs:5.825,0) rectangle (axis cs:6.175,0.442);
\end{axis}

\end{tikzpicture}
}
%\resizebox{.307\textwidth}{!}{% This file was created with tikzplotlib v0.10.1.
\begin{tikzpicture}

\definecolor{darkgray176}{RGB}{176,176,176}
\definecolor{darkviolet1270255}{RGB}{127,0,255}
\definecolor{deepskyblue3176236}{RGB}{3,176,236}
\definecolor{dodgerblue45123246}{RGB}{45,123,246}
\definecolor{lightgray204}{RGB}{204,204,204}
\definecolor{royalblue8762253}{RGB}{87,62,253}

\begin{axis}[
legend cell align={left},
legend style={fill opacity=0.8, draw opacity=1, text opacity=1, draw=lightgray204, legend columns=-1, legend pos=north west},
tick align=outside,
tick pos=left,
title={$\Delta=2$ Repair Precision},
x grid style={darkgray176},
xlabel={Seconds},
xmin=-0.4925, xmax=6.4925,
xtick style={color=black},
xtick={0,1,2,3,4,5,6},
xticklabels={20,60,100,140,180,220,260},
y grid style={darkgray176},
ylabel={\phantom{Precision@k}},
ymin=0, ymax=0.40635,
ytick style={color=black}
]
\addlegendimage{empty legend}
\addlegendentry{P@}
\draw[draw=none,fill=darkviolet1270255] (axis cs:-0.175,0) rectangle (axis cs:0.175,0.048);
\addlegendimage{ybar,ybar legend,draw=none,fill=darkviolet1270255}
\addlegendentry{All}

\draw[draw=none,fill=darkviolet1270255] (axis cs:0.825,0) rectangle (axis cs:1.175,0.081);
\draw[draw=none,fill=darkviolet1270255] (axis cs:1.825,0) rectangle (axis cs:2.175,0.177);
\draw[draw=none,fill=darkviolet1270255] (axis cs:2.825,0) rectangle (axis cs:3.175,0.274);
\draw[draw=none,fill=darkviolet1270255] (axis cs:3.825,0) rectangle (axis cs:4.175,0.306);
\draw[draw=none,fill=darkviolet1270255] (axis cs:4.825,0) rectangle (axis cs:5.175,0.355);
\draw[draw=none,fill=darkviolet1270255] (axis cs:5.825,0) rectangle (axis cs:6.175,0.387);
\draw[draw=none,fill=royalblue8762253] (axis cs:-0.175,0) rectangle (axis cs:0.175,0.048);
\addlegendimage{ybar,ybar legend,draw=none,fill=royalblue8762253}
\addlegendentry{10}

\draw[draw=none,fill=royalblue8762253] (axis cs:0.825,0) rectangle (axis cs:1.175,0.081);
\draw[draw=none,fill=royalblue8762253] (axis cs:1.825,0) rectangle (axis cs:2.175,0.113);
\draw[draw=none,fill=royalblue8762253] (axis cs:2.825,0) rectangle (axis cs:3.175,0.21);
\draw[draw=none,fill=royalblue8762253] (axis cs:3.825,0) rectangle (axis cs:4.175,0.226);
\draw[draw=none,fill=royalblue8762253] (axis cs:4.825,0) rectangle (axis cs:5.175,0.258);
\draw[draw=none,fill=royalblue8762253] (axis cs:5.825,0) rectangle (axis cs:6.175,0.242);
\draw[draw=none,fill=dodgerblue45123246] (axis cs:-0.175,0) rectangle (axis cs:0.175,0.048);
\addlegendimage{ybar,ybar legend,draw=none,fill=dodgerblue45123246}
\addlegendentry{5}

\draw[draw=none,fill=dodgerblue45123246] (axis cs:0.825,0) rectangle (axis cs:1.175,0.032);
\draw[draw=none,fill=dodgerblue45123246] (axis cs:1.825,0) rectangle (axis cs:2.175,0.065);
\draw[draw=none,fill=dodgerblue45123246] (axis cs:2.825,0) rectangle (axis cs:3.175,0.129);
\draw[draw=none,fill=dodgerblue45123246] (axis cs:3.825,0) rectangle (axis cs:4.175,0.113);
\draw[draw=none,fill=dodgerblue45123246] (axis cs:4.825,0) rectangle (axis cs:5.175,0.129);
\draw[draw=none,fill=dodgerblue45123246] (axis cs:5.825,0) rectangle (axis cs:6.175,0.113);
\draw[draw=none,fill=deepskyblue3176236] (axis cs:-0.175,0) rectangle (axis cs:0.175,0.016);
\addlegendimage{ybar,ybar legend,draw=none,fill=deepskyblue3176236}
\addlegendentry{1}

\draw[draw=none,fill=deepskyblue3176236] (axis cs:0.825,0) rectangle (axis cs:1.175,0);
\draw[draw=none,fill=deepskyblue3176236] (axis cs:1.825,0) rectangle (axis cs:2.175,0.016);
\draw[draw=none,fill=deepskyblue3176236] (axis cs:2.825,0) rectangle (axis cs:3.175,0.065);
\draw[draw=none,fill=deepskyblue3176236] (axis cs:3.825,0) rectangle (axis cs:4.175,0.048);
\draw[draw=none,fill=deepskyblue3176236] (axis cs:4.825,0) rectangle (axis cs:5.175,0.032);
\draw[draw=none,fill=deepskyblue3176236] (axis cs:5.825,0) rectangle (axis cs:6.175,0.032);
\end{axis}

\end{tikzpicture}
}
    \caption{Human repair benchmark. Note the y-axis across different edit distance plots has varying ranges.}\label{fig:human}
  \end{figure}

%  In the following benchmark, we measure the precision@k of our repair procedure against human repairs of varying edit distances and latency cutoffs, using an adaptive resampling procedure described in \S\ref{sec:adaptive}. This sampler maintains a buffer of successful repairs ranked by perplexity and uses stochastic local search to resample edits within a neighborhood. Initially, edits are sampled uniformly at random. Over time and as the admissible set grows, it prioritizes edits nearby low-perplexity repairs. This technique offers a significant advantage in the low-latency setting.

  Previously, we used a rejection-based sampler, which did not sample directly from the admissible set, but the entire Levenshtein ball, then rejected invalid samples. Although rejection sampling has a much lower minimum latency threshold, the average time required to attain a fixed precision is much higher. We present the results from the earlier evaluation for comparison below.

  \begin{figure}[H]
    \resizebox{.24\textwidth}{!}{% This file was created with tikzplotlib v0.10.1.
\begin{tikzpicture}

\definecolor{darkgray176}{RGB}{176,176,176}
\definecolor{darkviolet1270255}{RGB}{127,0,255}
\definecolor{deepskyblue3176236}{RGB}{3,176,236}
\definecolor{dodgerblue45123246}{RGB}{45,123,246}
\definecolor{lightgray204}{RGB}{204,204,204}
\definecolor{royalblue8762253}{RGB}{87,62,253}

\begin{axis}[
legend cell align={left},
legend style={fill opacity=0.8, draw opacity=1, text opacity=1, draw=lightgray204, legend columns=-1, legend pos=north west},
tick align=outside,
tick pos=left,
title={\(\displaystyle \Delta\in[1,3]\) Repair Precision},
x grid style={darkgray176},
xlabel={Seconds},
xmin=-0.5925, xmax=8.5925,
ymin=0.0, ymax=0.7,
xtick style={color=black},
xtick={0,1,2,3,4,5,6,7,8},
xticklabels={0.1,0.3,0.5,0.7,0.9,1.1,1.3,1.5,1.7},
y grid style={darkgray176},
ylabel={Precision@k},
ytick style={color=black},
width=0.5\textwidth,
height=0.48\textwidth,
]
\addlegendimage{empty legend}
\addlegendentry{P@}
\draw[draw=none,fill=darkviolet1270255] (axis cs:-0.175,0) rectangle (axis cs:0.175,0.245);
\addlegendimage{ybar,ybar legend,draw=none,fill=darkviolet1270255}
\addlegendentry{All}

\draw[draw=none,fill=darkviolet1270255] (axis cs:0.825,0) rectangle (axis cs:1.175,0.417);
\draw[draw=none,fill=darkviolet1270255] (axis cs:1.825,0) rectangle (axis cs:2.175,0.496);
\draw[draw=none,fill=darkviolet1270255] (axis cs:2.825,0) rectangle (axis cs:3.175,0.535);
\draw[draw=none,fill=darkviolet1270255] (axis cs:3.825,0) rectangle (axis cs:4.175,0.562);
\draw[draw=none,fill=darkviolet1270255] (axis cs:4.825,0) rectangle (axis cs:5.175,0.584);
\draw[draw=none,fill=darkviolet1270255] (axis cs:5.825,0) rectangle (axis cs:6.175,0.603);
\draw[draw=none,fill=darkviolet1270255] (axis cs:6.825,0) rectangle (axis cs:7.175,0.619);
\draw[draw=none,fill=darkviolet1270255] (axis cs:7.825,0) rectangle (axis cs:8.175,0.629);
\draw[draw=none,fill=royalblue8762253] (axis cs:-0.175,0) rectangle (axis cs:0.175,0.234);
\addlegendimage{ybar,ybar legend,draw=none,fill=royalblue8762253}
\addlegendentry{10}

\draw[draw=none,fill=royalblue8762253] (axis cs:0.825,0) rectangle (axis cs:1.175,0.387);
\draw[draw=none,fill=royalblue8762253] (axis cs:1.825,0) rectangle (axis cs:2.175,0.451);
\draw[draw=none,fill=royalblue8762253] (axis cs:2.825,0) rectangle (axis cs:3.175,0.481);
\draw[draw=none,fill=royalblue8762253] (axis cs:3.825,0) rectangle (axis cs:4.175,0.502);
\draw[draw=none,fill=royalblue8762253] (axis cs:4.825,0) rectangle (axis cs:5.175,0.517);
\draw[draw=none,fill=royalblue8762253] (axis cs:5.825,0) rectangle (axis cs:6.175,0.53);
\draw[draw=none,fill=royalblue8762253] (axis cs:6.825,0) rectangle (axis cs:7.175,0.54);
\draw[draw=none,fill=royalblue8762253] (axis cs:7.825,0) rectangle (axis cs:8.175,0.546);
\draw[draw=none,fill=dodgerblue45123246] (axis cs:-0.175,0) rectangle (axis cs:0.175,0.218);
\addlegendimage{ybar,ybar legend,draw=none,fill=dodgerblue45123246}
\addlegendentry{5}

\draw[draw=none,fill=dodgerblue45123246] (axis cs:0.825,0) rectangle (axis cs:1.175,0.353);
\draw[draw=none,fill=dodgerblue45123246] (axis cs:1.825,0) rectangle (axis cs:2.175,0.406);
\draw[draw=none,fill=dodgerblue45123246] (axis cs:2.825,0) rectangle (axis cs:3.175,0.43);
\draw[draw=none,fill=dodgerblue45123246] (axis cs:3.825,0) rectangle (axis cs:4.175,0.444);
\draw[draw=none,fill=dodgerblue45123246] (axis cs:4.825,0) rectangle (axis cs:5.175,0.455);
\draw[draw=none,fill=dodgerblue45123246] (axis cs:5.825,0) rectangle (axis cs:6.175,0.467);
\draw[draw=none,fill=dodgerblue45123246] (axis cs:6.825,0) rectangle (axis cs:7.175,0.474);
\draw[draw=none,fill=dodgerblue45123246] (axis cs:7.825,0) rectangle (axis cs:8.175,0.479);
\draw[draw=none,fill=deepskyblue3176236] (axis cs:-0.175,0) rectangle (axis cs:0.175,0.117);
\addlegendimage{ybar,ybar legend,draw=none,fill=deepskyblue3176236}
\addlegendentry{1}

\draw[draw=none,fill=deepskyblue3176236] (axis cs:0.825,0) rectangle (axis cs:1.175,0.187);
\draw[draw=none,fill=deepskyblue3176236] (axis cs:1.825,0) rectangle (axis cs:2.175,0.219);
\draw[draw=none,fill=deepskyblue3176236] (axis cs:2.825,0) rectangle (axis cs:3.175,0.233);
\draw[draw=none,fill=deepskyblue3176236] (axis cs:3.825,0) rectangle (axis cs:4.175,0.244);
\draw[draw=none,fill=deepskyblue3176236] (axis cs:4.825,0) rectangle (axis cs:5.175,0.25);
\draw[draw=none,fill=deepskyblue3176236] (axis cs:5.825,0) rectangle (axis cs:6.175,0.258);
\draw[draw=none,fill=deepskyblue3176236] (axis cs:6.825,0) rectangle (axis cs:7.175,0.262);
\draw[draw=none,fill=deepskyblue3176236] (axis cs:7.825,0) rectangle (axis cs:8.175,0.265);
\addplot [red, thick] coordinates {(-0.8925,0.35) (14.8925,0.35)};
\end{axis}

\end{tikzpicture}
}
    \resizebox{.25\textwidth}{!}{% This file was created with tikzplotlib v0.10.1.
\begin{tikzpicture}

\definecolor{darkgray176}{RGB}{176,176,176}
\definecolor{darkviolet1270255}{RGB}{127,0,255}
\definecolor{deepskyblue3176236}{RGB}{3,176,236}
\definecolor{dodgerblue45123246}{RGB}{45,123,246}
\definecolor{lightgray204}{RGB}{204,204,204}
\definecolor{royalblue8762253}{RGB}{87,62,253}

\begin{axis}[
legend cell align={left},
legend style={fill opacity=0.8, draw opacity=1, text opacity=1, draw=lightgray204, legend columns=-1, legend pos=north west},
tick align=outside,
tick pos=left,
title={$\Delta=1$ Repair Precision},
x grid style={darkgray176},
xlabel={Seconds},
xmin=-0.5925, xmax=8.5925,
xtick style={color=black},
xtick={0,1,2,3,4,5,6,7,8},
xticklabels={0.1,0.3,0.5,0.7,0.9,1.1,1.3,1.5,1.7},
y grid style={darkgray176},
ymin=0, ymax=1.0,
ytick style={color=black}
width=0.43\textwidth,
height=0.49\textwidth,
]
\addlegendimage{empty legend}
\addlegendentry{P@}
\draw[draw=none,fill=darkviolet1270255] (axis cs:-0.175,0) rectangle (axis cs:0.175,0.416);
\addlegendimage{ybar,ybar legend,draw=none,fill=darkviolet1270255}
\addlegendentry{All}

\draw[draw=none,fill=darkviolet1270255] (axis cs:0.825,0) rectangle (axis cs:1.175,0.672);
\draw[draw=none,fill=darkviolet1270255] (axis cs:1.825,0) rectangle (axis cs:2.175,0.773);
\draw[draw=none,fill=darkviolet1270255] (axis cs:2.825,0) rectangle (axis cs:3.175,0.822);
\draw[draw=none,fill=darkviolet1270255] (axis cs:3.825,0) rectangle (axis cs:4.175,0.849);
\draw[draw=none,fill=darkviolet1270255] (axis cs:4.825,0) rectangle (axis cs:5.175,0.873);
\draw[draw=none,fill=darkviolet1270255] (axis cs:5.825,0) rectangle (axis cs:6.175,0.891);
\draw[draw=none,fill=darkviolet1270255] (axis cs:6.825,0) rectangle (axis cs:7.175,0.904);
\draw[draw=none,fill=darkviolet1270255] (axis cs:7.825,0) rectangle (axis cs:8.175,0.912);
\draw[draw=none,fill=royalblue8762253] (axis cs:-0.175,0) rectangle (axis cs:0.175,0.402);
\addlegendimage{ybar,ybar legend,draw=none,fill=royalblue8762253}
\addlegendentry{10}

\draw[draw=none,fill=royalblue8762253] (axis cs:0.825,0) rectangle (axis cs:1.175,0.645);
\draw[draw=none,fill=royalblue8762253] (axis cs:1.825,0) rectangle (axis cs:2.175,0.741);
\draw[draw=none,fill=royalblue8762253] (axis cs:2.825,0) rectangle (axis cs:3.175,0.786);
\draw[draw=none,fill=royalblue8762253] (axis cs:3.825,0) rectangle (axis cs:4.175,0.812);
\draw[draw=none,fill=royalblue8762253] (axis cs:4.825,0) rectangle (axis cs:5.175,0.834);
\draw[draw=none,fill=royalblue8762253] (axis cs:5.825,0) rectangle (axis cs:6.175,0.851);
\draw[draw=none,fill=royalblue8762253] (axis cs:6.825,0) rectangle (axis cs:7.175,0.862);
\draw[draw=none,fill=royalblue8762253] (axis cs:7.825,0) rectangle (axis cs:8.175,0.87);
\draw[draw=none,fill=dodgerblue45123246] (axis cs:-0.175,0) rectangle (axis cs:0.175,0.379);
\addlegendimage{ybar,ybar legend,draw=none,fill=dodgerblue45123246}
\addlegendentry{5}

\draw[draw=none,fill=dodgerblue45123246] (axis cs:0.825,0) rectangle (axis cs:1.175,0.601);
\draw[draw=none,fill=dodgerblue45123246] (axis cs:1.825,0) rectangle (axis cs:2.175,0.686);
\draw[draw=none,fill=dodgerblue45123246] (axis cs:2.825,0) rectangle (axis cs:3.175,0.724);
\draw[draw=none,fill=dodgerblue45123246] (axis cs:3.825,0) rectangle (axis cs:4.175,0.743);
\draw[draw=none,fill=dodgerblue45123246] (axis cs:4.825,0) rectangle (axis cs:5.175,0.762);
\draw[draw=none,fill=dodgerblue45123246] (axis cs:5.825,0) rectangle (axis cs:6.175,0.777);
\draw[draw=none,fill=dodgerblue45123246] (axis cs:6.825,0) rectangle (axis cs:7.175,0.785);
\draw[draw=none,fill=dodgerblue45123246] (axis cs:7.825,0) rectangle (axis cs:8.175,0.791);
\draw[draw=none,fill=deepskyblue3176236] (axis cs:-0.175,0) rectangle (axis cs:0.175,0.211);
\addlegendimage{ybar,ybar legend,draw=none,fill=deepskyblue3176236}
\addlegendentry{1}

\draw[draw=none,fill=deepskyblue3176236] (axis cs:0.825,0) rectangle (axis cs:1.175,0.327);
\draw[draw=none,fill=deepskyblue3176236] (axis cs:1.825,0) rectangle (axis cs:2.175,0.378);
\draw[draw=none,fill=deepskyblue3176236] (axis cs:2.825,0) rectangle (axis cs:3.175,0.401);
\draw[draw=none,fill=deepskyblue3176236] (axis cs:3.825,0) rectangle (axis cs:4.175,0.418);
\draw[draw=none,fill=deepskyblue3176236] (axis cs:4.825,0) rectangle (axis cs:5.175,0.428);
\draw[draw=none,fill=deepskyblue3176236] (axis cs:5.825,0) rectangle (axis cs:6.175,0.437);
\draw[draw=none,fill=deepskyblue3176236] (axis cs:6.825,0) rectangle (axis cs:7.175,0.442);
\draw[draw=none,fill=deepskyblue3176236] (axis cs:7.825,0) rectangle (axis cs:8.175,0.446);
\addplot [red, thick] coordinates {(-0.8925,0.45) (14.8925,0.45)};
\end{axis}

\end{tikzpicture}
}
    \resizebox{.24\textwidth}{!}{% This file was created with tikzplotlib v0.10.1.
\begin{tikzpicture}

\definecolor{darkgray176}{RGB}{176,176,176}
\definecolor{darkviolet1270255}{RGB}{127,0,255}
\definecolor{deepskyblue3176236}{RGB}{3,176,236}
\definecolor{dodgerblue45123246}{RGB}{45,123,246}
\definecolor{lightgray204}{RGB}{204,204,204}
\definecolor{royalblue8762253}{RGB}{87,62,253}

\begin{axis}[
legend cell align={left},
legend style={fill opacity=0.8, draw opacity=1, text opacity=1, draw=lightgray204, legend columns=-1, legend pos=north west},
tick align=outside,
tick pos=left,
title={$\Delta=2$ Repair Precision},
x grid style={darkgray176},
xlabel={Seconds},
xmin=-0.5925, xmax=8.5925,
xtick style={color=black},
xtick={0,1,2,3,4,5,6,7,8},
xticklabels={0.1,0.3,0.5,0.7,0.9,1.1,1.3,1.5,1.7},
y grid style={darkgray176},
ymin=0, ymax=0.3843,
ytick style={color=black},
width=0.55\textwidth,
height=0.49\textwidth,
]
\addlegendimage{empty legend}
\addlegendentry{P@}
\draw[draw=none,fill=darkviolet1270255] (axis cs:-0.175,0) rectangle (axis cs:0.175,0.067);
\addlegendimage{ybar,ybar legend,draw=none,fill=darkviolet1270255}
\addlegendentry{All}

\draw[draw=none,fill=darkviolet1270255] (axis cs:0.825,0) rectangle (axis cs:1.175,0.165);
\draw[draw=none,fill=darkviolet1270255] (axis cs:1.825,0) rectangle (axis cs:2.175,0.225);
\draw[draw=none,fill=darkviolet1270255] (axis cs:2.825,0) rectangle (axis cs:3.175,0.261);
\draw[draw=none,fill=darkviolet1270255] (axis cs:3.825,0) rectangle (axis cs:4.175,0.292);
\draw[draw=none,fill=darkviolet1270255] (axis cs:4.825,0) rectangle (axis cs:5.175,0.309);
\draw[draw=none,fill=darkviolet1270255] (axis cs:5.825,0) rectangle (axis cs:6.175,0.335);
\draw[draw=none,fill=darkviolet1270255] (axis cs:6.825,0) rectangle (axis cs:7.175,0.353);
\draw[draw=none,fill=darkviolet1270255] (axis cs:7.825,0) rectangle (axis cs:8.175,0.366);
\draw[draw=none,fill=royalblue8762253] (axis cs:-0.175,0) rectangle (axis cs:0.175,0.06);
\addlegendimage{ybar,ybar legend,draw=none,fill=royalblue8762253}
\addlegendentry{10}

\draw[draw=none,fill=royalblue8762253] (axis cs:0.825,0) rectangle (axis cs:1.175,0.128);
\draw[draw=none,fill=royalblue8762253] (axis cs:1.825,0) rectangle (axis cs:2.175,0.164);
\draw[draw=none,fill=royalblue8762253] (axis cs:2.825,0) rectangle (axis cs:3.175,0.182);
\draw[draw=none,fill=royalblue8762253] (axis cs:3.825,0) rectangle (axis cs:4.175,0.2);
\draw[draw=none,fill=royalblue8762253] (axis cs:4.825,0) rectangle (axis cs:5.175,0.208);
\draw[draw=none,fill=royalblue8762253] (axis cs:5.825,0) rectangle (axis cs:6.175,0.221);
\draw[draw=none,fill=royalblue8762253] (axis cs:6.825,0) rectangle (axis cs:7.175,0.23);
\draw[draw=none,fill=royalblue8762253] (axis cs:7.825,0) rectangle (axis cs:8.175,0.238);
\draw[draw=none,fill=dodgerblue45123246] (axis cs:-0.175,0) rectangle (axis cs:0.175,0.05);
\addlegendimage{ybar,ybar legend,draw=none,fill=dodgerblue45123246}
\addlegendentry{5}

\draw[draw=none,fill=dodgerblue45123246] (axis cs:0.825,0) rectangle (axis cs:1.175,0.1);
\draw[draw=none,fill=dodgerblue45123246] (axis cs:1.825,0) rectangle (axis cs:2.175,0.123);
\draw[draw=none,fill=dodgerblue45123246] (axis cs:2.825,0) rectangle (axis cs:3.175,0.132);
\draw[draw=none,fill=dodgerblue45123246] (axis cs:3.825,0) rectangle (axis cs:4.175,0.142);
\draw[draw=none,fill=dodgerblue45123246] (axis cs:4.825,0) rectangle (axis cs:5.175,0.144);
\draw[draw=none,fill=dodgerblue45123246] (axis cs:5.825,0) rectangle (axis cs:6.175,0.155);
\draw[draw=none,fill=dodgerblue45123246] (axis cs:6.825,0) rectangle (axis cs:7.175,0.163);
\draw[draw=none,fill=dodgerblue45123246] (axis cs:7.825,0) rectangle (axis cs:8.175,0.167);
\draw[draw=none,fill=deepskyblue3176236] (axis cs:-0.175,0) rectangle (axis cs:0.175,0.016);
\addlegendimage{ybar,ybar legend,draw=none,fill=deepskyblue3176236}
\addlegendentry{1}

\draw[draw=none,fill=deepskyblue3176236] (axis cs:0.825,0) rectangle (axis cs:1.175,0.038);
\draw[draw=none,fill=deepskyblue3176236] (axis cs:1.825,0) rectangle (axis cs:2.175,0.052);
\draw[draw=none,fill=deepskyblue3176236] (axis cs:2.825,0) rectangle (axis cs:3.175,0.057);
\draw[draw=none,fill=deepskyblue3176236] (axis cs:3.825,0) rectangle (axis cs:4.175,0.061);
\draw[draw=none,fill=deepskyblue3176236] (axis cs:4.825,0) rectangle (axis cs:5.175,0.064);
\draw[draw=none,fill=deepskyblue3176236] (axis cs:5.825,0) rectangle (axis cs:6.175,0.072);
\draw[draw=none,fill=deepskyblue3176236] (axis cs:6.825,0) rectangle (axis cs:7.175,0.076);
\draw[draw=none,fill=deepskyblue3176236] (axis cs:7.825,0) rectangle (axis cs:8.175,0.078);
\addplot [red, thick] coordinates {(-0.8925,0.16) (14.8925,0.16)};
\end{axis}

\end{tikzpicture}
}
    \resizebox{.24\textwidth}{!}{% This file was created with tikzplotlib v0.10.1.
\begin{tikzpicture}
\begin{axis}[
legend cell align={left},
legend style={fill opacity=0.8, draw opacity=1, text opacity=1, draw=lightgray204, legend columns=-1, legend pos=north west},
tick align=outside,
tick pos=left,
axis lines*=left,
title={$\Delta=3$ Repair Precision},
x grid style={darkgray176},
xlabel={Seconds},
xmin=-0.5925, xmax=8.5925,
xtick style={color=black},
xtick={0,1,2,3,4,5,6,7,8},
xticklabels={0.1,0.3,0.5,0.7,0.9,1.1,1.3,1.5,1.7},
yticklabels={0, 0.05, 0.10, 0.15, 0.20},
y grid style={darkgray176},
ymin=0, ymax=0.15225,
ytick style={color=black},
width=0.53\textwidth,
height=0.48\textwidth,
]
\draw[draw=none,fill=darkviolet1270255] (axis cs:-0.175,0) rectangle (axis cs:0.175,0.01);

\draw[draw=none,fill=darkviolet1270255] (axis cs:0.825,0) rectangle (axis cs:1.175,0.035);
\draw[draw=none,fill=darkviolet1270255] (axis cs:1.825,0) rectangle (axis cs:2.175,0.064);
\draw[draw=none,fill=darkviolet1270255] (axis cs:2.825,0) rectangle (axis cs:3.175,0.072);
\draw[draw=none,fill=darkviolet1270255] (axis cs:3.825,0) rectangle (axis cs:4.175,0.09);
\draw[draw=none,fill=darkviolet1270255] (axis cs:4.825,0) rectangle (axis cs:5.175,0.109);
\draw[draw=none,fill=darkviolet1270255] (axis cs:5.825,0) rectangle (axis cs:6.175,0.116);
\draw[draw=none,fill=darkviolet1270255] (axis cs:6.825,0) rectangle (axis cs:7.175,0.133);
\draw[draw=none,fill=darkviolet1270255] (axis cs:7.825,0) rectangle (axis cs:8.175,0.145);
\draw[draw=none,fill=royalblue8762253] (axis cs:-0.175,0) rectangle (axis cs:0.175,0.003);

\draw[draw=none,fill=royalblue8762253] (axis cs:0.825,0) rectangle (axis cs:1.175,0.01);
\draw[draw=none,fill=royalblue8762253] (axis cs:1.825,0) rectangle (axis cs:2.175,0.01);
\draw[draw=none,fill=royalblue8762253] (axis cs:2.825,0) rectangle (axis cs:3.175,0.014);
\draw[draw=none,fill=royalblue8762253] (axis cs:3.825,0) rectangle (axis cs:4.175,0.018);
\draw[draw=none,fill=royalblue8762253] (axis cs:4.825,0) rectangle (axis cs:5.175,0.019);
\draw[draw=none,fill=royalblue8762253] (axis cs:5.825,0) rectangle (axis cs:6.175,0.018);
\draw[draw=none,fill=royalblue8762253] (axis cs:6.825,0) rectangle (axis cs:7.175,0.023);
\draw[draw=none,fill=royalblue8762253] (axis cs:7.825,0) rectangle (axis cs:8.175,0.023);
\draw[draw=none,fill=dodgerblue45123246] (axis cs:-0.175,0) rectangle (axis cs:0.175,0.002);

\draw[draw=none,fill=dodgerblue45123246] (axis cs:0.825,0) rectangle (axis cs:1.175,0.005);
\draw[draw=none,fill=dodgerblue45123246] (axis cs:1.825,0) rectangle (axis cs:2.175,0.008);
\draw[draw=none,fill=dodgerblue45123246] (axis cs:2.825,0) rectangle (axis cs:3.175,0.01);
\draw[draw=none,fill=dodgerblue45123246] (axis cs:3.825,0) rectangle (axis cs:4.175,0.011);
\draw[draw=none,fill=dodgerblue45123246] (axis cs:4.825,0) rectangle (axis cs:5.175,0.011);
\draw[draw=none,fill=dodgerblue45123246] (axis cs:5.825,0) rectangle (axis cs:6.175,0.011);
\draw[draw=none,fill=dodgerblue45123246] (axis cs:6.825,0) rectangle (axis cs:7.175,0.014);
\draw[draw=none,fill=dodgerblue45123246] (axis cs:7.825,0) rectangle (axis cs:8.175,0.018);
\draw[draw=none,fill=deepskyblue3176236] (axis cs:-0.175,0) rectangle (axis cs:0.175,0);

\draw[draw=none,fill=deepskyblue3176236] (axis cs:0.825,0) rectangle (axis cs:1.175,0.003);
\draw[draw=none,fill=deepskyblue3176236] (axis cs:1.825,0) rectangle (axis cs:2.175,0.008);
\draw[draw=none,fill=deepskyblue3176236] (axis cs:2.825,0) rectangle (axis cs:3.175,0.008);
\draw[draw=none,fill=deepskyblue3176236] (axis cs:3.825,0) rectangle (axis cs:4.175,0.011);
\draw[draw=none,fill=deepskyblue3176236] (axis cs:4.825,0) rectangle (axis cs:5.175,0.011);
\draw[draw=none,fill=deepskyblue3176236] (axis cs:5.825,0) rectangle (axis cs:6.175,0.011);
\draw[draw=none,fill=deepskyblue3176236] (axis cs:6.825,0) rectangle (axis cs:7.175,0.013);
\draw[draw=none,fill=deepskyblue3176236] (axis cs:7.825,0) rectangle (axis cs:8.175,0.013);
\addplot [red, thick] coordinates {(-0.8925,0.096) (14.8925,0.096)};
\addplot [orange, thick] coordinates {(-0.8925,0.02) (14.8925,0.02)};
\end{axis}

\end{tikzpicture}
}
    \caption{Adaptive sampling repairs. The red line indicates Seq2Parse precision@1 on the same dataset. Since it only supports generating one repair, we do not report precision@k or the intermediate latency cutoffs.}\label{fig:adaptive}
  \end{figure}

  We also evaluate Seq2Parse on the same dataset. Seq2Parse only supports precision@1 repairs, and so we only report Seq2parse precision@1 from the StackOverflow benchmark for comparison. Unlike our approach which only produces syntactically correct repairs, Seq2Parse also produces syntactically incorrect repairs and so we report the percentage of repairs matching the human repair for both our method and Seq2Parse. Seq2Parse latency varies depending on the length of the repair, averaging 1.5s for $\Delta=1$ to 2.7s for $\Delta=3$, across the entire StackOverflow dataset.

%  While adaptive sampling is able to saturate the admissible set for 1- and 2-edit repairs before the timeout elapses, 3-edit throughput is heavily constrained by compute around 16 lexical tokens, when Python's Levenshtein ball has a volume of roughly $6\times 10^8$ edits. This bottleneck can be relaxed with a longer timeout or additional CPU cores. Despite the high computational cost of sampling multi-edit repairs, our precision@all remains competitive with the Seq2Parse neurosymbolic baseline at the same latency. We provide some qualitative examples of repairs in Table~\ref{sec:appendix}.

  \begin{wrapfigure}{r}{0.35\textwidth}
    \vspace{-10pt}
    \resizebox{.35\textwidth}{!}{% This file was created with tikzplotlib v0.10.1.
\begin{tikzpicture}
\begin{axis}[
  legend cell align={left},
  legend style={fill opacity=0.8, draw opacity=1, text opacity=1, legend columns=-1, legend pos=north east},
  tick align=outside,
  tick pos=left,
  axis lines*=left,
  title={\textbf{Average throughput}},
  x grid style={darkgray176},
  xlabel={Lexical tokens},
  xmin=-0.95, xmax=19.95,
  xtick style={color=black},
  xtick={0,1,2,3,4,5,6,7,8,9,10,11,12,13,14,15,16,17,18,19},
  xticklabels={4, , 8, , 12, , 16, , 20, , 24, , 28, , 32, , 36, , 40,},
  y grid style={darkgray176},
  ylabel={Total repairs discovered},
  ymin=0, ymax=200000,
  ymode=log,
  legend pos=north east,
  ytick style={color=black}
]
\addlegendimage{empty legend}
\addlegendentry{$\Delta=$}
\addlegendentry{\footnotesize{$1,$}}
\addlegendentry{\footnotesize{$2,$}}
\addlegendentry{\footnotesize{$3,$}}
\addlegendentry{\footnotesize{$4$}}
\addplot [semithick, green, mark=*, mark size=3, mark options={solid}]
table {%
  0 7.14285714285714
  1 12.1666666666667
  2 9.27272727272727
  3 10.6041666666667
  4 7.22857142857143
  5 6.4
  6 10.3956043956044
  7 7.86075949367089
  8 9.81818181818182
  9 8.28089887640449
  10 8.08695652173913
  11 10.0430107526882
  12 6.90476190476191
  13 6.94029850746269
  14 8.74698795180723
  15 7.70422535211268
  16 12.1414141414141
  17 4.1
};
\addplot [semithick, blue, mark=*, mark size=3, mark options={solid}]
table {%
0 46.625
1 135.653846153846
2 175.676923076923
3 79.0090909090909
4 87.6589147286822
5 69.9683544303797
6 83.9896373056995
7 85.9777777777778
8 73.3684210526316
9 67.7715517241379
10 92.1878172588832
11 186.568527918782
12 116.195945945946
13 87.8901734104046
14 99.6514285714286
15 154.44
16 123.658064516129
17 36.6111111111111
};
\addplot [semithick, orange, mark=*, mark size=3, mark options={solid}]
table {%
0 1506.8
1 2076.57142857143
2 5194.91304347826
3 413.939393939394
4 3749.34693877551
5 1130.83076923077
6 431.141025641026
7 601.172839506173
8 2135.26666666667
9 1233.85106382979
10 960.263888888889
11 932.225352112676
12 612.833333333333
13 880.2625
14 965.863636363636
15 725.139784946237
16 1199.05882352941
17 114.5
};
\addplot [semithick, red, mark=*, mark size=3, mark options={solid}]
table {%
0 nan
1 73461
2 24965
3 4220.4
4 35145.8181818182
5 2613.28571428571
6 68402.25
7 3300.5
8 10010
9 24536.5217391304
10 17497.6315789474
11 8883.58333333333
12 1948.23076923077
13 4175.27272727273
14 5145.45454545455
15 27785.8095238095
16 2468
17 4027.25
};
\end{axis}

\end{tikzpicture}}
    \caption{Total repairs found in 30s.}
    \label{fig:throughput}
    \vspace{-10pt}
  \end{wrapfigure}

  End-to-end throughput varies significantly with the edit distance of the repair. Some errors are trivial to fix, while others require a large number of edits to be sampled before a syntactically valid edit is discovered. We evaluate throughput by sampling edits across invalid strings $|\err\sigma| \leq 40$ from the StackOverflow dataset of varying length, and measure the total number of syntactically valid edits discovered, as a function of string length and language edit distance $\Delta\in[1, 3]$. Each trial is terminated after 10 seconds, and the experiment is repeated across 7.3k total repairs. Note the y-axis is log-scaled, as the number of admissible repairs increases sharply with language edit distance. Our approach discovers a large number of syntactically valid repairs in a relatively short amount of time, and is able to quickly saturate the admissible set for $\Delta(\err\sigma, \sigma) \in [1, 4]$ before timeout. As Seq2Parse only generates one syntactically valid repair per string, we do not report its throughput.

  \begin{wrapfigure}{l}{0.35\textwidth}
    \resizebox{.35\textwidth}{!}{% This file was created with matplot2tikz v0.3.3.
\tikzset{mark size=0.5}

\begin{tikzpicture}

\begin{axis}[
log basis y={10},
tick align=outside,
tick pos=left,
x grid style={darkdarkgray176},
xlabel={Snippet length},
xmin=0, xmax=80,
title={\textbf{Latency across repair size}},
xtick style={color=black},
y grid style={darkdarkgray176},
log basis y={10},
axis lines*=left,
ylabel={Repair latency (ms)},
ymin=6.99556128596625, ymax=147694.796423657,
ymode=log,
mark size=1.2pt,
ytick style={color=black},
yticklabels={
  \(\displaystyle {10^{0}}\),
  \(\displaystyle {10^{1}}\),
  \(\displaystyle {10^{2}}\),
  \(\displaystyle {10^{3}}\),
  \(\displaystyle {10^{4}}\),
  \(\displaystyle {10^{5}}\),
}
]
\addplot [draw=darkgray, fill=darkgray, mark=*, only marks]
table{%
x  y
9 1549
12 116
16 1976
18 288
39 1303
63 2641
11 56
73 4245
45 1416
34 706
27 1489
37 918
29 542
29 389
15 142
39 968
29 434
45 1260
19 879
61 2273
7 120
23 924
51 1570
49 1441
61 2300
42 1175
68 3944
60 2896
72 4049
53 2217
8 113
13 265
62 17742
33 3818
47 9889
41 1394
60 4760
77 4309
72 4271
27 484
75 3934
16 96
32 677
21 339
49 1836
45 1435
58 3188
18 120
30 816
15 463
19 714
49 1691
43 1476
41 831
14 192
21 182
19 10979
78 4436
52 1445
23 652
21 1557
33 974
17 118
30 933
30 374
24 541
66 25769
50 1829
47 1638
46 1360
35 4446
51 1482
59 2163
13 150
41 947
33 1189
26 312
17 182
17 178
20 253
27 1030
34 20964
26 315
8 25
20 152
53 2119
59 2414
41 26227
19 259
63 2519
63 10671
41 1815
32 423
65 4079
74 4863
36 986
9 37
46 1446
78 5210
61 4684
69 4292
14 79
45 1274
14 232
33 540
38 953
16 103
31 908
69 31228
33 709
31 13133
15 253
48 2581
66 19478
36 629
33 495
30 396
40 941
71 4825
10 983
28 426
69 3913
25 994
44 2016
7 21
50 2080
53 2054
29 382
34 763
63 2470
34 955
9 60
16 119
22 10030
40 1040
22 512
77 5203
30 616
25 242
39 958
69 5508
59 2896
21 1549
51 40442
10 134
30 722
54 29715
36 990
40 988
52 1745
10 340
48 1874
29 574
52 2219
39 794
34 732
33 516
23 470
50 15557
30 641
16 496
23 328
22 459
33 742
31 765
7 102
48 2084
45 1484
25 258
30 691
71 3900
27 548
26 353
66 3080
63 4977
40 1250
18 289
14 1780
27 2357
29 623
40 1008
22 213
64 4137
24 398
36 862
46 1424
13 113
68 4167
10 54
53 6662
40 1054
9 66
26 611
68 3685
50 4591
19 798
30 485
52 1583
12 74
34 644
77 6474
76 85265
21 429
11 170
23 287
25 496
34 1099
6 21
28 882
13 149
25 397
35 1044
23 4948
25 1657
15 243
36 848
23 237
27 1713
19 217
25 520
37 919
24 965
39 1717
15 792
49 1469
77 5034
38 3093
40 1097
51 1957
6 63
10 135
17 597
39 753
63 3249
25 447
41 2468
56 2072
74 11972
49 1742
25 494
34 93928
67 6025
64 3035
13 79
56 1714
67 3955
66 3282
63 3261
16 105
5 85
26 545
28 413
8 90
13 106
37 805
69 5137
28 369
29 673
10 44
42 1239
67 4648
27 792
51 7387
20 1096
30 585
37 790
30 462
28 986
9 44
50 1843
70 3327
52 1525
11 79
30 515
16 210
16 196
27 554
11 184
53 1947
8 67
37 929
26 971
26 367
33 648
32 616
14 134
47 1524
5 11
24 913
32 418
75 12513
35 1148
59 2608
11 157
50 2808
15 191
14 184
41 1019
10 157
71 4695
69 4455
34 540
69 4740
51 3258
44 1029
51 1498
50 1420
46 1850
40 1080
17 149
59 2862
21 14398
37 681
37 2124
11 222
23 463
51 2020
33 530
42 17922
4 13
39 698
52 2063
21 328
78 7145
28 429
19 250
7 97
43 1210
59 38172
22 362
42 2745
47 1206
28 379
62 2811
55 2179
75 4071
45 1063
37 1899
12 112
40 745
21 884
9 40
70 3981
36 888
28 392
56 5825
27 393
23 223
24 10408
18 174
23 2177
53 1662
39 1384
41 989
71 3464
60 2194
35 977
67 4303
50 9881
21 379
55 2905
58 2756
45 1769
20 609
43 38706
40 3024
32 664
23 217
18 135
65 5228
67 3550
51 1385
14 107
19 212
33 1746
75 3857
72 4455
44 1118
68 4834
8 55
39 29045
10 43
22 336
48 7343
72 19347
24 269
49 2791
19 785
39 1120
49 1322
21 760
28 521
63 62239
49 1629
23 6078
37 27981
41 1375
51 2274
5 65
17 191
62 3688
27 1016
41 1326
34 2739
33 2785
35 916
63 2554
40 1149
11 56
72 4190
18 145
62 3069
16 379
65 2780
46 1193
50 1621
76 4978
45 1264
16 382
23 655
20 292
27 705
58 2380
26 539
42 7325
15 264
55 2162
76 25267
49 2954
46 1565
55 2499
35 568
24 303
36 783
36 725
29 392
53 6744
19 185
52 1495
11 149
16 583
26 471
76 5789
51 1362
48 1514
54 2037
23 306
8 145
31 623
12 105
70 3810
21 167
11 186
43 11621
15 897
77 30152
8 219
14 2669
15 198
29 907
11 46
38 708
31 894
41 1293
30 24580
60 31750
64 3403
26 527
27 783
21 189
34 980
15 138
9 64
9 32
8 90
59 2109
15 94
33 695
17 14201
36 733
75 5222
36 1902
18 535
67 4110
29 950
35 1798
42 1463
33 766
41 1037
33 503
30 1323
21 252
27 757
32 916
37 1450
54 2323
49 1980
44 9744
26 226
40 11847
22 919
21 4868
42 1916
27 326
53 2114
50 1959
13 178
24 2361
8 92
19 298
29 454
56 3157
27 466
55 2120
71 4493
12 97
25 544
39 1089
38 1161
36 580
53 2653
16 203
10 3282
12 57
30 644
55 1814
34 17457
14 70
25 365
19 265
38 1074
16 203
23 626
46 1762
20 538
13 385
53 2187
37 1069
33 702
9 104
62 4072
32 632
47 1670
50 2049
65 3868
38 810
16 262
62 2518
51 1981
11 109
12 2936
28 520
76 11739
71 3836
49 2243
};
\addplot [draw=darkgray, fill=darkgray, mark=*, only marks]
table{%
x  y
37 967
13 576
31 725
13 82
62 3570
27 286
11 65
55 1667
29 342
18 467
27 300
16 272
57 2199
75 4119
20 205
17 366
16 125
63 4737
45 1024
70 3483
26 608
21 563
30 1598
38 1097
12 62
33 613
61 2243
19 280
27 902
35 556
63 3421
26 714
18 110
47 5579
47 1203
30 955
55 2276
36 1207
46 1206
58 2320
41 855
73 3738
30 646
73 4086
34 805
26 254
48 1192
33 562
14 286
50 1452
64 2511
31 576
28 290
73 4351
45 3090
28 614
36 667
52 2867
77 5575
31 489
50 1848
35 820
42 1054
28 397
76 4091
37 736
64 2432
52 1436
10 169
41 775
9 54
67 3335
27 451
28 303
31 427
57 1809
42 1335
18 216
45 978
39 846
12 157
42 872
20 163
17 398
19 2462
23 198
19 367
26 310
69 3317
15 249
77 4157
42 856
17 121
20 758
36 558
18 149
22 496
62 2332
24 569
29 376
17 939
40 993
24 297
8 70
18 285
57 2095
23 670
29 336
25 1004
30 396
72 3449
38 1492
58 2403
54 2802
43 1186
20 423
26 1536
15 76
5 19
78 5718
39 670
26 281
43 874
44 1863
13 67
17 471
43 1152
24 266
43 956
29 363
48 1478
39 992
16 153
36 693
57 1936
14 390
66 5587
52 2022
31 1252
30 384
20 248
20 150
42 1961
27 464
11 106
21 581
11 99
33 720
29 536
56 6622
61 2275
12 99
78 4706
9 158
78 5312
64 3852
43 935
36 573
33 1219
21 1063
13 59
19 243
23 199
12 185
29 1062
59 2021
25 290
34 585
5 19
39 974
20 186
25 260
74 4492
25 468
46 1182
28 310
44 913
38 735
23 361
31 388
25 252
77 4451
53 3401
10 67
10 110
55 2453
61 2455
71 3342
43 921
63 3298
78 4901
60 13008
30 599
68 2986
32 559
13 310
22 216
20 368
51 2238
29 691
37 608
28 320
35 797
53 3644
23 263
58 2283
24 1170
50 4202
11 130
60 2883
34 2120
71 3378
22 180
36 858
19 217
5 47
18 384
22 392
26 1020
30 430
9 75
62 2291
8 143
39 753
73 3607
57 2080
35 2589
47 1105
20 591
38 1445
38 718
31 728
60 11098
34 496
35 1681
41 781
14 208
65 5798
61 2345
18 116
17 124
21 178
8 93
39 736
49 1481
46 1083
25 554
24 347
63 2938
16 180
15 202
43 1072
11 54
22 216
36 572
27 451
18 292
27 333
37 923
54 1697
26 745
18 223
17 564
48 1236
13 102
24 313
47 1151
24 418
73 4691
22 186
50 1315
72 3757
32 846
29 851
43 1050
34 480
31 419
30 762
17 394
33 557
30 1017
32 422
20 162
16 94
33 690
70 4054
27 626
44 2331
52 1935
22 749
71 3332
20 1017
34 2256
14 107
29 559
18 146
38 761
39 1008
18 226
48 2323
60 2418
38 653
26 273
27 336
33 455
32 416
13 214
77 4543
49 2068
20 355
61 2236
55 4802
11 163
20 179
65 2701
48 6149
77 4137
51 4601
55 1865
26 273
35 4314
67 2978
25 303
24 699
76 4592
23 236
25 265
7 36
34 499
62 2235
22 344
27 787
70 3060
30 2236
20 457
36 1361
72 5525
23 1335
13 373
16 263
30 1738
33 901
9 61
71 3235
40 775
33 638
30 366
16 401
33 3066
40 807
69 3141
9 60
28 398
59 2206
72 5128
36 1173
38 755
30 396
59 9951
73 4086
18 244
19 287
31 629
58 1929
48 1854
29 543
15 574
76 4735
37 3785
28 3065
33 3925
20 315
13 128
21 779
35 586
27 461
38 728
49 2232
15 578
24 237
24 234
39 831
17 264
34 499
14 560
40 799
31 403
24 227
21 418
50 1515
61 2860
48 1200
62 2411
75 4749
35 511
39 687
51 2613
15 99
40 784
4 21
24 2146
37 665
46 1132
24 275
22 396
31 403
33 456
31 615
23 324
16 216
54 1671
34 1328
23 467
56 1767
9 181
36 560
20 191
63 2774
27 371
8 215
27 346
33 3668
41 6073
16 109
56 2557
42 854
29 923
66 7498
28 328
50 2110
17 183
73 4756
39 907
47 1142
74 4066
17 114
18 134
46 1133
24 243
45 1212
35 1622
19 718
50 1286
74 3944
52 1596
22 417
32 1180
24 807
20 175
45 1019
23 203
25 255
65 2716
37 612
40 799
9 75
36 668
16 181
39 3541
16 211
49 2661
25 495
59 1979
37 595
16 403
16 280
53 2048
62 2206
28 308
54 1583
47 1330
21 195
15 394
55 2754
37 1058
32 818
14 95
41 811
15 91
34 561
15 517
13 298
48 1138
28 458
44 1152
24 223
30 376
79 4556
35 814
43 2771
53 1486
21 202
42 848
45 995
18 115
12 170
37 616
19 546
57 1925
26 322
21 173
60 2626
51 1515
27 2046
19 366
18 332
71 3346
31 656
73 3772
44 944
70 3714
43 1159
68 3472
19 336
55 1778
24 251
45 1963
18 154
58 2220
10 220
45 1072
43 1188
33 517
40 876
16 646
48 1214
22 561
37 930
63 4137
33 1414
13 397
30 469
25 306
15 130
52 1676
51 1391
42 1326
29 596
65 2703
60 3376
71 3470
18 109
40 737
35 827
30 444
31 495
59 2695
79 4538
37 619
31 538
20 376
57 2979
23 410
29 908
28 376
36 955
13 225
43 1404
29 472
35 604
23 220
16 168
17 151
13 151
13 77
12 396
57 1895
41 860
57 2269
75 3797
22 176
21 658
27 286
19 172
54 1513
10 138
17 319
44 987
27 286
25 473
17 119
64 2514
28 447
42 1606
53 1810
14 160
33 1491
53 2038
58 2610
21 236
36 578
27 473
24 275
74 3879
79 5334
67 3619
47 2079
41 833
15 254
53 1662
26 265
20 142
22 415
52 1511
11 74
31 505
50 1314
9 219
20 178
53 1688
55 1727
25 714
21 380
45 1511
73 3864
20 169
14 178
35 2191
63 2616
29 603
49 2332
47 1248
26 793
7 19
28 335
18 416
36 1161
20 182
24 714
14 88
49 1390
37 637
66 3852
29 337
12 64
32 468
34 1461
17 104
42 912
17 406
28 464
20 661
34 1508
32 501
41 785
74 3767
17 125
40 1750
30 468
11 256
53 1588
35 716
28 523
20 132
38 1072
62 2652
45 1865
42 888
71 3634
55 1915
66 2838
39 826
66 3678
62 2941
33 862
32 1045
20 463
26 407
14 88
51 1526
32 1004
32 810
27 396
5 34
15 458
31 530
11 116
36 1260
21 204
46 1272
64 3805
53 1946
56 2260
44 981
29 394
34 764
29 503
24 267
20 357
70 5314
21 212
19 271
25 1602
36 668
54 1692
41 879
27 314
29 469
43 1248
26 288
78 7398
46 1522
42 1005
19 162
20 659
16 352
29 369
41 1444
35 584
29 493
72 4540
36 565
21 948
16 111
34 1065
59 2490
22 253
16 195
51 1527
49 1373
53 1549
36 677
64 2722
55 1864
25 269
24 230
37 748
32 447
16 151
15 122
16 872
65 3000
17 100
12 296
29 466
32 674
29 337
15 230
45 1084
35 960
79 5158
34 563
21 394
33 633
24 330
27 295
27 329
36 565
34 1367
41 802
71 4002
35 548
47 1487
18 198
29 448
57 3080
56 2030
39 706
37 669
11 104
41 877
27 318
36 1874
22 231
42 1185
8 40
42 894
32 855
64 3359
33 603
41 823
22 306
29 448
15 94
34 824
6 15
41 860
12 153
4 28
35 906
72 3895
64 2509
16 427
15 113
31 737
18 138
42 1059
11 63
12 149
5 37
65 2822
17 278
19 303
17 335
18 175
17 562
22 605
23 240
35 3819
26 259
45 997
52 1586
16 429
34 490
29 1283
22 555
33 991
34 520
70 4662
13 213
13 97
52 1514
22 210
50 1817
29 383
33 675
18 422
18 117
53 2349
24 2137
39 1467
18 157
78 5258
45 1250
29 342
10 82
28 322
24 292
71 4662
16 211
27 294
15 271
42 1225
33 532
57 1970
34 566
36 1764
34 592
32 499
9 81
72 5270
31 1433
31 725
61 2268
50 2300
19 925
32 446
8 73
71 4323
25 763
18 310
31 442
54 1689
22 248
17 298
19 148
39 715
46 1237
24 378
33 620
55 1928
33 448
54 2785
28 2556
16 114
17 143
29 414
6 15
25 820
76 4052
35 846
27 347
44 1173
73 3785
34 486
15 148
69 4745
31 1132
12 130
29 553
77 4168
47 1263
54 3108
40 2104
25 370
32 1279
52 12972
36 1281
29 417
20 184
38 1157
42 889
11 164
46 1070
47 2268
5 46
13 530
9 55
30 898
45 1442
20 179
34 2244
11 49
14 85
5 29
12 48
26 402
25 586
28 415
38 708
40 1013
24 470
26 771
28 380
36 1000
19 609
48 1182
52 1710
26 289
36 605
13 390
23 376
28 3022
20 985
26 405
42 853
20 159
11 89
15 99
60 2482
38 870
35 743
20 890
19 1640
14 92
25 295
25 913
5 41
32 472
25 233
22 446
15 232
68 3026
19 514
37 2185
78 4881
39 773
26 295
41 909
30 432
12 304
13 88
32 454
76 4302
26 819
74 6097
35 603
48 4513
21 378
40 834
29 380
66 2766
71 3677
23 1391
24 261
38 747
8 146
26 1750
39 732
22 588
9 155
33 485
23 1342
15 459
44 1375
76 4937
44 1381
15 372
34 1294
22 195
29 333
49 6540
18 163
57 2083
44 963
34 504
19 229
59 2052
34 624
35 551
59 2096
23 195
46 1312
74 3939
29 628
26 633
27 306
20 651
18 122
50 1552
43 2398
27 559
12 49
34 648
65 2751
27 443
33 518
56 1772
12 56
37 1870
36 600
29 2453
19 321
22 622
29 1666
62 2886
43 976
15 110
54 2059
53 1487
16 124
24 383
37 602
35 531
34 716
58 2505
14 164
22 323
35 664
23 400
78 5086
};
\addplot [draw=darkgray, fill=darkgray, mark=*, only marks]
table{%
x  y
19 524
42 1166
45 3390
69 18648
51 3254
32 10954
36 680
76 7230
40 2763
64 3123
35 970
75 26775
60 4154
37 3369
28 406
48 36005
70 5503
54 4443
43 4796
35 4535
43 2163
11 255
30 1756
31 701
61 13931
37 2268
76 30644
44 925
43 2764
15 1076
50 6609
60 12517
30 2429
29 1154
34 694
9 655
29 1101
31 3358
33 620
12 3279
32 5493
65 15497
26 507
29 1146
65 5376
24 858
36 1494
40 1621
26 787
71 4837
36 1460
39 1047
28 485
29 1114
25 3603
67 3174
};
\end{axis}

\end{tikzpicture}
}
    \caption{End-to-end repair timings across snippet length.}
    \label{fig:timings}
    \vspace{-20pt}
  \end{wrapfigure}

  In Fig.~\ref{fig:timings}, we plot the end-to-end repair timings across snippet length. To measure the timings, we collect 1000 repair samples of varying lengths and edit distances, then measure the time until the sampler locates the exact human repair. For each code snippet, we measure the total time taken, for repairs between 1-4 Levenshtein edits. Note the logarithmic scale on the y-axis.

  \subsection{Synthetic repair benchmark}\label{sec:latency}

  Pairwise naturally-occurring errors and human fixes are the most authentic source of real-world syntax repairs, but can be difficult to obtain due to the paucity of parallel syntax error corpi. In the absence of natural syntax repairs, one viable alternative is to collect a dataset of syntactically valid code, and synthetically corrupt it. The original source code becomes the ground truth repair for the synthetically generated typo, and the target for evaluating the precision of our repair procedure.

  In our synthetic repair benchmark, we synthetically corrupt Python code snippets using a learned corruption model. To train the model, we compute the Levenshtein alignment on the StackOverflow dataset, then approximate the conditional probability of each edit given the local context. During evaluation, we sample a corruption from the learned typo distribution, and measure the precision of our model at recovering the originally valid lexical sequence.

  Suppose we have a dataset of Levenshtein edits and their local context. For simplicity, we shall assume a trigram language model, i.e., $P(\sigma_i' \mid \sigma_{i-1}, \sigma_i, \sigma_{i+1})$, however the approach can be generalized to higher-order Markov models. Given a string $\sigma$, we can sample error trajectories $q^1(\sigma), q^2(\sigma), \ldots, q^n(\sigma)$ by defining $q(\sigma)$ to sample a single edit from the set of all relevant edit actions $Q(\sigma)$, then recursively applying $q$ to the resulting string. More formally,

  \begin{enumerate}
    \item Given a string $\sigma$, compute $Q(\sigma)$, the set of all relevant edit actions for all possible edit locations by unioning the set of all possible edits at each location, i.e., $Q(\sigma) \coloneqq \bigcup_{i=1}^{|\sigma| - 1} \big\{\sigma_i' \mid  0 < P(\sigma_i \mid \sigma_{i-1}, \sigma_i, \sigma_{i+1})\big\}$.
    \item Renormalize the probabilities of each edit $P(q \mid \sigma)$ by $\sum_{q \in Q(\sigma)} P(q)$. This ensures the probability of sampling a particular edit is proportional to its relative probability under the language model and sums to 1.
    \item Sample an edit $q(\sigma) \sim Q(\sigma)$, then repeat for $n$ steps where $n$ is sampled from a geometric distribution with mean $\mu$ matching the average edit distance of the dataset (this assumes the edit distance is independent of the edits).
  \end{enumerate}

  For example, suppose we have the following patch in our initial dataset:\\

  \texttt{BOS \hlred{def} NAME ( NAME ) : NEWLINE \hlred{INDENT} return \hlorange{NAME} NEWLINE \hlgreen{INDENT} NEWLINE EOS}\\

  From this patch, the following contextual typo probabilities will be incremented:

  \begin{align*}
    P(\texttt{BOS \hlred{def} NAME}) &\mathrel{+}= 1 &P(\texttt{NEWLINE \hlred{INDENT} return}) &\mathrel{+}= 1\\
    P(\texttt{return \hlorange{NAME} NEWLINE}) &\mathrel{+}= 1 & P(\texttt{NEWLINE \hlgreen{INDENT} NEWLINE}) &\mathrel{+}= 1
  \end{align*}

  Later, these contextual probabilities will allow us to sample a synthetic corruption matching the distribution of typos in the dataset. We then measure the precision at recovering the originally valid string.

  Here, we conduct an ablation study to compare the effectiveness of PCFG sampling vs. the enumerative sampler. In both experiments, we run the corresponding sampler for 30 seconds (either enumerative or PCFG), then rerank all repairs by n-gram perplexity and measure the Precision@1 for each each across varying edit distances.

  \begin{figure}[h]
    \begin{tikzpicture}[scale=0.9]
  \begin{axis}[
  ybar,
  xlabel={$\Delta(\err\sigma, \sigma)$},
  ylabel={Precision@1},
  title={Enumeration Only},
  axis x line*=bottom,
  axis y line*=left,
  ymin=0,
  ymax=1,
  xtick=data,
  xticklabels={1, 2, 3, 4},
  width=5cm,
  height=4cm,
  enlarge x limits=0.15,
  legend style={at={(0.5,-0.15)},
  anchor=north,legend columns=-1},
  ]
  \addplot[fill=black!30] table[x=Lev, y=P@1] {
    Lev P@1
    1 0.98
    2 0.61
    3 0.35
    4 0.29
  };
  \end{axis}
\end{tikzpicture}
    \begin{tikzpicture}
  \begin{axis}[
  ybar,
  xlabel={$\Delta(\err\sigma, \sigma)$},
  ylabel={Precision},
  title={PCFG Only},
  axis x line*=bottom,
  axis y line*=left,
  ymin=0,
  ymax=1,
  xtick=data,
  width=5cm,
  height=4cm,
  xticklabels={1, 2, 3, 4},
  enlarge x limits=0.15,
  legend style={at={(0.5,-0.15)},
  anchor=north,legend columns=-1},
  ]
  \addplot table[x=Lev, y=P@1] {
    Lev P@1
    1 0.60
    2 0.39
    3 0.10
    4 0.05
  };
  \end{axis}
\end{tikzpicture}
  \end{figure}

  While the overall precision is notably lower for PCFG sampling than enumeration, the average number of samples drawn is also significantly lower, indicating a relatively higher sample efficiency. This illustrates the tradeoff between sample efficiency and diversity, and suggests that a hybrid approach may be the most effective. When the repair language is very large, a PCFG offers a more informed prior, albeit at the cost of lower coverage.

  In general enumeration has an advantage when the CFL is small. For example, if the CFL contains 2,000 sentences, enumeration will recover all 2,000, whereas PCFG sampling may only recover 100 of the most likely samples. However, if the CFL has 200,000 sentences, enumeration may only be able to recover 10,000 uniform random samples and the PCFG may only recover 5,000, but due to the higher sample efficiency, the PCFG samples are more likely to contain the human repair.

  \section{Related Work}\label{sec:related}

  Three important questions arise when repairing syntax errors: (1) is the program broken in the first place? (2) if so, where are the errors located? (3) how should those locations then be altered? In the case of syntax correction, those questions are addressed by three related research areas, (1) parsing, (2) language equations and (3) repair. We survey each of those areas in turn.

  \subsection{Parsing}

  Context-free language (CFL) parsing is the well-studied problem of how to turn a string into a unique tree, with many different algorithms and implementations (e.g., shift-reduce, recursive-descent, LR). Many of those algorithms expect grammars to be expressed in a certain form (e.g., left- or right- recursive) or are optimized for a narrow class of grammars (e.g., regular, linear).

  General CFL parsing allows ambiguity (non-unique trees) and can be formulated as a dynamic programming problem, as shown by Cocke-Younger-Kasami (CYK)~\cite{sakai1961syntax}, Earley~\cite{earley1970efficient} and others. These parsers have roughly cubic complexity with respect to the length of the input string.

  As shown by Valiant~\cite{valiant1975general}, Lee~\cite{lee2002fast} and others, general CFL recognition is in some sense equivalent to binary matrix multiplication, another well-studied combinatorial problem with broad applications, known to be at worst subcubic. This reduction unlocks the door to a wide range of complexity-theoretic and practical speedups to CFL recognition and fast general parsing algorithms.

  Okhotin (2001)~\cite{okhotin2001conjunctive} extends CFGs with language conjunction in \textit{conjunctive grammars}, followed by Zhang \& Su (2017)~\cite{zhang2017context} who apply conjunctive language reachability to dataflow analysis.

  \subsection{Language equations}

  Language equations are a powerful tool for reasoning about formal languages and their inhabitants. First proposed by Ginsburg et al.~\cite{ginsburg1962two} for the ALGOL language, language equations are essentially systems of inequalities with variables representing \textit{holes}, i.e., unknown values, in the language or grammar. Solutions to these equations can be obtained using various fixpoint techniques, yielding members of the language. This insight reveals the true algebraic nature of CFLs and their cousins.

  Being an algebraic formalism, language equations naturally give rise to a kind of calculus, vaguely reminiscent of Leibniz' and Newton's. First studied by Brzozowski~\cite{brzozowski1964derivatives, brzozowski1980equations} and Antimirov~\cite{antimirov1996partial}, one can take the derivative of a language equation, yielding another equation. This can be interpreted as a kind of continuation or language quotient, revealing the suffixes that complete a given prefix. This technique leads to an elegant family of algorithms for incremental parsing~\cite{might2011parsing, adams2016complexity} and automata minimization~\cite{brzozowski1962canonical}. In our setting, differentiation corresponds to code completion.

  In this paper, we restrict our attention to language equations over context-free and weakly context-sensitive languages, whose variables coincide with edit locations in the source code of a computer program, and solutions correspond to syntax repairs. Although prior work has studied the use of language equations for parsing~\cite{might2011parsing}, to our knowledge they have never previously been considered for the purpose of code completion or syntax error correction.

%Rosenkrantz~\cite{rosenkrantz1967matrix} introduces , and

%When a sentence contains so-called \textit{holes}, parsing becomes slightly more challenging. Holes are special tokens that can be filled by words in the grammar. A sentence might have multiple holes, representing a set of contextually valid words. For example, ``\textit{\_ \_ to eat \_}'' which could be filled by, e.g., ``\textit{I like\ldots sushi}'', ``\textit{They want\ldots pizza}'', or ``\textit{We need\ldots something}''.

  \subsection{Syntax repair}

  In finite languages, syntax repair corresponds to spelling correction, a more restrictive and largely solved problem. Schulz and Stoyan~\cite{schulz2002fast} construct a finite automaton that returns the nearest dictionary entry by Levenshtein edit distance. Though considerably simpler than syntax correction, their work shares similar challenges and offers insights for handling more general repair scenarios.

  When a sentence is grammatically invalid, parsing grows more challenging. Like spelling, the problem is to find the minimum number of edits required to transform an arbitrary string into a syntactically valid one, where validity is defined as containment in a (typically) context-free language. Early work, including Irons~\cite{irons1963error} and Aho~\cite{aho1972minimum} propose a dynamic programming algorithm to compute the minimum number of edits required to fix an invalid string. Prior work on error correcting parsing only considers the shortest edit(s), and does not study multiple edits over the Levenshtein ball. Furthermore, the problem of actually generating the repairs is not well-posed, as there are usually many valid strings that can be obtained within a given number of edits. We instead focus on bounded Levenshtein reachability, which is the problem of finding useful repairs within a fixed Levenshtein distance of the broken string, which requires language intersection.

%Bar-Hillel~\cite{bar1961formal} establishes the closure of CFLs under intersection with regular languages, and can be used to construct the corresponding minimum-cost CFG in a straightforward manner. However while generally quite expressive, CFLs are themselves not closed under intersection and have other practical limitations, i.e., are unable to express indentation or variable binding. These limitations motivate us towards more expressive yet still efficiently parsable formalisms.

%Conveniently, Okhotin~\cite{okhotin2001conjunctive} offers exactly the formalism needed: language equations augmented with logical operators like conjunction or disjunction. These operators afford us the flexibility to encode both language union and intersection, which are difficult or impossible to express using a single grammar, as well as incorporate extra-syntactic side-constraints during solving.

%Naively, one might think to enumerate all edits within a certain Levenshtein radius and reject invalid strings. However, this approach is intractable for long strings. Another approach is to compute the \textit{edit distance} between the string and the language as proposed by Aho. This approach is also intractable depending on the size of the grammar. A third approach is to use a language equation to compute the intersection between the Levenshtein hypersphere and the language.

%The literature on parsers is vast and deep, covering far more ground than we could possibly hope to survey. We take inspiration from their findings, but restrict ourselves to a dozen most closely related papers, in four major research areas: (1) formal language theory, (2) constraint satisfaction (3) program synthesis, and (4) error correction. We survey each of these areas in turn.
%
%It was Noam Chomsky who first developed the algebraic theory of \textit{context-free grammars} (CFGs) in 1959~\cite{chomsky1959algebraic}, and since then, CFGs have been the subject of extensive research in formal language theory and program analysis. In particular, we take inspiration from Leslie Valiant~\cite{valiant1975general} who first discovered the connection to matrix multiplication in 1975 and Alexander Okhotin~\cite{okhotin2001conjunctive} who later introduced the idea of \textit{conjunctive grammars} in 2001. More recently, Azimov \& Grigorev~\cite{azimov2018cfpq} and Qirun Zhang shed light into the practical applicability of these ideas and made the theory of CFG reachability more accessible to the general public. In our work, we show how to compile their ideas onto a SAT solver to fix syntax errors, which seems like a perfectly natural extension, but was heretofore previously never considered to the best of our knowledge.

  \subsection{Classical program synthesis}

  There is related work on string constraint solving in the constraint programing literature, featuring solvers like CFGAnalyzer and HAMPI~\cite{kiezun2009hampi}, which consider bounded context free grammars and intersections thereof. Axelson et al. (2008)~\cite{axelsson2008analyzing} has some work on incremental SAT encoding but does not exploit the linear-algebraic structure of parsing, conjunctive reachability nor provide real-time guarantees. D'Antoni et al. (2014) introduces \textit{symbolic automata}~\cite{dantoni2014minimization}, a generalization of finite automata which allow infinite alphabets and symbolic expressions over them. In none of the constraint programming literature we surveyed do any of the approaches employ matrix-based parsing, and therefor do not enjoy the optimality guarantees of Valiant's parser. Our solver can handle context-free and conjunctive grammars with finite alphabets and does not require any special grammar encoding. The matrix encoding makes it particularly amenable to parallelization.

  \subsection{Error correcting codes}

  Our work focuses on errors arising from human factors in computer programming, in particular \textit{syntax error correction}, which is the problem of fixing partially corrupted programs. Modern research on error correction, however, can be traced back to the early days of coding theory when researchers designed \textit{error-correcting codes} (ECCs) to denoise transmission errors induced by external interference, e.g., collision with a high-energy proton, manipulation by an adversary or even typographical mistake. In this context, \textit{code} can be any logical representation for communicating information between two parties (such as a human and a computer), and an ECC is a carefully-designed scheme which ensures that even if some portion of the message should become corrupted, one can still recover the original message by solving a linear system of equations. When designing ECCs, one typically assumes a noise model over a certain event space, such as the Hamming~\cite{titsias2017hamming, dong2023number} or Levenshtein~\cite{becerra2008learning, barlev2021levenshtein} balls, from which we draw inspiration for our work.

  \subsection{Neural program repair}

  The recent success of deep learning has lead to a variety of work on neural program repair~\cite{allamanis2021self, chirkova2021empirical, drain2021generating}. These approaches typically employ Transformer-based neural language models (NLMs) and model the problem as a sequence-to-sequence transformation. Although recent work on circuit lower bounds have cast doubt on the ability of Transformers to truly learn formal languages~\cite{merrill2022saturated, chiang2023tighter}, expressivity aside, these models have been widely adopted for practical program repair tasks. In particular, two papers stand out being most closely related to our own: Break-It-Fix-It (BIFI)~\cite{yasunaga2021break} and Seq2Parse~\cite{sakkas2022seq2parse}. BIFI adapts techniques from semi-supervised machine translation to generate synthetic errors in clean code and fixes them. This reduces the amount of pairwise training data, but may generalize poorly to natural errors. Seq2Parse combines a transformer-based model with an augmented version of the Early parser to suggest error rules, but only suggests a single repair. Our work differs from both in that we suggest multiple repairs with much lower latency, do not require a pairwise repair dataset, and can fix syntax errors in any language with a well-defined grammar. We note our approach is complementary to existing work in neural program repair, and may be used to generate synthetic repairs for training or employ a NLM model to rank its repairs.

  \section{Discussion}\label{sec:discussion}

  The main lesson we draw from our experiments is that it is possible to leverage compute to compete with large language models on practical program repair tasks. Though extremely sample-efficient, their size comes at the cost of higher latency, and domain adaptation requires fine-tuning or retraining on pairwise repairs. Our approach uses a tiny grammar and a relatively cheap ranking metric to achieve comparable accuracy at the same latency. This allows us to repair errors in languages with little to no training data and provides far more flexibility and controllability.

  Our primary insight leading to SoTA precision@\{5,10\} is that repairs are typically concentrated near a small number of edit locations, and by biasing the search toward previously successful edit locations, we can achieve a significant speedup over a na\"ive search. We note this heuristic may not be applicable to all grammars, and it may be possible to construct less natural counterexamples where this heuristic fails, although we consider these unlikely in practice.

  Latency can vary depending on many factors including string length and grammar size, and critically the Levenshtein edit distance. This can be an advantage because in the absence of any contextual or statistical information, syntax and Levenshtein edits are often sufficiently constrained to determine a small number of valid repairs. It is also a limitation because as the number of edits grows, the admissible set grows rapidly and the number of valid repairs may become too large to be useful without a good metric, depending on the language and source code snippet under repair.

  Tidyparse in its current form has several other technical shortcomings: firstly, it does not incorporate any neural language modeling technology at present, an omission we hope to address. Training a language model to predict likely repair locations and rank admissible results could lead to lower overall latency and more natural repairs. We also hope to explore the use of Metropolis-Hastings and determinantal point processes to encourage sampling diversity.

  Secondly, our current method does not specialize language intersection to the grammar family, nor employ Bar-Hillel's~\cite{bar1961formal} construction for REG-CFL intersection, which would lead to a more efficient encoding of Levenshtein-CFL reachability. Furthermore, considering recent extensions of Boolean matrix-based parsing to linear context-free rewriting systems (LCFRS)~\cite{cohen2016parsing}, it may be feasible to search through richer language families within the SAT solver without employing an external stochastic search to generate and validate candidate repairs.

  Lastly and perhaps most significantly, Tidyparse does not incorporate any semantic constraints, so its repairs whilst syntactically admissible, are not guaranteed to be semantically valid. This can be partly alleviated by filtering the results through an incremental compiler or linter, however, the latency introduced may be non-negligible. It is also possible to encode type-based semantic constraints into the solver and we intend to explore this direction more fully in future work.

  We envision a few primary use cases for our tool: (1) helping novice programmers become more quickly familiar with a new programming language, (2) autocorrecting common typos among proficient but forgetful programmers, (3) as a prototyping tool for PL designers and educators, and (4) as a pluggable library or service for parser-generators and language servers. Featuring a grammar editor and built-in SAT solver, Tidyparse helps developers navigate the language design space, visualize syntax trees, debug parsing errors and quickly generate simple examples and counterexamples for benchmarking and testing.

  \section{Conclusion}\label{sec:conclusion}

  Our work, while a case study on syntax repair, is part of a broader trend in program synthesis that investigates how to combine the strengths of formal language theory and machine learning to build more powerful and flexible programming tools. One approach is to filter the outputs of a generative language model to satisfy a formal specification, typically by constrained sampling. Alternatively, some attempt to use a formal language to guide the search for valid programs via a reinforcement learning or hybrid neurosymbolic approach.

  In our work, we take a more pragmatic tack - by incorporating the similarity metric into our formal language, we try to exhaustively enumerate repairs by increasing distance, then use the stochastic language model to rank the resulting solutions by naturalness. The more constraints we can encode into the formal language, the more efficient sampling becomes, and the more precise control we have over the output. This reduces the need for training a large, expensive language model to relearn syntax, and allows us to leverage compute for more efficient search.

  The great compromise in program synthesis is one of efficiency versus expressiveness. The more expressive a language, the more concise and varied the programs it can represent, but the harder those programs are to synthesize without resorting to domain-specific heuristics. Likewise, the simpler a language is to synthesize, the weaker its concision and expressive power.

  Most existing work on program synthesis has focused on general $\lambda$-calculi, or narrow languages such as finite sets or regular expressions. The former are too expressive to be efficiently synthesized or verified, whilst the latter are too restrictive to be useful. In our work, we focus on context-free and mildly context-sensitive grammars, which are expressive enough to capture a variety of useful programming language features, but not so expressive as to be unsynthesizable.

  The second great compromise in program synthesis is that of reusability versus specialization. In programming, as in human communications, there is a vast constellation of languages, each requiring specialized generators and interpreters. Are these languages truly irreconcilable? Or, as Noam Chomsky argues, are these merely dialects of a universal language? \textit{Synthesis} then, might be a misnomer, and more aptly called \textit{recognition}, in the analytic tradition.

  In our work, we argue these two compromises are not mutually exclusive, but complementary and reciprocal. Programs and the languages they inhabit are indeed synthetic, but can be analyzed and reused in the metalanguage of context-free grammars closed under conjunction. Not only does this admit an efficient synthesis algorithm, but allows users to introduce additional constraints without breaking compositionality, one of the most sacred tenets in programming language design.

  Over the last few years, there has been a surge of progress in applying language models to write programs. That work is primarily based on methods from differential calculus and continuous optimization, leading to the so-called \textit{naturalness hypothesis}, which suggests programming languages are not so different from natural ones. In contrast, programming language theory takes the view that languages are essentially discrete and finitely-generated sets, although perhaps uncomputably large ones, governed by logical calculi. Programming, thus viewed, is more like a mathematical exercise in constraint satisfaction, an oft-overlooked but unavoidable aspect of translating abstract ideas into computing machinery. These constraints naturally arise at various stages of syntax validation, type-checking and runtime verification, and help to ensure programs fulfill their intended purpose.

  As our work shows, not only is linear algebra over finite fields an expressive language for probabilistic inference, but also an efficient framework for inference on languages themselves. Borrowing analysis techniques from multilinear algebra and tensor completion in the machine learning setting, we develop an equational theory that allows us to translate various decision problems on formal languages into a system of inequalities over finite fields. As a practical consequence, this means we can efficiently encode a number of problems in parsing, code completion and program repair using SAT solvers, which are known to be highly efficient, scalable and flexible to domain-specific constraints. We demonstrate the effectiveness of our approach in a variety of practical scenarios, including code completion and program repair for linear conjunctive languages, and show that our approach is competitive with state-of-the-art methods in terms of both accuracy and efficiency. In contrast with LL and LR-style parsers, our technique can recover partial forests from invalid strings, and handle arbitrary conjunctive languages. In future work, we hope to extend our method to more natural grammars like PCFG, LCFRS and other mildly context-sensitive languages.

  Syntax correction tools should be as user-friendly and widely-accessible as autocorrection tools in word processors. From a practical standpoint, we argue it is possible to reduce disruption from manual syntax repair and improve the efficiency of working programmers by driving down the latency needed to synthesize an acceptable repair. In contrast with program synthesizers that require intermediate editor states to be well-formed, our synthesizer does not impose any constraints on the code itself being written and can be used in a live programming environment.

  Despite its computational complexity, the design of the tool itself is relatively simple. Tidyparse accepts a linear conjunctive language and a string. If the string is valid, it returns the parse forest, otherwise, it returns a set of repairs, ordered by their perplexity. Tidyparse compiles the grammar and candidate string onto a discrete dynamical system using an extended version of Valiant's algorithm and solves for its fixedpoints using an incremental SAT solver. By allowing the string to contain holes, repairs may contain either concrete tokens or nonterminals, which can be expanded by the user or a neural-guided search procedure to produce a valid program. This approach to parsing has many advantages, enabling us to repair syntax errors, correct typos and recover from errors in real time, as well as being provably sound and complete with respect to the grammatical specification. It is also compatible with neural program synthesis and repair techniques and shares the same masked language modeling (MLM) target used by Transformer-based LLMs.

  We have implemented our approach as an IDE plugin and demonstrated its viability as a practical tool for realtime programming. A considerable amount of effort was devoted to supporting fast error correction functionality. Tidyparse is capable of generating repairs for invalid code in a range of practical languages with very little downstream language integration required. We plan to continue expanding its grammar and autocorrection functionality to cover a broader range of languages and hope to conduct a more thorough user study to validate its effectiveness.

%\subsection{Ranking}
%
%Since the number of solutions can be very large, we can use a language model to rank the results maximizing likelihood, or minimizing perplexity, subject to the constraints. This ranking can be used to guide the propagation, sample the choice function, sample hole locations or as a post-processing step after a fixed timeout has expired.

%Alternatively, this expression can be rewritten as a polynomial over GF(2):
%
%\[
%  (v_1 \times w_2 + y_3 + 1) \Leftrightarrow [S \in Y] \Leftrightarrow [Q R \in L(G)]
%\]

%
% ---- Bibliography ----
%
% BibTeX users should specify bibliography style 'splncs04'.
% References will then be sorted and formatted in the correct style.
%
%  \bibliographystyle{splncs04}
  \bibliography{../bib/acmart}
\end{document}