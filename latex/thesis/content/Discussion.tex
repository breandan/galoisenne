\chapter{\rm\bfseries Discussion}
\label{ch:discussion}

Our work shows a surprising connection between advanced structured prediction and formal language learning. Large language models are very sample efficient, but are expensive to train. Verifying the correctness of a large language model is a hard problem and open research question. However, we show that if the specification can be expressed as a context-free grammar or finite intersection thereof, one can easily force the model to produce only valid text. Furthermore, if one is careful in their modeling assumptions, they can ensure every valid sentence has a nonzero probability of being generated.

Not only from a safety perspective, pairing constraints with a weak autoregressive model such as a low-order Markov chain can be competitive with SoTA neural language models on certain kinds of sequence modeling tasks. Most of the LLM is dedicated to [poorly] relearning syntax, and if we can remove the burden of modeling the syntax, we can use constrained decoding and a weak model to obtain higher precision at a fraction of the cost.

This manifests as a practical tool for repairing syntax in an IDE, as well as an interesting case study on language modeling. We can treat the grammar as an incremental verifier. If you have access to such a verifier (which allows you to preemptively reject continuations of partial trajectories before evaluating a full rollout) and massively parallelize the sampler, then you can often saturate the entire sample space or finite slices thereof. Together with a cheap ranking function, this method is highly competitive with large, expensive LLMs.

%\mcgillguidelines The discussion of findings must be in line with disciplinary expectations. A comprehensive discussion is expected to be a minimum of 10 pages, double-spaced for doctoral students and a minimum of 5 pages, double-spaced for Master’s students (including figures, images, and tables). It pertains to the entirety of a thesis. The discussion of findings should provide an final, overarching summary of study themes, limitations, and future directions.


%In the case of a manuscript-based thesis, the comprehensive discussion should encompass all of the chapters of the thesis and should not be a repetition of the individual chapters. This section can be used to address issues not sufficiently covered in the preceding chapters or papers (e.g., critiques raised by reviewers that could not be incorporated into published works, or reintroducing discussion arguments removed from published papers upon reviewer request). This section can also be used to elaborate on the practical/applied aspects of published findings in a manner that is more accessible to less expert readers.