\chapter*{\rm\bfseries Abstract}
\label{ch:abstraten}

%\href{https://www.mcgill.ca/gps/thesis/thesis-guidelines/preparation}{Official McGill Guidelines}: If the language of the thesis is neither English nor French (only allowed for specific language Units) then a third abstract in the language of the thesis is required.
%
%Abstracts in English and French are mandatory and must be text only, i.e. no images, special characters (apart from the West European character set excluding the “Œ” and “œ”), chemical or mathematical formulae, or special formatting (e.g. lists, tables). Abstracts have a maximum limit of 4000 characters.

We introduce an edit calculus for correcting syntax errors in arbitrary context-free languages, and by extension, any programming language with a context-free grammar. Syntax errors with a small repair seldom have many unique small repairs, which can usually be enumerated up to a small edit distance then quickly reranked. Our work places a heavy emphasis on precision: the enumerated set must contain every possible repair within a given radius and no invalid repairs. To do so, we construct a grammar representing the language intersection between a Levenshtein automaton and a context-free grammar, then decode it in order of probability. This produces an ordered set of repairs that contains with high probability the intended revision.