\chapter*{\rm\bfseries Terminology}
\label{ch:terminology}

Technical and vernacular collisions induce a strange semantic synesthesia, e.g., complete, consistent, kernel, reflexive, regression, regular, sound. The intension may be distantly related to standard English, but if one tries to interpret such jargon colloquially, there is no telling how far astray they will go. For this reason, we provide a glossary of terms to help the non-technical reader navigate the landscape of this thesis.

\begin{itemize}
    \item \textbf{Automaton}: A mathematical model of computation that can occupy one of a finite number of states at any given time, and makes transitions between states according to a set of rules.
    \item \textbf{Deterministic}: A property of a system that, given the same input, will always produce the same output.
    \item \textbf{Grammar}: A set of rules that define the syntax of a language.
    \item \textbf{Language}: A set of words generated by a grammar. For the purposes of this thesis, the language can be finite or infinite.
    \item \textbf{Word}: A member of a language, consisting of a sequence of terminals. For the purposes of this thesis, a word is always finite.
    \item \textbf{Terminal}: A single token from an alphabet. For the purposes of this thesis, the alphabet is always finite.
    \item \textbf{Intersection}: The set of elements common to two or more sets.
    \item \textbf{Probabilistic}: A property of a system that, given the same input, may produce different outputs.
    \item \textbf{Theory}: A set of sentences in a formal language.
\end{itemize}