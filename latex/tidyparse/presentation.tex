\documentclass{beamer}
\usetheme{Madrid}
\beamertemplatenavigationsymbolsempty

\usepackage{dsfont}
\usepackage{stmaryrd}
\usepackage{colortbl}
\usepackage{hyperref}

\usepackage{amsmath}
\DeclareMathOperator*{\argmax}{argmax}
\DeclareMathOperator*{\argmin}{argmin}
\usepackage{amssymb}

\usepackage[dvipsnames, table]{xcolor}
\usepackage{textcomp}

% Packages
\usepackage[pdf]{graphviz}
\usepackage{mathrsfs}

\newcommand*\circled[1]{\tikz[baseline=-0.1cm]{
  \node[shape=circle,draw,inner sep=0.48pt] (char) {\fontsize{7}{12}\textsf{#1}};}}

\DeclareMathAlphabet{\mathcal}{OMS}{cmsy}{m}{n}
\usepackage{cancel}
\newcommand\ccancel[2][red]{\renewcommand\CancelColor{\color{#1}}\cancel{#2}}
\newcommand{\nDownarrow}{\ensuremath{\text{ }\cancel{\Downarrow}\text{ }}}
\usepackage{centernot}

\usepackage{pgfplots, pgfplotstable}
\pgfplotsset{compat=1.7}
\usepgfplotslibrary{fillbetween}
\usetikzlibrary{patterns}
\pgfmathdeclarefunction{gauss}{2}{\pgfmathparse{1/(#2*sqrt(2*pi))*exp(-((x-#1)^2)/(2*#2^2))}}
\pgfmathdeclarefunction{nil}{1}{\pgfmathparse{0.001}}

\usepackage{arydshln}
\usepackage{adjustbox}
\usepackage{enumerate}
\usepackage{enumitem}
\usepackage{tikz-cd}
\usetikzlibrary{calc}
\usepackage{amsfonts}
%\usepackage{prooftrees}
\usepackage{bussproofs}
\renewcommand{\sectionautorefname}{\S}
\renewcommand{\subsectionautorefname}{\S}
\usepackage{float}

\usepackage{tikz-3dplot}
\usetikzlibrary{3d}
\usetikzlibrary{calligraphy}
\newif\ifshowcellnumber
\showcellnumbertrue

\usepackage{algorithm}
\usepackage{algpseudocode}
\usepackage{algorithmicx}
\usepackage{sourcecodepro}
\usepackage{tikz-qtree}
\usepackage{amsthm}
\usepackage{bm}
\usetikzlibrary{bayesnet}
\usetikzlibrary{arrows}
\usepackage{subcaption}
\usetikzlibrary{backgrounds}
\usetikzlibrary{tikzmark}

\newcommand{\E}{\mathbb{E}}
\newcommand{\Var}{\mathrm{Var}}
\newcommand{\Cov}{\mathrm{Cov}}

\newcommand{\CompOrder}{\mathcal{O}}
\def\graphspace{\mathbf{G}}
\def\Uniform{\mbox{\rm Uniform}}
\def\Gaussian{\mbox{\rm Gaussian}}
\def\Bernoulli{\mbox{\rm Bernoulli}}
\def\Dirichlet{\mbox{\rm Dirichlet}}

\usepackage{mathtools}% superior to amsmath
\usepackage{tikz}
% Packages
\usepackage{listings}
\DeclareRobustCommand{\hlred}[1]{{\sethlcolor{pink}\hl{#1}}}
\usepackage{fontspec}

\setmonofont[Scale=0.8]{JetBrainsMono}[
  Contextuals={Alternate},
  Path=./font/,
  Extension = .ttf,
  UprightFont=*-Regular,
  BoldFont=*-Bold,
  ItalicFont=*-Italic,
  BoldItalicFont=*-BoldItalic
]

\usepackage[skins,breakable,listings]{tcolorbox}

\lstdefinelanguage{kotlin}{
  comment=[l]{//},
  commentstyle={\color{gray}\ttfamily},
  emph={delegate, filter, firstOrNull, forEach, it, lazy, mapNotNull, println, repeat, assert, with, head, tail, len, return@},
  numberstyle=\noncopyable,
  identifierstyle=\color{black},
  keywords={abstract, actual, as, as?, break, by, class, companion, continue, data, do, dynamic, else, enum, expect, false, final, for, fun, get, if, import, in, infix, interface, internal, is, null, object, open, operator, override, package, private, public, return, sealed, set, super, suspend, this, throw, true, try, catch, typealias, val, var, vararg, when, where, while, tailrec, reified},
  keywordstyle={\bfseries},
  morecomment=[s]{/*}{*/},
  morestring=[b]",
  morestring=[s]{"""*}{*"""},
  ndkeywords={@Deprecated, @JvmField, @JvmName, @JvmOverloads, @JvmStatic, @JvmSynthetic, Array, Byte, Double, Float, Boolean, Int, Integer, Iterable, Long, Runnable, Short, String, int},
  ndkeywordstyle={\bfseries},
  sensitive=true,
  stringstyle={\ttfamily},
  literate={`}{{\char0}}1,
  escapeinside={(*@}{@*)}
}
\lstdefinelanguage{tidy}{
  comment=[l]{//},
  commentstyle={\color{gray}\ttfamily},
  emph={|, ->, ---},
  emphstyle={\color{red}},
  identifierstyle=\color{black},
  keywords={\|, ->, ---},
  otherkeywords={|,->},
  morekeywords={|,->},
  keywordstyle={\color{blue}\bfseries},
  morecomment=[s]{/*}{*/},
  morestring=[b]",
  morestring=[s]{"""*}{*"""},
  ndkeywords={@Deprecated, @JvmField, @JvmName, @JvmOverloads, @JvmStatic, @JvmSynthetic, Array, Byte, Double, Float, Int, Integer, Iterable, Long, Runnable, Short, String},
  ndkeywordstyle={\color{orange}\bfseries},
  sensitive=true,
  stringstyle={\color{green}\ttfamily},
  literate={`}{{\char0}}1
}

%%%%%%%%%%%%%%%%%%%%%%%%%%%%%%%%%%%%%%%%%%%
%
% Color boxes
%
%%%%%%%%%%%%%%%%%%%%%%%%%%%%%%%%%%%%%%%%%%%

\tcbset{
  enhanced jigsaw,
  breakable,
  listing only,
%  boxsep=-1pt,
%  top=-1pt,
  bottom=0.1cm,
  right=0.5cm,
  overlay first={
    \node[black!50] (S) at (frame.south) {\Large\ding{34}};
    \draw[dashed,black!50] (frame.south west) -- (S) -- (frame.south east);
  },
  overlay middle={
    \node[black!50] (S) at (frame.south) {\Large\ding{34}};
    \draw[dashed,black!50] (frame.south west) -- (S) -- (frame.south east);
    \node[black!50] (S) at (frame.north) {\Large\ding{34}};
    \draw[dashed,black!50] (frame.north west) -- (S) -- (frame.north east);
  },
  overlay last={
    \node[black!50] (S) at (frame.north) {\Large\ding{34}};
    \draw[dashed,black!50] (frame.north west) -- (S) -- (frame.north east);
  },
  before={\par\vspace{5pt}},
  after={\par\vspace{\parskip}\noindent}
}

\definecolor{slightgray}{rgb}{0.90, 0.90, 0.90}

\usepackage{soul}
\makeatletter
\def\SOUL@hlpreamble{%
  \setul{}{3.0ex}%
  \let\SOUL@stcolor\SOUL@hlcolor%
  \SOUL@stpreamble%
}
\makeatother

\newcommand{\inline}[1]{%
  \begingroup%
  \sethlcolor{slightgray}%
  \hl{\ttfamily\footnotesize #1}%
  \endgroup
}

\newcommand{\tinline}[1]{%
  \begingroup%
  \sethlcolor{slightgray}%
  \hl{\ttfamily\tiny #1}%
  \endgroup
}

\newtcblisting{halftidyinput}[1][]{%
  left skip=0.7cm,
  width=6cm,
%  left=-0.01cm,
  top=-0.1cm,
  bottom=-0.35cm,
  listing options={
    language=tidy,
    basicstyle=\ttfamily\small,
%numberstyle=\footnotesize,
    showstringspaces=false,
    tabsize=2,
    breaklines=true,
    numbers=none,
    inputencoding=utf8,
    escapeinside={(*@}{@*)},
    #1
  },
  underlay unbroken and first={%
    \path[draw=none] (interior.north west) rectangle node[white]{\includegraphics[width=4mm]{../figures/tidyparse_logo.png}} ([xshift=-10mm,yshift=-7mm]interior.north west);
  }
}

\newtcblisting{wholetidyinput}[1][]{%
  left skip=0.7cm,
  top=0.1cm,
  middle=0mm,
  boxsep=0mm,
  listing options={
    language=tidy,
    basicstyle=\ttfamily\small,
%numberstyle=\footnotesize,
    showstringspaces=false,
    tabsize=2,
    breaklines=true,
    numbers=none,
    inputencoding=utf8,
    escapeinside={(*@}{@*)},
    #1
  },
  underlay unbroken and first={%
      \path[draw=none] (interior.north west) rectangle node[white]{\includegraphics[width=4mm]{../figures/tidyparse_logo.png}} ([xshift=-10mm,yshift=-9mm]interior.north west);
  }
}

\definecolor{A}{RGB}{6,150,104}
\definecolor{B}{RGB}{196,74,137}
\definecolor{C}{RGB}{117,237,133}
\definecolor{D}{RGB}{246,46,243}
\definecolor{E}{RGB}{89,162,12}
\definecolor{F}{RGB}{113,12,158}
\definecolor{G}{RGB}{191,205,142}
\definecolor{H}{RGB}{51,58,158}
\definecolor{I}{RGB}{244,212,3}
\definecolor{J}{RGB}{37,36,249}
\definecolor{K}{RGB}{253,165,71}
\definecolor{L}{RGB}{27,81,29}
\colorlet{LA}{A!30}
\colorlet{LB}{B!30}
\colorlet{LC}{C!30}
\colorlet{LD}{D!30}
\colorlet{LE}{E!30}
\colorlet{LF}{F!30}
\colorlet{LG}{G!30}
\colorlet{LH}{H!30}
\colorlet{LI}{I!30}
\colorlet{LJ}{J!30}
\colorlet{LK}{K!30}
\colorlet{LL}{L!30}
\newcommand{\hiliA}[1]{%
  \colorbox{LA}{$#1$}}
\newcommand{\hiliB}[1]{%
  \colorbox{LB}{$#1$}}
\newcommand{\hiliC}[1]{%
  \colorbox{LC}{$#1$}}
\newcommand{\hiliD}[1]{%
  \colorbox{LD}{$#1$}}
\newcommand{\hiliE}[1]{%
  \colorbox{LE}{$#1$}}
\newcommand{\hiliF}[1]{%
  \colorbox{LF}{$#1$}}
\newcommand{\hiliG}[1]{%
  \colorbox{LG}{$#1$}}
\newcommand{\hiliH}[1]{%
  \colorbox{LH}{$#1$}}
\newcommand{\hiliI}[1]{%
  \colorbox{LI}{$#1$}}
\newcommand{\hiliJ}[1]{%
  \colorbox{LJ}{$#1$}}
\newcommand{\hiliK}[1]{%
  \colorbox{LK}{$#1$}}
\newcommand{\hiliL}[1]{%
  \colorbox{LL}{$#1$}}
\newcommand{\highlight}[1]{%
  \colorbox{lgray}{$#1$}}
\colorlet{lred}{red!30}
\colorlet{lorange}{orange!30}
\colorlet{lgreen}{green!30}
\colorlet{lgray}{black!15}
\colorlet{dgray}{black!75}
\DeclareRobustCommand{\hlred}[1]{{\sethlcolor{lred}\hl{#1}}}
\DeclareRobustCommand{\hlorange}[1]{{\sethlcolor{lorange}\hl{#1}}}
\DeclareRobustCommand{\hlgreen}[1]{{\sethlcolor{lgreen}\hl{#1}}}
\DeclareRobustCommand{\hlgray}[1]{{\sethlcolor{lgray}\hl{#1}}}
\DeclareRobustCommand{\caret}[1]{{\sethlcolor{dgray}\textcolor{white}{\hl{#1}}}}

\usepackage{url}
\usepackage{qtree}

\usepackage{filecontents}
\usepackage{pstricks-add}
\usepackage{emoji}
\usepackage{alltt}
\usepackage{nicematrix}
\usepackage{graphicx}
\usepackage{ulem}
\usepackage{upquote}
\tikzstyle{every picture}+=[remember picture]
\usepackage{menukeys}
\pgfplotstableread[col sep=comma,]{timings_loc.csv}\loctimings
\pgfplotstableread[col sep=comma,]{timings_unloc.csv}\unloctimings

\makeatletter
\DeclareRobustCommand{\cev}[1]{%
  {\mathpalette\do@cev{#1}}%
}
\newcommand{\do@cev}[2]{%
  \vbox{\offinterlineskip
  \sbox\z@{$\m@th#1 x$}%
  \ialign{##\cr
  \hidewidth\reflectbox{$\m@th#1\vec{}\mkern4mu$}\hidewidth\cr
  \noalign{\kern-\ht\z@}
    $\m@th#1#2$\cr
  }%
  }%
}
\makeatother

\makeatletter
\DeclareRobustCommand{\pder}[1]{%
  \@ifnextchar\bgroup{\@pder{#1}}{\@pder{}{#1}}}
\newcommand{\@pder}[2]{\frac{\partial#1}{\partial#2}}
\makeatother

\newcommand{\shup}{\shortuparrow}
\newcommand{\shri}{\shortrightarrow}
\newcommand{\shur}{\shup\hspace{-5pt}\shri}

\makeatletter
\def\squiggly{\bgroup \markoverwith{\textcolor{red}{\lower3\p@\hbox{\sixly \char58}}}\ULon}
\makeatother

\newcommand{\err}[1]{\smash{\squiggly{#1}{}}}
\newcommand{\stirlingii}{\genfrac{\{}{\}}{0pt}{}}

%======== Arrows =========
\newcommand{\knightarrow}{
  \tikz{
    \fill (0pt,0pt) circle [radius = 1pt];
    \fill (0pt,6pt) circle [radius = 1pt];
    \fill (6pt,0pt) circle [radius = 1pt];
    \fill (6pt,6pt) circle [radius = 1pt];
    \fill (12pt,0pt) circle [radius = 1pt];
    \fill (12pt,6pt) circle [radius = 1pt];
    \fill (6pt,0pt) circle [radius = 1pt];
    \fill (12pt,0pt) circle [radius = 1pt];
    \draw [-to] (0pt,0pt) -- (12pt,6pt);
  }
}

\newcommand{\kingarrow}{
  \tikz{
    \fill (0pt,0pt) circle [radius = 1pt];
    \fill (6pt,0pt) circle [radius = 1pt];
    \fill (0pt,6pt) circle [radius = 1pt];
    \fill (6pt,6pt) circle [radius = 1pt];
    \draw [-to] (0pt,0pt) -- (6pt,6pt);
    \draw [-to] (0pt,0pt) -- (0pt,6pt);
    \draw [-to] (0pt,0pt) -- (6pt,0pt);
  }
}

\newcommand{\knightkingarrow}{
  \tikz{
    \fill (0pt,0pt) circle [radius = 1pt];
    \fill (0pt,6pt) circle [radius = 1pt];
    \fill (6pt,0pt) circle [radius = 1pt];
    \fill (6pt,6pt) circle [radius = 1pt];
    \fill (12pt,0pt) circle [radius = 1pt];
    \fill (12pt,6pt) circle [radius = 1pt];
    \draw [-to] (0pt,0pt) -- (6pt,6pt);
    \draw [-to] (0pt,0pt) -- (0pt,6pt);
    \draw [-to] (0pt,0pt) -- (6pt,0pt);
    \draw [-to] (0pt,0pt) -- (12pt,6pt);
  }
}

%======== Arrows =========

\usetikzlibrary{decorations.pathreplacing,automata,calc,positioning,matrix,fit}
\usepackage{wrapfig}

\newcommand{\mkTrellis}[1]{
  \begin{tikzpicture}
    \def\dx{20pt}
    \def\dy{30pt}
    \newcounter{i}
    \stepcounter{i}
    \node[circle, draw, fill=black!30] (\arabic{i}) at (0,0){};
    \foreach [count=\i] \x in {2,...,#1}{
      \pgfmathsetmacro{\lox}{\x-1}%
      \pgfmathsetmacro{\loxt}{\x-3}%
      \foreach [count=\j] \xx in {-\lox,-\loxt,...,\lox}{
        \pgfmathsetmacro{\jj}{\j-1}%
        \stepcounter{i}
        \pgfmathsetmacro{\kk}{\xx-2}%
        \pgfmathsetmacro{\lbl}{\lox!/(\jj!*(\lox-\jj)!)}
        \ifnum\x<\kk
        \pgfmath\node[circle, draw]  (\arabic{i}) at (\xx*\dx, -\lox*\dy) {};
        \else
        \pgfmath\node[circle, draw, fill=black!30]  (\arabic{i}) at (\xx*\dx, -\lox*\dy) {};
        \fi
      }
    }
    \newcounter{z}
    \newcounter{xn}
    \newcounter{xnn}
    \pgfmathsetmacro{\maxx}{#1 - 1}
    \foreach \x in {1,...,\maxx}{
      \foreach \xx in {1,...,\x}{
        \stepcounter{z}
        \setcounter{xn}{\arabic{z}}
        \addtocounter{xn}{\x}
        \setcounter{xnn}{\arabic{xn}}
        \stepcounter{xnn}
        \draw [<-] (\arabic{z}) -- (\arabic{xn});
        \draw [<-] (\arabic{z}) -- (\arabic{xnn});
      }
    }
  \end{tikzpicture}
}

\newcommand{\dx}{20pt}
\newcommand{\dy}{30pt}
\newcounter{i}
\newcounter{z}
\newcounter{xn}
\newcounter{xnn}
\newcommand{\mkTrellisAppend}[1]{
  \begin{tikzpicture}
    \setcounter{i}{0}
    \setcounter{z}{0}
    \setcounter{xn}{0}
    \setcounter{xnn}{0}
    \stepcounter{i}
    \node[circle, draw] (\arabic{i}) at (0,0){};
    \foreach [count=\i] \x in {2,...,#1}{
      \pgfmathsetmacro{\lox}{\x-1}%
      \pgfmathsetmacro{\loxt}{\x-3}%
      \foreach [count=\j] \xx in {-\lox,-\loxt,...,\lox}{
        \pgfmathsetmacro{\jj}{\j-1}%
        \stepcounter{i}
        \pgfmathsetmacro{\kk}{\xx+2}%
        \pgfmathsetmacro{\lbl}{\lox!/(\jj!*(\lox-\jj)!)}
        \ifnum\x>\kk
        \pgfmath\node[circle, draw, fill=black!30]  (\arabic{i}) at (\xx*\dx, -\lox*\dy) {};
        \else
        \pgfmath\node[circle, draw]  (\arabic{i}) at (\xx*\dx, -\lox*\dy) {};
        \fi
      }
    }
    \pgfmathsetmacro{\maxx}{#1 - 1}
    \foreach \x in {1,...,\maxx}{
      \foreach \xx in {1,...,\x}{
        \stepcounter{z}
        \setcounter{xn}{\arabic{z}}
        \addtocounter{xn}{\x}
        \setcounter{xnn}{\arabic{xn}}
        \stepcounter{xnn}
        \draw [<-] (\arabic{z}) -- (\arabic{xn});
        \draw [<-] (\arabic{z}) -- (\arabic{xnn});
      }
    }
  \end{tikzpicture}
}

\newcommand{\mkTrellisInsert}[1]{
  \begin{tikzpicture}
    \setcounter{i}{0}
    \setcounter{z}{0}
    \setcounter{xn}{0}
    \setcounter{xnn}{0}
    \stepcounter{i}
    \node[circle, draw] (\arabic{i}) at (0,0){};
    \foreach [count=\i] \x in {2,...,#1}{
      \pgfmathsetmacro{\lox}{\x-1}%
      \pgfmathsetmacro{\loxt}{\x-3}%
      \foreach [count=\j] \xx in {-\lox,-\loxt,...,\lox}{
        \pgfmathsetmacro{\jj}{\j-1}%
        \stepcounter{i}
        \pgfmathsetmacro{\mp}{\xx+#1}%
        \pgfmathsetmacro{\mq}{\xx+\x}%
        \pgfmathsetmacro{\lbl}{\lox!/(\jj!*(\lox-\jj)!)}
        \ifnum\x>\mp
        \pgfmath\node[circle, draw, fill=black!30]  (\arabic{i}) at (\xx*\dx, -\lox*\dy) {};
        \else
        \ifnum#1<\mq
        \pgfmath\node[circle, draw, fill=black!30]  (\arabic{i}) at (\xx*\dx, -\lox*\dy) {};
        \else
        \pgfmath\node[circle, draw]  (\arabic{i}) at (\xx*\dx, -\lox*\dy) {};
        \fi
        \fi

      }
    }
    \pgfmathsetmacro{\maxx}{#1 - 1}
    \foreach \x in {1,...,\maxx}{
      \foreach \xx in {1,...,\x}{
        \stepcounter{z}
        \setcounter{xn}{\arabic{z}}
        \addtocounter{xn}{\x}
        \setcounter{xnn}{\arabic{xn}}
        \stepcounter{xnn}
        \draw [<-] (\arabic{z}) -- (\arabic{xn});
        \draw [<-] (\arabic{z}) -- (\arabic{xnn});
      }
    }
  \end{tikzpicture}
}

\usetikzlibrary{automata, positioning, arrows}

\newcommand{\nobarfrac}{\genfrac{}{}{0pt}{}}
\pgfplotstableread[col sep=comma,]{timings_loc.csv}\loctimings
\pgfplotstableread[col sep=comma,]{timings_unloc.csv}\unloctimings
\pgfplotstableread[col sep=comma,]{natural_errors.csv}\naturalerrors
\pgfplotstableread[col sep=comma,]{synthetic_errors.csv}\syntheticerrors


\usepackage{pgf-soroban}
\usetikzlibrary{shapes.geometric,calc,decorations.text,bayesnet,arrows,backgrounds}
\usepackage{circledsteps}
\usepackage{epigraph}
\usepackage{array}
\setmonofont{JetBrains Mono}[
  Contextuals = Alternate,
  Ligatures = TeX,
]
\lstset{
  basicstyle = \ttfamily,
  columns = flexible,
}
\makeatletter
\renewcommand*\verbatim@nolig@list{}
\makeatother
\usepackage{pmboxdraw}
\usetikzlibrary{cd}
\usepackage{pifont}
\newcommand{\cmark}{\ding{51}}%
\newcommand{\xmark}{\ding{55}}%

\setbeamertemplate{footline}
{
  \leavevmode%
  \hbox{%
    \begin{beamercolorbox}[wd=.25\paperwidth,ht=2.25ex,dp=1ex,center]{author in head/foot}%
      \usebeamerfont{author in head/foot}\insertshortauthor{}{~~(\insertshortinstitute)}
    \end{beamercolorbox}%
    \begin{beamercolorbox}[wd=.5\paperwidth,ht=2.25ex,dp=1ex,center]{title in head/foot}%
      \usebeamerfont{title in head/foot}\insertshorttitle
    \end{beamercolorbox}%
    \begin{beamercolorbox}[wd=.3\paperwidth,ht=2.25ex,dp=1ex,right]{date in head/foot}%
      \usebeamerfont{date in head/foot}\insertshortdate{}\hspace*{2em}
      \insertframenumber{} / \inserttotalframenumber\hspace*{2ex}
    \end{beamercolorbox}}%
  \vskip0pt%
}
\makeatother

\makeatletter
\let\HL\hl
\renewcommand\hl{%
  \let\set@color\beamerorig@set@color
  \let\reset@color\beamerorig@reset@color
  \HL}
\makeatother

\newcommand\ddd{\Ddots}
\newcommand\vdd{\Vdots}
\newcommand\cdd{\Cdots}
\newcommand\lds{\ldots}
\newcommand\vno{\varnothing}
\newcommand{\ts}[1]{\textsuperscript{#1}}
\newcommand\non{1\ts{st}}
\newcommand\ntw{2\ts{nd}}
\newcommand\nth{3\ts{rd}}
\newcommand\nfo{4\ts{th}}
\newcommand\nfi{5\ts{th}}
\newcommand\nsi{6\ts{th}}
\newcommand\nse{7\ts{th}}
\newcommand{\vs}[1]{\sigma_{#1}^{\shur}}
\newcommand\rcr{\rowcolor{black!15}}
\newcommand\rcw{\rowcolor{white}}
\newcommand\pcd{\cdot}
\newcommand\pcp{\phantom\cdot}
\newcommand\ppp{\phantom{\nse}}
\newcommand\hole{\underline{\hspace{0.25cm}}}

\title[A Pragmatic Approach to Syntax Repair]{A Pragmatic Approach to Syntax Repair}
\titlegraphic{\includegraphics[width=0.2\textwidth]{../figures/tidyparse_logo.png}}
\author[Considine, Guo, Si]{\textbf{Breandan Considine}, Jin Guo, Xujie Si}
\institute[McGill]{
  McGill University, Mila IQIA\\
  \medskip
  \textit{bre@ndan.co}
}
\date{\today}

\begin{document}
\begin{frame}
  \titlepage
\end{frame}

%\begin{frame}{Overview}
%  \tableofcontents
%\end{frame}

\begin{frame}[fragile]{Can you spot the error?}
  \begin{center}
    \begin{tabular}{|m{5.5cm}|m{5.5cm}|}
      \hline \rule{0pt}{2.5ex}\textbf{Original code}\rule[-1ex]{0pt}{2ex} &  \rule{0pt}{2.5ex}\textbf{Human repair}\rule[-1ex]{0pt}{2ex} \\\hline
      \begin{lstlisting}[escapechar=!, basicstyle=\linespread{1.3}\ttfamily\footnotesize]

  newlist = []
  i = set([5, 3, 1)]
  z = set([5, 0, 4)]


      \end{lstlisting} & \begin{lstlisting}[escapechar=!, basicstyle=\linespread{1.3}\ttfamily\footnotesize]

      \end{lstlisting} \\\hline
    \end{tabular}
  \end{center}
\end{frame}

\begin{frame}[fragile]{Can you spot the error?}
  \begin{center}
    \begin{tabular}{|m{5.5cm}|m{5.5cm}|}
      \hline \rule{0pt}{2.5ex}\textbf{Original code}\rule[-1ex]{0pt}{2ex} &  \rule{0pt}{2.5ex}\textbf{Human repair}\rule[-1ex]{0pt}{2ex} \\\hline
      \begin{lstlisting}[escapechar=!, basicstyle=\linespread{1.3}\ttfamily\footnotesize]

  newlist = []
  i = set([5, 3, 1)]
  z = set([5, 0, 4)]

      \end{lstlisting} & \begin{lstlisting}[escapechar=!, basicstyle=\linespread{1.3}\ttfamily\footnotesize]

  newlist = []
  i = set([5, 3, 1!\hlorange{])}!
  z = set([5, 0, 4!\hlorange{])}!

      \end{lstlisting} \\\hline
    \end{tabular}
  \end{center}
\end{frame}

\begin{frame}[fragile]{Can you spot the error?}
  \begin{center}
    \begin{tabular}{|m{5.5cm}|m{5.5cm}|}
      \hline \rule{0pt}{2.5ex}\textbf{Original code}\rule[-1ex]{0pt}{2ex} &  \rule{0pt}{2.5ex}\textbf{Human repair}\rule[-1ex]{0pt}{2ex} \\\hline
      \begin{lstlisting}[escapechar=!, basicstyle=\linespread{1.3}\ttfamily\footnotesize]

  def average(values):
    if values == (1,2,3):
      return (1+2+3)/3
    else if values == (-3,2):
      return (-3+2+8-1)/4

      \end{lstlisting} & \begin{lstlisting}[escapechar=!, basicstyle=\linespread{1.3}\ttfamily\footnotesize]

      \end{lstlisting} \\\hline
    \end{tabular}
  \end{center}
\end{frame}

\begin{frame}[fragile]{Can you spot the error?}
  \begin{center}
    \begin{tabular}{|m{5.5cm}|m{5.5cm}|}
      \hline \rule{0pt}{2.5ex}\textbf{Original code}\rule[-1ex]{0pt}{2ex} &  \rule{0pt}{2.5ex}\textbf{Human repair}\rule[-1ex]{0pt}{2ex} \\\hline
      \begin{lstlisting}[escapechar=!, basicstyle=\linespread{1.3}\ttfamily\footnotesize]

  def average(values):
    if values == (1,2,3):
      return (1+2+3)/3
    !\hlorange{else}! !\hlred{if}! values == (-3,2):
      return (-3+2+8-1)/4

      \end{lstlisting} & \begin{lstlisting}[escapechar=!, basicstyle=\linespread{1.3}\ttfamily\footnotesize]

  def average(values):
    if values == (1,2,3):
      return (1+2+3)/3
    !\hlorange{elif}! values == (-3,2):
      return (-3+2+8-1)/4

      \end{lstlisting} \\\hline
    \end{tabular}
  \end{center}
\end{frame}

\begin{frame}[fragile]{Can you spot the error?}
  \begin{center}
    \begin{tabular}{|m{5.5cm}|m{5.5cm}|}
      \hline \rule{0pt}{2.5ex}\textbf{Original code}\rule[-1ex]{0pt}{2ex} &  \rule{0pt}{2.5ex}\textbf{Human repair}\rule[-1ex]{0pt}{2ex} \\\hline
      \begin{lstlisting}[escapechar=!, basicstyle=\linespread{1.3}\ttfamily\footnotesize]

  import Global from Global
  globalObj = Global()
  print(str(globalObj.Test()))

      \end{lstlisting} & \begin{lstlisting}[escapechar=!, basicstyle=\linespread{1.3}\ttfamily\footnotesize]

      \end{lstlisting} \\\hline
    \end{tabular}
  \end{center}
\end{frame}

\begin{frame}[fragile]{Can you spot the error?}
  \begin{center}
    \begin{tabular}{|m{5.5cm}|m{5.5cm}|}
      \hline \rule{0pt}{2.5ex}\textbf{Original code}\rule[-1ex]{0pt}{2ex} &  \rule{0pt}{2.5ex}\textbf{Human repair}\rule[-1ex]{0pt}{2ex} \\\hline
      \begin{lstlisting}[escapechar=!, basicstyle=\linespread{1.3}\ttfamily\footnotesize]

  !\hlorange{import}! Global !\hlorange{from}! Global
  globalObj = Global()
  print(str(globalObj.Test()))

      \end{lstlisting} & \begin{lstlisting}[escapechar=!, basicstyle=\linespread{1.3}\ttfamily\footnotesize]

  !\hlorange{from}! Global !\hlorange{import}! Global
  globalObj = Global()
  print(str(globalObj.Test()))

      \end{lstlisting} \\\hline
    \end{tabular}
  \end{center}
\end{frame}

%\begin{frame}[fragile]{Can you spot the error?}
%  \begin{center}
%    \begin{tabular}{|m{5.5cm}|m{5.5cm}|}
%      \hline \rule{0pt}{2.5ex}\textbf{Original code}\rule[-1ex]{0pt}{2ex} &  \rule{0pt}{2.5ex}\textbf{Human repair}\rule[-1ex]{0pt}{2ex} \\\hline
%      \begin{lstlisting}[escapechar=!, basicstyle=\linespread{1.3}\ttfamily\footnotesize]
%
%   try:
%     something()
%   catch AttributeError:
%     pass
%
%      \end{lstlisting} & \begin{lstlisting}[escapechar=!, basicstyle=\linespread{1.3}\ttfamily\footnotesize]
%
%      \end{lstlisting} \\\hline
%    \end{tabular}
%  \end{center}
%\end{frame}
%
%\begin{frame}[fragile]{Can you spot the error?}
%  \begin{center}
%    \begin{tabular}{|m{5.5cm}|m{5.5cm}|}
%      \hline \rule{0pt}{2.5ex}\textbf{Original code}\rule[-1ex]{0pt}{2ex} &  \rule{0pt}{2.5ex}\textbf{Human repair}\rule[-1ex]{0pt}{2ex} \\\hline
%      \begin{lstlisting}[escapechar=!, basicstyle=\linespread{1.3}\ttfamily\footnotesize]
%
%   try:
%     something()
%   !\hlorange{catch}! AttributeError:
%     pass
%
%      \end{lstlisting} & \begin{lstlisting}[escapechar=!, basicstyle=\linespread{1.3}\ttfamily\footnotesize]
%
%   try:
%     something()
%   !\hlorange{except}! AttributeError:
%     pass
%
%      \end{lstlisting} \\\hline
%    \end{tabular}
%  \end{center}
%\end{frame}

\begin{frame}[t,fragile]{How many repairs could there possibly be?}
  Consider the following Python snippet, which contains a small syntax error:\\

  \begin{lstlisting}[escapechar=!, basicstyle=\linespread{1.3}\ttfamily\footnotesize]
    def prepend(i, k, L=[])
      n and [prepend(i - 1, k, [b] + L) for b in range(k)]
  \end{lstlisting}
\end{frame}

\begin{frame}[t,fragile]{How many repairs could there possibly be?}
  Consider the following Python snippet, which contains a small syntax error:\\

  \begin{lstlisting}[escapechar=!, basicstyle=\linespread{1.3}\ttfamily\footnotesize]
    def prepend(i, k, L=[])
      n and [prepend(i - 1, k, [b] + L) for b in range(k)]
  \end{lstlisting}

  It can be fixed by appending a colon after the function signature, yielding:\\

    \begin{lstlisting}[escapechar=!, basicstyle=\linespread{1.3}\ttfamily\footnotesize]
    def prepend(i, k, L=[])!\hlgreen{:}!
      n and [prepend(i - 1, k, [b] + L) for b in range(k)]
    \end{lstlisting}
\end{frame}

\begin{frame}[t,fragile]{How many repairs could there possibly be?}
  Consider the following Python snippet, which contains a small syntax error:\\

  \begin{lstlisting}[escapechar=!, basicstyle=\linespread{1.3}\ttfamily\footnotesize]
    def prepend(i, k, L=[])
      n and [prepend(i - 1, k, [b] + L) for b in range(k)]
  \end{lstlisting}

  It can be fixed by appending a colon after the function signature, yielding:\\

  \begin{lstlisting}[escapechar=!, basicstyle=\linespread{1.3}\ttfamily\footnotesize]
    def prepend(i, k, L=[])!\hlgreen{:}!
      n and [prepend(i - 1, k, [b] + L) for b in range(k)]
  \end{lstlisting}

  \vspace{0.5cm}

  \normalsize Let us consider a slightly more ambiguous error: \footnotesize{\texttt{v = df.iloc(5:, 2:)}}. \normalsize Assuming an alphabet of just a hundred lexical tokens, this statement has millions of two-token edits, yet only six are accepted by the Python parser:
\end{frame}

\begin{frame}[t,fragile]{How many repairs could there possibly be?}
  Consider the following Python snippet, which contains a small syntax error:\\

  \begin{lstlisting}[escapechar=!, basicstyle=\linespread{1.3}\ttfamily\footnotesize]
    def prepend(i, k, L=[])
      n and [prepend(i - 1, k, [b] + L) for b in range(k)]
  \end{lstlisting}

  It can be fixed by appending a colon after the function signature, yielding:\\

  \begin{lstlisting}[escapechar=!, basicstyle=\linespread{1.3}\ttfamily\footnotesize]
    def prepend(i, k, L=[])!\hlgreen{:}!
      n and [prepend(i - 1, k, [b] + L) for b in range(k)]
  \end{lstlisting}

  \vspace{0.5cm}

  \normalsize Let us consider a slightly more ambiguous error: \footnotesize{\texttt{v = df.iloc(5:, 2:)}}. \normalsize Assuming an alphabet of just a hundred lexical tokens, this statement has millions of two-token edits, yet only six are accepted by the Python parser:

    \scriptsize
    \begin{figure}[h!]
      \noindent\begin{tabular}{@{}l@{\hspace{10pt}}l@{\hspace{10pt}}l@{}}
      (1) \texttt{v = df.iloc(5\hlred{:}, 2\hlorange{,})} & (3) \texttt{v = df.iloc(5\hlgreen{[}:, 2:\hlgreen{]})} & (5) \texttt{v = df.iloc\hlorange{[}5:, 2:\hlorange{]}} \\\\
      (2) \texttt{v = df.iloc(5\hlorange{)}, 2\hlorange{(})} & (4) \texttt{v = df.iloc(5\hlred{:}, 2\hlred{:})} & (6) \texttt{v = df.iloc(5\hlgreen{[}:, 2\hlorange{]})} \\
      \end{tabular}
    \end{figure}
\end{frame}

\begin{frame}[fragile]{On the virtues of pragmatism}
  \textbf{Pragmatism}: \textit{a reasonable and logical way of solving problems that is based on dealing with specific situations instead of abstract theories.}\newline\\
  \begin{itemize}
    \item Often framed as a compromise, \textit{``Let's be pragmatic...''}
    \item Pragmatism is a principled approach to problem solving.
    \item Taken seriously, pragmatism is difficult because it requires modeling the needs of multiple stakeholders and balancing competing interests.
    \item Putting it into practice requires knowing your customer, understanding their workflow, considering the most appropriate solution out of a set of possible alternatives.
  \end{itemize}
    \setlength{\epigraphwidth}{0.97\textwidth}
    \epigraph{``\textit{What is the use of studying philosophy if all that it does for you is to enable you to talk about some abstruse questions of logic and does not improve your thinking about the important questions of everyday life?}''}{Ludwig Wittgenstein, 1889–1951}
\end{frame}

\begin{frame}[fragile]{On the virtues of pragmatism}
  \begin{columns}[T]
    \begin{column}{0.9\textwidth}
      \begin{itemize}
        \item Pioneered in the 19th century by Peirce, James, Dewey, et al.
        \item Wittgenstein was a pragmatist, early work on language games.
        \item Pragmatism is a philosophy of language that emphasizes the role of intent in human communication.
        \item Language is a tool for communication, \\not just a arbitrary set of rules.
        \item Must actively imagine the mindset of\\ the speaker, not just the literal\\meaning of their words.
        \item Language is a bit like a game whose goal \\is to understand the speaker's intent.
        \item Assume a proficient speaker, who is trying \\to communicate something meaningful.
      \end{itemize}
    \end{column}
    \begin{column}{0.6\textwidth}
      \vspace*{2cm}
      \hspace*{-3cm}
      \includegraphics[width=1.7cm]{../figures/photo_pierce.png}
      \includegraphics[width=2.6cm]{../figures/photo_wittgenstein.png}
    \end{column}
  \end{columns}
\end{frame}

%\begin{frame}[fragile]{Common sources of syntax errors}
%  \begin{itemize}
%    \item Reading impairments (e.g., dyslexia, dysgraphia)
%    \item Motor impairments (e.g., tremors, Parkinson's)
%    \item Speech impediments (e.g., stuttering, apraxia, Tourette's)
%    \item Visual impairments (e.g., poor eyesight, blindness)
%    \item Language barriers (e.g., foreign and non-native speakers)
%    \item Inexperience (e.g., novice programmers)
%    \item Distraction (e.g., multitasking, fatigue, stress)
%    \item Time pressure (e.g., deadlines, interview coding)
%    \item Inattention (e.g., typographic mistakes, boredom, apathy)
%    \item Lack of feedback (e.g., no syntax highlighting or IDE)
%  \end{itemize}
%\end{frame}

\begin{frame}{From Error-Correcting Codes to Correcting Coding Errors}
  \begin{itemize}
    \item Error-correcting codes are a well-studied topic in information theory used to detect and correct errors in data transmission.
    \item Introduces parity bits to detect and correct transmission errors assuming a certain noise model (e.g., Hamming distance).
    \item Like ECCs, we also assume a certain noise model (Levenshtein distance) and error tolerance (n-lexical tokens).
    \item Instead of injecting parity bits, we use the grammar and mutual information between tokens to detect and correct errors.
    \item Unlike ECCs, we do not assume a unique solution, but a set of admissible solutions ranked by statistical likliehood.
    \end{itemize}
  \setlength{\epigraphwidth}{0.97\textwidth}
  \epigraph{``\textit{Damn it, if the machine can detect an error, why can't it\\\phantom{``}locate the position of the error and correct it?'}''}{Richard Hamming, 1915-1998}
\end{frame}

\begin{frame}[fragile]{Syntax repair as a language game}
  \begin{itemize}
    \item Imagine a game between two players, \textit{Editor} and \textit{Author}.
    \item They both see the same grammar, $\mathcal{G}$ and invalid string $\err\sigma \notin \mathcal{L}(\mathcal{G})$.
    % Both players move simultaneously after a short period of deliberation. As soon as Player A repairs $\err\sigma$, Player B must move immediately.
    \item Author moves by modifying $\err\sigma$ to produce a valid string, $\sigma \in \mathcal{L}(\mathcal{G})$.
    \item Editor moves continuously, sampling a set $\tilde{\bm\sigma} \subseteq \mathcal{L}(\mathcal{G})$.
    \item As soon as Author repairs $\err\sigma$, the turn immediately ends.
    \item Neither player sees the other's move(s) before making their own.
    \item If Editor anticipates Author's move, i.e., $\sigma \in \tilde{\bm\sigma}$, they both win.
    \item If Author surprises Editor with a valid move, i.e., $\sigma \notin \tilde{\bm\sigma}$, Author wins.
    \item We may consider a refinement where Editor wins in proportion to the time taken to anticipate Author's move.
  \end{itemize}
\end{frame}

\begin{frame}[fragile]{Problem Statement: Validity and naturalness}
  Syntax repair can be treated as a language intersection problem between a context-free language (CFL) and a regular language.

  \begin{definition}[Reachable edits]
    Given a CFL, $\ell$, and an invalid string, $\err{\sigma}: \ell^\complement$, find every valid string reachable within $d$ edits of $\err{\sigma}$, i.e., letting $\Delta$ be the Levenshtein metric and $L(\err\sigma, d) = \{\sigma' \mid \Delta(\err{\sigma}, \sigma') \leq d\}$ be the edit ball, we seek $A = L(\err\sigma, d) \cap \ell$.
  \end{definition}

  \begin{definition}[Ranked repair]
  Given a finite language $A = L(\err\sigma, d) \cap \ell$ and a probabilistic language model $\text{P}_\theta: \Sigma^* \rightarrow [0, 1] \subset \mathbb{R}$, the ranked repair problem is to find the top-$k$ maximum likelihood repairs under the language model. That is,
  \begin{equation}
    R(A, P_\theta) = \argmax_{\bm{\sigma} \subseteq A, |\bm{\sigma}| \leq k} \sum_{\sigma \in \bm{\sigma}}\text{P}_\theta(\sigma)
  \end{equation}
    \end{definition}
%  To solve this problem, we will first pose a simpler problem that only considers localized edits, then turn our attention back to BCFLR.
%
%  \begin{definition}[Porous completion]
%    Let $\underline\Sigma \coloneqq \Sigma \cup \{\hole\}$, where \hole denotes a hole. We denote $\sqsubseteq: \Sigma^n \times \underline\Sigma^n$ as the relation $\{\langle\sigma', \sigma\rangle \mid \sigma_i \in \Sigma \implies \sigma_i' = \sigma_i\}$ and the set of all inhabitants $\{\sigma' \mid \sigma' \sqsubseteq \sigma\}$ as $\text{H}(\sigma)$. Given a \textit{porous string}, $\sigma: \underline\Sigma^*$ we seek all syntactically admissible inhabitants, i.e., $A(\sigma)\coloneqq\text{H}(\sigma)\cap\ell$.
%  \end{definition}
\end{frame}

\begin{frame}[fragile]{Problem Statement: Temporal constraints}
  Find every syntactically admissible edit $\{\sigma' \in \ell \mid \Delta(\err{\sigma}, \sigma') \leq d\}$, ranked by a probability metric $P_\theta$, and return them in a reasonable amount of time.

  \begin{definition}[Linear convergence]\label{def:linear-convergence}
  Given a finite CFL, $\ell$, we want a generating function, $\bm{\varphi}: \mathbb{N}_{<|\ell|} \rightarrow 2^\ell$, that converges linearly in expectation, i.e., $\mathbb{E}_{i \in [1, n]}|\bm{\varphi}(i)| \propto n$.
  \end{definition}

  \begin{center}
    \resizebox{0.42\textwidth}{!}{
      \def\secondcirclepath{(1.15,0) coordinate (e) circle (2cm)}
      \begin{tikzpicture}[
        dot/.style = {circle, inner sep=0pt, minimum size=1mm, fill,
        node contents={}}
      ]
        \def\firstcircle{(-2.1,0) coordinate (a) circle (2.4cm)}
        \def\firstcirclea{(-2.1,0) coordinate (b) circle (0.6cm)}
        \def\firstcircleb{(-2.1,0) coordinate (c) circle (1.2cm)}
        \def\firstcirclec{(-2.1,0) coordinate (d) circle (1.8cm)}
        \def\secondcircle{(1.2,0) coordinate (e) circle (1.5cm)}

        \begin{scope}
          \clip[decorate, decoration={snake, amplitude=0.6mm, segment length=5.01mm}] \secondcirclepath;
          \fill[black!15] \firstcircle;
        \end{scope}

        \draw \firstcircle node[dot,label=$\err{\sigma}$](z0);
        \draw [dashed] \firstcirclea;
        \draw [dashed] \firstcircleb;
        \draw [dashed] \firstcirclec;
        \draw[-stealth] (-2.1,0) -- (-1.5, 0) node[midway,below]{$d_1$};
        \draw[-stealth] (-1.5,0) -- (-0.9, 0) node[midway,below]{$d_2$};
        \draw[-stealth] (-0.9,0) -- (-0.3, 0) node[midway,below]{$d^*$};
        \draw[-stealth] (-0.3,0) -- (0.3, 0) node[midway,above]{$\tilde{\sigma}$};
        \draw[-stealth] (-0.3,0) -- (0.3, 0) node[midway,below]{$d^+$};

        \draw[decorate, decoration={snake, amplitude=0.6mm, segment length=5.01mm}] \secondcirclepath;
        \node [above] at (current bounding box.north -| a) {$\mathcal{L}\bigl(L(\err\sigma, d^*)\bigr)$};
        \node [above,yshift=2.1cm] at (e) {$\mathcal{L}(\mathcal{G}')$};
      \end{tikzpicture}
    }
    \scalebox{7}{\large$^+$\raisebox{0.2ex}{\huge\fontspec{Arial Unicode MS}^^^^231b}}
  \end{center}

  \textbf{Natural language:}
  \textit{Retrieve as many syntactically valid repairs as possible within a small neighborhood and time frame, ranked by naturalness.}
\end{frame}

%\begin{frame}[fragile]{Ranked repair under realtime constraints}
%  $A(\sigma)$ is often a very large-cardinality set, so we want a procedure which prioritizes likely repairs first, without exhaustive enumeration. Specifically,
%  \begin{definition}[Ranked repair]
%    Given a finite language $\ell_\cap = L(\err\sigma, d) \cap \ell$ and a probabilistic language model $P_\theta: \Sigma^* \rightarrow [0, 1] \subset \mathbb{R}$, the ranked repair problem is to find the top-$k$ repairs by likelihood under the language model. That is,
%    \begin{equation}
%      R(\ell_\cap, P_\theta) \coloneqq \argmax_{\{\bm{\sigma} \mid \bm{\sigma} \subseteq \ell_\cap, |\bm{\sigma}| \leq k\}} \sum_{\sigma \in \bm{\sigma}}\text{P}(\sigma\mid\err\sigma, \theta)
%    \end{equation}
%    % On average, across all $G, \sigma$ $\hat{R}$ should approximate $R$.
%    We want a procedure $\hat{R}$, minimizing $\mathbb{E}_{G, \sigma}\big[D_{\text{KL}}(\hat{R} \parallel R)\big]$ and total latency.
%  \end{definition}
%
%  Since $R$ is intractable in general, we want a procedure that approximates it for a representative sampling of natural context-free grammars and strings, i.e., real-world programming languages and source code snippets.
%\end{frame}

%\begin{frame}[fragile]{From CFL Reachability to Real World Program Repair}
%  To fix real code, we needed to overcome a variety of interesting challenges:\vspace{10pt}
%
%  \begin{itemize}
%    \item \textbf{Syntax mismatch}: The syntax of real-world programming languages does not exactly correspond to the theory of formal languages.
%    \item \textbf{Source code $\approx$ PL}: Most of the time, source code in the wild is incomplete or only loosely approximates a programming language.
%    \item \textbf{Responsiveness}: The usefulness of synthetic repairs is inversely proportional to the amount of time required to generate them.
%    \item \textbf{Edit generation}: How do we generate edits that are (1) syntactically admissible (2) statistically plausible and (3) semantically meaningful?
%    \item \textbf{Evaluation}: Big code and version control is too coarse-grained, contains irrelevant edits, not representative of small errors/fixes.
%  \end{itemize}
%\end{frame}


\begin{frame}[fragile]{High-level architecture overview}
  \vspace{0.4cm}
  \begin{figure}[h!]
    \vspace{-1cm}
      \begin{center}
      \resizebox{0.7\textwidth}{!}{
      \begin{tikzpicture}[node distance=5cm]
      \node (start) [io] {Broken code};
      \node (node1) [plain, right of=start] {\phantom{...}\textbf{Language intersection}\phantom{...}};
      \node (gram1) [io2, above of=node1, yshift=-3.2cm] {Grammar};
      %        \node (node2) [plain, right of=node1] {\textbf{Repair extraction}};
      %        \node (ptree) [io, above of=node2, yshift=-3cm] {$\mathbb{T}_2$};
      \node (node3) [plain, right of=node1, xshift=0.5cm] {\textbf{Repair decoding}};
      \node (ngram) [io2, above of=node3, yshift=-3.2cm] {Markov chain};
      \node (node4) [io, right of=node3] {Repairs};
      \draw [arrow] (start) -- (node1);
      \draw [arrow] (gram1) -- (node1);
      \draw [arrow] (node1) -- (node3);
      \draw [arrow] (node3) -- (node4);
      \draw [arrow] (ngram) -- (node3);
      %        \draw [arrow] (ptree) -- (node2);
      \end{tikzpicture}
      }
      \end{center}
  \end{figure}

  \vspace{0.4cm}

\begin{wrapfigure}{r}{0.4\textwidth}
%\begin{figure}[h!]
\vspace{-0.7cm}
\resizebox{0.4\textwidth}{!}{
\begin{tikzpicture}[node distance=2cm]
\node (start) [startstop, draw=none];
\node (pro1) [process, below of=start, yshift=-0.3cm] {$G_\cap \gets G\cap L(\err\sigma, d)$};
\node (pcfg) [io2, left of=pro1, xshift=-3cm] {CFG};
\node (lnfa) [io, right of=pro1, xshift=3cm] {L-NFA};

\node (code) [io, right of=start,xshift=3cm] {Code};
\node (synt) [io2, left of=start,xshift=-3cm] {Syntax};

\node (dec1) [decision, below of=pro1, yshift=-0.5cm] {$[G_{\cap} = \varnothing]$};

\node (t2) [process, left of=dec1, xshift=-3cm] {Construct $\mathbb{T}_2$ from $G_\cap'$};

\node (pro2b) [process, right of=dec1, xshift=3cm] {Increase radius, $d$};

\node [below=0.7cm of pro2b, xshift=-0.3cm] {\Large\textbf{Language intersection}};
\draw[thick,dotted, rounded corners] ($(pcfg.north west)+(-1.9,0.8)$) rectangle ($(pro2b.south east)+(0.3,-1.5)$);

\node (const) [process, below of=dec1, yshift=-1.8cm, xshift=-1.5cm] {Enumerate trees from $\mathbb{T}_2$};

%          \node (dec2) [decision, below of=const, yshift=-0.5cm] {$|\mathcal{L}(G_\cap)|$};
%
%          \node (samp1) [process, left of=dec2, xshift=-3cm] {Enumerate $\sigma' \in \mathcal{L}(G_\cap)$};
%          \node [above=0.07cm of samp1] {(\S~\ref{sec:ptree})};
%          \node (samp2) [process, right of=dec2, xshift=3cm] {Sample $\sigma' \sim P(G_\cap)$};
%          \node [above=0.07cm of samp2] {(\S~\ref{sec:ptree})};

%          \draw[thick,dotted, rounded corners] ($(const.north west)+(-5.3,0.7)$) rectangle ($(samp2.south east)+(0.3,-0.6)$);

\node (rrpc) [process, below of=const, yshift=-0.5cm] {Rerank by PCFG score};
\node [above=0.07cm of rrpc, xshift=1.7cm] {(Algorithm 1)};
\node (rank) [process, below of=rrpc, yshift=-0.5cm] {Convert to DFA and walk};
\node [above=0.1cm of rank, xshift=1.7cm] {(Algorithm 3)};
%          \node (vlmc) [io2, right of=rank, xshift=3cm] {Markov chain};
\node [below=0.3cm of rank, xshift=-3.8cm] {\Large\textbf{Repair decoding}};
\draw [thick,dotted, rounded corners] ($(rank.north west)+(-3.8,5.8)$) rectangle ($(rank.south east)+(6.8,-1.1)$);

\node (rrng) [process, right of=const, xshift=4.5cm] {Rerank by n-gram score};
\node [above=0.1cm of rrng, xshift=0.7cm] {(Algorithm 2)};
\node (results) [io, right of=rank, xshift=4.5cm] {Repairs};
%  \node (out1) [io, below of=pro2a] {Output};
\node (stop) [startstop, right of=rank, xshift=3cm];
\node (stop1) [startstop, right of=rrpc, xshift=3cm];
\node (stop2) [startstop, right of=const, xshift=3cm];

%  \draw [arrow] (dec0) -- node[anchor=east] {no} (pro1);

%          \draw [->,thick] (-5, 1.3) -- (synt);
%          \draw [->,thick] (5, 1.3) -- (code);

%          \draw [arrow] (start) -- (code);
%          \draw [arrow] (start) -- (synt);
\draw [arrow] (code) -- (lnfa);
\draw [arrow] (const) -- (rrng);
\draw [arrow] (rrng) -- (results);
\draw [arrow] (synt) -- (pcfg);
\draw [arrow] (lnfa) -- (pro1);
\draw [arrow] (pcfg) -- (pro1);

%          \draw [arrow] (grwa) -- (results);
%          \draw [line width=0.8pt] (stop.west) -- (stop1.west);
%          \draw [line width=0.8pt] (stop2.west) -- (stop1.west);

%  \draw [arrow] (in1) -- (pro1);
\draw [arrow] (pro1) -- (dec1);
\draw [arrow] (dec1) -- node[anchor=south] {yes} (pro2b);
\draw [arrow] (dec1) -- node[anchor=south] {no} (t2);
\draw [arrow] (const) -- (rrpc);
\draw [arrow] (pro2b) -- (lnfa);
%          \draw [arrow] (dec2) -- node[anchor=south] {small} (samp1);
%          \draw [arrow] (dec2) -- node[anchor=south] {large} (samp2);

\draw [arrow] (t2) |- ([shift={(-1.3cm,0)}]const.west)--(const.west);
%          \draw [arrow] (t2) |- ([shift={(-1.3cm,0)}]grwa.west)--(grwa.west);
\draw [arrow] (t2) |- ([shift={(-1.3cm,0)}]rank.west)--(rank.west);
%          \draw [arrow] (vlmc) -- (rank);
\draw [arrow] (rrpc) |- ([shift={(4.37cm,0)}]rrpc.east)--(results.north);
%          \draw [arrow] (samp2) |- ([shift={(0,1.3cm)}]rank.north)--(rank.north);
%  \draw [arrow] (pro2a) -- (out1);
\draw [arrow] (rank) -- (results);
%          \draw [line width=0.8pt] (grwa) -- (stop1);
%          \draw [line width=0.8pt] (const) -- (stop2);
%          \draw [arrow] (dec2) -- node[anchor=east] {1} (stop);

\end{tikzpicture}
}
\end{wrapfigure}

  Our syntax repair procedure can be described in three high-level steps. First, we generate a synthetic grammar $(G_\cap)$ representing the intersection between the syntax $(G)$ and Levenshtein ball around the source code $\big(\Delta(\err\sigma, d)\big)$. During repair extraction, we retrieve as many repairs as possible from the intersection grammar via sampling or enumeration. Finally, we rank all repairs discovered by likelihood.

\end{frame}


\begin{frame}[fragile]{High-level architecture overview}
This process can be depicted as series of staged transformations lowering the CFL intersection problem onto a finite automaton. Below, we consider a simplified version based on the language of balanced parentheses.

\vspace{0.3cm}
\begin{figure}[H]
\centering
\includegraphics[width=\textwidth]{flow_short.pdf}
\vspace{0.1cm}
\caption{Simplified dataflow of the language intersection pipeline. Given a grammar and broken code fragment, we (1) create a automaton generating the language of small edits, then (2) construct a grammar representing the intersection of the two languages. This grammar can be (3) converted into a finite automaton, (4) determinized, then (5) decoded to produce a list of repairs.}
\label{fig:exampleDFA}
\end{figure}
\end{frame}

\section{Formal Language Theory}\label{sec:fltheory}

%------------------------------------------------------------------------------------------------

\begin{frame}[fragile]{Background: Regular grammars}
  A regular grammar (RG) is a quadruple $\mathcal{G} = \langle V, \Sigma, P, S\rangle$ where $V$ are nonterminals, $\Sigma$ are terminals, $P: V\times (V \cup \Sigma)^{\leq 2}$ are the productions, and $S\in V$ is the start symbol, i.e., all productions are of the form $A \rightarrow a$, $A \rightarrow a B$ (right-regular), or $A \rightarrow B a$ (left-regular). E.g., the following RG and NFA correspond to the language defined by the \textit{regex} \tinline{(a(ab)*)*(ba)*}:

  % https://www3.nd.edu/~kogge/courses/cse30151-fa17/Public/other/tikz_tutorial.pdf
  % Glushkov's algorithm: https://www.irif.fr/~jep/PDF/MPRI/MPRI.pdf#subsection.3.5.2
  \begin{figure}
    \hspace{-1cm}
    \begin{minipage}[t]{0.25\linewidth}
      \vspace{-2.4cm}\scalebox{0.6}{
        \begin{aligned}[t]
          S &\rightarrow Q_0 \mid Q_2 \mid Q_3 \mid Q_5\\
          Q_0 &\rightarrow \varepsilon \\
          Q_1 &\rightarrow Q_0 b \mid Q_2 b\\
          Q_2 &\rightarrow Q_1 a \\
          Q_3 &\rightarrow Q_0 a \mid Q_3 a \mid Q_5 a \\
          Q_4 &\rightarrow Q_3 a \mid Q_5 a \\
          Q_5 &\rightarrow Q_4 b \\
        \end{aligned}}
    \end{minipage}
    \hspace{0.5cm}
    \begin{minipage}[t]{0.48\linewidth}
      \scalebox{0.5}{
        \begin{tikzpicture}
          [->, >=stealth,]
          \node[state, initial above, accepting] (Q0) {$Q_0$};
          \node[state, left of=Q0] (Q1) {$Q_1$};
          \node[state, accepting, left of=Q1] (Q2) {$Q_2$};
          \node[state, accepting, right of=Q0] (Q3) {$Q_3$};
          \node[state, above right of=Q3] (Q4) {$Q_4$};
          \node[state, accepting, below right of=Q3] (Q5) {$Q_5$};
          \draw
          %        (Q0) edge[loop above] (Q0)
          (Q0) edge node{\ttinline b} (Q1)
          (Q0) edge node{\ttinline a} (Q3)
          (Q1) edge[bend right] node{\ttinline a} (Q2)
          (Q2) edge[bend right] node{\ttinline b} (Q1)
          (Q3) edge[loop above] node{\ttinline a} (Q3)
          (Q3) edge node{\ttinline a} (Q4)
          (Q5) edge node{\ttinline a} (Q3)
          (Q4) edge[bend left] node{\ttinline b} (Q5)
          (Q5) edge[bend left] node{\ttinline a} (Q4)
        \end{tikzpicture}
      }
    \end{minipage}
  \end{figure}

  \begin{center}
    \scalebox{0.8}{
      \begin{tikzpicture}[font=\sffamily,breathe dist/.initial=4ex]
        \foreach \X [count=\Y,remember=\Y as \LastY] in
          {finite,regular}
          {\ifnum\Y=1
        \node[ellipse,draw,outer sep=0pt] (F-\Y) {\X};
        \else
        \path[decoration={text along path,
        text={|\sffamily|\X},text align=center,raise=0.9ex},decorate]
        let \p1=($(F-\LastY.north)-(F-\LastY.west)$)
        in (F-\LastY.west) arc(180:0:\x1 and \y1);
        \path let \p1=($([yshift=\pgfkeysvalueof{/tikz/breathe dist}]F-\LastY.north)
        -(F-\LastY.south)$),
        \p2=($(F-1.east)-(F-1.west)$),\p3=($(F-1.north)-(F-1.south)$)
        in ($([yshift=\pgfkeysvalueof{/tikz/breathe dist}]F-\LastY.north)!0.5!(F-\LastY.south)$)
        node[minimum height=\y1,minimum width={\y1*\x2/\y3},
        draw,ellipse,inner sep=0pt, fill=black!30!white] (F-\Y){};
        \fi
        }
        \foreach \X [count=\Y,remember=\Y as \LastY] in
          {finite,regular,context-free}
          {\ifnum\Y=1
        \node[ellipse,draw,outer sep=0pt] (F-\Y) {\X};
        \else
        \path[decoration={text along path,
        text={|\sffamily|\X},text align=center,raise=0.9ex},decorate]
        let \p1=($(F-\LastY.north)-(F-\LastY.west)$)
        in (F-\LastY.west) arc(180:0:\x1 and \y1);
        \path let \p1=($([yshift=\pgfkeysvalueof{/tikz/breathe dist}]F-\LastY.north)
        -(F-\LastY.south)$),
        \p2=($(F-1.east)-(F-1.west)$),\p3=($(F-1.north)-(F-1.south)$)
        in ($([yshift=\pgfkeysvalueof{/tikz/breathe dist}]F-\LastY.north)!0.5!(F-\LastY.south)$)
        node[minimum height=\y1,minimum width={\y1*\x2/\y3},
        draw,ellipse,inner sep=0pt] (F-\Y){};
        \fi}
      \end{tikzpicture}
    }
  \end{center}
\end{frame}

    \begin{frame}[fragile]{Levenshtein automaton customization}

Consider the string $\err\sigma=$ \texttt{\scriptsize ( ) )} and the alphabet $\Sigma = \{\texttt{\scriptsize)}, \texttt{\scriptsize(}\}$. Every string within one edit of $\err\sigma$ is recognizing by an NFA with the following structure:

\begin{figure}[h!]
\resizebox{0.45\textwidth}{!}{
\begin{tikzpicture}[
%->, % makes the edges directed
>=stealth',
node distance=2.5cm, % specifies the minimum distance between two nodes. Change if necessary.
%  every state/.style={thick, fill=gray!10}, % sets the properties for each ’state’ node
initial text=$ $, % sets the text that appears on the start arrow
]
\node[state, initial]                (00) {$q_{0,0}$};
\node[state, right of=00]            (10) {$q_{1,0}$};
\node[accepting, state, right of=10] (20) {$q_{2,0}$};
\node[accepting, state, right of=20] (30) {$q_{3,0}$};

\node[state, above of=00, shift={(-2cm,0cm)}] (01) {$q_{0,1}$};
\node[state, right of=01]                     (11) {$q_{1,1}$};
\node[state, right of=11]                     (21) {$q_{2,1}$};
\node[accepting, state, right of=21]          (31) {$q_{3,1}$};

\draw [->] (00) edge[below] node{$\texttt{(}$} (10);
\draw [->] (10) edge[below] node{$\texttt{)}$} (20);
\draw [->] (20) edge[below] node{$\texttt{)}$} (30);

\draw [->] (01) edge[below] node{$\texttt{(}$}                       (11);
\draw [->] (11) edge[below] node[shift={(-0.2cm,0cm)}]{$\texttt{)}$} (21);
\draw [->] (21) edge[below] node[shift={(-0.2cm,0cm)}]{$\texttt{)}$} (31);

\draw [->] (00) edge[bend left=10] node[shift={(-0.15cm,0cm)}]{\tiny{$\texttt{(}$}} (11);
\draw [->] (10) edge[bend left=10] node[shift={(-0.15cm,0cm)}]{\tiny{$\texttt{(}$}} (21);
\draw [->] (20) edge[bend left=10] node[shift={(-0.15cm,0cm)}]{\tiny{$\texttt{(}$}} (31);

\draw [->] (00) edge[bend left=10, left] node[shift={(-0.1cm,0cm)}]{\tiny{$\texttt{(}$}} (01);
\draw [->] (10) edge[bend left=10, left] node[shift={(-0.1cm,0cm)}]{\tiny{$\texttt{(}$}} (11);
\draw [->] (20) edge[bend left=10, left] node[shift={(-0.1cm,0cm)}]{\tiny{$\texttt{(}$}} (21);

\draw [->] (00) edge[bend right=10, right] node{\tiny{$\texttt{)}$}} (11);
\draw [->] (10) edge[bend right=10, right] node{\tiny{$\texttt{)}$}} (21);
\draw [->] (20) edge[bend right=10, right] node{\tiny{$\texttt{)}$}} (31);

\draw [->] (00) edge[bend right=10, right] node{\tiny{$\texttt{)}$}} (01);
\draw [->] (10) edge[bend right=10, right] node{\tiny{$\texttt{)}$}} (11);
\draw [->] (20) edge[bend right=10, right] node{\tiny{$\texttt{)}$}} (21);

\draw [->] (30) edge[bend left=10, left] node[shift={(-0.1cm,0cm)}]{\tiny{$\texttt{(}$}} (31);
\draw [->] (30) edge[bend right=10, right] node{\tiny{$\texttt{)}$}} (31);

\draw [->, blue] (00) edge[bend right=11,below] node[shift={(0.4cm,0.9cm)}]{$\texttt{)}$}    (21);
\draw [->, blue] (10) edge[bend right=11,below] node[shift={(0.4cm,0.9cm)}]{$\texttt{)}$}    (31);
\node[align=center, yshift=2em, xshift=-1cm] (title) at (current bounding box.north) {Original Levenshtein automaton};
\end{tikzpicture}
}
\resizebox{0.515\textwidth}{!}{
\begin{tikzpicture}[
%->, % makes the edges directed
>=stealth',
node distance=2.5cm, % specifies the minimum distance between two nodes. Change if necessary.
%  every state/.style={thick, fill=gray!10}, % sets the properties for each ’state’ node
initial text=$ $, % sets the text that appears on the start arrow
]
\draw[orange,->] (-4cm,1.2cm) -- (-3cm,1.2cm);

\node[state, initial]                (00) {$q_{0,0}$};
\node[state, right of=00]            (10) {$q_{1,0}$};
\node[accepting, state, right of=10] (20) {$q_{2,0}$};
\node[accepting, state, right of=20] (30) {$q_{3,0}$};

\node[state, above of=00, shift={(-2cm,0cm)}] (01) {$q_{0,1}$};
\node[state, right of=01]                     (11) {$q_{1,1}$};
\node[state, right of=11]                     (21) {$q_{2,1}$};
\node[accepting, state, right of=21]          (31) {$q_{3,1}$};

\draw [->] (00) edge[below] node{\tiny{$[= \texttt{(}]$}} (10);
\draw [->] (10) edge[below] node{\tiny{$[= \texttt{)}]$}} (20);
\draw [->] (20) edge[below] node{\tiny{$[= \texttt{)}]$}} (30);

\draw [->] (01) edge[below] node{\tiny{$[= \texttt{(}]$}}                       (11);
\draw [->] (11) edge[below] node[shift={(-0.2cm,0cm)}]{\tiny{$[= \texttt{)}]$}} (21);
\draw [->] (21) edge[below] node[shift={(-0.2cm,0cm)}]{\tiny{$[= \texttt{)}]$}} (31);

\draw [->] (00) edge[left] node{\tiny{$[\neq \texttt{(}]$}} (11);
\draw [->] (10) edge[left] node{\tiny{$[\neq \texttt{)}]$}} (21);
\draw [->] (20) edge[left] node{\tiny{$[\neq \texttt{)}]$}} (31);

\draw [->] (00) edge[bend left=10, left] node{\tiny{$[\neq \texttt{(}]$}} (01);
\draw [->] (10) edge[bend left=10, left] node{\tiny{$[\neq \texttt{)}]$}} (11);
\draw [->] (20) edge[bend left=10, left] node{\tiny{$[\neq \texttt{)}]$}} (21);
\draw [->] (30) edge[bend left=10, left] node{\tiny{$[=.]$}} (31);


\draw [->, blue] (00) edge[bend right=11,below] node[shift={(0.2cm,0.8cm)}]{\tiny{$[= \texttt{)}]$}}    (21);
\draw [->, blue] (10) edge[bend right=11,below] node[shift={(0.2cm,0.8cm)}]{\tiny{$[= \texttt{)}]$}}    (31);
\node[align=center, yshift=2em, xshift=-0.4cm] (title) at (current bounding box.north) {Nominal Levenshtein automaton (ours)};
\end{tikzpicture}
}
\caption{Automaton recognizing every single patch of the broken string \texttt{( ) )} within Levenshtein distance 1. We nominalize the original Levenshtein automaton, ensuring upward arcs denote a mutation, and replace terminals with a symbolic predicate, which deduplicates parallel arcs in large alphabets.}\label{fig:lev_automaton}\vspace{-5pt}
\end{figure}
\\
\tiny{\url{https://fulmicoton.com/posts/levenshtein/#observations-lets-count-states}}
\end{frame}

\begin{frame}[fragile]{Levenshtein reachability}
  \begin{figure}[H]
    \resizebox{\textwidth}{!}{
      \begin{tikzpicture}[
%->, % makes the edges directed
        >=stealth',
        node distance=2.5cm, % specifies the minimum distance between two nodes. Change if necessary.
%  every state/.style={thick, fill=gray!10}, % sets the properties for each ’state’ node
        initial text=$ $, % sets the text that appears on the start arrow
      ]
        \node[state, initial]                (00) {$q_{0,0}$};
        \node[state, right of=00]            (10) {$q_{1,0}$};
        \node[accepting, state, right of=10] (20) {$q_{2,0}$};
        \node[accepting, state, right of=20] (30) {$q_{3,0}$};
        \node[accepting, state, right of=30] (40) {$q_{4,0}$};
        \node[accepting, state, right of=40] (50) {$q_{5,0}$};

        \node[state, above of=00, shift={(-2cm,0cm)}] (01) {$q_{0,1}$};
        \node[state, right of=01]                          (11) {$q_{1,1}$};
        \node[state, right of=11]                          (21) {$q_{2,1}$};
        \node[accepting, state, right of=21]               (31) {$q_{3,1}$};
        \node[accepting, state, right of=31]               (41) {$q_{4,1}$};
        \node[accepting, state, right of=41]               (51) {$q_{5,1}$};

        \node[state, above of=01, shift={(-2cm,0cm)}] (0j) {$q_{0,2}$};
        \node[state, right of=0j]                          (1j) {$q_{1,2}$};
        \node[state, right of=1j]                          (2j) {$q_{2,2}$};
        \node[state, right of=2j]                          (3j) {$q_{3,2}$};
        \node[accepting, state, right of=3j]               (4j) {$q_{4,2}$};
        \node[accepting, state, right of=4j]               (5j) {$q_{5,2}$};

        \node[state, above of=0j, shift={(-2cm,0cm)}] (0k) {$q_{0,3}$};
        \node[state, right of=0k]                         (1k) {$q_{1,3}$};
        \node[state, right of=1k]                         (2k) {$q_{2,3}$};
        \node[state, right of=2k]                         (3k) {$q_{3,3}$};
        \node[state, right of=3k]                         (4k) {$q_{4,3}$};
        \node[accepting, state, right of=4k]              (5k) {$q_{5,3}$};

        \draw [->] (00) edge[below] node{$\sigma_1$} (10);
        \draw [->] (10) edge[below] node{$\sigma_2$} (20);
        \draw [->] (20) edge[below] node{$\sigma_3$} (30);
        \draw [->] (30) edge[below] node{$\sigma_4$} (40);
        \draw [->] (40) edge[below] node{$\sigma_5$} (50);

        \draw [->] (01) edge[below] node{$\sigma_1$} (11);
        \draw [->] (11) edge[below] node[shift={(-0.2cm,0cm)}]{$\sigma_2$} (21);
        \draw [->] (21) edge[below] node[shift={(-0.2cm,0cm)}]{$\sigma_3$} (31);
        \draw [->] (31) edge[below] node[shift={(-0.2cm,0cm)}]{$\sigma_4$} (41);
        \draw [->] (41) edge[below] node{$\sigma_5$} (51);

        \draw [->] (0j) edge[below] node{$\sigma_1$} (1j);
        \draw [->] (1j) edge[below] node{$\sigma_2$} (2j);
        \draw [->] (2j) edge[below] node{$\sigma_3$} (3j);
        \draw [->] (3j) edge[below] node{$\sigma_4$} (4j);
        \draw [->] (4j) edge[below] node{$\sigma_5$} (5j);

        \draw [->] (0k) edge[below] node{$\sigma_1$} (1k);
        \draw [->] (1k) edge[below] node{$\sigma_2$} (2k);
        \draw [->] (2k) edge[below] node{$\sigma_3$} (3k);
        \draw [->] (3k) edge[below] node{$\sigma_4$} (4k);
        \draw [->] (4k) edge[below] node{$\sigma_5$} (5k);

        \draw [->] (00) edge[left] node{$\phantom{\cdot}$} (11);
        \draw [->] (10) edge[left] node{$\phantom{\cdot}$} (21);
        \draw [->] (20) edge[left] node{$\phantom{\cdot}$} (31);
        \draw [->] (30) edge[left] node{$\phantom{\cdot}$} (41);
        \draw [->] (40) edge[left] node{$\phantom{\cdot}$} (51);

% Super-knight arcs
        \draw [->, red] (00) edge[bend right=8] node[east, shift={(-0.2cm,-0.7cm)}]{$\color{red}\sigma_3$}         (3j);
        \draw [->, red] (10) edge[bend right=8] node[east, shift={(-0.2cm,-0.7cm)}]{$\color{red}\sigma_4$}         (4j);
        \draw [->, red] (20) edge[bend right=8] node[east, shift={(-0.2cm,-0.7cm)}]{$\color{red}\sigma_5$}         (5j);

        \draw [->, red] (01) edge[bend left=8] node[east, shift={(-0.2cm,-0.7cm)}]{$\color{red}\sigma_3$}         (3k);
        \draw [->, red] (11) edge[bend left=8] node[east, shift={(-0.2cm,-0.7cm)}]{$\color{red}\sigma_4$}         (4k);
        \draw [->, red] (21) edge[bend left=8] node[east, shift={(-0.2cm,-0.7cm)}]{$\color{red}\sigma_5$}         (5k);

        \draw [->, violet] (00) edge node[east, shift={(-0.1cm,-0.8cm)}]{$\color{violet}\sigma_4$}  (4k);
        \draw [->, violet] (10) edge node[east, shift={(-0.1cm,-0.8cm)}]{$\color{violet}\sigma_5$}  (5k);

        \draw [->] (01) edge[left] node{$\phantom{\cdot}$} (1j);
        \draw [->] (11) edge[left] node{$\phantom{\cdot}$} (2j);
        \draw [->] (21) edge[left] node{$\phantom{\cdot}$} (3j);
        \draw [->] (31) edge[left] node{$\phantom{\cdot}$} (4j);
        \draw [->] (41) edge[left] node{$\phantom{\cdot}$} (5j);

        \draw [->] (0j) edge[left] node{$\phantom{\cdot}$} (1k);
        \draw [->] (1j) edge[left] node{$\phantom{\cdot}$} (2k);
        \draw [->] (2j) edge[left] node{$\phantom{\cdot}$} (3k);
        \draw [->] (3j) edge[left] node{$\phantom{\cdot}$} (4k);
        \draw [->] (4j) edge[left] node{$\phantom{\cdot}$} (5k);

        \draw [->] (00) edge[bend left=10, left] node{$\phantom{\cdot}$} (01);
        \draw [->] (10) edge[bend left=10, left] node{$\phantom{\cdot}$} (11);
        \draw [->] (20) edge[bend left=10, left] node{$\phantom{\cdot}$} (21);
        \draw [->] (30) edge[bend left=10, left] node{$\phantom{\cdot}$} (31);
        \draw [->] (40) edge[bend left=10, left] node{$\phantom{\cdot}$} (41);
        \draw [->] (50) edge[bend left=10, left] node{$\phantom{\cdot}$} (51);

        \draw [->] (01) edge[bend left=10, left] node{$\phantom{\cdot}$} (0j);
        \draw [->] (11) edge[bend left=10, left] node{$\phantom{\cdot}$} (1j);
        \draw [->] (21) edge[bend left=10, left] node{$\phantom{\cdot}$} (2j);
        \draw [->] (31) edge[bend left=10, left] node{$\phantom{\cdot}$} (3j);
        \draw [->] (41) edge[bend left=10, left] node{$\phantom{\cdot}$} (4j);
        \draw [->] (51) edge[bend left=10, left] node{$\phantom{\cdot}$} (5j);

        \draw [->] (0j) edge[bend left=10, left] node{$\phantom{\cdot}$} (0k);
        \draw [->] (1j) edge[bend left=10, left] node{$\phantom{\cdot}$} (1k);
        \draw [->] (2j) edge[bend left=10, left] node{$\phantom{\cdot}$} (2k);
        \draw [->] (3j) edge[bend left=10, left] node{$\phantom{\cdot}$} (3k);
        \draw [->] (4j) edge[bend left=10, left] node{$\phantom{\cdot}$} (4k);
        \draw [->] (5j) edge[bend left=10, left] node{$\phantom{\cdot}$} (5k);

        \draw [->, blue] (00) edge[bend right=11,below] node[shift={(0.5cm,0.3cm)}]{$\color{blue}\sigma_2$}    (21);
        \draw [->, blue] (10) edge[bend right=11,below] node[shift={(0.5cm,0.3cm)}]{$\color{blue}\sigma_3$}    (31);
        \draw [->, blue] (20) edge[bend right=11,below] node[shift={(0.5cm,0.3cm)}]{$\color{blue}\sigma_4$}    (41);
        \draw [->, blue] (30) edge[bend right=11,below] node[shift={(0.5cm,0.3cm)}]{$\color{blue}\sigma_5$}    (51);

        \draw [->, blue] (01) edge[bend right=3,below] node[shift={(0.3cm,0.2cm)}]{$\color{blue}\sigma_2$}    (2j);
        \draw [->, blue] (11) edge[bend right=3,below] node[shift={(0.3cm,0.2cm)}]{$\color{blue}\sigma_3$}    (3j);
        \draw [->, blue] (21) edge[bend right=3,below] node[shift={(0.3cm,0.2cm)}]{$\color{blue}\sigma_4$}    (4j);
        \draw [->, blue] (31) edge[bend right=3,below] node[shift={(0.3cm,0.2cm)}]{$\color{blue}\sigma_4$}    (5j);

        \draw [->, blue] (0j) edge[bend left=8,below] node[shift={(-0.45cm,-0.55cm)}]{$\color{blue}\sigma_2$}    (2k);
        \draw [->, blue] (1j) edge[bend left=8,below] node[shift={(-0.45cm,-0.55cm)}]{$\color{blue}\sigma_3$}    (3k);
        \draw [->, blue] (2j) edge[bend left=8,below] node[shift={(-0.45cm,-0.55cm)}]{$\color{blue}\sigma_4$}    (4k);
        \draw [->, blue] (3j) edge[bend left=8,below] node[shift={(-0.45cm,-0.55cm)}]{$\color{blue}\sigma_5$}    (5k);

%https://tex.stackexchange.com/a/20986/139648
        \draw [decorate,decoration={brace,amplitude=10pt,raise=10pt,mirror}] (00.south west) -- (50.south east) node[midway,yshift=-3em]{\textbf{String length}};
        \draw [decorate,decoration={brace,amplitude=10pt,raise=20pt}] (00.south west) -- (0k.north west) node[midway,xshift=-1cm,yshift=-1cm,rotate=-54]{\textbf{Edit distance}};
      \end{tikzpicture}
    }
    \caption{Bounded Levenshtein reachability from $\sigma: \Sigma^n$ is expressible as an NFA populated by accept states within radius $k$ of $S=q_{n,0}$, which accepts all strings $\sigma'$ within Levenshtein radius $k$ of $\sigma$.}
  \end{figure}
\end{frame}

\begin{frame}[fragile]{Geometrically interpreting the edit calculus}

Each arc plays a specific role. $\duparrow$ handles insertions, $\ddiagarrow$ handles substitutions and $\knightarrow$ handles deletions of $\geq 1$ tokens. Consider some illustrative cases:

\vspace{0.5cm}

\newcommand{\substitutionExample}{
\tikz{
\foreach \x in {0,8,16,24,32,40}{
\fill (\x pt,0pt) circle [radius = 1pt];
\fill (\x pt,8pt) circle [radius = 1pt];
}
\phantom{\fill (0pt,-8pt) circle [radius = 1pt];}
\draw [-to] (0pt,0pt) -- (8pt,0pt);
\draw [-to] (8pt,0pt) -- (16pt,0pt);
\draw [-to] (16pt,0pt) -- (24pt,8pt);
\draw [-to] (24pt,8pt) -- (32pt,8pt);
\draw [-to] (32pt,8pt) -- (40pt,8pt);
}
}

\newcommand{\insertionExample}{
\tikz{
\foreach \x in {0,8,16,24,32,40}{
\fill (\x pt,0pt) circle [radius = 1pt];
\fill (\x pt,8pt) circle [radius = 1pt];
}
\phantom{\fill (0pt,-8pt) circle [radius = 1pt];}
\fill[white] (16pt,0pt) circle [radius = 1.2pt];
\fill[white] (24pt,8pt) circle [radius = 1.2pt];
\draw [-to] (0pt,0pt) -- (8pt,0pt);
\draw [-to] (8pt,0pt) -- (24pt,0pt);
\draw [-to] (24pt,0pt) -- (16pt,8pt);
\draw [-to] (16pt,8pt) -- (32pt,8pt);
\draw [-to] (32pt,8pt) -- (40pt,8pt);
}
}

\newcommand{\deletionExample}{
\tikz{
\foreach \x in {0,8,16,24,32,40}{
\fill (\x pt,0pt) circle [radius = 1pt];
\fill (\x pt,8pt) circle [radius = 1pt];
}
\phantom{\fill (0pt,-8pt) circle [radius = 1pt];}
\draw [-to] (0pt,0pt) -- (8pt,0pt);
\draw [-to] (8pt,0pt) -- (16pt,0pt);
\draw [-to] (16pt,0pt) -- (24pt,0pt);
\draw [-to] (24pt,0pt) -- (40pt,8pt);
}
}

\newcommand{\doubleDeletionExample}{
\tikz{
\foreach \x in {0,8,16,24,32,40}{
\fill (\x pt,0pt) circle [radius = 1pt];
\fill (\x pt,8pt) circle [radius = 1pt];
\fill (\x pt,16pt) circle [radius = 1pt];
}
\draw [-to] (0pt,0pt) -- (24pt,16pt);
\draw [-to] (24pt,16pt) -- (32pt,16pt);
\draw [-to] (32pt,16pt) -- (40pt,16pt);
}
}

\newcommand{\subDelExample}{
\tikz{
\foreach \x in {0,8,16,24,32,40}{
\fill (\x pt,0pt) circle [radius = 1pt];
\fill (\x pt,8pt) circle [radius = 1pt];
\fill (\x pt,16pt) circle [radius = 1pt];
}
\draw [-to] (0pt,0pt) -- (8pt,0pt);
\draw [-to] (8pt,0pt) -- (16pt,8pt);
\draw [-to] (16pt,8pt) -- (32pt,16pt);
\draw [-to] (32pt,16pt) -- (40pt,16pt);
}
}

\newcommand{\subSubExample}{
\tikz{
\foreach \x in {0,8,16,24,32,40}{
\fill (\x pt,0pt) circle [radius = 1pt];
\fill (\x pt,8pt) circle [radius = 1pt];
\fill (\x pt,16pt) circle [radius = 1pt];
}
\draw [-to] (0pt,0pt) -- (8pt,0pt);
\draw [-to] (8pt,0pt) -- (16pt,8pt);
\draw [-to] (16pt,8pt) -- (24pt,16pt);
\draw [-to] (24pt,16pt) -- (32pt,16pt);
\draw [-to] (32pt,16pt) -- (40pt,16pt);
}
}

\newcommand{\insertDeleteExample}{
\tikz{
\foreach \x in {0,8,16,24,32,40,48}{
\fill (\x pt,0pt) circle [radius = 1pt];
\fill (\x pt,8pt) circle [radius = 1pt];
\fill (\x pt,16pt) circle [radius = 1pt];
}
\fill[white] (16pt,16pt) circle [radius = 1.2pt];
\fill[white] (8pt,0pt) circle [radius = 1.2pt];
\fill[white] (16pt,8pt) circle [radius = 1.2pt];
\draw [-to] (0pt,0pt) -- (16pt,0pt);
\draw [-to] (16pt,0pt) -- (8pt,8pt);
\draw [-to] (8pt,8pt) -- (24pt,8pt);
\draw [-to] (24pt,8pt) -- (40pt,16pt);
\draw [-to] (40pt,16pt) -- (48pt,16pt);
}
}

\footnotesize{
\begin{table}[h!]
\begin{tabular}{ccccc}

\texttt{f\hspace{3pt}.\hspace{3pt}\hlorange{[}\hspace{3pt}x\hspace{3pt})} &
\texttt{f\hspace{3pt}.\hspace{3pt}\phantom{(}\hspace{3pt}x\hspace{3pt})} &
\texttt{f\hspace{3pt}.\hspace{3pt}(\hspace{3pt}\hlred{x}\hspace{3pt})} &
\texttt{\hlred{.}\hspace{3pt}\hlred{+}\hspace{3pt}(\hspace{3pt}x\hspace{3pt})} &
\texttt{f\hspace{3pt}\hlorange{.}\hspace{3pt}\hlred{(}\hspace{3pt}x\hspace{3pt};} \\

\texttt{f\hspace{3pt}.\hspace{3pt}\hlorange{(}\hspace{3pt}x\hspace{3pt})} &
\texttt{f\hspace{3pt}.\hspace{3pt}\hlgreen{(}\hspace{3pt}x\hspace{3pt})} &
\texttt{f\hspace{3pt}.\hspace{3pt}(\hspace{3pt}\phantom{x}\hspace{3pt})} &
\texttt{\phantom{f}\hspace{3pt}\phantom{.}\hspace{3pt}(\hspace{3pt}x\hspace{3pt})} &
\texttt{f\hspace{3pt}\hlorange{*}\hspace{3pt}\phantom{(}\hspace{3pt}x\hspace{3pt};} \\

\substitutionExample & \insertionExample & \deletionExample & \doubleDeletionExample & \subDelExample
\end{tabular}
\end{table}
}

\normalsize{Note that the same $\langle\err\sigma, \sigma'\rangle$ pair can have multiple Levenshtein alignments:}
\vspace{0.5cm}

\footnotesize{
\begin{table}[h!]
\begin{tabular}{cc}

\texttt{[\hspace{3pt}\hlorange{,}\hspace{3pt}\hlorange{x}\hspace{3pt}y\hspace{3pt}]} &
\texttt{[\hspace{3pt}\phantom{,}\hspace{3pt},\hspace{3pt}\hlred{x}\hspace{3pt}y\hspace{3pt}]} \\

\texttt{[\hspace{3pt}\hlorange{x}\hspace{3pt}\hlorange{,}\hspace{3pt}y\hspace{3pt}]} &
\texttt{[\hspace{3pt}\hlgreen{x}\hspace{3pt},\hspace{3pt}\phantom{x}\hspace{3pt}y\hspace{3pt}]} \\

\subSubExample & \insertDeleteExample
\end{tabular}
\end{table}
}

\normalsize{Non-uniqueness of geodesics has implications for CFG $\cap$ L-NFA ambiguity.}
\end{frame}

\begin{frame}[fragile]{The nominal Levenshtein automaton}
  The original Levenshtein automaton (Schulz \& Stoyan, 2002):

  \begin{prooftree}
    \AxiomC{$s\in\Sigma \phantom{\land} i \in [0, n] \phantom{\land} j \in [1, k]$}
    \RightLabel{$\duparrow$}
    \UnaryInfC{$(q_{i, j-1} \overset{s}{\rightarrow} q_{i,j}) \in \delta$}
    \DisplayProof
    \hskip 1.5em
    \AxiomC{$s\in\Sigma \phantom{\land} i \in [1, n] \phantom{\land} j \in [1, k]$}
    \RightLabel{$\ddiagarrow$}
    \UnaryInfC{$(q_{i-1, j-1} \overset{s}{\rightarrow} q_{i,j}) \in \delta$}
  \end{prooftree}
  \begin{prooftree}
    \AxiomC{$s=\sigma_i \phantom{\land} i \in [1, n] \phantom{\land} j \in [0, k]$}
    \RightLabel{$\drightarrow$}
    \UnaryInfC{$(q_{i-1, j} \overset{s}{\rightarrow} q_{i,j}) \in \delta$}
    \DisplayProof
    \hskip 1.5em
    \AxiomC{$s=\sigma_i \phantom{\land} i \in [2, n] \phantom{\land} j \in [1, k]$}
    \RightLabel{$\knightarrow$}
    \UnaryInfC{$(q_{i-2, j-1} \overset{s}{\rightarrow} q_{i,j}) \in \delta$}
  \end{prooftree}
  \begin{prooftree}
    \AxiomC{$\vphantom{|}$}
    \RightLabel{$\textsc{Init}$}
    \UnaryInfC{$q_{0,0} \in I$}
    \DisplayProof
    \hskip 1.5em
    \AxiomC{$q_{i, j}$}
    \AxiomC{$|n-i+j| \leq k$}
    \RightLabel{$\textsc{Done}$}
    \BinaryInfC{$q_{i, j}\in F$}
  \end{prooftree}

  We modify the original automaton with a nominal predicate:

  \begin{prooftree}
    \AxiomC{$i \in [0, n] \phantom{\land} j \in [1, k]$}
    \RightLabel{$\duparrow$}
    \UnaryInfC{$(q_{i, j-1} \overset{{\color{orange}[\neq \sigma_{i+1}]}}{\rightarrow} q_{i,j}) \in \delta$}
    \DisplayProof
    \hskip 1.5em
    \AxiomC{$i \in [1, n] \phantom{\land} j \in [1, k]$}
    \RightLabel{$\ddiagarrow$}
    \UnaryInfC{$(q_{i-1, j-1} \overset{{\color{orange}[\neq \sigma_i]}}{\rightarrow} q_{i,j}) \in \delta$}
  \end{prooftree}
  \begin{prooftree}
    \AxiomC{$i \in [1, n] \phantom{\land} j \in [0, k]$}
    \RightLabel{$\drightarrow$}
    \UnaryInfC{$(q_{i-1, j} \overset{{\color{orange}[=\sigma_i]}}{\rightarrow} q_{i,j}) \in \delta$}
    \DisplayProof
    \hskip 1.5em
    \AxiomC{$d \in [1, d_{\max}] \phantom{\land} i \in [d + 1, n] \phantom{\land} j \in [d, k]$}
    \RightLabel{$\knightarrow$}
    \UnaryInfC{$(q_{i-d-1, j-d} \overset{{\color{orange}[=\sigma_i]}}{\rightarrow} q_{i,j}) \in \delta$}
  \end{prooftree}
\end{frame}

\begin{frame}{Background: Context-free grammars}
  In a context-free grammar $\mathcal{G} = \langle V, \Sigma, P, S\rangle$ all productions are of the form $P: V\times (V \cup \Sigma)^+$, i.e., RHS may contain any number of nonterminals, $V$. Recognition decidable in $\mathcal{O}(n^\omega)$, n.b. CFLs are \textbf{not} closed under $\cap$!\newline\\
  %
  For example, consider the grammar $\underline{S \rightarrow S S \mid ( S ) \mid ()}$. This represents the language of balanced parentheses, e.g. $(), ()(), (()), ()(()), (()()), (())()\ldots$\newline\\
  %
  Every CFG has a normal form $P^*: V \times (V^2 \mid \Sigma)$, i.e., every production can be refactored into either $v_0 \rightarrow v_1 v_2$ or $v_0 \rightarrow \sigma$, where $v_{0\ldots2}: V$ and $\sigma: \Sigma$, e.g., $\{S \rightarrow S S \mid ( S ) \mid ()\}\Leftrightarrow^*\{S\rightarrow XR \mid SS \mid LR, L \rightarrow (, R \rightarrow ), X\rightarrow LS\}$

  \begin{center}
    \scalebox{0.8}{
      \begin{tikzpicture}[font=\sffamily,breathe dist/.initial=4ex]
        \foreach \X [count=\Y,remember=\Y as \LastY] in
          {finite,regular,context-free}
          {\ifnum\Y=1
        \node[ellipse,draw,outer sep=0pt] (F-\Y) {\X};
        \else
        \path[decoration={text along path,
        text={|\sffamily|\X},text align=center,raise=0.9ex},decorate]
        let \p1=($(F-\LastY.north)-(F-\LastY.west)$)
        in (F-\LastY.west) arc(180:0:\x1 and \y1);
        \path let \p1=($([yshift=\pgfkeysvalueof{/tikz/breathe dist}]F-\LastY.north)
        -(F-\LastY.south)$),
        \p2=($(F-1.east)-(F-1.west)$),\p3=($(F-1.north)-(F-1.south)$)
        in ($([yshift=\pgfkeysvalueof{/tikz/breathe dist}]F-\LastY.north)!0.5!(F-\LastY.south)$)
        node[minimum height=\y1,minimum width={\y1*\x2/\y3},
        draw,ellipse,inner sep=0pt, fill=black!30!white] (F-\Y){};
        \fi
        }
        \foreach \X [count=\Y,remember=\Y as \LastY] in
          {finite,regular,context-free,conjunctive}
          {\ifnum\Y=1
        \node[ellipse,draw,outer sep=0pt] (F-\Y) {\X};
        \else
        \path[decoration={text along path,
        text={|\sffamily|\X},text align=center,raise=0.9ex},decorate]
        let \p1=($(F-\LastY.north)-(F-\LastY.west)$)
        in (F-\LastY.west) arc(180:0:\x1 and \y1);
        \path let \p1=($([yshift=\pgfkeysvalueof{/tikz/breathe dist}]F-\LastY.north)
        -(F-\LastY.south)$),
        \p2=($(F-1.east)-(F-1.west)$),\p3=($(F-1.north)-(F-1.south)$)
        in ($([yshift=\pgfkeysvalueof{/tikz/breathe dist}]F-\LastY.north)!0.5!(F-\LastY.south)$)
        node[minimum height=\y1,minimum width={\y1*\x2/\y3},
        draw,ellipse,inner sep=0pt] (F-\Y){};
        \fi}
      \end{tikzpicture}
    }
  \end{center}
\end{frame}

%\begin{frame}{Background: Conjunctive grammars}
%  Conjunctive grammars naturally extend CFGs with CFL union and intersection, respecting closure under those operations. Equivalent to trellis automata, which are like contractive elementary cellular automata. Language inclusion is decidable in P.\\
%
%  \begin{prooftree}
%    \AxiomC{$\Gamma \vdash \mathcal{G}_1, \mathcal{G}_2 : \texttt{CG}$}
%    \RightLabel{$\cap$}
%    \UnaryInfC{$\Gamma \vdash \exists \:\mathcal{G}_3: \texttt{CG}\:.\:\mathcal{L}_{\mathcal{G}_1} \cap \mathcal{L}_{\mathcal{G}_1} \leftrightarrow \mathcal{G}_3$}
%  \end{prooftree}
%
%  \begin{center}
%    \scalebox{0.8}{
%      \begin{tikzpicture}[font=\sffamily,breathe dist/.initial=4ex]
%        \foreach \X [count=\Y,remember=\Y as \LastY] in
%          {finite,regular,context-free,conjunctive}
%          {\ifnum\Y=1
%        \node[ellipse,draw,outer sep=0pt] (F-\Y) {\X};
%        \else
%        \path[decoration={text along path,
%        text={|\sffamily|\X},text align=center,raise=0.9ex},decorate]
%        let \p1=($(F-\LastY.north)-(F-\LastY.west)$)
%        in (F-\LastY.west) arc(180:0:\x1 and \y1);
%        \path let \p1=($([yshift=\pgfkeysvalueof{/tikz/breathe dist}]F-\LastY.north)
%        -(F-\LastY.south)$),
%        \p2=($(F-1.east)-(F-1.west)$),\p3=($(F-1.north)-(F-1.south)$)
%        in ($([yshift=\pgfkeysvalueof{/tikz/breathe dist}]F-\LastY.north)!0.5!(F-\LastY.south)$)
%        node[minimum height=\y1,minimum width={\y1*\x2/\y3},
%        draw,ellipse,inner sep=0pt, fill=black!30!white] (F-\Y){};
%        \fi
%        }
%        \foreach \X [count=\Y,remember=\Y as \LastY] in
%          {finite,regular,context-free,conjunctive,context sensitive}
%          {\ifnum\Y=1
%        \node[ellipse,draw,outer sep=0pt] (F-\Y) {\X};
%        \else
%        \path[decoration={text along path,
%        text={|\sffamily|\X},text align=center,raise=0.9ex},decorate]
%        let \p1=($(F-\LastY.north)-(F-\LastY.west)$)
%        in (F-\LastY.west) arc(180:0:\x1 and \y1);
%        \path let \p1=($([yshift=\pgfkeysvalueof{/tikz/breathe dist}]F-\LastY.north)
%        -(F-\LastY.south)$),
%        \p2=($(F-1.east)-(F-1.west)$),\p3=($(F-1.north)-(F-1.south)$)
%        in ($([yshift=\pgfkeysvalueof{/tikz/breathe dist}]F-\LastY.north)!0.5!(F-\LastY.south)$)
%        node[minimum height=\y1,minimum width={\y1*\x2/\y3},
%        draw,ellipse,inner sep=0pt] (F-\Y){};
%        \fi}
%      \end{tikzpicture}
%    }
%  \end{center}
%\end{frame}


\begin{frame}[fragile]{Background: Closure properties of formal languages}
  Formal languages are not always closed under set-theoretic operations, e.g., CFL $\cap$ CFL is not CFL in general. Let $\cdot$ denote concatenation, $*$ be Kleene star, and $\complement$ be complementation:\\
  \begin{table}
    \begin{tabular}{c|ccccc}
      & $\cup$ & $\cap$ & $\cdot$ & $*$ & $\complement$ \\\hline
      Finite$^1$                                  & \cmark & \cmark     & \cmark  & \cmark  & \cmark \\
      Regular$^1$                                 & \cmark & \cmark     & \cmark  & \cmark  & \cmark \\
      \rowcolor{slightgray} Context-free$^{1, 2}$ & \cmark & \xmark$^\dagger$ & \cmark  & \cmark  & \xmark \\
      Conjunctive$^{1,2}$                         & \cmark & \cmark     & \cmark  & \cmark  & ?      \\
      Context-sensitive$^2$                       & \cmark & \cmark     & \cmark  & +       & \cmark \\
      Recursively Enumerable$^2$                  & \cmark & \cmark     & \cmark  & \cmark  & \xmark \\
    \end{tabular}
  \end{table}
  We would like a language family that is (1) tractable, i.e., has polynomial recognition and search complexity and (2) reasonably expressive, i.e., can represent syntactic properties of real-world programming languages.\vspace{0.2cm}

  $^\dagger$ However, CFLs are closed under intersection with regular languages.
\end{frame}

\begin{frame}[fragile]{The Bar-Hillel construction and its specialization}
  The original Bar-Hillel construction provides a way to construct a grammar for the intersection of a regular and context-free language.
  \noindent\begin{prooftree}
             \AxiomC{$q \in I \phantom{\land} r \in F\vphantom{\overset{a}{\rightarrow}}$}
             \RightLabel{$\downarrow$}
             \UnaryInfC{$\big(S\rightarrow q S r\big) \in P_\cap$}
             \DisplayProof
             \hskip 2em
             \AxiomC{$(A \rightarrow a) \in P$}
             \AxiomC{$(q\overset{a}{\rightarrow}r) \in \delta$}
             \RightLabel{$\uparrow$}
             \BinaryInfC{$\big(qAr\rightarrow a\big)\in P_\cap$}
             \DisplayProof
             \AxiomC{$(w \rightarrow xz) \in P\vphantom{\overset{a}{\rightarrow}}$}
             \AxiomC{$p,q,r \in Q$}
             \RightLabel{$\Join$}
             \BinaryInfC{$\big(pwr\rightarrow (pxq)(qzr)\big) \in P_\cap$}
  \end{prooftree}

  We specialize the Bar-Hillel construction to nominal Levenshtein automata:

  \begin{prooftree}
    \AxiomC{$(A \rightarrow a) \in P$}
    \AxiomC{$(q\overset{{\color{orange}[\cdot]}}{\rightarrow}r) \in \delta$}
    \AxiomC{$\color{orange}a[\cdot]$}
    \RightLabel{$\hat\uparrow$}
    \TrinaryInfC{$\big(qAr\rightarrow a\big)\in P_\cap$}
    \DisplayProof
    \AxiomC{$\vphantom{\overset{[\cdot]}{\rightarrow}}\color{orange} w \triangleleft pr \phantom{\land} x \triangleleft pq \phantom{\land} z \triangleleft qr$}
    \AxiomC{$(w \rightarrow xz) \in P\vphantom{\overset{a}{\rightarrow}}$}
    \AxiomC{$p,q,r \in Q$}
    \RightLabel{$\hat\Join$}
    \TrinaryInfC{$\big(pwr\rightarrow (pxq)(qzr)\big) \in P_\cap$}
  \end{prooftree}

  Where $\triangleleft$ denotes compatibility between the Parikh map of a nonterminal and Levenshtein margin between two NFA states, see our paper for details.
\end{frame}

\section{Algebraic Parsing}\label{sec:algebraic-parsing}

\begin{frame}[fragile]{Parsing for linear algebraists}
  Given a CFG $\mathcal{G} \coloneqq \langle V, \Sigma, P, S\rangle$ in Chomsky Normal Form, we can construct a recognizer $R_\mathcal{G}: \Sigma^n \rightarrow \mathbb{B}$ for strings $\sigma: \Sigma^n$ as follows. Let $2^V$ be our domain, $0$ be $\varnothing$, $\oplus$ be $\cup$, and $\otimes$ be defined as follows:

  \vspace{-7pt}
  \[
    s_1 \otimes s_2 \coloneqq \{C \mid \langle A, B\rangle \in s_1 \times s_2, (C\rightarrow AB) \in P\}\\
    \text{e.g.},
    \{A \rightarrow BC, C \rightarrow AD, D \rightarrow BA\} \subseteq P \vdash \{A, B, C\} \otimes \{B, C, D\} = \{A, C\}
  \]
  \vspace{-1.5cm}

  \noindent If we define $\sigma_r^{\shur} \coloneqq \{w \mid (w \rightarrow \sigma_r) \in P\}$, then initialize $M^0_{r+1=c}(\mathcal{G}', e) := \;\sigma_r^{\shur}$ and solve for the fixpoint $M^* = M + M^2$,\vspace{-10pt}

  \begin{align*}
    M^0:=
    \begin{pNiceMatrix}[nullify-dots,xdots/line-style=loosely dotted]
      \varnothing & \sigma_1^\shri & \varnothing & \Cdots & \varnothing \\
      \Vdots      & \Ddots         & \Ddots      & \Ddots & \Vdots\\
                  &                &             &        & \varnothing\\
                  &                &             &        & \sigma_n^\shup \\
      \varnothing & \Cdots         &             &        & \varnothing
    \end{pNiceMatrix} &\Rightarrow \ldots \Rightarrow M^* =
    \begin{pNiceMatrix}[nullify-dots,xdots/line-style=loosely dotted]
      \varnothing & \sigma_1^\shri & \Lambda & \Cdots & \Lambda^*_\sigma\\
      \Vdots      & \Ddots         & \Ddots  & \Ddots & \Vdots\\
                  &                &         &        & \Lambda\\
                  &                &         &        & \sigma_n^\shup \\
      \varnothing & \Cdots         &         &        & \varnothing
    \end{pNiceMatrix}
  \end{align*}

  \noindent $S \Rightarrow^* \sigma \iff \sigma \in \mathcal{L}(\mathcal{G})$ iff $S \in \Lambda^*_\sigma$, i.e., $\mathds{1}_{\Lambda^*_\sigma}(S) \iff \mathds{1}_{\mathcal{L}(\mathcal{G})}(\sigma)$.
\end{frame}

\begin{frame}[fragile]{Lattices, Matrices and Trellises}
  The art of treillage has been practiced from ancient through modern times.
  \begin{center}
    \begin{tabular}{ c c c c c }
      \includegraphics[width=0.17\textwidth]{../figures/trellis.png} & & \includegraphics[width=0.12\textwidth]{../figures/grid_topiary.jpeg} & & \includegraphics[width=0.23\textwidth]{../figures/tree_pyramid.jpeg} \\\\
      \includegraphics[width=0.17\textwidth]{../figures/jiaxian_triangle.png} & &
      \rotatebox{37}{\scalebox{0.9}{$
      \begin{pNiceMatrix}[nullify-dots,xdots/line-style=loosely dotted,delimiters-color=gray]
        \sigma_1^\shri & \Lambda & \Cdots & \Lambda^*_\sigma\\
                       & \Ddots  & \Ddots & \Vdots\\
                       &         &        & \Lambda\\
                       &         &        & \sigma_n^\shup \\
      \end{pNiceMatrix}$}} & & \scalebox{0.23}{\mkTrellis{9}}\\
      Jia Xian Triangle & & CYK Parsing & & Trellis Automaton\\
      Jia, $\sim$1030 A.D. & & Sakai, 1961 A.D. & & Dyer, 1980 A.D.\\
    \end{tabular}
  \end{center}
\end{frame}

\begin{frame}{A few observations on algebraic parsing}
  \begin{itemize}
    \item The matrix $\mathbf M^*$ is strictly upper triangular, i.e., nilpotent of degree $n$
    \item Recognizer can be translated into a parser by storing backpointers\\\\
  \end{itemize}\vspace{0.2cm}
  \begin{tabular}{ c c c }
    \small{$\mathbf{M}_1 = \mathbf{M}_0 + \mathbf{M}_0^2$} & \small{$\mathbf{M}_2 = \mathbf{M}_1 + \mathbf{M}_1^2$} & \small{$\mathbf{M}_3 = \mathbf{M}_2 + \mathbf{M}_2^2 = \mathbf{M}_4$} \\
    \includegraphics[trim=420 288 0 0,clip, width=3.6cm]{../figures/parse2.png} &
    \includegraphics[trim=420 285 0 0,clip, width=3.6cm]{../figures/parse3.png} &
    \includegraphics[trim=420 287 0 0,clip, width=3.63cm]{../figures/parse4.png}
  \end{tabular}
  \begin{itemize}
    \item The $\otimes$ operator is \textit{not} associative: $S \otimes (S \otimes S) \neq (S \otimes S) \otimes S$
    \item Built-in error recovery: nonempty submatrices = parsable fragments
    \item \texttt{seekFixpoint \{ it + it * it \}} is sufficient but unnecessary
    \item If we had a way to solve for $\mathbf{M = M + M}^2$ directly, power iteration would be unnecessary, could solve for $\mathbf{M = M}^2$ above superdiagonal
  \end{itemize}
\end{frame}

\begin{frame}[fragile]{Satisfiability + holes}
  \begin{itemize}
    \item Can be lowered onto a Boolean tensor $\mathbb{B}_2^{n\times n \times |V|}$ (Valiant, 1975)
    \item Binarized CYK parser can be efficiently compiled to a SAT solver
    \item Enables sketch-based synthesis in either $\sigma$ or $\mathcal G$: just use variables!
    \item We simply encode the characteristic function, i.e. $\mathds{1}_{\subseteq V}: V\rightarrow \mathbb{Z}_2^{|V|}$
    \item $\oplus, \otimes$ are defined as $\boxplus, \boxtimes$, so that the following diagram commutes:
    \[\begin{tikzcd}
        2^V \times 2^V \arrow[r, "\oplus/\otimes"] \arrow[d, "\mathds{1}^2"]
        & 2^V \arrow[d, "\mathds{1}\phantom{^{-1}}"] \\
        \mathbb{Z}_2^{|V|} \times \mathbb{Z}_2^{|V|} \arrow[r, "\boxplus/\boxtimes", labels=below] \arrow[u, "\mathds{1}^{-2}"]
        & \mathbb{Z}_2^{|V|} \arrow[u, "\mathds{1}^{-1}"]
    \end{tikzcd}\]
    \item These operators can be lifted into matrices/tensors in the usual way
    \item In most cases, only a few nonterminals are active at any given time
%    \item More sophisticated representations are known for $\binom{n}{0 \leq k}$ subsets
%    \item If density is desired, possible to use the Maculay representation
%            \item Set joins are an active topic of research in SQL query optimization
  \end{itemize}
\end{frame}

\begin{frame}[fragile]{Satisfiability + holes}
  Let us consider an example with two holes, $\sigma = 1$ \hole\phantom{.}\hole, and the grammar being $G\coloneqq\{S\rightarrow N O N, O \rightarrow + \mid \times, N \rightarrow 0 \mid 1\}$. This can be rewritten into CNF as $G'\coloneqq \{S \rightarrow N L, N \rightarrow 0 \mid 1, O \rightarrow × \mid +, L \rightarrow O N\}$. Using the algebra where $\oplus=\cup$, $X \otimes Z = \big\{\;w \mid \langle x, z\rangle \in X \times Z, (w\rightarrow xz) \in P\;\big\}$, the fixpoint $M' = M + M^2$ can be computed as follows:\\\vspace{10pt}

  \resizebox{\textwidth}{!}{
{\renewcommand{\arraystretch}{1.2}
\noindent\phantom{...}\begin{tabular}{|c|c|c|c|}
  \hline
  & $2^V$ & $\mathbb{Z}_2^{|V|}$ & $\mathbb{Z}_2^{|V|}\rightarrow\mathbb{Z}_2^{|V|}$\\\hline
  $M_0$ & \begin{pmatrix}
  \phantom{V} & \tiny{\{N\}} &         &             \\
              &              & \{N,O\} &             \\
              &              &         & \{N,O\} \\
              &              &         &
  \end{pmatrix} & \begin{pmatrix}
  \phantom{V} & \ws\bs\ws\ws &              &              \\
              &              & \ws\bs\bs\ws &              \\
              &              &              & \ws\bs\bs\ws \\
              &              &              &
  \end{pmatrix} & \begin{pmatrix}
     \phantom{V} & V_{0, 1} &          &          \\
                 &          & V_{1, 2} &          \\
                 &          &          & V_{2, 3} \\
                 &          &          &
  \end{pmatrix} \\\hline
  $M_1$ & \begin{pmatrix}
  \phantom{V} & \tiny{\{N\}} & \varnothing &         \\
              &              & \{N,O\}     & \{L\}   \\
              &              &             & \{N,O\} \\
              &              &             &
  \end{pmatrix} & \begin{pmatrix}
  \phantom{V} & \ws\bs\ws\ws & \ws\ws\ws\ws &              \\
              &              & \ws\bs\bs\ws & \bs\ws\ws\ws \\
              &              &              & \ws\bs\bs\ws \\
              &              &              &
  \end{pmatrix} & \begin{pmatrix}
                   \phantom{V} & V_{0, 1} & V_{0, 2} &          \\
                   &          & V_{1, 2} & V_{1, 3} \\
                   &          &          & V_{2, 3} \\
                   &          &          &
  \end{pmatrix} \\\hline
  $M_\infty$ & \begin{pmatrix}
  \phantom{V} & \tiny{\{N\}} & \varnothing & \{S\}   \\
              &              & \{N,O\}     & \{L\}   \\
              &              &             & \{N,O\} \\
              &              &             &
  \end{pmatrix} & \begin{pmatrix}
  \phantom{V} & \ws\bs\ws\ws & \ws\ws\ws\ws & \ws\ws\ws\bs \\
              &              & \ws\bs\bs\ws & \bs\ws\ws\ws \\
              &              &              & \ws\bs\bs\ws \\
              &              &              &
  \end{pmatrix} & \begin{pmatrix}
                   \phantom{V} & V_{0, 1} & V_{0, 2} & V_{0, 3} \\
                   &          & V_{1, 2} & V_{1, 3} \\
                   &          &          & V_{2, 3} \\
                   &          &          &
  \end{pmatrix}\\\hline
\end{tabular}
}
  }
\end{frame}

\begin{frame}[fragile]{Semiring algebras: Part I}
  The prior solution tell us whether $A(\sigma)$ is nonempty, but forgets the solution(s). To solve for $A(\sigma)$, a na\"ive approach accumulates a mapping of nonterminals to sets of strings. Letting $D = V \rightarrow \mathcal{P}(\Sigma^*)$, we define $\oplus, \otimes: D \times D \rightarrow D$. Initially, we construct $M_0[r+1=c] = p(\sigma_r)$ using:

  \begin{equation*}
    p(s: \Sigma) \mapsto \{w \mid (w \rightarrow s)\in P\} \text{ and } p(\hole) \mapsto \bigcup_{s\in \Sigma} p(s)
  \end{equation*}

  $p(\cdot)$ constructs the superdiagonal, then we solve for $\Lambda_\sigma^*$ using the algebra:

  \begin{equation*}
    X \oplus Z \mapsto \big\{w \stackrel{+}{\Rightarrow} (X \circ w) \cup (Z \circ w) \mid w \in \pi_1(X \cup Z)\big\}
  \end{equation*}

  \begin{equation*}
    X \otimes Z \mapsto \bigoplus_{w, x, z}\big\{w \stackrel{+}{\Rightarrow} (X\circ x)(Z\circ z) \mid (w\rightarrow xz) \in P, x\in X, z\in Z\big\}
  \end{equation*}

  \noindent After $M_\infty$ is attained, the solutions can be read off via $\Lambda_\sigma^* \circ S$. The issue here is exponential growth when eagerly computing the transitive closure.
\end{frame}

\begin{frame}[fragile]{Semiring algebras: Part II}
  The prior encoding can be improved using an ADT $\mathbb{T}_3 = (V \cup \Sigma) \rightharpoonup \mathbb{T}_2$ where $\mathbb{T}_2 = (V \cup \Sigma) \times (\mathbb{N} \rightharpoonup \mathbb{T}_2\times\mathbb{T}_2)$. We construct $\hat\sigma_r = \dot{p}(\sigma_r)$ using:

  \begin{equation*}
    \dot{p}(s: \Sigma) \mapsto \Big\{\mathbb{T}_2\big(w, \big[\langle\mathbb{T}_2(s), \mathbb{T}_2(\varepsilon)\rangle\big]\big) \mid (w \rightarrow s)\in P\Big\} \text{ and } \dot{p}(\hole) \mapsto \bigoplus_{s\in \Sigma} p(s)
  \end{equation*}

  \noindent We then compute the fixpoint $M_\infty$ by redefining $\oplus, \otimes: \mathbb{T}_3 \times \mathbb{T}_3 \rightarrow \mathbb{T}_3$ as:

  \begin{equation*}
    X \oplus Z \mapsto \bigcup_{\mathclap{k\in \pi_1(X \cup Z)}}\Big\{k \Rightarrow \mathbb{T}_2(k, x \cup z) \mid x \in \pi_2(X\circ k), z \in \pi_2(Z\circ k)\Big\}
  \end{equation*}

  \begin{equation*}
    X \otimes Z \mapsto \bigoplus_{\mathclap{(w\rightarrow xz) \in P}}\Big\{\mathbb{T}_2\big(w, \big[\langle X\circ x, Z\circ z\rangle\big]\big) \mid x \in \pi_1(X), z \in \pi_1(Z)\Big\}
  \end{equation*}
\end{frame}

\begin{frame}[fragile]{Semiring algebras: Part III}
  \begin{equation*}
    \hspace{-3cm}L(p) = 1 + p L(p) \phantom{space} P(a) = \Sigma + V + V L\big(V^2P(a)^2\big)
  \end{equation*}
  \vspace{-1cm}\begin{figure}[H]
                 \resizebox{0.9\columnwidth}{!}{
                   \begin{tikzpicture}
                   [
                     grow                    = right,
                     sibling distance        = 3em,
                     level distance          = 5em,
                     edge from parent/.style = {draw, -latex},
                     every node/.style       = {font=\footnotesize},
                     sloped
                   ]
                     \node [root] {S}
                     child { node [env] {BC}
                     child { node [root] {B}
                     child { node [env] {RD}
                     child { node [root] {R} edge from parent node [above] }
                     child { node [root] {D} edge from parent node [above] }
                     edge from parent node [above] }
                     edge from parent node [below] }
                     child { node [root] {C}
                     child { node [env] {\ldots\vphantom{BB}} edge from parent node [below] }
%  child { edge from parent node [above] {\ldots} }
                     edge from parent node [below] }
                     edge from parent node [above] }
                     child { node [env] {\ldots\vphantom{BB}} edge from parent node [below] }
                     child { node [env] {AB}
                     child { node [root] {A}
                     child {
                       node [env] {QC}
                       child { node [root] {Q} edge from parent node [above] }
                       child { node [root] {C} edge from parent node [above] }
                       edge from parent node [above]
                     }
%    child { node [env] {ZQ} edge from parent node [above] }
                     child { node [env] {\ldots\vphantom{BB}} edge from parent node [below] }
                     edge from parent node [below] }
                     child { node [root] {B}
                     child { node [env] {RD}
                     child { node [root] {R} edge from parent node [above] }
                     child { node [root] {D} edge from parent node [above] }
                     edge from parent node [above] }
                     edge from parent node [below] }
                     edge from parent node [above] };
                   \end{tikzpicture}
                 }
                 \vspace{-0.5cm}\caption{A partial $\mathbb{T}_2$ for the grammar with productions $P=\{S \rightarrow BC \mid \ldots \mid AB, B\rightarrow RD \mid \ldots, A\rightarrow QC \mid \ldots\}$.}
  \end{figure}
\end{frame}

\begin{frame}[fragile]{Sampling trees with replacement}
  Given a probabilistic CFG whose productions indexed by each nonterminal are decorated with a probability vector $\mathbf{p}$ (this may be uniform in the non-probabilistic case), we define a tree sampler $\Gamma: \mathbb{T}_2 \rightsquigarrow \mathbb{T}$ which recursively samples children according to a Multinoulli distribution:

  \begin{equation*}
    \Gamma(T) \mapsto \begin{cases}
      \text{Multi} \big(\texttt{children}(T), \mathbf{p}\big) & \text{ if $T$ is a \texttt{root}} \\
      \big\langle \Gamma\big(\pi_1(T)\big), \Gamma\big(\pi_2(T)\big) \big\rangle & \text{ if $T$ is a \texttt{child} }
    \end{cases}
  \end{equation*}

  This is closely related to the generating function for the ordinary Boltzmann sampler from analytic combinatorics,

  \begin{equation*}
    \Gamma C(x) \mapsto \begin{cases}
                          \text{Bern} \left(\frac{A(x)}{A(x) + B(x)}\right) \rightarrow \Gamma A(x) \mid \Gamma B(x) & \text{ if } \mathcal{C}=\mathcal{A}+\mathcal{B} \\
                          \big\langle \Gamma A(x), \Gamma B(x)\big\rangle & \text{ if } \mathcal{C}=\mathcal{A} \times \mathcal{B}
    \end{cases}
  \end{equation*}

  \noindent however unlike Duchon et al. (2004), rejection is unnecessary to ensure exact-size sampling, as all trees in $\mathbb{T}_2$ will necessarily be the same size.
\end{frame}

\begin{frame}[fragile]{A pairing function for replacement-free tree sampling}
  The total number of trees induced by a given sketch template is given by:

  \begin{equation*}
    |T: \mathbb{T}_2| \mapsto \begin{cases}
%    \big|\{s \mid \big(\texttt{root}(T) \rightarrow s\big) \in P_\cap\}\big| & \text{if $T$ is a leaf,} \\
                                1  & \text{if $T$ is a leaf,} \\
                                \sum_{\langle T_1, T_2\rangle \in \texttt{children}(T)} |T_1| \cdot |T_2| & \text{otherwise.}
    \end{cases}
  \end{equation*}\\

  To sample from $\mathbb{T}_2$ without replacement, we define a pairing function:

% Small text
\begin{footnotesize}
  \begin{equation*}
  \varphi(T: \mathbb{T}_2, i: \mathbb{Z}_{|T|}) \mapsto \begin{cases}
  \texttt{BTree}\big(\texttt{root}(T)\big) & \text{if $T$ is a leaf,} \vspace{5pt}\\
  \textbf{let } r = |\texttt{children}(T)|,\\
  \phantom{\textbf{Let }} F(n) = \sum_{\langle l, r\rangle \in \texttt{children}[0 \ldots n]}}|l|\cdot|r|,\\
  \phantom{\textbf{Let }} F^{-1}(u)=\inf \big\{x \mid u \leq F(x)\big\},\\
  \phantom{\textbf{Let }} q=i - F\big(F^{-1}(i)\big),\\
  \phantom{\textbf{Let }} l, r = \texttt{children}[t],\\
  \phantom{\textbf{Let }} q_1, q_2 =\big\langle\lfloor\frac{q}{|r|}\rfloor, q \pmod{|r|}\big\rangle,\\
  \phantom{\textbf{Let }} T_1, T_2 = \big\langle\varphi(l, q_1), \varphi(r, q_2)\big\rangle \textbf{ in } \\
  \texttt{BTree}\big(\texttt{root}(T), T_1, T_2\big) & \text{otherwise.} \\
  \end{cases}
  \end{equation*}
\end{footnotesize}
\end{frame}

\begin{frame}[fragile]{Enumerative search with reranking}
Given $\sigma: \Sigma^*$, let $P_\theta(\sigma) = \prod_{i=1}^{|\sigma|}P_\theta(\sigma_i \mid \sigma_{i-1}\ldots\sigma_{i-n})$. This defines an ordering over $\Sigma^*$. Then, for each retrieved set $\sigma \in \hat{a} \subseteq A$ drawn before timeout, we score the repair and return $\hat{A}$ in ascending order:

\begin{algorithm}[H]
\caption{Enumerative tree sampling with n-gram reranking}
\label{alg:enum_ngram}
\begin{algorithmic}[1]
\Require $T: \mathbb{T}_2$ intersection grammar, $P_\theta: \Sigma^d \rightarrow \mathbb{R}$ Markov chain
\State $\hat{A} \gets \varnothing, \texttt{seed} \gets 0$ \Comment{Initialize set of repairs.}
%\State $[\mathcal{T, P}]\texttt{.comparator} = \lambda\langle\sigma, q, \gamma\rangle.(\frac{\gamma}{|\sigma|})$
\For{$\texttt{seed} < |T|$ and uninterrupted}
\State $t \gets \varphi'(T, \texttt{seed++})}$ \Comment{Decode fresh tree and add.}
\State $\hat{A} \gets \hat{A} \cup \{\text{\textleaf}(t)\}$
%\Until{interrupted or $\texttt{seed} = |T|$.}
\EndFor
\State \Return $[\sigma \in \hat{A} \textbf{ ranked by } \text{NLL}(\sigma)]$ \Comment{Rerank by n-gram likelihood.}
\end{algorithmic}
\end{algorithm}

The issue is, the bijection is defined over labeled binary trees and does not guarantee unique strings, as the CFG may be ambiguous. In practice, this matters if we care about $\mathcal{O}(n)$ convergence!
\end{frame}

\begin{frame}[fragile]{Potential ambiguity of Levenshtein-Bar-Hillel grammars}
The previous technique enumerates parse trees in a given $\mathbb{T}_2$, but does not guarantee string uniqueness since the CFG may be ambiguous.

\begin{lemma}\label{lemma:ambiguity}
If the FSA, $\alpha$, is ambiguous, the intersection CFG, $G_\cap$, can be ambiguous.
\end{lemma}

\begin{proof}
Let $\ell$ be the language defined by $G=\{S\rightarrow LR, L \rightarrow\texttt{(}, R \rightarrow\texttt{)}\}$, where $\alpha=L(\err\sigma, 2)$, the broken string $\err\sigma$ is $\texttt{)(}$, and $\mathcal{L}(G_\cap) = \ell \cap \mathcal{L}(\alpha)$. Then, $\mathcal{L}(G_\cap)$ contains the following two identical repairs: \texttt{\hlred{)}(\hlgreen{)}} with the parse $S \rightarrow q_{00}Lq_{21}\phantom{.}q_{21}Rq_{22}$, and \texttt{\hlorange{(}\hlorange{)}} with the parse $S \rightarrow q_{00}Lq_{11}\phantom{.}q_{11}Rq_{22}$.
\end{proof}

We can eliminate ambiguity and thereby improve the rate of convergence for natural syntax repair by first translating $\mathbb{T}_2$ into an FSA.
\end{frame}

\begin{frame}[fragile]{Existence of an FSA that generates $\mathcal{L}(G_\cap)$}
There is an FSA generating $\mathcal{L}(G_\cap)$. We first show this non-constructively:

\begin{lemma}\label{lemma:upper-bound}
The intersection grammar, $G_\cap$, is acyclic.
\end{lemma}

\begin{proof}
Assume $G_\cap$ is cyclic. Then $\mathcal{L}(G_\cap)$ must be infinite. But since $G_\cap$ generates $\ell \cap \mathcal{L}(\alpha)$ by construction and $\alpha$ is acyclic, $\mathcal{L}(G_\cap)$ is necessarily finite. Therefore, $G_\cap$ must not be cyclic.
\end{proof}

Since $G_\cap$ is acyclic and thus finite, it must be representable as an FSA. Using an FSA for decoding has many advantages, notably, it can be efficiently minimized and decoded in order of n-gram likelihood using a Markov chain or standard pretrained autoregressive language model.
\end{frame}

\begin{frame}[fragile]{Translating from $T_2$ to a DFA}
Let $+, *: \mathcal{A}\times \mathcal{A} \rightarrow \mathcal{A}$ be automata operators satisfying the property $\mathcal{L}(A_1 + A_2) = \mathcal{L}(A_1)\cup\mathcal{L}(A_2)$, and $\mathcal{L}(A_1 * A_2) = \mathcal{L}(A_1)\times\mathcal{L}(A_2)$. We can translate $\mathbb{T}_2$ to $\mathcal{A}$, as follows, recalling FSAs are closed over $+, *$:

\begin{equation*}
\mathcal{Y}(T:\mathbb{T}_2) \mapsto \begin{cases}
\alpha \mid \mathcal{L}(\alpha) = \{T\} & T: \Sigma, \\
\sum_{\langle T_1, T_2\rangle \in \texttt{children}(T)} \mathcal{Y}(T_1)*\mathcal{Y}(T_2) & T: VL\big(V^2P(a)^2\big)
\end{cases}
\end{equation*}
%
In the case of LBH intersections, $\mathcal{Y}(G_\cap')$ yields $\alpha: \mathcal{A} \mid \mathcal{L}(\alpha) = \ell\cap L(\err\sigma, d)$, which can be minimized via Brzozowski's algorithm then decoded:

\begin{figure}[H]
\centering
\includegraphics[width=0.9\textwidth]{../popl2025/exampleDFA.pdf}
\caption{$L(\scriptsize{\texttt{NAME = NAME . NAME ( NUM : , NUM : )}}, 2) \cap \ell_\textsc{Python}$}
\label{fig:exampleDFA}
\end{figure}
\end{frame}

\begin{frame}[fragile]{Decoding the DFA in order of normalized log likelihood}
\vspace{-0.3cm}
\begin{algorithm}[H]
\caption{Steerable DFA walk}
\label{alg:adaptive}
\begin{algorithmic}[1]
\Require $\mathcal{A} = \langle Q, \Sigma, \delta, I, F\rangle$ DFA, $P_\theta: \Sigma^d \rightarrow \mathbb{R}$ Markov chain

\State $\mathcal{T} \gets \varnothing \text{ total trajectories}, \mathcal{P} \gets \big[\langle \varepsilon, i, 0\rangle \mid i \in I\big] \text{ partial trajectories}$
%\State $[\mathcal{T, P}]\texttt{.comparator} = \lambda\langle\sigma, q, \gamma\rangle.(\frac{\gamma}{|\sigma|})$
\Repeat
\State \textbf{let }$\langle \sigma, q, \gamma \rangle = \texttt{head}(\mathcal{P})$ \textbf{in}
%\State $\mathcal{P} \gets \texttt{tail}(\mathcal{P})$
% For loop:
\State \phantom{\textbf{let }}$\mathbf{T} = \big\{\langle s\sigma, q', \gamma - \log P_\theta(s \mid \sigma_{1..d-1}) \rangle\mid (q\overset{s}{\rightarrow}q') \in \delta\big\}$
\For{$\langle \sigma, q, \gamma \rangle = T \in \mathbf{T}$}
\If {$\exists s: \Sigma, q': Q \mid (q\overset{s}{\rightarrow} q')\in\delta$}
\State $\mathcal{P} \gets \texttt{tail}(\mathcal{P}) \oplus T$ \Comment{Add partial trajectory to PQ.}
\EndIf
\If {$q \in F$}
\State $\mathcal{T} \gets \mathcal{T} \oplus T$ \Comment{Accepting state reached, add to queue.}
\EndIf
\EndFor
\Until{interrupted or $\mathcal{P}=\varnothing$.}
\State \Return $\big[\sigma_{|\sigma|..1} \mid \langle \sigma, q, \gamma \rangle = T \in \mathcal{T}\big]$ \Comment{Return in sorted order}
\end{algorithmic}
\end{algorithm}
\end{frame}

%\begin{frame}[fragile]{A birds eye view of the algorithm}
%  \begin{figure}[H]
%    \adjustbox{scale=0.75,center}{%
%      \[\begin{tikzcd}[row sep=large, col sep=huge]
%          \text{String} && \text{Grammar} && \err{\text{String}} \\
%          [-20pt] \highlight{\Sigma}^{n-1} \arrow[dr] & \text{\underline{Parsing}} & \mathcal{G}_\varepsilon \arrow[u, dashed, no head, color=gray] \arrow[dl]\arrow[ddl, bend left] \arrow[ddd, dashed, no head, color=gray] \arrow[dr]\arrow[ddr, bend right] & \text{\underline{Repair}} & \err{\Sigma^{n-1}} \arrow[dl] \\
%          & V^{n-1} \arrow[d] & & (V_\varepsilon \cup \{\texttt{\_}\})^{n+q} \arrow[d, shift left]\arrow[d, shift right] & \\
%          & \mathbb{Z}_2^{n\times n \times |V|} \arrow[d, shift left] \arrow[d, shift right] & & \mathbb{Z}^{(n+q)^2 \times |V_\varepsilon|}_2 \rightarrow \mathbb{Z}_2^{|V_{\varepsilon}|} \arrow[d, shift left]\arrow[d, shift right] & \\
%          & CST & \phantom{.} & \left\langle (\highlight{\Sigma} \setminus \{\varepsilon\})^*, CST \right\rangle &
%      \end{tikzcd}\]
%    }
%  \end{figure}
%  Our algorithm produces set of concrete syntax trees (CSTs) for a given valid string. Otherwise, if the string contains an error, the algorithm generates a set of admissible corrections, alongside their CSTs.
%\end{frame}

\section{Error Correction}\label{sec:error-correction}

%\begin{frame}[fragile]{Sampling without replacement}
%Let $A$ be a complicated space we do not know how to sample from.
%\begin{figure}
%\centering
%\begin{tikzpicture}[ele/.style={fill=black,circle,minimum width=.8pt,inner sep=1pt},every fit/.style={ellipse,draw,minimum width=5pt,inner sep=10pt}]
%\node[ele,label=left:$a$] (a1) at (0,3) {};
%\node[ele,label=left:$b$] (a2) at (0,2) {};
%\node[ele,label=left:$c$] (a3) at (0,1) {};
%%\node[ele,label=left:$d$] (a4) at (0,1) {};
%
%\node[ele,,label=right:$1$] (b1) at (4,4) {};
%\node[ele,,label=right:$2$] (b2) at (4,3) {};
%\node[ele,,label=right:$3$] (b3) at (4,2) {};
%\node[ele,,label=right:$4$] (b4) at (4,1) {};
%
%\node[draw,fit= (a1) (a2) (a3),minimum width=2.5cm,label=below:$A$] {} ;
%\node[draw,fit= (b1) (b2) (b3) (b4),minimum width=2.5cm,label=below:$\mathbb{N}$] {} ;
%\draw[->,shorten <=2pt,shorten >=2pt] (a1) -- (b4);
%\draw[->,shorten <=2pt,shorten >=2] (a2) -- (b2);
%\draw[->,shorten <=2pt,shorten >=2] (a3) -- (b1);
%%\draw[->,shorten <=2pt,shorten >=2] (a4) -- (b3);
%\end{tikzpicture}
%\end{figure}
%We want a permutation mapping $f: A \rightarrow \mathbb{N} \mid \forall a \in A \exists i \in \mathbb{N}.f^{-1}(i) = a$. Then we can just sample $i \sim \mathbb{N}$ without replacement and apply $f^{-1}(i)$.
%\end{frame}

%\begin{frame}[fragile]{Error Correction: Levenshtein q-Balls}
%  Now that we have a reliable method to fix \textit{localized} errors, $S: \mathcal{G} \times (\Sigma\cup\{\varepsilon, \texttt{\texttt{\_}}\})^n \rightarrow \{\Sigma^n\}\subseteq \mathcal{L}_\mathcal{G}$, given some unparseable string, i.e., $\err{\sigma_1\ldots\:\sigma_n}: \highlight{\Sigma}^n \cap\mathcal{L}_\mathcal{G}^\complement$, where should we put holes to obtain a parseable $\sigma' \in \mathcal{L}_\mathcal{G}$? One way to do so is by sampling repairs, $\bm{\sigma}\sim\Sigma^{n\pm q}\cap\Delta_{q}(\err{\sigma})$ from the Levenshtein q-ball centered on $\err{\sigma}$, i.e., the space of all admissible edits with Levenshtein distance $\leq q$ (this is loosely analogous to a finite difference approximation). To admit variable-length edits, we first add an $\varepsilon^+$-production to each unit production:\vspace{5pt}
%
%  \begin{prooftree}
%    \AxiomC{$\mathcal{G} \vdash \varepsilon \in \Sigma$}
%    \RightLabel{$\varepsilon\textsc{-dup}$}
%    \UnaryInfC{$\mathcal{G} \vdash (\varepsilon^+ \rightarrow \varepsilon \mid \varepsilon^+\:\varepsilon^+) \in P$}
%  \end{prooftree}
%
%  \begin{prooftree}
%    \AxiomC{$\mathcal{G} \vdash (A \rightarrow B) \in P$}
%    \RightLabel{$\varepsilon^+\textsc{-int}$}
%    \UnaryInfC{$\mathcal{G} \vdash (A \rightarrow B\:\varepsilon^+ \mid \varepsilon^+\:B \mid B) \in P$}
%  \end{prooftree}
%\end{frame}
%
%\begin{frame}[fragile]{Error Correction: d-Subset Sampling}
%  \noindent Next, suppose $U: \mathbb{Z}_2^{m\times m}$ is a matrix whose structure is shown in Eq.~\ref{eq:lfsr}, wherein $C$ is a primitive polynomial over $\mathbb{Z}_2^m$ with coefficients $C_{1\ldots m}$ and semiring operators $\oplus := \veebar, \otimes := \land$:\vspace{-5pt}
%
%  \begin{align}
%    U^tV = \begin{pNiceMatrix}[nullify-dots,xdots/line-style=loosely dotted]
%             C_1    & \cdd  &       &       & C_m \\
%             \top   & \circ & \cdd  &       & \circ \\
%             \circ  & \ddd  & \ddd  &       & \vdd \\
%             \vdd   & \ddd  &       &       & \\
%             \circ  & \cdd  & \circ & \top  & \circ
%    \end{pNiceMatrix}^t
%    \begin{pNiceMatrix}[nullify-dots,xdots/line-style=loosely dotted]
%      V_1 \\
%      \vdd\\
%      \\
%      \\
%      V_m
%    \end{pNiceMatrix}\label{eq:lfsr}
%  \end{align}
%
%  \noindent Since $C$ is primitive, the sequence $\mathbf{S} = (U^{0 \ldots 2^m-1}V)$ must have \textit{full periodicity}, i.e., for all $i, j \in[0, 2^m)$, ${\mathbf{S}_i = \mathbf{S}_j \Rightarrow i = j}$. To uniformly sample $\bm\sigma$ without replacement, we first form an injection $\mathbb{Z}_2^m\rightharpoonup\stirlingii{n}{d}\times\Sigma_\varepsilon^{2d}$ using a combinatorial number system, cycle over $\mathbf{S}$, then discard samples which have no witness in $\stirlingii{n}{d}\times\Sigma_\varepsilon^{2d}$. This method requires $\widetilde{\mathcal O}(1)$ per sample and $\widetilde{\mathcal O}\left({n \choose d}|\Sigma + 1|^{2d}\right)$ to exhaustively search $\stirlingii{n}{d}\times\Sigma_\varepsilon^{2d}$.
%\end{frame}
%
%\begin{frame}[fragile]{Error Correction: Sketch Templates}
%  Finally, to sample $\bm{\sigma}\sim\Delta_{q}(\err{\sigma})$, we enumerate a series of sketch templates $H(\sigma, i) = \sigma_{1\ldots i-1}\:\text{\texttt{\_} \texttt{\_}}\:\sigma_{i+1\ldots n}$ for each $i \in \cdot \in \stirlingii{n}{d}$ and $d \in 1\ldots q$, then solve for $\mathcal{M}_{\bm\sigma}^*$. If $S \in \Lambda^*_{\bm\sigma}?$ has a solution, each edit in each $\sigma' \in \bm\sigma$ will match exactly one of the following seven edit patterns:\vspace{-10pt}
%
%  \begin{align*}
%    \text{Deletion}&=\begin{cases}
%                       \,\ldots\sigma_{i-1}\:\text{\hlred{$\gamma_1$}\:\hlred{$\gamma_2$}}\:\sigma_{i+1}\ldots\hspace{0.2cm}\gamma_{1, 2} = \varepsilon\label{eq:del}
%    \end{cases}\\
%    \text{Substitution}&=\begin{cases}
%         \ldots\sigma_{i-1}\:\text{\hlorange{$\gamma_1$}\:\hlred{$\gamma_2$}}\:\sigma_{i+1}\ldots\hspace{0.2cm}\gamma_1 \neq \varepsilon \land \gamma_2 = \varepsilon\\
%         \ldots\sigma_{i-1}\:\text{\hlred{$\gamma_1$}\:\hlorange{$\gamma_2$}}\:\sigma_{i+1}\ldots\hspace{0.2cm}\gamma_1 = \varepsilon \land \gamma_2 \neq \varepsilon\\
%         \ldots\sigma_{i-1}\:\text{\hlorange{$\gamma_1$}\:\hlorange{$\gamma_2$}}\:\sigma_{i+1}\ldots\hspace{0.2cm}\{\gamma_1, \gamma_2\}\cap\{\varepsilon, \sigma_i\} = \varnothing
%    \end{cases}\\
%    \text{Insertion}&=\begin{cases}
%        \ldots\sigma_{i-1}\:\text{\hlgreen{$\gamma_1$}\:\hlorange{$\gamma_2$}}\:\sigma_{i+1}\ldots\hspace{0.2cm}\gamma_1 = \sigma_i \land \gamma_2 \notin \{\varepsilon,  \sigma_i\}\label{eq:ins2}\\
%        \ldots\sigma_{i-1}\:\text{\hlorange{$\gamma_1$}\:\hlgreen{$\gamma_2$}}\:\sigma_{i+1}\ldots\hspace{0.2cm}\gamma_1 \notin \{\varepsilon, \sigma_i\} \land \gamma_2 = \sigma_i\label{eq:ins1}\\
%        \ldots\sigma_{i-1}\:\text{\hlgreen{$\gamma_1$}\:\hlgreen{$\gamma_2$}}\:\sigma_{i+1}\ldots\hspace{0.2cm}\gamma_{1,2} = \sigma_i\label{eq:copy}
%    \end{cases}
%  \end{align*}
%\end{frame}



%\begin{frame}[fragile]{Level I: Known Error Locations}
%  \begin{figure}[H]
%    \hspace{-0.25cm}
%    \begin{tikzpicture}
%      \begin{axis}[
%        width=8.3cm,
%        height=6cm,
%        title={\hspace{-1cm}\textbf{Latency with known locations}},
%        ybar,
%        bar width=6pt,
%        xlabel={Number of holes},
%        ylabel={ms to synthesize 10 repairs},
%        xtick=data,
%        axis x line*=bottom,
%        axis y line*=left,
%        ytick pos=left,
%        xticklabels from table={\loctimings}{holes},
%        ymajorgrids,
%        legend pos=north west,
%        legend columns=2,
%        error bars/y dir=both,
%        error bars/y explicit
%      ]
%        \addplot table [x expr=\coordindex, y=d1, y error=d1err]{\loctimings};
%        \addplot table [x expr=\coordindex, y=d2, y error=d2err]{\loctimings};
%        \addplot table [x expr=\coordindex, y=d3, y error=d3err]{\loctimings};
%        \addplot table [x expr=\coordindex, y=d4, y error=d4err]{\loctimings};
%        \legend{Dyck-1, Dyck-2, Dyck-3, Dyck-4}
%      \end{axis}
%    \end{tikzpicture}
%  \end{figure}
%\end{frame}
%
%\begin{frame}[fragile]{Level II: Unknown Locations, Fixed Error Count}
%  \begin{figure}[H]
%    \hspace{-0.25cm}
%    \begin{tikzpicture}
%      \begin{axis}[
%        width=8.3cm,
%        height=6cm,
%        title={\hspace{-1cm}\textbf{Latency with unknown locations}},
%        ybar,
%        bar width=20pt,
%        xlabel={Number of errors},
%        ylabel={ms to synthesize 1 repair},
%        xtick=data,
%        axis x line*=bottom,
%        axis y line*=left,
%        enlarge x limits={abs=0.5},
%        ymode=log,
%        ytick pos=left,
%        xticklabels from table={\unloctimings}{errors},
%        ymajorgrids,
%        legend pos=north west,
%        error bars/y dir=both,
%        error bars/y explicit
%      ]
%        \addplot table [x expr=\coordindex, y=d1]{\unloctimings};
%        \addplot table [x expr=\coordindex, y=d2]{\unloctimings};
%        \addplot table [x expr=\coordindex, y=d3]{\unloctimings};
%        \legend{Dyck-1, Dyck-2, Dyck-3}
%      \end{axis}
%    \end{tikzpicture}
%  \end{figure}
%\end{frame}
%
%\begin{frame}[fragile]{Level III: MiniGithub, Unknown Count, Synthetic Errors}
%  \begin{figure}[H]
%    \centering{\textbf{Synthetic error correction accuracy}}\\
%    \vspace{0.25cm}
%%\adjustbox{scale=0.48,center}{%
%    \begin{tikzpicture}
%      \begin{axis}
%        [
%        width=8.3cm,
%        height=6.6cm,
%%    title={\hspace{-1cm}\textbf{Java Brackets}},
%        ybar,
%        bar width=10pt,
%        xlabel={$|\Sigma^*|$},
%        ylabel={Parser acceptance},
%        xtick=data,
%        axis x line*=bottom,
%        axis y line*=left,
%        enlarge x limits={abs=0.5},
%        ytick pos=left,
%        xticklabels from table={\syntheticerrors}{len},
%        ymajorgrids,
%        legend pos=north east,
%        legend columns=3,
%        error bars/y dir=both,
%        error bars/y explicit
%        ]
%        \addplot table [x expr=\coordindex, y=10s]{\syntheticerrors};
%        \addplot table [x expr=\coordindex, y=30s]{\syntheticerrors};
%        \addplot table [x expr=\coordindex, y=60s]{\syntheticerrors};
%        \legend{10s, 30s, 60s}
%      \end{axis}
%    \end{tikzpicture}
%  \end{figure}
%\end{frame}
%
%\begin{frame}[fragile]{Level IV: BIFI Dataset, Unknown Count, Real Errors}
%  \begin{figure}[H]
%    \hspace{-0.25cm}
%    \begin{tikzpicture}
%      \begin{axis}
%        [
%        width=8.3cm,
%        height=6.6cm,
%        title={\hspace{-1cm}\textbf{Organic error correction accuracy}},
%        ybar,
%        bar width=10pt,
%        xlabel={$|\Sigma^*|$},
%        ylabel={Parser acceptance},
%        xtick=data,
%        axis x line*=bottom,
%        axis y line*=left,
%        enlarge x limits={abs=0.5},
%        ytick pos=left,
%        xticklabels from table={\naturalerrors}{len},
%        ymajorgrids,
%        y tick label style={/pgf/number format/.cd,%
%        scaled y ticks = false,
%        set thousands separator={},
%        fixed},
%        legend pos=north east,
%        legend columns=3,
%        error bars/y dir=both,
%        error bars/y explicit
%        ]
%        \addplot table [x expr=\coordindex, y=10s]{\naturalerrors};
%        \addplot table [x expr=\coordindex, y=30s]{\naturalerrors};
%        \addplot table [x expr=\coordindex, y=60s]{\naturalerrors};
%        \legend{10s, 30s, 60s}
%      \end{axis}
%    \end{tikzpicture}
%  \end{figure}
%\end{frame}

\begin{frame}[fragile]{Characteristics of the repair dataset}
\begin{figure}[h!]
\begin{tikzpicture}[scale=0.52]
\begin{axis}[
ybar,
bar width=5pt,
xlabel={Length $(|\err\sigma|)$},
ylabel={Frequency},
title={Cumulative length distribution},
axis x line*=bottom,
axis y line*=left,
ymin=0,
ymax=65,
xtick=data,
xticklabels={,,<30,,,<60,,,<90,},
ymajorgrids=true,
grid style=dashed,
width=0.45\textwidth,
height=0.3\textwidth
]

\addplot[fill=black!30] table {
X Y
1 7.60
2 14.52
3 22.01
4 30.54
5 37.82
6 44.30
7 49.68
8 55.21
9 59.75
10 63.59
};
\draw[red, dashed] (axis cs:8.5,0) -- (axis cs:8.5,65);
\end{axis}
\end{tikzpicture}
\begin{tikzpicture}[scale=0.52]
\begin{axis}[
ybar,
bar width=5pt,
title={Human repair distance},
xlabel={Edits, $\Delta(\err\sigma, \sigma')$},
ylabel={Frequency},
axis x line*=bottom,
axis y line*=left,
xtick=data,
ymajorgrids=true,
grid style=dashed,
xticklabels={,\leq 2,,\leq 4,,\leq 6,,\leq 8,,\leq 10},
ytick={0, 20, 40, 60, 80, 100},
ymin=0,
width=0.45\textwidth,
height=0.3\textwidth
]
\addplot[fill=black!30] table {
X Y
1  31.48
2  47.52
3  54.89
4  60.44
5  63.88
6  66.38
7  68.02
8  70.04
9  71.49
10 72.22
};
\draw[red, dashed] (axis cs:4.5,0) -- (axis cs:4.5,80);
\end{axis}
\end{tikzpicture}
\begin{tikzpicture}[scale=0.52]
\begin{axis}[
ybar,
bar width=5pt,
xlabel={Beginning $\longleftrightarrow$ End},
ylabel={Frequency},
title={Normalized edit locations},
axis x line*=bottom,
axis y line*=left,
ymin=0,
ymax=35,
xtick=data,
xticklabels={0\%,,,,,,,,,100\%},
ymajorgrids=true,
grid style=dashed,
width=0.45\textwidth,
height=0.3\textwidth
]

\addplot[fill=black!30] table {
X Y
10 11.6539
20 5.7252
30 6.2087
40 5.9542
50 5.5980
60 7.9389
70 7.0738
80 6.9466
90 12.4173
100 30.4835
};
\end{axis}
\end{tikzpicture}
%    \begin{tikzpicture}
%      \begin{axis}[
%        ybar,
%        bar width=5pt,
%        title={Intra-patch edit distance},
%        xlabel={Caret distance},
%        ylabel={Frequency},
%        axis x line*=bottom,
%        axis y line*=left,
%        xtick=data,
%        ymajorgrids=true,
%        grid style=dashed,
%        xticklabels={1,2,3,4,5,6,7,8,9,10+},
%        width=0.45\textwidth,
%        height=0.3\textwidth
%      ]
%
%        \addplot table {
%          X Y
%          1 40.66
%          2 15.00
%          3 5.80
%          4 4.86
%          5 4.26
%          6 2.98
%          7 2.05
%          8 2.73
%          9 1.62
%          10 13.64
%        };
%      \end{axis}
%    \end{tikzpicture}
\begin{tikzpicture}[scale=0.52]
\begin{axis}[
legend cell align={left},
legend style={fill opacity=1, draw opacity=1, text opacity=1, draw=lightgray204, legend columns=-1, legend pos=north east, legend style={at={(0.5,1.8)},anchor=north}},
xlabel={$|\err\sigma|$},
ylabel={Stable region},
title={Stability profile},
ybar,
axis lines*=left,
xtick={0, 10, 20, 30, 40, 50, 60, 70},
ytick={0, 0.2, 0.4, 0.6, 0.8, 1.0},
xticklabels={, {[}10{,}20{)}, , {[}30{,}40{)}, , {[}50{,}60{)}, , {[}70{,}80{)}},
yticklabels={0, 0.2, 0.4, 0.6, 0.8, 1.0},
x tick label style={font=\scriptsize},
ymax=1.0,
ymin=0.0,
bar width=3pt,
grid style=dashed,
ymajorgrids=true,
width=0.45\textwidth,
height=0.3\textwidth
]
\addlegendimage{empty legend}
\addlegendentry{$\Delta(\err\sigma, \sigma')=$}
\addlegendimage{ybar,ybar legend,draw=none,green,fill=green!50}
\addlegendentry{1,}
\addlegendimage{ybar,ybar legend,draw=none,blue,fill=blue!50}
\addlegendentry{2,}
\addlegendimage{ybar,ybar legend,draw=none,orange,fill=orange!50}
\addlegendentry{3}
\addplot[green, fill=green!50] coordinates {(0, 0.80) (10, 0.91) (20, 0.96) (30, 0.97) (40, 0.99) (50, 0.99) (60, 0.99) (70, 0.99)};
\addplot[blue, fill=blue!50] coordinates {(0, 0.35) (10, 0.59) (20, 0.69) (30, 0.73) (40, 0.79) (50, 0.82) (60, 0.84) (70, 0.86)};
\addplot[orange, fill=orange!50] coordinates {(0, 0.23) (10, 0.45) (20, 0.58) (30, 0.66) (40, 0.70) (50, 0.77) (60, 0.78) (70, 0.86)};
\end{axis}
\end{tikzpicture}
\caption{Repair statistics across the StackOverflow dataset, of which Tidyparse can handle about half in under $\sim$30s and $\sim$150 GB. Larger repairs and edit distances are possible, albeit requiring additional time and memory. The stability profile measures the average fraction of all edit locations that were never altered by any repair in the $L\big(\err\sigma, \Delta(\err\sigma, \sigma')\big)$-ball across repairs of varying length and distance.}\label{fig:patch_stats}
\end{figure}
\end{frame}

\begin{frame}[fragile]{Ranked repair}
We train on lexical n-grams using the standard MLE for Markov chains. To score the repairs, we use the conventional length-normalized NLL:

\begin{equation}
\text{NLL}(\sigma) = -\frac{1}{|\sigma|}\sum_{i=1}^{|\sigma|}\log P_\theta(\sigma_i \mid \sigma_{i-1}\ldots\sigma_{i-n})
\end{equation}

For each retrieved set $\hat{A} \subseteq A$ drawn before a predetermined timeout and each $\sigma \in \hat{A}$, we score the repair and return $\hat{A}$ in ascending order.

To evaluate the quality of our ranking, we use the Precision@k statistic. Specifically, given a repair model, $R: \Sigma^* \rightarrow 2^{\Sigma^*}$ and a parallel corpus, $\mathcal{D}_{\text{test}}$, of errors ($\sigma^\dagger$) and repairs ($\sigma'$), we define Precision@k as:

\begin{equation}
\text{Precision@k}(R) = \frac{1}{|\mathcal{D}_{\text{test}}|}\sum_{\langle\sigma^\dagger, \sigma'\rangle \in \mathcal{D}_{\text{test}}} \mathds{1}\big[\sigma' \in \argmax_{\bm{\sigma} \subset R(\sigma^\dagger), |\bm{\sigma}| \leq k}\sum_{\sigma \in \bm{\sigma}}\text{NLL}(\sigma)\big]
\end{equation}
\end{frame}

\begin{frame}[fragile]{Precision and latency comparison}
\begin{figure}[h!]
\resizebox{.24\textwidth}{!}{\begin{tikzpicture}
  \begin{axis}[
    xlabel={$|\sigma|$},
    ylabel={Precision@1},
    title={Tidyparse Repair Precision},
    ybar,
    axis lines*=left,
    xtick={0, 10, 20, 30, 40, 50, 60, 70},
    ytick={0, 0.1, 0.2, 0.3, 0.4, 0.5, 0.6, 0.7, 0.8, 0.9, 1.0},
    ymax=1.0,
    ymin=0.0,
    bar width=4pt,
  ]

  \addplot[green, fill=green] coordinates {(0, 1.0) (10, 1.0) (20, 1.0) (30, 1.0) (40, 1.0) (50, 1.0) (60, 1.0) (70, 1.0)};
  \addplot[blue, fill=blue] coordinates {(0, 0.3) (10, 0.286) (20, 0.205) (30, 0.433) (40, 0.256) (50, 0.296) (60, 0.236) (70, 0.315)};
  \addplot[orange, fill=orange] coordinates {(0, 0.46875) (10, 0.321) (20, 0.366) (30, 0.24) (40, 0.407) (50, 0.454) (60, 0.574) (70, 0.526)};

%  \legend{Δ=1,Δ=2,Δ=3}
  \end{axis}
\end{tikzpicture}}
\resizebox{.24\textwidth}{!}{\begin{tikzpicture}
  \begin{axis}[
  xlabel={Snippet length, $|\sigma|$},
  ylabel={Precision@20k},
  title={\textbf{BIFI Repair Precision@20k}},
  legend cell align={left},
  legend style={fill opacity=0.8, draw opacity=1, text opacity=1, draw=lightgray204, legend columns=-1, legend pos=north east},
  ybar,
  axis lines*=left,
  xtick={0, 10, 20, 30, 40, 50, 60, 70},
  ytick={0, 0.1, 0.2, 0.3, 0.4, 0.5, 0.6, 0.7, 0.8, 0.9, 1.0},
  xticklabels={{(}0{,}10{)}, {[}10{,}20{)}, {[}20{,}30{)}, {[}30{,}40{)}, {[}40{,}50{)}, {[}50{,}60{)}, {[}60{,}70{)}, {[}70{,}80{)}},
  x tick label style={font=\scriptsize},
  ymax=1.0,
  ymin=0.0,
  bar width=4pt,
  ]

  \addlegendimage{empty legend}
  \addlegendentry{$\Delta(\err\sigma, \sigma')=$}
  \addlegendimage{ybar,ybar legend,draw=none,green,fill=green!50}
  \addlegendentry{1,}
  \addlegendimage{ybar,ybar legend,draw=none,blue,fill=blue!50}
  \addlegendentry{2,}
  \addlegendimage{ybar,ybar legend,draw=none,orange,fill=orange!50}
  \addlegendentry{3}

  \addplot[green, fill=green!50] coordinates   {(0, 0.65) (10, 0.67) (20, 0.71) (30, 0.63) (40, 0.60) (50, 0.62) (60, 0.59) (70, 0.64)};
  \addplot[blue, fill=blue!50] coordinates     {(0, 0.52) (10, 0.41) (20, 0.35) (30, 0.31) (40, 0.27) (50, 0.27) (60, 0.21) (70, 0.22)};
  \addplot[orange, fill=orange!50] coordinates {(0, 0.25) (10, 0.08) (20, 0.08) (30, 0.17) (40, 0.11) (50, 0.17) (60, 0.08) (70, 0.08)};

  \end{axis}
\end{tikzpicture}
}
\resizebox{.24\textwidth}{!}{\begin{tikzpicture}
  \begin{axis}[
    xlabel={Snippet length, $|\sigma|$},
    ylabel={Precision@1},
    title={\textbf{Seq2Parse Repair Precision@1}},
    ybar,
    axis lines*=left,
    xtick={0, 10, 20, 30, 40, 50, 60, 70},
    ytick={0, 0.1, 0.2, 0.3, 0.4, 0.5, 0.6, 0.7, 0.8, 0.9, 1.0},
    xticklabels={{(}0{,}10{)}, {[}10{,}20{)}, {[}20{,}30{)}, {[}30{,}40{)}, {[}40{,}50{)}, {[}50{,}60{)}, {[}60{,}70{)}, {[}70{,}80{)}},
    x tick label style={font=\scriptsize},
    ymax=1.0,
    ymin=0.0,
    bar width=4pt,
  ]

  \addplot[green, fill=green!50] coordinates {(0, 0.352631) (10, 0.413115) (20, 0.400502) (30, 0.378440) (40, 0.308869) (50, 0.287755) (60, 0.268817) (70, 0.210526)};
  \addplot[blue, fill=blue!50] coordinates {(0, 0.122529) (10, 0.126453) (20, 0.144192) (30, 0.118483) (40, 0.108007) (50, 0.106849) (60, 0.097403) (70, 0.122047)};
  \addplot[orange, fill=orange!50] coordinates {(0, 0.03125) (10, 0.070922) (20, 0.077348) (30, 0.087629) (40, 0.094675) (50, 0.02) (60, 0.066038) (70, 0.063291)};

  \end{axis}
\end{tikzpicture}}
\resizebox{.24\textwidth}{!}{\begin{tikzpicture}
  \begin{axis}[
    legend cell align={left},
    legend style={fill opacity=0.8, draw opacity=1, text opacity=1, draw=lightgray204, legend columns=-1, legend pos=north east},
    xlabel={Snippet length, $|\sigma|$},
    ylabel={Precision@1},
    title={\Large\textbf{BIFI Repair Precision@1}},
    ybar,
    axis lines*=left,
    xtick={0, 10, 20, 30, 40, 50, 60, 70},
    ytick={0, 0.1, 0.2, 0.3, 0.4, 0.5, 0.6, 0.7, 0.8, 0.9, 1.0},
    xticklabels={{(}0{,}10{)}, {[}10{,}20{)}, {[}20{,}30{)}, {[}30{,}40{)}, {[}40{,}50{)}, {[}50{,}60{)}, {[}60{,}70{)}, {[}70{,}80{)}},
    x tick label style={font=\scriptsize},
    ymax=0.6,
    ymin=0.0,
    bar width=4pt,
  ]
    \addlegendimage{empty legend}
    \addlegendentry{$\Delta(\err\sigma, \sigma')=$}
    \addlegendimage{ybar,ybar legend,draw=none,green,fill=green!50}
    \addlegendentry{1,}
    \addlegendimage{ybar,ybar legend,draw=none,blue,fill=blue!50}
    \addlegendentry{2,}
    \addlegendimage{ybar,ybar legend,draw=none,orange,fill=orange!50}
    \addlegendentry{3}

    \addplot[green, fill=green!50] coordinates {(0, 0.196013) (10, 0.326401) (20, 0.318538) (30, 0.272843) (40, 0.213894) (50, 0.206651) (60, 0.247525) (70, 0.179245)};
    \addplot[blue, fill=blue!50] coordinates {(0, 0.174603) (10, 0.176651) (20, 0.209573) (30, 0.19195) (40, 0.18851) (50, 0.176166) (60, 0.110787) (70, 0.106383)};
    \addplot[orange, fill=orange!50] coordinates {(0, 0.015873) (10, 0.021858) (20, 0.030435) (30, 0.02439) (40, 0.032922) (50, 0.045) (60, 0.027397) (70, 0.017094)};
  \end{axis}
\end{tikzpicture}}
\caption{Tidyparse, Seq2Parse and BIFI repair precision across length and edits.}
\end{figure}
\begin{figure}[h!]
%    \resizebox{.19\textwidth}{!}{% This file was created with tikzplotlib v0.10.1.
\begin{tikzpicture}

\definecolor{darkgray176}{RGB}{176,176,176}
\definecolor{darkviolet1270255}{RGB}{127,0,255}
\definecolor{deepskyblue3176236}{RGB}{3,176,236}
\definecolor{dodgerblue45123246}{RGB}{45,123,246}
\definecolor{royalblue8762253}{RGB}{87,62,253}

\begin{axis}[
tick align=outside,
tick pos=left,
axis lines=left,
title={\(\displaystyle \Delta\in[1,3]\) Repair Precision},
legend style={fill opacity=0.8, draw opacity=1, text opacity=1, legend columns=1, legend pos=north west},
x grid style={darkgray176},
xlabel={Seconds},
xmin=-0.5925, xmax=8.5925,
xtick style={color=black},
xtick={0,1,2,3,4,5,6,7,8},
xticklabels={1,2,3,4,5,6,7,8,9},
y grid style={darkgray176},
ylabel={Precision@k},
ymin=0, ymax=0.6993,
ytick style={color=black}
]
\draw[draw=none,fill=darkviolet1270255] (axis cs:-0.175,0) rectangle (axis cs:0.175,0.149);
\addlegendimage{ybar,ybar legend,draw=none,fill=darkviolet1270255}
\addlegendentry{P@All}

\draw[draw=none,fill=darkviolet1270255] (axis cs:0.825,0) rectangle (axis cs:1.175,0.303);
\draw[draw=none,fill=darkviolet1270255] (axis cs:1.825,0) rectangle (axis cs:2.175,0.412);
\draw[draw=none,fill=darkviolet1270255] (axis cs:2.825,0) rectangle (axis cs:3.175,0.501);
\draw[draw=none,fill=darkviolet1270255] (axis cs:3.825,0) rectangle (axis cs:4.175,0.569);
\draw[draw=none,fill=darkviolet1270255] (axis cs:4.825,0) rectangle (axis cs:5.175,0.605);
\draw[draw=none,fill=darkviolet1270255] (axis cs:5.825,0) rectangle (axis cs:6.175,0.63);
\draw[draw=none,fill=darkviolet1270255] (axis cs:6.825,0) rectangle (axis cs:7.175,0.645);
\draw[draw=none,fill=darkviolet1270255] (axis cs:7.825,0) rectangle (axis cs:8.175,0.666);
\draw[draw=none,fill=royalblue8762253] (axis cs:-0.175,0) rectangle (axis cs:0.175,0.128);
\addlegendimage{ybar,ybar legend,draw=none,fill=royalblue8762253}
\addlegendentry{P@10}

\draw[draw=none,fill=royalblue8762253] (axis cs:0.825,0) rectangle (axis cs:1.175,0.239);
\draw[draw=none,fill=royalblue8762253] (axis cs:1.825,0) rectangle (axis cs:2.175,0.323);
\draw[draw=none,fill=royalblue8762253] (axis cs:2.825,0) rectangle (axis cs:3.175,0.374);
\draw[draw=none,fill=royalblue8762253] (axis cs:3.825,0) rectangle (axis cs:4.175,0.421);
\draw[draw=none,fill=royalblue8762253] (axis cs:4.825,0) rectangle (axis cs:5.175,0.446);
\draw[draw=none,fill=royalblue8762253] (axis cs:5.825,0) rectangle (axis cs:6.175,0.459);
\draw[draw=none,fill=royalblue8762253] (axis cs:6.825,0) rectangle (axis cs:7.175,0.469);
\draw[draw=none,fill=royalblue8762253] (axis cs:7.825,0) rectangle (axis cs:8.175,0.477);
\draw[draw=none,fill=dodgerblue45123246] (axis cs:-0.175,0) rectangle (axis cs:0.175,0.11);
\addlegendimage{ybar,ybar legend,draw=none,fill=dodgerblue45123246}
\addlegendentry{P@5}

\draw[draw=none,fill=dodgerblue45123246] (axis cs:0.825,0) rectangle (axis cs:1.175,0.206);
\draw[draw=none,fill=dodgerblue45123246] (axis cs:1.825,0) rectangle (axis cs:2.175,0.282);
\draw[draw=none,fill=dodgerblue45123246] (axis cs:2.825,0) rectangle (axis cs:3.175,0.325);
\draw[draw=none,fill=dodgerblue45123246] (axis cs:3.825,0) rectangle (axis cs:4.175,0.369);
\draw[draw=none,fill=dodgerblue45123246] (axis cs:4.825,0) rectangle (axis cs:5.175,0.392);
\draw[draw=none,fill=dodgerblue45123246] (axis cs:5.825,0) rectangle (axis cs:6.175,0.405);
\draw[draw=none,fill=dodgerblue45123246] (axis cs:6.825,0) rectangle (axis cs:7.175,0.414);
\draw[draw=none,fill=dodgerblue45123246] (axis cs:7.825,0) rectangle (axis cs:8.175,0.422);
\draw[draw=none,fill=deepskyblue3176236] (axis cs:-0.175,0) rectangle (axis cs:0.175,0.092);
\addlegendimage{ybar,ybar legend,draw=none,fill=deepskyblue3176236}
\addlegendentry{P@1}

\draw[draw=none,fill=deepskyblue3176236] (axis cs:0.825,0) rectangle (axis cs:1.175,0.18);
\draw[draw=none,fill=deepskyblue3176236] (axis cs:1.825,0) rectangle (axis cs:2.175,0.244);
\draw[draw=none,fill=deepskyblue3176236] (axis cs:2.825,0) rectangle (axis cs:3.175,0.285);
\draw[draw=none,fill=deepskyblue3176236] (axis cs:3.825,0) rectangle (axis cs:4.175,0.324);
\draw[draw=none,fill=deepskyblue3176236] (axis cs:4.825,0) rectangle (axis cs:5.175,0.343);
\draw[draw=none,fill=deepskyblue3176236] (axis cs:5.825,0) rectangle (axis cs:6.175,0.356);
\draw[draw=none,fill=deepskyblue3176236] (axis cs:6.825,0) rectangle (axis cs:7.175,0.365);
\draw[draw=none,fill=deepskyblue3176236] (axis cs:7.825,0) rectangle (axis cs:8.175,0.37);
\addplot [red, thick] coordinates {(-0.8925,0.29) (14.8925,0.29)};
\end{axis}

\end{tikzpicture}
}
\resizebox{.24\textwidth}{!}{% This file was created with tikzplotlib v0.10.1.
\begin{tikzpicture}

\definecolor{darkgray176}{RGB}{176,176,176}
\definecolor{darkviolet1270255}{RGB}{127,0,255}
\definecolor{deepskyblue3176236}{RGB}{3,176,236}
\definecolor{dodgerblue45123246}{RGB}{45,123,246}
\definecolor{royalblue8762253}{RGB}{87,62,253}

\begin{axis}[
tick align=outside,
tick pos=left,
axis lines*=left,
title={{\Large\(\displaystyle \Delta=1\) Repair Precision}},
legend style={fill opacity=0.8, draw opacity=1, text opacity=1, legend columns=1, legend pos=north west},
x grid style={darkgray176},
xlabel={Seconds},
xmin=-0.5925, xmax=8.5925,
xtick style={color=black},
xtick={0,1,2,3,4,5,6,7,8},
xticklabels={1,2,3,4,5,6,7,8,9},
y grid style={darkgray176},
ylabel={Precision@k},
ymin=0, ymax=1.0311,
ytick style={color=black}
]
\draw[draw=none,fill=darkviolet1270255] (axis cs:-0.175,0) rectangle (axis cs:0.175,0.546);
\addlegendimage{ybar,ybar legend,draw=none,fill=darkviolet1270255}
\addlegendentry{P@All}

\draw[draw=none,fill=darkviolet1270255] (axis cs:0.825,0) rectangle (axis cs:1.175,0.752);
\draw[draw=none,fill=darkviolet1270255] (axis cs:1.825,0) rectangle (axis cs:2.175,0.88);
\draw[draw=none,fill=darkviolet1270255] (axis cs:2.825,0) rectangle (axis cs:3.175,0.908);
\draw[draw=none,fill=darkviolet1270255] (axis cs:3.825,0) rectangle (axis cs:4.175,0.933);
\draw[draw=none,fill=darkviolet1270255] (axis cs:4.825,0) rectangle (axis cs:5.175,0.954);
\draw[draw=none,fill=darkviolet1270255] (axis cs:5.825,0) rectangle (axis cs:6.175,0.966);
\draw[draw=none,fill=darkviolet1270255] (axis cs:6.825,0) rectangle (axis cs:7.175,0.975);
\draw[draw=none,fill=darkviolet1270255] (axis cs:7.825,0) rectangle (axis cs:8.175,0.982);
\draw[draw=none,fill=royalblue8762253] (axis cs:-0.175,0) rectangle (axis cs:0.175,0.546);
\addlegendimage{ybar,ybar legend,draw=none,fill=royalblue8762253}
\addlegendentry{P@10}

\draw[draw=none,fill=royalblue8762253] (axis cs:0.825,0) rectangle (axis cs:1.175,0.752);
\draw[draw=none,fill=royalblue8762253] (axis cs:1.825,0) rectangle (axis cs:2.175,0.88);
\draw[draw=none,fill=royalblue8762253] (axis cs:2.825,0) rectangle (axis cs:3.175,0.908);
\draw[draw=none,fill=royalblue8762253] (axis cs:3.825,0) rectangle (axis cs:4.175,0.933);
\draw[draw=none,fill=royalblue8762253] (axis cs:4.825,0) rectangle (axis cs:5.175,0.954);
\draw[draw=none,fill=royalblue8762253] (axis cs:5.825,0) rectangle (axis cs:6.175,0.966);
\draw[draw=none,fill=royalblue8762253] (axis cs:6.825,0) rectangle (axis cs:7.175,0.975);
\draw[draw=none,fill=royalblue8762253] (axis cs:7.825,0) rectangle (axis cs:8.175,0.982);
\draw[draw=none,fill=dodgerblue45123246] (axis cs:-0.175,0) rectangle (axis cs:0.175,0.546);
\addlegendimage{ybar,ybar legend,draw=none,fill=dodgerblue45123246}
\addlegendentry{P@5}

\draw[draw=none,fill=dodgerblue45123246] (axis cs:0.825,0) rectangle (axis cs:1.175,0.752);
\draw[draw=none,fill=dodgerblue45123246] (axis cs:1.825,0) rectangle (axis cs:2.175,0.88);
\draw[draw=none,fill=dodgerblue45123246] (axis cs:2.825,0) rectangle (axis cs:3.175,0.908);
\draw[draw=none,fill=dodgerblue45123246] (axis cs:3.825,0) rectangle (axis cs:4.175,0.933);
\draw[draw=none,fill=dodgerblue45123246] (axis cs:4.825,0) rectangle (axis cs:5.175,0.954);
\draw[draw=none,fill=dodgerblue45123246] (axis cs:5.825,0) rectangle (axis cs:6.175,0.966);
\draw[draw=none,fill=dodgerblue45123246] (axis cs:6.825,0) rectangle (axis cs:7.175,0.975);
\draw[draw=none,fill=dodgerblue45123246] (axis cs:7.825,0) rectangle (axis cs:8.175,0.982);
\draw[draw=none,fill=deepskyblue3176236] (axis cs:-0.175,0) rectangle (axis cs:0.175,0.546);
\addlegendimage{ybar,ybar legend,draw=none,fill=deepskyblue3176236}
\addlegendentry{P@1}

\draw[draw=none,fill=deepskyblue3176236] (axis cs:0.825,0) rectangle (axis cs:1.175,0.752);
\draw[draw=none,fill=deepskyblue3176236] (axis cs:1.825,0) rectangle (axis cs:2.175,0.88);
\draw[draw=none,fill=deepskyblue3176236] (axis cs:2.825,0) rectangle (axis cs:3.175,0.908);
\draw[draw=none,fill=deepskyblue3176236] (axis cs:3.825,0) rectangle (axis cs:4.175,0.933);
\draw[draw=none,fill=deepskyblue3176236] (axis cs:4.825,0) rectangle (axis cs:5.175,0.954);
\draw[draw=none,fill=deepskyblue3176236] (axis cs:5.825,0) rectangle (axis cs:6.175,0.966);
\draw[draw=none,fill=deepskyblue3176236] (axis cs:6.825,0) rectangle (axis cs:7.175,0.975);
\draw[draw=none,fill=deepskyblue3176236] (axis cs:7.825,0) rectangle (axis cs:8.175,0.982);
\addplot [red, thick] coordinates {(-0.8925,0.39) (14.8925,0.39)};
\end{axis}

\end{tikzpicture}
}
\resizebox{.24\textwidth}{!}{% This file was created with tikzplotlib v0.10.1.
\begin{tikzpicture}

\definecolor{darkgray176}{RGB}{176,176,176}
\definecolor{darkviolet1270255}{RGB}{127,0,255}
\definecolor{deepskyblue3176236}{RGB}{3,176,236}
\definecolor{dodgerblue45123246}{RGB}{45,123,246}
\definecolor{royalblue8762253}{RGB}{87,62,253}

\begin{axis}[
tick align=outside,
tick pos=left,
title={\(\displaystyle \Delta=2\) Repair Precision},
legend style={fill opacity=0.8, draw opacity=1, text opacity=1, legend columns=1, legend pos=north west},
x grid style={darkgray176},
xlabel={Seconds},
xmin=-0.6425, xmax=9.6425,
xtick style={color=black},
xtick={0,1,2,3,4,5,6,7,8,9},
xticklabels={4,8,12,16,20,24,28,32,36,40},
y grid style={darkgray176},
ylabel={Precision@k},
ymin=0, ymax=0.74025,
ytick style={color=black}
]
\draw[draw=none,fill=darkviolet1270255] (axis cs:-0.175,0) rectangle (axis cs:0.175,0.149);
\addlegendimage{ybar,ybar legend,draw=none,fill=darkviolet1270255}
\addlegendentry{P@All}

\draw[draw=none,fill=darkviolet1270255] (axis cs:0.825,0) rectangle (axis cs:1.175,0.305);
\draw[draw=none,fill=darkviolet1270255] (axis cs:1.825,0) rectangle (axis cs:2.175,0.464);
\draw[draw=none,fill=darkviolet1270255] (axis cs:2.825,0) rectangle (axis cs:3.175,0.569);
\draw[draw=none,fill=darkviolet1270255] (axis cs:3.825,0) rectangle (axis cs:4.175,0.631);
\draw[draw=none,fill=darkviolet1270255] (axis cs:4.825,0) rectangle (axis cs:5.175,0.672);
\draw[draw=none,fill=darkviolet1270255] (axis cs:5.825,0) rectangle (axis cs:6.175,0.682);
\draw[draw=none,fill=darkviolet1270255] (axis cs:6.825,0) rectangle (axis cs:7.175,0.697);
\draw[draw=none,fill=darkviolet1270255] (axis cs:7.825,0) rectangle (axis cs:8.175,0.7);
\draw[draw=none,fill=darkviolet1270255] (axis cs:8.825,0) rectangle (axis cs:9.175,0.705);
\draw[draw=none,fill=royalblue8762253] (axis cs:-0.175,0) rectangle (axis cs:0.175,0.072);
\addlegendimage{ybar,ybar legend,draw=none,fill=royalblue8762253}
\addlegendentry{P@10}

\draw[draw=none,fill=royalblue8762253] (axis cs:0.825,0) rectangle (axis cs:1.175,0.162);
\draw[draw=none,fill=royalblue8762253] (axis cs:1.825,0) rectangle (axis cs:2.175,0.231);
\draw[draw=none,fill=royalblue8762253] (axis cs:2.825,0) rectangle (axis cs:3.175,0.282);
\draw[draw=none,fill=royalblue8762253] (axis cs:3.825,0) rectangle (axis cs:4.175,0.297);
\draw[draw=none,fill=royalblue8762253] (axis cs:4.825,0) rectangle (axis cs:5.175,0.318);
\draw[draw=none,fill=royalblue8762253] (axis cs:5.825,0) rectangle (axis cs:6.175,0.321);
\draw[draw=none,fill=royalblue8762253] (axis cs:6.825,0) rectangle (axis cs:7.175,0.331);
\draw[draw=none,fill=royalblue8762253] (axis cs:7.825,0) rectangle (axis cs:8.175,0.333);
\draw[draw=none,fill=royalblue8762253] (axis cs:8.825,0) rectangle (axis cs:9.175,0.336);
\draw[draw=none,fill=dodgerblue45123246] (axis cs:-0.175,0) rectangle (axis cs:0.175,0.072);
\addlegendimage{ybar,ybar legend,draw=none,fill=dodgerblue45123246}
\addlegendentry{P@5}

\draw[draw=none,fill=dodgerblue45123246] (axis cs:0.825,0) rectangle (axis cs:1.175,0.154);
\draw[draw=none,fill=dodgerblue45123246] (axis cs:1.825,0) rectangle (axis cs:2.175,0.213);
\draw[draw=none,fill=dodgerblue45123246] (axis cs:2.825,0) rectangle (axis cs:3.175,0.264);
\draw[draw=none,fill=dodgerblue45123246] (axis cs:3.825,0) rectangle (axis cs:4.175,0.279);
\draw[draw=none,fill=dodgerblue45123246] (axis cs:4.825,0) rectangle (axis cs:5.175,0.295);
\draw[draw=none,fill=dodgerblue45123246] (axis cs:5.825,0) rectangle (axis cs:6.175,0.297);
\draw[draw=none,fill=dodgerblue45123246] (axis cs:6.825,0) rectangle (axis cs:7.175,0.308);
\draw[draw=none,fill=dodgerblue45123246] (axis cs:7.825,0) rectangle (axis cs:8.175,0.31);
\draw[draw=none,fill=dodgerblue45123246] (axis cs:8.825,0) rectangle (axis cs:9.175,0.313);
\draw[draw=none,fill=deepskyblue3176236] (axis cs:-0.175,0) rectangle (axis cs:0.175,0.054);
\addlegendimage{ybar,ybar legend,draw=none,fill=deepskyblue3176236}
\addlegendentry{P@1}

\draw[draw=none,fill=deepskyblue3176236] (axis cs:0.825,0) rectangle (axis cs:1.175,0.126);
\draw[draw=none,fill=deepskyblue3176236] (axis cs:1.825,0) rectangle (axis cs:2.175,0.174);
\draw[draw=none,fill=deepskyblue3176236] (axis cs:2.825,0) rectangle (axis cs:3.175,0.218);
\draw[draw=none,fill=deepskyblue3176236] (axis cs:3.825,0) rectangle (axis cs:4.175,0.233);
\draw[draw=none,fill=deepskyblue3176236] (axis cs:4.825,0) rectangle (axis cs:5.175,0.244);
\draw[draw=none,fill=deepskyblue3176236] (axis cs:5.825,0) rectangle (axis cs:6.175,0.246);
\draw[draw=none,fill=deepskyblue3176236] (axis cs:6.825,0) rectangle (axis cs:7.175,0.256);
\draw[draw=none,fill=deepskyblue3176236] (axis cs:7.825,0) rectangle (axis cs:8.175,0.259);
\draw[draw=none,fill=deepskyblue3176236] (axis cs:8.825,0) rectangle (axis cs:9.175,0.262);
\addplot [red, thick] coordinates {(-0.8925,0.15) (14.8925,0.15)};
\end{axis}

\end{tikzpicture}
}
\resizebox{.24\textwidth}{!}{% This file was created with tikzplotlib v0.10.1.
\begin{tikzpicture}

\definecolor{darkgray176}{RGB}{176,176,176}
\definecolor{darkviolet1270255}{RGB}{127,0,255}
\definecolor{deepskyblue3176236}{RGB}{3,176,236}
\definecolor{dodgerblue45123246}{RGB}{45,123,246}
\definecolor{royalblue8762253}{RGB}{87,62,253}

\begin{axis}[
tick align=outside,
tick pos=left,
title={\(\displaystyle \Delta=3\) Repair Precision},
legend style={fill opacity=0.8, draw opacity=1, text opacity=1, legend columns=1, legend pos=north west},
x grid style={darkgray176},
xlabel={Seconds},
xmin=-0.6425, xmax=9.6425,
xtick style={color=black},
xtick={0,1,2,3,4,5,6,7,8,9},
xticklabels={6,12,18,24,30,36,42,48,54,60},
y grid style={darkgray176},
ylabel={Precision@k},
ymin=0, ymax=0.5313,
ytick style={color=black}
]
\draw[draw=none,fill=darkviolet1270255] (axis cs:-0.175,0) rectangle (axis cs:0.175,0.013);
\addlegendimage{ybar,ybar legend,draw=none,fill=darkviolet1270255}
\addlegendentry{P@All}

\draw[draw=none,fill=darkviolet1270255] (axis cs:0.825,0) rectangle (axis cs:1.175,0.065);
\draw[draw=none,fill=darkviolet1270255] (axis cs:1.825,0) rectangle (axis cs:2.175,0.156);
\draw[draw=none,fill=darkviolet1270255] (axis cs:2.825,0) rectangle (axis cs:3.175,0.221);
\draw[draw=none,fill=darkviolet1270255] (axis cs:3.825,0) rectangle (axis cs:4.175,0.325);
\draw[draw=none,fill=darkviolet1270255] (axis cs:4.825,0) rectangle (axis cs:5.175,0.377);
\draw[draw=none,fill=darkviolet1270255] (axis cs:5.825,0) rectangle (axis cs:6.175,0.403);
\draw[draw=none,fill=darkviolet1270255] (axis cs:6.825,0) rectangle (axis cs:7.175,0.455);
\draw[draw=none,fill=darkviolet1270255] (axis cs:7.825,0) rectangle (axis cs:8.175,0.468);
\draw[draw=none,fill=darkviolet1270255] (axis cs:8.825,0) rectangle (axis cs:9.175,0.506);
\draw[draw=none,fill=royalblue8762253] (axis cs:-0.175,0) rectangle (axis cs:0.175,0);
\addlegendimage{ybar,ybar legend,draw=none,fill=royalblue8762253}
\addlegendentry{P@10}

\draw[draw=none,fill=royalblue8762253] (axis cs:0.825,0) rectangle (axis cs:1.175,0.013);
\draw[draw=none,fill=royalblue8762253] (axis cs:1.825,0) rectangle (axis cs:2.175,0.052);
\draw[draw=none,fill=royalblue8762253] (axis cs:2.825,0) rectangle (axis cs:3.175,0.065);
\draw[draw=none,fill=royalblue8762253] (axis cs:3.825,0) rectangle (axis cs:4.175,0.104);
\draw[draw=none,fill=royalblue8762253] (axis cs:4.825,0) rectangle (axis cs:5.175,0.117);
\draw[draw=none,fill=royalblue8762253] (axis cs:5.825,0) rectangle (axis cs:6.175,0.13);
\draw[draw=none,fill=royalblue8762253] (axis cs:6.825,0) rectangle (axis cs:7.175,0.143);
\draw[draw=none,fill=royalblue8762253] (axis cs:7.825,0) rectangle (axis cs:8.175,0.156);
\draw[draw=none,fill=royalblue8762253] (axis cs:8.825,0) rectangle (axis cs:9.175,0.156);
\draw[draw=none,fill=dodgerblue45123246] (axis cs:-0.175,0) rectangle (axis cs:0.175,0);
\addlegendimage{ybar,ybar legend,draw=none,fill=dodgerblue45123246}
\addlegendentry{P@5}

\draw[draw=none,fill=dodgerblue45123246] (axis cs:0.825,0) rectangle (axis cs:1.175,0.013);
\draw[draw=none,fill=dodgerblue45123246] (axis cs:1.825,0) rectangle (axis cs:2.175,0.052);
\draw[draw=none,fill=dodgerblue45123246] (axis cs:2.825,0) rectangle (axis cs:3.175,0.065);
\draw[draw=none,fill=dodgerblue45123246] (axis cs:3.825,0) rectangle (axis cs:4.175,0.104);
\draw[draw=none,fill=dodgerblue45123246] (axis cs:4.825,0) rectangle (axis cs:5.175,0.104);
\draw[draw=none,fill=dodgerblue45123246] (axis cs:5.825,0) rectangle (axis cs:6.175,0.117);
\draw[draw=none,fill=dodgerblue45123246] (axis cs:6.825,0) rectangle (axis cs:7.175,0.13);
\draw[draw=none,fill=dodgerblue45123246] (axis cs:7.825,0) rectangle (axis cs:8.175,0.143);
\draw[draw=none,fill=dodgerblue45123246] (axis cs:8.825,0) rectangle (axis cs:9.175,0.143);
\draw[draw=none,fill=deepskyblue3176236] (axis cs:-0.175,0) rectangle (axis cs:0.175,0);
\addlegendimage{ybar,ybar legend,draw=none,fill=deepskyblue3176236}
\addlegendentry{P@1}

\draw[draw=none,fill=deepskyblue3176236] (axis cs:0.825,0) rectangle (axis cs:1.175,0.013);
\draw[draw=none,fill=deepskyblue3176236] (axis cs:1.825,0) rectangle (axis cs:2.175,0.052);
\draw[draw=none,fill=deepskyblue3176236] (axis cs:2.825,0) rectangle (axis cs:3.175,0.065);
\draw[draw=none,fill=deepskyblue3176236] (axis cs:3.825,0) rectangle (axis cs:4.175,0.091);
\draw[draw=none,fill=deepskyblue3176236] (axis cs:4.825,0) rectangle (axis cs:5.175,0.091);
\draw[draw=none,fill=deepskyblue3176236] (axis cs:5.825,0) rectangle (axis cs:6.175,0.104);
\draw[draw=none,fill=deepskyblue3176236] (axis cs:6.825,0) rectangle (axis cs:7.175,0.117);
\draw[draw=none,fill=deepskyblue3176236] (axis cs:7.825,0) rectangle (axis cs:8.175,0.13);
\draw[draw=none,fill=deepskyblue3176236] (axis cs:8.825,0) rectangle (axis cs:9.175,0.13);
\addplot [red, thick] coordinates {(-0.8925,0.07) (14.8925,0.07)};
\end{axis}

\end{tikzpicture}
}
\resizebox{.24\textwidth}{!}{% This file was created with tikzplotlib v0.10.1.
\begin{tikzpicture}
\begin{axis}[
tick align=outside,
tick pos=left,
axis lines=left,
title={\textbf{\Large Repair Precision \(\mathbf{(\bm\Delta=4)}\)}},
legend style={fill opacity=0.8, draw opacity=1, text opacity=1, legend columns=1, legend pos=north west},
x grid style={darkgray176},
xlabel={Seconds},
xmin=-0.5925, xmax=8.5925,
xtick style={color=black},
xtick={0,1,2,3,4,5,6,7,8},
xticklabels={10,20,30,40,50,60,70,80,90},
y grid style={darkgray176},
ylabel={Precision@k},
ymin=0, ymax=0.54915,
ytick style={color=black}
]
\draw[draw=none,fill=darkviolet1270255] (axis cs:-0.175,0) rectangle (axis cs:0.175,0.123);

\draw[draw=none,fill=darkviolet1270255] (axis cs:0.825,0) rectangle (axis cs:1.175,0.323);
\draw[draw=none,fill=darkviolet1270255] (axis cs:1.825,0) rectangle (axis cs:2.175,0.415);
\draw[draw=none,fill=darkviolet1270255] (axis cs:2.825,0) rectangle (axis cs:3.175,0.431);
\draw[draw=none,fill=darkviolet1270255] (axis cs:3.825,0) rectangle (axis cs:4.175,0.477);
\draw[draw=none,fill=darkviolet1270255] (axis cs:4.825,0) rectangle (axis cs:5.175,0.508);
\draw[draw=none,fill=darkviolet1270255] (axis cs:5.825,0) rectangle (axis cs:6.175,0.523);
\draw[draw=none,fill=darkviolet1270255] (axis cs:6.825,0) rectangle (axis cs:7.175,0.523);
\draw[draw=none,fill=darkviolet1270255] (axis cs:7.825,0) rectangle (axis cs:8.175,0.523);
\draw[draw=none,fill=royalblue8762253] (axis cs:-0.175,0) rectangle (axis cs:0.175,0.092);

\draw[draw=none,fill=royalblue8762253] (axis cs:0.825,0) rectangle (axis cs:1.175,0.169);
\draw[draw=none,fill=royalblue8762253] (axis cs:1.825,0) rectangle (axis cs:2.175,0.215);
\draw[draw=none,fill=royalblue8762253] (axis cs:2.825,0) rectangle (axis cs:3.175,0.231);
\draw[draw=none,fill=royalblue8762253] (axis cs:3.825,0) rectangle (axis cs:4.175,0.277);
\draw[draw=none,fill=royalblue8762253] (axis cs:4.825,0) rectangle (axis cs:5.175,0.308);
\draw[draw=none,fill=royalblue8762253] (axis cs:5.825,0) rectangle (axis cs:6.175,0.323);
\draw[draw=none,fill=royalblue8762253] (axis cs:6.825,0) rectangle (axis cs:7.175,0.323);
\draw[draw=none,fill=royalblue8762253] (axis cs:7.825,0) rectangle (axis cs:8.175,0.323);
\draw[draw=none,fill=dodgerblue45123246] (axis cs:-0.175,0) rectangle (axis cs:0.175,0.092);

\draw[draw=none,fill=dodgerblue45123246] (axis cs:0.825,0) rectangle (axis cs:1.175,0.169);
\draw[draw=none,fill=dodgerblue45123246] (axis cs:1.825,0) rectangle (axis cs:2.175,0.215);
\draw[draw=none,fill=dodgerblue45123246] (axis cs:2.825,0) rectangle (axis cs:3.175,0.231);
\draw[draw=none,fill=dodgerblue45123246] (axis cs:3.825,0) rectangle (axis cs:4.175,0.277);
\draw[draw=none,fill=dodgerblue45123246] (axis cs:4.825,0) rectangle (axis cs:5.175,0.308);
\draw[draw=none,fill=dodgerblue45123246] (axis cs:5.825,0) rectangle (axis cs:6.175,0.323);
\draw[draw=none,fill=dodgerblue45123246] (axis cs:6.825,0) rectangle (axis cs:7.175,0.323);
\draw[draw=none,fill=dodgerblue45123246] (axis cs:7.825,0) rectangle (axis cs:8.175,0.323);
\draw[draw=none,fill=deepskyblue3176236] (axis cs:-0.175,0) rectangle (axis cs:0.175,0.077);

\draw[draw=none,fill=deepskyblue3176236] (axis cs:0.825,0) rectangle (axis cs:1.175,0.138);
\draw[draw=none,fill=deepskyblue3176236] (axis cs:1.825,0) rectangle (axis cs:2.175,0.185);
\draw[draw=none,fill=deepskyblue3176236] (axis cs:2.825,0) rectangle (axis cs:3.175,0.2);
\draw[draw=none,fill=deepskyblue3176236] (axis cs:3.825,0) rectangle (axis cs:4.175,0.246);
\draw[draw=none,fill=deepskyblue3176236] (axis cs:4.825,0) rectangle (axis cs:5.175,0.277);
\draw[draw=none,fill=deepskyblue3176236] (axis cs:5.825,0) rectangle (axis cs:6.175,0.292);
\draw[draw=none,fill=deepskyblue3176236] (axis cs:6.825,0) rectangle (axis cs:7.175,0.292);
\draw[draw=none,fill=deepskyblue3176236] (axis cs:7.825,0) rectangle (axis cs:8.175,0.292);
\addplot [red, thick] coordinates {(-0.8925,0.10) (14.8925,0.10)};
\addplot [orange, thick] coordinates {(-0.8925,0.01) (14.8925,0.01)};
\end{axis}

\end{tikzpicture}
}
%    \resizebox{.24\textwidth}{!}{% This file was created with tikzplotlib v0.10.1.
\begin{tikzpicture}
\begin{axis}[
tick align=outside,
tick pos=left,
axis lines=left,
title={\Large\(\displaystyle \Delta=5\) Repair Precision},
legend style={fill opacity=0.8, draw opacity=1, text opacity=1, legend columns=1, legend pos=north west},
x grid style={darkgray176},
xlabel={Seconds},
xmin=-0.5925, xmax=8.5925,
xtick={0,1,2,3,4,5,6,7,8},
xticklabels={10,20,30,40,50,60,70,80,90},
y grid style={darkgray176},
ylabel={Precision@k},
ymin=0, ymax=0.54915,
]
\draw[draw=none,fill=darkviolet1270255] (axis cs:-0.175,0) rectangle (axis cs:0.175,0.123);
\addlegendimage{ybar,ybar legend,draw=none,fill=darkviolet1270255}
\addlegendentry{P@All}

\draw[draw=none,fill=darkviolet1270255] (axis cs:0.825,0) rectangle (axis cs:1.175,0.323);
\draw[draw=none,fill=darkviolet1270255] (axis cs:1.825,0) rectangle (axis cs:2.175,0.415);
\draw[draw=none,fill=darkviolet1270255] (axis cs:2.825,0) rectangle (axis cs:3.175,0.431);
\draw[draw=none,fill=darkviolet1270255] (axis cs:3.825,0) rectangle (axis cs:4.175,0.477);
\draw[draw=none,fill=darkviolet1270255] (axis cs:4.825,0) rectangle (axis cs:5.175,0.508);
\draw[draw=none,fill=darkviolet1270255] (axis cs:5.825,0) rectangle (axis cs:6.175,0.523);
\draw[draw=none,fill=darkviolet1270255] (axis cs:6.825,0) rectangle (axis cs:7.175,0.523);
\draw[draw=none,fill=darkviolet1270255] (axis cs:7.825,0) rectangle (axis cs:8.175,0.523);
\draw[draw=none,fill=royalblue8762253] (axis cs:-0.175,0) rectangle (axis cs:0.175,0.092);
\addlegendimage{ybar,ybar legend,draw=none,fill=royalblue8762253}
\addlegendentry{P@10}

\draw[draw=none,fill=royalblue8762253] (axis cs:0.825,0) rectangle (axis cs:1.175,0.169);
\draw[draw=none,fill=royalblue8762253] (axis cs:1.825,0) rectangle (axis cs:2.175,0.215);
\draw[draw=none,fill=royalblue8762253] (axis cs:2.825,0) rectangle (axis cs:3.175,0.231);
\draw[draw=none,fill=royalblue8762253] (axis cs:3.825,0) rectangle (axis cs:4.175,0.277);
\draw[draw=none,fill=royalblue8762253] (axis cs:4.825,0) rectangle (axis cs:5.175,0.308);
\draw[draw=none,fill=royalblue8762253] (axis cs:5.825,0) rectangle (axis cs:6.175,0.323);
\draw[draw=none,fill=royalblue8762253] (axis cs:6.825,0) rectangle (axis cs:7.175,0.323);
\draw[draw=none,fill=royalblue8762253] (axis cs:7.825,0) rectangle (axis cs:8.175,0.323);
\draw[draw=none,fill=dodgerblue45123246] (axis cs:-0.175,0) rectangle (axis cs:0.175,0.092);
\addlegendimage{ybar,ybar legend,draw=none,fill=dodgerblue45123246}
\addlegendentry{P@5}

\draw[draw=none,fill=dodgerblue45123246] (axis cs:0.825,0) rectangle (axis cs:1.175,0.169);
\draw[draw=none,fill=dodgerblue45123246] (axis cs:1.825,0) rectangle (axis cs:2.175,0.215);
\draw[draw=none,fill=dodgerblue45123246] (axis cs:2.825,0) rectangle (axis cs:3.175,0.231);
\draw[draw=none,fill=dodgerblue45123246] (axis cs:3.825,0) rectangle (axis cs:4.175,0.277);
\draw[draw=none,fill=dodgerblue45123246] (axis cs:4.825,0) rectangle (axis cs:5.175,0.308);
\draw[draw=none,fill=dodgerblue45123246] (axis cs:5.825,0) rectangle (axis cs:6.175,0.323);
\draw[draw=none,fill=dodgerblue45123246] (axis cs:6.825,0) rectangle (axis cs:7.175,0.323);
\draw[draw=none,fill=dodgerblue45123246] (axis cs:7.825,0) rectangle (axis cs:8.175,0.323);
\draw[draw=none,fill=deepskyblue3176236] (axis cs:-0.175,0) rectangle (axis cs:0.175,0.077);
\addlegendimage{ybar,ybar legend,draw=none,fill=deepskyblue3176236}
\addlegendentry{P@1}

\draw[draw=none,fill=deepskyblue3176236] (axis cs:0.825,0) rectangle (axis cs:1.175,0.138);
\draw[draw=none,fill=deepskyblue3176236] (axis cs:1.825,0) rectangle (axis cs:2.175,0.185);
\draw[draw=none,fill=deepskyblue3176236] (axis cs:2.825,0) rectangle (axis cs:3.175,0.2);
\draw[draw=none,fill=deepskyblue3176236] (axis cs:3.825,0) rectangle (axis cs:4.175,0.246);
\draw[draw=none,fill=deepskyblue3176236] (axis cs:4.825,0) rectangle (axis cs:5.175,0.277);
\draw[draw=none,fill=deepskyblue3176236] (axis cs:5.825,0) rectangle (axis cs:6.175,0.292);
\draw[draw=none,fill=deepskyblue3176236] (axis cs:6.825,0) rectangle (axis cs:7.175,0.292);
\draw[draw=none,fill=deepskyblue3176236] (axis cs:7.825,0) rectangle (axis cs:8.175,0.292);
\addplot [red, thick] coordinates {(-0.8925,0.10) (14.8925,0.10)};
\end{axis}

\end{tikzpicture}
}
%\resizebox{.3\textwidth}{!}{% This file was created with tikzplotlib v0.10.1.
\begin{tikzpicture}

\definecolor{darkgray176}{RGB}{176,176,176}
\definecolor{darkviolet1270255}{RGB}{127,0,255}
\definecolor{deepskyblue3176236}{RGB}{3,176,236}
\definecolor{dodgerblue45123246}{RGB}{45,123,246}
\definecolor{lightgray204}{RGB}{204,204,204}
\definecolor{royalblue8762253}{RGB}{87,62,253}

\begin{axis}[
legend cell align={left},
legend style={fill opacity=0.8, draw opacity=1, text opacity=1, draw=lightgray204, legend columns=-1, legend pos=north west},
tick align=outside,
tick pos=left,
title={$\Delta=1$ Repair Precision},
x grid style={darkgray176},
xlabel={Seconds},
xmin=-0.4925, xmax=6.4925,
xtick style={color=black},
xtick={0,1,2,3,4,5,6},
xticklabels={20,60,100,140,180,220,260},
y grid style={darkgray176},
ylabel={\phantom{Precision@k}},
ymin=0, ymax=0.77595,
ytick style={color=black}
]
\addlegendimage{empty legend}
\addlegendentry{P@}
\draw[draw=none,fill=darkviolet1270255] (axis cs:-0.175,0) rectangle (axis cs:0.175,0.145);
\addlegendimage{ybar,ybar legend,draw=none,fill=darkviolet1270255}
\addlegendentry{All}

\draw[draw=none,fill=darkviolet1270255] (axis cs:0.825,0) rectangle (axis cs:1.175,0.304);
\draw[draw=none,fill=darkviolet1270255] (axis cs:1.825,0) rectangle (axis cs:2.175,0.42);
\draw[draw=none,fill=darkviolet1270255] (axis cs:2.825,0) rectangle (axis cs:3.175,0.507);
\draw[draw=none,fill=darkviolet1270255] (axis cs:3.825,0) rectangle (axis cs:4.175,0.594);
\draw[draw=none,fill=darkviolet1270255] (axis cs:4.825,0) rectangle (axis cs:5.175,0.71);
\draw[draw=none,fill=darkviolet1270255] (axis cs:5.825,0) rectangle (axis cs:6.175,0.739);
\draw[draw=none,fill=royalblue8762253] (axis cs:-0.175,0) rectangle (axis cs:0.175,0.145);
\addlegendimage{ybar,ybar legend,draw=none,fill=royalblue8762253}
\addlegendentry{10}

\draw[draw=none,fill=royalblue8762253] (axis cs:0.825,0) rectangle (axis cs:1.175,0.304);
\draw[draw=none,fill=royalblue8762253] (axis cs:1.825,0) rectangle (axis cs:2.175,0.42);
\draw[draw=none,fill=royalblue8762253] (axis cs:2.825,0) rectangle (axis cs:3.175,0.507);
\draw[draw=none,fill=royalblue8762253] (axis cs:3.825,0) rectangle (axis cs:4.175,0.594);
\draw[draw=none,fill=royalblue8762253] (axis cs:4.825,0) rectangle (axis cs:5.175,0.703);
\draw[draw=none,fill=royalblue8762253] (axis cs:5.825,0) rectangle (axis cs:6.175,0.732);
\draw[draw=none,fill=dodgerblue45123246] (axis cs:-0.175,0) rectangle (axis cs:0.175,0.145);
\addlegendimage{ybar,ybar legend,draw=none,fill=dodgerblue45123246}
\addlegendentry{5}

\draw[draw=none,fill=dodgerblue45123246] (axis cs:0.825,0) rectangle (axis cs:1.175,0.304);
\draw[draw=none,fill=dodgerblue45123246] (axis cs:1.825,0) rectangle (axis cs:2.175,0.42);
\draw[draw=none,fill=dodgerblue45123246] (axis cs:2.825,0) rectangle (axis cs:3.175,0.493);
\draw[draw=none,fill=dodgerblue45123246] (axis cs:3.825,0) rectangle (axis cs:4.175,0.565);
\draw[draw=none,fill=dodgerblue45123246] (axis cs:4.825,0) rectangle (axis cs:5.175,0.638);
\draw[draw=none,fill=dodgerblue45123246] (axis cs:5.825,0) rectangle (axis cs:6.175,0.667);
\draw[draw=none,fill=deepskyblue3176236] (axis cs:-0.175,0) rectangle (axis cs:0.175,0.116);
\addlegendimage{ybar,ybar legend,draw=none,fill=deepskyblue3176236}
\addlegendentry{1}

\draw[draw=none,fill=deepskyblue3176236] (axis cs:0.825,0) rectangle (axis cs:1.175,0.203);
\draw[draw=none,fill=deepskyblue3176236] (axis cs:1.825,0) rectangle (axis cs:2.175,0.268);
\draw[draw=none,fill=deepskyblue3176236] (axis cs:2.825,0) rectangle (axis cs:3.175,0.312);
\draw[draw=none,fill=deepskyblue3176236] (axis cs:3.825,0) rectangle (axis cs:4.175,0.355);
\draw[draw=none,fill=deepskyblue3176236] (axis cs:4.825,0) rectangle (axis cs:5.175,0.428);
\draw[draw=none,fill=deepskyblue3176236] (axis cs:5.825,0) rectangle (axis cs:6.175,0.442);
\end{axis}

\end{tikzpicture}
}
%\resizebox{.307\textwidth}{!}{% This file was created with tikzplotlib v0.10.1.
\begin{tikzpicture}

\definecolor{darkgray176}{RGB}{176,176,176}
\definecolor{darkviolet1270255}{RGB}{127,0,255}
\definecolor{deepskyblue3176236}{RGB}{3,176,236}
\definecolor{dodgerblue45123246}{RGB}{45,123,246}
\definecolor{lightgray204}{RGB}{204,204,204}
\definecolor{royalblue8762253}{RGB}{87,62,253}

\begin{axis}[
legend cell align={left},
legend style={fill opacity=0.8, draw opacity=1, text opacity=1, draw=lightgray204, legend columns=-1, legend pos=north west},
tick align=outside,
tick pos=left,
title={$\Delta=2$ Repair Precision},
x grid style={darkgray176},
xlabel={Seconds},
xmin=-0.4925, xmax=6.4925,
xtick style={color=black},
xtick={0,1,2,3,4,5,6},
xticklabels={20,60,100,140,180,220,260},
y grid style={darkgray176},
ylabel={\phantom{Precision@k}},
ymin=0, ymax=0.40635,
ytick style={color=black}
]
\addlegendimage{empty legend}
\addlegendentry{P@}
\draw[draw=none,fill=darkviolet1270255] (axis cs:-0.175,0) rectangle (axis cs:0.175,0.048);
\addlegendimage{ybar,ybar legend,draw=none,fill=darkviolet1270255}
\addlegendentry{All}

\draw[draw=none,fill=darkviolet1270255] (axis cs:0.825,0) rectangle (axis cs:1.175,0.081);
\draw[draw=none,fill=darkviolet1270255] (axis cs:1.825,0) rectangle (axis cs:2.175,0.177);
\draw[draw=none,fill=darkviolet1270255] (axis cs:2.825,0) rectangle (axis cs:3.175,0.274);
\draw[draw=none,fill=darkviolet1270255] (axis cs:3.825,0) rectangle (axis cs:4.175,0.306);
\draw[draw=none,fill=darkviolet1270255] (axis cs:4.825,0) rectangle (axis cs:5.175,0.355);
\draw[draw=none,fill=darkviolet1270255] (axis cs:5.825,0) rectangle (axis cs:6.175,0.387);
\draw[draw=none,fill=royalblue8762253] (axis cs:-0.175,0) rectangle (axis cs:0.175,0.048);
\addlegendimage{ybar,ybar legend,draw=none,fill=royalblue8762253}
\addlegendentry{10}

\draw[draw=none,fill=royalblue8762253] (axis cs:0.825,0) rectangle (axis cs:1.175,0.081);
\draw[draw=none,fill=royalblue8762253] (axis cs:1.825,0) rectangle (axis cs:2.175,0.113);
\draw[draw=none,fill=royalblue8762253] (axis cs:2.825,0) rectangle (axis cs:3.175,0.21);
\draw[draw=none,fill=royalblue8762253] (axis cs:3.825,0) rectangle (axis cs:4.175,0.226);
\draw[draw=none,fill=royalblue8762253] (axis cs:4.825,0) rectangle (axis cs:5.175,0.258);
\draw[draw=none,fill=royalblue8762253] (axis cs:5.825,0) rectangle (axis cs:6.175,0.242);
\draw[draw=none,fill=dodgerblue45123246] (axis cs:-0.175,0) rectangle (axis cs:0.175,0.048);
\addlegendimage{ybar,ybar legend,draw=none,fill=dodgerblue45123246}
\addlegendentry{5}

\draw[draw=none,fill=dodgerblue45123246] (axis cs:0.825,0) rectangle (axis cs:1.175,0.032);
\draw[draw=none,fill=dodgerblue45123246] (axis cs:1.825,0) rectangle (axis cs:2.175,0.065);
\draw[draw=none,fill=dodgerblue45123246] (axis cs:2.825,0) rectangle (axis cs:3.175,0.129);
\draw[draw=none,fill=dodgerblue45123246] (axis cs:3.825,0) rectangle (axis cs:4.175,0.113);
\draw[draw=none,fill=dodgerblue45123246] (axis cs:4.825,0) rectangle (axis cs:5.175,0.129);
\draw[draw=none,fill=dodgerblue45123246] (axis cs:5.825,0) rectangle (axis cs:6.175,0.113);
\draw[draw=none,fill=deepskyblue3176236] (axis cs:-0.175,0) rectangle (axis cs:0.175,0.016);
\addlegendimage{ybar,ybar legend,draw=none,fill=deepskyblue3176236}
\addlegendentry{1}

\draw[draw=none,fill=deepskyblue3176236] (axis cs:0.825,0) rectangle (axis cs:1.175,0);
\draw[draw=none,fill=deepskyblue3176236] (axis cs:1.825,0) rectangle (axis cs:2.175,0.016);
\draw[draw=none,fill=deepskyblue3176236] (axis cs:2.825,0) rectangle (axis cs:3.175,0.065);
\draw[draw=none,fill=deepskyblue3176236] (axis cs:3.825,0) rectangle (axis cs:4.175,0.048);
\draw[draw=none,fill=deepskyblue3176236] (axis cs:4.825,0) rectangle (axis cs:5.175,0.032);
\draw[draw=none,fill=deepskyblue3176236] (axis cs:5.825,0) rectangle (axis cs:6.175,0.032);
\end{axis}

\end{tikzpicture}
}
\caption{Latency benchmarks. Note the varying y-axis ranges. The red line marks Seq2Parse and the orange line marks BIFI's Precision@1 on the same repairs.}\label{fig:human}
\end{figure}
\end{frame}

\begin{frame}[fragile]{Results from sample efficiency experiments}
\begin{figure}[h!]
% This file was created with tikzplotlib v0.10.1.
\begin{tikzpicture}[scale=0.75]
\begin{axis}[
legend cell align={left},
legend style={fill opacity=0.8, draw opacity=1, text opacity=1, draw=lightgray204, legend columns=-1, legend pos=north west},
width=\textwidth,
height=.4\textwidth,
log basis x={10},
tick align=outside,
tick pos=left,
axis lines*=left,
title={\textbf{Uniform Sampling Efficiency}},
x grid style={darkgray176},
xlabel={Samples drawn (log scale)},
xmin=0.34892211545542, xmax=1001,
xmode=log,
%xtick style={color=black},
xtick={0.01,0.1,1,10,100,1000,10000},
xticklabels={
  \(\displaystyle 10^-2\),
  \(\displaystyle 10^-1\),
  \(\displaystyle 10^2\),
  \(\displaystyle 10^3\),
  \(\displaystyle 10^4\),
  \(\displaystyle 10^5\),
  \(\displaystyle 10^6\)
},
y grid style={darkgray176},
ylabel={Precision@All},
ymin=0, ymax=101,
%ytick style={color=black}
ytick={0,20,40,60,80,100}
]
\draw[draw=none,fill=green!80] (axis cs:-0.5,0) rectangle (axis cs:0.9,100);
\addlegendimage{empty legend}
\addlegendentry{$\Delta(\err\sigma, \sigma')=$}
\addlegendimage{ybar,ybar legend,draw=none,fill=green!80}
\addlegendentry{\footnotesize{$1,$}}
\addlegendimage{ybar,ybar legend,draw=none,fill=blue!80}
\addlegendentry{\footnotesize{$2,$}}
\addlegendimage{ybar,ybar legend,draw=none,fill=orange!80}
\addlegendentry{\footnotesize{$3$}}

\draw[draw=none,fill=green!80] (axis cs:0.1,0) rectangle (axis cs:1.9,100);
\draw[draw=none,fill=green!80] (axis cs:1.1,0) rectangle (axis cs:2.9,100);
\draw[draw=none,fill=green!80] (axis cs:2.1,0) rectangle (axis cs:3.9,100);
\draw[draw=none,fill=green!80] (axis cs:3.1,0) rectangle (axis cs:4.9,100);
\draw[draw=none,fill=green!80] (axis cs:4.1,0) rectangle (axis cs:5.9,100);
\draw[draw=none,fill=green!80] (axis cs:5.1,0) rectangle (axis cs:6.9,100);
\draw[draw=none,fill=green!80] (axis cs:6.1,0) rectangle (axis cs:7.9,100);
\draw[draw=none,fill=green!80] (axis cs:7.1,0) rectangle (axis cs:8.9,100);
\draw[draw=none,fill=green!80] (axis cs:8.1,0) rectangle (axis cs:9.9,100);
\draw[draw=none,fill=green!80] (axis cs:9.1,0) rectangle (axis cs:10.9,100);
\draw[draw=none,fill=green!80] (axis cs:10.1,0) rectangle (axis cs:11.9,100);
\draw[draw=none,fill=green!80] (axis cs:11.1,0) rectangle (axis cs:12.9,100);
\draw[draw=none,fill=green!80] (axis cs:12.1,0) rectangle (axis cs:13.9,100);
\draw[draw=none,fill=green!80] (axis cs:13.1,0) rectangle (axis cs:14.9,100);
\draw[draw=none,fill=green!80] (axis cs:14.1,0) rectangle (axis cs:15.9,100);
\draw[draw=none,fill=green!80] (axis cs:15.1,0) rectangle (axis cs:16.9,100);
\draw[draw=none,fill=green!80] (axis cs:16.1,0) rectangle (axis cs:17.9,100);
\draw[draw=none,fill=green!80] (axis cs:17.1,0) rectangle (axis cs:18.9,100);
\draw[draw=none,fill=green!80] (axis cs:18.1,0) rectangle (axis cs:19.9,100);
\draw[draw=none,fill=green!80] (axis cs:19.1,0) rectangle (axis cs:20.9,100);
\draw[draw=none,fill=green!80] (axis cs:20.1,0) rectangle (axis cs:21.9,100);
\draw[draw=none,fill=green!80] (axis cs:21.1,0) rectangle (axis cs:22.9,100);
\draw[draw=none,fill=green!80] (axis cs:22.1,0) rectangle (axis cs:23.9,100);
\draw[draw=none,fill=green!80] (axis cs:23.1,0) rectangle (axis cs:24.9,100);
\draw[draw=none,fill=green!80] (axis cs:24.1,0) rectangle (axis cs:25.9,100);
\draw[draw=none,fill=green!80] (axis cs:25.1,0) rectangle (axis cs:26.9,100);
\draw[draw=none,fill=green!80] (axis cs:26.1,0) rectangle (axis cs:27.9,100);
\draw[draw=none,fill=green!80] (axis cs:27.1,0) rectangle (axis cs:28.9,100);
\draw[draw=none,fill=green!80] (axis cs:28.1,0) rectangle (axis cs:29.9,100);
\draw[draw=none,fill=green!80] (axis cs:29.1,0) rectangle (axis cs:30.9,100);
\draw[draw=none,fill=green!80] (axis cs:30.1,0) rectangle (axis cs:31.9,100);
\draw[draw=none,fill=green!80] (axis cs:31.1,0) rectangle (axis cs:32.9,100);
\draw[draw=none,fill=green!80] (axis cs:32.1,0) rectangle (axis cs:33.9,100);
\draw[draw=none,fill=green!80] (axis cs:33.1,0) rectangle (axis cs:34.9,100);
\draw[draw=none,fill=green!80] (axis cs:34.1,0) rectangle (axis cs:35.9,100);
\draw[draw=none,fill=green!80] (axis cs:35.1,0) rectangle (axis cs:36.9,100);
\draw[draw=none,fill=green!80] (axis cs:36.1,0) rectangle (axis cs:37.9,100);
\draw[draw=none,fill=green!80] (axis cs:37.1,0) rectangle (axis cs:38.9,100);
\draw[draw=none,fill=green!80] (axis cs:38.1,0) rectangle (axis cs:39.9,100);
\draw[draw=none,fill=green!80] (axis cs:39.1,0) rectangle (axis cs:40.9,100);
\draw[draw=none,fill=green!80] (axis cs:40.1,0) rectangle (axis cs:41.9,100);
\draw[draw=none,fill=green!80] (axis cs:41.1,0) rectangle (axis cs:42.9,100);
\draw[draw=none,fill=green!80] (axis cs:42.1,0) rectangle (axis cs:43.9,100);
\draw[draw=none,fill=green!80] (axis cs:43.1,0) rectangle (axis cs:44.9,100);
\draw[draw=none,fill=green!80] (axis cs:44.1,0) rectangle (axis cs:45.9,100);
\draw[draw=none,fill=green!80] (axis cs:45.1,0) rectangle (axis cs:46.9,100);
\draw[draw=none,fill=green!80] (axis cs:46.1,0) rectangle (axis cs:47.9,100);
\draw[draw=none,fill=green!80] (axis cs:47.1,0) rectangle (axis cs:48.9,100);
\draw[draw=none,fill=green!80] (axis cs:48.1,0) rectangle (axis cs:49.9,100);
\draw[draw=none,fill=green!80] (axis cs:49.1,0) rectangle (axis cs:50.9,100);
\draw[draw=none,fill=green!80] (axis cs:50.1,0) rectangle (axis cs:51.9,100);
\draw[draw=none,fill=green!80] (axis cs:51.1,0) rectangle (axis cs:52.9,100);
\draw[draw=none,fill=green!80] (axis cs:52.1,0) rectangle (axis cs:53.9,100);
\draw[draw=none,fill=green!80] (axis cs:53.1,0) rectangle (axis cs:54.9,100);
\draw[draw=none,fill=green!80] (axis cs:54.1,0) rectangle (axis cs:55.9,100);
\draw[draw=none,fill=green!80] (axis cs:55.1,0) rectangle (axis cs:56.9,100);
\draw[draw=none,fill=green!80] (axis cs:56.1,0) rectangle (axis cs:57.9,100);
\draw[draw=none,fill=green!80] (axis cs:57.1,0) rectangle (axis cs:58.9,100);
\draw[draw=none,fill=green!80] (axis cs:58.1,0) rectangle (axis cs:59.9,100);
\draw[draw=none,fill=green!80] (axis cs:59.1,0) rectangle (axis cs:60.9,100);
\draw[draw=none,fill=green!80] (axis cs:60.1,0) rectangle (axis cs:61.9,100);
\draw[draw=none,fill=green!80] (axis cs:61.1,0) rectangle (axis cs:62.9,100);
\draw[draw=none,fill=green!80] (axis cs:62.1,0) rectangle (axis cs:63.9,100);
\draw[draw=none,fill=green!80] (axis cs:63.1,0) rectangle (axis cs:64.9,100);
\draw[draw=none,fill=green!80] (axis cs:64.1,0) rectangle (axis cs:65.9,100);
\draw[draw=none,fill=green!80] (axis cs:65.1,0) rectangle (axis cs:66.9,100);
\draw[draw=none,fill=green!80] (axis cs:66.1,0) rectangle (axis cs:67.9,100);
\draw[draw=none,fill=green!80] (axis cs:67.1,0) rectangle (axis cs:68.9,100);
\draw[draw=none,fill=green!80] (axis cs:68.1,0) rectangle (axis cs:69.9,100);
\draw[draw=none,fill=green!80] (axis cs:69.1,0) rectangle (axis cs:70.9,100);
\draw[draw=none,fill=green!80] (axis cs:70.1,0) rectangle (axis cs:71.9,100);
\draw[draw=none,fill=green!80] (axis cs:71.1,0) rectangle (axis cs:72.9,100);
\draw[draw=none,fill=green!80] (axis cs:72.1,0) rectangle (axis cs:73.9,100);
\draw[draw=none,fill=green!80] (axis cs:73.1,0) rectangle (axis cs:74.9,100);
\draw[draw=none,fill=green!80] (axis cs:74.1,0) rectangle (axis cs:75.9,100);
\draw[draw=none,fill=green!80] (axis cs:75.1,0) rectangle (axis cs:76.9,100);
\draw[draw=none,fill=green!80] (axis cs:76.1,0) rectangle (axis cs:77.9,100);
\draw[draw=none,fill=green!80] (axis cs:77.1,0) rectangle (axis cs:78.9,100);
\draw[draw=none,fill=green!80] (axis cs:78.1,0) rectangle (axis cs:79.9,100);
\draw[draw=none,fill=green!80] (axis cs:79.1,0) rectangle (axis cs:80.9,100);
\draw[draw=none,fill=green!80] (axis cs:80.1,0) rectangle (axis cs:81.9,100);
\draw[draw=none,fill=green!80] (axis cs:81.1,0) rectangle (axis cs:82.9,100);
\draw[draw=none,fill=green!80] (axis cs:82.1,0) rectangle (axis cs:83.9,100);

\draw[draw=none,fill=blue!80] (axis cs:0.1,0) rectangle (axis cs:1.9,19.7452229299363);
\draw[draw=none,fill=blue!80] (axis cs:1.1,0) rectangle (axis cs:2.9,26.7515923566879);
\draw[draw=none,fill=blue!80] (axis cs:2.1,0) rectangle (axis cs:3.9,31.5286624203822);
\draw[draw=none,fill=blue!80] (axis cs:3.1,0) rectangle (axis cs:4.9,35.3503184713376);
\draw[draw=none,fill=blue!80] (axis cs:4.1,0) rectangle (axis cs:5.9,38.2165605095541);
\draw[draw=none,fill=blue!80] (axis cs:5.1,0) rectangle (axis cs:6.9,40.4458598726115);
\draw[draw=none,fill=blue!80] (axis cs:6.1,0) rectangle (axis cs:7.9,44.9044585987261);
\draw[draw=none,fill=blue!80] (axis cs:7.1,0) rectangle (axis cs:8.9,48.0891719745223);
\draw[draw=none,fill=blue!80] (axis cs:8.1,0) rectangle (axis cs:9.9,51.5923566878981);
\draw[draw=none,fill=blue!80] (axis cs:9.1,0) rectangle (axis cs:10.9,54.7770700636943);
\draw[draw=none,fill=blue!80] (axis cs:10.1,0) rectangle (axis cs:11.9,57.6433121019108);
\draw[draw=none,fill=blue!80] (axis cs:11.1,0) rectangle (axis cs:12.9,58.5987261146497);
\draw[draw=none,fill=blue!80] (axis cs:12.1,2) rectangle (axis cs:13.9,61.1464968152866);
\draw[draw=none,fill=blue!80] (axis cs:13.1,2) rectangle (axis cs:14.9,63.3757961783439);
\draw[draw=none,fill=blue!80] (axis cs:14.1,4) rectangle (axis cs:15.9,66.8789808917197);
\draw[draw=none,fill=blue!80] (axis cs:15.1,4) rectangle (axis cs:16.9,69.7452229299363);
\draw[draw=none,fill=blue!80] (axis cs:16.1,6) rectangle (axis cs:17.9,70.3821656050955);
\draw[draw=none,fill=blue!80] (axis cs:17.1,6) rectangle (axis cs:18.9,72.9299363057325);
\draw[draw=none,fill=blue!80] (axis cs:18.1,6) rectangle (axis cs:19.9,75.1592356687898);
\draw[draw=none,fill=blue!80] (axis cs:19.1,6) rectangle (axis cs:20.9,76.1146496815287);
\draw[draw=none,fill=blue!80] (axis cs:20.1,8) rectangle (axis cs:21.9,78.343949044586);
\draw[draw=none,fill=blue!80] (axis cs:21.1,10) rectangle (axis cs:22.9,79.2993630573248);
\draw[draw=none,fill=blue!80] (axis cs:22.1,10) rectangle (axis cs:23.9,80.2547770700637);
\draw[draw=none,fill=blue!80] (axis cs:23.1,12) rectangle (axis cs:24.9,80.5732484076433);
\draw[draw=none,fill=blue!80] (axis cs:24.1,12) rectangle (axis cs:25.9,82.1656050955414);
\draw[draw=none,fill=blue!80] (axis cs:25.1,12) rectangle (axis cs:26.9,82.484076433121);
\draw[draw=none,fill=blue!80] (axis cs:26.1,12) rectangle (axis cs:27.9,82.8025477707006);
\draw[draw=none,fill=blue!80] (axis cs:27.1,14) rectangle (axis cs:28.9,84.0764331210191);
\draw[draw=none,fill=blue!80] (axis cs:28.1,14) rectangle (axis cs:29.9,85.6687898089172);
\draw[draw=none,fill=blue!80] (axis cs:29.1,14) rectangle (axis cs:30.9,86.3057324840764);
\draw[draw=none,fill=blue!80] (axis cs:30.1,16) rectangle (axis cs:31.9,86.3057324840764);
\draw[draw=none,fill=blue!80] (axis cs:31.1,16) rectangle (axis cs:32.9,87.5796178343949);
\draw[draw=none,fill=blue!80] (axis cs:32.1,16) rectangle (axis cs:33.9,89.171974522293);
\draw[draw=none,fill=blue!80] (axis cs:33.1,16) rectangle (axis cs:34.9,90.1273885350319);
\draw[draw=none,fill=blue!80] (axis cs:34.1,18) rectangle (axis cs:35.9,90.4458598726115);
\draw[draw=none,fill=blue!80] (axis cs:35.1,18) rectangle (axis cs:36.9,90.7643312101911);
\draw[draw=none,fill=blue!80] (axis cs:36.1,18) rectangle (axis cs:37.9,91.7197452229299);
\draw[draw=none,fill=blue!80] (axis cs:37.1,18) rectangle (axis cs:38.9,91.7197452229299);
\draw[draw=none,fill=blue!80] (axis cs:38.1,18) rectangle (axis cs:39.9,92.0382165605096);
\draw[draw=none,fill=blue!80] (axis cs:39.1,18) rectangle (axis cs:40.9,92.9936305732484);
\draw[draw=none,fill=blue!80] (axis cs:40.1,18) rectangle (axis cs:41.9,93.312101910828);
\draw[draw=none,fill=blue!80] (axis cs:41.1,18) rectangle (axis cs:42.9,93.312101910828);
\draw[draw=none,fill=blue!80] (axis cs:42.1,18) rectangle (axis cs:43.9,93.312101910828);
\draw[draw=none,fill=blue!80] (axis cs:43.1,20) rectangle (axis cs:44.9,93.6305732484076);
\draw[draw=none,fill=blue!80] (axis cs:44.1,20) rectangle (axis cs:45.9,95.2229299363057);
\draw[draw=none,fill=blue!80] (axis cs:45.1,20) rectangle (axis cs:46.9,95.2229299363057);
\draw[draw=none,fill=blue!80] (axis cs:46.1,22) rectangle (axis cs:47.9,95.2229299363057);
\draw[draw=none,fill=blue!80] (axis cs:47.1,22) rectangle (axis cs:48.9,95.2229299363057);
\draw[draw=none,fill=blue!80] (axis cs:48.1,22) rectangle (axis cs:49.9,95.2229299363057);
\draw[draw=none,fill=blue!80] (axis cs:49.1,22) rectangle (axis cs:50.9,95.5414012738854);
\draw[draw=none,fill=blue!80] (axis cs:50.1,22) rectangle (axis cs:51.9,95.5414012738854);
\draw[draw=none,fill=blue!80] (axis cs:51.1,24) rectangle (axis cs:52.9,96.1783439490446);
\draw[draw=none,fill=blue!80] (axis cs:52.1,26) rectangle (axis cs:53.9,96.1783439490446);
\draw[draw=none,fill=blue!80] (axis cs:53.1,26) rectangle (axis cs:54.9,96.1783439490446);
\draw[draw=none,fill=blue!80] (axis cs:54.1,26) rectangle (axis cs:55.9,96.1783439490446);
\draw[draw=none,fill=blue!80] (axis cs:55.1,26) rectangle (axis cs:56.9,96.1783439490446);
\draw[draw=none,fill=blue!80] (axis cs:56.1,26) rectangle (axis cs:57.9,96.8152866242038);
\draw[draw=none,fill=blue!80] (axis cs:57.1,26) rectangle (axis cs:58.9,97.1337579617834);
\draw[draw=none,fill=blue!80] (axis cs:58.1,26) rectangle (axis cs:59.9,97.1337579617834);
\draw[draw=none,fill=blue!80] (axis cs:59.1,26) rectangle (axis cs:60.9,97.1337579617834);
\draw[draw=none,fill=blue!80] (axis cs:60.1,26) rectangle (axis cs:61.9,97.1337579617834);
\draw[draw=none,fill=blue!80] (axis cs:61.1,26) rectangle (axis cs:62.9,97.4522292993631);
\draw[draw=none,fill=blue!80] (axis cs:62.1,26) rectangle (axis cs:63.9,98.0891719745223);
\draw[draw=none,fill=blue!80] (axis cs:63.1,26) rectangle (axis cs:64.9,98.0891719745223);
\draw[draw=none,fill=blue!80] (axis cs:64.1,26) rectangle (axis cs:65.9,98.4076433121019);
\draw[draw=none,fill=blue!80] (axis cs:65.1,26) rectangle (axis cs:66.9,98.7261146496815);
\draw[draw=none,fill=blue!80] (axis cs:66.1,26) rectangle (axis cs:67.9,98.7261146496815);
\draw[draw=none,fill=blue!80] (axis cs:67.1,26) rectangle (axis cs:68.9,98.7261146496815);
\draw[draw=none,fill=blue!80] (axis cs:68.1,26) rectangle (axis cs:69.9,99.0445859872611);
\draw[draw=none,fill=blue!80] (axis cs:69.1,26) rectangle (axis cs:70.9,99.0445859872611);
\draw[draw=none,fill=blue!80] (axis cs:70.1,26) rectangle (axis cs:71.9,99.0445859872611);
\draw[draw=none,fill=blue!80] (axis cs:71.1,26) rectangle (axis cs:72.9,99.0445859872611);
\draw[draw=none,fill=blue!80] (axis cs:72.1,26) rectangle (axis cs:73.9,99.3630573248408);
\draw[draw=none,fill=blue!80] (axis cs:73.1,26) rectangle (axis cs:74.9,99.3630573248408);
\draw[draw=none,fill=blue!80] (axis cs:74.1,26) rectangle (axis cs:75.9,99.3630573248408);
\draw[draw=none,fill=blue!80] (axis cs:75.1,28) rectangle (axis cs:76.9,99.3630573248408);
\draw[draw=none,fill=blue!80] (axis cs:76.1,28) rectangle (axis cs:77.9,99.3630573248408);
\draw[draw=none,fill=blue!80] (axis cs:77.1,28) rectangle (axis cs:78.9,99.6815286624204);
\draw[draw=none,fill=blue!80] (axis cs:78.1,30) rectangle (axis cs:786.9,100);

\draw[draw=none,fill=orange!80] (axis cs:12.0,0) rectangle (axis cs:14.99,2);
\draw[draw=none,fill=orange!80] (axis cs:14.0,0) rectangle (axis cs:15.99,4);
\draw[draw=none,fill=orange!80] (axis cs:15.0,0) rectangle (axis cs:16.99,4);
\draw[draw=none,fill=orange!80] (axis cs:16.0,0) rectangle (axis cs:17.99,6);
\draw[draw=none,fill=orange!80] (axis cs:17.0,0) rectangle (axis cs:18.99,6);
\draw[draw=none,fill=orange!80] (axis cs:18.0,0) rectangle (axis cs:19.99,6);
\draw[draw=none,fill=orange!80] (axis cs:19.0,0) rectangle (axis cs:20.99,6);
\draw[draw=none,fill=orange!80] (axis cs:20.0,0) rectangle (axis cs:21.99,8);
\draw[draw=none,fill=orange!80] (axis cs:21.0,0) rectangle (axis cs:22.99,10);
\draw[draw=none,fill=orange!80] (axis cs:22.0,0) rectangle (axis cs:23.99,10);
\draw[draw=none,fill=orange!80] (axis cs:23.0,0) rectangle (axis cs:24.99,12);
\draw[draw=none,fill=orange!80] (axis cs:24.0,0) rectangle (axis cs:25.99,12);
\draw[draw=none,fill=orange!80] (axis cs:25.0,0) rectangle (axis cs:26.99,12);
\draw[draw=none,fill=orange!80] (axis cs:26.0,0) rectangle (axis cs:27.99,12);
\draw[draw=none,fill=orange!80] (axis cs:27.0,0) rectangle (axis cs:28.99,14);
\draw[draw=none,fill=orange!80] (axis cs:28.0,0) rectangle (axis cs:29.99,14);
\draw[draw=none,fill=orange!80] (axis cs:29.0,0) rectangle (axis cs:30.99,14);
\draw[draw=none,fill=orange!80] (axis cs:30.0,0) rectangle (axis cs:31.99,16);
\draw[draw=none,fill=orange!80] (axis cs:31.0,0) rectangle (axis cs:32.99,16);
\draw[draw=none,fill=orange!80] (axis cs:32.0,0) rectangle (axis cs:33.99,16);
\draw[draw=none,fill=orange!80] (axis cs:33.0,0) rectangle (axis cs:34.99,16);
\draw[draw=none,fill=orange!80] (axis cs:34.0,0) rectangle (axis cs:43.99,18);
\draw[draw=none,fill=orange!80] (axis cs:43.0,0) rectangle (axis cs:44.99,20);
\draw[draw=none,fill=orange!80] (axis cs:44.0,0) rectangle (axis cs:45.99,20);
\draw[draw=none,fill=orange!80] (axis cs:45.0,0) rectangle (axis cs:46.99,20);
\draw[draw=none,fill=orange!80] (axis cs:46.0,0) rectangle (axis cs:47.99,22);
\draw[draw=none,fill=orange!80] (axis cs:47.0,0) rectangle (axis cs:48.99,22);
\draw[draw=none,fill=orange!80] (axis cs:48.0,0) rectangle (axis cs:49.99,22);
\draw[draw=none,fill=orange!80] (axis cs:49.0,0) rectangle (axis cs:50.99,22);
\draw[draw=none,fill=orange!80] (axis cs:50.0,0) rectangle (axis cs:51.99,22);
\draw[draw=none,fill=orange!80] (axis cs:51.0,0) rectangle (axis cs:52.99,24);
\draw[draw=none,fill=orange!80] (axis cs:52.0,0) rectangle (axis cs:75.99,26);
\draw[draw=none,fill=orange!80] (axis cs:75.0,0) rectangle (axis cs:76.99,28);
\draw[draw=none,fill=orange!80] (axis cs:76.0,0) rectangle (axis cs:77.99,28);
\draw[draw=none,fill=orange!80] (axis cs:77.0,0) rectangle (axis cs:78.99,28);
\draw[draw=none,fill=orange!80] (axis cs:78.0,0) rectangle (axis cs:79.99,30);
\draw[draw=none,fill=orange!80] (axis cs:79.0,0) rectangle (axis cs:80.99,30);
\draw[draw=none,fill=orange!80] (axis cs:80.0,0) rectangle (axis cs:81.99,30);
\draw[draw=none,fill=orange!80] (axis cs:81.0,0) rectangle (axis cs:82.99,30);
\draw[draw=none,fill=orange!80] (axis cs:82.0,0) rectangle (axis cs:83.99,30);
\draw[draw=none,fill=orange!80] (axis cs:83.0,0) rectangle (axis cs:84.99,30);
\draw[draw=none,fill=orange!80] (axis cs:84.0,0) rectangle (axis cs:85.99,30);
\draw[draw=none,fill=orange!80] (axis cs:85.0,0) rectangle (axis cs:86.99,32);
\draw[draw=none,fill=orange!80] (axis cs:86.0,0) rectangle (axis cs:87.99,32);
\draw[draw=none,fill=orange!80] (axis cs:87.0,0) rectangle (axis cs:88.99,32);
\draw[draw=none,fill=orange!80] (axis cs:88.0,0) rectangle (axis cs:89.99,32);
\draw[draw=none,fill=orange!80] (axis cs:89.0,0) rectangle (axis cs:90.99,32);
\draw[draw=none,fill=orange!80] (axis cs:90.0,0) rectangle (axis cs:91.99,32);
\draw[draw=none,fill=orange!80] (axis cs:91.0,0) rectangle (axis cs:116.99,36);
\draw[draw=none,fill=orange!80] (axis cs:116.0,0) rectangle (axis cs:130.99,38);
\draw[draw=none,fill=orange!80] (axis cs:130.0,0) rectangle (axis cs:138.99,40);
\draw[draw=none,fill=orange!80] (axis cs:138.0,0) rectangle (axis cs:139.99,42);
\draw[draw=none,fill=orange!80] (axis cs:139.0,0) rectangle (axis cs:140.99,42);
\draw[draw=none,fill=orange!80] (axis cs:140.0,0) rectangle (axis cs:141.99,42);
\draw[draw=none,fill=orange!80] (axis cs:141.0,0) rectangle (axis cs:142.99,42);
\draw[draw=none,fill=orange!80] (axis cs:142.0,0) rectangle (axis cs:143.99,42);
\draw[draw=none,fill=orange!80] (axis cs:143.0,0) rectangle (axis cs:144.99,42);
\draw[draw=none,fill=orange!80] (axis cs:144.0,0) rectangle (axis cs:145.99,42);
\draw[draw=none,fill=orange!80] (axis cs:145.0,0) rectangle (axis cs:170.99,44);
\draw[draw=none,fill=orange!80] (axis cs:170.0,0) rectangle (axis cs:191.99,46);
\draw[draw=none,fill=orange!80] (axis cs:191.0,0) rectangle (axis cs:192.99,48);
\draw[draw=none,fill=orange!80] (axis cs:192.0,0) rectangle (axis cs:193.99,48);
\draw[draw=none,fill=orange!80] (axis cs:193.0,0) rectangle (axis cs:194.99,48);
\draw[draw=none,fill=orange!80] (axis cs:194.0,0) rectangle (axis cs:195.99,48);
\draw[draw=none,fill=orange!80] (axis cs:195.0,0) rectangle (axis cs:196.99,48);
\draw[draw=none,fill=orange!80] (axis cs:196.0,0) rectangle (axis cs:197.99,48);
\draw[draw=none,fill=orange!80] (axis cs:197.0,0) rectangle (axis cs:211.99,50);
\draw[draw=none,fill=orange!80] (axis cs:211.0,0) rectangle (axis cs:212.99,52);
\draw[draw=none,fill=orange!80] (axis cs:212.0,0) rectangle (axis cs:213.99,52);
\draw[draw=none,fill=orange!80] (axis cs:213.0,0) rectangle (axis cs:214.99,52);
\draw[draw=none,fill=orange!80] (axis cs:214.0,0) rectangle (axis cs:215.99,56);
\draw[draw=none,fill=orange!80] (axis cs:215.0,0) rectangle (axis cs:216.99,56);
\draw[draw=none,fill=orange!80] (axis cs:216.0,0) rectangle (axis cs:217.99,56);
\draw[draw=none,fill=orange!80] (axis cs:217.0,0) rectangle (axis cs:218.99,56);
\draw[draw=none,fill=orange!80] (axis cs:218.0,0) rectangle (axis cs:219.99,56);
\draw[draw=none,fill=orange!80] (axis cs:219.0,0) rectangle (axis cs:220.99,56);
\draw[draw=none,fill=orange!80] (axis cs:220.0,0) rectangle (axis cs:221.99,56);
\draw[draw=none,fill=orange!80] (axis cs:221.0,0) rectangle (axis cs:222.99,56);
\draw[draw=none,fill=orange!80] (axis cs:222.0,0) rectangle (axis cs:223.99,56);
\draw[draw=none,fill=orange!80] (axis cs:223.0,0) rectangle (axis cs:224.99,56);
\draw[draw=none,fill=orange!80] (axis cs:224.0,0) rectangle (axis cs:236.99,58);
\draw[draw=none,fill=orange!80] (axis cs:236.0,0) rectangle (axis cs:247.99,60);
\draw[draw=none,fill=orange!80] (axis cs:247.0,0) rectangle (axis cs:288.99,64);
\draw[draw=none,fill=orange!80] (axis cs:288.0,0) rectangle (axis cs:289.99,68);
\draw[draw=none,fill=orange!80] (axis cs:289.0,0) rectangle (axis cs:290.99,68);
\draw[draw=none,fill=orange!80] (axis cs:290.0,0) rectangle (axis cs:304.99,70);
\draw[draw=none,fill=orange!80] (axis cs:304.0,0) rectangle (axis cs:305.99,72);
\draw[draw=none,fill=orange!80] (axis cs:305.0,0) rectangle (axis cs:306.99,72);
\draw[draw=none,fill=orange!80] (axis cs:306.0,0) rectangle (axis cs:307.99,72);
\draw[draw=none,fill=orange!80] (axis cs:307.0,0) rectangle (axis cs:308.99,72);
\draw[draw=none,fill=orange!80] (axis cs:308.0,0) rectangle (axis cs:309.99,74);
\draw[draw=none,fill=orange!80] (axis cs:309.0,0) rectangle (axis cs:310.99,74);
\draw[draw=none,fill=orange!80] (axis cs:310.0,0) rectangle (axis cs:311.99,74);
\draw[draw=none,fill=orange!80] (axis cs:311.0,0) rectangle (axis cs:366.99,76);
\draw[draw=none,fill=orange!80] (axis cs:366.0,0) rectangle (axis cs:394.99,78);
\draw[draw=none,fill=orange!80] (axis cs:394.0,0) rectangle (axis cs:415.99,80);
\draw[draw=none,fill=orange!80] (axis cs:415.0,0) rectangle (axis cs:416.99,82);
\draw[draw=none,fill=orange!80] (axis cs:416.0,0) rectangle (axis cs:440.99,84);
\draw[draw=none,fill=orange!80] (axis cs:440.0,0) rectangle (axis cs:490.99,86);
\draw[draw=none,fill=orange!80] (axis cs:490.0,0) rectangle (axis cs:513.99,88);
\draw[draw=none,fill=orange!80] (axis cs:513.0,0) rectangle (axis cs:522.99,90);
\draw[draw=none,fill=orange!80] (axis cs:522.0,0) rectangle (axis cs:554.99,92);
\draw[draw=none,fill=orange!80] (axis cs:554.0,0) rectangle (axis cs:583.99,94);
\draw[draw=none,fill=orange!80] (axis cs:583.0,0) rectangle (axis cs:597.99,96);
\draw[draw=none,fill=orange!80] (axis cs:597.0,0) rectangle (axis cs:786.99,98);
\end{axis}

\end{tikzpicture}\\
%    \caption{Sample efficiency of Tidyparse at varying Levenshtein radii. After drawing up to $\sim10^5$ samples without replacement we can usually retrieve the human repair for almost all repairs fewer than four edits.}\label{fig:sample_efficiency}
\resizebox{.35\textwidth}{!}{% This file was created with tikzplotlib v0.10.1.
\begin{tikzpicture}
\begin{axis}[
  legend cell align={left},
  legend style={fill opacity=0.8, draw opacity=1, text opacity=1, legend columns=-1, legend pos=north east},
  tick align=outside,
  tick pos=left,
  axis lines*=left,
  title={\textbf{Average throughput}},
  x grid style={darkgray176},
  xlabel={Lexical tokens},
  xmin=-0.95, xmax=19.95,
  xtick style={color=black},
  xtick={0,1,2,3,4,5,6,7,8,9,10,11,12,13,14,15,16,17,18,19},
  xticklabels={4, , 8, , 12, , 16, , 20, , 24, , 28, , 32, , 36, , 40,},
  y grid style={darkgray176},
  ylabel={Total repairs discovered},
  ymin=0, ymax=200000,
  ymode=log,
  legend pos=north east,
  ytick style={color=black}
]
\addlegendimage{empty legend}
\addlegendentry{$\Delta=$}
\addlegendentry{\footnotesize{$1,$}}
\addlegendentry{\footnotesize{$2,$}}
\addlegendentry{\footnotesize{$3,$}}
\addlegendentry{\footnotesize{$4$}}
\addplot [semithick, green, mark=*, mark size=3, mark options={solid}]
table {%
  0 7.14285714285714
  1 12.1666666666667
  2 9.27272727272727
  3 10.6041666666667
  4 7.22857142857143
  5 6.4
  6 10.3956043956044
  7 7.86075949367089
  8 9.81818181818182
  9 8.28089887640449
  10 8.08695652173913
  11 10.0430107526882
  12 6.90476190476191
  13 6.94029850746269
  14 8.74698795180723
  15 7.70422535211268
  16 12.1414141414141
  17 4.1
};
\addplot [semithick, blue, mark=*, mark size=3, mark options={solid}]
table {%
0 46.625
1 135.653846153846
2 175.676923076923
3 79.0090909090909
4 87.6589147286822
5 69.9683544303797
6 83.9896373056995
7 85.9777777777778
8 73.3684210526316
9 67.7715517241379
10 92.1878172588832
11 186.568527918782
12 116.195945945946
13 87.8901734104046
14 99.6514285714286
15 154.44
16 123.658064516129
17 36.6111111111111
};
\addplot [semithick, orange, mark=*, mark size=3, mark options={solid}]
table {%
0 1506.8
1 2076.57142857143
2 5194.91304347826
3 413.939393939394
4 3749.34693877551
5 1130.83076923077
6 431.141025641026
7 601.172839506173
8 2135.26666666667
9 1233.85106382979
10 960.263888888889
11 932.225352112676
12 612.833333333333
13 880.2625
14 965.863636363636
15 725.139784946237
16 1199.05882352941
17 114.5
};
\addplot [semithick, red, mark=*, mark size=3, mark options={solid}]
table {%
0 nan
1 73461
2 24965
3 4220.4
4 35145.8181818182
5 2613.28571428571
6 68402.25
7 3300.5
8 10010
9 24536.5217391304
10 17497.6315789474
11 8883.58333333333
12 1948.23076923077
13 4175.27272727273
14 5145.45454545455
15 27785.8095238095
16 2468
17 4027.25
};
\end{axis}

\end{tikzpicture}}\hspace{1cm}
\resizebox{.35\textwidth}{!}{% This file was created with matplot2tikz v0.3.3.
\tikzset{mark size=0.5}

\begin{tikzpicture}

\begin{axis}[
log basis y={10},
tick align=outside,
tick pos=left,
x grid style={darkdarkgray176},
xlabel={Snippet length},
xmin=0, xmax=80,
title={\textbf{Latency across repair size}},
xtick style={color=black},
y grid style={darkdarkgray176},
log basis y={10},
axis lines*=left,
ylabel={Repair latency (ms)},
ymin=6.99556128596625, ymax=147694.796423657,
ymode=log,
mark size=1.2pt,
ytick style={color=black},
yticklabels={
  \(\displaystyle {10^{0}}\),
  \(\displaystyle {10^{1}}\),
  \(\displaystyle {10^{2}}\),
  \(\displaystyle {10^{3}}\),
  \(\displaystyle {10^{4}}\),
  \(\displaystyle {10^{5}}\),
}
]
\addplot [draw=darkgray, fill=darkgray, mark=*, only marks]
table{%
x  y
9 1549
12 116
16 1976
18 288
39 1303
63 2641
11 56
73 4245
45 1416
34 706
27 1489
37 918
29 542
29 389
15 142
39 968
29 434
45 1260
19 879
61 2273
7 120
23 924
51 1570
49 1441
61 2300
42 1175
68 3944
60 2896
72 4049
53 2217
8 113
13 265
62 17742
33 3818
47 9889
41 1394
60 4760
77 4309
72 4271
27 484
75 3934
16 96
32 677
21 339
49 1836
45 1435
58 3188
18 120
30 816
15 463
19 714
49 1691
43 1476
41 831
14 192
21 182
19 10979
78 4436
52 1445
23 652
21 1557
33 974
17 118
30 933
30 374
24 541
66 25769
50 1829
47 1638
46 1360
35 4446
51 1482
59 2163
13 150
41 947
33 1189
26 312
17 182
17 178
20 253
27 1030
34 20964
26 315
8 25
20 152
53 2119
59 2414
41 26227
19 259
63 2519
63 10671
41 1815
32 423
65 4079
74 4863
36 986
9 37
46 1446
78 5210
61 4684
69 4292
14 79
45 1274
14 232
33 540
38 953
16 103
31 908
69 31228
33 709
31 13133
15 253
48 2581
66 19478
36 629
33 495
30 396
40 941
71 4825
10 983
28 426
69 3913
25 994
44 2016
7 21
50 2080
53 2054
29 382
34 763
63 2470
34 955
9 60
16 119
22 10030
40 1040
22 512
77 5203
30 616
25 242
39 958
69 5508
59 2896
21 1549
51 40442
10 134
30 722
54 29715
36 990
40 988
52 1745
10 340
48 1874
29 574
52 2219
39 794
34 732
33 516
23 470
50 15557
30 641
16 496
23 328
22 459
33 742
31 765
7 102
48 2084
45 1484
25 258
30 691
71 3900
27 548
26 353
66 3080
63 4977
40 1250
18 289
14 1780
27 2357
29 623
40 1008
22 213
64 4137
24 398
36 862
46 1424
13 113
68 4167
10 54
53 6662
40 1054
9 66
26 611
68 3685
50 4591
19 798
30 485
52 1583
12 74
34 644
77 6474
76 85265
21 429
11 170
23 287
25 496
34 1099
6 21
28 882
13 149
25 397
35 1044
23 4948
25 1657
15 243
36 848
23 237
27 1713
19 217
25 520
37 919
24 965
39 1717
15 792
49 1469
77 5034
38 3093
40 1097
51 1957
6 63
10 135
17 597
39 753
63 3249
25 447
41 2468
56 2072
74 11972
49 1742
25 494
34 93928
67 6025
64 3035
13 79
56 1714
67 3955
66 3282
63 3261
16 105
5 85
26 545
28 413
8 90
13 106
37 805
69 5137
28 369
29 673
10 44
42 1239
67 4648
27 792
51 7387
20 1096
30 585
37 790
30 462
28 986
9 44
50 1843
70 3327
52 1525
11 79
30 515
16 210
16 196
27 554
11 184
53 1947
8 67
37 929
26 971
26 367
33 648
32 616
14 134
47 1524
5 11
24 913
32 418
75 12513
35 1148
59 2608
11 157
50 2808
15 191
14 184
41 1019
10 157
71 4695
69 4455
34 540
69 4740
51 3258
44 1029
51 1498
50 1420
46 1850
40 1080
17 149
59 2862
21 14398
37 681
37 2124
11 222
23 463
51 2020
33 530
42 17922
4 13
39 698
52 2063
21 328
78 7145
28 429
19 250
7 97
43 1210
59 38172
22 362
42 2745
47 1206
28 379
62 2811
55 2179
75 4071
45 1063
37 1899
12 112
40 745
21 884
9 40
70 3981
36 888
28 392
56 5825
27 393
23 223
24 10408
18 174
23 2177
53 1662
39 1384
41 989
71 3464
60 2194
35 977
67 4303
50 9881
21 379
55 2905
58 2756
45 1769
20 609
43 38706
40 3024
32 664
23 217
18 135
65 5228
67 3550
51 1385
14 107
19 212
33 1746
75 3857
72 4455
44 1118
68 4834
8 55
39 29045
10 43
22 336
48 7343
72 19347
24 269
49 2791
19 785
39 1120
49 1322
21 760
28 521
63 62239
49 1629
23 6078
37 27981
41 1375
51 2274
5 65
17 191
62 3688
27 1016
41 1326
34 2739
33 2785
35 916
63 2554
40 1149
11 56
72 4190
18 145
62 3069
16 379
65 2780
46 1193
50 1621
76 4978
45 1264
16 382
23 655
20 292
27 705
58 2380
26 539
42 7325
15 264
55 2162
76 25267
49 2954
46 1565
55 2499
35 568
24 303
36 783
36 725
29 392
53 6744
19 185
52 1495
11 149
16 583
26 471
76 5789
51 1362
48 1514
54 2037
23 306
8 145
31 623
12 105
70 3810
21 167
11 186
43 11621
15 897
77 30152
8 219
14 2669
15 198
29 907
11 46
38 708
31 894
41 1293
30 24580
60 31750
64 3403
26 527
27 783
21 189
34 980
15 138
9 64
9 32
8 90
59 2109
15 94
33 695
17 14201
36 733
75 5222
36 1902
18 535
67 4110
29 950
35 1798
42 1463
33 766
41 1037
33 503
30 1323
21 252
27 757
32 916
37 1450
54 2323
49 1980
44 9744
26 226
40 11847
22 919
21 4868
42 1916
27 326
53 2114
50 1959
13 178
24 2361
8 92
19 298
29 454
56 3157
27 466
55 2120
71 4493
12 97
25 544
39 1089
38 1161
36 580
53 2653
16 203
10 3282
12 57
30 644
55 1814
34 17457
14 70
25 365
19 265
38 1074
16 203
23 626
46 1762
20 538
13 385
53 2187
37 1069
33 702
9 104
62 4072
32 632
47 1670
50 2049
65 3868
38 810
16 262
62 2518
51 1981
11 109
12 2936
28 520
76 11739
71 3836
49 2243
};
\addplot [draw=darkgray, fill=darkgray, mark=*, only marks]
table{%
x  y
37 967
13 576
31 725
13 82
62 3570
27 286
11 65
55 1667
29 342
18 467
27 300
16 272
57 2199
75 4119
20 205
17 366
16 125
63 4737
45 1024
70 3483
26 608
21 563
30 1598
38 1097
12 62
33 613
61 2243
19 280
27 902
35 556
63 3421
26 714
18 110
47 5579
47 1203
30 955
55 2276
36 1207
46 1206
58 2320
41 855
73 3738
30 646
73 4086
34 805
26 254
48 1192
33 562
14 286
50 1452
64 2511
31 576
28 290
73 4351
45 3090
28 614
36 667
52 2867
77 5575
31 489
50 1848
35 820
42 1054
28 397
76 4091
37 736
64 2432
52 1436
10 169
41 775
9 54
67 3335
27 451
28 303
31 427
57 1809
42 1335
18 216
45 978
39 846
12 157
42 872
20 163
17 398
19 2462
23 198
19 367
26 310
69 3317
15 249
77 4157
42 856
17 121
20 758
36 558
18 149
22 496
62 2332
24 569
29 376
17 939
40 993
24 297
8 70
18 285
57 2095
23 670
29 336
25 1004
30 396
72 3449
38 1492
58 2403
54 2802
43 1186
20 423
26 1536
15 76
5 19
78 5718
39 670
26 281
43 874
44 1863
13 67
17 471
43 1152
24 266
43 956
29 363
48 1478
39 992
16 153
36 693
57 1936
14 390
66 5587
52 2022
31 1252
30 384
20 248
20 150
42 1961
27 464
11 106
21 581
11 99
33 720
29 536
56 6622
61 2275
12 99
78 4706
9 158
78 5312
64 3852
43 935
36 573
33 1219
21 1063
13 59
19 243
23 199
12 185
29 1062
59 2021
25 290
34 585
5 19
39 974
20 186
25 260
74 4492
25 468
46 1182
28 310
44 913
38 735
23 361
31 388
25 252
77 4451
53 3401
10 67
10 110
55 2453
61 2455
71 3342
43 921
63 3298
78 4901
60 13008
30 599
68 2986
32 559
13 310
22 216
20 368
51 2238
29 691
37 608
28 320
35 797
53 3644
23 263
58 2283
24 1170
50 4202
11 130
60 2883
34 2120
71 3378
22 180
36 858
19 217
5 47
18 384
22 392
26 1020
30 430
9 75
62 2291
8 143
39 753
73 3607
57 2080
35 2589
47 1105
20 591
38 1445
38 718
31 728
60 11098
34 496
35 1681
41 781
14 208
65 5798
61 2345
18 116
17 124
21 178
8 93
39 736
49 1481
46 1083
25 554
24 347
63 2938
16 180
15 202
43 1072
11 54
22 216
36 572
27 451
18 292
27 333
37 923
54 1697
26 745
18 223
17 564
48 1236
13 102
24 313
47 1151
24 418
73 4691
22 186
50 1315
72 3757
32 846
29 851
43 1050
34 480
31 419
30 762
17 394
33 557
30 1017
32 422
20 162
16 94
33 690
70 4054
27 626
44 2331
52 1935
22 749
71 3332
20 1017
34 2256
14 107
29 559
18 146
38 761
39 1008
18 226
48 2323
60 2418
38 653
26 273
27 336
33 455
32 416
13 214
77 4543
49 2068
20 355
61 2236
55 4802
11 163
20 179
65 2701
48 6149
77 4137
51 4601
55 1865
26 273
35 4314
67 2978
25 303
24 699
76 4592
23 236
25 265
7 36
34 499
62 2235
22 344
27 787
70 3060
30 2236
20 457
36 1361
72 5525
23 1335
13 373
16 263
30 1738
33 901
9 61
71 3235
40 775
33 638
30 366
16 401
33 3066
40 807
69 3141
9 60
28 398
59 2206
72 5128
36 1173
38 755
30 396
59 9951
73 4086
18 244
19 287
31 629
58 1929
48 1854
29 543
15 574
76 4735
37 3785
28 3065
33 3925
20 315
13 128
21 779
35 586
27 461
38 728
49 2232
15 578
24 237
24 234
39 831
17 264
34 499
14 560
40 799
31 403
24 227
21 418
50 1515
61 2860
48 1200
62 2411
75 4749
35 511
39 687
51 2613
15 99
40 784
4 21
24 2146
37 665
46 1132
24 275
22 396
31 403
33 456
31 615
23 324
16 216
54 1671
34 1328
23 467
56 1767
9 181
36 560
20 191
63 2774
27 371
8 215
27 346
33 3668
41 6073
16 109
56 2557
42 854
29 923
66 7498
28 328
50 2110
17 183
73 4756
39 907
47 1142
74 4066
17 114
18 134
46 1133
24 243
45 1212
35 1622
19 718
50 1286
74 3944
52 1596
22 417
32 1180
24 807
20 175
45 1019
23 203
25 255
65 2716
37 612
40 799
9 75
36 668
16 181
39 3541
16 211
49 2661
25 495
59 1979
37 595
16 403
16 280
53 2048
62 2206
28 308
54 1583
47 1330
21 195
15 394
55 2754
37 1058
32 818
14 95
41 811
15 91
34 561
15 517
13 298
48 1138
28 458
44 1152
24 223
30 376
79 4556
35 814
43 2771
53 1486
21 202
42 848
45 995
18 115
12 170
37 616
19 546
57 1925
26 322
21 173
60 2626
51 1515
27 2046
19 366
18 332
71 3346
31 656
73 3772
44 944
70 3714
43 1159
68 3472
19 336
55 1778
24 251
45 1963
18 154
58 2220
10 220
45 1072
43 1188
33 517
40 876
16 646
48 1214
22 561
37 930
63 4137
33 1414
13 397
30 469
25 306
15 130
52 1676
51 1391
42 1326
29 596
65 2703
60 3376
71 3470
18 109
40 737
35 827
30 444
31 495
59 2695
79 4538
37 619
31 538
20 376
57 2979
23 410
29 908
28 376
36 955
13 225
43 1404
29 472
35 604
23 220
16 168
17 151
13 151
13 77
12 396
57 1895
41 860
57 2269
75 3797
22 176
21 658
27 286
19 172
54 1513
10 138
17 319
44 987
27 286
25 473
17 119
64 2514
28 447
42 1606
53 1810
14 160
33 1491
53 2038
58 2610
21 236
36 578
27 473
24 275
74 3879
79 5334
67 3619
47 2079
41 833
15 254
53 1662
26 265
20 142
22 415
52 1511
11 74
31 505
50 1314
9 219
20 178
53 1688
55 1727
25 714
21 380
45 1511
73 3864
20 169
14 178
35 2191
63 2616
29 603
49 2332
47 1248
26 793
7 19
28 335
18 416
36 1161
20 182
24 714
14 88
49 1390
37 637
66 3852
29 337
12 64
32 468
34 1461
17 104
42 912
17 406
28 464
20 661
34 1508
32 501
41 785
74 3767
17 125
40 1750
30 468
11 256
53 1588
35 716
28 523
20 132
38 1072
62 2652
45 1865
42 888
71 3634
55 1915
66 2838
39 826
66 3678
62 2941
33 862
32 1045
20 463
26 407
14 88
51 1526
32 1004
32 810
27 396
5 34
15 458
31 530
11 116
36 1260
21 204
46 1272
64 3805
53 1946
56 2260
44 981
29 394
34 764
29 503
24 267
20 357
70 5314
21 212
19 271
25 1602
36 668
54 1692
41 879
27 314
29 469
43 1248
26 288
78 7398
46 1522
42 1005
19 162
20 659
16 352
29 369
41 1444
35 584
29 493
72 4540
36 565
21 948
16 111
34 1065
59 2490
22 253
16 195
51 1527
49 1373
53 1549
36 677
64 2722
55 1864
25 269
24 230
37 748
32 447
16 151
15 122
16 872
65 3000
17 100
12 296
29 466
32 674
29 337
15 230
45 1084
35 960
79 5158
34 563
21 394
33 633
24 330
27 295
27 329
36 565
34 1367
41 802
71 4002
35 548
47 1487
18 198
29 448
57 3080
56 2030
39 706
37 669
11 104
41 877
27 318
36 1874
22 231
42 1185
8 40
42 894
32 855
64 3359
33 603
41 823
22 306
29 448
15 94
34 824
6 15
41 860
12 153
4 28
35 906
72 3895
64 2509
16 427
15 113
31 737
18 138
42 1059
11 63
12 149
5 37
65 2822
17 278
19 303
17 335
18 175
17 562
22 605
23 240
35 3819
26 259
45 997
52 1586
16 429
34 490
29 1283
22 555
33 991
34 520
70 4662
13 213
13 97
52 1514
22 210
50 1817
29 383
33 675
18 422
18 117
53 2349
24 2137
39 1467
18 157
78 5258
45 1250
29 342
10 82
28 322
24 292
71 4662
16 211
27 294
15 271
42 1225
33 532
57 1970
34 566
36 1764
34 592
32 499
9 81
72 5270
31 1433
31 725
61 2268
50 2300
19 925
32 446
8 73
71 4323
25 763
18 310
31 442
54 1689
22 248
17 298
19 148
39 715
46 1237
24 378
33 620
55 1928
33 448
54 2785
28 2556
16 114
17 143
29 414
6 15
25 820
76 4052
35 846
27 347
44 1173
73 3785
34 486
15 148
69 4745
31 1132
12 130
29 553
77 4168
47 1263
54 3108
40 2104
25 370
32 1279
52 12972
36 1281
29 417
20 184
38 1157
42 889
11 164
46 1070
47 2268
5 46
13 530
9 55
30 898
45 1442
20 179
34 2244
11 49
14 85
5 29
12 48
26 402
25 586
28 415
38 708
40 1013
24 470
26 771
28 380
36 1000
19 609
48 1182
52 1710
26 289
36 605
13 390
23 376
28 3022
20 985
26 405
42 853
20 159
11 89
15 99
60 2482
38 870
35 743
20 890
19 1640
14 92
25 295
25 913
5 41
32 472
25 233
22 446
15 232
68 3026
19 514
37 2185
78 4881
39 773
26 295
41 909
30 432
12 304
13 88
32 454
76 4302
26 819
74 6097
35 603
48 4513
21 378
40 834
29 380
66 2766
71 3677
23 1391
24 261
38 747
8 146
26 1750
39 732
22 588
9 155
33 485
23 1342
15 459
44 1375
76 4937
44 1381
15 372
34 1294
22 195
29 333
49 6540
18 163
57 2083
44 963
34 504
19 229
59 2052
34 624
35 551
59 2096
23 195
46 1312
74 3939
29 628
26 633
27 306
20 651
18 122
50 1552
43 2398
27 559
12 49
34 648
65 2751
27 443
33 518
56 1772
12 56
37 1870
36 600
29 2453
19 321
22 622
29 1666
62 2886
43 976
15 110
54 2059
53 1487
16 124
24 383
37 602
35 531
34 716
58 2505
14 164
22 323
35 664
23 400
78 5086
};
\addplot [draw=darkgray, fill=darkgray, mark=*, only marks]
table{%
x  y
19 524
42 1166
45 3390
69 18648
51 3254
32 10954
36 680
76 7230
40 2763
64 3123
35 970
75 26775
60 4154
37 3369
28 406
48 36005
70 5503
54 4443
43 4796
35 4535
43 2163
11 255
30 1756
31 701
61 13931
37 2268
76 30644
44 925
43 2764
15 1076
50 6609
60 12517
30 2429
29 1154
34 694
9 655
29 1101
31 3358
33 620
12 3279
32 5493
65 15497
26 507
29 1146
65 5376
24 858
36 1494
40 1621
26 787
71 4837
36 1460
39 1047
28 485
29 1114
25 3603
67 3174
};
\end{axis}

\end{tikzpicture}
}
\end{figure}
\end{frame}

\begin{frame}[fragile]{Outcomes in the syntax repair pipeline}
\vspace{-1cm}
\begin{figure}
\resizebox{.83\textwidth}{!}{% This file was created with matplot2tikz v0.3.3.
\begin{tikzpicture}

\definecolor{darkgray176}{RGB}{176,176,176}
\definecolor{teal9147104}{RGB}{9,147,104}

\begin{axis}[
hide x axis,
hide y axis,
tick align=outside,
tick pos=left,
x grid style={darkgray176},
xmin=-0.65, xmax=2.10714285714286,
xtick style={color=black},
y grid style={darkgray176},
ymin=-1.22685803692431, ymax=1.22218305326489,
ytick style={color=black}
]
\path [draw=teal9147104, fill=teal9147104]
(axis cs:-0.25,0.315857142857143)
--(axis cs:-0.125,0.315857142857143)
.. controls (axis cs:-0.03125,0.315857142857143) and (axis cs:0.03125,0.315857142857143) .. (axis cs:0.125,0.315857142857143)
--(axis cs:0.25,0.315857142857143)
--(axis cs:0.4,0.315857142857143)
.. controls (axis cs:0.4265114773,0.315857142857143) and (axis cs:0.4519642327,0.326400019357143) .. (axis cs:0.4707106781,0.345146464757143)
.. controls (axis cs:0.4894571235,0.363892910157143) and (axis cs:0.5,0.389345665557143) .. (axis cs:0.5,0.415857142857143)
--(axis cs:0.5,0.565857142857143)
--(axis cs:0.47,0.565857142857143)
--(axis cs:0.596714285714286,0.672183053264892)
--(axis cs:0.723428571428572,0.565857142857143)
--(axis cs:0.693428571428572,0.565857142857143)
--(axis cs:0.693428571428572,0.415857142857143)
.. controls (axis cs:0.693428571428572,0.338064893751143) and (axis cs:0.662492759527143,0.263379237191714) .. (axis cs:0.607485332596286,0.208371810260857)
.. controls (axis cs:0.552477905665429,0.15336438333) and (axis cs:0.477792249106,0.122428571428571) .. (axis cs:0.4,0.122428571428571)
--(axis cs:0.693428571428572,0.122428571428571)
--(axis cs:0.843428571428572,0.122428571428571)
.. controls (axis cs:0.869940048728572,0.122428571428571) and (axis cs:0.895392804128571,0.132971447928571) .. (axis cs:0.914139249528572,0.151717893328571)
.. controls (axis cs:0.932885694928572,0.170464338728571) and (axis cs:0.943428571428572,0.195917094128571) .. (axis cs:0.943428571428572,0.222428571428571)
--(axis cs:0.943428571428572,0.565857142857143)
--(axis cs:0.913428571428571,0.565857142857143)
--(axis cs:0.992571428571429,0.63226588509603)
--(axis cs:1.07171428571429,0.565857142857143)
--(axis cs:1.04171428571429,0.565857142857143)
--(axis cs:1.04171428571429,0.222428571428571)
.. controls (axis cs:1.04171428571429,0.169860099296571) and (axis cs:1.02080926774,0.119390921446286) .. (axis cs:0.983637744575429,0.0822193982817143)
.. controls (axis cs:0.946466221410857,0.0450478751171429) and (axis cs:0.895997043560572,0.0241428571428572) .. (axis cs:0.843428571428572,0.0241428571428572)
--(axis cs:1.04171428571429,0.0241428571428571)
--(axis cs:1.19171428571429,0.0241428571428571)
.. controls (axis cs:1.21822576301429,0.0241428571428571) and (axis cs:1.24367851841429,0.0346857336428571) .. (axis cs:1.26242496381429,0.0534321790428571)
.. controls (axis cs:1.28117140921429,0.0721786244428572) and (axis cs:1.29171428571429,0.0976313798428571) .. (axis cs:1.29171428571429,0.124142857142857)
--(axis cs:1.29171428571429,0.565857142857143)
--(axis cs:1.26171428571429,0.565857142857143)
--(axis cs:1.30557142857143,0.602657655253061)
--(axis cs:1.34942857142857,0.565857142857143)
--(axis cs:1.31942857142857,0.565857142857143)
--(axis cs:1.31942857142857,0.124142857142857)
.. controls (axis cs:1.31942857142857,0.0902839132768571) and (axis cs:1.30596381201286,0.0577771085231429) .. (axis cs:1.28202192317343,0.0338352196837143)
.. controls (axis cs:1.258080034334,0.00989333084428572) and (axis cs:1.22557322958029,-0.00357142857142857) .. (axis cs:1.19171428571429,-0.00357142857142857)
--(axis cs:1.31942857142857,-0.00357142857142857)
--(axis cs:1.46942857142857,-0.00357142857142857)
.. controls (axis cs:1.49594004872857,-0.00357142857142857) and (axis cs:1.52139280412857,0.00697144792857142) .. (axis cs:1.54013924952857,0.0257178933285714)
.. controls (axis cs:1.55888569492857,0.0444643387285714) and (axis cs:1.56942857142857,0.0699170941285714) .. (axis cs:1.56942857142857,0.0964285714285714)
--(axis cs:1.56942857142857,0.565857142857143)
--(axis cs:1.53942857142857,0.565857142857143)
--(axis cs:1.62328571428571,0.636221640500152)
--(axis cs:1.70714285714286,0.565857142857143)
--(axis cs:1.67714285714286,0.565857142857143)
--(axis cs:1.67714285714286,0.0964285714285714)
.. controls (axis cs:1.67714285714286,0.0413604457225714) and (axis cs:1.65524379652714,-0.0115085633511429) .. (axis cs:1.61630475136771,-0.0504476085105714)
.. controls (axis cs:1.57736570620829,-0.08938665367) and (axis cs:1.52449669713457,-0.111285714285714) .. (axis cs:1.46942857142857,-0.111285714285714)
--(axis cs:0.4,-0.111285714285714)
.. controls (axis cs:0.480746385148,-0.111285714285714) and (axis cs:0.558268205880571,-0.143396303854286) .. (axis cs:0.615364522441714,-0.200492620415429)
.. controls (axis cs:0.672460839002857,-0.257588936976571) and (axis cs:0.704571428571429,-0.335110757709143) .. (axis cs:0.704571428571429,-0.415857142857143)
--(axis cs:0.704571428571429,-0.565857142857143)
--(axis cs:0.734571428571429,-0.565857142857143)
--(axis cs:0.602285714285714,-0.676858036924309)
--(axis cs:0.47,-0.565857142857143)
--(axis cs:0.5,-0.565857142857143)
--(axis cs:0.5,-0.415857142857143)
.. controls (axis cs:0.5,-0.389345665557143) and (axis cs:0.4894571235,-0.363892910157143) .. (axis cs:0.4707106781,-0.345146464757143)
.. controls (axis cs:0.4519642327,-0.326400019357143) and (axis cs:0.4265114773,-0.315857142857143) .. (axis cs:0.4,-0.315857142857143)
--(axis cs:0.25,-0.315857142857143)
--(axis cs:0.25,-0.315857142857143)
--(axis cs:0.125,-0.315857142857143)
.. controls (axis cs:0.03125,-0.315857142857143) and (axis cs:-0.03125,-0.315857142857143) .. (axis cs:-0.125,-0.315857142857143)
--(axis cs:-0.25,-0.315857142857143)
--(axis cs:-0.25,-0.315857142857143)
--(axis cs:0.015035612076138,0)
--(axis cs:-0.25,0.315857142857143)
--(axis cs:0,0.315857142857143)
--(axis cs:-0.25,0.315857142857143)
--cycle;
\draw (axis cs:-0.134964387923862,0) node[
  scale=0.5,
  text=black,
  rotate=0.0,
  align=center
]{Total\\
2211};
\draw (axis cs:0.596714285714286,0.822183053264892) node[
  scale=0.5,
  text=black,
  rotate=0.0,
  align=center
]{Top-1\\677};
\draw (axis cs:0.992571428571429,0.78226588509603) node[
  scale=0.5,
  text=black,
  rotate=0.0,
  align=center
]{[2-10]\\344};
\draw (axis cs:1.30557142857143,0.752657655253061) node[
  scale=0.5,
  text=black,
  rotate=0.0,
  align=center
]{[11-99]\\97};
\draw (axis cs:1.62328571428571,0.786221640500152) node[
  scale=0.5,
  text=black,
  rotate=0.0,
  align=center
]{Top-100+\\377};
\draw (axis cs:0.602285714285714,-0.826858036924309) node[
  scale=0.5,
  text=black,
  rotate=0.0,
  align=center
]{NR\\716};
\end{axis}

\end{tikzpicture}
}
\vspace{-1cm}
\caption{Sankey diagram of 967 total repair instances sampled uniformly from the StackOverflow Python dataset balanced acrost snippet lengths and edit distances ($\lfloor|\err\sigma| / 10\rfloor \in [0, 8], \Delta(\err\sigma, \sigma') < 4$) with a sampling timeout of 30s per repair.}
\end{figure}
\end{frame}

\begin{frame}[fragile]{Feature comparison matrix}
\begin{table}
\begin{tabular}{c|cccccc}
             & \textbf{Sound}   & \textbf{Complete} & \textbf{Natural} & \textbf{Theory} & ||      & \textbf{Tool} \\\hline
Tidyparse    & \cmark           & \cmark            & \cmark           & CFG$_\cap$      & \cmark  & IDE-ready     \\
Seq2Parse    & \cmark$^\dagger$ & \xmark            & \cmark           & CFG             & \xmark  & Python        \\
BIFI         & \xmark           & \xmark            & \cmark           & $\Sigma^*$      & \xmark  & Python        \\
OrdinalFix   & \cmark           & \xmark            & \xmark           & CFG+            & \xmark  & Rust          \\
Outlines$^1$ & \cmark$^\dagger$ & \cmark$^\dagger$  & \cmark           & EBNF            & \xmark  & Python        \\
SynCode$^1$  & \cmark           & \cmark            & \cmark           & EBNF            & \xmark  & Python        \\
Aho/Irons    & \cmark           & \xmark            & \xmark           & CFG             & \xmark  & None          \\
\end{tabular}
\end{table}

\textbf{Sound} $=$ generated repairs always syntactically valid.

\textbf{Complete} $=$ all valid repairs are eventually generated.

\textbf{Natural} $\approx$ statistically likely / designed to model human preferability.

|| = Trivially parallelizable, i.e., designed to be executed on multiple cores.\\\vspace{0.3cm}

$^1$ Not specifically intended for syntax repair, but can be adapted.

$^\dagger$ Claimed by the authors, but counterexamples known to exist.
\end{frame}

\begin{frame}{Abbreviated history of algebraic parsing}
  \begin{itemize}
    \item \href{http://www-igm.univ-mlv.fr/~berstel/Mps/Travaux/A/1963-7ChomskyAlgebraic.pdf}{Chomsky \& Sch\"utzenberger (1959) - The algebraic theory of CFLs}
    \item Cocke–Younger–Kasami (1961) - Bottom-up matrix-based parsing
    \item \href{https://dl.acm.org/doi/10.1145/321239.321249}{Brzozowski (1964) - Derivatives of regular expressions}
%    \item \href{https://dl.acm.org/doi/10.1145/362007.362035}{Earley (1968) - top-down dynamic programming (no CNF needed)}
    \item \href{http://theory.stanford.edu/~virgi/cs367/papers/valiantcfg.pdf}{Valiant (1975) - first realizes the Boolean matrix correspondence}
    \begin{itemize}
      \item Na\"ively, has complexity $\mathcal{O}(n^4)$, can be reduced to $\mathcal{O}(n^\omega)$, $\omega < 2.763$
    \end{itemize}
    \item \href{https://www.cs.cornell.edu/home/llee/papers/bmmcfl-jacm.pdf}{Lee (1997) - Fast CFG Parsing $\Longleftrightarrow$ Fast BMM, formalizes reduction}
    \item \href{https://matt.might.net/papers/might2011derivatives.pdf}{Might et al. (2011) - Parsing with derivatives (Brzozowski $\Rightarrow$ CFL)}
    \item \href{https://users.math-cs.spbu.ru/~okhotin/papers/formal_languages_gf2.pdf}{Bakinova, Okhotin et al. (2010) - Formal languages over GF(2)}
    \item \href{https://arxiv.org/pdf/1601.07724.pdf}{Bernady \& Jansson (2015) - Certifies Valiant (1975) in Agda}
    \item \href{https://arxiv.org/pdf/1504.08342.pdf}{Cohen \& Gildea (2016) - Generalizes Valiant (1975) to parse and recognize mildly context sensitive languages, e.g. LCFRS, TAG, CCG}
    \item \textbf{Considine, Guo \& Si (2022) - SAT + Valiant (1975) + holes}
    \item \textbf{Considine, Guo \& Si (2024) - Levenshtein Bar-Hillel repairs}
  \end{itemize}
\end{frame}

%\section{Graph Programming}
%
%\begin{frame}[fragile]{Classical programs are graphs}
%  Programs can be compiled into DFGs and represented using a big matrix.
%  \begin{table}[H]
%    \centering
%    \begin{tabular}{lcc}
%      \textbf{Program} & \textbf{Dataflow Graph} & \textbf{Matrix} \\
%%            \begin{tabular}[c]{@{}l@{}} $\hat y = θx + b$\\ $l = ||\hat y - y||_2$\end{tabular}
%%              &
%      \begin{adjustbox}{minipage={.25\textwidth}, height=0.14\textwidth, margin*=-0.8cm 0cm 0cm 0.1cm}
%        \begin{lstlisting}[basicstyle=\ttfamily\tiny]
%sum = 0
%l = [0, 0, 0, 0]
%for i in range(0, 4):
%  l[i] += θ[i] * x[i]
%for i in range(0, 4):
%  l[i] -= y[i] - b
%for i in range(0, 4):
%  l[i] *= l[i]
%for i in range(0, 4):
%  sum += l[i]
%l = sqrt(sum)
%        \end{lstlisting}
%      \end{adjustbox}
%      & \begin{adjustbox}{minipage={.25\textwidth}, height=0.14\textwidth, margin*=0cm 0cm 0cm 0.1cm}
%          \digraph[scale=0.1]{prograph}{
%            node[ fontname="Helvetica" fontsize=20 shape=Mrecord ];
%            edge[ fontname="Helvetica" fontsize=18 ];
%
%            graph ["concentrate"="true","rankdir"="LR","bgcolor"="transparent","margin"="0.0","compound"="true","nslimit"="20"]
%            "eeba8" ["label"="+"]
%            "a8416" ["label"="+"]
%            "4500f" ["label"="pow"]
%            "a67f9" ["label"="*"]
%            "0.5" ["label"="0.5"]
%            "f14a3" ["label"="*"]
%            "9c49d" ["label"="*"]
%            "59c48" ["label"="+"]
%            "980bd" ["label"="+"]
%            "8f532" ["label"="+"]
%            "1a609" ["label"="+"]
%            "e58c4" ["label"="+"]
%            "23f5b" ["label"="+"]
%            "d829b" ["label"="+"]
%            "y₀" ["label"="y₀"]
%            "2783d" ["label"="+"]
%            "8bd47" ["label"="+"]
%            "y₂" ["label"="y₂"]
%            "517e6" ["label"="+"]
%            "8caa0" ["label"="+"]
%            "y₁" ["label"="y₁"]
%            "7da0d" ["label"="+"]
%            "b12cb" ["label"="+"]
%            "b" ["label"="b"]
%            "f8941" ["label"="+"]
%            "3eecd" ["label"="+"]
%            "2e83a" ["label"="+"]
%            "b59fd" ["label"="+"]
%            "dae83" ["label"="+"]
%            "b11ba" ["label"="*"]
%            "3bb89" ["label"="*"]
%            "b2454" ["label"="*"]
%            "7bed4" ["label"="*"]
%            "39644" ["label"="*"]
%            "12c32" ["label"="*"]
%            "d58d1" ["label"="*"]
%            "6c64d" ["label"="*"]
%            "fb0f0" ["label"="*"]
%            "6c2be" ["label"="*"]
%            "57fd4" ["label"="*"]
%            "a9bc3" ["label"="*"]
%            "x₀" ["label"="x₀"]
%            "θ₀" ["label"="θ₀"]
%            "x₂" ["label"="x₂"]
%            "θ₁" ["label"="θ₁"]
%            "x₂" ["label"="x₂"]
%            "x₄" ["label"="x₄"]
%            "x₁" ["label"="x₁"]
%            "x₃" ["label"="x₃"]
%            "eeba8" -> "a8416"
%            "a8416" -> "4500f"
%            "a67f9" -> "a8416"
%            "0.5" -> "4500f"
%            "f14a3" -> "eeba8"
%            "9c49d" -> "eeba8"
%            "59c48" -> "a67f9"
%            "980bd" -> "a67f9"
%            "8f532" -> "f14a3"
%            "1a609" -> "f14a3"
%            "e58c4" -> "9c49d"
%            "23f5b" -> "9c49d"
%            "d829b" -> "59c48"
%            "y₀" -> "59c48"
%            "y₀" -> "980bd"
%            "2783d" -> "980bd"
%            "8bd47" -> "8f532"
%            "y₂" -> "8f532"
%            "y₂" -> "1a609"
%            "517e6" -> "1a609"
%            "8caa0" -> "e58c4"
%            "y₁" -> "e58c4"
%            "y₁" -> "23f5b"
%            "7da0d" -> "23f5b"
%            "b12cb" -> "d829b"
%            "b" -> "d829b"
%            "b" -> "2783d"
%            "b" -> "8bd47"
%            "b" -> "517e6"
%            "b" -> "8caa0"
%            "b" -> "7da0d"
%            "f8941" -> "2783d"
%            "3eecd" -> "8bd47"
%            "2e83a" -> "517e6"
%            "b59fd" -> "8caa0"
%            "dae83" -> "7da0d"
%            "b11ba" -> "b12cb"
%            "3bb89" -> "b12cb"
%            "b2454" -> "f8941"
%            "7bed4" -> "f8941"
%            "39644" -> "3eecd"
%            "12c32" -> "3eecd"
%            "d58d1" -> "2e83a"
%            "6c64d" -> "2e83a"
%            "fb0f0" -> "b59fd"
%            "6c2be" -> "b59fd"
%            "57fd4" -> "dae83"
%            "a9bc3" -> "dae83"
%            "x₀" -> "b11ba"
%            "x₀" -> "b2454"
%            "θ₀" -> "b11ba"
%            "θ₀" -> "b2454"
%            "θ₀" -> "39644"
%            "θ₀" -> "d58d1"
%            "θ₀" -> "fb0f0"
%            "θ₀" -> "57fd4"
%            "x₂" -> "3bb89"
%            "x₂" -> "7bed4"
%            "θ₁" -> "3bb89"
%            "θ₁" -> "7bed4"
%            "θ₁" -> "12c32"
%            "θ₁" -> "6c64d"
%            "θ₁" -> "6c2be"
%            "θ₁" -> "a9bc3"
%            "x₂" -> "39644"
%            "x₂" -> "d58d1"
%            "x₄" -> "12c32"
%            "x₄" -> "6c64d"
%            "x₁" -> "fb0f0"
%            "x₁" -> "57fd4"
%            "x₃" -> "6c2be"
%            "x₃" -> "a9bc3"
%          }
%      \end{adjustbox} &
%      \begin{adjustbox}{minipage={.15\textwidth}, height=0.14\textwidth, margin*=-0.2cm 0cm 0cm 0.5cm}
%        \includegraphics[scale=0.15]{../figures/adj_prog.png}
%      \end{adjustbox}
%    \end{tabular}
%  \end{table}
%  This representation allows us to solve for their fixedpoints as eigenvectors.
%\end{frame}
%
%\begin{frame}[fragile]{Probabilistic programs are also graphs}
%  A Bayesian Belief Network (BN) is an acyclic DGM of the following form:
%
%  \tikzset{latent/.append style={minimum size=14pt, inner sep=1pt, node distance=10pt, draw,circle, inner sep=1pt}, obs/.append style={minimum size=14pt, inner sep=1pt, node distance=10pt, draw,circle, inner sep=1pt}}
%  \makeatletter
%  \newcommand\ccirc[1]{%
%    \mathpalette\@ccirc{#1}%
%  }
%  \newcommand\@ccirc[2]{%
%    \tikz[baseline=(math.base)] \node (math) {$\m@th#1#2$};%
%  }
%  \newcommand\gcirc[1]{%
%    \mathpalette\@gcirc{#1}%
%  }
%  \newcommand\@gcirc[2]{%
%    \tikz[baseline=(math.base)] \node[fill=gray!30] (math) {$\m@th#1#2$};%
%  }
%  \makeatother
%% https://maximustann.github.io/mach/2015/07/06/belief-network-2/
%% https://frnsys.com/notes/ai/foundations/probabilistic_graphical_models.html
%  \begin{prooftree}
%    \AxiomC{$X \cancel\perp Y \mid Z$}
%    \RightLabel{\textsc{V}}
%    \UnaryInfC{
%      \begin{tikzpicture}
%        \node[obs] (z) {$Z$};%
%        \node[latent,above=of z,xshift=-1cm,fill] (x) {$X$}; %
%        \node[latent,above=of z,xshift=1cm] (y) {$Y$}; %
%        \edge {x,y} {z}
%      \end{tikzpicture}
%    }
%    \DisplayProof
%    \AxiomC{$X \perp Y \mid Z$}
%    \RightLabel{\textsc{Fork}}
%    \UnaryInfC{
%      \begin{tikzpicture}
%        \node[obs] (z) {$Z$};%
%        \node[latent,below=of z,xshift=-1cm,fill] (x) {$X$}; %
%        \node[latent,below=of z,xshift=1cm] (y) {$Y$}; %
%        \edge {z} {x,y}
%      \end{tikzpicture}
%    }
%    \DisplayProof
%    \AxiomC{$X \perp Y \mid Z$}
%    \RightLabel{\textsc{Ch}}
%    \UnaryInfC{
%      \begin{tikzpicture}
%        \node[obs] (z) {$Z$};%
%        \node[latent,above=of z,yshift=-11pt, xshift=-32pt,fill] (x) {$X$}; %
%        \node[latent,below=of z,yshift=11pt, xshift=32pt] (y) {$Y$}; %
%        \edge {x} {z}
%        \edge {z} {y}
%      \end{tikzpicture}
%    }
%  \end{prooftree}
%
%%  \begin{equation*}
%%    P(x_1,\ldots,x_D)=\prod_{i=1}^D P(x_i \mid \texttt{parents}(x_i))
%%  \end{equation*}
%
%  Translatable to a probabilistic circuit a.k.a. Sum Product Network (SPN):
%
%  \begin{tabular}{cc}
%    \hspace{-1.8cm}
%    \begin{minipage}[c]{0.5\textwidth}
%      \centering
%      \begin{tabular}{l}
%        $PC \rightarrow v \sim \mathcal{D}$ \\
%        $PC \rightarrow PC \oplus PC$       \\
%        $PC \rightarrow PC \otimes PC$
%      \end{tabular}
%    \end{minipage}
%    &
%    \begin{minipage}[c]{0.5\textwidth}
%      \centering
%      \digraph[scale=0.4]{spn1}{
%        margin=0
%        compound=true
%        rankdir=LR
%        node [shape=Mrecord,fontname="JetBrains Mono"]
%        edge [fontsize=8,fontcolor=indigo]
%        bgcolor=transparent
%        nslimit=20
%
%        g0 [label="{{μ|σ}|Normal|{<Out0>g0}}"]
%        g1 [label="{{μ|σ}|Normal|{<Out0>g1}}"]
%        g2 [label="{{μ|σ}|Normal|{<Out0>g2}}"]
%        g3 [label="{{μ|σ}|Normal|{<Out0>g3}}"]
%
%        f4 [label="{{<In0>g0|<In1>g1|<In2>g2}|Σ|{<Out0>f4}}"]
%        f5 [label="{{<In0>f4|<In1>g3}|Π|{<Out0>f5}}"]
%
%        g0:Out0 -> f4:In0
%        g1:Out0 -> f4:In1
%        g2:Out0 -> f4:In2
%        g3:Out0 -> f5:In1
%        f4:Out0 -> f5:In0
%
%
%        out1 [style=invis,shape=point]
%        out2 [style=invis,shape=point]
%
%        f5 -> out1
%      }
%    \end{minipage}
%  \end{tabular}
%\end{frame}
%
%\begin{frame}[fragile]{Message passing \& path algebras}
%  A semiring algebra, denoted $(S, \oplus, \otimes, \circled{0}, \circled{1})$, is a set together with two binary operators $\oplus, \otimes: S \times S \rightarrow S$ such that $(S, \oplus, \circled{0})$ is a commutative monoid and $(S, \otimes, \circled{1})$ is a monoid. Furthermore, we have distributivity:
%
%  \begin{prooftree}
%    \bottomAlignProof
%    \AxiomC{$a \cdot (b \cdot c)$}
%    \RightLabel{\small\textsc{Assoc}}
%    \UnaryInfC{$(a \cdot b) \cdot c$}
%    \DisplayProof
%    \hskip 2.5em
%    \bottomAlignProof
%    \AxiomC{$a \cdot \circled 1$}
%    \RightLabel{\small\textsc{Neutral}}
%    \UnaryInfC{$a$\vphantom{$()$}}
%    \DisplayProof
%    \hskip 2.5em
%    \bottomAlignProof
%    \AxiomC{$a \cdot b$}
%    \RightLabel{\small\textsc{Comm}}
%    \UnaryInfC{$b \cdot a$\vphantom{$()$}}
%  \end{prooftree}
%
%  \begin{prooftree}
%    \bottomAlignProof
%    \AxiomC{$a \otimes (b \oplus c)$}
%    \UnaryInfC{$(a \otimes b) \oplus (a \otimes c)$}
%    \AxiomC{$(a \oplus b) \otimes c$}
%    \RightLabel{\small\textsc{Dist}}
%    \UnaryInfC{$(a \otimes c) \oplus (b \otimes c)$}
%    \DisplayProof
%    \hskip 2.5em
%    \bottomAlignProof
%    \AxiomC{$a \otimes \circled 0$}
%    \RightLabel{\small\textsc{Annhil}}
%    \UnaryInfC{$\circled 0$\vphantom{$()$}}
%  \end{prooftree}
%  \vspace{2pt}
%  These operators can be lifted to matrices to form \textit{path algebras}:
%  \begin{center}
%    \begin{tabular}{lcr}
%      $\delta_{st} = \overbrace{\underset{P\in P_{st}^*}{\bigoplus}\underbrace{\underset{e\in P}{\bigotimes}W_{e}}_{\text{Aggregate}}}^{\text{Update}}$ & &
%      \bgroup
%      \def\arraystretch{1.2}
%      \begin{tabular}{c{1cm}c{1cm}|c{1cm}c{1cm}|c}
%        $\oplus$         & $\otimes$ & $\circled{0}$ & $\circled{1}$ & Path     \\\hline
%        min              & +         & $\infty$      & 0             & Shortest \\
%        max              & +         & $-\infty$     & 0             & Longest  \\
%        max              & min       & 0             & $\infty$      & Widest   \\
%        $\underline\vee$ & $\land$   & $\circ$       & $\top$        & Random   \\
%      \end{tabular}
%      \egroup
%    \end{tabular}
%%        \begin{tabular}{lc|cr}
%%            $δ_{st} = \overbrace{\underset{P\in P_{st}^*}{\bigoplus}\underbrace{\underset{e\in P}{\bigotimes}W_{e}}_{\text{Aggregate}}}^{\text{Update}}$ & & &
%%            \bgroup
%%            \def\arraystretch{1.2}
%%            \begin{tabular}{c|c{1cm}c{1cm}|c{1cm}c{1cm}|c}
%%                S                           & $\oplus$ & $\otimes$ & $\circled{0}$ & $\circled{1}$ & Path     \\\hline
%%                $\mathbb R \cup \{\infty\}$ & min      & +         &   $\infty$    &      0        & Shortest \\
%%                $\mathbb R \cup \{\infty\}$ & max      & +         &   $-\infty$   &      0        & Longest  \\
%%                $\mathbb R \cup \{\infty\}$ & max      & min       &       0       &   $\infty$    & Widest   \\
%%            \end{tabular}
%%            \egroup
%%        \end{tabular}
%  \end{center}
%\end{frame}
%
%
%\section{Finite Fields}\label{sec:finite-fields}
%
%\begin{frame}{Recap: Classical logic in a nutshell}
%  \begin{prooftree}
%    \bottomAlignProof
%    \AxiomC{$a \veebar b$}
%    \RightLabel{XOR}
%    \UnaryInfC{$(p \vee q)\land \neg (p\land q)\phantom{()}$}
%    \DisplayProof
%    \hskip 1.5em
%    \bottomAlignProof
%    \AxiomC{$a \to b$}
%    \RightLabel{Impl}
%    \UnaryInfC{$\neg a \vee b$}
%    \DisplayProof
%    \hskip 1.5em
%    \bottomAlignProof
%    \AxiomC{$a \leftrightarrow b$}
%    \RightLabel{Iff}
%    \UnaryInfC{$(\neg a \vee b)\land (\neg b\vee a)$}
%    \DisplayProof
%    \vskip 2.5em
%
%    \bottomAlignProof
%    \AxiomC{$\neg\neg a\phantom{()}$}
%    \RightLabel{2Neg}
%    \UnaryInfC{$a\phantom{()}$}
%    \DisplayProof
%    \hskip 2.5em
%    \bottomAlignProof
%    \AxiomC{$a \cdot (b \cdot c)$}
%    \RightLabel{Assoc$_{\land\vee}$}
%    \UnaryInfC{$(a \cdot b) \cdot c$}
%    \DisplayProof
%    \hskip 2.5em
%    \bottomAlignProof
%    \AxiomC{$a \cdot b$}
%    \RightLabel{Comm$_{\land\vee}$}
%    \UnaryInfC{$b \cdot a$}
%    \DisplayProof
%    \vskip 2.5em
%
%    \bottomAlignProof
%    \AxiomC{$a \land (b \vee c)$}
%    \RightLabel{Dist$_\land$}
%    \UnaryInfC{$(a \land b) \vee (a \land c)$}
%    \DisplayProof
%    \hskip 2.5em
%    \bottomAlignProof
%    \AxiomC{$a \vee (b \land c)$}
%    \RightLabel{Dist$_\vee$}
%    \UnaryInfC{$(a \vee b) \land (a \vee c)$}
%    \DisplayProof
%    \vskip 2.5em
%
%    \bottomAlignProof
%    \AxiomC{$\neg (a \vee b)$}
%    \RightLabel{DeMorgan$_\vee$}
%    \UnaryInfC{$\neg a \land \neg b$}
%    \DisplayProof
%    \hskip 2.5em
%    \bottomAlignProof
%    \AxiomC{$\neg (a \land b)$}
%    \RightLabel{DeMorgan$_\land$}
%    \UnaryInfC{$\neg a \vee \neg b$}
%  \end{prooftree}
%\end{frame}
%
%\begin{frame}{Normalization in classical logic}
%  \begin{columns}[t]
%    \begin{column}{0.55\textwidth}
%      \begin{center}
%        \textbf{Conjunctive Normal Form}
%        \phantom{()}\\
%        \begin{align*}
%          \textsc{Conj} &\rightarrow (\textsc{Disj}) \mid \textsc{Conj} \land (\textsc{Disj})\\
%          \textsc{Unit} &\rightarrow \textsc{Var} \mid \neg \textsc{Var} \mid \bot \mid \top \\
%          \textsc{Disj} &\rightarrow \textsc{Unit} \mid \textsc{Disj} \vee \textsc{Disj}
%        \end{align*}
%        \begin{prooftree}
%          \AxiomC{$\neg (x\vee \neg y)\vee \neg \neg z$}
%          \RightLabel{2Neg}
%          \UnaryInfC{$\neg (x\vee \neg y)\vee z$}
%          \RightLabel{DeMorgan}
%          \UnaryInfC{$(\neg x\land  \neg \neg y)\vee z$}
%          \RightLabel{2Neg}
%          \UnaryInfC{$(\neg x\land  y)\vee z$}
%          \RightLabel{Dist}
%          \UnaryInfC{$(\neg x\vee z)\land  (y\vee z)$}
%        \end{prooftree}\\
%        \phantom{()}\\
%        \phantom{Solved with SAT (e.g., DPLL/CDCL/Survey Prop).}\\
%      \end{center}
%    \end{column}
%    \begin{column}{0.45\textwidth}
%      \begin{center}
%        \textbf{Zhegalkin Normal Form}
%        \phantom{()}\\
%        \[
%          f(x_1, \ldots x_n) = \bigoplus_{i\subseteq\{1,\ldots, n\}}a_i x^i
%        \]
%        i.e., $a_i$'s filter the powerset.
%        \phantom{()}\\
%        \begin{prooftree}
%          \AxiomC{$x + (y \land \neg z)$}
%          \RightLabel{}
%          \UnaryInfC{$x + y(1 \oplus z)$}
%          \RightLabel{}
%          \UnaryInfC{$x + (y \oplus yz)$}
%          \RightLabel{}
%          \UnaryInfC{$x \oplus (y \oplus yz) \oplus x(y \oplus yz)$}
%          \RightLabel{}
%          \UnaryInfC{$x \oplus y \oplus xy \oplus yz \oplus xyz$}
%        \end{prooftree}
%      \end{center}\\
%      \phantom{()}\\
%      \phantom{Solved with rootfinding (e.g., power iteration, Gr\"obner basis).}\\
%      \phantom{()}\\
%    \end{column}
%  \end{columns}
%\end{frame}
%
%\begin{frame}{Some common algebraic and logical forms}
%  \begingroup
%  \setlength{\tabcolsep}{7.5pt} % Default value: 6pt
%  \begin{tabular}{c|c|c|c|c|c|c}
%    a_1 & a_2 & a_3 & a_4 & ZNF            & Logical                    & CNF                                     \\
%    \hline
%    0   & 0   & 0   & 0   & 0              & $\bot                      $                     & $x \land \neg x                       $                        \\
%    1   & 0   & 0   & 0   & 1              & $\top                      $                     & $x \vee \neg x                        $                         \\
%    0   & 1   & 0   & 0   & x              & $x                         $                        & $x                                    $                                     \\
%    1   & 1   & 0   & 0   & 1 + x          & $\neg x                    $                   & $\neg x                               $                                \\
%    0   & 0   & 1   & 0   & y              & $y                         $                        & $y                                    $                                     \\
%    1   & 0   & 1   & 0   & 1 + y          & $\neg y                    $                   & $\neg y                               $                                \\
%    0   & 1   & 1   & 0   & x + y          & $x \oplus y                $               & $(x \vee y) \land (\neg x \vee \neg y)$ \\
%    1   & 1   & 1   & 0   & 1 + x + y      & $ x \Longleftrightarrow y  $ & $(x \vee \neg y) \land (\neg x \vee y)$ \\
%    0   & 0   & 0   & 1   & xy             & $x \land y                 $                & $x \land y                            $                             \\
%    1   & 0   & 0   & 1   & 1 + xy         & $\neg(x \land y)           $          & $(\neg x) \vee (\neg y)               $                \\
%    0   & 1   & 0   & 1   & x + xy         & $x \land (\neg y)          $         & $x \land (\neg y)                     $                      \\
%    1   & 1   & 0   & 1   & 1 + x + xy     & $x \Longrightarrow y       $      & $(\neg x) \vee y                      $                       \\
%    0   & 0   & 1   & 1   & y + xy         & $(\neg x) \land y          $         & $(\neg x) \land y                     $                      \\
%    1   & 0   & 1   & 1   & 1 + y + xy     & $x \Longleftarrow y        $       & $x \vee (\neg y)                      $                       \\
%    0   & 1   & 1   & 1   & x + y + xy     & $x \vee y                  $                 & $x \vee y                             $                              \\
%    1   & 1   & 1   & 1   & 1 + x + y + xy & $\neg(x \vee y)            $           & $(\neg x) \land (\neg y)              $               \\
%  \end{tabular}
%  \endgroup
%\end{frame}
%
%\begin{frame}{Facts about finite fields}
%  \begin{itemize}
%    \item For every prime number p and positive integer n, there exists a finite field with $p^n$ elements, denoted $GF(p^n)$, $\mathbb{Z}/p^n$ or $\mathbb{F}_p^n$.
%    \item The following instruction sets have identical expressivity:
%    \begin{itemize}
%      \item Pairs: $\{\vee, \neg\}, \{\wedge, \neg\}, \{\to, \neg\}, \{\to, \bot\}, \{\to, \veebar\}, \{\land, \veebar\}, \ldots$
%      \item Triples: $\{\lor, =, \veebar\}, \{\lor, \veebar, \top\}, \{\land, =, \bot\}, \{\land, =, \veebar\}, \{\land, \veebar, \top\}, \ldots$
%    \end{itemize}
%    \item In other words, we can compute any Boolean function $\mathbb{B}^n\rightarrow\mathbb{B}$ by composing any one of the above operator sets in an orderly fashion.
%    \item $\mathbb{F}_2$ corresponds to arithmetic modulo 2, i.e., $\oplus := \veebar, \otimes := \land$.
%    \item There are (at least) two schools of thought about Boolean circuits:
%    \begin{itemize}
%      \item Logical: Conjunctive Normal Form (CNF). May not be unique.
%      \item Algebra: Zhegalkin Normal Form (ZNF). Always unique.
%    \end{itemize}
%    \item The type $\mathbb{F}_2^n\rightarrow\mathbb{F}_2$ possesses $2^{2^{n}}$ inhabitants.
%  \end{itemize}
%\end{frame}
%
%\begin{frame}{Preface to “Two Memoirs on Pure Analysis”}
%  \setlength{\epigraphwidth}{0.97\textwidth}
%  \epigraph{``\textit{Long algebraic calculations were at first hardly necessary for mathematical progress... It was only since Euler that concision has become indispensable to continuing the work this great geometer has given to science. Since Euler, calculation has become more and more necessary and... the algorithms so complicated that progress would be nearly impossible without the elegance that modern geometers have brought to bear on their research, and by which means the mind can promptly and with a glance grasp a large number of operations.}\\
%
%  \vspace{5pt}
%  \ldots\\
%  \vspace{5pt}
%
%  \textit{It is clear that elegance, so admirably and aptly named, has no other purpose.}\\
%
%  \vspace{5pt}
%  \ldots\\
%  \vspace{5pt}
%
%  \textit{Jump headlong into the calculations! Group the operations, classify them by their difficulties and not their appearances. This, I believe, is the mission of future geometers. This is the road on which I am embarking in this work.}''}{\'Evariste Galois, 1811-1832}
%\end{frame}


%\section{Future Work}

%\begin{frame}{What's the point?}
%  \begin{itemize}
%    \item Algebraists have developed a powerful language for rootfinding
%    \item Tradition handed down from Fermat, Euler, Galois, Kleene, Chomsky
%    \item We have closed forms for exponentials of structured matrices
%%   \item Characteristic polynomials, companion matrices, eigenvalues
%    \item Solving these forms can be much faster than power iteration
%    \item Unifies many problems in PL, probability and graph theory
%    \item Context-free parsing is just rootfinding on a semiring algebra
%    \item Unification/simplification a form of lazy hypergraph search
%    \item Bounded program synthesis is matrix factorization/completion
%    \item By doing so, we can leverage well-known algebraic techniques
%  \end{itemize}
%\end{frame}
%
%\begin{frame}{Future work}
%  Parsing
%  \begin{itemize}
%    \item Error propagation in discrete dynamical systems and TRS
%    \item Dynamic matrix inverse and incremental transitive closure
%    \item Language Edit Distance with metrizable Boolean semirings
%    \item Unify parser-lexer for scannerless ECP on a real language
%    \item Investigate the feasibility of grammar induction and repair
%    \item Strengthen the connection to Leibnizian differentiability
%%    \item What is the meaning of abstract algebraic eigenvalues?
%  \end{itemize}
%  \phantom{space}\\
%  Probability
%  \begin{itemize}
%    \item Look into Markov chains (detailed balance, stationarity, reversibility)
%    \item Fuse Valiant parser and probabilistic context-free grammar
%    \item Contextualize belief propagation and graph diffusion processes
%    \item Look into constrained optimization (e.g., L/QP) to rank feasible set
%  \end{itemize}
%\end{frame}

\begin{frame}{Special thanks}
  \begin{center}
    \LARGE{
      Jin Guo, Xujie Si, David Bieber,\\
      David Chiang, Brigitte Pientka, David Hui,\\
      Ori Roth, Younesse Kaddar, Michael Schröder\\
      Will Chrichton, Kristopher Micinski, Alex Lew\\
      Matthijs Vákár, Michael Coblenz, Maddy Bowers\\
      \phantom{}\\
    }
    \href{https://cs.mcgill.ca}{\includegraphics[scale=0.06]{../figures/mcgill_logo.png}}
    \href{https://mila.quebec}{\includegraphics[scale=0.1]{../figures/mila_logo.png}}
  \end{center}
\end{frame}

\begin{frame}
  \begin{center}
    \huge{Learn more at: \\~\\
    \href{https://tidyparse.github.io}{\color{blue}{https://tidyparse.github.io}}}
  \end{center}
\end{frame}

%\begin{frame}[fragile]{Chomsky Denormalization}
%  Chomksy normalization is needed for matrix-based parsing, however produces lopsided parse trees. We can denormalize them using a simple recusive procedure to restore the natural shape of the original CFG:\vspace{0.5cm}\\
%
%  \begin{minipage}[l]{6cm}
%    \vspace{0.3cm}\resizebox{\textwidth}{!}{
%      \begin{tabular}{ll}
%        \Tree [.\texttt{S} \tikz\node(v1){\texttt{true}} [.$\ccancel{\texttt{and.S}}$ \tikz\node(v3){\texttt{and}} [.\texttt{S} \tikz\node(v5){\texttt{(}} [.$\ccancel{\texttt{S.)}}$ [.\texttt{S} \tikz\node(v9){\texttt{false}} [.$\ccancel{\texttt{or.S}}$ \tikz\node(v11){\texttt{or}} [.\texttt{S} \texttt{!} \texttt{true} ] ] ] \tikz\node(v7){\texttt{)}} ] ] ] ]
%%    \Tree [.S [.NP John ] [.VP [.\tikz\node(v1){V}; sleeps ] ] ]
%        \hspace{-2cm}
%        &
%        \Tree [.\texttt{S} \tikz\node(v2){\texttt{true}} \tikz\node(v4){\texttt{and}} [.\texttt{S} \tikz\node(v6){\texttt{(}} [.\texttt{S} \tikz\node(v10){\texttt{false}} \tikz\node(v12){\texttt{or}} [.\texttt{S} \texttt{!} \texttt{true} ] ] \tikz\node(v8){\texttt{)}} ] ]\\\\
%%    \Tree [.\tikz\node(v2){V}; [.\tikz\node(v3){V}; ] [.Adv {a lot} ] ]
%        \hspace{1cm}\huge{Pre-Denormalization} & \hspace{3cm}\huge{Post-Denormalization}
%      \end{tabular}
%      \begin{tikzpicture}[overlay]
%%    \draw [red,dashed,-stealth] (v1) to[bend left] (v2);
%        \draw [red,dashed,-stealth] (v3) to[bend left] (v4);
%%    \draw [red,dashed,-stealth] (v5) to[bend left] (v6);
%        \draw [red,dashed,-stealth] (v7) to[bend left] (v8);
%%    \draw [red,dashed,-stealth] (v9) to[bend right] (v10);
%        \draw [red,dashed,-stealth] (v11) to[bend right] (v12);
%      \end{tikzpicture}
%    }
%%    \caption{Result of applying Algorithm~\ref{alg:cap} to the tree obtained by parsing the string: \texttt{true and ( false or ! true )}.}
%  \end{minipage}
%  \hspace{-0.7cm}\scalebox{0.6}{
%    \begin{minipage}[l]{10cm}
%      \begin{algorithm}[H]
%        \caption{Rewrite procedure for tree denormalization}\label{alg:cap}
%        \begin{algorithmic}
%          \Procedure{Cut}{\texttt{t: Tree}}
%            \State $\texttt{stems} \leftarrow \{\:\textsc{Cut}(\texttt{c}) \mid \texttt{c} \in \texttt{t.children}\:\}$
%            \If{$\texttt{t.root} \in (V_{\mathcal{G}'} \setminus V_{\mathcal{G}})$}
%              \State \textbf{return } \texttt{stems} %\Comment{Drop synthetic nonterminals.}
%            \Else%\Comment{Graft the denormalized children on root.}
%              \State \textbf{return } $\{\:\texttt{Tree(t.root, stems)}\:\}$
%            \EndIf
%          \EndProcedure
%        \end{algorithmic}
%      \end{algorithm}
%    \end{minipage}
%  }
%
%  \vspace{1cm}All synthetic nonterminals are excised during Chomsky denormalization. Rewriting improves legibility but does not alter the underlying semantics.
%\end{frame}
%
%\begin{frame}[fragile]{Incremental parsing}
%  Should only need to recompute submatrices affected by individual edits. In the worst case, each edit requires quadratic complexity in terms of $|\Sigma^*|$, assuming $\mathcal{O}(1)$ cost for each CNF-nonterminal subset join, $\mathbf{V}'_1\otimes \mathbf{V}'_2$.
%  \begin{center}
%    \begin{tabular}{ c c c c c }
%      \scalebox{0.32}{\mkTrellisAppend{7}} & & \scalebox{0.32}{\mkTrellisInsert{6}}         & & \scalebox{0.32}{\mkTrellisInsert{7}}         \\
%      Append                               & & Delete                                       & & Insert                                       \\
%      $\mathcal{O}(n+1)$                     & & $\mathcal{O}\left(\frac{1}{4}(n-1)^2\right)$ & & $\mathcal{O}\left(\frac{1}{4}(n+1)^2\right)$ \\
%    \end{tabular}
%  \end{center}
%  Related to \textbf{dynamic matrix inverse} and \textbf{incremental transitive closure} with vertex updates. With a careful encoding, we can incrementally update SAT constraints as new keystrokes are received to eliminate redundancy.
%\end{frame}
%
%%\begin{frame}{Error Recovery}
%%  \begin{figure}[H]
%%    \adjustbox{scale=0.75,center}{%
%%    \hspace{-0.5cm}\begin{minipage}[l]{6cm}
%%      \[
%%        \begin{NiceMatrix}
%%              \leftarrow & \nse & \nsi & \nfi & \nfo & \nth & \ntw & \non & \leftarrow & \ppp \\
%%                         &      & \ddd & \ddd & \ddd & \ddd & \ddd & \ddd & \ddd & \ppp \\
%%          \sigma_1^\shri & \cdd &      & A    &      &      &      &      &      & \ppp \\
%%                    \vno & \ddd &  T_A & \vdd &      &      &      &      &      & \ppp \\
%%                    \vdd & \ddd &      & \pcd & \cdd &      & B    &      &      & \ppp \\
%%                         &      &      &      &      & T_B  & \vdd &      &      & \ppp \\
%%                         &      &      &      &      &      & \pcd & \cdd &      & C    \\
%%                         &      &      &      &      &      &      &      & T_C  & \vdd \\
%%                         &      &      &      &      &      &      & \text{\emoji{cross-mark}} &      & \\
%%                         &      &      &      &      &      &      &      &      & \\
%%                         &      &      &      &      &      &      &      &      & \\
%%                         &      &      &      &      &      &      &      & \ppp & \sigma_n^\shup \\
%%                    \vno & \cdd &      &      &      &      &      &      & \vno &
%%        \end{NiceMatrix}
%%      \]
%%    \end{minipage}
%%    }
%%    \hspace{1cm}
%%
%%    \caption{By recursing over upper diagonals of decreasing elevation and discarding subtrees that fall under the shadow of another's canopy, we can recover parseable subtrees.}
%%  \end{figure}
%%\end{frame}
%
%\begin{frame}[fragile]{Conjunctive parsing}
%  It is well-known that the family of CFLs is not closed under intersection. For example, consider $\mathcal{L}_\cap := \mathcal{L}_{\mathcal{G}_1} \cap \mathcal{L}_{\mathcal{G}_2}$:
%
%  \begin{table}[H]
%    \begin{tabular}{llll}
%      $P_1 := \big\{\;S \rightarrow L R,$ & $L \rightarrow a b \mid a L b,$ & $R \rightarrow c \mid c R\;\big\}$\vspace{5pt} \\
%      $P_2 := \big\{\;S \rightarrow L R,$ & $R \rightarrow b c \mid b R c,$ & $L \rightarrow a \mid a L\;\big\}$
%    \end{tabular}
%  \end{table}
%
%  \noindent Note that $\mathcal{L}_\cap$ generates the language $\big\{\;a^d b^d c^d \mid d > 0\;\big\}$, which according to the pumping lemma is not context-free. To encode $\mathcal{L}_\cap$, we intersect all terminals $\Sigma_\cap := \bigcap_{i=1}^c \Sigma_i$, then for each $t_\cap \in \Sigma_\cap$ and CFG, construct an equivalence class $E(t_\cap, \mathcal{G}_i) = \{ w_i \mid (w_i \rightarrow t_\cap) \in P_i\}$ as follows:\vspace{-5pt}
%
%  \begin{align}
%    \bigwedge_{t\in\Sigma_\cap}\bigwedge_{j = 1}^{c-1}\bigwedge_{i=1}^{|\sigma|} E(t_{\cap}, \mathcal{G}_j) \equiv_{\sigma_i} E(t_{\cap}, \mathcal{G}_{j+1})
%  \end{align}
%  % Generated by cfl4_intersection.vox, open with https://voxelator.com/
%  \begin{figure}[H]
%    \includegraphics[height=0.093\textwidth]{../figures/angle1.png}\hspace{-5pt}
%    \includegraphics[height=0.093\textwidth]{../figures/angle2.png}\hspace{-5pt}
%    \includegraphics[height=0.093\textwidth]{../figures/angle5.png}\hspace{-5pt}
%    \includegraphics[height=0.093\textwidth]{../figures/angle3.png}\hspace{-3pt}
%    \includegraphics[height=0.093\textwidth]{../figures/angle4.png}
%  \end{figure}
%\end{frame}
%
%\begin{frame}[fragile]{An Simple Reachability Proof}
%  \begin{lemma}
%    For any nonempty language $\ell: \mathcal{L}(\mathcal{G})$ and invalid string $\err{\sigma}: \Sigma^n$, there exists an $(\tilde{\sigma}, m)$ such that $\tilde{\sigma} \in \ell\cap\Sigma^m$ and $0 < \Delta(\err{\sigma}, \ell) \leq \max(m, n) < \infty$, where $\Delta$ denotes the Levenshtein edit distance.\\
%  \end{lemma}
%
%  \begin{proof}
%    Since $\ell$ is nonempty, it must have at least one inhabitant $\sigma \in \ell$. Let $\tilde{\sigma}$ be the smallest such member. Since $\tilde{\sigma}$ is a valid sentence in $\ell$, by definition it must be that $|\tilde{\sigma}|<\infty$. Let $m:=|\tilde{\sigma}|$. Since we know $\err{\sigma} \notin \ell$, it follows that $0 < \Delta(\err{\sigma}, \ell)$. Let us consider two cases, either $\tilde{\sigma} = \varepsilon$, or $0 < |\tilde{\sigma}|$:
%
%    \begin{itemize}
%      \item If $\tilde{\sigma} = \varepsilon$, then $\Delta(\err{\sigma}, \tilde{\sigma}) = n$ by full erasure of $\err{\sigma}$, or
%      \item If $0 < m$, then $\Delta(\err{\sigma}, \tilde{\sigma}) \leq \max(m, n)$ by overwriting.
%    \end{itemize}
%
%    In either case, it follows $\Delta(\err{\sigma}, \ell) \leq \max(m, n)$ and $\ell$ is always reachable via a finite nonempty set of Levenshtein edits, i.e., $0 < \Delta(\err{\sigma}, \ell) < \infty$.
%  \end{proof}
%\end{frame}
%
%\begin{frame}[fragile]{Probabilistic repair generation}
%  \resizebox{\textwidth}{!}{
%    \begin{minipage}{1.4\textwidth}
%      \begin{algorithm}[H]
  \caption{Probabilistic reachability with adaptive resampling}
  \label{alg:adaptive}
  \begin{algorithmic}[1]
    \Require $\mathcal{G}$ grammar, $\err{\sigma}$ broken string, $p$ process ID, $c$ total CPU cores, $t_{\text{total}}$ timeout.
    \State $\mathcal{Q} \gets \varnothing, \mathcal{R} \gets \varnothing, \epsilon \gets 1, i \gets 0, Y \sim \mathbb{Z}_2^m, t_0 \gets t_{\text{now}}$ \Comment{Initialize replay buffer $\mathcal{Q}$ and reservoir $\mathcal{R}$.}
    \Repeat
      \If {$\mathcal{Q} = \varnothing$ or \textbf{Rand}(0, 1) $< \epsilon$}
        \State $\hat\sigma \gets \varphi^{-1}\left(\langle\kappa, \rho\rangle^{-1}(U^{ci+p}Y), \err{\sigma}\right), i \gets i + 1$ \Comment{Sample WoR using the leapfrog method.}
      \Else
        \State $\hat\sigma \sim \mathcal{Q} + \textbf{Noise}(\mathcal{Q})$ \Comment{Sample replay buffer with additive noise.}
      \EndIf
      \State $\mathcal{R} \gets \mathcal{R} \cup \{\hat\sigma\}$ \Comment{Insert repair candidate $\hat\sigma$ into reservoir $\mathcal{R}$.}
      \If{$\mathcal{R}$ is full}
        \State $\hat\sigma \gets \argmin_{\hat\sigma \in \mathcal{R}} PP(\hat\sigma)$ \Comment{Select lowest perplexity repair candidate.}
        \If{$\hat\sigma \in \mathcal{L}(\mathcal{G})$}
          \State $\mathcal{Q} \gets \mathcal{Q} \cup \{\hat\sigma\}$ \Comment{Insert successful repair into replay buffer.}
        \EndIf
        \State $\mathcal{R} \gets \mathcal{R} \setminus \{\hat\sigma\}$ \Comment{Remove checked sample from the reservoir.}
      \EndIf
      \State $\epsilon \leftarrow \textbf{Schedule}\big((t_{\text{now}} - t_0) / t_{\text{total}}\big)$ \Comment{Update exploration/exploitation rate.}
    \Until{$t_{\text{total}}$ elapses.}
    \State \Return $\tilde\sigma \in \mathcal{Q}$ ranked by $PP(\tilde\sigma)$.
  \end{algorithmic}
\end{algorithm}
%    \end{minipage}
%  }
%\end{frame}
%
%\begin{frame}[fragile]{Multicore Scaling Results (aarch64)}
%  \begin{tikzpicture}
%    \begin{axis}[
%      ybar,
%      enlargelimits=0.15,
%      legend style={at={(0.03,0.97)},anchor=north west},
%      title={\textbf{Relative Total Distinct Solutions Found vs. Single Core}},
%      x=1cm,
%      ylabel={Relative improvement},
%      xlabel={Number of assigned cores},
%      symbolic x coords={2,3,4,5,6,7,8,9,10},
%      xtick=data,
%      bar width=3pt,
%    ]
%      \addplot coordinates {
%        (2,0.1343) (3,0.1955) (4,0.2249) (5,0.2475)
%        (6,0.2760) (7,0.2994) (8,0.3073) (9,0.3151) (10,0.3229)
%      };
%      \addplot coordinates {
%        (2,0.2655) (3,0.4353) (4,0.5614) (5,0.6644)
%        (6,0.7462) (7,0.7783) (8,0.8347) (9,0.8005) (10,0.7798)
%      };
%      \addplot coordinates {
%        (2,0.3972) (3,0.6928) (4,0.9327) (5,1.1834)
%        (6,1.3138) (7,1.3988) (8,1.6039) (9,1.5500) (10,1.5691)
%      };
%      \addplot coordinates {
%        (2,0.4863) (3,0.8326) (4,1.1368) (5,1.3879)
%        (6,1.5873) (7,1.7494) (8,1.8802) (9,1.9059) (10,1.9625)
%      };
%      \addplot coordinates {
%        (2,0.4315) (3,0.7583) (4,1.0122) (5,1.2593)
%        (6,1.4586) (7,1.6349) (8,1.7813) (9,1.8324) (10,1.8695)
%      };
%      \legend{Holes=2,Holes=3,Holes=4,Holes=5,Holes=6}
%    \end{axis}
%  \end{tikzpicture}
%\end{frame}
\end{document}