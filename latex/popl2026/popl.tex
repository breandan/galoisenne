%! suppress = LineBreak
%% For double-blind review submission, w/o CCS and ACM Reference (max submission space)
%\documentclass[sigplan,10pt,review,anonymous]{acmart}
%\settopmatter{printfolios=false,printccs=false,printacmref=false}
%% For double-blind review submission, w/ CCS and ACM Reference
%\documentclass[sigplan,review,anonymous]{acmart}\settopmatter{printfolios=true}
%% For single-blind review submission, w/o CCS and ACM Reference (max submission space)
%\documentclass[sigplan,review]{acmart}\settopmatter{printfolios=true,printccs=false,printacmref=false}
%% For single-blind review submission, w/ CCS and ACM Reference
%\documentclass[sigplan,review]{acmart}\settopmatter{printfolios=true}
%% For final camera-ready submission, w/ required CCS and ACM Reference
%\documentclass[sigplan,nonacm]{acmart}
\documentclass[sigplan,review,acmsmall,nonacm,screen,anonymous]{acmart}\settopmatter{printfolios=false,printccs=false,printacmref=false}

%% Conference information
%% Supplied to authors by publisher for camera-ready submission;
%% use defaults for review submission.
%\acmConference[SPLASH'24]{ACM SIGPLAN conference on Systems, Programming, Languages, and Applications: Software for Humanity}{October 22-27, 2024}{Pasadena, California, United States}
%\acmConference{}{}{}
%\acmYear{2018}
%\acmISBN{} % \acmISBN{978-x-xxxx-xxxx-x/YY/MM}
%\acmDOI{} % \acmDOI{10.1145/nnnnnnn.nnnnnnn}
%\startPage{1}

%% Copyright information
%% Supplied to authors (based on authors' rights management selection;
%% see authors.acm.org) by publisher for camera-ready submission;
%% use 'none' for review submission.
\setcopyright{none}
%\setcopyright{acmcopyright}
%\setcopyright{acmlicensed}
%\setcopyright{rightsretained}
%\copyrightyear{2018}           %% If different from \acmYear

%% Bibliography style
\bibliographystyle{acmart}

\usepackage{dsfont}
\usepackage{stmaryrd}
\usepackage{colortbl}
\usepackage{hyperref}

\usepackage{amsmath}
\DeclareMathOperator*{\argmax}{argmax}
\DeclareMathOperator*{\argmin}{argmin}
\usepackage{amssymb}

\usepackage[dvipsnames, table]{xcolor}
\usepackage{textcomp}

% Packages
\usepackage[pdf]{graphviz}
\usepackage{mathrsfs}

\newcommand*\circled[1]{\tikz[baseline=-0.1cm]{
  \node[shape=circle,draw,inner sep=0.48pt] (char) {\fontsize{7}{12}\textsf{#1}};}}

\DeclareMathAlphabet{\mathcal}{OMS}{cmsy}{m}{n}
\usepackage{cancel}
\newcommand\ccancel[2][red]{\renewcommand\CancelColor{\color{#1}}\cancel{#2}}
\newcommand{\nDownarrow}{\ensuremath{\text{ }\cancel{\Downarrow}\text{ }}}
\usepackage{centernot}

\usepackage{pgfplots, pgfplotstable}
\pgfplotsset{compat=1.7}
\usepgfplotslibrary{fillbetween}
\usetikzlibrary{patterns}
\pgfmathdeclarefunction{gauss}{2}{\pgfmathparse{1/(#2*sqrt(2*pi))*exp(-((x-#1)^2)/(2*#2^2))}}
\pgfmathdeclarefunction{nil}{1}{\pgfmathparse{0.001}}

\usepackage{arydshln}
\usepackage{adjustbox}
\usepackage{enumerate}
\usepackage{enumitem}
\usepackage{tikz-cd}
\usetikzlibrary{calc}
\usepackage{amsfonts}
%\usepackage{prooftrees}
\usepackage{bussproofs}
\renewcommand{\sectionautorefname}{\S}
\renewcommand{\subsectionautorefname}{\S}
\usepackage{float}

\usepackage{tikz-3dplot}
\usetikzlibrary{3d}
\usetikzlibrary{calligraphy}
\newif\ifshowcellnumber
\showcellnumbertrue

\usepackage{algorithm}
\usepackage{algpseudocode}
\usepackage{algorithmicx}
\usepackage{sourcecodepro}
\usepackage{tikz-qtree}
\usepackage{amsthm}
\usepackage{bm}
\usetikzlibrary{bayesnet}
\usetikzlibrary{arrows}
\usepackage{subcaption}
\usetikzlibrary{backgrounds}
\usetikzlibrary{tikzmark}

\newcommand{\E}{\mathbb{E}}
\newcommand{\Var}{\mathrm{Var}}
\newcommand{\Cov}{\mathrm{Cov}}

\newcommand{\CompOrder}{\mathcal{O}}
\def\graphspace{\mathbf{G}}
\def\Uniform{\mbox{\rm Uniform}}
\def\Gaussian{\mbox{\rm Gaussian}}
\def\Bernoulli{\mbox{\rm Bernoulli}}
\def\Dirichlet{\mbox{\rm Dirichlet}}

\usepackage{mathtools}% superior to amsmath
\usepackage{tikz}
% Packages
\usepackage{listings}
\DeclareRobustCommand{\hlred}[1]{{\sethlcolor{pink}\hl{#1}}}
\usepackage{fontspec}

\setmonofont[Scale=0.8]{JetBrainsMono}[
  Contextuals={Alternate},
  Path=./font/,
  Extension = .ttf,
  UprightFont=*-Regular,
  BoldFont=*-Bold,
  ItalicFont=*-Italic,
  BoldItalicFont=*-BoldItalic
]

\usepackage[skins,breakable,listings]{tcolorbox}

\lstdefinelanguage{kotlin}{
  comment=[l]{//},
  commentstyle={\color{gray}\ttfamily},
  emph={delegate, filter, firstOrNull, forEach, it, lazy, mapNotNull, println, repeat, assert, with, head, tail, len, return@},
  numberstyle=\noncopyable,
  identifierstyle=\color{black},
  keywords={abstract, actual, as, as?, break, by, class, companion, continue, data, do, dynamic, else, enum, expect, false, final, for, fun, get, if, import, in, infix, interface, internal, is, null, object, open, operator, override, package, private, public, return, sealed, set, super, suspend, this, throw, true, try, catch, typealias, val, var, vararg, when, where, while, tailrec, reified},
  keywordstyle={\bfseries},
  morecomment=[s]{/*}{*/},
  morestring=[b]",
  morestring=[s]{"""*}{*"""},
  ndkeywords={@Deprecated, @JvmField, @JvmName, @JvmOverloads, @JvmStatic, @JvmSynthetic, Array, Byte, Double, Float, Boolean, Int, Integer, Iterable, Long, Runnable, Short, String, int},
  ndkeywordstyle={\bfseries},
  sensitive=true,
  stringstyle={\ttfamily},
  literate={`}{{\char0}}1,
  escapeinside={(*@}{@*)}
}
\lstdefinelanguage{tidy}{
  comment=[l]{//},
  commentstyle={\color{gray}\ttfamily},
  emph={|, ->, ---},
  emphstyle={\color{red}},
  identifierstyle=\color{black},
  keywords={\|, ->, ---},
  otherkeywords={|,->},
  morekeywords={|,->},
  keywordstyle={\color{blue}\bfseries},
  morecomment=[s]{/*}{*/},
  morestring=[b]",
  morestring=[s]{"""*}{*"""},
  ndkeywords={@Deprecated, @JvmField, @JvmName, @JvmOverloads, @JvmStatic, @JvmSynthetic, Array, Byte, Double, Float, Int, Integer, Iterable, Long, Runnable, Short, String},
  ndkeywordstyle={\color{orange}\bfseries},
  sensitive=true,
  stringstyle={\color{green}\ttfamily},
  literate={`}{{\char0}}1
}

%%%%%%%%%%%%%%%%%%%%%%%%%%%%%%%%%%%%%%%%%%%
%
% Color boxes
%
%%%%%%%%%%%%%%%%%%%%%%%%%%%%%%%%%%%%%%%%%%%

\tcbset{
  enhanced jigsaw,
  breakable,
  listing only,
%  boxsep=-1pt,
%  top=-1pt,
  bottom=0.1cm,
  right=0.5cm,
  overlay first={
    \node[black!50] (S) at (frame.south) {\Large\ding{34}};
    \draw[dashed,black!50] (frame.south west) -- (S) -- (frame.south east);
  },
  overlay middle={
    \node[black!50] (S) at (frame.south) {\Large\ding{34}};
    \draw[dashed,black!50] (frame.south west) -- (S) -- (frame.south east);
    \node[black!50] (S) at (frame.north) {\Large\ding{34}};
    \draw[dashed,black!50] (frame.north west) -- (S) -- (frame.north east);
  },
  overlay last={
    \node[black!50] (S) at (frame.north) {\Large\ding{34}};
    \draw[dashed,black!50] (frame.north west) -- (S) -- (frame.north east);
  },
  before={\par\vspace{5pt}},
  after={\par\vspace{\parskip}\noindent}
}

\definecolor{slightgray}{rgb}{0.90, 0.90, 0.90}

\usepackage{soul}
\makeatletter
\def\SOUL@hlpreamble{%
  \setul{}{3.0ex}%
  \let\SOUL@stcolor\SOUL@hlcolor%
  \SOUL@stpreamble%
}
\makeatother

\newcommand{\inline}[1]{%
  \begingroup%
  \sethlcolor{slightgray}%
  \hl{\ttfamily\footnotesize #1}%
  \endgroup
}

\newcommand{\tinline}[1]{%
  \begingroup%
  \sethlcolor{slightgray}%
  \hl{\ttfamily\tiny #1}%
  \endgroup
}

\newtcblisting{halftidyinput}[1][]{%
  left skip=0.7cm,
  width=6cm,
%  left=-0.01cm,
  top=-0.1cm,
  bottom=-0.35cm,
  listing options={
    language=tidy,
    basicstyle=\ttfamily\small,
%numberstyle=\footnotesize,
    showstringspaces=false,
    tabsize=2,
    breaklines=true,
    numbers=none,
    inputencoding=utf8,
    escapeinside={(*@}{@*)},
    #1
  },
  underlay unbroken and first={%
    \path[draw=none] (interior.north west) rectangle node[white]{\includegraphics[width=4mm]{../figures/tidyparse_logo.png}} ([xshift=-10mm,yshift=-7mm]interior.north west);
  }
}

\newtcblisting{wholetidyinput}[1][]{%
  left skip=0.7cm,
  top=0.1cm,
  middle=0mm,
  boxsep=0mm,
  listing options={
    language=tidy,
    basicstyle=\ttfamily\small,
%numberstyle=\footnotesize,
    showstringspaces=false,
    tabsize=2,
    breaklines=true,
    numbers=none,
    inputencoding=utf8,
    escapeinside={(*@}{@*)},
    #1
  },
  underlay unbroken and first={%
      \path[draw=none] (interior.north west) rectangle node[white]{\includegraphics[width=4mm]{../figures/tidyparse_logo.png}} ([xshift=-10mm,yshift=-9mm]interior.north west);
  }
}

\definecolor{A}{RGB}{6,150,104}
\definecolor{B}{RGB}{196,74,137}
\definecolor{C}{RGB}{117,237,133}
\definecolor{D}{RGB}{246,46,243}
\definecolor{E}{RGB}{89,162,12}
\definecolor{F}{RGB}{113,12,158}
\definecolor{G}{RGB}{191,205,142}
\definecolor{H}{RGB}{51,58,158}
\definecolor{I}{RGB}{244,212,3}
\definecolor{J}{RGB}{37,36,249}
\definecolor{K}{RGB}{253,165,71}
\definecolor{L}{RGB}{27,81,29}
\colorlet{LA}{A!30}
\colorlet{LB}{B!30}
\colorlet{LC}{C!30}
\colorlet{LD}{D!30}
\colorlet{LE}{E!30}
\colorlet{LF}{F!30}
\colorlet{LG}{G!30}
\colorlet{LH}{H!30}
\colorlet{LI}{I!30}
\colorlet{LJ}{J!30}
\colorlet{LK}{K!30}
\colorlet{LL}{L!30}
\newcommand{\hiliA}[1]{%
  \colorbox{LA}{$#1$}}
\newcommand{\hiliB}[1]{%
  \colorbox{LB}{$#1$}}
\newcommand{\hiliC}[1]{%
  \colorbox{LC}{$#1$}}
\newcommand{\hiliD}[1]{%
  \colorbox{LD}{$#1$}}
\newcommand{\hiliE}[1]{%
  \colorbox{LE}{$#1$}}
\newcommand{\hiliF}[1]{%
  \colorbox{LF}{$#1$}}
\newcommand{\hiliG}[1]{%
  \colorbox{LG}{$#1$}}
\newcommand{\hiliH}[1]{%
  \colorbox{LH}{$#1$}}
\newcommand{\hiliI}[1]{%
  \colorbox{LI}{$#1$}}
\newcommand{\hiliJ}[1]{%
  \colorbox{LJ}{$#1$}}
\newcommand{\hiliK}[1]{%
  \colorbox{LK}{$#1$}}
\newcommand{\hiliL}[1]{%
  \colorbox{LL}{$#1$}}
\newcommand{\highlight}[1]{%
  \colorbox{lgray}{$#1$}}
\colorlet{lred}{red!30}
\colorlet{lorange}{orange!30}
\colorlet{lgreen}{green!30}
\colorlet{lgray}{black!15}
\colorlet{dgray}{black!75}
\DeclareRobustCommand{\hlred}[1]{{\sethlcolor{lred}\hl{#1}}}
\DeclareRobustCommand{\hlorange}[1]{{\sethlcolor{lorange}\hl{#1}}}
\DeclareRobustCommand{\hlgreen}[1]{{\sethlcolor{lgreen}\hl{#1}}}
\DeclareRobustCommand{\hlgray}[1]{{\sethlcolor{lgray}\hl{#1}}}
\DeclareRobustCommand{\caret}[1]{{\sethlcolor{dgray}\textcolor{white}{\hl{#1}}}}

\usepackage{url}
\usepackage{qtree}

\usepackage{filecontents}
\usepackage{pstricks-add}
\usepackage{emoji}
\usepackage{alltt}
\usepackage{nicematrix}
\usepackage{graphicx}
\usepackage{ulem}
\usepackage{upquote}
\tikzstyle{every picture}+=[remember picture]
\usepackage{menukeys}
\pgfplotstableread[col sep=comma,]{timings_loc.csv}\loctimings
\pgfplotstableread[col sep=comma,]{timings_unloc.csv}\unloctimings

\makeatletter
\DeclareRobustCommand{\cev}[1]{%
  {\mathpalette\do@cev{#1}}%
}
\newcommand{\do@cev}[2]{%
  \vbox{\offinterlineskip
  \sbox\z@{$\m@th#1 x$}%
  \ialign{##\cr
  \hidewidth\reflectbox{$\m@th#1\vec{}\mkern4mu$}\hidewidth\cr
  \noalign{\kern-\ht\z@}
    $\m@th#1#2$\cr
  }%
  }%
}
\makeatother

\makeatletter
\DeclareRobustCommand{\pder}[1]{%
  \@ifnextchar\bgroup{\@pder{#1}}{\@pder{}{#1}}}
\newcommand{\@pder}[2]{\frac{\partial#1}{\partial#2}}
\makeatother

\newcommand{\shup}{\shortuparrow}
\newcommand{\shri}{\shortrightarrow}
\newcommand{\shur}{\shup\hspace{-5pt}\shri}

\makeatletter
\def\squiggly{\bgroup \markoverwith{\textcolor{red}{\lower3\p@\hbox{\sixly \char58}}}\ULon}
\makeatother

\newcommand{\err}[1]{\smash{\squiggly{#1}{}}}
\newcommand{\stirlingii}{\genfrac{\{}{\}}{0pt}{}}

%======== Arrows =========
\newcommand{\knightarrow}{
  \tikz{
    \fill (0pt,0pt) circle [radius = 1pt];
    \fill (0pt,6pt) circle [radius = 1pt];
    \fill (6pt,0pt) circle [radius = 1pt];
    \fill (6pt,6pt) circle [radius = 1pt];
    \fill (12pt,0pt) circle [radius = 1pt];
    \fill (12pt,6pt) circle [radius = 1pt];
    \fill (6pt,0pt) circle [radius = 1pt];
    \fill (12pt,0pt) circle [radius = 1pt];
    \draw [-to] (0pt,0pt) -- (12pt,6pt);
  }
}

\newcommand{\kingarrow}{
  \tikz{
    \fill (0pt,0pt) circle [radius = 1pt];
    \fill (6pt,0pt) circle [radius = 1pt];
    \fill (0pt,6pt) circle [radius = 1pt];
    \fill (6pt,6pt) circle [radius = 1pt];
    \draw [-to] (0pt,0pt) -- (6pt,6pt);
    \draw [-to] (0pt,0pt) -- (0pt,6pt);
    \draw [-to] (0pt,0pt) -- (6pt,0pt);
  }
}

\newcommand{\knightkingarrow}{
  \tikz{
    \fill (0pt,0pt) circle [radius = 1pt];
    \fill (0pt,6pt) circle [radius = 1pt];
    \fill (6pt,0pt) circle [radius = 1pt];
    \fill (6pt,6pt) circle [radius = 1pt];
    \fill (12pt,0pt) circle [radius = 1pt];
    \fill (12pt,6pt) circle [radius = 1pt];
    \draw [-to] (0pt,0pt) -- (6pt,6pt);
    \draw [-to] (0pt,0pt) -- (0pt,6pt);
    \draw [-to] (0pt,0pt) -- (6pt,0pt);
    \draw [-to] (0pt,0pt) -- (12pt,6pt);
  }
}

%======== Arrows =========

\usetikzlibrary{decorations.pathreplacing,automata,calc,positioning,matrix,fit}
\usepackage{wrapfig}

\newcommand{\mkTrellis}[1]{
  \begin{tikzpicture}
    \def\dx{20pt}
    \def\dy{30pt}
    \newcounter{i}
    \stepcounter{i}
    \node[circle, draw, fill=black!30] (\arabic{i}) at (0,0){};
    \foreach [count=\i] \x in {2,...,#1}{
      \pgfmathsetmacro{\lox}{\x-1}%
      \pgfmathsetmacro{\loxt}{\x-3}%
      \foreach [count=\j] \xx in {-\lox,-\loxt,...,\lox}{
        \pgfmathsetmacro{\jj}{\j-1}%
        \stepcounter{i}
        \pgfmathsetmacro{\kk}{\xx-2}%
        \pgfmathsetmacro{\lbl}{\lox!/(\jj!*(\lox-\jj)!)}
        \ifnum\x<\kk
        \pgfmath\node[circle, draw]  (\arabic{i}) at (\xx*\dx, -\lox*\dy) {};
        \else
        \pgfmath\node[circle, draw, fill=black!30]  (\arabic{i}) at (\xx*\dx, -\lox*\dy) {};
        \fi
      }
    }
    \newcounter{z}
    \newcounter{xn}
    \newcounter{xnn}
    \pgfmathsetmacro{\maxx}{#1 - 1}
    \foreach \x in {1,...,\maxx}{
      \foreach \xx in {1,...,\x}{
        \stepcounter{z}
        \setcounter{xn}{\arabic{z}}
        \addtocounter{xn}{\x}
        \setcounter{xnn}{\arabic{xn}}
        \stepcounter{xnn}
        \draw [<-] (\arabic{z}) -- (\arabic{xn});
        \draw [<-] (\arabic{z}) -- (\arabic{xnn});
      }
    }
  \end{tikzpicture}
}

\newcommand{\dx}{20pt}
\newcommand{\dy}{30pt}
\newcounter{i}
\newcounter{z}
\newcounter{xn}
\newcounter{xnn}
\newcommand{\mkTrellisAppend}[1]{
  \begin{tikzpicture}
    \setcounter{i}{0}
    \setcounter{z}{0}
    \setcounter{xn}{0}
    \setcounter{xnn}{0}
    \stepcounter{i}
    \node[circle, draw] (\arabic{i}) at (0,0){};
    \foreach [count=\i] \x in {2,...,#1}{
      \pgfmathsetmacro{\lox}{\x-1}%
      \pgfmathsetmacro{\loxt}{\x-3}%
      \foreach [count=\j] \xx in {-\lox,-\loxt,...,\lox}{
        \pgfmathsetmacro{\jj}{\j-1}%
        \stepcounter{i}
        \pgfmathsetmacro{\kk}{\xx+2}%
        \pgfmathsetmacro{\lbl}{\lox!/(\jj!*(\lox-\jj)!)}
        \ifnum\x>\kk
        \pgfmath\node[circle, draw, fill=black!30]  (\arabic{i}) at (\xx*\dx, -\lox*\dy) {};
        \else
        \pgfmath\node[circle, draw]  (\arabic{i}) at (\xx*\dx, -\lox*\dy) {};
        \fi
      }
    }
    \pgfmathsetmacro{\maxx}{#1 - 1}
    \foreach \x in {1,...,\maxx}{
      \foreach \xx in {1,...,\x}{
        \stepcounter{z}
        \setcounter{xn}{\arabic{z}}
        \addtocounter{xn}{\x}
        \setcounter{xnn}{\arabic{xn}}
        \stepcounter{xnn}
        \draw [<-] (\arabic{z}) -- (\arabic{xn});
        \draw [<-] (\arabic{z}) -- (\arabic{xnn});
      }
    }
  \end{tikzpicture}
}

\newcommand{\mkTrellisInsert}[1]{
  \begin{tikzpicture}
    \setcounter{i}{0}
    \setcounter{z}{0}
    \setcounter{xn}{0}
    \setcounter{xnn}{0}
    \stepcounter{i}
    \node[circle, draw] (\arabic{i}) at (0,0){};
    \foreach [count=\i] \x in {2,...,#1}{
      \pgfmathsetmacro{\lox}{\x-1}%
      \pgfmathsetmacro{\loxt}{\x-3}%
      \foreach [count=\j] \xx in {-\lox,-\loxt,...,\lox}{
        \pgfmathsetmacro{\jj}{\j-1}%
        \stepcounter{i}
        \pgfmathsetmacro{\mp}{\xx+#1}%
        \pgfmathsetmacro{\mq}{\xx+\x}%
        \pgfmathsetmacro{\lbl}{\lox!/(\jj!*(\lox-\jj)!)}
        \ifnum\x>\mp
        \pgfmath\node[circle, draw, fill=black!30]  (\arabic{i}) at (\xx*\dx, -\lox*\dy) {};
        \else
        \ifnum#1<\mq
        \pgfmath\node[circle, draw, fill=black!30]  (\arabic{i}) at (\xx*\dx, -\lox*\dy) {};
        \else
        \pgfmath\node[circle, draw]  (\arabic{i}) at (\xx*\dx, -\lox*\dy) {};
        \fi
        \fi

      }
    }
    \pgfmathsetmacro{\maxx}{#1 - 1}
    \foreach \x in {1,...,\maxx}{
      \foreach \xx in {1,...,\x}{
        \stepcounter{z}
        \setcounter{xn}{\arabic{z}}
        \addtocounter{xn}{\x}
        \setcounter{xnn}{\arabic{xn}}
        \stepcounter{xnn}
        \draw [<-] (\arabic{z}) -- (\arabic{xn});
        \draw [<-] (\arabic{z}) -- (\arabic{xnn});
      }
    }
  \end{tikzpicture}
}

\usetikzlibrary{automata, positioning, arrows}

\newcommand{\nobarfrac}{\genfrac{}{}{0pt}{}}
\pgfplotstableread[col sep=comma,]{timings_loc.csv}\loctimings
\pgfplotstableread[col sep=comma,]{timings_unloc.csv}\unloctimings
\pgfplotstableread[col sep=comma,]{natural_errors.csv}\naturalerrors
\pgfplotstableread[col sep=comma,]{synthetic_errors.csv}\syntheticerrors


%\usepackage{draftwatermark}
%\SetWatermarkLightness{0.75}
%\SetWatermarkText{DRAFT}
%\makeatletter
%\let\@authorsaddresses\@empty
%\makeatother

\begin{document}
%
\title{Syntax Repair as Language Intersection}
%
\begin{abstract}
We introduce a new technique for correcting syntax errors in arbitrary context-free languages. Our work addresses the problem of syntax error correction, which we solve by defining a finite language that provably generates every repair within a certain edit distance. To do this, we adapt the Bar-Hillel construction from formal languages, guaranteeing this language is sound and complete with respect to a programming language's grammar. This technique also admits a polylogarithmic time algorithm for deciding intersection nonemptiness between CFLs and acyclic NFAs, the first of its kind in the parsing literature.
\keywords{Error correction \and CFL reachability \and Language games.}
\end{abstract}

%\titlerunning{Abbreviated paper title}
% If the paper title is too long for the running head, you can set
% an abbreviated paper title here
\author{Breandan Considine}
\email{bre@ndan.co}

\maketitle

\section{Introduction}

When programming, one invariably encounters a recurring scenario in which the editor occupies an unparseable state. Faced with this predicament, programmers must spend time to locate and repair the error before proceeding. In the following paper, we propose to solve this problem automatically by generating a list of candidate repairs which contains with high probability the true repair, assuming this repair differs by no more than a few edits from the broken source code.

Prior research on syntax repair can be classified into exact and approximate methods. In the former, rule-based methods are used to locate a suitable alternative. While appealing for their interpretability and well-understood algorithmic properties, these methods are too weak to model the full distribution of natural source code and must rely on relatively brittle heuristics.

In the latter case, the set of all strings is typically used as the sample space for a distribution whose parameters are learned from a dataset of pairwise errors and fixes. Though statistically more robust, large language models typically use approximate inference and thus require some form of postprocessing or rejection sampling to ensure the generated results conform to the grammar.

The primary shortcoming with both of these approaches is they generate too few repairs. Even if the model in question guarantees grammatical soundness or has good statistical generalization properties, it is likely to miss the intended repair in the presence of ambiguity or when there are many candidates from which to choose. Note however, that most syntax errors need relatively minor alterations to repair, of which there are only a finite number of possibilities to consider.

Thus we arrive at the core problem this paper aims to solve: how do we efficiently recover the most probable repairs in close proximity to a syntactically broken code snippet? To address this problem, we propose to exhaustively evaluate every repair within a fixed edit distance.

Our algorithm first constructs an automaton representing all possible strings within a certain edit distance. This is used to compute a matrix denoting all valid repairs in the programming language and edit distance, then we construct a regular expression for that language. Finally, this regular expression is decoded using a fast pretrained statistical model to produce a finite list, which is then reranked and truncated to obtain the final repairs. The full pipeline is depicted in Fig.~\ref{fig:arch_simp}.

\begin{figure}[h!]
  \includegraphics[width=\textwidth]{flow}\vspace{-1pt}
  \caption{Simplified dataflow. Given a grammar and broken code fragment, we create an automaton generating the language of small edits, then construct a regular expression representing the intersection of the two languages. This regular expression can be decoded to produce the final list of repairs.}\label{fig:arch_simp}
\end{figure}

To operationalize this technique, we design, develop and benchmark a new developer tool for syntax repair. This tool makes aggressive use of communication-free parallelism, making it readily executable by off-the-shelf GPU and SIMD co-processors. We provide a reference implementation of our tool on the WebGPU platform and show these computational resources, which typically sit idle during text editing, can be profitably used to accelerate real-time program repair.

Finally, we evaluate our approach on a dataset of human syntax errors and fixes fewer than five lexical edits and shorter than 120 tokens, large enough to fit a few lines of source code in realistic programming languages. Our work shows this technique is highly effective at predicting the true repair across a dataset of Python source code, on average 5x more accurately than previous state of the art methods at comparable latency and compute thresholds.

\section{Background}

Recall that a CFG, $\mathcal{G} = \langle \Sigma, V, P, S\rangle$, is a quadruple consisting of terminals $(\Sigma)$, nonterminals $(V)$, productions $\big(P\colon V \rightarrow (V \mid \Sigma)^+\big)$, and a start symbol, $(S)$. Every CFG is reducible to so-called \textit{Chomsky Normal Form}~\cite{chomsky1959certain}, $P'\colon V \rightarrow (V^2 \mid \Sigma)$, where every production is either (1) a binary production $w \rightarrow xz$, or (2) a unit production $w \rightarrow t$, where $w, x, z: V$ and $t: \Sigma$. For example:\vspace{-3pt}

\begin{table}[H]
  \begin{tabular}{llll}
    $G = \big\{\;S \rightarrow S\:S \mid (\:S\:) \mid (\:)\;\big\} \Longrightarrow G' = \big\{\;S\rightarrow Q\:R \mid S\:S \mid L\:R,$ & $R \rightarrow\:),$ & $L \rightarrow (,$ & $Q\rightarrow L\:S\;\big\}$
  \end{tabular}
\end{table}\vspace{-8pt}

Likewise, a finite state automaton (FSA) is a quintuple $\mathcal{A} = \langle Q, \Sigma, \delta, q_\alpha, F\rangle$, where $Q$ is a finite set of states, $\Sigma$ is a finite alphabet, $\delta \subseteq Q \times \Sigma \times Q$ is the transition function, $q_\alpha$ is the initial state, and $F \subseteq Q$ are the accepting states. These generally come in two varieties, deterministic and nondeterministic depending on whether or not $\delta$ maps each pair $\langle q, s \rangle$ to a unique $q'$.

There is an equivalent characterization of the regular languages via an inductively defined datatype, which is often more elegant than FSAs to work with. Consider the generalized regular expression (GRE) fragment containing concatenation, conjunction and disjunction:

\begin{definition}[Star-free GRE fragment]
  Let \( e \) be an expression defined by the grammar:
  \[
    e \rightarrow \varnothing \mid \varepsilon \mid \Sigma \mid e \cdot e \mid e \lor e \mid e \land e
  \]

where $\varepsilon$ is the empty symbol. Semantically, we interpret these expressions as denoting languages:\vspace{-0.8cm}

  \setlength{\columnseprule}{0pt}
  \setlength{\columnsep}{-3cm}
  \begin{multicols}{2}
    \begin{eqnarray*}
      \mathcal{L}(&\hspace{-0.35cm} \varnothing \hspace{-0.35cm}&) = \varnothing \\
      \mathcal{L}(&\hspace{-0.35cm} \varepsilon \hspace{-0.35cm}&) = \{\varepsilon\} \\
      \mathcal{L}(&\hspace{-0.35cm} a           \hspace{-0.35cm}&) = \{a\}
    \end{eqnarray*} \break\vspace{-0.45cm}
    \begin{eqnarray*}
      \mathcal{L}(&\hspace{-0.35cm} x\cdot z \hspace{-0.35cm}&) = \mathcal{L}(x) \circ \mathcal{L}(z)\text{\footnotemark}\\
      \mathcal{L}(&\hspace{-0.35cm} x\vee  z \hspace{-0.35cm}&) = \mathcal{L}(x) \cup  \mathcal{L}(z)\\
      \mathcal{L}(&\hspace{-0.35cm} x\land z \hspace{-0.35cm}&) = \mathcal{L}(x) \cap  \mathcal{L}(z)
    \end{eqnarray*}
  \end{multicols}
  \footnotetext{Where $\mathcal{L}(x)\circ\mathcal{L}(z)$ is defined as $\big\{a \cdot b \mid a \in \mathcal{L}(x) \land b \in \mathcal{L}(z) \big\}$.}
\end{definition}\vspace{-0.2cm}

\noindent Brzozowski~\cite{brzozowski1964derivatives} introduces an operator, $\partial$, which lets us quotient a language by some prefix,

\begin{definition}[Brzozowski, 1964]
  To compute the quotient \(\partial_a(L) = \{b \mid ab \in L\}\), we:

  \vspace{-0.8cm}
  \begin{multicols}{2}
    \begin{eqnarray*}
      \phantom{--}\partial_a(&\hspace{-0.35cm} \varnothing \hspace{-0.35cm}&) = \varnothing                                           \\
      \phantom{--}\partial_a(&\hspace{-0.35cm} \varepsilon \hspace{-0.35cm}&) = \varnothing                                           \\
      \phantom{--}\partial_a(&\hspace{-0.35cm} b           \hspace{-0.35cm}&) = \begin{cases}\varepsilon &\text{ if } a = b\\ \varnothing &\text{ if } a \neq b \end{cases}\\
      \phantom{--}\partial_a(&\hspace{-0.35cm} x\cdot z    \hspace{-0.35cm}&) = (\partial_a x)\cdot z \vee \delta(x)\cdot\partial_a z \\
      \phantom{--}\partial_a(&\hspace{-0.35cm} x\vee  z    \hspace{-0.35cm}&) =  \partial_a x \vee  \partial_a z                       \\
      \phantom{--}\partial_a(&\hspace{-0.35cm} x\land z    \hspace{-0.35cm}&) =  \partial_a x \land \partial_a z
    \end{eqnarray*} \break\vspace{-0.45cm}
    \begin{eqnarray*}
      \delta(&\hspace{-0.35cm} \varnothing \hspace{-0.35cm}&) = \varnothing                                      \\
      \delta(&\hspace{-0.35cm} \varepsilon \hspace{-0.35cm}&) = \varepsilon                                      \\
      \delta(&\hspace{-0.35cm} a           \hspace{-0.35cm}&) = \varnothing\phantom{\begin{cases}\varepsilon\\\varnothing\end{cases}}\\
      \delta(&\hspace{-0.35cm} x\cdot z    \hspace{-0.35cm}&) = \delta(x) \land \delta(z)                        \\
      \delta(&\hspace{-0.35cm} x\vee  z    \hspace{-0.35cm}&) = \delta(x) \vee  \delta(z)                        \\
      \delta(&\hspace{-0.35cm} x\land z    \hspace{-0.35cm}&) = \delta(x) \land \delta(z)
    \end{eqnarray*}
  \end{multicols}
\end{definition}

Primarily, this gadget was designed to handle membership queries, for which purpose it has received considerable attention~\cite{might2011parsing,adams2016complexity,stanford2021symbolic,varatalu2025re} in recent years:

\begin{theorem}[Recognition]
  For any regex \(e\) and \(\sigma: \Sigma^*\), \(\sigma \in \mathcal{L}(e) \Longleftrightarrow \varepsilon \in \mathcal{L}(\partial_\sigma e)\), where:

  \[
    \partial_\sigma (e): E \rightarrow E = \begin{cases}e &\text{ if } \sigma = \varepsilon\\\partial_b(\partial_a e) &\text{ if } \sigma = a \cdot b, a \in \Sigma, b \in \Sigma^* \end{cases}
  \]
\end{theorem}

Variations on this basic procedure can also be used for functional parsing and regular expression tasks. Less well known, perhaps, is that Brzozowski's derivative can also be used to decode witnesses. We will first focus on the nonempty disjunctive fragment, and define this process in two steps:

\begin{theorem}[Generation]\label{thm:generation}
  For any nonempty $(\varepsilon, \land)$-free regex, \(e\), to witness $\sigma \in \mathcal{L}(e)$:\\

  \hspace{1.6cm}$\texttt{follow}(e): E \rightarrow 2^\Sigma$ = \begin{cases}
   \{e\} &\text{ if } e \in \Sigma \\
   \texttt{follow}(x) &\text{ if } e = x \cdot z\\
   \texttt{follow}(x)\cup\texttt{follow}(z) &\text{ if } e = x \lor z
  \end{cases}\\\\

  \hspace{1.6cm}$\texttt{choose}(e): E \rightarrow \Sigma^+$ = \begin{cases}
   e &\text{ if } e \in \Sigma \\
   \big(s \stackrel{\$}{\gets} \texttt{follow}(e)\big)\cdot \texttt{choose}(\partial_s e) &\text{ if } e = x \cdot z\\
   \texttt{choose}\big(e' \stackrel{\$}{\gets} \{x, z\}\big) &\text{ if } e = x \lor z
  \end{cases}
\end{theorem}

Here, we use the $\stackrel{\$}{\gets}$ operator to denote probabilistic choice, however any deterministic choice function will also suffice to generate a witness. Now we are equipped to handle conjunction.

Recall that every regular language is also context-free a fortiori. So, given an $(\varepsilon, \land)$-free regular expression, we can construct an equivalent CFG with productions $P(e)$ as follows:

\begin{equation}
P(e): E \rightarrow \big(V \rightarrow (\Sigma \mid V \mid V^2)\big) = \begin{cases}
 \{ S_e \rightarrow e \} & \text{if } e \in \Sigma \\
 P(x) \cup P(z) \cup \{ S_e \rightarrow S_x S_z \} & \text{if } e = x \cdot z \\
 P(x) \cup P(z) \cup \{ S_e \rightarrow S_x, S_e \rightarrow S_z \} & \text{if } e = x \lor z \\
\end{cases}
\end{equation}\vspace{0.2cm}

\noindent where the CFG is $G(e) = \langle V, \Sigma, P(e), S_e\rangle$ with $V$ being all nonterminals in $P(e)$. Therefor, to intersect two regular languages, we can treat one of them as a CFL. Alternatively, we can take the intersection between some truly non-regular CFL (say, a programming language syntax) and a regular language.

\begin{theorem}[Bar-Hillel, 1961]
  For any CFG, $G = \langle V, \Sigma, P, S\rangle$, and nondeterministic finite automata (NFA), $A = \langle Q, \Sigma, \delta, q_\alpha, F\rangle$, there is a CFG, \(G_\cap=\langle V_\cap, \Sigma_\cap, P_\cap, S_\cap\rangle\) s.t. $\mathcal{L}(G_\cap) = \mathcal{L}(G)\cap\mathcal{L}(A)$.
\end{theorem}

\noindent Salomaa~\cite{salomaa1973formal} introduces a direct, but inefficient construction for the intersection grammar:

\begin{definition}[Salomaa, 1973]
  One could construct $G_\cap$ like so,

  \noindent\begin{prooftree}
      \hskip -0.5em
      \AxiomC{$q_\omega \in F\vphantom{\overset{a}{\rightarrow}}$}
      \RightLabel{$\mathcal{S}$}
      \UnaryInfC{$\big(S\rightarrow q_\alpha S q_\omega\big) \in P_\cap$}
      \DisplayProof
      \hskip 1em
      \AxiomC{$(w \rightarrow a) \in P$}
      \AxiomC{$(q\overset{a}{\rightarrow}r) \in \delta$}
      \RightLabel{$\uparrow$}
      \BinaryInfC{$\big(qwr\rightarrow a\big)\in P_\cap$}
      \DisplayProof
      \hskip 1em
      \AxiomC{$(w \rightarrow xz) \in P$}
      \AxiomC{$\vphantom{(}p,q,r \in Q\vphantom{\overset{a}{\rightarrow}}$}
      \RightLabel{$\Join$}
      \BinaryInfC{$\big(pwr\rightarrow (pxq)(qzr)\big) \in P_\cap$}
  \end{prooftree}
\end{definition}\vspace{0.2cm}

\noindent however most synthetic productions in $P_\cap$ will be non-generating or unreachable. This method will construct a synthetic production for state pairs which are not even connected by any path, which is clearly excessive. In \S~\ref{sec:method}, we will present a far more efficient construction for the special case when the intersection is finite. But first, let us return to the broader question of syntax repair.% We will instead proceed by considering a simpler problem, then construct a parse chart which efficiently computes the intersection.

\subsection{Informal statement}

Assume there exists a transducer from Unicode tokens to grammatical tokens, $\tau: \Sigma_U^* \rightarrow \Sigma_G^*$. In the compiler nomenclature $\tau$ is called a \textit{lexer} and would typically be regular under mild conditions. In this paper, we do not consider $\tau$ and strictly deal with languages over $\Sigma_G^*$, or simply $\Sigma^*$ for brevity.

%Thus, the full source language can be described as $\tau^{-1}\big(L(G)\big)$

%We designate a special token for tokens which are not recognized by the lexer, which are simply replaced by a hole.

Now suppose we have a syntax, $\ell \subset \Sigma^*$, containing every acceptable program. A syntax error is an unacceptable string, $\err\sigma \notin \ell$, that we wish to repair. We can model syntax repair as a language intersection between a context-free language (CFL) and a regular language. Henceforth, $\err\sigma$ will always and only be used to denote a syntactically invalid string whose target language is known.

\begin{wrapfigure}{r}{0.4\textwidth}
\vspace{-0.3cm}
\resizebox{0.42\textwidth}{!}{
  \def\secondcirclepath{(1.15,0) coordinate (e) circle (2cm)}
  \begin{tikzpicture}[
    dot/.style = {circle, inner sep=0pt, minimum size=1mm, fill,
    node contents={}}
  ]
    \def\firstcircle{(-2.1,0) coordinate (a) circle (2.4cm)}
    \def\firstcirclea{(-2.1,0) coordinate (b) circle (0.6cm)}
    \def\firstcircleb{(-2.1,0) coordinate (c) circle (1.2cm)}
    \def\firstcirclec{(-2.1,0) coordinate (d) circle (1.8cm)}
    \def\secondcircle{(1.2,0) coordinate (e) circle (1.5cm)}

    \begin{scope}
      \clip[decorate, decoration={snake, amplitude=0.6mm, segment length=5.01mm}] \secondcirclepath;
      \fill[black!35] \firstcircle;
    \end{scope}

    \draw \firstcircle node[dot,label=$\err{\sigma}$](z0);
    \draw [dashed] \firstcirclea;
    \draw [dashed] \firstcircleb;
    \draw [dashed] \firstcirclec;
    \draw[-stealth] (-2.1,0) -- (-1.5, 0) node[midway,below]{$d_1$};
    \draw[-stealth] (-1.5,0) -- (-0.9, 0) node[midway,below]{$d_2$};
    \draw[-stealth] (-0.9,0) -- (-0.3, 0) node[midway,below]{$d_3$};
    \draw[-stealth] (-0.3,0) -- (0.3, 0) node[midway,above]{$\tilde{\sigma}$};
    \draw[-stealth] (-0.3,0) -- (0.3, 0) node[midway,below]{$d_4$};

    \draw[decorate, decoration={snake, amplitude=0.6mm, segment length=5.01mm}] \secondcirclepath;
    \node [above] at (current bounding box.north -| a) {$\mathcal{L}\bigl(L(\err\sigma, d^*)\bigr)$};
    \node [above,yshift=2.1cm] at (e) {$\mathcal{L}(G)$};
    \node [above,yshift=1.8cm, xshift=-1.4cm] at (e) {$\ell_\cap$};
  \end{tikzpicture}
}
\vspace{-0.3cm}
\caption{CFL intersection with the local edit region of a given broken code snippet.}
\vspace{-0.2cm}
\end{wrapfigure}

Given a lexical representation of a broken computer program $\err\sigma$ and a grammar $G$, our goal is to find every valid string $\sigma$ consistent with the grammar $G$ and within a certain edit distance, $d$. Consider the language of nearby strings: if intersected with the language of grammatically valid programs, $\mathcal{L}(G)$, the result ($\ell_\cap$) will contain every possible repair within the given edit distance, a subset of which will be natural or statistically probable. If we can locate these repairs then we can map them back into Unicode, adding placeholders for fresh names, numbers, and string literals, then finally apply an off-the-shelf code formatter to display them. Both the preprocessing and the cosmetic postprocessing steps are tangential to this work, in which we confine ourselves to a lexical alphabet.

\subsection{Formal statement}\label{sec:problem}

Let us now restate our informal description of the syntax repair problem in more formal terms.

\begin{definition}[Bounded Levenshtein-CFL reachability]\label{def:bcflr}
Given a CFL, $\ell$, and an invalid string, $\err{\sigma}: \bar\ell$, find every valid string reachable within $d$ edits of $\err{\sigma}$, i.e., letting $\Delta$ be the Levenshtein metric and $\mathcal{L}\big(L(\err\sigma, d)\big) = \{\sigma' \mid \Delta(\err{\sigma}, \sigma') \leq d\}$ be the Levenshtein $d$-ball, we seek to find $\ell_\cap = \mathcal{L}\big(L(\err\sigma, d)\big) \cap \ell$.
\end{definition}

%  To solve this problem, it is convenient to first consider intersections with a finite-length string with holes, then turn our attention back to BCFLR.
%

As the admissible set $\ell_\cap$ is typically under-constrained, we want a procedure which surfaces natural and valid repairs over unnatural but valid repairs:

\begin{definition}[Ranked repair]\label{def:ranked-repair}
Given a finite language $\ell_\cap = \mathcal{L}\big(L(\err\sigma, d)\big) \cap \ell$ and a probabilistic language model $\text{P}_\theta: \Sigma^* \rightarrow [0, 1] \subset \mathbb{R}$, find the top-$k$ maximum probability repairs. That is,
\begin{equation}
R(\ell_\cap, P_\theta): 2^{\Sigma^*} \times (\Sigma^* \rightarrow \mathbb{R}) \rightarrow (\Sigma^*)^{\leq k} = \argmax_{\bm{\sigma} \subseteq \ell_\cap, |\bm{\sigma}| \leq k} \sum_{\sigma \in \bm{\sigma}}\text{P}_\theta(\sigma)
\end{equation}
% On average, across all $G, \sigma$ $\hat{R}$ should approximate $R$.
%    We want a procedure $\hat{R}$, minimizing $\mathbb{E}_{G, \sigma}\big[D_{\text{KL}}(\hat{R} \parallel R)\big]$ and wallclock runtime.
\end{definition}

A popular approach to ranked repair involves learning a distribution over strings, however this is highly sample-inefficient and generalizes poorly to new languages. Approximating a distribution over $\Sigma^*$ forces the model to jointly learn syntax and stylometry. Furthermore, even with an extremely efficient approximate sampler for $\sigma \sim \ell_\cap$, due to the size of the languages involved, it would be intractable to sample either $\ell$ or $\mathcal{L}\big(L(\err\sigma, d)\big)$, reject duplicates, then reject unreachable or invalid edits, and completely out of the question to sample $\sigma \sim \Sigma^*$ as do many neural language models.

As we will demonstrate, the ranked repair problem can be factorized into two steps: first exact representation, then decoding. Instead of working with strings, we will explicitly construct a grammar which soundly and completely generates the set $\ell \cap \mathcal{L}\big(L(\err\sigma, d)\big)$, then retrieve repairs from its language. By ensuring retrieval is sufficiently precise and exhaustive, maximizing probability over the retrieved set can be achieved with a much simpler, syntax-oblivious language model.

%Assuming we have a grammar that recognizes the Levenshtein-CFL intersection, the question then becomes how to maximize the number of unique valid sentences in a given number of samples. Top-down incremental sampling with replacement eventually converges to the language, but does so superlinearly~\cite{flajolet1992birthday}. Due to practical considerations including latency, we require the sampler to converge linearly, ensuring with much higher probability that natural repairs are retrieved in a timely manner. This motivates the need for a specialized generating function. More precisely,
%
%\begin{definition}[Linear convergence]\label{def:linear-convergence}
%Given a finite CFL, $\ell$, we want a randomized generating function, $\bm{\varphi}: \mathbb{N}_{\leq|\ell|} \rightarrow 2^\ell$, whose rate of convergence is linear in expectation, i.e., $\mathbb{E}_{i \in [1, n]}|\bm{\varphi}(i)| \propto n$.
%\end{definition}
%
%\noindent This will ensure that if $|\ell_\cap|$ is sufficiently small and enough samples are drawn, $\bm\varphi$ is sure to include a representative subset, and additionally, will terminate after exhausting all valid repairs.
%
%To satisfy Def.~\ref{def:linear-convergence}, we can construct a bijection from syntax trees to integers (\S~\ref{sec:ptree}), sample integers uniformly without replacement, then decode them as trees. This will produce a set of unique trees, and each tree, assuming grammatical unambiguity, will correspond to a unique sentence in the language.  Finally, sentences can be scored and ranked by likelihood under a language model.
%
%Otherwise, if the grammar, $G_\ell$, is ambiguous, it can be translated into a DFA, then decoded (\S~\ref{sec:decoding}) using an autoregressive language model or any suitably fast scoring function of the implementer's choice. In our case, we use a low-order Markov model for its inference speed, data efficiency, and simplicity. So long as the decoder samples $\ell$ without replacement, it will satisfy Def.~\ref{def:linear-convergence}.

%  Finally, once we have a set of small and valid repairs, the problem of ranked repair reduces to sorting retrieved samples by likelihood, which can be approximated using an autoregressive language model or any suitable scoring function of the implementer's choice.

\clearpage\section{Method}\label{sec:method}

The key to solving this problem is to treat finite language intersections as matrix exponentiation. We will show a certain correspondence between CFL-REG intersections and a semiring algebra that allows us to quickly decide and witness intersection nonemptiness for finite languages.

\begin{theorem}%[Considine, 2025]
  For every CFG, G, and every acyclic NFA (ANFA), $A = \langle Q, \Sigma, \delta, q_\alpha: Q, F \subseteq Q\rangle$, there exists a decision procedure $\Psi: \text{CFG} \rightarrow \text{ANFA} \rightarrow \mathbb{B}$ such that $\Psi(G, A) \models [\mathcal{L}(G)\cap\mathcal{L}(A) \neq \varnothing]$ which requires $\mathcal{O}\big(\log^c |Q||V|\big)$ time using $\mathcal{O}\big((|Q||V|)^k\big)$ parallel processors for some $c, k < \infty$.
\end{theorem}

\begin{proof}[Proof]
  To prove nonemptiness, we must show there exists a path $q_\alpha \rightsquigarrow q_\omega$ in $A$ such that $q_\omega: F$ where $q_\alpha \rightsquigarrow q_\omega \vdash S$. At least one of two cases must hold for $w \in V$ to parse a given $p \rightsquigarrow r$ pair:

  \begin{enumerate}
    \item $p$ steps directly to $r$ in which case it suffices to check $\exists a.\big((p \overset{a}{\rightarrow} r)\in \delta \land (w \rightarrow a) \in P\big)$, or,
    \item there is some midpoint $q \in Q$, $p \rightsquigarrow q \rightsquigarrow r$ such that $\exists x, z.\big((w \rightarrow xz) \in P\land\overbrace{\underbrace{p \rightsquigarrow q}_x, \underbrace{q \rightsquigarrow r}_z}^w\big)$.
  \end{enumerate}

  \noindent This decomposition immediately suggests a dynamic programming solution. Let M be a matrix of type $E^{|Q|\times|Q|\times|V|}$  indexed by $Q$. Since we assumed $\delta$ is acyclic, there exists a topological sort of $\delta$ imposing a total order on $Q$ such that $M$ is strictly upper triangular (SUT). Initiate it thusly:

  \begin{align}
    M_0[r, c, w] = \bigvee_{a\::\:\Sigma} \big\{a \mid (w \rightarrow a) \in P \land (q_r \overset{a}{\rightarrow} q_c)\in \delta\big\}
  \end{align}

  Now, our goal will be to find $M=M^2$ such that $\big[M_0[r, c, w] \neq \varnothing\big] \implies \big[M[r, c, w] \neq \varnothing\big]$ under a certain near-semiring. The algebraic operations $\oplus, \otimes: E^{2|V|} \rightarrow E^{|V|}$ we will define elementwise:

  \begin{equation}
    [\ell \oplus r]_w  = [\ell_w \lor r_w]\hspace{0.5cm}\text{and}\hspace{0.5cm}
    [\ell \otimes r]_w = \bigvee_{\mathclap{x, z\:\in\:V}}\big\{\ell_x \cdot r_z \mid (w \rightarrow xz) \in P\big\}
  \end{equation}

  \noindent By slight abuse of notation,\footnote{Customarily, there is a $\frac{1}{k!}$ factor to suppress exploding entries, but alas this domain has no multiplicative inverse.} we will redefine the matrix exponential over this domain as:

  \begin{align}
    \exp(M) &= \sum_{i = 0}^\infty M_0^i = \sum_{i = 0}^{\mathclap{|Q||V|}} M_0^i \text { (since $M$ is SUT.)}
  \end{align}

  \noindent While $|Q||V|$ is an upper-bound and $\exp(M)$ may converge sooner, incremental evaluation grows expensive even with unbounded parallelism. Instead, we will employ exponentiation-by-squaring:

  \begin{align}
    T(2n) \;=\; \begin{cases}
       M_0, & \text{if } n = 1,\\
       T(n) + T(n)^2 & \text{otherwise}.
    \end{cases}
  \end{align}

  \noindent Therefor, the complexity can be reduced to at most $\lceil\log_2 |Q||V|\rceil$ sequential steps in the limit. Finally, we will union all the languages of every state pair deriving $S$ into a new nonterminal, $S_\cap$.

  \begin{align}
    S_\cap = \bigvee_{\mathclap{\:q_\omega \in F}}\exp(M)[q_\alpha, q_\omega, S] \text{ and } \Psi = [S_\cap \neq \varnothing]
  \end{align}

  \noindent Note we can check $\Psi$ before each recurrence of $T$ and escape immediately thereafter in case of nonemptiness. Should that occur, one may simply $\texttt{choose}(S_\cap)$ to decode a witness (see Thm.~\ref{thm:generation}). In either case, the algorithm provably terminates in $\mathcal{O}\big(\log^c |Q||V|\big)$ parallel time for finite $c$.
\end{proof}\clearpage

\section{Examples}

In this section, we will consider three examples of intersections with finite languages. First, parsing can be viewed as a special case of intersection with a singleton language. Second, we will introduce completion as intersection admitting terminal wildcards in fixed locations. Thirdly, we consider syntax repair, where we will intersect a language representing all possible edit paths within a certain distance to determine the location(s) and fill them with appropriate terminal(s).

\subsection{Recognition as intersection}

In the case of ordinary CFL recognition, the automaton accepts just a single word:

\begin{figure}[H]
\resizebox{0.5\textwidth}{!}{
  \begin{tikzpicture}[>=stealth', node distance=2.5cm, initial text=$ $]
    \node[state, initial]         (00) {$q_{0,0}$};
    \node[state, right of=00]     (10) {$q_{1,0}$};
    \node[state, right of=10, draw=none]     (20) {$\ldots$};
    \node[state, accepting, right of=20] (30) {$q_{n,0}$};

    \draw [->] (00) edge[below] node{$\sigma_1$} (10);
    \draw [->] (10) edge[below] node{$\sigma_2$} (20);
    \draw [->] (20) edge[below] node{$\sigma_n$} (30);
  \end{tikzpicture}
}
\end{figure}

Given a CFG, $G' : \mathcal{G}$ in Chomsky Normal Form (CNF), we can construct a recognizer for strings $\sigma: \Sigma^n$ as follows. Let $2^V$ be our domain, $0$ be $\varnothing$, $\oplus$ be $\cup$, and $\otimes$ be defined as:\vspace{-10pt}

\begin{align}
  X \otimes Z = \big\{\;w \mid \langle x, z\rangle \in X \times Z, (w\rightarrow xz) \in P\;\big\}
\end{align}

\noindent If we define $\hat\sigma_r = \{w \mid (w \rightarrow \sigma_r) \in P\}$, then construct a matrix with nonterminals on the superdiagonal representing each token, $M_0[r+1=c](G', \sigma) = \;\hat\sigma_r$, the fixpoint $M_{i+1} = M_i + M_i^2$ is uniquely determined by the superdiagonal entries. Omitting the exponentiation-by-squaring detail, the ordinary fixedpoint iteration simply fills successive diagonals:\vspace{-10pt}

\begin{align*}
  M_0=
  \begin{pNiceMatrix}[nullify-dots,xdots/line-style=loosely dotted]
    \varnothing & \hat\sigma_1 & \varnothing & \Cdots & \varnothing  \\
    \Vdots      & \Ddots       & \Ddots      & \Ddots & \Vdots       \\
                &              &             &        & \varnothing  \\
                &              &             &        & \hat\sigma_n \\
    \varnothing & \Cdots       &             &        & \varnothing
  \end{pNiceMatrix} & \Rightarrow
  \begin{pNiceMatrix}[nullify-dots,xdots/line-style=loosely dotted]
    \varnothing & \hat\sigma_1 & \Lambda     & \Cdots & \varnothing  \\
    \Vdots      & \Ddots       & \Ddots      & \Ddots & \Vdots       \\
                &              &             &        & \Lambda      \\
                &              &             &        & \hat\sigma_n \\
    \varnothing & \Cdots       &             &        & \varnothing
  \end{pNiceMatrix} & \Rightarrow \ldots \Rightarrow M_\infty =
  \begin{pNiceMatrix}[nullify-dots,xdots/line-style=loosely dotted]
    \varnothing & \hat\sigma_1 & \Lambda     & \Cdots & \Lambda^*_\sigma \\
    \Vdots      & \Ddots       & \Ddots      & \Ddots & \Vdots           \\
                &              &             &        & \Lambda          \\
                &              &             &        & \hat\sigma_n     \\
    \varnothing & \Cdots       &             &        & \varnothing
  \end{pNiceMatrix}
\end{align*}

Once the fixpoint $M_\infty$ is attained, the proposition $[S \in \Lambda^*_\sigma]$~\footnote{Hereinafter, we use Iverson brackets to denote the indicator function of a predicate with free variables, i.e., $[P] \Leftrightarrow \mathds{1}(P)$.} decides language membership, i.e., $[\sigma \in \mathcal{L}(G)]$. So far, this procedure is essentially the textbook CYK algorithm in a linear algebraic notation~\cite{goodman1999semiring} and a well-established technique in the parsing literature~\cite{Grune2008}.

\subsection{Completion as intersection}

We may also consider a problem of intermediate difficulty, wherein we are given a string template admitting edits at fixed locations, which can be filled by any terminal. When intersected with a CFL, this specifies a finite language whose contents are the set of all words consistent with the template. This problem we call \textit{completion}. Formally,

\begin{definition}[Completion]
  Let $\underline\Sigma = \Sigma \cup \{\_\}$, where $\_$ denotes a hole. We denote $\sqsubseteq: \Sigma^n \times \underline\Sigma^n$ as the relation $\{\langle\sigma', \sigma\rangle \mid \sigma_i \in \Sigma \implies \sigma_i' = \sigma_i\}$ and the set of all inhabitants $\{\sigma': \Sigma^+ \mid \sigma' \sqsubseteq \sigma\}$ as $\text{H}(\sigma)$. Given a \textit{porous string}, $\sigma: \underline\Sigma^*$ we seek all syntactically valid inhabitants, i.e., $A(\sigma)=\text{H}(\sigma)\cap\ell$.
\end{definition}

Here, the FSA takes a similar shape but can have multiple arcs between adjacent states, e.g.:

\begin{figure}[H]
  \resizebox{0.5\textwidth}{!}{
    \begin{tikzpicture}[>=stealth', node distance=2.5cm, initial text=$ $]
      \node[state, initial]                (00) {$q_{0,0}$};
      \node[state, right of=00]            (10) {$q_{1,0}$};
      \node[state, right of=10]            (20) {$q_{2,0}$};
      \node[state, accepting, right of=20] (30) {$q_{3,0}$};

      \draw [->] (00) edge[below]             node{$\sigma_1$} (10);
      \draw [->] (10) edge[below]             node{$\ldots$}   (20);
      \draw [->] (10) edge[below, bend left]  node{$\Sigma_1$} (20);
      \draw [->] (10) edge[below, bend right] node{$\Sigma_n$} (20);
      \draw [->] (20) edge[below]             node{$\ldots$}   (30);
      \draw [->] (20) edge[below, bend left]  node{$\Sigma_1$} (30);
      \draw [->] (20) edge[below, bend right] node{$\Sigma_n$} (30);
    \end{tikzpicture}
  }
\end{figure}

\noindent This corresponds to a template with two holes, $\sigma = 1$ \_ \_. Suppose the context-free grammar is $G=\{S\rightarrow N O N, O \rightarrow + \mid \times, N \rightarrow 0 \mid 1\}$. This grammar will first be rewritten into CNF as $G'= \{S \rightarrow N L, N \rightarrow 0 \mid 1, O \rightarrow \times \mid +, L \rightarrow O N\}$. Using the powerset algebra we just defined, the matrix fixpoint $M' = M + M^2$ can be computed as follows, shown in the leftmost column below:\vspace{0.3cm}

\begin{small}
{\renewcommand{\arraystretch}{1.2}
\noindent\phantom{...}\begin{tabular}{|c|c|c|c|}
  \hline
  & $2^V$ & $\mathbb{Z}_2^{|V|}$ & GRE$^{|V|}$\\\hline
  $M_0$ & \begin{pmatrix}
            \phantom{V} & \tiny{\{N\}} &              &              \\
                        &              & \{N,O\}      &              \\
                        &              &              & \{N,O\}      \\
                        &              &              &
  \end{pmatrix} & \begin{pmatrix}
            \phantom{V} & \overset{L}{\ws}\overset{N}{\bs}\overset{O}{\ws}\overset{S}{\ws} &              &              \\
                        &              & \ws\bs\bs\ws &              \\
                        &              &              & \ws\bs\bs\ws \\
                        &              &              &
  \end{pmatrix} & \begin{pmatrix}
            \phantom{V} & E_{0, 1}     &              &              \\
                        &              & E_{1, 2}     &              \\
                        &              &              & E_{2, 3}     \\
                        &              &              &
  \end{pmatrix} \\\hline
  $M_1$ & \begin{pmatrix}
            \phantom{V} & \tiny{\{N\}} & \varnothing  &              \\
                        &              & \{N,O\}      & \{L\}        \\
                        &              &              & \{N,O\}      \\
                        &              &              &
  \end{pmatrix} & \begin{pmatrix}
            \phantom{V} & \ws\bs\ws\ws & \ws\ws\ws\ws &              \\
                        &              & \ws\bs\bs\ws & \bs\ws\ws\ws \\
                        &              &              & \ws\bs\bs\ws \\
                        &              &              &
  \end{pmatrix} & \begin{pmatrix}
            \phantom{V} & E_{0, 1}     & E_{0, 2}     &              \\
                        &              & E_{1, 2}     & E_{1, 3}     \\
                        &              &              & E_{2, 3}     \\
                        &              &              &
  \end{pmatrix} \\\hline
  \begin{tabular}{@{}c@{}}$M_2$\\$=$\\$M_\infty$\end{tabular} & \begin{pmatrix}
            \phantom{V} & \tiny{\{N\}} & \varnothing  & \{S\}        \\
                        &              & \{N,O\}      & \{L\}        \\
                        &              &              & \{N,O\}      \\
                        &              &              &
  \end{pmatrix} & \begin{pmatrix}
            \phantom{V} & \ws\bs\ws\ws & \ws\ws\ws\ws & \ws\ws\ws\bs \\
                        &              & \ws\bs\bs\ws & \bs\ws\ws\ws \\
                        &              &              & \ws\bs\bs\ws \\
                        &              &              &
  \end{pmatrix} & \begin{pmatrix}
            \phantom{V} & E_{0, 1}     & E_{0, 2}     & E_{0, 3}     \\
                        &              & E_{1, 2}     & E_{1, 3}     \\
                        &              &              & E_{2, 3}     \\
                        &              &              &
  \end{pmatrix} \\\hline
\end{tabular}\\
}
\end{small}

\vspace{8pt}The same procedure can be translated, without loss of generality, into the bit domain ($\mathbb{Z}_2^{|V|}$) using a lexicographic nonterminal ordering, however $M_\infty$ in both $2^V$ and $\mathbb{Z}_2^{|V|}$ represents a decision procedure, i.e., $\big[S\in M_\infty[0, 3]\big]\Leftrightarrow \big[M_\infty[0, 3, 3]=\bs\big] \Leftrightarrow \big[A(\sigma) \neq \varnothing\big]$. Since $M_\infty[0, 3] = \{S\}$, we know there is at least one $\sigma' \in A(\sigma)$, but neither $M_\infty$ in $2^V$ or $\mathbb{Z}_2^V$ lets us recover a witness.

%$\{\text{xor}, \land, \top\}$ is a functionally complete set is equivalent to $\mathbb{Z}_2$ $\top := 1, \land := \times, \text{xor} := +$. We can define $=$ as $(a = b) \Leftrightarrow (a \text{ xor } b) \text{ xor } \top \Leftrightarrow (a + b) + \top$.

To witness $\sigma' \in A(\sigma)$, we can translate the matrix exponential to the GRE domain. We first define $X \boxtimes Z = [X_2 \cdot Z_1, \varnothing, \varnothing, X_1 \cdot Z_0]$ and $X \boxplus Z = [X_i \lor Z_i]_{i \in [0, |V|)}$, mirroring $\oplus, \otimes$ from the powerset domain. Since the unit nonterminals $O, N$ can only occur on the superdiagonal, they may be safely ignored by $\boxtimes$. To solve for $M_\infty$, we proceed by first computing $E_{0, 2}, E_{1, 3}$:\vspace{-8pt}

\begin{small}
\begin{align*}
  E_{0, 2} &= E_{0, j} \cdot E_{j, 2} = E_{0, 1} \boxtimes E_{1, 2}                         &  E_{1, 3} &= E_{1, j} \cdot E_{j, 3} = E_{1, 2} \boxtimes E_{2, 3}\\
  &= [L \in E_{0, 2}, \varnothing, \varnothing, S \in E_{0, 2}]                                           &  &= [L \in E_{1, 3}, \varnothing, \varnothing, S \in E_{1, 3}]\\
  &= [O \in E_{0, 1} \cdot N \in E_{1, 2}, \varnothing, \varnothing, N \in E_{0, 1} \cdot L \in E_{1, 2}] &  &= [O \in E_{1, 2} \cdot N \in E_{2, 3}, \varnothing, \varnothing, N \in E_{1, 2} \cdot L \in E_{2, 3}]\\
  &= [E_{0, 1, 2} \cdot E_{1, 2, 1}, \varnothing, \varnothing, E_{0, 1, 1} \cdot E_{1, 2, 0}]             &  &= [E_{1, 2, 2} \cdot E_{2, 3, 1}, \varnothing, \varnothing, E_{1, 2, 1} \cdot E_{2, 3, 0}]
\end{align*}
\end{small}\vspace{-8pt}

\noindent Now we solve for the corner entry $E_{0, 3}$ by dotting the first row and last column, which yields:\vspace{-8pt}

\begin{align*}
  E_{0, 3} &= E_{0, j} \cdot E_{j, 3} = (E_{0, 1} \boxtimes E_{1, 3}) \boxplus (E_{0, 2} \boxtimes E_{2, 3})\\
%  &= [E_{0, 1, 2} \cdot E_{1, 3, 1}, \varnothing, \varnothing, E_{0, 1, 1} \cdot E_{1, 3, 0}] + [E_{0, 2, 2} \cdot E_{2, 3, 1}, \varnothing, \varnothing, E_{0, 2, 1} \cdot E_{2, 3, 0}]\\
  &= [E_{0, 1, 2} \cdot E_{1, 3, 1} \lor E_{0, 2, 2} \cdot E_{2, 3, 1}, \varnothing, \varnothing, E_{0, 1, 1} \cdot E_{1, 3, 0} \lor E_{0, 2, 1} \cdot E_{2, 3, 0}]
\end{align*}

\noindent Since we only care about $E_{0, 3, 3} \Leftrightarrow [S \in E_{0, 3}]$, we can ignore the first three entries and solve for:\vspace{-8pt}

\begin{align*}
  E_{0, 3, 3} &= E_{0, 1, 1} \cdot E_{1, 3, 0} \lor E_{0, 2, 1} \cdot E_{2, 3, 0}\\
  &= E_{0, 1, 1} \cdot (E_{1, 2, 2} \cdot E_{2, 3, 1}) \lor E_{0, 2, 1} \cdot \varnothing\\
  &= E_{0, 1, 1} \cdot E_{1, 2, 2} \cdot E_{2, 3, 1} \big(= [N \in E_{0, 1}] \cdot [O \in E_{1, 2}] \cdot [N \in E_{2, 3}]\big)\\
  &= 1 \cdot \{+, \times\} \cdot \{0, 1\}
\end{align*}

\noindent Finally, to recover a witness, we can simply $\texttt{choose}\big(1 \cdot \{+, \times\} \cdot \{0, 1\}\big)$.

%Now we know that $\sigma =$ 1 \underline{O} \underline{N} is a valid solution, and we can take the product $\{1\}\times \hat\sigma_2^{-1}(O) \times \hat\sigma_3^{-1}(N)$ to recover the inhabitants, yielding $A=\{1+0, 1+1, 1\times 0, 1\times 1\}$. In this case, since $G$ is unambiguous, there is only one parse tree satisfying $V_{0, |\sigma|, 3}$.%, but in general, there can be multiple valid parse trees.

\subsection{Repair as intersection}\label{sec:repair_ex}

Now, we are ready to consider the general case of syntax repair, in which case the edit locations are not localized but can occur anywhere inside the snippet. In this case, we construct a lattice of all possible edit paths up to a fixed distance. This structure is called a Levenshtein automaton.

\begin{wrapfigure}{r}{0.5\textwidth}
  \vspace{-0.3cm}
  \begin{center}
    \begin{minipage}[c]{0.45\textwidth}
  \centering
  \underline{NIA}\vspace{10pt}
  \resizebox{\textwidth}{!}{
  \begin{tikzpicture}[
%->, % makes the edges directed
  >=stealth',
  node distance=2.5cm, % specifies the minimum distance between two nodes. Change if necessary.
%  every state/.style={thick, fill=gray!10}, % sets the properties for each ’state’ node
  initial text=$ $, % sets the text that appears on the start arrow
  ]
  \node[state, initial]                (00) {$q_{0,0}$};
  \node[state, right of=00]            (10) {$q_{1,0}$};
  \node[state, right of=10]            (20) {$q_{2,0}$};
  \node[state, right of=20]            (30) {$q_{3,0}$};
  \node[right of=30]                   (40) {$\vphantom{\vdots}\cdots$};
  \node[accepting, state, right of=40] (n0) {$q_{n,0}$};

  \node[state, above of=00]            (01) {$q_{0,1}$};
  \node[state, right of=01]            (11) {$q_{1,1}$};
  \node[state, right of=11]            (21) {$q_{2,1}$};
  \node[state, right of=21]            (31) {$q_{3,1}$};
  \node[right of=31]                   (41) {$\vphantom{\vdots}\cdots$};
  \node[accepting, state, right of=41] (n1) {$q_{n,1}$};

  \node[above of=01]                   (0j) {$\mathmakebox[\widthof{$\cdots$}]{\vdots}$};
\node[right of=0j]                   (1j) {$\mathmakebox[\widthof{$\cdots$}]{\vdots}$};
\node[right of=1j]                   (2j) {$\mathmakebox[\widthof{$\cdots$}]{\vdots}$};
\node[right of=2j]                   (3j) {$\mathmakebox[\widthof{$\cdots$}]{\vdots}$};
\node[right of=3j]                   (4j) {$\iddots$};
\node[accepting, right of=4j]        (nj) {$\mathmakebox[\widthof{$\cdots$}]{\vdots}$};

\node[state, above of=0j]            (0k) {$q_{0,k}$};
\node[state, right of=0k]            (1k) {$q_{1,k}$};
\node[state, right of=1k]            (2k) {$q_{2,k}$};
\node[state, right of=2k]            (3k) {$q_{3,k}$};
\node[right of=3k]                   (4k) {$\vphantom{\vdots}\cdots$};
\node[accepting, state, right of=4k] (nk) {$q_{n,k}$};

\draw [->] (00) edge[below] node{$\sigma_1$} (10);
\draw [->] (10) edge[below] node{$\sigma_2$} (20);
\draw [->] (20) edge[below] node{$\sigma_3$} (30);
\draw [->] (30) edge[below] node{$\sigma_4$} (40);
\draw [->] (40) edge[below] node{$\sigma_n$} (n0);

\draw [->] (01) edge[below] node{$\sigma_1$} (11);
\draw [->] (11) edge[below] node{$\sigma_2$} (21);
\draw [->] (21) edge[below] node{$\sigma_3$} (31);
\draw [->] (31) edge[below] node{$\sigma_4$} (41);
\draw [->] (41) edge[below] node{$\sigma_n$} (n1);

\draw [->] (0j) edge[below] node{$\sigma_1$} (1j);
\draw [->] (1j) edge[below] node{$\sigma_2$} (2j);
\draw [->] (2j) edge[below] node{$\sigma_3$} (3j);
\draw [->] (3j) edge[below] node{$\sigma_4$} (4j);
\draw [->] (4j) edge[below] node{$\sigma_n$} (nj);

\draw [->] (0k) edge[below] node{$\sigma_1$} (1k);
\draw [->] (1k) edge[below] node{$\sigma_2$} (2k);
\draw [->] (2k) edge[below] node{$\sigma_3$} (3k);
\draw [->] (3k) edge[below] node{$\sigma_4$} (4k);
\draw [->] (4k) edge[below] node{$\sigma_n$} (nk);

\draw [->] (00) edge[left] node{$*$}         (11);
\draw [->] (10) edge[left] node{$*$}         (21);
\draw [->] (20) edge[left] node{$*$}         (31);
\draw [->] (30) edge[left] node{$*$}         (41);
\draw [->] (30) edge[bend right, below] node{$\sigma_5$} (41);
\draw [->] (40) edge[           right] node{$\sigma_n$}  (n1);
\draw [->] (40) edge[bend right, left] node{$*$}         (n1);

\draw [->] (01) edge[left] node{$*$}                     (1j);
\draw [->] (11) edge[left] node{$*$}                     (2j);
\draw [->] (21) edge[left] node{$*$}                     (3j);
\draw [->] (31) edge[left] node{$*$}                     (4j);
\draw [->] (31) edge[bend right, below] node{$\sigma_5$} (4j);
\draw [->] (41) edge[           right] node{$\sigma_n$}  (nj);
\draw [->] (41) edge[bend right, left] node{$*$}         (nj);

\draw [->] (0j) edge[left] node{$*$}                     (1k);
\draw [->] (1j) edge[left] node{$*$}                     (2k);
\draw [->] (2j) edge[left] node{$*$}                     (3k);
\draw [->] (3j) edge[left] node{$*$}                     (4k);
\draw [->] (3j) edge[bend right, below] node{$\sigma_5$} (4k);
\draw [->] (4j) edge[           right] node{$\sigma_n$}  (nk);
\draw [->] (4j) edge[bend right, left] node{$*$}         (nk);

\draw [->] (00) edge[bend left, left] node{$*$}   (01);
\draw [->] (10) edge[bend left, left] node{$*$}   (11);
\draw [->] (20) edge[bend left, left] node{$*$}   (21);
\draw [->] (30) edge[bend left, left] node{$*$}   (31);
\draw [->] (40) edge[right] node{$*$}             (41);
\draw [->] (n0) edge[bend right, right] node{$*$} (n1);

\draw [->] (01) edge[bend left, left] node{$*$}   (0j);
\draw [->] (11) edge[bend left, left] node{$*$}   (1j);
\draw [->] (21) edge[bend left, left] node{$*$}   (2j);
\draw [->] (31) edge[bend left, left] node{$*$}   (3j);
\draw [->] (41) edge[right] node{$*$}             (4j);
\draw [->] (n1) edge[bend right, right] node{$*$} (nj);

\draw [->] (0j) edge[bend left, left] node{$*$}   (0k);
\draw [->] (1j) edge[bend left, left] node{$*$}   (1k);
\draw [->] (2j) edge[bend left, left] node{$*$}   (2k);
\draw [->] (3j) edge[bend left, left] node{$*$}   (3k);
\draw [->] (4j) edge[right] node{$*$}             (4k);
\draw [->] (nj) edge[bend right, right] node{$*$} (nk);

\draw [->] (00) edge[below] node{$\sigma_2$}    (21);
\draw [->] (10) edge[below] node{$\sigma_3$}    (31);
\draw [->] (20) edge[below] node{$\sigma_4$}    (41);

\draw [->] (01) edge[below] node{$\sigma_2$}    (2j);
\draw [->] (11) edge[below] node{$\sigma_3$}    (3j);
\draw [->] (21) edge[below] node{$\sigma_4$}    (4j);

\draw [->] (0j) edge[below] node{$\sigma_2$}    (2k);
\draw [->] (1j) edge[below] node{$\sigma_3$}    (3k);
\draw [->] (2j) edge[below] node{$\sigma_4$}    (4k);

%https://tex.stackexchange.com/a/20986/139648
\draw [decorate,decoration={brace,amplitude=10pt,raise=10pt,mirror}] (00.south west) -- (n0.south east) node[midway,yshift=-3em]{\textbf{String length}};
\draw [decorate,decoration={brace,amplitude=10pt,raise=20pt}] (00.south west) -- (0k.north west) node[midway,xshift=-40pt,rotate=90]{\textbf{Levenshtein edit distance}};
\end{tikzpicture}
}
\end{minipage}
\hfill
\begin{minipage}[l]{5 cm}
\centering
\underline{CFG}\vspace{7pt}
\begin{align*}
S &\Rightarrow \{\cdot \in Q \mid \delta(\cdot, q_{n,0}) \leq k\}\\
* &\Rightarrow \{\cdot \in \Sigma\}\\
\big\{q_{i, j} &\Rightarrow \{q_{i, j-1}*\} \mid i, j \in [1, n]\times[1, k]\big\}\\
\big\{q_{i, j} &\Rightarrow \{q_{i-1, j-1}*\}\mid i, j\in[1, n]\times [1, k]\big\}\\
\big\{q_{i, j} &\Rightarrow \{q_{i-1, j} \sigma_i \}\mid i, j \in [1, n]\times[0, k]\big\} \\
\big\{q_{i, j} &\Rightarrow \{q_{i-2, j-1} \sigma_i\} \mid i, j \in [2, n]\times[1, k] \big\}\\
\end{align*}
\end{minipage}
  \end{center}
  \caption{Levenshtein NFA recognizing $\mathcal{L}\big(L(\sigma: \Sigma^5, 3)\big)$.}\label{fig:lev_nfa}
  \vspace{-0.5cm}
\end{wrapfigure}

As the original construction defined by Schultz and Mihov~\cite{schulz2002fast} contains cycles and $\varepsilon$-transitions, we propose a variant which is $\varepsilon$-free and acyclic. Furthermore, we adopt a nominal form which supports infinite alphabets and simplifies the description to follow. Illustrated in Fig.~\ref{fig:lev_nfa} is an example of a small Levenshtein automaton recognizing $\mathcal{L}\big(L(\sigma: \Sigma^5, 3)\big)$. Unlabeled arcs accept any terminal from the alphabet, $\Sigma$. Equivalently, this transition system can be viewed as a kind of proof system within an unlabeled lattice. The following construction is equivalent to Schultz and Mihov's original Levenshtein automaton, but is more amenable to our purposes as it does not any contain $\varepsilon$-arcs, and instead uses skip connections to recognize consecutive deletions of varying lengths.

\begin{prooftree}
  \AxiomC{$s\in\Sigma \phantom{\land} i \in [0, n] \phantom{\land} j \in [1, d_{\max}]$}
  \RightLabel{$\duparrow$}
  \UnaryInfC{$(q_{i, j-1} \overset{s}{\rightarrow} q_{i,j}) \in \delta$}
  \DisplayProof
  \hskip 1.5em
  \AxiomC{$s\in\Sigma \phantom{\land} i \in [1, n] \phantom{\land} j \in [1, d_{\max}]$}
  \RightLabel{$\ddiagarrow$}
  \UnaryInfC{$(q_{i-1, j-1} \overset{s}{\rightarrow} q_{i,j}) \in \delta$}
\end{prooftree}
\begin{prooftree}
  \AxiomC{$i \in [1, n] \phantom{\land} j \in [0, d_{\max}]$}
  \RightLabel{$\drightarrow$}
  \UnaryInfC{$(q_{i-1, j} \overset{\sigma_i}{\rightarrow} q_{i,j}) \in \delta$}
  \DisplayProof
  \hskip 1.5em
  \AxiomC{$d \in [1, d_{\max}] \phantom{\land} i \in [d + 1, n] \phantom{\land} j \in [d, d_{\max}]$}
  \RightLabel{$\knightarrow$}
  \UnaryInfC{$(q_{i-d-1, j-d} \overset{\sigma_i}{\rightarrow} q_{i,j}) \in \delta$}
\end{prooftree}
\begin{prooftree}
  \AxiomC{$\vphantom{|}$}
  \RightLabel{$\textsc{Init}$}
  \UnaryInfC{$q_{0,0} \in I$}
  \DisplayProof
  \hskip 1.5em
  \AxiomC{$q_{i, j} \in Q$}
  \AxiomC{$|n-i+j| \leq d_{\max}$}
  \RightLabel{$\textsc{Done}$}
  \BinaryInfC{$q_{i, j}\in F$}
\end{prooftree}

\newcommand{\substitutionExample}{
  \tikz{
    \foreach \x in {0,8,16,24,32,40}{
      \fill (\x pt,0pt) circle [radius = 1pt];
      \fill (\x pt,8pt) circle [radius = 1pt];
    }
    \phantom{\fill (0pt,-8pt) circle [radius = 1pt];}
    \draw [-to] (0pt,0pt) -- (8pt,0pt);
    \draw [-to] (8pt,0pt) -- (16pt,0pt);
    \draw [-to] (16pt,0pt) -- (24pt,8pt);
    \draw [-to] (24pt,8pt) -- (32pt,8pt);
    \draw [-to] (32pt,8pt) -- (40pt,8pt);
  }
}

\newcommand{\insertionExample}{
  \tikz{
    \foreach \x in {0,8,16,24,32,40}{
      \fill (\x pt,0pt) circle [radius = 1pt];
      \fill (\x pt,8pt) circle [radius = 1pt];
    }
    \phantom{\fill (0pt,-8pt) circle [radius = 1pt];}
    \fill[white] (16pt,0pt) circle [radius = 1.2pt];
    \fill[white] (24pt,8pt) circle [radius = 1.2pt];
    \draw [-to] (0pt,0pt) -- (8pt,0pt);
    \draw [-to] (8pt,0pt) -- (24pt,0pt);
    \draw [-to] (24pt,0pt) -- (16pt,8pt);
    \draw [-to] (16pt,8pt) -- (32pt,8pt);
    \draw [-to] (32pt,8pt) -- (40pt,8pt);
  }
}

\newcommand{\deletionExample}{
  \tikz{
    \foreach \x in {0,8,16,24,32,40}{
      \fill (\x pt,0pt) circle [radius = 1pt];
      \fill (\x pt,8pt) circle [radius = 1pt];
    }
    \phantom{\fill (0pt,-8pt) circle [radius = 1pt];}
    \draw [-to] (0pt,0pt) -- (8pt,0pt);
    \draw [-to] (8pt,0pt) -- (16pt,0pt);
    \draw [-to] (16pt,0pt) -- (24pt,0pt);
    \draw [-to] (24pt,0pt) -- (40pt,8pt);
  }
}

\newcommand{\doubleDeletionExample}{
  \tikz{
    \foreach \x in {0,8,16,24,32,40}{
      \fill (\x pt,0pt) circle [radius = 1pt];
      \fill (\x pt,8pt) circle [radius = 1pt];
      \fill (\x pt,16pt) circle [radius = 1pt];
    }
    \draw [-to] (0pt,0pt) -- (24pt,16pt);
    \draw [-to] (24pt,16pt) -- (32pt,16pt);
    \draw [-to] (32pt,16pt) -- (40pt,16pt);
  }
}

\newcommand{\subDelExample}{
  \tikz{
    \foreach \x in {0,8,16,24,32,40}{
      \fill (\x pt,0pt) circle [radius = 1pt];
      \fill (\x pt,8pt) circle [radius = 1pt];
      \fill (\x pt,16pt) circle [radius = 1pt];
    }
    \draw [-to] (0pt,0pt) -- (8pt,0pt);
    \draw [-to] (8pt,0pt) -- (16pt,8pt);
    \draw [-to] (16pt,8pt) -- (32pt,16pt);
    \draw [-to] (32pt,16pt) -- (40pt,16pt);
  }
}

\newcommand{\subSubExample}{
  \tikz{
    \foreach \x in {0,8,16,24,32,40}{
      \fill (\x pt,0pt) circle [radius = 1pt];
      \fill (\x pt,8pt) circle [radius = 1pt];
      \fill (\x pt,16pt) circle [radius = 1pt];
    }
    \draw [-to] (0pt,0pt) -- (8pt,0pt);
    \draw [-to] (8pt,0pt) -- (16pt,8pt);
    \draw [-to] (16pt,8pt) -- (24pt,16pt);
    \draw [-to] (24pt,16pt) -- (32pt,16pt);
    \draw [-to] (32pt,16pt) -- (40pt,16pt);
  }
}

\newcommand{\insertDeleteExample}{
  \tikz{
    \foreach \x in {0,8,16,24,32,40,48}{
      \fill (\x pt,0pt) circle [radius = 1pt];
      \fill (\x pt,8pt) circle [radius = 1pt];
      \fill (\x pt,16pt) circle [radius = 1pt];
    }
    \fill[white] (16pt,16pt) circle [radius = 1.2pt];
    \fill[white] (8pt,0pt) circle [radius = 1.2pt];
    \fill[white] (16pt,8pt) circle [radius = 1.2pt];
    \draw [-to] (0pt,0pt) -- (16pt,0pt);
    \draw [-to] (16pt,0pt) -- (8pt,8pt);
    \draw [-to] (8pt,8pt) -- (24pt,8pt);
    \draw [-to] (24pt,8pt) -- (40pt,16pt);
    \draw [-to] (40pt,16pt) -- (48pt,16pt);
  }
}

Each type of arc plays a specific role. $\duparrow$ handles insertions, $\ddiagarrow$ handles substitutions and $\knightarrow$ handles deletions of one or more terminals. Let us consider some illustrative cases.

\begin{table}[h!]
  \begin{tabular}{ccccccc}

    \texttt{f\hspace{3pt}.\hspace{3pt}\hlorange{[}\hspace{3pt}x\hspace{3pt})} &
    \texttt{f\hspace{3pt}.\hspace{3pt}\phantom{(}\hspace{3pt}x\hspace{3pt})} &
    \texttt{f\hspace{3pt}.\hspace{3pt}(\hspace{3pt}\hlred{x}\hspace{3pt})} &
    \texttt{\hlred{.}\hspace{3pt}\hlred{+}\hspace{3pt}(\hspace{3pt}x\hspace{3pt})} &
    \texttt{f\hspace{3pt}\hlorange{.}\hspace{3pt}\hlred{(}\hspace{3pt}x\hspace{3pt};} &
    \texttt{[\hspace{3pt}\hlorange{,}\hspace{3pt}\hlorange{x}\hspace{3pt}y\hspace{3pt}]} &
    \texttt{[\hspace{3pt}\phantom{,}\hspace{3pt},\hspace{3pt}\hlred{x}\hspace{3pt}y\hspace{3pt}]} \\

    \texttt{f\hspace{3pt}.\hspace{3pt}\hlorange{(}\hspace{3pt}x\hspace{3pt})} &
    \texttt{f\hspace{3pt}.\hspace{3pt}\hlgreen{(}\hspace{3pt}x\hspace{3pt})} &
    \texttt{f\hspace{3pt}.\hspace{3pt}(\hspace{3pt}\phantom{x}\hspace{3pt})} &
    \texttt{\phantom{f}\hspace{3pt}\phantom{.}\hspace{3pt}(\hspace{3pt}x\hspace{3pt})} &
    \texttt{f\hspace{3pt}\hlorange{*}\hspace{3pt}\phantom{(}\hspace{3pt}x\hspace{3pt};} &
    \texttt{[\hspace{3pt}\hlorange{x}\hspace{3pt}\hlorange{,}\hspace{3pt}y\hspace{3pt}]} &
    \texttt{[\hspace{3pt}\hlgreen{x}\hspace{3pt},\hspace{3pt}\phantom{x}\hspace{3pt}y\hspace{3pt}]} \\

    \substitutionExample & \insertionExample & \deletionExample & \doubleDeletionExample & \subDelExample & \subSubExample & \insertDeleteExample
  \end{tabular}
\end{table}\vspace{-0.3cm}

Note that the same patch can have multiple Levenshtein alignments. $\textsc{Done}$ constructs the final states, which are all states accepting strings $\sigma'$ whose Levenshtein distance $\Delta(\sigma, \sigma') \leq d_\max$.

To avoid creating a parallel bundle of arcs for each insertion and substitution point, we instead decorate each arc with a nominal predicate, accepting or rejecting $\sigma_i$. To distinguish this nominal variant from the original construction, we highlight the modified rules in orange below.

\begin{prooftree}
  \AxiomC{$i \in [0, n] \phantom{\land} j \in [1, d_{\max}]$}
  \RightLabel{$\duparrow$}
  \UnaryInfC{$(q_{i, j-1} \overset{{\color{orange}[\neq \sigma_{i+1}]}}{\rightarrow} q_{i,j}) \in \delta$}
  \DisplayProof
  \hskip 1.5em
  \AxiomC{$i \in [1, n] \phantom{\land} j \in [1, d_{\max}]$}
  \RightLabel{$\ddiagarrow$}
  \UnaryInfC{$(q_{i-1, j-1} \overset{{\color{orange}[\neq \sigma_i]}}{\rightarrow} q_{i,j}) \in \delta$}
\end{prooftree}
\begin{prooftree}
  \AxiomC{$i \in [1, n] \phantom{\land} j \in [0, d_{\max}]$}
  \RightLabel{$\drightarrow$}
  \UnaryInfC{$(q_{i-1, j} \overset{{\color{orange}[=\sigma_i]}}{\rightarrow} q_{i,j}) \in \delta$}
  \DisplayProof
  \hskip 1.5em
  \AxiomC{$d \in [1, d_{\max}] \phantom{\land} i \in [d + 1, n] \phantom{\land} j \in [d, d_{\max}]$}
  \RightLabel{$\knightarrow$}
  \UnaryInfC{$(q_{i-d-1, j-d} \overset{{\color{orange}[=\sigma_i]}}{\rightarrow} q_{i,j}) \in \delta$}
\end{prooftree}

Nominalizing the NFA eliminates the creation of $2(|\Sigma| - 1)\cdot|\sigma|\cdot d_\max$ unnecessary arcs and drastically reduces the representation size of the Levenshtein automaton, but does not affect the underlying semantics. Thus, it is important to first nominalize the automaton before proceeding.

\begin{wrapfigure}{r}{0.40\textwidth}
\resizebox{0.4\textwidth}{!}{%
\begin{tikzpicture}[
%->, % makes the edges directed
  >=stealth',
  node distance=2.5cm, % specifies the minimum distance between two nodes. Change if necessary.
%  every state/.style={thick, fill=gray!10}, % sets the properties for each ’state’ node
  initial text=$ $, % sets the text that appears on the start arrow
]
  \node[state, initial]                (00) {$q_{0,0}$};
  \node[state, right of=00]            (10) {$q_{1,0}$};
  \node[accepting, state, right of=10] (20) {$q_{2,0}$};
  \node[accepting, state, right of=20] (30) {$q_{3,0}$};

  \node[state, above of=00, shift={(-2cm,0cm)}] (01) {$q_{0,1}$};
  \node[state, right of=01]                     (11) {$q_{1,1}$};
  \node[state, right of=11]                     (21) {$q_{2,1}$};
  \node[accepting, state, right of=21]          (31) {$q_{3,1}$};

  \draw [->] (00) edge[below] node{\tiny{$[= \texttt{(}]$}} (10);
  \draw [->] (10) edge[below] node{\tiny{$[= \texttt{)}]$}} (20);
  \draw [->] (20) edge[below] node{\tiny{$[= \texttt{)}]$}} (30);

  \draw [->] (01) edge[below] node{\tiny{$[= \texttt{(}]$}}                       (11);
  \draw [->] (11) edge[below] node[shift={(-0.2cm,0cm)}]{\tiny{$[= \texttt{)}]$}} (21);
  \draw [->] (21) edge[below] node[shift={(-0.2cm,0cm)}]{\tiny{$[= \texttt{)}]$}} (31);

  \draw [->] (00) edge[left] node{\tiny{$[\neq \texttt{(}]$}} (11);
  \draw [->] (10) edge[left] node{\tiny{$[\neq \texttt{)}]$}} (21);
  \draw [->] (20) edge[left] node{\tiny{$[\neq \texttt{)}]$}} (31);

  \draw [->] (00) edge[bend left=10, left] node{\tiny{$[\neq \texttt{(}]$}} (01);
  \draw [->] (10) edge[bend left=10, left] node{\tiny{$[\neq \texttt{)}]$}} (11);
  \draw [->] (20) edge[bend left=10, left] node{\tiny{$[\neq \texttt{)}]$}} (21);
  \draw [->] (30) edge[bend left=10, left] node{\tiny{$[=.]$}} (31);


  \draw [->, blue] (00) edge[bend right=11,below] node[shift={(0.2cm,0.8cm)}]{\tiny{$[= \texttt{)}]$}}    (21);
  \draw [->, blue] (10) edge[bend right=11,below] node[shift={(0.2cm,0.8cm)}]{\tiny{$[= \texttt{)}]$}}    (31);
\end{tikzpicture}

}
\caption{Simple Levenshtein automaton.}\label{fig:ex_atm}

\vspace{0.3cm}
\resizebox{0.4\textwidth}{!}{%
\[
  \begin{tikzcd}[row sep=1.7em, column sep=1.7em]
  (0,3) \arrow[r]  & (1,3) \arrow[dr] & (2,3) \arrow[r]  & (3,3) \arrow[dr] & (4,3) \\
  (0,2) \arrow[dr] & (1,2) \arrow[ul] & (2,2) \arrow[dr] & (3,2) \arrow[ul] & (4,2) \arrow[u] \\
  (0,1) \arrow[u]  & (1,1) \arrow[dr] & (2,1) \arrow[ul] & (3,1) \arrow[dr] & (4,1) \arrow[ul] \\
  (0,0) \arrow[r]  & (1,0) \arrow[ul] & (2,0) \arrow[r]  & (3,0) \arrow[ul] & (4,0) \arrow[u]
  \end{tikzcd}
\]
}
\caption{Pairing function over $\mathcal{L}\big(L(\sigma: \Sigma^3, 1)\big)$.}\label{fig:pairing_fun}

\vspace{0.3cm}
\begin{center}
\resizebox{0.35\textwidth}{!}{%
\begin{tikzpicture}[x=0.3cm, y=0.3cm, draw=gray, very thin]
  \path[fill=white] (0,19) rectangle ++(1,1);
  \path[fill=black] (1,19) rectangle ++(1,1);
  \path[fill=black] (2,19) rectangle ++(1,1);
  \path[fill=white] (3,19) rectangle ++(1,1);
  \path[fill=black] (4,19) rectangle ++(1,1);
  \path[fill=white] (5,19) rectangle ++(1,1);
  \path[fill=white] (6,19) rectangle ++(1,1);
  \path[fill=white] (7,19) rectangle ++(1,1);
  \path[fill=black] (8,19) rectangle ++(1,1);
  \path[fill=white] (9,19) rectangle ++(1,1);
  \path[fill=white] (10,19) rectangle ++(1,1);
  \path[fill=white] (11,19) rectangle ++(1,1);
  \path[fill=white] (12,19) rectangle ++(1,1);
  \path[fill=white] (13,19) rectangle ++(1,1);
  \path[fill=white] (14,19) rectangle ++(1,1);
  \path[fill=black] (15,19) rectangle ++(1,1);
  \path[fill=white] (16,19) rectangle ++(1,1);
  \path[fill=white] (17,19) rectangle ++(1,1);
  \path[fill=white] (18,19) rectangle ++(1,1);
  \path[fill=black] (19,19) rectangle ++(1,1);
  \path[fill=white] (0,18) rectangle ++(1,1);
  \path[fill=white] (1,18) rectangle ++(1,1);
  \path[fill=white] (2,18) rectangle ++(1,1);
  \path[fill=black] (3,18) rectangle ++(1,1);
  \path[fill=black] (4,18) rectangle ++(1,1);
  \path[fill=white] (5,18) rectangle ++(1,1);
  \path[fill=white] (6,18) rectangle ++(1,1);
  \path[fill=black] (7,18) rectangle ++(1,1);
  \path[fill=white] (8,18) rectangle ++(1,1);
  \path[fill=white] (9,18) rectangle ++(1,1);
  \path[fill=white] (10,18) rectangle ++(1,1);
  \path[fill=black] (11,18) rectangle ++(1,1);
  \path[fill=white] (12,18) rectangle ++(1,1);
  \path[fill=white] (13,18) rectangle ++(1,1);
  \path[fill=white] (14,18) rectangle ++(1,1);
  \path[fill=white] (15,18) rectangle ++(1,1);
  \path[fill=white] (16,18) rectangle ++(1,1);
  \path[fill=black] (17,18) rectangle ++(1,1);
  \path[fill=white] (18,18) rectangle ++(1,1);
  \path[fill=white] (19,18) rectangle ++(1,1);
  \path[fill=white] (0,17) rectangle ++(1,1);
  \path[fill=white] (1,17) rectangle ++(1,1);
  \path[fill=white] (2,17) rectangle ++(1,1);
  \path[fill=white] (3,17) rectangle ++(1,1);
  \path[fill=black] (4,17) rectangle ++(1,1);
  \path[fill=black] (5,17) rectangle ++(1,1);
  \path[fill=white] (6,17) rectangle ++(1,1);
  \path[fill=white] (7,17) rectangle ++(1,1);
  \path[fill=black] (8,17) rectangle ++(1,1);
  \path[fill=white] (9,17) rectangle ++(1,1);
  \path[fill=white] (10,17) rectangle ++(1,1);
  \path[fill=white] (11,17) rectangle ++(1,1);
  \path[fill=black] (12,17) rectangle ++(1,1);
  \path[fill=white] (13,17) rectangle ++(1,1);
  \path[fill=white] (14,17) rectangle ++(1,1);
  \path[fill=white] (15,17) rectangle ++(1,1);
  \path[fill=white] (16,17) rectangle ++(1,1);
  \path[fill=white] (17,17) rectangle ++(1,1);
  \path[fill=black] (18,17) rectangle ++(1,1);
  \path[fill=white] (19,17) rectangle ++(1,1);
  \path[fill=white] (0,16) rectangle ++(1,1);
  \path[fill=white] (1,16) rectangle ++(1,1);
  \path[fill=white] (2,16) rectangle ++(1,1);
  \path[fill=white] (3,16) rectangle ++(1,1);
  \path[fill=white] (4,16) rectangle ++(1,1);
  \path[fill=white] (5,16) rectangle ++(1,1);
  \path[fill=black] (6,16) rectangle ++(1,1);
  \path[fill=black] (7,16) rectangle ++(1,1);
  \path[fill=white] (8,16) rectangle ++(1,1);
  \path[fill=white] (9,16) rectangle ++(1,1);
  \path[fill=black] (10,16) rectangle ++(1,1);
  \path[fill=white] (11,16) rectangle ++(1,1);
  \path[fill=white] (12,16) rectangle ++(1,1);
  \path[fill=white] (13,16) rectangle ++(1,1);
  \path[fill=black] (14,16) rectangle ++(1,1);
  \path[fill=white] (15,16) rectangle ++(1,1);
  \path[fill=white] (16,16) rectangle ++(1,1);
  \path[fill=white] (17,16) rectangle ++(1,1);
  \path[fill=white] (18,16) rectangle ++(1,1);
  \path[fill=white] (19,16) rectangle ++(1,1);
  \path[fill=white] (0,15) rectangle ++(1,1);
  \path[fill=white] (1,15) rectangle ++(1,1);
  \path[fill=white] (2,15) rectangle ++(1,1);
  \path[fill=white] (3,15) rectangle ++(1,1);
  \path[fill=white] (4,15) rectangle ++(1,1);
  \path[fill=white] (5,15) rectangle ++(1,1);
  \path[fill=white] (6,15) rectangle ++(1,1);
  \path[fill=black] (7,15) rectangle ++(1,1);
  \path[fill=black] (8,15) rectangle ++(1,1);
  \path[fill=white] (9,15) rectangle ++(1,1);
  \path[fill=white] (10,15) rectangle ++(1,1);
  \path[fill=black] (11,15) rectangle ++(1,1);
  \path[fill=white] (12,15) rectangle ++(1,1);
  \path[fill=white] (13,15) rectangle ++(1,1);
  \path[fill=white] (14,15) rectangle ++(1,1);
  \path[fill=black] (15,15) rectangle ++(1,1);
  \path[fill=white] (16,15) rectangle ++(1,1);
  \path[fill=white] (17,15) rectangle ++(1,1);
  \path[fill=white] (18,15) rectangle ++(1,1);
  \path[fill=black] (19,15) rectangle ++(1,1);
  \path[fill=white] (0,14) rectangle ++(1,1);
  \path[fill=white] (1,14) rectangle ++(1,1);
  \path[fill=white] (2,14) rectangle ++(1,1);
  \path[fill=white] (3,14) rectangle ++(1,1);
  \path[fill=white] (4,14) rectangle ++(1,1);
  \path[fill=white] (5,14) rectangle ++(1,1);
  \path[fill=white] (6,14) rectangle ++(1,1);
  \path[fill=white] (7,14) rectangle ++(1,1);
  \path[fill=black] (8,14) rectangle ++(1,1);
  \path[fill=black] (9,14) rectangle ++(1,1);
  \path[fill=white] (10,14) rectangle ++(1,1);
  \path[fill=white] (11,14) rectangle ++(1,1);
  \path[fill=black] (12,14) rectangle ++(1,1);
  \path[fill=white] (13,14) rectangle ++(1,1);
  \path[fill=white] (14,14) rectangle ++(1,1);
  \path[fill=white] (15,14) rectangle ++(1,1);
  \path[fill=black] (16,14) rectangle ++(1,1);
  \path[fill=white] (17,14) rectangle ++(1,1);
  \path[fill=white] (18,14) rectangle ++(1,1);
  \path[fill=white] (19,14) rectangle ++(1,1);
  \path[fill=white] (0,13) rectangle ++(1,1);
  \path[fill=white] (1,13) rectangle ++(1,1);
  \path[fill=white] (2,13) rectangle ++(1,1);
  \path[fill=white] (3,13) rectangle ++(1,1);
  \path[fill=white] (4,13) rectangle ++(1,1);
  \path[fill=white] (5,13) rectangle ++(1,1);
  \path[fill=white] (6,13) rectangle ++(1,1);
  \path[fill=white] (7,13) rectangle ++(1,1);
  \path[fill=white] (8,13) rectangle ++(1,1);
  \path[fill=white] (9,13) rectangle ++(1,1);
  \path[fill=black] (10,13) rectangle ++(1,1);
  \path[fill=white] (11,13) rectangle ++(1,1);
  \path[fill=white] (12,13) rectangle ++(1,1);
  \path[fill=white] (13,13) rectangle ++(1,1);
  \path[fill=white] (14,13) rectangle ++(1,1);
  \path[fill=white] (15,13) rectangle ++(1,1);
  \path[fill=white] (16,13) rectangle ++(1,1);
  \path[fill=white] (17,13) rectangle ++(1,1);
  \path[fill=white] (18,13) rectangle ++(1,1);
  \path[fill=white] (19,13) rectangle ++(1,1);
  \path[fill=white] (0,12) rectangle ++(1,1);
  \path[fill=white] (1,12) rectangle ++(1,1);
  \path[fill=white] (2,12) rectangle ++(1,1);
  \path[fill=white] (3,12) rectangle ++(1,1);
  \path[fill=white] (4,12) rectangle ++(1,1);
  \path[fill=white] (5,12) rectangle ++(1,1);
  \path[fill=white] (6,12) rectangle ++(1,1);
  \path[fill=white] (7,12) rectangle ++(1,1);
  \path[fill=white] (8,12) rectangle ++(1,1);
  \path[fill=white] (9,12) rectangle ++(1,1);
  \path[fill=black] (10,12) rectangle ++(1,1);
  \path[fill=black] (11,12) rectangle ++(1,1);
  \path[fill=white] (12,12) rectangle ++(1,1);
  \path[fill=white] (13,12) rectangle ++(1,1);
  \path[fill=black] (14,12) rectangle ++(1,1);
  \path[fill=white] (15,12) rectangle ++(1,1);
  \path[fill=white] (16,12) rectangle ++(1,1);
  \path[fill=black] (17,12) rectangle ++(1,1);
  \path[fill=white] (18,12) rectangle ++(1,1);
  \path[fill=white] (19,12) rectangle ++(1,1);
  \path[fill=white] (0,11) rectangle ++(1,1);
  \path[fill=white] (1,11) rectangle ++(1,1);
  \path[fill=white] (2,11) rectangle ++(1,1);
  \path[fill=white] (3,11) rectangle ++(1,1);
  \path[fill=white] (4,11) rectangle ++(1,1);
  \path[fill=white] (5,11) rectangle ++(1,1);
  \path[fill=white] (6,11) rectangle ++(1,1);
  \path[fill=white] (7,11) rectangle ++(1,1);
  \path[fill=white] (8,11) rectangle ++(1,1);
  \path[fill=white] (9,11) rectangle ++(1,1);
  \path[fill=white] (10,11) rectangle ++(1,1);
  \path[fill=black] (11,11) rectangle ++(1,1);
  \path[fill=black] (12,11) rectangle ++(1,1);
  \path[fill=white] (13,11) rectangle ++(1,1);
  \path[fill=white] (14,11) rectangle ++(1,1);
  \path[fill=black] (15,11) rectangle ++(1,1);
  \path[fill=white] (16,11) rectangle ++(1,1);
  \path[fill=white] (17,11) rectangle ++(1,1);
  \path[fill=black] (18,11) rectangle ++(1,1);
  \path[fill=white] (19,11) rectangle ++(1,1);
  \path[fill=white] (0,10) rectangle ++(1,1);
  \path[fill=white] (1,10) rectangle ++(1,1);
  \path[fill=white] (2,10) rectangle ++(1,1);
  \path[fill=white] (3,10) rectangle ++(1,1);
  \path[fill=white] (4,10) rectangle ++(1,1);
  \path[fill=white] (5,10) rectangle ++(1,1);
  \path[fill=white] (6,10) rectangle ++(1,1);
  \path[fill=white] (7,10) rectangle ++(1,1);
  \path[fill=white] (8,10) rectangle ++(1,1);
  \path[fill=white] (9,10) rectangle ++(1,1);
  \path[fill=white] (10,10) rectangle ++(1,1);
  \path[fill=white] (11,10) rectangle ++(1,1);
  \path[fill=black] (12,10) rectangle ++(1,1);
  \path[fill=black] (13,10) rectangle ++(1,1);
  \path[fill=white] (14,10) rectangle ++(1,1);
  \path[fill=white] (15,10) rectangle ++(1,1);
  \path[fill=black] (16,10) rectangle ++(1,1);
  \path[fill=white] (17,10) rectangle ++(1,1);
  \path[fill=white] (18,10) rectangle ++(1,1);
  \path[fill=white] (19,10) rectangle ++(1,1);
  \path[fill=white] (0,9) rectangle ++(1,1);
  \path[fill=white] (1,9) rectangle ++(1,1);
  \path[fill=white] (2,9) rectangle ++(1,1);
  \path[fill=white] (3,9) rectangle ++(1,1);
  \path[fill=white] (4,9) rectangle ++(1,1);
  \path[fill=white] (5,9) rectangle ++(1,1);
  \path[fill=white] (6,9) rectangle ++(1,1);
  \path[fill=white] (7,9) rectangle ++(1,1);
  \path[fill=white] (8,9) rectangle ++(1,1);
  \path[fill=white] (9,9) rectangle ++(1,1);
  \path[fill=white] (10,9) rectangle ++(1,1);
  \path[fill=white] (11,9) rectangle ++(1,1);
  \path[fill=white] (12,9) rectangle ++(1,1);
  \path[fill=white] (13,9) rectangle ++(1,1);
  \path[fill=black] (14,9) rectangle ++(1,1);
  \path[fill=white] (15,9) rectangle ++(1,1);
  \path[fill=white] (16,9) rectangle ++(1,1);
  \path[fill=white] (17,9) rectangle ++(1,1);
  \path[fill=white] (18,9) rectangle ++(1,1);
  \path[fill=white] (19,9) rectangle ++(1,1);
  \path[fill=white] (0,8) rectangle ++(1,1);
  \path[fill=white] (1,8) rectangle ++(1,1);
  \path[fill=white] (2,8) rectangle ++(1,1);
  \path[fill=white] (3,8) rectangle ++(1,1);
  \path[fill=white] (4,8) rectangle ++(1,1);
  \path[fill=white] (5,8) rectangle ++(1,1);
  \path[fill=white] (6,8) rectangle ++(1,1);
  \path[fill=white] (7,8) rectangle ++(1,1);
  \path[fill=white] (8,8) rectangle ++(1,1);
  \path[fill=white] (9,8) rectangle ++(1,1);
  \path[fill=white] (10,8) rectangle ++(1,1);
  \path[fill=white] (11,8) rectangle ++(1,1);
  \path[fill=white] (12,8) rectangle ++(1,1);
  \path[fill=white] (13,8) rectangle ++(1,1);
  \path[fill=black] (14,8) rectangle ++(1,1);
  \path[fill=black] (15,8) rectangle ++(1,1);
  \path[fill=white] (16,8) rectangle ++(1,1);
  \path[fill=black] (17,8) rectangle ++(1,1);
  \path[fill=white] (18,8) rectangle ++(1,1);
  \path[fill=black] (19,8) rectangle ++(1,1);
  \path[fill=white] (0,7) rectangle ++(1,1);
  \path[fill=white] (1,7) rectangle ++(1,1);
  \path[fill=white] (2,7) rectangle ++(1,1);
  \path[fill=white] (3,7) rectangle ++(1,1);
  \path[fill=white] (4,7) rectangle ++(1,1);
  \path[fill=white] (5,7) rectangle ++(1,1);
  \path[fill=white] (6,7) rectangle ++(1,1);
  \path[fill=white] (7,7) rectangle ++(1,1);
  \path[fill=white] (8,7) rectangle ++(1,1);
  \path[fill=white] (9,7) rectangle ++(1,1);
  \path[fill=white] (10,7) rectangle ++(1,1);
  \path[fill=white] (11,7) rectangle ++(1,1);
  \path[fill=white] (12,7) rectangle ++(1,1);
  \path[fill=white] (13,7) rectangle ++(1,1);
  \path[fill=white] (14,7) rectangle ++(1,1);
  \path[fill=black] (15,7) rectangle ++(1,1);
  \path[fill=black] (16,7) rectangle ++(1,1);
  \path[fill=white] (17,7) rectangle ++(1,1);
  \path[fill=black] (18,7) rectangle ++(1,1);
  \path[fill=white] (19,7) rectangle ++(1,1);
  \path[fill=white] (0,6) rectangle ++(1,1);
  \path[fill=white] (1,6) rectangle ++(1,1);
  \path[fill=white] (2,6) rectangle ++(1,1);
  \path[fill=white] (3,6) rectangle ++(1,1);
  \path[fill=white] (4,6) rectangle ++(1,1);
  \path[fill=white] (5,6) rectangle ++(1,1);
  \path[fill=white] (6,6) rectangle ++(1,1);
  \path[fill=white] (7,6) rectangle ++(1,1);
  \path[fill=white] (8,6) rectangle ++(1,1);
  \path[fill=white] (9,6) rectangle ++(1,1);
  \path[fill=white] (10,6) rectangle ++(1,1);
  \path[fill=white] (11,6) rectangle ++(1,1);
  \path[fill=white] (12,6) rectangle ++(1,1);
  \path[fill=white] (13,6) rectangle ++(1,1);
  \path[fill=white] (14,6) rectangle ++(1,1);
  \path[fill=white] (15,6) rectangle ++(1,1);
  \path[fill=black] (16,6) rectangle ++(1,1);
  \path[fill=white] (17,6) rectangle ++(1,1);
  \path[fill=white] (18,6) rectangle ++(1,1);
  \path[fill=white] (19,6) rectangle ++(1,1);
  \path[fill=white] (0,5) rectangle ++(1,1);
  \path[fill=white] (1,5) rectangle ++(1,1);
  \path[fill=white] (2,5) rectangle ++(1,1);
  \path[fill=white] (3,5) rectangle ++(1,1);
  \path[fill=white] (4,5) rectangle ++(1,1);
  \path[fill=white] (5,5) rectangle ++(1,1);
  \path[fill=white] (6,5) rectangle ++(1,1);
  \path[fill=white] (7,5) rectangle ++(1,1);
  \path[fill=white] (8,5) rectangle ++(1,1);
  \path[fill=white] (9,5) rectangle ++(1,1);
  \path[fill=white] (10,5) rectangle ++(1,1);
  \path[fill=white] (11,5) rectangle ++(1,1);
  \path[fill=white] (12,5) rectangle ++(1,1);
  \path[fill=white] (13,5) rectangle ++(1,1);
  \path[fill=white] (14,5) rectangle ++(1,1);
  \path[fill=white] (15,5) rectangle ++(1,1);
  \path[fill=white] (16,5) rectangle ++(1,1);
  \path[fill=black] (17,5) rectangle ++(1,1);
  \path[fill=white] (18,5) rectangle ++(1,1);
  \path[fill=white] (19,5) rectangle ++(1,1);
  \path[fill=white] (0,4) rectangle ++(1,1);
  \path[fill=white] (1,4) rectangle ++(1,1);
  \path[fill=white] (2,4) rectangle ++(1,1);
  \path[fill=white] (3,4) rectangle ++(1,1);
  \path[fill=white] (4,4) rectangle ++(1,1);
  \path[fill=white] (5,4) rectangle ++(1,1);
  \path[fill=white] (6,4) rectangle ++(1,1);
  \path[fill=white] (7,4) rectangle ++(1,1);
  \path[fill=white] (8,4) rectangle ++(1,1);
  \path[fill=white] (9,4) rectangle ++(1,1);
  \path[fill=white] (10,4) rectangle ++(1,1);
  \path[fill=white] (11,4) rectangle ++(1,1);
  \path[fill=white] (12,4) rectangle ++(1,1);
  \path[fill=white] (13,4) rectangle ++(1,1);
  \path[fill=white] (14,4) rectangle ++(1,1);
  \path[fill=white] (15,4) rectangle ++(1,1);
  \path[fill=white] (16,4) rectangle ++(1,1);
  \path[fill=black] (17,4) rectangle ++(1,1);
  \path[fill=black] (18,4) rectangle ++(1,1);
  \path[fill=black] (19,4) rectangle ++(1,1);
  \path[fill=white] (0,3) rectangle ++(1,1);
  \path[fill=white] (1,3) rectangle ++(1,1);
  \path[fill=white] (2,3) rectangle ++(1,1);
  \path[fill=white] (3,3) rectangle ++(1,1);
  \path[fill=white] (4,3) rectangle ++(1,1);
  \path[fill=white] (5,3) rectangle ++(1,1);
  \path[fill=white] (6,3) rectangle ++(1,1);
  \path[fill=white] (7,3) rectangle ++(1,1);
  \path[fill=white] (8,3) rectangle ++(1,1);
  \path[fill=white] (9,3) rectangle ++(1,1);
  \path[fill=white] (10,3) rectangle ++(1,1);
  \path[fill=white] (11,3) rectangle ++(1,1);
  \path[fill=white] (12,3) rectangle ++(1,1);
  \path[fill=white] (13,3) rectangle ++(1,1);
  \path[fill=white] (14,3) rectangle ++(1,1);
  \path[fill=white] (15,3) rectangle ++(1,1);
  \path[fill=white] (16,3) rectangle ++(1,1);
  \path[fill=white] (17,3) rectangle ++(1,1);
  \path[fill=black] (18,3) rectangle ++(1,1);
  \path[fill=white] (19,3) rectangle ++(1,1);
  \path[fill=white] (0,2) rectangle ++(1,1);
  \path[fill=white] (1,2) rectangle ++(1,1);
  \path[fill=white] (2,2) rectangle ++(1,1);
  \path[fill=white] (3,2) rectangle ++(1,1);
  \path[fill=white] (4,2) rectangle ++(1,1);
  \path[fill=white] (5,2) rectangle ++(1,1);
  \path[fill=white] (6,2) rectangle ++(1,1);
  \path[fill=white] (7,2) rectangle ++(1,1);
  \path[fill=white] (8,2) rectangle ++(1,1);
  \path[fill=white] (9,2) rectangle ++(1,1);
  \path[fill=white] (10,2) rectangle ++(1,1);
  \path[fill=white] (11,2) rectangle ++(1,1);
  \path[fill=white] (12,2) rectangle ++(1,1);
  \path[fill=white] (13,2) rectangle ++(1,1);
  \path[fill=white] (14,2) rectangle ++(1,1);
  \path[fill=white] (15,2) rectangle ++(1,1);
  \path[fill=white] (16,2) rectangle ++(1,1);
  \path[fill=white] (17,2) rectangle ++(1,1);
  \path[fill=white] (18,2) rectangle ++(1,1);
  \path[fill=black] (19,2) rectangle ++(1,1);
  \path[fill=white] (0,1) rectangle ++(1,1);
  \path[fill=white] (1,1) rectangle ++(1,1);
  \path[fill=white] (2,1) rectangle ++(1,1);
  \path[fill=white] (3,1) rectangle ++(1,1);
  \path[fill=white] (4,1) rectangle ++(1,1);
  \path[fill=white] (5,1) rectangle ++(1,1);
  \path[fill=white] (6,1) rectangle ++(1,1);
  \path[fill=white] (7,1) rectangle ++(1,1);
  \path[fill=white] (8,1) rectangle ++(1,1);
  \path[fill=white] (9,1) rectangle ++(1,1);
  \path[fill=white] (10,1) rectangle ++(1,1);
  \path[fill=white] (11,1) rectangle ++(1,1);
  \path[fill=white] (12,1) rectangle ++(1,1);
  \path[fill=white] (13,1) rectangle ++(1,1);
  \path[fill=white] (14,1) rectangle ++(1,1);
  \path[fill=white] (15,1) rectangle ++(1,1);
  \path[fill=white] (16,1) rectangle ++(1,1);
  \path[fill=white] (17,1) rectangle ++(1,1);
  \path[fill=white] (18,1) rectangle ++(1,1);
  \path[fill=black] (19,1) rectangle ++(1,1);
  \path[fill=white] (0,0) rectangle ++(1,1);
  \path[fill=white] (1,0) rectangle ++(1,1);
  \path[fill=white] (2,0) rectangle ++(1,1);
  \path[fill=white] (3,0) rectangle ++(1,1);
  \path[fill=white] (4,0) rectangle ++(1,1);
  \path[fill=white] (5,0) rectangle ++(1,1);
  \path[fill=white] (6,0) rectangle ++(1,1);
  \path[fill=white] (7,0) rectangle ++(1,1);
  \path[fill=white] (8,0) rectangle ++(1,1);
  \path[fill=white] (9,0) rectangle ++(1,1);
  \path[fill=white] (10,0) rectangle ++(1,1);
  \path[fill=white] (11,0) rectangle ++(1,1);
  \path[fill=white] (12,0) rectangle ++(1,1);
  \path[fill=white] (13,0) rectangle ++(1,1);
  \path[fill=white] (14,0) rectangle ++(1,1);
  \path[fill=white] (15,0) rectangle ++(1,1);
  \path[fill=white] (16,0) rectangle ++(1,1);
  \path[fill=white] (17,0) rectangle ++(1,1);
  \path[fill=white] (18,0) rectangle ++(1,1);
  \path[fill=white] (19,0) rectangle ++(1,1);
\end{tikzpicture}

\begin{tikzpicture}[x=0.3cm, y=0.3cm, draw=gray, very thin]
  \path[fill=white] (0,19) rectangle ++(1,1);
  \path[fill=black] (1,19) rectangle ++(1,1);
  \path[fill=black] (2,19) rectangle ++(1,1);
  \path[fill=black] (3,19) rectangle ++(1,1);
  \path[fill=black] (4,19) rectangle ++(1,1);
  \path[fill=white] (5,19) rectangle ++(1,1);
  \path[fill=white] (6,19) rectangle ++(1,1);
  \path[fill=black] (7,19) rectangle ++(1,1);
  \path[fill=black] (8,19) rectangle ++(1,1);
  \path[fill=white] (9,19) rectangle ++(1,1);
  \path[fill=black] (10,19) rectangle ++(1,1);
  \path[fill=black] (11,19) rectangle ++(1,1);
  \path[fill=white] (12,19) rectangle ++(1,1);
  \path[fill=white] (13,19) rectangle ++(1,1);
  \path[fill=black] (14,19) rectangle ++(1,1);
  \path[fill=black] (15,19) rectangle ++(1,1);
  \path[fill=white] (16,19) rectangle ++(1,1);
  \path[fill=black] (17,19) rectangle ++(1,1);
  \path[fill=black] (18,19) rectangle ++(1,1);
  \path[fill=black] (19,19) rectangle ++(1,1);
  \path[fill=white] (0,18) rectangle ++(1,1);
  \path[fill=white] (1,18) rectangle ++(1,1);
  \path[fill=white] (2,18) rectangle ++(1,1);
  \path[fill=black] (3,18) rectangle ++(1,1);
  \path[fill=black] (4,18) rectangle ++(1,1);
  \path[fill=white] (5,18) rectangle ++(1,1);
  \path[fill=black] (6,18) rectangle ++(1,1);
  \path[fill=black] (7,18) rectangle ++(1,1);
  \path[fill=white] (8,18) rectangle ++(1,1);
  \path[fill=white] (9,18) rectangle ++(1,1);
  \path[fill=black] (10,18) rectangle ++(1,1);
  \path[fill=black] (11,18) rectangle ++(1,1);
  \path[fill=white] (12,18) rectangle ++(1,1);
  \path[fill=white] (13,18) rectangle ++(1,1);
  \path[fill=black] (14,18) rectangle ++(1,1);
  \path[fill=white] (15,18) rectangle ++(1,1);
  \path[fill=white] (16,18) rectangle ++(1,1);
  \path[fill=black] (17,18) rectangle ++(1,1);
  \path[fill=white] (18,18) rectangle ++(1,1);
  \path[fill=black] (19,18) rectangle ++(1,1);
  \path[fill=white] (0,17) rectangle ++(1,1);
  \path[fill=white] (1,17) rectangle ++(1,1);
  \path[fill=white] (2,17) rectangle ++(1,1);
  \path[fill=white] (3,17) rectangle ++(1,1);
  \path[fill=black] (4,17) rectangle ++(1,1);
  \path[fill=black] (5,17) rectangle ++(1,1);
  \path[fill=white] (6,17) rectangle ++(1,1);
  \path[fill=black] (7,17) rectangle ++(1,1);
  \path[fill=black] (8,17) rectangle ++(1,1);
  \path[fill=white] (9,17) rectangle ++(1,1);
  \path[fill=white] (10,17) rectangle ++(1,1);
  \path[fill=black] (11,17) rectangle ++(1,1);
  \path[fill=black] (12,17) rectangle ++(1,1);
  \path[fill=white] (13,17) rectangle ++(1,1);
  \path[fill=black] (14,17) rectangle ++(1,1);
  \path[fill=black] (15,17) rectangle ++(1,1);
  \path[fill=white] (16,17) rectangle ++(1,1);
  \path[fill=black] (17,17) rectangle ++(1,1);
  \path[fill=black] (18,17) rectangle ++(1,1);
  \path[fill=black] (19,17) rectangle ++(1,1);
  \path[fill=white] (0,16) rectangle ++(1,1);
  \path[fill=white] (1,16) rectangle ++(1,1);
  \path[fill=white] (2,16) rectangle ++(1,1);
  \path[fill=white] (3,16) rectangle ++(1,1);
  \path[fill=white] (4,16) rectangle ++(1,1);
  \path[fill=white] (5,16) rectangle ++(1,1);
  \path[fill=black] (6,16) rectangle ++(1,1);
  \path[fill=black] (7,16) rectangle ++(1,1);
  \path[fill=white] (8,16) rectangle ++(1,1);
  \path[fill=white] (9,16) rectangle ++(1,1);
  \path[fill=black] (10,16) rectangle ++(1,1);
  \path[fill=white] (11,16) rectangle ++(1,1);
  \path[fill=white] (12,16) rectangle ++(1,1);
  \path[fill=white] (13,16) rectangle ++(1,1);
  \path[fill=black] (14,16) rectangle ++(1,1);
  \path[fill=white] (15,16) rectangle ++(1,1);
  \path[fill=white] (16,16) rectangle ++(1,1);
  \path[fill=white] (17,16) rectangle ++(1,1);
  \path[fill=white] (18,16) rectangle ++(1,1);
  \path[fill=white] (19,16) rectangle ++(1,1);
  \path[fill=white] (0,15) rectangle ++(1,1);
  \path[fill=white] (1,15) rectangle ++(1,1);
  \path[fill=white] (2,15) rectangle ++(1,1);
  \path[fill=white] (3,15) rectangle ++(1,1);
  \path[fill=white] (4,15) rectangle ++(1,1);
  \path[fill=white] (5,15) rectangle ++(1,1);
  \path[fill=white] (6,15) rectangle ++(1,1);
  \path[fill=black] (7,15) rectangle ++(1,1);
  \path[fill=black] (8,15) rectangle ++(1,1);
  \path[fill=white] (9,15) rectangle ++(1,1);
  \path[fill=black] (10,15) rectangle ++(1,1);
  \path[fill=black] (11,15) rectangle ++(1,1);
  \path[fill=white] (12,15) rectangle ++(1,1);
  \path[fill=white] (13,15) rectangle ++(1,1);
  \path[fill=black] (14,15) rectangle ++(1,1);
  \path[fill=black] (15,15) rectangle ++(1,1);
  \path[fill=white] (16,15) rectangle ++(1,1);
  \path[fill=black] (17,15) rectangle ++(1,1);
  \path[fill=black] (18,15) rectangle ++(1,1);
  \path[fill=black] (19,15) rectangle ++(1,1);
  \path[fill=white] (0,14) rectangle ++(1,1);
  \path[fill=white] (1,14) rectangle ++(1,1);
  \path[fill=white] (2,14) rectangle ++(1,1);
  \path[fill=white] (3,14) rectangle ++(1,1);
  \path[fill=white] (4,14) rectangle ++(1,1);
  \path[fill=white] (5,14) rectangle ++(1,1);
  \path[fill=white] (6,14) rectangle ++(1,1);
  \path[fill=white] (7,14) rectangle ++(1,1);
  \path[fill=black] (8,14) rectangle ++(1,1);
  \path[fill=black] (9,14) rectangle ++(1,1);
  \path[fill=white] (10,14) rectangle ++(1,1);
  \path[fill=black] (11,14) rectangle ++(1,1);
  \path[fill=black] (12,14) rectangle ++(1,1);
  \path[fill=white] (13,14) rectangle ++(1,1);
  \path[fill=white] (14,14) rectangle ++(1,1);
  \path[fill=black] (15,14) rectangle ++(1,1);
  \path[fill=black] (16,14) rectangle ++(1,1);
  \path[fill=black] (17,14) rectangle ++(1,1);
  \path[fill=black] (18,14) rectangle ++(1,1);
  \path[fill=black] (19,14) rectangle ++(1,1);
  \path[fill=white] (0,13) rectangle ++(1,1);
  \path[fill=white] (1,13) rectangle ++(1,1);
  \path[fill=white] (2,13) rectangle ++(1,1);
  \path[fill=white] (3,13) rectangle ++(1,1);
  \path[fill=white] (4,13) rectangle ++(1,1);
  \path[fill=white] (5,13) rectangle ++(1,1);
  \path[fill=white] (6,13) rectangle ++(1,1);
  \path[fill=white] (7,13) rectangle ++(1,1);
  \path[fill=white] (8,13) rectangle ++(1,1);
  \path[fill=white] (9,13) rectangle ++(1,1);
  \path[fill=black] (10,13) rectangle ++(1,1);
  \path[fill=white] (11,13) rectangle ++(1,1);
  \path[fill=white] (12,13) rectangle ++(1,1);
  \path[fill=white] (13,13) rectangle ++(1,1);
  \path[fill=white] (14,13) rectangle ++(1,1);
  \path[fill=white] (15,13) rectangle ++(1,1);
  \path[fill=white] (16,13) rectangle ++(1,1);
  \path[fill=white] (17,13) rectangle ++(1,1);
  \path[fill=white] (18,13) rectangle ++(1,1);
  \path[fill=white] (19,13) rectangle ++(1,1);
  \path[fill=white] (0,12) rectangle ++(1,1);
  \path[fill=white] (1,12) rectangle ++(1,1);
  \path[fill=white] (2,12) rectangle ++(1,1);
  \path[fill=white] (3,12) rectangle ++(1,1);
  \path[fill=white] (4,12) rectangle ++(1,1);
  \path[fill=white] (5,12) rectangle ++(1,1);
  \path[fill=white] (6,12) rectangle ++(1,1);
  \path[fill=white] (7,12) rectangle ++(1,1);
  \path[fill=white] (8,12) rectangle ++(1,1);
  \path[fill=white] (9,12) rectangle ++(1,1);
  \path[fill=black] (10,12) rectangle ++(1,1);
  \path[fill=black] (11,12) rectangle ++(1,1);
  \path[fill=white] (12,12) rectangle ++(1,1);
  \path[fill=white] (13,12) rectangle ++(1,1);
  \path[fill=black] (14,12) rectangle ++(1,1);
  \path[fill=white] (15,12) rectangle ++(1,1);
  \path[fill=white] (16,12) rectangle ++(1,1);
  \path[fill=black] (17,12) rectangle ++(1,1);
  \path[fill=white] (18,12) rectangle ++(1,1);
  \path[fill=white] (19,12) rectangle ++(1,1);
  \path[fill=white] (0,11) rectangle ++(1,1);
  \path[fill=white] (1,11) rectangle ++(1,1);
  \path[fill=white] (2,11) rectangle ++(1,1);
  \path[fill=white] (3,11) rectangle ++(1,1);
  \path[fill=white] (4,11) rectangle ++(1,1);
  \path[fill=white] (5,11) rectangle ++(1,1);
  \path[fill=white] (6,11) rectangle ++(1,1);
  \path[fill=white] (7,11) rectangle ++(1,1);
  \path[fill=white] (8,11) rectangle ++(1,1);
  \path[fill=white] (9,11) rectangle ++(1,1);
  \path[fill=white] (10,11) rectangle ++(1,1);
  \path[fill=black] (11,11) rectangle ++(1,1);
  \path[fill=black] (12,11) rectangle ++(1,1);
  \path[fill=white] (13,11) rectangle ++(1,1);
  \path[fill=black] (14,11) rectangle ++(1,1);
  \path[fill=black] (15,11) rectangle ++(1,1);
  \path[fill=white] (16,11) rectangle ++(1,1);
  \path[fill=black] (17,11) rectangle ++(1,1);
  \path[fill=black] (18,11) rectangle ++(1,1);
  \path[fill=black] (19,11) rectangle ++(1,1);
  \path[fill=white] (0,10) rectangle ++(1,1);
  \path[fill=white] (1,10) rectangle ++(1,1);
  \path[fill=white] (2,10) rectangle ++(1,1);
  \path[fill=white] (3,10) rectangle ++(1,1);
  \path[fill=white] (4,10) rectangle ++(1,1);
  \path[fill=white] (5,10) rectangle ++(1,1);
  \path[fill=white] (6,10) rectangle ++(1,1);
  \path[fill=white] (7,10) rectangle ++(1,1);
  \path[fill=white] (8,10) rectangle ++(1,1);
  \path[fill=white] (9,10) rectangle ++(1,1);
  \path[fill=white] (10,10) rectangle ++(1,1);
  \path[fill=white] (11,10) rectangle ++(1,1);
  \path[fill=black] (12,10) rectangle ++(1,1);
  \path[fill=black] (13,10) rectangle ++(1,1);
  \path[fill=white] (14,10) rectangle ++(1,1);
  \path[fill=black] (15,10) rectangle ++(1,1);
  \path[fill=black] (16,10) rectangle ++(1,1);
  \path[fill=white] (17,10) rectangle ++(1,1);
  \path[fill=black] (18,10) rectangle ++(1,1);
  \path[fill=black] (19,10) rectangle ++(1,1);
  \path[fill=white] (0,9) rectangle ++(1,1);
  \path[fill=white] (1,9) rectangle ++(1,1);
  \path[fill=white] (2,9) rectangle ++(1,1);
  \path[fill=white] (3,9) rectangle ++(1,1);
  \path[fill=white] (4,9) rectangle ++(1,1);
  \path[fill=white] (5,9) rectangle ++(1,1);
  \path[fill=white] (6,9) rectangle ++(1,1);
  \path[fill=white] (7,9) rectangle ++(1,1);
  \path[fill=white] (8,9) rectangle ++(1,1);
  \path[fill=white] (9,9) rectangle ++(1,1);
  \path[fill=white] (10,9) rectangle ++(1,1);
  \path[fill=white] (11,9) rectangle ++(1,1);
  \path[fill=white] (12,9) rectangle ++(1,1);
  \path[fill=white] (13,9) rectangle ++(1,1);
  \path[fill=black] (14,9) rectangle ++(1,1);
  \path[fill=white] (15,9) rectangle ++(1,1);
  \path[fill=white] (16,9) rectangle ++(1,1);
  \path[fill=white] (17,9) rectangle ++(1,1);
  \path[fill=white] (18,9) rectangle ++(1,1);
  \path[fill=white] (19,9) rectangle ++(1,1);
  \path[fill=white] (0,8) rectangle ++(1,1);
  \path[fill=white] (1,8) rectangle ++(1,1);
  \path[fill=white] (2,8) rectangle ++(1,1);
  \path[fill=white] (3,8) rectangle ++(1,1);
  \path[fill=white] (4,8) rectangle ++(1,1);
  \path[fill=white] (5,8) rectangle ++(1,1);
  \path[fill=white] (6,8) rectangle ++(1,1);
  \path[fill=white] (7,8) rectangle ++(1,1);
  \path[fill=white] (8,8) rectangle ++(1,1);
  \path[fill=white] (9,8) rectangle ++(1,1);
  \path[fill=white] (10,8) rectangle ++(1,1);
  \path[fill=white] (11,8) rectangle ++(1,1);
  \path[fill=white] (12,8) rectangle ++(1,1);
  \path[fill=white] (13,8) rectangle ++(1,1);
  \path[fill=black] (14,8) rectangle ++(1,1);
  \path[fill=black] (15,8) rectangle ++(1,1);
  \path[fill=white] (16,8) rectangle ++(1,1);
  \path[fill=black] (17,8) rectangle ++(1,1);
  \path[fill=white] (18,8) rectangle ++(1,1);
  \path[fill=black] (19,8) rectangle ++(1,1);
  \path[fill=white] (0,7) rectangle ++(1,1);
  \path[fill=white] (1,7) rectangle ++(1,1);
  \path[fill=white] (2,7) rectangle ++(1,1);
  \path[fill=white] (3,7) rectangle ++(1,1);
  \path[fill=white] (4,7) rectangle ++(1,1);
  \path[fill=white] (5,7) rectangle ++(1,1);
  \path[fill=white] (6,7) rectangle ++(1,1);
  \path[fill=white] (7,7) rectangle ++(1,1);
  \path[fill=white] (8,7) rectangle ++(1,1);
  \path[fill=white] (9,7) rectangle ++(1,1);
  \path[fill=white] (10,7) rectangle ++(1,1);
  \path[fill=white] (11,7) rectangle ++(1,1);
  \path[fill=white] (12,7) rectangle ++(1,1);
  \path[fill=white] (13,7) rectangle ++(1,1);
  \path[fill=white] (14,7) rectangle ++(1,1);
  \path[fill=black] (15,7) rectangle ++(1,1);
  \path[fill=black] (16,7) rectangle ++(1,1);
  \path[fill=black] (17,7) rectangle ++(1,1);
  \path[fill=black] (18,7) rectangle ++(1,1);
  \path[fill=black] (19,7) rectangle ++(1,1);
  \path[fill=white] (0,6) rectangle ++(1,1);
  \path[fill=white] (1,6) rectangle ++(1,1);
  \path[fill=white] (2,6) rectangle ++(1,1);
  \path[fill=white] (3,6) rectangle ++(1,1);
  \path[fill=white] (4,6) rectangle ++(1,1);
  \path[fill=white] (5,6) rectangle ++(1,1);
  \path[fill=white] (6,6) rectangle ++(1,1);
  \path[fill=white] (7,6) rectangle ++(1,1);
  \path[fill=white] (8,6) rectangle ++(1,1);
  \path[fill=white] (9,6) rectangle ++(1,1);
  \path[fill=white] (10,6) rectangle ++(1,1);
  \path[fill=white] (11,6) rectangle ++(1,1);
  \path[fill=white] (12,6) rectangle ++(1,1);
  \path[fill=white] (13,6) rectangle ++(1,1);
  \path[fill=white] (14,6) rectangle ++(1,1);
  \path[fill=white] (15,6) rectangle ++(1,1);
  \path[fill=black] (16,6) rectangle ++(1,1);
  \path[fill=white] (17,6) rectangle ++(1,1);
  \path[fill=black] (18,6) rectangle ++(1,1);
  \path[fill=white] (19,6) rectangle ++(1,1);
  \path[fill=white] (0,5) rectangle ++(1,1);
  \path[fill=white] (1,5) rectangle ++(1,1);
  \path[fill=white] (2,5) rectangle ++(1,1);
  \path[fill=white] (3,5) rectangle ++(1,1);
  \path[fill=white] (4,5) rectangle ++(1,1);
  \path[fill=white] (5,5) rectangle ++(1,1);
  \path[fill=white] (6,5) rectangle ++(1,1);
  \path[fill=white] (7,5) rectangle ++(1,1);
  \path[fill=white] (8,5) rectangle ++(1,1);
  \path[fill=white] (9,5) rectangle ++(1,1);
  \path[fill=white] (10,5) rectangle ++(1,1);
  \path[fill=white] (11,5) rectangle ++(1,1);
  \path[fill=white] (12,5) rectangle ++(1,1);
  \path[fill=white] (13,5) rectangle ++(1,1);
  \path[fill=white] (14,5) rectangle ++(1,1);
  \path[fill=white] (15,5) rectangle ++(1,1);
  \path[fill=white] (16,5) rectangle ++(1,1);
  \path[fill=black] (17,5) rectangle ++(1,1);
  \path[fill=white] (18,5) rectangle ++(1,1);
  \path[fill=white] (19,5) rectangle ++(1,1);
  \path[fill=white] (0,4) rectangle ++(1,1);
  \path[fill=white] (1,4) rectangle ++(1,1);
  \path[fill=white] (2,4) rectangle ++(1,1);
  \path[fill=white] (3,4) rectangle ++(1,1);
  \path[fill=white] (4,4) rectangle ++(1,1);
  \path[fill=white] (5,4) rectangle ++(1,1);
  \path[fill=white] (6,4) rectangle ++(1,1);
  \path[fill=white] (7,4) rectangle ++(1,1);
  \path[fill=white] (8,4) rectangle ++(1,1);
  \path[fill=white] (9,4) rectangle ++(1,1);
  \path[fill=white] (10,4) rectangle ++(1,1);
  \path[fill=white] (11,4) rectangle ++(1,1);
  \path[fill=white] (12,4) rectangle ++(1,1);
  \path[fill=white] (13,4) rectangle ++(1,1);
  \path[fill=white] (14,4) rectangle ++(1,1);
  \path[fill=white] (15,4) rectangle ++(1,1);
  \path[fill=white] (16,4) rectangle ++(1,1);
  \path[fill=black] (17,4) rectangle ++(1,1);
  \path[fill=black] (18,4) rectangle ++(1,1);
  \path[fill=black] (19,4) rectangle ++(1,1);
  \path[fill=white] (0,3) rectangle ++(1,1);
  \path[fill=white] (1,3) rectangle ++(1,1);
  \path[fill=white] (2,3) rectangle ++(1,1);
  \path[fill=white] (3,3) rectangle ++(1,1);
  \path[fill=white] (4,3) rectangle ++(1,1);
  \path[fill=white] (5,3) rectangle ++(1,1);
  \path[fill=white] (6,3) rectangle ++(1,1);
  \path[fill=white] (7,3) rectangle ++(1,1);
  \path[fill=white] (8,3) rectangle ++(1,1);
  \path[fill=white] (9,3) rectangle ++(1,1);
  \path[fill=white] (10,3) rectangle ++(1,1);
  \path[fill=white] (11,3) rectangle ++(1,1);
  \path[fill=white] (12,3) rectangle ++(1,1);
  \path[fill=white] (13,3) rectangle ++(1,1);
  \path[fill=white] (14,3) rectangle ++(1,1);
  \path[fill=white] (15,3) rectangle ++(1,1);
  \path[fill=white] (16,3) rectangle ++(1,1);
  \path[fill=white] (17,3) rectangle ++(1,1);
  \path[fill=black] (18,3) rectangle ++(1,1);
  \path[fill=black] (19,3) rectangle ++(1,1);
  \path[fill=white] (0,2) rectangle ++(1,1);
  \path[fill=white] (1,2) rectangle ++(1,1);
  \path[fill=white] (2,2) rectangle ++(1,1);
  \path[fill=white] (3,2) rectangle ++(1,1);
  \path[fill=white] (4,2) rectangle ++(1,1);
  \path[fill=white] (5,2) rectangle ++(1,1);
  \path[fill=white] (6,2) rectangle ++(1,1);
  \path[fill=white] (7,2) rectangle ++(1,1);
  \path[fill=white] (8,2) rectangle ++(1,1);
  \path[fill=white] (9,2) rectangle ++(1,1);
  \path[fill=white] (10,2) rectangle ++(1,1);
  \path[fill=white] (11,2) rectangle ++(1,1);
  \path[fill=white] (12,2) rectangle ++(1,1);
  \path[fill=white] (13,2) rectangle ++(1,1);
  \path[fill=white] (14,2) rectangle ++(1,1);
  \path[fill=white] (15,2) rectangle ++(1,1);
  \path[fill=white] (16,2) rectangle ++(1,1);
  \path[fill=white] (17,2) rectangle ++(1,1);
  \path[fill=white] (18,2) rectangle ++(1,1);
  \path[fill=black] (19,2) rectangle ++(1,1);
  \path[fill=white] (0,1) rectangle ++(1,1);
  \path[fill=white] (1,1) rectangle ++(1,1);
  \path[fill=white] (2,1) rectangle ++(1,1);
  \path[fill=white] (3,1) rectangle ++(1,1);
  \path[fill=white] (4,1) rectangle ++(1,1);
  \path[fill=white] (5,1) rectangle ++(1,1);
  \path[fill=white] (6,1) rectangle ++(1,1);
  \path[fill=white] (7,1) rectangle ++(1,1);
  \path[fill=white] (8,1) rectangle ++(1,1);
  \path[fill=white] (9,1) rectangle ++(1,1);
  \path[fill=white] (10,1) rectangle ++(1,1);
  \path[fill=white] (11,1) rectangle ++(1,1);
  \path[fill=white] (12,1) rectangle ++(1,1);
  \path[fill=white] (13,1) rectangle ++(1,1);
  \path[fill=white] (14,1) rectangle ++(1,1);
  \path[fill=white] (15,1) rectangle ++(1,1);
  \path[fill=white] (16,1) rectangle ++(1,1);
  \path[fill=white] (17,1) rectangle ++(1,1);
  \path[fill=white] (18,1) rectangle ++(1,1);
  \path[fill=black] (19,1) rectangle ++(1,1);
  \path[fill=white] (0,0) rectangle ++(1,1);
  \path[fill=white] (1,0) rectangle ++(1,1);
  \path[fill=white] (2,0) rectangle ++(1,1);
  \path[fill=white] (3,0) rectangle ++(1,1);
  \path[fill=white] (4,0) rectangle ++(1,1);
  \path[fill=white] (5,0) rectangle ++(1,1);
  \path[fill=white] (6,0) rectangle ++(1,1);
  \path[fill=white] (7,0) rectangle ++(1,1);
  \path[fill=white] (8,0) rectangle ++(1,1);
  \path[fill=white] (9,0) rectangle ++(1,1);
  \path[fill=white] (10,0) rectangle ++(1,1);
  \path[fill=white] (11,0) rectangle ++(1,1);
  \path[fill=white] (12,0) rectangle ++(1,1);
  \path[fill=white] (13,0) rectangle ++(1,1);
  \path[fill=white] (14,0) rectangle ++(1,1);
  \path[fill=white] (15,0) rectangle ++(1,1);
  \path[fill=white] (16,0) rectangle ++(1,1);
  \path[fill=white] (17,0) rectangle ++(1,1);
  \path[fill=white] (18,0) rectangle ++(1,1);
  \path[fill=white] (19,0) rectangle ++(1,1);
\end{tikzpicture}

\begin{tikzpicture}[x=0.3cm, y=0.3cm, draw=gray, very thin]
  \path[fill=white] (0,19) rectangle ++(1,1);
  \path[fill=black] (1,19) rectangle ++(1,1);
  \path[fill=black] (2,19) rectangle ++(1,1);
  \path[fill=black] (3,19) rectangle ++(1,1);
  \path[fill=black] (4,19) rectangle ++(1,1);
  \path[fill=white] (5,19) rectangle ++(1,1);
  \path[fill=black] (6,19) rectangle ++(1,1);
  \path[fill=black] (7,19) rectangle ++(1,1);
  \path[fill=black] (8,19) rectangle ++(1,1);
  \path[fill=white] (9,19) rectangle ++(1,1);
  \path[fill=black] (10,19) rectangle ++(1,1);
  \path[fill=black] (11,19) rectangle ++(1,1);
  \path[fill=black] (12,19) rectangle ++(1,1);
  \path[fill=white] (13,19) rectangle ++(1,1);
  \path[fill=black] (14,19) rectangle ++(1,1);
  \path[fill=black] (15,19) rectangle ++(1,1);
  \path[fill=white] (16,19) rectangle ++(1,1);
  \path[fill=black] (17,19) rectangle ++(1,1);
  \path[fill=black] (18,19) rectangle ++(1,1);
  \path[fill=black] (19,19) rectangle ++(1,1);
  \path[fill=white] (0,18) rectangle ++(1,1);
  \path[fill=white] (1,18) rectangle ++(1,1);
  \path[fill=white] (2,18) rectangle ++(1,1);
  \path[fill=black] (3,18) rectangle ++(1,1);
  \path[fill=black] (4,18) rectangle ++(1,1);
  \path[fill=white] (5,18) rectangle ++(1,1);
  \path[fill=black] (6,18) rectangle ++(1,1);
  \path[fill=black] (7,18) rectangle ++(1,1);
  \path[fill=white] (8,18) rectangle ++(1,1);
  \path[fill=white] (9,18) rectangle ++(1,1);
  \path[fill=black] (10,18) rectangle ++(1,1);
  \path[fill=black] (11,18) rectangle ++(1,1);
  \path[fill=white] (12,18) rectangle ++(1,1);
  \path[fill=white] (13,18) rectangle ++(1,1);
  \path[fill=black] (14,18) rectangle ++(1,1);
  \path[fill=black] (15,18) rectangle ++(1,1);
  \path[fill=white] (16,18) rectangle ++(1,1);
  \path[fill=black] (17,18) rectangle ++(1,1);
  \path[fill=white] (18,18) rectangle ++(1,1);
  \path[fill=black] (19,18) rectangle ++(1,1);
  \path[fill=white] (0,17) rectangle ++(1,1);
  \path[fill=white] (1,17) rectangle ++(1,1);
  \path[fill=white] (2,17) rectangle ++(1,1);
  \path[fill=white] (3,17) rectangle ++(1,1);
  \path[fill=black] (4,17) rectangle ++(1,1);
  \path[fill=black] (5,17) rectangle ++(1,1);
  \path[fill=white] (6,17) rectangle ++(1,1);
  \path[fill=black] (7,17) rectangle ++(1,1);
  \path[fill=black] (8,17) rectangle ++(1,1);
  \path[fill=white] (9,17) rectangle ++(1,1);
  \path[fill=black] (10,17) rectangle ++(1,1);
  \path[fill=black] (11,17) rectangle ++(1,1);
  \path[fill=black] (12,17) rectangle ++(1,1);
  \path[fill=white] (13,17) rectangle ++(1,1);
  \path[fill=black] (14,17) rectangle ++(1,1);
  \path[fill=black] (15,17) rectangle ++(1,1);
  \path[fill=black] (16,17) rectangle ++(1,1);
  \path[fill=black] (17,17) rectangle ++(1,1);
  \path[fill=black] (18,17) rectangle ++(1,1);
  \path[fill=black] (19,17) rectangle ++(1,1);
  \path[fill=white] (0,16) rectangle ++(1,1);
  \path[fill=white] (1,16) rectangle ++(1,1);
  \path[fill=white] (2,16) rectangle ++(1,1);
  \path[fill=white] (3,16) rectangle ++(1,1);
  \path[fill=white] (4,16) rectangle ++(1,1);
  \path[fill=white] (5,16) rectangle ++(1,1);
  \path[fill=black] (6,16) rectangle ++(1,1);
  \path[fill=black] (7,16) rectangle ++(1,1);
  \path[fill=white] (8,16) rectangle ++(1,1);
  \path[fill=white] (9,16) rectangle ++(1,1);
  \path[fill=black] (10,16) rectangle ++(1,1);
  \path[fill=white] (11,16) rectangle ++(1,1);
  \path[fill=white] (12,16) rectangle ++(1,1);
  \path[fill=white] (13,16) rectangle ++(1,1);
  \path[fill=black] (14,16) rectangle ++(1,1);
  \path[fill=white] (15,16) rectangle ++(1,1);
  \path[fill=white] (16,16) rectangle ++(1,1);
  \path[fill=black] (17,16) rectangle ++(1,1);
  \path[fill=white] (18,16) rectangle ++(1,1);
  \path[fill=white] (19,16) rectangle ++(1,1);
  \path[fill=white] (0,15) rectangle ++(1,1);
  \path[fill=white] (1,15) rectangle ++(1,1);
  \path[fill=white] (2,15) rectangle ++(1,1);
  \path[fill=white] (3,15) rectangle ++(1,1);
  \path[fill=white] (4,15) rectangle ++(1,1);
  \path[fill=white] (5,15) rectangle ++(1,1);
  \path[fill=white] (6,15) rectangle ++(1,1);
  \path[fill=black] (7,15) rectangle ++(1,1);
  \path[fill=black] (8,15) rectangle ++(1,1);
  \path[fill=white] (9,15) rectangle ++(1,1);
  \path[fill=black] (10,15) rectangle ++(1,1);
  \path[fill=black] (11,15) rectangle ++(1,1);
  \path[fill=white] (12,15) rectangle ++(1,1);
  \path[fill=white] (13,15) rectangle ++(1,1);
  \path[fill=black] (14,15) rectangle ++(1,1);
  \path[fill=black] (15,15) rectangle ++(1,1);
  \path[fill=white] (16,15) rectangle ++(1,1);
  \path[fill=black] (17,15) rectangle ++(1,1);
  \path[fill=black] (18,15) rectangle ++(1,1);
  \path[fill=black] (19,15) rectangle ++(1,1);
  \path[fill=white] (0,14) rectangle ++(1,1);
  \path[fill=white] (1,14) rectangle ++(1,1);
  \path[fill=white] (2,14) rectangle ++(1,1);
  \path[fill=white] (3,14) rectangle ++(1,1);
  \path[fill=white] (4,14) rectangle ++(1,1);
  \path[fill=white] (5,14) rectangle ++(1,1);
  \path[fill=white] (6,14) rectangle ++(1,1);
  \path[fill=white] (7,14) rectangle ++(1,1);
  \path[fill=black] (8,14) rectangle ++(1,1);
  \path[fill=black] (9,14) rectangle ++(1,1);
  \path[fill=white] (10,14) rectangle ++(1,1);
  \path[fill=black] (11,14) rectangle ++(1,1);
  \path[fill=black] (12,14) rectangle ++(1,1);
  \path[fill=white] (13,14) rectangle ++(1,1);
  \path[fill=black] (14,14) rectangle ++(1,1);
  \path[fill=black] (15,14) rectangle ++(1,1);
  \path[fill=black] (16,14) rectangle ++(1,1);
  \path[fill=black] (17,14) rectangle ++(1,1);
  \path[fill=black] (18,14) rectangle ++(1,1);
  \path[fill=black] (19,14) rectangle ++(1,1);
  \path[fill=white] (0,13) rectangle ++(1,1);
  \path[fill=white] (1,13) rectangle ++(1,1);
  \path[fill=white] (2,13) rectangle ++(1,1);
  \path[fill=white] (3,13) rectangle ++(1,1);
  \path[fill=white] (4,13) rectangle ++(1,1);
  \path[fill=white] (5,13) rectangle ++(1,1);
  \path[fill=white] (6,13) rectangle ++(1,1);
  \path[fill=white] (7,13) rectangle ++(1,1);
  \path[fill=white] (8,13) rectangle ++(1,1);
  \path[fill=white] (9,13) rectangle ++(1,1);
  \path[fill=black] (10,13) rectangle ++(1,1);
  \path[fill=white] (11,13) rectangle ++(1,1);
  \path[fill=white] (12,13) rectangle ++(1,1);
  \path[fill=white] (13,13) rectangle ++(1,1);
  \path[fill=white] (14,13) rectangle ++(1,1);
  \path[fill=white] (15,13) rectangle ++(1,1);
  \path[fill=white] (16,13) rectangle ++(1,1);
  \path[fill=white] (17,13) rectangle ++(1,1);
  \path[fill=white] (18,13) rectangle ++(1,1);
  \path[fill=white] (19,13) rectangle ++(1,1);
  \path[fill=white] (0,12) rectangle ++(1,1);
  \path[fill=white] (1,12) rectangle ++(1,1);
  \path[fill=white] (2,12) rectangle ++(1,1);
  \path[fill=white] (3,12) rectangle ++(1,1);
  \path[fill=white] (4,12) rectangle ++(1,1);
  \path[fill=white] (5,12) rectangle ++(1,1);
  \path[fill=white] (6,12) rectangle ++(1,1);
  \path[fill=white] (7,12) rectangle ++(1,1);
  \path[fill=white] (8,12) rectangle ++(1,1);
  \path[fill=white] (9,12) rectangle ++(1,1);
  \path[fill=black] (10,12) rectangle ++(1,1);
  \path[fill=black] (11,12) rectangle ++(1,1);
  \path[fill=white] (12,12) rectangle ++(1,1);
  \path[fill=white] (13,12) rectangle ++(1,1);
  \path[fill=black] (14,12) rectangle ++(1,1);
  \path[fill=white] (15,12) rectangle ++(1,1);
  \path[fill=white] (16,12) rectangle ++(1,1);
  \path[fill=black] (17,12) rectangle ++(1,1);
  \path[fill=white] (18,12) rectangle ++(1,1);
  \path[fill=black] (19,12) rectangle ++(1,1);
  \path[fill=white] (0,11) rectangle ++(1,1);
  \path[fill=white] (1,11) rectangle ++(1,1);
  \path[fill=white] (2,11) rectangle ++(1,1);
  \path[fill=white] (3,11) rectangle ++(1,1);
  \path[fill=white] (4,11) rectangle ++(1,1);
  \path[fill=white] (5,11) rectangle ++(1,1);
  \path[fill=white] (6,11) rectangle ++(1,1);
  \path[fill=white] (7,11) rectangle ++(1,1);
  \path[fill=white] (8,11) rectangle ++(1,1);
  \path[fill=white] (9,11) rectangle ++(1,1);
  \path[fill=white] (10,11) rectangle ++(1,1);
  \path[fill=black] (11,11) rectangle ++(1,1);
  \path[fill=black] (12,11) rectangle ++(1,1);
  \path[fill=white] (13,11) rectangle ++(1,1);
  \path[fill=black] (14,11) rectangle ++(1,1);
  \path[fill=black] (15,11) rectangle ++(1,1);
  \path[fill=white] (16,11) rectangle ++(1,1);
  \path[fill=black] (17,11) rectangle ++(1,1);
  \path[fill=black] (18,11) rectangle ++(1,1);
  \path[fill=black] (19,11) rectangle ++(1,1);
  \path[fill=white] (0,10) rectangle ++(1,1);
  \path[fill=white] (1,10) rectangle ++(1,1);
  \path[fill=white] (2,10) rectangle ++(1,1);
  \path[fill=white] (3,10) rectangle ++(1,1);
  \path[fill=white] (4,10) rectangle ++(1,1);
  \path[fill=white] (5,10) rectangle ++(1,1);
  \path[fill=white] (6,10) rectangle ++(1,1);
  \path[fill=white] (7,10) rectangle ++(1,1);
  \path[fill=white] (8,10) rectangle ++(1,1);
  \path[fill=white] (9,10) rectangle ++(1,1);
  \path[fill=white] (10,10) rectangle ++(1,1);
  \path[fill=white] (11,10) rectangle ++(1,1);
  \path[fill=black] (12,10) rectangle ++(1,1);
  \path[fill=black] (13,10) rectangle ++(1,1);
  \path[fill=white] (14,10) rectangle ++(1,1);
  \path[fill=black] (15,10) rectangle ++(1,1);
  \path[fill=black] (16,10) rectangle ++(1,1);
  \path[fill=black] (17,10) rectangle ++(1,1);
  \path[fill=black] (18,10) rectangle ++(1,1);
  \path[fill=black] (19,10) rectangle ++(1,1);
  \path[fill=white] (0,9) rectangle ++(1,1);
  \path[fill=white] (1,9) rectangle ++(1,1);
  \path[fill=white] (2,9) rectangle ++(1,1);
  \path[fill=white] (3,9) rectangle ++(1,1);
  \path[fill=white] (4,9) rectangle ++(1,1);
  \path[fill=white] (5,9) rectangle ++(1,1);
  \path[fill=white] (6,9) rectangle ++(1,1);
  \path[fill=white] (7,9) rectangle ++(1,1);
  \path[fill=white] (8,9) rectangle ++(1,1);
  \path[fill=white] (9,9) rectangle ++(1,1);
  \path[fill=white] (10,9) rectangle ++(1,1);
  \path[fill=white] (11,9) rectangle ++(1,1);
  \path[fill=white] (12,9) rectangle ++(1,1);
  \path[fill=white] (13,9) rectangle ++(1,1);
  \path[fill=black] (14,9) rectangle ++(1,1);
  \path[fill=white] (15,9) rectangle ++(1,1);
  \path[fill=white] (16,9) rectangle ++(1,1);
  \path[fill=white] (17,9) rectangle ++(1,1);
  \path[fill=white] (18,9) rectangle ++(1,1);
  \path[fill=white] (19,9) rectangle ++(1,1);
  \path[fill=white] (0,8) rectangle ++(1,1);
  \path[fill=white] (1,8) rectangle ++(1,1);
  \path[fill=white] (2,8) rectangle ++(1,1);
  \path[fill=white] (3,8) rectangle ++(1,1);
  \path[fill=white] (4,8) rectangle ++(1,1);
  \path[fill=white] (5,8) rectangle ++(1,1);
  \path[fill=white] (6,8) rectangle ++(1,1);
  \path[fill=white] (7,8) rectangle ++(1,1);
  \path[fill=white] (8,8) rectangle ++(1,1);
  \path[fill=white] (9,8) rectangle ++(1,1);
  \path[fill=white] (10,8) rectangle ++(1,1);
  \path[fill=white] (11,8) rectangle ++(1,1);
  \path[fill=white] (12,8) rectangle ++(1,1);
  \path[fill=white] (13,8) rectangle ++(1,1);
  \path[fill=black] (14,8) rectangle ++(1,1);
  \path[fill=black] (15,8) rectangle ++(1,1);
  \path[fill=white] (16,8) rectangle ++(1,1);
  \path[fill=black] (17,8) rectangle ++(1,1);
  \path[fill=white] (18,8) rectangle ++(1,1);
  \path[fill=black] (19,8) rectangle ++(1,1);
  \path[fill=white] (0,7) rectangle ++(1,1);
  \path[fill=white] (1,7) rectangle ++(1,1);
  \path[fill=white] (2,7) rectangle ++(1,1);
  \path[fill=white] (3,7) rectangle ++(1,1);
  \path[fill=white] (4,7) rectangle ++(1,1);
  \path[fill=white] (5,7) rectangle ++(1,1);
  \path[fill=white] (6,7) rectangle ++(1,1);
  \path[fill=white] (7,7) rectangle ++(1,1);
  \path[fill=white] (8,7) rectangle ++(1,1);
  \path[fill=white] (9,7) rectangle ++(1,1);
  \path[fill=white] (10,7) rectangle ++(1,1);
  \path[fill=white] (11,7) rectangle ++(1,1);
  \path[fill=white] (12,7) rectangle ++(1,1);
  \path[fill=white] (13,7) rectangle ++(1,1);
  \path[fill=white] (14,7) rectangle ++(1,1);
  \path[fill=black] (15,7) rectangle ++(1,1);
  \path[fill=black] (16,7) rectangle ++(1,1);
  \path[fill=black] (17,7) rectangle ++(1,1);
  \path[fill=black] (18,7) rectangle ++(1,1);
  \path[fill=black] (19,7) rectangle ++(1,1);
  \path[fill=white] (0,6) rectangle ++(1,1);
  \path[fill=white] (1,6) rectangle ++(1,1);
  \path[fill=white] (2,6) rectangle ++(1,1);
  \path[fill=white] (3,6) rectangle ++(1,1);
  \path[fill=white] (4,6) rectangle ++(1,1);
  \path[fill=white] (5,6) rectangle ++(1,1);
  \path[fill=white] (6,6) rectangle ++(1,1);
  \path[fill=white] (7,6) rectangle ++(1,1);
  \path[fill=white] (8,6) rectangle ++(1,1);
  \path[fill=white] (9,6) rectangle ++(1,1);
  \path[fill=white] (10,6) rectangle ++(1,1);
  \path[fill=white] (11,6) rectangle ++(1,1);
  \path[fill=white] (12,6) rectangle ++(1,1);
  \path[fill=white] (13,6) rectangle ++(1,1);
  \path[fill=white] (14,6) rectangle ++(1,1);
  \path[fill=white] (15,6) rectangle ++(1,1);
  \path[fill=black] (16,6) rectangle ++(1,1);
  \path[fill=white] (17,6) rectangle ++(1,1);
  \path[fill=black] (18,6) rectangle ++(1,1);
  \path[fill=black] (19,6) rectangle ++(1,1);
  \path[fill=white] (0,5) rectangle ++(1,1);
  \path[fill=white] (1,5) rectangle ++(1,1);
  \path[fill=white] (2,5) rectangle ++(1,1);
  \path[fill=white] (3,5) rectangle ++(1,1);
  \path[fill=white] (4,5) rectangle ++(1,1);
  \path[fill=white] (5,5) rectangle ++(1,1);
  \path[fill=white] (6,5) rectangle ++(1,1);
  \path[fill=white] (7,5) rectangle ++(1,1);
  \path[fill=white] (8,5) rectangle ++(1,1);
  \path[fill=white] (9,5) rectangle ++(1,1);
  \path[fill=white] (10,5) rectangle ++(1,1);
  \path[fill=white] (11,5) rectangle ++(1,1);
  \path[fill=white] (12,5) rectangle ++(1,1);
  \path[fill=white] (13,5) rectangle ++(1,1);
  \path[fill=white] (14,5) rectangle ++(1,1);
  \path[fill=white] (15,5) rectangle ++(1,1);
  \path[fill=white] (16,5) rectangle ++(1,1);
  \path[fill=black] (17,5) rectangle ++(1,1);
  \path[fill=white] (18,5) rectangle ++(1,1);
  \path[fill=white] (19,5) rectangle ++(1,1);
  \path[fill=white] (0,4) rectangle ++(1,1);
  \path[fill=white] (1,4) rectangle ++(1,1);
  \path[fill=white] (2,4) rectangle ++(1,1);
  \path[fill=white] (3,4) rectangle ++(1,1);
  \path[fill=white] (4,4) rectangle ++(1,1);
  \path[fill=white] (5,4) rectangle ++(1,1);
  \path[fill=white] (6,4) rectangle ++(1,1);
  \path[fill=white] (7,4) rectangle ++(1,1);
  \path[fill=white] (8,4) rectangle ++(1,1);
  \path[fill=white] (9,4) rectangle ++(1,1);
  \path[fill=white] (10,4) rectangle ++(1,1);
  \path[fill=white] (11,4) rectangle ++(1,1);
  \path[fill=white] (12,4) rectangle ++(1,1);
  \path[fill=white] (13,4) rectangle ++(1,1);
  \path[fill=white] (14,4) rectangle ++(1,1);
  \path[fill=white] (15,4) rectangle ++(1,1);
  \path[fill=white] (16,4) rectangle ++(1,1);
  \path[fill=black] (17,4) rectangle ++(1,1);
  \path[fill=black] (18,4) rectangle ++(1,1);
  \path[fill=black] (19,4) rectangle ++(1,1);
  \path[fill=white] (0,3) rectangle ++(1,1);
  \path[fill=white] (1,3) rectangle ++(1,1);
  \path[fill=white] (2,3) rectangle ++(1,1);
  \path[fill=white] (3,3) rectangle ++(1,1);
  \path[fill=white] (4,3) rectangle ++(1,1);
  \path[fill=white] (5,3) rectangle ++(1,1);
  \path[fill=white] (6,3) rectangle ++(1,1);
  \path[fill=white] (7,3) rectangle ++(1,1);
  \path[fill=white] (8,3) rectangle ++(1,1);
  \path[fill=white] (9,3) rectangle ++(1,1);
  \path[fill=white] (10,3) rectangle ++(1,1);
  \path[fill=white] (11,3) rectangle ++(1,1);
  \path[fill=white] (12,3) rectangle ++(1,1);
  \path[fill=white] (13,3) rectangle ++(1,1);
  \path[fill=white] (14,3) rectangle ++(1,1);
  \path[fill=white] (15,3) rectangle ++(1,1);
  \path[fill=white] (16,3) rectangle ++(1,1);
  \path[fill=white] (17,3) rectangle ++(1,1);
  \path[fill=black] (18,3) rectangle ++(1,1);
  \path[fill=black] (19,3) rectangle ++(1,1);
  \path[fill=white] (0,2) rectangle ++(1,1);
  \path[fill=white] (1,2) rectangle ++(1,1);
  \path[fill=white] (2,2) rectangle ++(1,1);
  \path[fill=white] (3,2) rectangle ++(1,1);
  \path[fill=white] (4,2) rectangle ++(1,1);
  \path[fill=white] (5,2) rectangle ++(1,1);
  \path[fill=white] (6,2) rectangle ++(1,1);
  \path[fill=white] (7,2) rectangle ++(1,1);
  \path[fill=white] (8,2) rectangle ++(1,1);
  \path[fill=white] (9,2) rectangle ++(1,1);
  \path[fill=white] (10,2) rectangle ++(1,1);
  \path[fill=white] (11,2) rectangle ++(1,1);
  \path[fill=white] (12,2) rectangle ++(1,1);
  \path[fill=white] (13,2) rectangle ++(1,1);
  \path[fill=white] (14,2) rectangle ++(1,1);
  \path[fill=white] (15,2) rectangle ++(1,1);
  \path[fill=white] (16,2) rectangle ++(1,1);
  \path[fill=white] (17,2) rectangle ++(1,1);
  \path[fill=white] (18,2) rectangle ++(1,1);
  \path[fill=black] (19,2) rectangle ++(1,1);
  \path[fill=white] (0,1) rectangle ++(1,1);
  \path[fill=white] (1,1) rectangle ++(1,1);
  \path[fill=white] (2,1) rectangle ++(1,1);
  \path[fill=white] (3,1) rectangle ++(1,1);
  \path[fill=white] (4,1) rectangle ++(1,1);
  \path[fill=white] (5,1) rectangle ++(1,1);
  \path[fill=white] (6,1) rectangle ++(1,1);
  \path[fill=white] (7,1) rectangle ++(1,1);
  \path[fill=white] (8,1) rectangle ++(1,1);
  \path[fill=white] (9,1) rectangle ++(1,1);
  \path[fill=white] (10,1) rectangle ++(1,1);
  \path[fill=white] (11,1) rectangle ++(1,1);
  \path[fill=white] (12,1) rectangle ++(1,1);
  \path[fill=white] (13,1) rectangle ++(1,1);
  \path[fill=white] (14,1) rectangle ++(1,1);
  \path[fill=white] (15,1) rectangle ++(1,1);
  \path[fill=white] (16,1) rectangle ++(1,1);
  \path[fill=white] (17,1) rectangle ++(1,1);
  \path[fill=white] (18,1) rectangle ++(1,1);
  \path[fill=black] (19,1) rectangle ++(1,1);
  \path[fill=white] (0,0) rectangle ++(1,1);
  \path[fill=white] (1,0) rectangle ++(1,1);
  \path[fill=white] (2,0) rectangle ++(1,1);
  \path[fill=white] (3,0) rectangle ++(1,1);
  \path[fill=white] (4,0) rectangle ++(1,1);
  \path[fill=white] (5,0) rectangle ++(1,1);
  \path[fill=white] (6,0) rectangle ++(1,1);
  \path[fill=white] (7,0) rectangle ++(1,1);
  \path[fill=white] (8,0) rectangle ++(1,1);
  \path[fill=white] (9,0) rectangle ++(1,1);
  \path[fill=white] (10,0) rectangle ++(1,1);
  \path[fill=white] (11,0) rectangle ++(1,1);
  \path[fill=white] (12,0) rectangle ++(1,1);
  \path[fill=white] (13,0) rectangle ++(1,1);
  \path[fill=white] (14,0) rectangle ++(1,1);
  \path[fill=white] (15,0) rectangle ++(1,1);
  \path[fill=white] (16,0) rectangle ++(1,1);
  \path[fill=white] (17,0) rectangle ++(1,1);
  \path[fill=white] (18,0) rectangle ++(1,1);
  \path[fill=white] (19,0) rectangle ++(1,1);
\end{tikzpicture}
}
\end{center}
\vspace{-0.3cm}
\caption{Adjacency and reachability matrix.}\label{fig:reach_matr}

\vspace{0.3cm}
\resizebox{0.4\textwidth}{!}{
\[
  \begin{array}{c|cccccccc}
    M_0 & q_{00} & q_{01} & q_{10} & q_{11} & q_{20} & q_{21} & q_{30} & q_{31} \\ \hline
    q_{00}       &        &
    \cellcolor{lightgray}{\overset{S}{\ws}\overset{F}{\ws}\overset{L}{\ws}\overset{R}{\bs}} &
    \cellcolor{lightgray}{\overset{S}{\ws}\overset{F}{\ws}\overset{L}{\bs}\overset{R}{\ws}} &
    \cellcolor{lightgray}{\overset{S}{\ws}\overset{F}{\ws}\overset{L}{\ws}\overset{R}{\bs}} &
    \overset{S}{\ws}\overset{F}{\ws}\overset{L}{\ws}\overset{R}{\ws} &
    \cellcolor{lightgray}{\overset{S}{\ws}\overset{F}{\ws}\overset{L}{\ws}\overset{R}{\bs}} &
    \overset{S}{\ws}\overset{F}{\ws}\overset{L}{\ws}\overset{R}{\ws} &
    \overset{S}{\ws}\overset{F}{\ws}\overset{L}{\ws}\overset{R}{\ws} \\ [6pt]
%     q_{00}       &        & \ws\ws\ws\bs & \ws\ws\bs\ws & \ws\ws\ws\bs & \ws\ws\ws\ws & \ws\ws\ws\bs & \ws\ws\ws\ws & \ws\ws\ws\ws \\ [6pt]
    q_{01}       &        &              & \ws\ws\ws\ws & \cellcolor{lightgray}{\ws\ws\bs\ws} & \ws\ws\ws\ws & \ws\ws\ws\ws & \ws\ws\ws\ws & \ws\ws\ws\ws \\ [6pt]
    q_{10}       &        &              &              & \cellcolor{lightgray}{\ws\ws\bs\ws} & \cellcolor{lightgray}{\ws\ws\ws\bs} & \cellcolor{lightgray}{\ws\ws\bs\ws} & \ws\ws\ws\ws & \cellcolor{lightgray}{\ws\ws\ws\bs} \\ [6pt]
    q_{11}       &        &              &              &              & \ws\ws\ws\ws & \cellcolor{lightgray}{\ws\ws\ws\bs} & \ws\ws\ws\ws & \ws\ws\ws\ws \\ [6pt]
    q_{20}       &        &              &              &              &              & \cellcolor{lightgray}{\ws\ws\bs\ws} & \cellcolor{lightgray}{\ws\ws\ws\bs} & \cellcolor{lightgray}{\ws\ws\bs\ws} \\ [6pt]
    q_{21}       &        &              &              &              &              &              & \ws\ws\ws\ws & \cellcolor{lightgray}{\ws\ws\ws\bs} \\ [6pt]
    q_{30}       &        &              &              &              &              &              &              & \cellcolor{lightgray}{\ws\ws\bs\bs} \\ [6pt]
    q_{31}       &        &              &              &              &              &              &              &              \\ [6pt]
  \end{array}
\]
}
\caption{Initial parse chart configuration.}\label{fig:initial_pc}

\vspace{0.3cm}
\resizebox{0.4\textwidth}{!}{
\input{figures/pc_final}
}

\caption{Final parse chart configuration.}\label{fig:final_pc}

\begin{center}
\resizebox{0.4\textwidth}{!}{
  \includegraphics{figures/gre}
}
\end{center}
\vspace{-0.3cm}
\caption{Regular expression denoting $\mathcal{L}(G_\cap)$.}\label{fig:re_tree}
\end{wrapfigure}

As a concrete example, suppose we have the string, $\err\sigma=\texttt{())}$ and wish to balance the parentheses. We will initially have the Levenshtein automaton, $A$, depicted in Fig.~\ref{fig:ex_atm}. To check for non-emptiness, we will perform the following procedure. Suppose we have a CNF CFG, $G'= \big\{S \rightarrow L R, S \rightarrow L F, S \rightarrow S S, F \rightarrow S R, L \rightarrow (, R \rightarrow )\big\}$ and let us assume an ordering of $S, F, L, R$ on $V$.

First, we need to order the automata states by increasing longest-path distance from $q_0$. One approach would be to topologically sort the adjacency matrix. While some form of sorting is unavoidable for arbitrary ANFAs, if we know ahead of time that our structure is a Levenshtein automaton, we can simply enumerate its state space by increasing Manhattan distance from the origin. % using, e.g., the Cantor pairing function to construct a valid ordering. This ordering will form the row and column indices of our intersection matrix, and each entry will represent the existence of some path between a two states yielding a given nonterminal.
So, a valid ordering on $Q$ would be $q_{00}, q_{01}, q_{10}, q_{11}, q_{20}, q_{21}, q_{30}, q_{31}$. Now, we want to compute whether $[\mathcal{L}(G')\cap \mathcal{L}(A) \neq \varnothing]$.

Under such an ordering, the adjacency matrix takes an upper triangular form and becomes the template for the initial parse chart, $M_0$ (Fig.~\ref{fig:initial_pc}). Each entry of this chart corresponds to a vector of expressions $E^{|V|}$ with at least one expression denoting a nonempty language. Likewise, the reachability matrix signifies a subset of state pairs which can participate in the language intersection. The adjacency and reachability matrices will always cover the expression vectors of the initial and final parse charts, respectively. In other words, we may safely ignore absent $\langle q, q'\rangle$ pairs in the reachability matrix, as these state pairs definitely cannot participate in the intersection.

From the reachability matrix we can construct the parse chart via matrix exponentiation. We note that n-step reachability constraints n-step parseability, i.e., $\sum_{i=0}^n A^i[q, q'] = \ws \vdash M_n[q, q', v] = \ws$, thus we can avoid substantial work via memoization. In this example, since $M_\infty[q_{00}, q_{31}, S] = \bs$, this implies that $\mathcal{L}(A)\cap \mathcal{L}(G') \neq \varnothing$, hence $\text{LED}(\sigma, G) = 1$. Using the same matrix, we will then perform a second pass to construct regular expressions representing finite languages for each nonempty constituent. Once again, we can skip $\langle q, q', v\rangle$ entries when $M_\infty[q, q', v] = \ws$ to hasten convergence.

\enlargethispage{4\baselineskip}
Just as before, we will define $\boxplus, \boxtimes$ over GRE vectors, where $X \boxtimes Z = [X_x\cdot Z_z \mid (w\rightarrow xz) \in P]_{w\in V}$ and $X \boxplus Z= [ X_w\vee Z_w ]_{w\in V}$. Finally, we will repeat the matrix exponential, using $M_\infty$ in the binary domain as a guide. This allows us to construct the regular expression tree for $S_\cap = q_{00}Sq_{20}\vee q_{00}Sq_{31}$ shown in Fig.~\ref{fig:re_tree}. Once this regex is constructed, decoding becomes simply a matter of invoking \texttt{choose}$(S_\cap)$. In this case there are only a few choices, but in general, there can be a vast multitude.

\clearpage

\section{Measuring the language intersection}\label{sec:measurement}

We will now attempt to put a probability distribution over the language intersection. We shall start with a few cursory but illumative approaches, then proceed towards a more refined solution.

\subsection{Mode collapse}

Ordinarily, one might think to train a top-down PCFG sampler using a treebank of well-formed code snippets, however this method is highly degenerate in the finite case, exhibiting poor sample diversity. Consider an illustrative pathological case for top-down ancestral (TDA) sampling:
$$
G=\left\{ S \rightarrow A\:B \: \left(\frac{10^5 - 1}{10^5}\right), \hspace{2pt}
     S \rightarrow C\:C \: \left(\frac{1}{10^5}\right), \hspace{2pt}
     A \rightarrow a \: (1), \hspace{2pt}
     B  \rightarrow b \: (1), \hspace{2pt}
     C  \rightarrow a \: \left(\frac{1}{26}\right) \mid \ldots \mid z \: \left(\frac{1}{26}\right)\right\}
$$
Such a sampler will almost always yield $a b$, but most of $\mathcal{L}(G)$ is concealed in the hidden branch, $S \rightarrow C C$. Though a contrived example, it illustrates why TDA sampling is unviable: our sampler should match the true distribution over the finite CFL, not the PCFG's local approximation thereof.

\subsection{Exact enumeration}

To correct for mode collapse, a brute force solution would be to simply generate every tree. While the whole set can be materialized in some cases when the intersection language is small, this strategy is clearly suboptimal due to its worst-case complexity. Nevertheless, it is useful for checking completeness. To enumerate trees, we first need the total number of trees, which is denoted $|e|$.

\begin{definition}[Cardinality]
  $|e|: E \rightarrow \mathbb{N} =$ \begin{cases}
    1           & \text{if } e \in \Sigma \\
    x \times z  & \text{if } e = x \cdot z \\
    x + z       & \text{if } e = x \vee z
  \end{cases}\\
\end{definition}

\begin{theorem}[Enumeration]
  To enumerate, we can invoke $\bigcup_{i = 0}^{|R|}\{\texttt{enum}(R, i)\}$:\\

  $\texttt{enum}(e, n): E \times \mathbb{N} \rightarrow \Sigma^*$ = \begin{cases}
       e &\text{if } e \in \Sigma \\
       \texttt{enum}\big(x, \lfloor \frac{n}{|z|} \rfloor\big) \cdot \texttt{enum}\big(z,\, n \bmod |z|\big)  &\text{if } e = x \cdot z \\
       \texttt{enum}\big((x, z)_{\min(1, \lfloor\frac{n}{|x|}\rfloor)}, n-|x|\min(1, \lfloor\frac{n}{|x|}\rfloor)\big) &\text{if } e = x \vee z
  \end{cases}
\end{theorem}

This can be converted to a uniform sampler by drawing integers without replacement using a pseudorandom number generator, however, if $|e|$ is very large, \texttt{enum} can fail to capture modes.

\subsection{The problem with ambiguity}

The main problem with the previous approach is that it counts distinct trees, which overcounts the total number of words, $|\mathcal{L}(G_\cap)|$. Since the Levenshtein automaton can be ambiguous, this causes certain repairs to be overrepresented, resulting in a pernicious bias. Consider, for example,

\begin{lemma}\label{lemma:ambiguity}
If the FSA, $\alpha$, is ambiguous, then the intersection grammar, $G_\cap$, can be ambiguous.
\end{lemma}

\begin{proof}
Let $\ell$ be the language defined by $G=\{S\rightarrow LR, L \rightarrow\texttt{(}, R \rightarrow\texttt{)}\}$, where $\alpha=L(\err\sigma, 2)$, the broken string $\err\sigma$ is $\texttt{)(}$, and $\mathcal{L}(G_\cap) = \ell \cap \mathcal{L}(\alpha)$. Then, $\mathcal{L}(G_\cap)$ contains the following two identical repairs: \texttt{\hlred{)}(\hlgreen{)}} with the parse $S \rightarrow q_{00}Lq_{21}\phantom{.}q_{21}Rq_{22}$, and \texttt{\hlorange{(}\hlorange{)}} with the parse $S \rightarrow q_{00}Lq_{11}\phantom{.}q_{11}Rq_{22}$.
\end{proof}

\noindent We would expect the underlying sample space to be a proper set, \textit{not} a multiset.

\subsection{Disambiguation}\label{sec:transfer_method}

To count the number of distinct repairs, we will need to convert $G_\cap$ to an automaton. Since $\mathcal{L}(G_\cap)$ is finite, it must be regular a fortiori. Recalling the definition for an NFA, $\langle Q, \Sigma, \delta, q_\alpha: Q, F \subseteq Q \rangle$, and star-free regex, $e \rightarrow \Sigma \mid e \lor e \mid e \land e$, we will proceed by structural induction on the regex:

\begin{equation*}
N(e) =
\begin{cases}
  \begin{alignedat}{8}
    &\big\langle \{q_\alpha, q_\omega\}
    &&\,,\, \{ q_\alpha \overset{e}{\rightarrow} q_\omega \}
    &&\,,\, q_\alpha\,\,, \{q_{\omega}\}
    &&\big\rangle
    &&\:\: \text{if } e \in \Sigma \\[0.5em]

    &\big\langle Q_{x} \cup Q_{z}
    &&\,,\, \{ q \overset{s}{\rightarrow} q_{\alpha z} \mid (q \overset{s}{\rightarrow} q_{\omega}^{\in F_x}) \in \delta_x \} \cup \delta_{x} \cup \delta_{z}
    &&\,,\, q_{\alpha x}\,, F_{z}
    &&\big\rangle
    &&\:\: \text{if } e = x \cdot z \\[0.5em]

    &\Bigg\langle   \begin{matrix} Q_{x}\cup \{q_{\alpha e}\}\:\cup\\ Q_{z} \cup \{q_{\omega e}\}\phantom{\:\cup} \end{matrix}
    &&\,,\, \begin{matrix}\{ q_{\alpha e} \overset{s}{\rightarrow} q \mid (q_{\alpha x, \alpha z} \overset{s}{\rightarrow} q)\in \delta_{x, z} \} \cup \delta_{x}\:\cup\\
    \{ q \overset{s}{\rightarrow} q_{\omega e} \mid (q \overset{s}{\rightarrow} q_{\omega}^{\in F_{x,z}})\in \delta_{x, z} \}\cup \delta_{z}\phantom{\,\:\cup}\end{matrix}
    &&\,,\, q_{\alpha e}\,, \{q_{\omega e}\}
    &&\Bigg\rangle
    &&\:\:\text{if } e = x \lor z\hspace{-0.5cm}\\[-0.5em]
    \multicolumn{9}{c}{\tiny{\text{- - - - - - - - - - - - or - - - - - - - - - - - -}}} \\[-0.5em]
    &\big\langle Q_{x} \cup Q_{z} \cup \{q_{\alpha e}\}
    &&\,,\, \{ q_{\alpha e} \overset{s}{\rightarrow} q \mid (q_{\alpha x, \alpha z} \overset{s}{\rightarrow} q) \in \delta_{x, z} \}  \cup \delta_{x} \cup \delta_{z}
    &&\,,\, q_{\alpha e}\,, F_{x} \cup F_{z}
    &&\big\rangle
    &&\:\: \text{if } e = x \lor z
  \end{alignedat}
\end{cases}\vspace{0.2cm}
\end{equation*}

\noindent Though slightly more verbose, we find the topology induced by the first version of the $\lor$ case to be slightly more favorable for minimization. To minimize, we will use Brzozowski's algorithm~\cite{brzozowski1962canonical} to construct the unique minimal DFA, $D^*_\cap \equiv_\mathcal{L} G_\cap$. Since $\mathcal{L}(G_\cap)$ is finite, the DFA must be acyclic and thus representable as an upper triangular adjacency matrix under a topological ordering of $\delta$.

Continuing with our running example from \S~\ref{sec:repair_ex}, this will result in the following construction:

\begin{figure}[H]
  \centering
  \includegraphics[width=0.27\textwidth]{figures/dyck_nfa}
  \includegraphics[width=0.27\textwidth]{figures/dyck_nfa_orig}
  \includegraphics[width=0.30\textwidth]{figures/dyck_dfa}
  \caption{FSA for $\mathcal{L}\big(L(\texttt{())}, 1)\big)\cap\mathcal{L}(G')$ (a) with or (b) without $\lor$-merging, and then (c) post-determinization.}\label{fig:fsas_for_re}
\end{figure}
Now, for any DFA, we can ascertain the size of its language by counting walks from $q_\alpha$ to $q_\omega \in F$. Letting $A$ be the adjacency matrix for $D_\cap^*$, i.e., $A[q, q'] = \big[1 \text{ if } \exists s: \Sigma \text{ s.t. } (q \overset{s}{\rightarrow} q') \in \delta \text{ else } 0\big]$, the number of words it recognizes is given via the transfer matrix method~\cite{flajolet2009analytic}, that is,
\begin{align}
  C(A, q_\alpha, F): \mathbb{N}^{|Q|\times|Q|} \times Q \times 2^Q \rightarrow \mathbb{N} = \sum_{\mathclap{q_\omega \in F}}(I-A)^{-1}[q_\alpha, q_\omega] &= \sum_{\mathclap{q_\omega \in F}}\sum_{i = 0}^{\mathclap{|Q|-1}}A^i[q_\alpha, q_\omega]
\end{align}
\noindent Plugging in powers of the adjacency matrix for the DFA shown in Fig.~\ref{fig:fsas_for_re}.(c), we arrive at the total:
\begin{align}
(I-A)^{-1} &= \hspace{1cm}I + A \hspace{1cm}+\hspace{1.05cm} A^2 \hspace{1.07cm}+\hspace{1cm} A^3 \hspace{0.9cm}+\hspace{0.9cm} A^4\\
  &=\begin{tiny}\begin{pmatrix}
       1 & 1 &   &   &   &   \\
        & 1  & 1 & 1 &   &   \\
        &   & 1  &   & 1 &   \\
        &   &   & 1  & 1 &   \\
        &   &   &   & 1  & 1 \\
        &   &   &   &   & 1  \\
  \end{pmatrix} + \begin{pmatrix}
              &   & 1 & 1 &   &   \\
              &   &   &   & 2 &   \\
              &   &   &   &   & 1 \\
              &   &   &   &   & 1 \\
              &   &   &   &   &   \\
              &   &   &   &   &   \\
  \end{pmatrix} + \begin{pmatrix}
              &   &   &   & 2 &   \\
              &   &   &   &   & 2 \\
              &   &   &   &   &   \\
              &   &   &   &   &   \\
              &   &   &   &   &   \\
              &   &   &   &   &   \\
  \end{pmatrix} + \begin{pmatrix}
              &   &   &   &   & 2 \\
              &   &   &   &   &   \\
              &   &   &   &   &   \\
              &   &   &   &   &   \\
              &   &   &   &   &   \\
              &   &   &   &   &   \\
  \end{pmatrix}\end{tiny}\\
  &= \begin{tiny}\begin{pmatrix}
          1   & 1  & 1  & \underline{1} & 2 & \underline{2} \\
              & 1  & 1  & 1 & 2 & 2 \\
              &    & 1  &   & 1 & 1 \\
              &    &    & 1 & 1 & 1 \\
              &    &    &   & 1 & 1 \\
              &    &    &   &   & 1 \\
\end{pmatrix}\end{tiny} \text{ therefor, } \big|\mathcal{L}(D_\cap^*)\big| = C\big(A, q_0, \{q_3, q_5\}\big) = \underline{1} + \underline{2} = 3.
\end{align}
Note the model counting problem is strictly harder than deciding intersection nonemptiness as it requires a secondary determinization step, however, weak bounds may be obtained by applying $C$ to the automaton generated by $N(e)$ or by direct analysis of $e$. While the inequality $C_{D_\cap^*} \leq C_{N(e)} \leq |e|$ will hold, the bounds provided by the latter approximations may be vacuous, whereas $C_{D_\cap^*}$ is exact.

%The advantage of dealing with formal language representations is that we can reason about them algebraically. Consider the context-free grammar: the arrow $\rightarrow$ becomes an $=$ sign, $\mid$ becomes $+$ and $AB$ becomes $A \times B$. The ambiguous Dyck grammar, then, can be seen as a system of equations.
%
%\begin{equation}
%  S \rightarrow ( ) \mid ( S ) \mid S S \Longleftrightarrow f(x) = x^2 + x^2 f(x) + f(x)^2
%\end{equation}
%
%\noindent We will now solve for $f(x)$, giving us the generating function for the language:
%
%\begin{equation}
%  0 = f(x)^2 + x^2 f(x) - f(x) + x^2
%\end{equation}
%
%\noindent Now, using the quadratic equation, where $a = 1, b = x^2 - 1, c = x^2$, we have:
%
%\begin{equation}
%  f(x) = \frac{-b \pm \sqrt{b^2 - 4ac}}{2a} = \frac{-x^2 + 1 \pm \sqrt{x^4 - 6x^2 + 1}}{2}
%\end{equation}
%
%\noindent Note there are two solutions, but only one where $\lim_{x\rightarrow 0} = 1$. From the ordinary generating function (OGF), we also have that $f(x)=\sum _{n=0}^{\infty }f_nx^{n}$. Expanding $\sqrt{x^4 - 6x^2 + 1}$ via the generalized binomial theorem, we have:
%
%\begin{align}
%  f(x) = (1+u)^{\alpha }&=\sum _{k=0}^{\infty }\;{\binom {\alpha }{k}}\;u^{k}\\
%  &=\sum _{k=0}^{\infty }\;{\binom {\frac{1}{2} }{k}}\;(x^4 - 6x^2)^{k} \text{ where } u = x^4-6x^2
%\end{align}
%
%Now, to obtain the number of ambiguous Dyck trees of size $n$, we can extract the $x^n$-th coefficient using the binomial series:
%
%\begin{align}
%[x^n]f(x) &= [x^n]\frac{-x^2 + 1}{2} + \frac{1}{2}[x^n]\sum _{k=0}^{\infty }\;{\binom {\frac{1}{2} }{k}}\;(x^4 - 6x^2)^{k}\\
%[x^n]f(x) &= \frac{1}{2}{\binom {\frac{1}{2} }{n}}\;[x^n](x^4 - 6x^2)^n = \frac{1}{2}{\binom {\frac{1}{2} }{n}}\;[x^n](x^2 - 6x)^n
%\end{align}
%
%We can use this technique, first described by Flajolet \& Sedgewick~\cite{flajolet2009analytic}, to count the number of trees of a given size or distinct words in an unambiguous CFG. This lets us understand grammars as a kind of algebra, which is useful for enumerative combinatorics on words and syntax-guided synthesis. We use use this in our setting to count the total number of words in the intersection grammar.

\clearpage\section{Implementation}\label{sec:implementation}

The implementation essentially consists of four stages, each dependent on its predecessor.

\begin{enumerate}
  \item $\texttt{lev\_build}: \Sigma^{|Q|-1} \times \mathbb{N}^{3} \rightarrow \text{NFA}$ -- constructs a Levenshtein NFA from the broken string.
  \item $\texttt{cfl\_fixpt}: \text{NFA} \times \text{CFG} \rightarrow \mathbb{B}^{|Q|\times |Q| \times |V|}$ -- computes the matrix exponential.
  \item $\texttt{reg\_build}: \mathbb{B}^{|Q|\times |Q| \times |V|} \times \text{CFG} \rightarrow \text{GRE}$ -- constructs the regular expression for $G_\cap$.
  \item $\texttt{reg\_dcode}: \text{GRE} \times \mathbb{N}^{|\Sigma|^{c\approx 3}} \hspace{-0.05cm}\times \mathbb{N} \rightarrow\hspace{-0.02cm} (\Sigma^+)^{k\approx 10}$ -- returns a small set of the most probable repairs.
%  \item $\texttt{sel\_top\_k}: (\Sigma^* \times \mathbb{N})^{p\gg1} \times \mathbb{N} \rightarrow (\Sigma^*)^{k\ll p}$ -- returns a small set of the most probable repairs.
\end{enumerate}

\noindent We will now explore the imperative pseudocode for each stage, starting with the Levenshtein automata constructor, which is a straightforward translation of the inference rules in \S~\ref{sec:repair_ex}.

\begin{algorithm}[H]
\caption{\texttt{lev\_build} pseudocode}
\label{alg:lev_build}
\begin{algorithmic}[1]
  \Procedure{\texttt{lev\_build}$(\sigma: \Sigma^n, d_{\max}: \mathbb{N})$}{} \Comment{Takes a string and maximum edit distance.}
  \State $Q, \delta \gets \varnothing$
  \For{$\langle h, j, i, k \rangle \textbf{ in } [0, n]^2\times[0, d_{\max}]^2$\vspace{1.34cm}}
    \State \vspace{-1.65cm}\[\hspace{1.15cm}\delta\,\gets \delta\,\cup\:\!\left\{
        \begin{alignedat}{9}
          &\;& q_{h,i} &\hspace{-0.1cm}\overset{{\color{orange}[\neq\sigma_{j+1}]}}{\rightarrow} &q_{j,k} &\qquad& \text{if}\;& h = j   &\:\land\:& i = k-1  &\qquad& \duparrow\\[-2pt]
          && q_{h,i}   &\overset{{\color{orange}[\neq\sigma_j]}}{\rightarrow} &q_{j,k} &&       \text{if}\;& h = j-1 &\:\land\:& i = k-1  &&       \ddiagarrow\\[-2pt]
          && q_{h,i}   &\overset{{\color{orange}[=\sigma_j]}}{\rightarrow}    &q_{j,k} &&       \text{if}\;& h = j-1 &\:\land\:& i = k    &&       \drightarrow\\[-2pt]
          && q_{h,i}   &\overset{{\color{orange}[=\sigma_j]}}{\rightarrow}    &q_{j,k} &&       \text{if}\;& 1 \leq j - h - 1 \leq d_{\max} &\:\land\:& 1 \leq k - i \leq d_{\max}   && \knightarrow\;
        \end{alignedat}
      \right\}\]
    \State $Q \gets Q \cup \{q_{h,i}, q_{j,k}\}$
  \EndFor
  \State $I \gets \{q_{0,0}\}, F \gets \{q_{i, j} \mid n - i + j \leq d_{\max}\}$
  \State \Return $\langle Q, \Sigma, \delta = Q \times (\Sigma\:{\color{orange}\rightarrow \mathbb{B}}) \times Q, q_\alpha, F\rangle$  \Comment{Returns a [nominal] Levenshtein automaton.}
\end{algorithmic}
\end{algorithm}\vspace{-0.2cm}

Next, the chart parser expects an acyclic NFA, a CNF grammar and returns a Boolean 3-tensor.

\begin{algorithm}[H]
\caption{\texttt{cfl\_fixpt} pseudocode}
\label{alg:cfl_fixpt}
\begin{algorithmic}[1]
\Require CFG must be in CNF and the NFA must be $\varepsilon$-free and acyclic (i.e., denote a finite langauge).
\Procedure{\texttt{cfl\_fixpt}$\big(\langle \Sigma, V, P, S\rangle: \text{CFG}, \langle Q, \Sigma, \delta, q_\alpha, F\rangle: \text{NFA}\big)$}{}
\State $R: \mathbb{B}^{|Q|\times |Q|} \gets \big[\bs \textbf{ if } \exists \sigma \in \Sigma^+ \mid q \overset{\sigma}{\rightsquigarrow} q' \textbf{ else } \ws\big]_{q,\,q'\,:\, Q}$ \Comment{Solve for reachability matrix.}
\State $M: \mathbb{B}^{|Q|\times |Q| \times |V|} \gets \big[\bs \textbf{ if } \exists s: \Sigma \mid (v \rightarrow s) \in P \land (q \overset{{\color{orange}\varphi}}{\rightarrow} q') \in \delta \land {\color{orange}\varphi(}s{\color{orange})} \textbf{ else } \ws\big]_{q,\,q'\,:\,Q,\,v\,:\,V}$
\For{$i \textbf{ in } \big[0, \lceil\log_2(|Q||V|)\rceil\big]$} \Comment{Solves matrix exponential, $\exp(M_0)$.}
\State $\textsc{done} \gets \bs$
\For{$\langle p, r, w \rangle \textbf{ in } Q^2\times V$} \Comment{Iterates one step of $M_{i+1} = M_i + M_i^2$.}
  \State $\textbf{if } M[p, r, w] \textbf{ or not } R[p, r] \textbf{ then continue}$
  \State $Q_{pr} \gets \big\{q: Q \mid R[p, q] \land R[q, r]\big\}$ \Comment{Consider reachable states between p and r.}
  \State $M[p, r, w] \gets \bs \textbf{ if } \exists q: Q_{pr}, x, z: V \mid M[p, q, x] \land M[q, r, z] \land (w \rightarrow x z) \in P \textbf { else } \ws$
  \State $\textbf{if } M[p, r, w] \textbf{ then } \textsc{done} \gets \ws$
\EndFor
\State $\textbf{if }\textsc{done} \textbf{ then break}$
\EndFor
\State \Return $M$ \Comment{Returns the completed Boolean parse chart.}
  \end{algorithmic}
\end{algorithm}\vspace{-0.2cm}

Note we may short-circuit for three reasons, if: $M_{i+1} = M_i$, when two states $q, q'$ are unreachable, or whenever a $\langle q, q', v\rangle$ is already present. Once we obtain $M_\infty$, we can immediately tell whether $\ell_\cap \neq \varnothing$ by checking whether $M_\infty[q_\alpha, q_\omega, S] = \bs$ for some $q_\omega: F$. Otherwise if no such $q_\omega$ exists, then $\ell_\cap$ must be empty and $d_\max$ should be enlarged before proceeding.

Now we can perform a second sweep over nonempty entries of the Boolean parse chart, reconstructing the provenance of each $\langle q, q', v\rangle$ constituent. For compactness it will be convenient to use a pointer-based representation of the regular expression instead of manipulating strings.

\begin{algorithm}[H]
\caption{\texttt{reg\_build} pseudocode}
\label{alg:reg_build}
\begin{algorithmic}[1]
  \Require Same as \texttt{cfl\_fixpt} (Alg.~\ref{alg:cfl_fixpt}), $M_{\mathbb{B}}[q_\alpha, q_\omega: F, S] = \bs$ for some $q_\omega$, and $M_{\mathbb{B}} = M_{\mathbb{B}} + M_{\mathbb{B}}^2$.
  \Procedure{\texttt{reg\_build}$\big(M_{\mathbb{B}}: \mathbb{B}^{|Q|\times |Q| \times |V|}, \langle \Sigma, V, P, S\rangle: \text{CFG}, \langle Q, \Sigma, \delta, q_\alpha, F\rangle: \text{NFA}\big)$}{}
  \State $P: \mathbb{B}^{|Q|\times |Q|} \gets \big[\bs \textbf{ if } \exists q: Q, v, v': V \mid M_{\mathbb{B}}[p, q, v] \land M_{\mathbb{B}}[q, r, v'] \textbf{ else } \ws\big]_{p,\,r\,:\,Q}$
  \State $M: \text{GRE}^{|Q|\times |Q| \times |V|} \gets \big[\{s: \Sigma \mid M[q, q', v] \land (q \overset{{\color{orange}\varphi}}{\rightarrow} q') \in \delta \land (v\rightarrow s) \in P \land {\color{orange}\varphi(}s{\color{orange})}\}\big]_{q,\,q'\,:\, Q,\,v\,:\,V}$
  \For{$i \textbf{ in } \big[0, \lceil\log_2(|Q||V|)\rceil\big]$}
  \State $M' \gets M$
  \For{$\langle p, r, w \rangle \textbf{ in } Q^2\times V$}
  \State $\textbf{if not } M_\mathbb{B}[p, r, w] \textbf{ then continue}$
  \State $Q_{pr} \gets \big\{q: Q \mid P[p, q] \land P[q, r]\big\}$ \Comment{Consider parseable states between p and r.}\vspace{0.2cm}
  \State \vspace{-0.42cm}\[\hspace{0.62cm}M'[p, r, w] \gets M[p, r, w] \vee \bigvee_{\mathclap{\substack{q\,:\,Q_{pr}\\x,\,z\,:\,V}}} \big\{M[p, q, x]\cdot M[q, r, z] \mid (w \rightarrow x z) \in P\big\}\]\vspace{-0.2cm}
  \EndFor
  \State $\textbf{if } M=M' \textbf{ then break else } M \gets M'$
  \EndFor \vspace{0.2cm}
  \State \vspace{-0.42cm}\[\hspace{-8.5cm}\textbf{return }\bigvee_{\mathclap{q_\omega\,:\,F}} M[q_\alpha, q_\omega, S]\vspace{-0.8cm}\] \Comment{Union regexes for all total parses yielding S.}\vspace{0.31cm}
\end{algorithmic}
\end{algorithm}

Finally, once we have the expression for $G_\cap$, we can decode it to extract a small set of candidates. Various strategies are possible here, and we opt for the simplest one. We use two priority queues to store partial and total trajectories, which are ranked by probability as estimated by a pretrained c-gram count tensor, $C$. Partial trajectories are greedily extended until termination, after which the trajectory it is diverted to the total queue, and the top-k total trajectories are returned.

\begin{algorithm}[H]
  \caption{\texttt{reg\_dcode} pseudocode}
  \label{alg:reg_dcode}
  \begin{algorithmic}[1]
  \Require We expect the shortest word to exceed the Markov order in length, $c < |\sigma|, \forall\sigma: \mathcal{L}(e)$.
  \Procedure{\texttt{reg\_dcode}$\big(e: \text{GRE}, C: \mathbb{N}^{|\Sigma|^{c\approx 3}}, k: \mathbb{N}\big)$}{}
    \State $\mathcal{T} \gets [], \mathcal{E} \gets \big[\langle \varepsilon^{c-1}, e\cdot \varepsilon^{c-1}, 0\rangle\big]$ \Comment{Initialize total and partial trajectories.}\vspace{0.5cm}
    \State \vspace{-0.55cm}\[\hspace{-4.58cm}\textbf{let } P(s: \Sigma \mid \sigma: \Sigma^{\geq c-1}) = \frac{C[\sigma_{|\sigma| - c + 1, |\sigma|}\cdot s]+ 1}{\sum_{s' \in \Sigma} C[\sigma_{|\sigma| - c + 1, |\sigma|}\cdot s']}\vspace{-0.7cm}\]\Comment{Define Markov transition probability.}\vspace{0.3cm}
    \Repeat
        \State $\langle\sigma, e, p\rangle \gets \textbf{pop } \mathcal{E}_0 \textbf{ off }\mathcal{E}$
        \State $\mathcal{E}' \gets \big[\langle\sigma\cdot a, \partial_a e, p + \ln P(a\mid \sigma) \rangle \mid a \in \texttt{follow}(e)\big]$
        \State $\mathcal{T}\hspace{0.05cm} \gets \mathcal{T} \texttt{++} \big[\langle \sigma, p \rangle \mid \langle \sigma, e, p\rangle \in \mathcal{E}' \land \varepsilon \in \mathcal{L}(e)\big]$
        \State $\mathcal{E}\phantom{'} \gets \big[\langle\sigma, e, p\rangle \in (\mathcal{E} \texttt{++} \mathcal{E}')\textbf{ sorted by } p\big]$
    \Until{interrupted or $\mathcal{E}$ is empty.}
    \State \Return $[\sigma \mid \langle\sigma, p\rangle \in \mathcal{T}_{0..k}\textbf{ sorted by } p]$ \Comment{Skim off top-k repairs by c-gram probability.}
  \end{algorithmic}
\end{algorithm}

%Now we are ready to skim off the highest probability repairs using a standard selection algorithm.
%
%\begin{algorithm}[H]
%  \caption{\texttt{sel\_top\_k} pseudocode}
%  \label{alg:sel_top_k}
%  \begin{algorithmic}[1]
%    \Procedure{\texttt{sel\_top\_k}$\big(l: (\Sigma^* \times \mathbb{N})^{p\gg1}, k: \mathbb{N}\big)$}{}
%    \State $\hat{A}: (\Sigma^* \times \mathbb{N})^* \gets [l_0]}$ \Comment{Initialize a priority queue of nearly-optimal repairs.}
%    \For{$\langle\sigma, s\rangle \textbf{ in } l$}
%      \State $\textbf{if } s < \pi_2(\hat{A}_{|\hat{A}|}) \textbf{ then insert } \langle\sigma, s\rangle \textbf{ into } \hat{A} \textbf{ and drop } \hat{A}_{|k + 1|} \textbf{ if } |\hat{A}| > k$
%    \EndFor
%    \State \Return $[\pi_1(a) \mid a \in \hat{A}]$ \Comment{Return top-k candidates.}
%  \end{algorithmic}
%\end{algorithm}

  Now, we have our shortlist of repairs and after cosmetic postprocessing, can present them to the user. With this approach, we can quickly generate a representative subset of $\ell_\cap$ within a fixed latency budget, e.g., \~100ms, or otherwise terminate early should we succeed in exhaustively generating it.

\clearpage\subsection{GPU translation}

This architecture can be translated to series of high-performance GPU kernels, essentially vector and tensor operations which optimized for GPU acceleration. Our strategy will be to treat each stage of the repair pipeline as a massively parallel map-reduce job, where each thread is responsible for writing a single entry to a large buffer, independently and without inter-thread communication.

We will make the simplifying assumption that each GPU kernel is a pure function that takes as input a coordinate triple $r, c, v: \mathbb{N}$ and one or more flat buffers $b_1: \mathbb{N}^{d_1}, \ldots b_n: \mathbb{N}^{d_n}$ containing encoded data, does some arithmetic, and returns a single output buffer, $b_{\text{out}}: \mathbb{N}^{d}$. The main challenge of GPU programming becomes mapping complex datatypes to and from an integer format.

Morally, each $\langle r, c, v\rangle$ triple will dispatch a single isolated GPU thread with global read access to the input buffers and exclusive write access to a contiguous region of the output buffer. Absent a GPU, this can be rewritten as a triply-nested loop subject to additional latency overhead. Each thread is effectively executed simultaneously on a GPU but all memory must be sized ahead of time, as dynamic allocation is forbidden during a GPU kernel's execution. This presents a slight challenge for constructing the GRE datatype, though surmountable with careful reference counting.

For CFG and NFA datatypes, we elect to use a dense representation $\mathbb{B}^{|V|\times|V|\times|V|}$ and $\mathbb{B}^{|Q|\times|Q|\times |\Sigma|}$ due to the tripartite coordinate structure and thread dispatching API. While these datatypes can be encoded sparsely as $\mathbb{N}^{3|P|}$ and $\mathbb{N}^{3|\delta|}$, for most repair instances and memory configurations representation size is not a bottleneck. It will be helpful to define two indices $\texttt{enc}: \Sigma \rightarrow 2^V$ and $\texttt{dec}: V \rightarrow 2^\Sigma$ for nonterminal encoding and decoding, and bijections $\Sigma \leftrightarrow \mathbb{N}$, $V \leftrightarrow \mathbb{N}$ for getting into and out of the integer domain -- these we omit for brevity but are trivial to define.

The parse chart $M$ can be represented as a bit-packed integer matrix $\texttt{uint32}^{|Q|\times|Q|\times|V|}$, whose layout encodes up to four properties of each $\langle q, q', v\rangle$ triple: (1) the first bit encodes the dis/equality predicate ${\color{orange}\varphi}}$, (2) the next 25 bits designate terminal participation $\big(\text{if }\exists s: \Sigma. \varphi(s) \land (q \overset{{\color{orange}\varphi}}{\rightarrow} q')\in \delta\big)$, (3) the next five bits memoize the minimum iteration where $M_{i_{\min}}[q, q', v] \& 1 = \bs$ (if parsable) allowing us to short circuit on Line 7 of \texttt{reg\_build} (Alg.~\ref{alg:reg_build}), and the last bit denotes parsability, i.e., $q\rightsquigarrow q'\vdash v$. Note that a pair $(q, q')$ can be simultaneously an arc and a path, so both possibilities must be acknowledged during decoding. This is depicted below in little-endian format:\vspace{-0.1cm}

\[
\big[\overset{\overset{{\color{orange}=/\neq}}{\Updownarrow}}{\bs}, \overbrace{\ws, \ws, \ldots, \bs, \ws, \bs}^{s: \Sigma \Leftrightarrow \mathbb{B}^{25}}, \overbrace{\ws, \bs, \ws, \bs, \bs}^{i_{\min}:\mathbb{N}_{\leq 32}\Leftrightarrow \mathbb{B}^{5}}, \overset{\overset{v: V}{\Updownarrow}}{\bs}\big]: \texttt{uint32}
\]

Once \texttt{cfl\_fixpt} (Alg.~\ref{alg:cfl_fixpt}) is complete, we can calculate the total amount of memory needed to allocate $G_\cap$ by counting constituents in the parse chart. The GRE datatype is slightly more complex to flatten, as being an algebraic datatype it can be encoded in various ways. We will use the following memory layout to encode a GRE, however this choice is somewhat arbitrary,\vspace{-0.1cm}

\[
\big[\overbrace{\underset{\texttt{bp}_0}{2}, \underset{\texttt{bp}_1}{7}, \ldots, \underset{\texttt{bp}_{c-2}}{1}, \underset{\texttt{bp}_{c-1}}{3}}^{\texttt{bp\_counts}}, \overbrace{0, 4, \ldots, \underset{\texttt{bp}_{c-2}}{n-8}, \underset{\texttt{bp}_{c-1}}{n-6}}^{\texttt{bp\_offsets}}, \overbrace{\underbrace{\underline{59,83}, \underline{64, 152}}_{\texttt{bp}_0}, \ldots, \underbrace{\underline{34, 83}}_{\texttt{bp}_{c-2}}, \underbrace{\underline{22,74},  \underline{74, 90}, \underline{16, 66}}_{\texttt{bp}_{c-1}}}^{\texttt{bp\_storage}}\big]:  \texttt{uint32}^n
\]

\noindent where each $\texttt{bp}_i$ represents a nonempty $\langle q, q', v\rangle$ constituent with at least one back-pointer pair, $\texttt{bp\_count}(p, r, w) = \big|\big\{\langle q, x, z\rangle\mid  M[p, q, x] \land M[q, r, z] \land (w \rightarrow xz)\in P\big\}\big|$ counts the number of unique backpointers held by each nonterminal $w$ parseable from $p \rightsquigarrow r$, and $\texttt{bp\_storage}$ stores pointers to memory locations in the same data structure. These pointers must also be tied to locations in the parse chart $M[q, q', v]$ in order to obtain the terminal subsets for unit nonterminals when decoding. In total, the GPU should have at least 4 GB of onboard memory to safely handle intersections with up to 1k states and nonterminals $\big(|Q|^2\times|V|=10^6\times10^3\times 32 \text{ bits} \approx \text{4 GB}\big)$.

\clearpage\subsection{Training the reranker}

The candidate repairs contain every nearby edit, filtered by grammatical admissibility. So every repair will be valid, simple and at least somewhat plausible, however it is possible the list may be quite long. No reasonable user would be expected to skim through $10^5$ suggestions to select their intended repair, especially since they could have easily written it themselves in a few seconds.

So, we will sort this list by a neural reranker. The ensuing technique falls under the umbrella of the \textit{learning-to-rank} (LTR) problem in machine learning. In their terminology, the broken code snippet is a ``query'' and the list of candidates are ``documents''. If we can guarantee the documents are exhaustive, they will necessarily include the true repair which need only be surfaced.

To introduce the model, we must now overload some concepts, so the reader is trusted to contextually interpret $\mathcal{L}$ as denoting a ``loss'' instead of a langauge, and the derivative operator~\footnote{There is a connection to Brzozowski's derivative, but to refrain from digression here we refer the reader to ~\cite{elliott2019generalized} for details.} as the directional rate of change of a differentiable manifold over the parameter space of a neural network. Matrix multiplication remains more or less the same as before, except now over the reals.

We omit the definition of a transformer-encoder (see Strobl et al.'s survey~\cite{strobl2024formal}), except to say that it is a function $T_\theta: \Sigma^* \rightarrow \mathbb{R}^k$ where $\theta$ are learnable parameters trained via gradient descent. This we will use to embed the query and documents, then return a numerical score for the relevance labels by applying a series of linear layers from a multilayer perceptron (MLP):

\[
f_\theta(\err\sigma, \sigma): \bar\ell \times \ell_\cap \rightarrow \mathbb{R} = \text{MLP}_{\theta}\big(T_\theta(\err\sigma), T_\theta(\sigma)\big)
\]

The model will be used to predict a relevance score $f_\theta(q, d)$ for a given query $q$ and item $d$. The training objective is to minimize the listwise cross-entropy loss:
\begin{equation}
\mathcal{L}(\theta) = -\sum_{q \in \mathcal{Q}} \log \left( \frac{\exp\big(f_\theta(q, d_q^+)\big)}{\sum_{d \in \mathcal{D}_q} \exp\big(f_\theta(q, d)\big)} \right)
\end{equation}
where $\mathcal{Q}$ is the set of training queries, $\mathcal{D}_q$ is the set of candidate repairs for query $q$, and $d_q^+ \in \mathcal{D}_q$ is true repair. %This loss function encourages the model to assign higher scores to natural repairs, thereby learning to rank them above unnatural ones.



\clearpage\section{Evaluation}

We call our method Tidyparse and consider the following research questions:

\begin{itemize}
\item \textbf{RQ 1}: What statistical properties do human repairs exhibit? (e.g., length, edit distance)
\item \textbf{RQ 2}: How performant is Tidyparse at fixing syntax errors? (i.e., vs. Seq2Parse and BIFI)
\item \textbf{RQ 3}: Which design choices are most significant? (e.g., sampling, decoding, parallelism)
\end{itemize}

We address \textbf{RQ 1} in \S~\ref{sec:rq1} by analyzing the distribution of natural code snippet lengths and edit distances, \textbf{RQ 2} in \S~\ref{sec:rq2} by comparing Tidyparse against two existing syntax repair baselines, and \textbf{RQ 3} in \S~\ref{sec:rq3} by ablating various design choices and evaluating the impact on repair precision.

\subsection{Experimental setup}

We use syntax errors and fixes from the Python language to validate our approach.  Python source code fragments are abstracted as a sequence of lexical tokens using the official Python lexer, erasing numbers and identifiers, but retaining all other keywords. Accuracy is evaluated across a test set by checking for lexical equivalence with the ground-truth repair, following Sakkas et al. (2022)~\cite{sakkas2022seq2parse}.

To evaluate accuracy, we use the Precision@k statistic, which measures the frequency of repairs in the top-k results matching the true repair. Specifically, given a repair model, $R: \Sigma^* \rightarrow 2^{\Sigma^*}$ and a test set $\mathcal{D}_{\text{test}}$ of pairwise aligned errors ($\sigma^\dagger$) and fixes ($\sigma'$), we define Precision@k as:

\begin{equation}
\text{Precision@k}(R) = \frac{1}{|\mathcal{D}_{\text{test}}|}\sum_{\langle\sigma^\dagger, \sigma'\rangle \in \mathcal{D}_{\text{test}}} \mathds{1}\left[\sigma' \in \argmax_{\bm{\sigma} \subseteq R(\sigma^\dagger), |\bm{\sigma}| \leq k}\sum_{\sigma \in \bm{\sigma}}\text{Score}(\sigma)\right]
\end{equation}

This is a variation on a standard metric used in information retrieval and a common way of measuring the quality of ranked results in machine translation and recommender systems. Precision@All or completeness may be seen as a special case where $k=\infty$.

%By default, Tidyparse uses the DFA decoder (Alg. 3) for all experiments, however, we also include a comparison with a na\"ive rejection-based edit sampler, as well as enumerative sampling with PCFG reranking (Alg. 1) and c-gram reranking (Alg. 2) in our ablation study (\S~\ref{sec:rq3}).

We compare our method against two external baselines, Seq2Parse and Break-It-Fix-It (BIFI)~\cite{yasunaga2021break} on a single test set. This dataset~\cite{wong2019syntax} consists of 20k naturally-occurring pairs of Python errors and their corresponding human fixes from StackOverflow, and is used to compare the precision of each method at blind recovery of the ground truth repair across varying edit distances, snippet lengths and latency cutoffs. We preprocess all source code by filtering for broken-fixed snippet pairs shorter than 100 tokens and fewer than five Levenshtein edits apart, whose broken and fixed form is rejected and accepted, respectively, by the Python 3.8.11 parser. We then balance the dataset by sampling an equal number of repairs from each length and Levenshtein edit distance.

%  In our synthetic experiments, we apply the pretrained BIFI breaker to synthetically corrupt Python snippets from the BIFI good code test set, using the clean source as the ground truth repair, and filter broken-fixed snippet pairs by the same criteria.

The Seq2Parse and BIFI experiments were conducted on a single Nvidia V100 GPU with 32 GB of RAM. For Seq2Parse, we use the default pretrained model provided in commit \texttt{7ae0681}~\footnote{https://github.com/gsakkas/seq2parse/tree/7ae0681f1139cb873868727f035c1b7a369c3eb9}. Since it was unclear how to extract multiple repairs from their model, we only take a single repair prediction. For BIFI, we use the Round 2 breaker and fixer from commit \texttt{ee2a68c}\footnote{https://github.com/michiyasunaga/BIFI/tree/ee2a68cff8dbe88d2a2b2b5feabc7311d5f8338b}, the highest-performing model reported by the authors, with a variable-width beam search to control the number of predictions, and let the BIFI fixer model predict the top-k repairs, for $k=\{1, 5, 10, 2\times10^4\}$.

The language intersection experiments were conducted on a MacBook M4 Max with 16GB of memory. To train our c-gram model, we use an order-4 Markov chain trained on 55 million BIFI tokens. Training takes ~10 minutes, after which re-ranking is nearly instantaneous. Sequences are scored using NLL with Laplace smoothing and our evaluation measures Precision@\{1, 5, 10, All\}.

\clearpage\subsection{Dataset statistics}\label{sec:rq1}

In the following experiments, we use a dataset of Python snippets consisting of 20,500 pairwise-aligned human errors and fixes from StackOverflow~\cite{wong2019syntax}. We preprocess the dataset to lexicalize all code snippets, then filter by length and distance shorter than 80 lexical tokens and under five edits, i.e., with Levenshtein distance under five lexical edits $\big(|\Sigma| = 50, |\err{\sigma}| < 80, \Delta(\err{\sigma}, \sigma') < 5\big)$. We depict the length, edit distance, normalized edit locations and stability profile in Fig.~\ref{fig:patch_stats}.\vspace{-0.2cm}

\begin{figure}[h!]
\begin{tikzpicture}[scale=0.57]
  \begin{axis}[
ybar,
bar width=5pt,
xlabel={Snippet length, $|\err\sigma|$},
ylabel={Frequency},
title={Cumulative length distribution},
axis x line*=bottom,
axis y line*=left,
ymin=0,
ymax=65,
xtick=data,
xticklabels={,<20,,<40,,<60,,<80,,<100},
ymajorgrids=true,
grid style=dashed,
width=0.45\textwidth,
height=0.3\textwidth
]

\addplot[fill=black!30] table {
  X Y
  1 7.60
  2 14.52
  3 22.01
  4 30.54
  5 37.82
  6 44.30
  7 49.68
  8 55.21
  9 59.75
  10 63.59
};
\draw[red, dashed] (axis cs:8.5,0) -- (axis cs:8.5,65);
\end{axis}
\end{tikzpicture}
\begin{tikzpicture}[scale=0.57]
\begin{axis}[
ybar,
bar width=5pt,
title={Human repair distance},
xlabel={Edit distance, $\Delta(\err\sigma, \sigma')$},
ylabel={Frequency},
axis x line*=bottom,
axis y line*=left,
xtick=data,
ymajorgrids=true,
grid style=dashed,
xticklabels={,\leq 2,,\leq 4,,\leq 6,,\leq 8,,\leq 10},
ytick={0, 20, 40, 60, 80, 100},
ymin=0,
width=0.45\textwidth,
height=0.3\textwidth
]
\addplot[fill=black!30] table {
X Y
1  31.48
2  47.52
3  54.89
4  60.44
5  63.88
6  66.38
7  68.02
8  70.04
9  71.49
10 72.22
};
\draw[red, dashed] (axis cs:4.5,0) -- (axis cs:4.5,80);
\end{axis}
\end{tikzpicture}
\begin{tikzpicture}[scale=0.57]
\begin{axis}[
ybar,
bar width=5pt,
xlabel={Beginning $\longleftrightarrow$ End},
ylabel={Frequency},
title={Normalized edit locations},
axis x line*=bottom,
axis y line*=left,
ymin=0,
ymax=35,
xtick=data,
xticklabels={,20\%,,40\%,,60\%,,80\%,,100\%},
ymajorgrids=true,
grid style=dashed,
width=0.45\textwidth,
height=0.3\textwidth
]

\addplot[fill=black!30] table {
X Y
10 11.6539
20 5.7252
30 6.2087
40 5.9542
50 5.5980
60 7.9389
70 7.0738
80 6.9466
90 12.4173
100 30.4835
};
\end{axis}
\end{tikzpicture}
%    \begin{tikzpicture}
%      \begin{axis}[
%        ybar,
%        bar width=5pt,
%        title={Intra-patch edit distance},
%        xlabel={Caret distance},
%        ylabel={Frequency},
%        axis x line*=bottom,
%        axis y line*=left,
%        xtick=data,
%        ymajorgrids=true,
%        grid style=dashed,
%        xticklabels={1,2,3,4,5,6,7,8,9,10+},
%        width=0.45\textwidth,
%        height=0.3\textwidth
%      ]
%
%        \addplot table {
%          X Y
%          1 40.66
%          2 15.00
%          3 5.80
%          4 4.86
%          5 4.26
%          6 2.98
%          7 2.05
%          8 2.73
%          9 1.62
%          10 13.64
%        };
%      \end{axis}
%    \end{tikzpicture}
\begin{tikzpicture}[scale=0.57]
\begin{axis}[
legend cell align={left},
legend style={fill opacity=1, draw opacity=1, text opacity=1, draw=lightgray204, legend columns=-1, legend pos=south east},
xlabel={Snippet length, $|\err\sigma|$},
ylabel={Stable region},
title={Stability profile},
ybar,
axis lines*=left,
xtick={0, 10, 20, 30, 40, 50, 60, 70},
ytick={0, 0.2, 0.4, 0.6, 0.8, 1.0},
xticklabels={, {[}10{,}20{)}, , {[}30{,}40{)}, , {[}50{,}60{)}, , {[}70{,}80{)}},
yticklabels={0, 0.2, 0.4, 0.6, 0.8, 1.0},
x tick label style={font=\scriptsize},
ymax=1.0,
ymin=0.0,
bar width=3pt,
grid style=dashed,
ymajorgrids=true,
width=0.45\textwidth,
height=0.3\textwidth
]
\addlegendimage{empty legend}
\addlegendentry{$\Delta(\err\sigma, \sigma')=$}
\addlegendimage{ybar,ybar legend,draw=none,green,fill=green!50}
\addlegendentry{1,}
\addlegendimage{ybar,ybar legend,draw=none,blue,fill=blue!50}
\addlegendentry{2,}
\addlegendimage{ybar,ybar legend,draw=none,orange,fill=orange!50}
\addlegendentry{3}
\addplot[green, fill=green!50] coordinates {(0, 0.80) (10, 0.91) (20, 0.96) (30, 0.97) (40, 0.99) (50, 0.99) (60, 0.99) (70, 0.99)};
\addplot[blue, fill=blue!50] coordinates {(0, 0.35) (10, 0.59) (20, 0.69) (30, 0.73) (40, 0.79) (50, 0.82) (60, 0.84) (70, 0.86)};
\addplot[orange, fill=orange!50] coordinates {(0, 0.23) (10, 0.45) (20, 0.58) (30, 0.66) (40, 0.70) (50, 0.77) (60, 0.78) (70, 0.86)};
\end{axis}
\end{tikzpicture}
\vspace{-0.2cm}
\caption{Repair statistics across the StackOverflow dataset, of which Tidyparse can handle about half in under $\sim$3s and $\sim$4 GB. Larger repairs and edit distances are possible, albeit requiring additional time and memory.}\label{fig:patch_stats}\vspace{-0.2cm}
\end{figure}

We observe that slightly over half of the code snippet pairs in the StackOverflow dataset contain fewer than 80 tokens and five lexical edits, which our method can easily handle (\S~\ref{sec:rq2}). The distribution across edit locations indicates a large fraction of human edits occur near the boundaries of the broken code snippet, however we do not exploit this prior anywhere in the repair process.

For the stability profile, we enumerate repairs for each syntax error and estimate the average fraction of all edit locations that were never altered by any repair in the $L\big(\err\sigma, \Delta(\err\sigma, \sigma')\big)$-ball. For example, on average roughly half of the string is stable for 3-edit syntax repairs in the $[10-20)$ token range, whereas 1-edit repairs of the same length could modify only $\sim 10\%$ of all locations. For a fixed edit distance, we observe an overall decrease in the number of degrees of caret freedom with increasing length, which intuitively makes sense, as the repairs are more heavily constrained by the surrounding context and their locations grow more concentrated relative to the entire string.

\begin{wrapfigure}{r}{0.45\textwidth}
\vspace{-0.4cm}
\resizebox{.45\textwidth}{!}{% This file was created with matplot2tikz v0.3.3.
\tikzset{mark size=1}
\begin{tikzpicture}

\definecolor{darkgray176}{RGB}{176,176,176}
\definecolor{darkorange25512714}{RGB}{255,127,14}
\definecolor{forestgreen4416044}{RGB}{44,160,44}
\definecolor{lightgray204}{RGB}{204,204,204}
\definecolor{steelblue31119180}{RGB}{31,119,180}

\begin{axis}[
legend cell align={left},
legend style={fill opacity=0.8, draw opacity=1, text opacity=1, draw=lightgray204, legend columns=-1, legend pos=north west},
tick align=outside,
tick pos=left,
title={\textbf{Language intersection volume}},
x grid style={darkgray176},
xlabel={\(\displaystyle |\err\sigma|\)},
xmin=-0.75, xmax=103.75,
xtick style={color=black},
y grid style={darkgray176},
log basis y={2},
ymode=log,
axis lines*=left,
ylabel={\(\displaystyle |\ell_\cap|\)},
ymin=1, ymax=400000000,
ytick style={color=black}
]
\addlegendimage{empty legend}
\addlegendentry{$\Delta(\err\sigma, \sigma')= \text{LED} + \{$}
\addlegendentry{$0,$}
\addlegendentry{$1,$}
\addlegendentry{$2\}$}
\addplot [draw=forestgreen4416044, fill=forestgreen4416044, mark=*, only marks]
table{%
x  y
15 7
23 57
16 67
16 121
29 5
28 7
39 102
47 60
37 50
33 8
30 2
35 37
21 26
35 5
18 62
23 35
42 78
54 4
57 8
24 63
62 15
55 57
66 102
36 4
16 4
30 55
11 3
12 6
33 67
9 66
57 16
43 82
25 2
70 39
43 150
31 8
42 51
27 5
11 2
56 8
19 29
13 41
23 35
49 100
31 28
26 34
19 144
93 7
52 26
18 64
54 56
24 46
24 2
22 4
66 1795
98 32
34 9
21 48
16 48
45 78
79 5
15 81
14 14
34 1245
16 4
21 3
30 35
20 64
91 49
93 8
29 37
54 4
71 392
39 17
22 64
46 23
20 48
18 40
23 1716
42 3
27 14
96 29
93 94
26 2
17 58
21 86
28 15
29 70
9 549
89 58
52 30
47 14
51 102
26 95
34 2
16 23
22 42
32 5
33 8
60 254
32 37
21 1001
96 20
33 110
26 28
16 55
25 11
34 9290
16 2
91 36
20 65
36 8
13 56
20 2
19 18
78 23
53 207
67 152
59 176
15 20
59 4
78 28
31 6
16 25
47 3
24 48
71 22
48 57
54 2
32 3
26 43
25 58
18 46
72 72
16 7
16 15
9 226858
8 38
44 6
83 26
22 4
16 165
18 67
33 2
35 17
20 92
13 66
55 51
86 64
48 4
39 6
78 49
45 30
17 38
49 94
55 86
82 4
9 5
54 7
18 37
50 6
40 143380
34 2
11 8
25 16
63 102
25 2
15 80
60 13
49 16
81 56
84 49
73 4
20 13
92 4
39 28
74 8
39 4160
51 4
6 4
27 6
31 86
32 61
37 14
83 31
29 26
5 3
26 4
85 5
73 78
31 7
26 55
16 95
93 7
14 1285
29 28
20 13
14 111
8 59
42 22
43 2077
30 8
41 7
33 47
78 28
23 47
80 26
73 4
17 15
76 4
33 36
37 26
53 70
27 240
36 3
14 8
38 4
17 16
31 2
43 61
41 6
41 231
21 63
60 58
74 289
17 59
28 7
44 11
24 2660
34 4
27 117
50 1156
35 121
76 20
36 3
33 2
66 3
29 11
56 36
31 1067
52 15
33 20
63 841
36 27
35 88
5 35
49 80
13 15
63 2914
39 4
9 3
9 63
36 3
75 222
21 69
12 16
27 8
55 26
49 56
57 80
73 15
33 52
17 66
31 11
26 63
65 7
42 37
20 2
85 3
10 4095
56 6
8 47
41 28
92 38
29 7
8 32
9 90
16 112
40 39
82 14
67 4
78 20
27 3
40 1382
25 14
35 35
29 75
42 28
21 4351
59 49
25 146
28 15
7 11
18 2
64 38
18 3
28 33
62 64
29 33
27 14
19 64
83 237
45 28
36 78
94 7
56 154
39 2
41 13
23 82
24 26
76 29
41 90
9 2
27 20
83 4
32 10
58 33
34 4
9 46
22 18
65 88
27 53
36 58
16 14
97 90
89 19
23 98
56 335
35 3
62 29
42 13
74 47
47 28
9 64
25 64
37 101
34 112
72 71
29 26
76 120
35 85
42 1020
24 3
34 83
51 156
31 12
36 3
76 897
34 94
32 58
58 34
22 36
85 8
11 2
40 8
63 63
36 30
88 274
22 20
24 46
15 32
84 35
62 34
23 5
31 27
53 38
85 3
92 320
20 83
34 109
15 3
79 16
50 64
30 115
22 26
84 64
15 83
33 40
13 83
11 40
39 2
35 38
28 28
77 23
13 12
12 26
20 96
51 5
6 2
19 143
75 37
31 49
35 91
20 57
20 28
37 130
29 14
38 82
54 1223
36 58
25 11
24 76
95 2
97 70
60 62
72 26
58 4
13 78
28 6
33 20
42 53
20 37
44 41
35 29
12 12
57 28
78 56
50 7
35 40
29 9
16 10
79 29
12 102
17 41
68 17
29 36
72 80
30 2
23 311
29 19
63 8
36 81
13 68
22 85
11 17
36 79
86 121
86 9
25 94
25 31
77 58
97 13
39 28
25 34
36 2
16 10
43 49
37 13
50 76
95 80
98 49
27 3
4 46
32 11
18 31
10 58
29 15
35 4
33 222
17 58
11 32
24 30
44 628
22 19
57 8
30 102
69 32
63 7
26 47
24 12
77 1630
40 4
26 46
25 58
31 5
43 397
21 99
26 4
15 2
53 28
88 19
72 21
15 89
5 41
69 28
71 3
51 1156
66 69
57 35
33 38
48 555
12 61
40 82
31 9
93 65926
17 7
42 7
72 8
76 6395
43 38
25 58
34 9
54 52
45 7
19 86
35 3
30 19
48 36
68 49
15 262015
31 11547
82 9
28 101
37 15
34 18
32 35
39 8
36 35
33 5310
16 63
18 46
42 6
85 59
32 55
22 19
15 2
76 26
23 16
33 12
14 11
17 7
21 23
19 14
47 34
13 48
24 7
28 122
52 2
81 35
67 7
38 2
33 28
64 58
31 102
36 33
24 397
88 4
41 4
45 15
61 4
13 46
19 16
20 20
62 473
71 8
23 115
78 102
41 18
25 52
23 11
44 9
4 8
25 3
54 34
25 2
28 8
35 70
38 14
56 26
53 79
49 2
74 7
28 17
53 40
54 102
20 40
12 37
28 6648
28 64
88 7
27 27
95 50
33 81
30 4
27 41
45 2
46 34
50 35
82 35
52 289
48 8
25 113
79 47
19 66
89 5
45 37
40 3
16 28
24 43
64 78
27 35
17 88
15 79
24 27
32 80
16 37
10 41
39 63
17 72
55 102
26 102
35 86
96 43
40 8
30 33
25 11
26 102
21 91
26 7
14 48
47 58
14 58
34 165
14 23
23 10
24 47
15 19
96 8
43 65
29 29
38 68
20 13
18 10
34 3
48 96
84 4
28 139
43 38
15 5
60 46
19 62
11 14
66 58
22 19
18 37
74 5
61 41
50 11
76 7
16 6
39 1681
30 14
15 29
22 80
13 27
20 13
61 4
57 35
19 15
65 2
55 13
20 2
20 68
37 18
11 39
21 16
44 109
53 38
32 64
30 37
70 35
48 106
20 88
40 288
45 102
36 55
22 2
29 88
78 35
51 3069
69 6
33 4
52 4
71 3
52 40
27 2
53 11
58 38
11 13
11 7
28 2
35 41
16 68
37 29
30 52
38 15
38 6
36 12
27 47
21 3
25 28
19 38
53 7
21 3
12 35
35 40
12 2126
87 3
69 82
43 26
13 4
24 7
18 64
34 1836
76 117936
51 67
73 28
55 15
20 92
27 5
39 38
27 389
18 8
95 961
41 3845
12 36
31 2
39 36
11 8
23 27
14 894
76 3
12 39
31 5
19 102
17 45
18 15
16 2
52 82
43 35
22 166
32 63
50 2401
38 5
43 28
51 5
96 102
86 77
72 276
51 20
26 86
40 4
38 48
50 64
35 28
17 78
13 7
42 48
15 758
19 5
21 82
33 38
38 11
25 6
40 16
25 66
14 34
37 167
29 36
11 13
12 3
33 57
17 66
22 14
35 339
34 59
57 8
57 53
31 99
99 13
82 71952
18 8
17 58
9 16
53 10
18 15
26 64
49 7
47 559
17 8
41 4
23 2
35 11
10 56
19 1301
21 78
38 63
33 212
33 32
41 67
42 84
19 62
42 4
9 46
21 56
20 11
16 49
38 31
77 15
73 7
41 8
26 56
49 111
15 59
36 11
61 4
16 37
67 46
71 49
31 14
18 6
19 56
42 79
33 148
5 22
66 22
53 28
20 7
8 58
19 28
82 51
81 83
19 39
74 7
40 16
33 733
24 17
27 8
39 37
35 146
21 11
73 41
43 15
18 7
70 35
34 144
13 34
11 25
71 32
8 10
15 8
71 4
58 34
30 26
14 26
82 210
22 66
44 8
10 16
46 4
46 15
63 54
10 754
27 384
24 5
47 43
56 2
29 57
30 2078
88 185
16 34
29 3
21 841
35 211
13 81
48 10
37 2357
65 7
22 3
20 63
21 11
43 3
73 78
24 63
14 40
17 4896
24 70
99 28
25 81
65 4
63 35
22 47
59 21
36 79
34 18
65 2
34 3
28 13
61 2
15 32
11 20
15 398
12 108
46 42
28 3
48 34
53 80
27 27
24 2
34 2
63 15
42 41
30 58
13 80
27 32
21 1764
17 2
34 27
45 135
24 4
27 2
61 11
5 3
96 28
34 88
60 169
22 10404
80 3
29 22
25 398
63 27
70 102
67 17
10 37
69 10404
33 183
84 23
15 58
31 11
35 12
19 86
23 33
52 47
51 38
20 5
9 28
59 2442
39 7
78 58
75 3
55 5
57 26
42 14
22 5
60 2
31 2
41 2
29 6
50 41
24 26
9 12
89 6
15 17
43 30
29 82
11 72
46 10
14 4
26 7
28 37
30 3
22 3
20 16
71 5
20 86
16 10
19 31
30 9
29 47
24 17
60 2112
42 3782
34 15
23 890
20 7
18 4
34 80
29 37
60 88
29 44
16 30
21 188
22 34
13 57
37 86
48 4
34 101
90 26
17 6
32 57
47 157
36 26
92 16
79 80
13 18
22 35
50 30
79 5
30 82
52 13
10 15
35 79
17 31
27 7
38 43
15 3
40 29
45 15
54 8
35 4
45 2
18 30
27 37
83 28
43 5
15 65
50 38
74 81
13 6
37 29
15 5
70 58
24 18
38 58
69 14
55 13
29 3
8 102
18 48
21 784
29 150
46 9
};
\addlegendentry{Lev Margin 0}
\addplot [draw=steelblue31119180, fill=steelblue31119180, mark=*, only marks]
table{%
x  y
6 809
24 12540
36 53800
59 34162
51 700
18 519
28 4184
65 83294
62 55274
15 15979
53 13841
41 23538
33 9101
51 1022
24 27172
36 8140
41 32883
51 14603
40 32882
52 8119
5 113
89 49372
55 4266
92 14839
29 26412
10 7799
24 23211
49 41879
51 31387
12 5880
65 127928
41 57332
38 23940
96 66192
13 3378
75 11146
88 84840
58 52570
30 3151
5 4661
13 13464
23 20487
36 21811
61 3086
32 756205
11 7145
9 1667
19 12210
49 58508
6 4189
75 969
36 26836
30 17095
82 2758945
5 6848
27 21866
37 23858
49 43970
50 27314
39 24705
16 14948
12 3118
44 33727
11 6340
26 5204
29 25506
21 21102
29 72055
36 8400
30 55417
75 2063
52 13905
37 30039
78 13146
71 9942
43 49896
46 46395
29 72055
30 120759
69 67463
36 2377
27 18917
30 53999
73 59251
8 14929
14 2787
19 18452
72 17668
39 6549
25 20529
32 29476
46 2868
26 28347
23 14589
15 2252
49 5456
33 32191
15 5070
70 60123
51 68503
17 17684
35 31956
37 44278
86 89315
68 57625
45 52339
55 38678
55 33058
65 248824
15 8486
51 27138
20 15102
24 19652
16 12073
36 54859
42 20691
8 5195
40 4948
54 3676
17 3972
27 31012
23 24794
41 5128
14 163
8 3686
70 4222
89 63703
8 7244
29 29379
78 64496
31 20061
59 78685
12 4715
36 27955
33 2762
27 39578
40 1504
32 12456
62 35826
46 29168
19 18998
14 13963
66 216106
31 24452
7 6192
12 3118
49 12514
16 887
18 2590
11 362
20 7361
69 85262
58 91098
48 82285
7 8302
29 2810
71 36386
10 472
19 24546
66 37107
71 63518
87 67891
78 58763
51 3301
45 180311
30 3564
51 3714
19 11440
22 14867
60 560567
87 3126013
95 62227
32 27569
29 4247
36 28036
42 32482
30 18479
65 43194
46 19563
19 13341
25 13490
20 2201
55 50842
37 27616
37 20564
63 8766
11 7027
15 6857
89 1781
88 5762
27 22314
94 81020
71 58140
69 54058
35 5309
37 69824
15 11283
46 37720
34 5863
45 30873
64 59089
8 5195
35 21865
28 6636
66 24850
47 12594
51 6811
18 2748
30 181234
34 12707
16 5591
61 78395
28 2801
53 107048
33 1422
28 13620
74 83536
21 3035
68 34167
29 72055
39 24754
31 178497
33 20442
13 6609
35 36728
83 46073
88 5772
26 47726
77 66291
67 71018
34 22259
9 1362
92 1819
44 7314
59 2051
27 26257
37 36644
50 6160
41 22695
22 884
89 4407
26 18958
33 15173
86 4522
64 33980
53 24449
94 4845
37 180646
30 28123
48 1730888
49 38387
62 55628
82 123529
39 15518
25 2986
11 28715
16 439
40 29746
28 10537
28 13457
97 26397
31 33002
7 6750
39 41304
29 27044
30 18263
10 1829
10 5881
23 14586
81 89843
50 182108
26 22749
88 47697
83 74637
11 8319
24 6398
17 7346
71 30657
32 2037
20 31128
26 54
29 17651
52 10489
17 11000
8 5195
40 24180
19 24475
41 21634
55 42852
80 14924
10 6271
60 56666
49 37249
26 31592
34 35360
12 2861
9 159
63 10695
34 14302
26 3696
68 85236
15 18468
52 44712
65 88978
16 1704
8 12442
39 32449
20 24317
49 40860
37 35876
77 6441
59 2192
86 4109
18 17031
45 2880
52 2012
36 21298
26 10492
28 16348
24 9035
21 13143
42 42868
26 11273
77 68403
82 79680
67 87097
55 49331
25 19289
16 2056
11 1460
10 1277
16 11308
16 16821
67 55129
25 2250
23 23640
29 3131
97 24091
41 50199
31 57199
4 525
28 13610
40 22984
29 4045
63 56345
34 53381
26 46098
27 10938
53 48066
43 116944
29 72055
72 50163
23 24321
23 1645
33 9560
70 31564
19 4523
88 107498
56 44182
41 4713
48 23195
46 20691
19 11026
27 31741
58 35024
25 19459
16 13859
50 60321
9 3550
91 19714
70 29635
56 4184
94 30367
63 9365
13 1950
40 19920
63 54393
43 24367
23 19076
69 689287
45 28936
49 1395
47 33402
43 131806
19 12507
62 15802
30 1262
32 548
11 1460
32 55669
27 27139
69 114891
23 2830
38 27623
82 206450
65 3804
24 91102
26 3152
14 7998
19 9522
53 54284
75 402305
39 770
9 6021
37 1788
45 29875
60 17834
12 279347
38 8193
64 1091495
18 947
28 4675
30 23798
25 462391
36 37644
84 1286
30 19989
27 22707
61 728118
46 29499
23 3013
72 36718
28 38279
97 12049
63 2114
26 3012
9 1601
45 7204
23 8614
33 4900
54 91740
18 15379
21 2517
89 24243
67 61284
29 28516
64 76746
29 17422
76 1088863
41 67784
17 923
36 22089
53 54052
14 7067
10 17759
49 42904
40 5548
43 99403
39 31780
81 273594
14 1562
21 500
9 1506
30 10866
52 2276
25 11030
40 32242
45 41063
13 10177
33 26227
50 33097
75 70150
60 4209
88 74322
32 7632
59 22156
35 342491
18 16178
15 15916
45 50958
16 19251
76 59356
56 433
44 6241
27 959
48 34744
91 67968
21 17680
72 60290
38 1392
8 141
33 25963
94 91958
47 9866
22 20389
20 21992
21 10804
13 4033
47 41793
33 7504
11 7027
8 2305
39 4246
49 43250
25 15682
84 83492
17 8946
22 24708
40 31387
81 168515
50 51788
19 18998
37 31533
77 79838
27 18230
42 51796
12 2093
33 3280
11 5558
53 6586
15 12860
87 4734
34 23730
61 6973
7 419
79 60996
91 73876
32 599960
33 21434
};
\addlegendentry{Lev Margin 1}
\addplot [draw=darkorange25512714, fill=darkorange25512714, mark=*, only marks]
table{%
x  y
38 1761664
32 3855870
20 852624
45 14663997
47 376033
58 18231184
56 170260
31 73726
53 1831132
95 57830719
43 863387
67 12696630
17 2098301
59 24778010
17 2240035
44 204177
37 5681724
18 4105974
77 22229933
28 9289358
63 20663115
34 274048
24 373626
41 6573601
34 879291
14 1024069
12 1349310
20 2155283
18 72053
88 16734099
31 3708284
36 13084204
25 6835180
93 2808110
16 3553168
56 11481040
12 98281
9 97953
30 57774
44 1157505
13 198316
11 1144472
33 1564362
18 450595
89 64028789
87 70370367
17 524593
31 7766430
70 314048
38 1567762
20 3909452
28 2062680
28 91010
70 6947754
73 7661113
25 4855540
26 552551
99 31844218
40 1895407
37 13333940
78 277626
29 1168007
32 7162021
21 825779
34 10010353
23 1583408
41 1036751
55 11137630
13 3663628
16 363587
32 2753832
44 9426605
18 473850
73 32731442
38 3692891
18 1147019
37 3735747
35 8479370
10 905170
10 243005
94 64472605
9 647530
11 1151495
78 26113548
46 10765001
77 5541437
10 1043338
47 19502165
27 2655775
46 2574607
23 427206
67 3640486
28 3242461
51 101926
22 3068263
15 2499221
64 6120523
9 465895
22 142208
69 15620809
14 1944834
23 308507
15 1707842
93 132267574
94 30772385
49 53767
51 1288510
31 4157788
36 2245258
75 45804139
38 4189522
29 118014
19 3979566
28 8234093
71 225754
89 14196814
43 8750889
81 36495261
20 426307
67 5846382
32 5680355
60 4539524
48 11022011
65 209419
53 12279975
45 4109575
29 4622928
84 21220569
};
\addlegendentry{Lev Margin 2}
\end{axis}

\end{tikzpicture}
}
\vspace{-0.8cm}
\caption{Log language volume versus snippet length and edit distance for Python repairs.}
\label{fig:volumetric_plot}
\vspace{-0.2cm}
\end{wrapfigure}

For an intuition about the size of the langauge intersections involved in syntax repair, volumetric analysis will be helpful, particularly in understanding the influence of snippet length and edit distance on language intersection volume. To measure the intersection volume we will form the $L\big(\err\sigma, \Delta(\err{\sigma}, \sigma')\big)$ automaton, intersect it with the Python grammar, then automatize the resulting regular expression and finally compute the DFA transfer matrix using the method described in \S~\ref{sec:transfer_method} to obtain the exact volume. For a given (error, fix) pair, this tells us how many repairs of equal or lesser distance exist in the Python langauge. Plotting intersection volume across the full dataset (Fig.~\ref{fig:volumetric_plot}), we observe a strong positive correlation with the Levenshtein margin and a mild correlation with snippet length. Fully materializing $\ell_\cap$ is typically only feasible if we extend the Levenshtein radius up to one edit beyond the langauge edit distance (LED) $\big(\text{i.e., } d_{\max} \leq \text{LED}(\err\sigma)$\footnote{Where $\text{LED}(\err\sigma)$ is shorthand for $\text{LED}(\err\sigma, \ell) = \min \big\{d_{\max}: \mathbb{N}\mid \mathcal{L}\big(L(\err\sigma, d_{\max})\big) \cap \ell \neq \varnothing\big\}$ with $\ell$ being the Python language.}$+ 1\big)$ after which it grows too large to exhaustively generate and must be sampled. Across all snippets where $\Delta(\err{\sigma}, \sigma') < 5$, approximately 54\% matched $\text{LED}(\err\sigma)$, 35\% had an edit distance of $\text{LED}(\err\sigma) + 1$ and 11\% had a distance of $\text{LED}(\err\sigma) + 2$.
%Note that this analysis does not depend on our method, but is rather an empirical observation about naturally-occurring Python syntax errors and repairs. Furthermore, it does not tell us how many of those repairs were semantically admissible, i.e., typesafe, the quantity of which may be an order of magnitude smaller.

\clearpage\subsection{StackOverflow evaluation}\label{sec:rq2}

For our first experiment, we measure the precision of our repair procedure at various lengths and Levenshtein distances. We rebalance the StackOverflow dataset across each length interval and edit distance, sample uniformly from each category and compare Precision@1 of our method against Seq2Parse, vanilla BIFI and BIFI with a beam size and precision at $2\times10^4$ distinct samples.

\begin{figure}[h!]
\resizebox{.24\textwidth}{!}{\begin{tikzpicture}
  \begin{axis}[
    xlabel={$|\sigma|$},
    ylabel={Precision@1},
    title={Tidyparse Repair Precision},
    ybar,
    axis lines*=left,
    xtick={0, 10, 20, 30, 40, 50, 60, 70},
    ytick={0, 0.1, 0.2, 0.3, 0.4, 0.5, 0.6, 0.7, 0.8, 0.9, 1.0},
    ymax=1.0,
    ymin=0.0,
    bar width=4pt,
  ]

  \addplot[green, fill=green] coordinates {(0, 1.0) (10, 1.0) (20, 1.0) (30, 1.0) (40, 1.0) (50, 1.0) (60, 1.0) (70, 1.0)};
  \addplot[blue, fill=blue] coordinates {(0, 0.3) (10, 0.286) (20, 0.205) (30, 0.433) (40, 0.256) (50, 0.296) (60, 0.236) (70, 0.315)};
  \addplot[orange, fill=orange] coordinates {(0, 0.46875) (10, 0.321) (20, 0.366) (30, 0.24) (40, 0.407) (50, 0.454) (60, 0.574) (70, 0.526)};

%  \legend{Δ=1,Δ=2,Δ=3}
  \end{axis}
\end{tikzpicture}}
\resizebox{.24\textwidth}{!}{\begin{tikzpicture}
  \begin{axis}[
  xlabel={Snippet length, $|\sigma|$},
  ylabel={Precision@20k},
  title={\textbf{BIFI Repair Precision@20k}},
  legend cell align={left},
  legend style={fill opacity=0.8, draw opacity=1, text opacity=1, draw=lightgray204, legend columns=-1, legend pos=north east},
  ybar,
  axis lines*=left,
  xtick={0, 10, 20, 30, 40, 50, 60, 70},
  ytick={0, 0.1, 0.2, 0.3, 0.4, 0.5, 0.6, 0.7, 0.8, 0.9, 1.0},
  xticklabels={{(}0{,}10{)}, {[}10{,}20{)}, {[}20{,}30{)}, {[}30{,}40{)}, {[}40{,}50{)}, {[}50{,}60{)}, {[}60{,}70{)}, {[}70{,}80{)}},
  x tick label style={font=\scriptsize},
  ymax=1.0,
  ymin=0.0,
  bar width=4pt,
  ]

  \addlegendimage{empty legend}
  \addlegendentry{$\Delta(\err\sigma, \sigma')=$}
  \addlegendimage{ybar,ybar legend,draw=none,green,fill=green!50}
  \addlegendentry{1,}
  \addlegendimage{ybar,ybar legend,draw=none,blue,fill=blue!50}
  \addlegendentry{2,}
  \addlegendimage{ybar,ybar legend,draw=none,orange,fill=orange!50}
  \addlegendentry{3}

  \addplot[green, fill=green!50] coordinates   {(0, 0.65) (10, 0.67) (20, 0.71) (30, 0.63) (40, 0.60) (50, 0.62) (60, 0.59) (70, 0.64)};
  \addplot[blue, fill=blue!50] coordinates     {(0, 0.52) (10, 0.41) (20, 0.35) (30, 0.31) (40, 0.27) (50, 0.27) (60, 0.21) (70, 0.22)};
  \addplot[orange, fill=orange!50] coordinates {(0, 0.25) (10, 0.08) (20, 0.08) (30, 0.17) (40, 0.11) (50, 0.17) (60, 0.08) (70, 0.08)};

  \end{axis}
\end{tikzpicture}
}
\resizebox{.24\textwidth}{!}{\begin{tikzpicture}
  \begin{axis}[
    xlabel={Snippet length, $|\sigma|$},
    ylabel={Precision@1},
    title={\textbf{Seq2Parse Repair Precision@1}},
    ybar,
    axis lines*=left,
    xtick={0, 10, 20, 30, 40, 50, 60, 70},
    ytick={0, 0.1, 0.2, 0.3, 0.4, 0.5, 0.6, 0.7, 0.8, 0.9, 1.0},
    xticklabels={{(}0{,}10{)}, {[}10{,}20{)}, {[}20{,}30{)}, {[}30{,}40{)}, {[}40{,}50{)}, {[}50{,}60{)}, {[}60{,}70{)}, {[}70{,}80{)}},
    x tick label style={font=\scriptsize},
    ymax=1.0,
    ymin=0.0,
    bar width=4pt,
  ]

  \addplot[green, fill=green!50] coordinates {(0, 0.352631) (10, 0.413115) (20, 0.400502) (30, 0.378440) (40, 0.308869) (50, 0.287755) (60, 0.268817) (70, 0.210526)};
  \addplot[blue, fill=blue!50] coordinates {(0, 0.122529) (10, 0.126453) (20, 0.144192) (30, 0.118483) (40, 0.108007) (50, 0.106849) (60, 0.097403) (70, 0.122047)};
  \addplot[orange, fill=orange!50] coordinates {(0, 0.03125) (10, 0.070922) (20, 0.077348) (30, 0.087629) (40, 0.094675) (50, 0.02) (60, 0.066038) (70, 0.063291)};

  \end{axis}
\end{tikzpicture}}
\resizebox{.24\textwidth}{!}{\begin{tikzpicture}
  \begin{axis}[
    legend cell align={left},
    legend style={fill opacity=0.8, draw opacity=1, text opacity=1, draw=lightgray204, legend columns=-1, legend pos=north east},
    xlabel={Snippet length, $|\sigma|$},
    ylabel={Precision@1},
    title={\Large\textbf{BIFI Repair Precision@1}},
    ybar,
    axis lines*=left,
    xtick={0, 10, 20, 30, 40, 50, 60, 70},
    ytick={0, 0.1, 0.2, 0.3, 0.4, 0.5, 0.6, 0.7, 0.8, 0.9, 1.0},
    xticklabels={{(}0{,}10{)}, {[}10{,}20{)}, {[}20{,}30{)}, {[}30{,}40{)}, {[}40{,}50{)}, {[}50{,}60{)}, {[}60{,}70{)}, {[}70{,}80{)}},
    x tick label style={font=\scriptsize},
    ymax=0.6,
    ymin=0.0,
    bar width=4pt,
  ]
    \addlegendimage{empty legend}
    \addlegendentry{$\Delta(\err\sigma, \sigma')=$}
    \addlegendimage{ybar,ybar legend,draw=none,green,fill=green!50}
    \addlegendentry{1,}
    \addlegendimage{ybar,ybar legend,draw=none,blue,fill=blue!50}
    \addlegendentry{2,}
    \addlegendimage{ybar,ybar legend,draw=none,orange,fill=orange!50}
    \addlegendentry{3}

    \addplot[green, fill=green!50] coordinates {(0, 0.196013) (10, 0.326401) (20, 0.318538) (30, 0.272843) (40, 0.213894) (50, 0.206651) (60, 0.247525) (70, 0.179245)};
    \addplot[blue, fill=blue!50] coordinates {(0, 0.174603) (10, 0.176651) (20, 0.209573) (30, 0.19195) (40, 0.18851) (50, 0.176166) (60, 0.110787) (70, 0.106383)};
    \addplot[orange, fill=orange!50] coordinates {(0, 0.015873) (10, 0.021858) (20, 0.030435) (30, 0.02439) (40, 0.032922) (50, 0.045) (60, 0.027397) (70, 0.017094)};
  \end{axis}
\end{tikzpicture}}
\caption{Tidyparse, Seq2Parse and BIFI repair precision at various lengths and Levenshtein distances.}\label{fig:len_dist_prec}
\end{figure}

As we can see, Tidyparse has a highly competitive top-1 precision versus Seq2Parse and BIFI across all lengths and edit distances, and attains a significant advantage in the few-edit regime. The Precision@1 of our method is even competitive with BIFI's Precision@20k, whereas our Precision@All is Pareto-dominant across all lengths and edit distances, while requiring only a fraction of the data and compute. We report the raw data from these experiments in Appendix~\ref{sec:raw_prec_data}.

Next, we measure the precision at various ranking cutoffs and wall-clock timeouts. Our method attains the same precision as Seq2Parse and BIFI for 1-edit repairs at comparable latency, however Tidyparse takes longer to attain the same precision for 2- and 3-edit repairs. BIFI and Seq2Parse both have subsecond single-shot latency but are neural models trained on a much larger dataset.

\begin{figure}[h!]
%    \resizebox{.19\textwidth}{!}{% This file was created with tikzplotlib v0.10.1.
\begin{tikzpicture}

\definecolor{darkgray176}{RGB}{176,176,176}
\definecolor{darkviolet1270255}{RGB}{127,0,255}
\definecolor{deepskyblue3176236}{RGB}{3,176,236}
\definecolor{dodgerblue45123246}{RGB}{45,123,246}
\definecolor{royalblue8762253}{RGB}{87,62,253}

\begin{axis}[
tick align=outside,
tick pos=left,
axis lines=left,
title={\(\displaystyle \Delta\in[1,3]\) Repair Precision},
legend style={fill opacity=0.8, draw opacity=1, text opacity=1, legend columns=1, legend pos=north west},
x grid style={darkgray176},
xlabel={Seconds},
xmin=-0.5925, xmax=8.5925,
xtick style={color=black},
xtick={0,1,2,3,4,5,6,7,8},
xticklabels={1,2,3,4,5,6,7,8,9},
y grid style={darkgray176},
ylabel={Precision@k},
ymin=0, ymax=0.6993,
ytick style={color=black}
]
\draw[draw=none,fill=darkviolet1270255] (axis cs:-0.175,0) rectangle (axis cs:0.175,0.149);
\addlegendimage{ybar,ybar legend,draw=none,fill=darkviolet1270255}
\addlegendentry{P@All}

\draw[draw=none,fill=darkviolet1270255] (axis cs:0.825,0) rectangle (axis cs:1.175,0.303);
\draw[draw=none,fill=darkviolet1270255] (axis cs:1.825,0) rectangle (axis cs:2.175,0.412);
\draw[draw=none,fill=darkviolet1270255] (axis cs:2.825,0) rectangle (axis cs:3.175,0.501);
\draw[draw=none,fill=darkviolet1270255] (axis cs:3.825,0) rectangle (axis cs:4.175,0.569);
\draw[draw=none,fill=darkviolet1270255] (axis cs:4.825,0) rectangle (axis cs:5.175,0.605);
\draw[draw=none,fill=darkviolet1270255] (axis cs:5.825,0) rectangle (axis cs:6.175,0.63);
\draw[draw=none,fill=darkviolet1270255] (axis cs:6.825,0) rectangle (axis cs:7.175,0.645);
\draw[draw=none,fill=darkviolet1270255] (axis cs:7.825,0) rectangle (axis cs:8.175,0.666);
\draw[draw=none,fill=royalblue8762253] (axis cs:-0.175,0) rectangle (axis cs:0.175,0.128);
\addlegendimage{ybar,ybar legend,draw=none,fill=royalblue8762253}
\addlegendentry{P@10}

\draw[draw=none,fill=royalblue8762253] (axis cs:0.825,0) rectangle (axis cs:1.175,0.239);
\draw[draw=none,fill=royalblue8762253] (axis cs:1.825,0) rectangle (axis cs:2.175,0.323);
\draw[draw=none,fill=royalblue8762253] (axis cs:2.825,0) rectangle (axis cs:3.175,0.374);
\draw[draw=none,fill=royalblue8762253] (axis cs:3.825,0) rectangle (axis cs:4.175,0.421);
\draw[draw=none,fill=royalblue8762253] (axis cs:4.825,0) rectangle (axis cs:5.175,0.446);
\draw[draw=none,fill=royalblue8762253] (axis cs:5.825,0) rectangle (axis cs:6.175,0.459);
\draw[draw=none,fill=royalblue8762253] (axis cs:6.825,0) rectangle (axis cs:7.175,0.469);
\draw[draw=none,fill=royalblue8762253] (axis cs:7.825,0) rectangle (axis cs:8.175,0.477);
\draw[draw=none,fill=dodgerblue45123246] (axis cs:-0.175,0) rectangle (axis cs:0.175,0.11);
\addlegendimage{ybar,ybar legend,draw=none,fill=dodgerblue45123246}
\addlegendentry{P@5}

\draw[draw=none,fill=dodgerblue45123246] (axis cs:0.825,0) rectangle (axis cs:1.175,0.206);
\draw[draw=none,fill=dodgerblue45123246] (axis cs:1.825,0) rectangle (axis cs:2.175,0.282);
\draw[draw=none,fill=dodgerblue45123246] (axis cs:2.825,0) rectangle (axis cs:3.175,0.325);
\draw[draw=none,fill=dodgerblue45123246] (axis cs:3.825,0) rectangle (axis cs:4.175,0.369);
\draw[draw=none,fill=dodgerblue45123246] (axis cs:4.825,0) rectangle (axis cs:5.175,0.392);
\draw[draw=none,fill=dodgerblue45123246] (axis cs:5.825,0) rectangle (axis cs:6.175,0.405);
\draw[draw=none,fill=dodgerblue45123246] (axis cs:6.825,0) rectangle (axis cs:7.175,0.414);
\draw[draw=none,fill=dodgerblue45123246] (axis cs:7.825,0) rectangle (axis cs:8.175,0.422);
\draw[draw=none,fill=deepskyblue3176236] (axis cs:-0.175,0) rectangle (axis cs:0.175,0.092);
\addlegendimage{ybar,ybar legend,draw=none,fill=deepskyblue3176236}
\addlegendentry{P@1}

\draw[draw=none,fill=deepskyblue3176236] (axis cs:0.825,0) rectangle (axis cs:1.175,0.18);
\draw[draw=none,fill=deepskyblue3176236] (axis cs:1.825,0) rectangle (axis cs:2.175,0.244);
\draw[draw=none,fill=deepskyblue3176236] (axis cs:2.825,0) rectangle (axis cs:3.175,0.285);
\draw[draw=none,fill=deepskyblue3176236] (axis cs:3.825,0) rectangle (axis cs:4.175,0.324);
\draw[draw=none,fill=deepskyblue3176236] (axis cs:4.825,0) rectangle (axis cs:5.175,0.343);
\draw[draw=none,fill=deepskyblue3176236] (axis cs:5.825,0) rectangle (axis cs:6.175,0.356);
\draw[draw=none,fill=deepskyblue3176236] (axis cs:6.825,0) rectangle (axis cs:7.175,0.365);
\draw[draw=none,fill=deepskyblue3176236] (axis cs:7.825,0) rectangle (axis cs:8.175,0.37);
\addplot [red, thick] coordinates {(-0.8925,0.29) (14.8925,0.29)};
\end{axis}

\end{tikzpicture}
}
\resizebox{.24\textwidth}{!}{% This file was created with tikzplotlib v0.10.1.
\begin{tikzpicture}

\definecolor{darkgray176}{RGB}{176,176,176}
\definecolor{darkviolet1270255}{RGB}{127,0,255}
\definecolor{deepskyblue3176236}{RGB}{3,176,236}
\definecolor{dodgerblue45123246}{RGB}{45,123,246}
\definecolor{royalblue8762253}{RGB}{87,62,253}

\begin{axis}[
tick align=outside,
tick pos=left,
axis lines*=left,
title={{\Large\(\displaystyle \Delta=1\) Repair Precision}},
legend style={fill opacity=0.8, draw opacity=1, text opacity=1, legend columns=1, legend pos=north west},
x grid style={darkgray176},
xlabel={Seconds},
xmin=-0.5925, xmax=8.5925,
xtick style={color=black},
xtick={0,1,2,3,4,5,6,7,8},
xticklabels={1,2,3,4,5,6,7,8,9},
y grid style={darkgray176},
ylabel={Precision@k},
ymin=0, ymax=1.0311,
ytick style={color=black}
]
\draw[draw=none,fill=darkviolet1270255] (axis cs:-0.175,0) rectangle (axis cs:0.175,0.546);
\addlegendimage{ybar,ybar legend,draw=none,fill=darkviolet1270255}
\addlegendentry{P@All}

\draw[draw=none,fill=darkviolet1270255] (axis cs:0.825,0) rectangle (axis cs:1.175,0.752);
\draw[draw=none,fill=darkviolet1270255] (axis cs:1.825,0) rectangle (axis cs:2.175,0.88);
\draw[draw=none,fill=darkviolet1270255] (axis cs:2.825,0) rectangle (axis cs:3.175,0.908);
\draw[draw=none,fill=darkviolet1270255] (axis cs:3.825,0) rectangle (axis cs:4.175,0.933);
\draw[draw=none,fill=darkviolet1270255] (axis cs:4.825,0) rectangle (axis cs:5.175,0.954);
\draw[draw=none,fill=darkviolet1270255] (axis cs:5.825,0) rectangle (axis cs:6.175,0.966);
\draw[draw=none,fill=darkviolet1270255] (axis cs:6.825,0) rectangle (axis cs:7.175,0.975);
\draw[draw=none,fill=darkviolet1270255] (axis cs:7.825,0) rectangle (axis cs:8.175,0.982);
\draw[draw=none,fill=royalblue8762253] (axis cs:-0.175,0) rectangle (axis cs:0.175,0.546);
\addlegendimage{ybar,ybar legend,draw=none,fill=royalblue8762253}
\addlegendentry{P@10}

\draw[draw=none,fill=royalblue8762253] (axis cs:0.825,0) rectangle (axis cs:1.175,0.752);
\draw[draw=none,fill=royalblue8762253] (axis cs:1.825,0) rectangle (axis cs:2.175,0.88);
\draw[draw=none,fill=royalblue8762253] (axis cs:2.825,0) rectangle (axis cs:3.175,0.908);
\draw[draw=none,fill=royalblue8762253] (axis cs:3.825,0) rectangle (axis cs:4.175,0.933);
\draw[draw=none,fill=royalblue8762253] (axis cs:4.825,0) rectangle (axis cs:5.175,0.954);
\draw[draw=none,fill=royalblue8762253] (axis cs:5.825,0) rectangle (axis cs:6.175,0.966);
\draw[draw=none,fill=royalblue8762253] (axis cs:6.825,0) rectangle (axis cs:7.175,0.975);
\draw[draw=none,fill=royalblue8762253] (axis cs:7.825,0) rectangle (axis cs:8.175,0.982);
\draw[draw=none,fill=dodgerblue45123246] (axis cs:-0.175,0) rectangle (axis cs:0.175,0.546);
\addlegendimage{ybar,ybar legend,draw=none,fill=dodgerblue45123246}
\addlegendentry{P@5}

\draw[draw=none,fill=dodgerblue45123246] (axis cs:0.825,0) rectangle (axis cs:1.175,0.752);
\draw[draw=none,fill=dodgerblue45123246] (axis cs:1.825,0) rectangle (axis cs:2.175,0.88);
\draw[draw=none,fill=dodgerblue45123246] (axis cs:2.825,0) rectangle (axis cs:3.175,0.908);
\draw[draw=none,fill=dodgerblue45123246] (axis cs:3.825,0) rectangle (axis cs:4.175,0.933);
\draw[draw=none,fill=dodgerblue45123246] (axis cs:4.825,0) rectangle (axis cs:5.175,0.954);
\draw[draw=none,fill=dodgerblue45123246] (axis cs:5.825,0) rectangle (axis cs:6.175,0.966);
\draw[draw=none,fill=dodgerblue45123246] (axis cs:6.825,0) rectangle (axis cs:7.175,0.975);
\draw[draw=none,fill=dodgerblue45123246] (axis cs:7.825,0) rectangle (axis cs:8.175,0.982);
\draw[draw=none,fill=deepskyblue3176236] (axis cs:-0.175,0) rectangle (axis cs:0.175,0.546);
\addlegendimage{ybar,ybar legend,draw=none,fill=deepskyblue3176236}
\addlegendentry{P@1}

\draw[draw=none,fill=deepskyblue3176236] (axis cs:0.825,0) rectangle (axis cs:1.175,0.752);
\draw[draw=none,fill=deepskyblue3176236] (axis cs:1.825,0) rectangle (axis cs:2.175,0.88);
\draw[draw=none,fill=deepskyblue3176236] (axis cs:2.825,0) rectangle (axis cs:3.175,0.908);
\draw[draw=none,fill=deepskyblue3176236] (axis cs:3.825,0) rectangle (axis cs:4.175,0.933);
\draw[draw=none,fill=deepskyblue3176236] (axis cs:4.825,0) rectangle (axis cs:5.175,0.954);
\draw[draw=none,fill=deepskyblue3176236] (axis cs:5.825,0) rectangle (axis cs:6.175,0.966);
\draw[draw=none,fill=deepskyblue3176236] (axis cs:6.825,0) rectangle (axis cs:7.175,0.975);
\draw[draw=none,fill=deepskyblue3176236] (axis cs:7.825,0) rectangle (axis cs:8.175,0.982);
\addplot [red, thick] coordinates {(-0.8925,0.39) (14.8925,0.39)};
\end{axis}

\end{tikzpicture}
}
\resizebox{.24\textwidth}{!}{% This file was created with tikzplotlib v0.10.1.
\begin{tikzpicture}

\definecolor{darkgray176}{RGB}{176,176,176}
\definecolor{darkviolet1270255}{RGB}{127,0,255}
\definecolor{deepskyblue3176236}{RGB}{3,176,236}
\definecolor{dodgerblue45123246}{RGB}{45,123,246}
\definecolor{royalblue8762253}{RGB}{87,62,253}

\begin{axis}[
tick align=outside,
tick pos=left,
title={\(\displaystyle \Delta=2\) Repair Precision},
legend style={fill opacity=0.8, draw opacity=1, text opacity=1, legend columns=1, legend pos=north west},
x grid style={darkgray176},
xlabel={Seconds},
xmin=-0.6425, xmax=9.6425,
xtick style={color=black},
xtick={0,1,2,3,4,5,6,7,8,9},
xticklabels={4,8,12,16,20,24,28,32,36,40},
y grid style={darkgray176},
ylabel={Precision@k},
ymin=0, ymax=0.74025,
ytick style={color=black}
]
\draw[draw=none,fill=darkviolet1270255] (axis cs:-0.175,0) rectangle (axis cs:0.175,0.149);
\addlegendimage{ybar,ybar legend,draw=none,fill=darkviolet1270255}
\addlegendentry{P@All}

\draw[draw=none,fill=darkviolet1270255] (axis cs:0.825,0) rectangle (axis cs:1.175,0.305);
\draw[draw=none,fill=darkviolet1270255] (axis cs:1.825,0) rectangle (axis cs:2.175,0.464);
\draw[draw=none,fill=darkviolet1270255] (axis cs:2.825,0) rectangle (axis cs:3.175,0.569);
\draw[draw=none,fill=darkviolet1270255] (axis cs:3.825,0) rectangle (axis cs:4.175,0.631);
\draw[draw=none,fill=darkviolet1270255] (axis cs:4.825,0) rectangle (axis cs:5.175,0.672);
\draw[draw=none,fill=darkviolet1270255] (axis cs:5.825,0) rectangle (axis cs:6.175,0.682);
\draw[draw=none,fill=darkviolet1270255] (axis cs:6.825,0) rectangle (axis cs:7.175,0.697);
\draw[draw=none,fill=darkviolet1270255] (axis cs:7.825,0) rectangle (axis cs:8.175,0.7);
\draw[draw=none,fill=darkviolet1270255] (axis cs:8.825,0) rectangle (axis cs:9.175,0.705);
\draw[draw=none,fill=royalblue8762253] (axis cs:-0.175,0) rectangle (axis cs:0.175,0.072);
\addlegendimage{ybar,ybar legend,draw=none,fill=royalblue8762253}
\addlegendentry{P@10}

\draw[draw=none,fill=royalblue8762253] (axis cs:0.825,0) rectangle (axis cs:1.175,0.162);
\draw[draw=none,fill=royalblue8762253] (axis cs:1.825,0) rectangle (axis cs:2.175,0.231);
\draw[draw=none,fill=royalblue8762253] (axis cs:2.825,0) rectangle (axis cs:3.175,0.282);
\draw[draw=none,fill=royalblue8762253] (axis cs:3.825,0) rectangle (axis cs:4.175,0.297);
\draw[draw=none,fill=royalblue8762253] (axis cs:4.825,0) rectangle (axis cs:5.175,0.318);
\draw[draw=none,fill=royalblue8762253] (axis cs:5.825,0) rectangle (axis cs:6.175,0.321);
\draw[draw=none,fill=royalblue8762253] (axis cs:6.825,0) rectangle (axis cs:7.175,0.331);
\draw[draw=none,fill=royalblue8762253] (axis cs:7.825,0) rectangle (axis cs:8.175,0.333);
\draw[draw=none,fill=royalblue8762253] (axis cs:8.825,0) rectangle (axis cs:9.175,0.336);
\draw[draw=none,fill=dodgerblue45123246] (axis cs:-0.175,0) rectangle (axis cs:0.175,0.072);
\addlegendimage{ybar,ybar legend,draw=none,fill=dodgerblue45123246}
\addlegendentry{P@5}

\draw[draw=none,fill=dodgerblue45123246] (axis cs:0.825,0) rectangle (axis cs:1.175,0.154);
\draw[draw=none,fill=dodgerblue45123246] (axis cs:1.825,0) rectangle (axis cs:2.175,0.213);
\draw[draw=none,fill=dodgerblue45123246] (axis cs:2.825,0) rectangle (axis cs:3.175,0.264);
\draw[draw=none,fill=dodgerblue45123246] (axis cs:3.825,0) rectangle (axis cs:4.175,0.279);
\draw[draw=none,fill=dodgerblue45123246] (axis cs:4.825,0) rectangle (axis cs:5.175,0.295);
\draw[draw=none,fill=dodgerblue45123246] (axis cs:5.825,0) rectangle (axis cs:6.175,0.297);
\draw[draw=none,fill=dodgerblue45123246] (axis cs:6.825,0) rectangle (axis cs:7.175,0.308);
\draw[draw=none,fill=dodgerblue45123246] (axis cs:7.825,0) rectangle (axis cs:8.175,0.31);
\draw[draw=none,fill=dodgerblue45123246] (axis cs:8.825,0) rectangle (axis cs:9.175,0.313);
\draw[draw=none,fill=deepskyblue3176236] (axis cs:-0.175,0) rectangle (axis cs:0.175,0.054);
\addlegendimage{ybar,ybar legend,draw=none,fill=deepskyblue3176236}
\addlegendentry{P@1}

\draw[draw=none,fill=deepskyblue3176236] (axis cs:0.825,0) rectangle (axis cs:1.175,0.126);
\draw[draw=none,fill=deepskyblue3176236] (axis cs:1.825,0) rectangle (axis cs:2.175,0.174);
\draw[draw=none,fill=deepskyblue3176236] (axis cs:2.825,0) rectangle (axis cs:3.175,0.218);
\draw[draw=none,fill=deepskyblue3176236] (axis cs:3.825,0) rectangle (axis cs:4.175,0.233);
\draw[draw=none,fill=deepskyblue3176236] (axis cs:4.825,0) rectangle (axis cs:5.175,0.244);
\draw[draw=none,fill=deepskyblue3176236] (axis cs:5.825,0) rectangle (axis cs:6.175,0.246);
\draw[draw=none,fill=deepskyblue3176236] (axis cs:6.825,0) rectangle (axis cs:7.175,0.256);
\draw[draw=none,fill=deepskyblue3176236] (axis cs:7.825,0) rectangle (axis cs:8.175,0.259);
\draw[draw=none,fill=deepskyblue3176236] (axis cs:8.825,0) rectangle (axis cs:9.175,0.262);
\addplot [red, thick] coordinates {(-0.8925,0.15) (14.8925,0.15)};
\end{axis}

\end{tikzpicture}
}
\resizebox{.24\textwidth}{!}{% This file was created with tikzplotlib v0.10.1.
\begin{tikzpicture}

\definecolor{darkgray176}{RGB}{176,176,176}
\definecolor{darkviolet1270255}{RGB}{127,0,255}
\definecolor{deepskyblue3176236}{RGB}{3,176,236}
\definecolor{dodgerblue45123246}{RGB}{45,123,246}
\definecolor{royalblue8762253}{RGB}{87,62,253}

\begin{axis}[
tick align=outside,
tick pos=left,
title={\(\displaystyle \Delta=3\) Repair Precision},
legend style={fill opacity=0.8, draw opacity=1, text opacity=1, legend columns=1, legend pos=north west},
x grid style={darkgray176},
xlabel={Seconds},
xmin=-0.6425, xmax=9.6425,
xtick style={color=black},
xtick={0,1,2,3,4,5,6,7,8,9},
xticklabels={6,12,18,24,30,36,42,48,54,60},
y grid style={darkgray176},
ylabel={Precision@k},
ymin=0, ymax=0.5313,
ytick style={color=black}
]
\draw[draw=none,fill=darkviolet1270255] (axis cs:-0.175,0) rectangle (axis cs:0.175,0.013);
\addlegendimage{ybar,ybar legend,draw=none,fill=darkviolet1270255}
\addlegendentry{P@All}

\draw[draw=none,fill=darkviolet1270255] (axis cs:0.825,0) rectangle (axis cs:1.175,0.065);
\draw[draw=none,fill=darkviolet1270255] (axis cs:1.825,0) rectangle (axis cs:2.175,0.156);
\draw[draw=none,fill=darkviolet1270255] (axis cs:2.825,0) rectangle (axis cs:3.175,0.221);
\draw[draw=none,fill=darkviolet1270255] (axis cs:3.825,0) rectangle (axis cs:4.175,0.325);
\draw[draw=none,fill=darkviolet1270255] (axis cs:4.825,0) rectangle (axis cs:5.175,0.377);
\draw[draw=none,fill=darkviolet1270255] (axis cs:5.825,0) rectangle (axis cs:6.175,0.403);
\draw[draw=none,fill=darkviolet1270255] (axis cs:6.825,0) rectangle (axis cs:7.175,0.455);
\draw[draw=none,fill=darkviolet1270255] (axis cs:7.825,0) rectangle (axis cs:8.175,0.468);
\draw[draw=none,fill=darkviolet1270255] (axis cs:8.825,0) rectangle (axis cs:9.175,0.506);
\draw[draw=none,fill=royalblue8762253] (axis cs:-0.175,0) rectangle (axis cs:0.175,0);
\addlegendimage{ybar,ybar legend,draw=none,fill=royalblue8762253}
\addlegendentry{P@10}

\draw[draw=none,fill=royalblue8762253] (axis cs:0.825,0) rectangle (axis cs:1.175,0.013);
\draw[draw=none,fill=royalblue8762253] (axis cs:1.825,0) rectangle (axis cs:2.175,0.052);
\draw[draw=none,fill=royalblue8762253] (axis cs:2.825,0) rectangle (axis cs:3.175,0.065);
\draw[draw=none,fill=royalblue8762253] (axis cs:3.825,0) rectangle (axis cs:4.175,0.104);
\draw[draw=none,fill=royalblue8762253] (axis cs:4.825,0) rectangle (axis cs:5.175,0.117);
\draw[draw=none,fill=royalblue8762253] (axis cs:5.825,0) rectangle (axis cs:6.175,0.13);
\draw[draw=none,fill=royalblue8762253] (axis cs:6.825,0) rectangle (axis cs:7.175,0.143);
\draw[draw=none,fill=royalblue8762253] (axis cs:7.825,0) rectangle (axis cs:8.175,0.156);
\draw[draw=none,fill=royalblue8762253] (axis cs:8.825,0) rectangle (axis cs:9.175,0.156);
\draw[draw=none,fill=dodgerblue45123246] (axis cs:-0.175,0) rectangle (axis cs:0.175,0);
\addlegendimage{ybar,ybar legend,draw=none,fill=dodgerblue45123246}
\addlegendentry{P@5}

\draw[draw=none,fill=dodgerblue45123246] (axis cs:0.825,0) rectangle (axis cs:1.175,0.013);
\draw[draw=none,fill=dodgerblue45123246] (axis cs:1.825,0) rectangle (axis cs:2.175,0.052);
\draw[draw=none,fill=dodgerblue45123246] (axis cs:2.825,0) rectangle (axis cs:3.175,0.065);
\draw[draw=none,fill=dodgerblue45123246] (axis cs:3.825,0) rectangle (axis cs:4.175,0.104);
\draw[draw=none,fill=dodgerblue45123246] (axis cs:4.825,0) rectangle (axis cs:5.175,0.104);
\draw[draw=none,fill=dodgerblue45123246] (axis cs:5.825,0) rectangle (axis cs:6.175,0.117);
\draw[draw=none,fill=dodgerblue45123246] (axis cs:6.825,0) rectangle (axis cs:7.175,0.13);
\draw[draw=none,fill=dodgerblue45123246] (axis cs:7.825,0) rectangle (axis cs:8.175,0.143);
\draw[draw=none,fill=dodgerblue45123246] (axis cs:8.825,0) rectangle (axis cs:9.175,0.143);
\draw[draw=none,fill=deepskyblue3176236] (axis cs:-0.175,0) rectangle (axis cs:0.175,0);
\addlegendimage{ybar,ybar legend,draw=none,fill=deepskyblue3176236}
\addlegendentry{P@1}

\draw[draw=none,fill=deepskyblue3176236] (axis cs:0.825,0) rectangle (axis cs:1.175,0.013);
\draw[draw=none,fill=deepskyblue3176236] (axis cs:1.825,0) rectangle (axis cs:2.175,0.052);
\draw[draw=none,fill=deepskyblue3176236] (axis cs:2.825,0) rectangle (axis cs:3.175,0.065);
\draw[draw=none,fill=deepskyblue3176236] (axis cs:3.825,0) rectangle (axis cs:4.175,0.091);
\draw[draw=none,fill=deepskyblue3176236] (axis cs:4.825,0) rectangle (axis cs:5.175,0.091);
\draw[draw=none,fill=deepskyblue3176236] (axis cs:5.825,0) rectangle (axis cs:6.175,0.104);
\draw[draw=none,fill=deepskyblue3176236] (axis cs:6.825,0) rectangle (axis cs:7.175,0.117);
\draw[draw=none,fill=deepskyblue3176236] (axis cs:7.825,0) rectangle (axis cs:8.175,0.13);
\draw[draw=none,fill=deepskyblue3176236] (axis cs:8.825,0) rectangle (axis cs:9.175,0.13);
\addplot [red, thick] coordinates {(-0.8925,0.07) (14.8925,0.07)};
\end{axis}

\end{tikzpicture}
}
\resizebox{.24\textwidth}{!}{% This file was created with tikzplotlib v0.10.1.
\begin{tikzpicture}
\begin{axis}[
tick align=outside,
tick pos=left,
axis lines=left,
title={\textbf{\Large Repair Precision \(\mathbf{(\bm\Delta=4)}\)}},
legend style={fill opacity=0.8, draw opacity=1, text opacity=1, legend columns=1, legend pos=north west},
x grid style={darkgray176},
xlabel={Seconds},
xmin=-0.5925, xmax=8.5925,
xtick style={color=black},
xtick={0,1,2,3,4,5,6,7,8},
xticklabels={10,20,30,40,50,60,70,80,90},
y grid style={darkgray176},
ylabel={Precision@k},
ymin=0, ymax=0.54915,
ytick style={color=black}
]
\draw[draw=none,fill=darkviolet1270255] (axis cs:-0.175,0) rectangle (axis cs:0.175,0.123);

\draw[draw=none,fill=darkviolet1270255] (axis cs:0.825,0) rectangle (axis cs:1.175,0.323);
\draw[draw=none,fill=darkviolet1270255] (axis cs:1.825,0) rectangle (axis cs:2.175,0.415);
\draw[draw=none,fill=darkviolet1270255] (axis cs:2.825,0) rectangle (axis cs:3.175,0.431);
\draw[draw=none,fill=darkviolet1270255] (axis cs:3.825,0) rectangle (axis cs:4.175,0.477);
\draw[draw=none,fill=darkviolet1270255] (axis cs:4.825,0) rectangle (axis cs:5.175,0.508);
\draw[draw=none,fill=darkviolet1270255] (axis cs:5.825,0) rectangle (axis cs:6.175,0.523);
\draw[draw=none,fill=darkviolet1270255] (axis cs:6.825,0) rectangle (axis cs:7.175,0.523);
\draw[draw=none,fill=darkviolet1270255] (axis cs:7.825,0) rectangle (axis cs:8.175,0.523);
\draw[draw=none,fill=royalblue8762253] (axis cs:-0.175,0) rectangle (axis cs:0.175,0.092);

\draw[draw=none,fill=royalblue8762253] (axis cs:0.825,0) rectangle (axis cs:1.175,0.169);
\draw[draw=none,fill=royalblue8762253] (axis cs:1.825,0) rectangle (axis cs:2.175,0.215);
\draw[draw=none,fill=royalblue8762253] (axis cs:2.825,0) rectangle (axis cs:3.175,0.231);
\draw[draw=none,fill=royalblue8762253] (axis cs:3.825,0) rectangle (axis cs:4.175,0.277);
\draw[draw=none,fill=royalblue8762253] (axis cs:4.825,0) rectangle (axis cs:5.175,0.308);
\draw[draw=none,fill=royalblue8762253] (axis cs:5.825,0) rectangle (axis cs:6.175,0.323);
\draw[draw=none,fill=royalblue8762253] (axis cs:6.825,0) rectangle (axis cs:7.175,0.323);
\draw[draw=none,fill=royalblue8762253] (axis cs:7.825,0) rectangle (axis cs:8.175,0.323);
\draw[draw=none,fill=dodgerblue45123246] (axis cs:-0.175,0) rectangle (axis cs:0.175,0.092);

\draw[draw=none,fill=dodgerblue45123246] (axis cs:0.825,0) rectangle (axis cs:1.175,0.169);
\draw[draw=none,fill=dodgerblue45123246] (axis cs:1.825,0) rectangle (axis cs:2.175,0.215);
\draw[draw=none,fill=dodgerblue45123246] (axis cs:2.825,0) rectangle (axis cs:3.175,0.231);
\draw[draw=none,fill=dodgerblue45123246] (axis cs:3.825,0) rectangle (axis cs:4.175,0.277);
\draw[draw=none,fill=dodgerblue45123246] (axis cs:4.825,0) rectangle (axis cs:5.175,0.308);
\draw[draw=none,fill=dodgerblue45123246] (axis cs:5.825,0) rectangle (axis cs:6.175,0.323);
\draw[draw=none,fill=dodgerblue45123246] (axis cs:6.825,0) rectangle (axis cs:7.175,0.323);
\draw[draw=none,fill=dodgerblue45123246] (axis cs:7.825,0) rectangle (axis cs:8.175,0.323);
\draw[draw=none,fill=deepskyblue3176236] (axis cs:-0.175,0) rectangle (axis cs:0.175,0.077);

\draw[draw=none,fill=deepskyblue3176236] (axis cs:0.825,0) rectangle (axis cs:1.175,0.138);
\draw[draw=none,fill=deepskyblue3176236] (axis cs:1.825,0) rectangle (axis cs:2.175,0.185);
\draw[draw=none,fill=deepskyblue3176236] (axis cs:2.825,0) rectangle (axis cs:3.175,0.2);
\draw[draw=none,fill=deepskyblue3176236] (axis cs:3.825,0) rectangle (axis cs:4.175,0.246);
\draw[draw=none,fill=deepskyblue3176236] (axis cs:4.825,0) rectangle (axis cs:5.175,0.277);
\draw[draw=none,fill=deepskyblue3176236] (axis cs:5.825,0) rectangle (axis cs:6.175,0.292);
\draw[draw=none,fill=deepskyblue3176236] (axis cs:6.825,0) rectangle (axis cs:7.175,0.292);
\draw[draw=none,fill=deepskyblue3176236] (axis cs:7.825,0) rectangle (axis cs:8.175,0.292);
\addplot [red, thick] coordinates {(-0.8925,0.10) (14.8925,0.10)};
\addplot [orange, thick] coordinates {(-0.8925,0.01) (14.8925,0.01)};
\end{axis}

\end{tikzpicture}
}
%    \resizebox{.24\textwidth}{!}{% This file was created with tikzplotlib v0.10.1.
\begin{tikzpicture}
\begin{axis}[
tick align=outside,
tick pos=left,
axis lines=left,
title={\Large\(\displaystyle \Delta=5\) Repair Precision},
legend style={fill opacity=0.8, draw opacity=1, text opacity=1, legend columns=1, legend pos=north west},
x grid style={darkgray176},
xlabel={Seconds},
xmin=-0.5925, xmax=8.5925,
xtick={0,1,2,3,4,5,6,7,8},
xticklabels={10,20,30,40,50,60,70,80,90},
y grid style={darkgray176},
ylabel={Precision@k},
ymin=0, ymax=0.54915,
]
\draw[draw=none,fill=darkviolet1270255] (axis cs:-0.175,0) rectangle (axis cs:0.175,0.123);
\addlegendimage{ybar,ybar legend,draw=none,fill=darkviolet1270255}
\addlegendentry{P@All}

\draw[draw=none,fill=darkviolet1270255] (axis cs:0.825,0) rectangle (axis cs:1.175,0.323);
\draw[draw=none,fill=darkviolet1270255] (axis cs:1.825,0) rectangle (axis cs:2.175,0.415);
\draw[draw=none,fill=darkviolet1270255] (axis cs:2.825,0) rectangle (axis cs:3.175,0.431);
\draw[draw=none,fill=darkviolet1270255] (axis cs:3.825,0) rectangle (axis cs:4.175,0.477);
\draw[draw=none,fill=darkviolet1270255] (axis cs:4.825,0) rectangle (axis cs:5.175,0.508);
\draw[draw=none,fill=darkviolet1270255] (axis cs:5.825,0) rectangle (axis cs:6.175,0.523);
\draw[draw=none,fill=darkviolet1270255] (axis cs:6.825,0) rectangle (axis cs:7.175,0.523);
\draw[draw=none,fill=darkviolet1270255] (axis cs:7.825,0) rectangle (axis cs:8.175,0.523);
\draw[draw=none,fill=royalblue8762253] (axis cs:-0.175,0) rectangle (axis cs:0.175,0.092);
\addlegendimage{ybar,ybar legend,draw=none,fill=royalblue8762253}
\addlegendentry{P@10}

\draw[draw=none,fill=royalblue8762253] (axis cs:0.825,0) rectangle (axis cs:1.175,0.169);
\draw[draw=none,fill=royalblue8762253] (axis cs:1.825,0) rectangle (axis cs:2.175,0.215);
\draw[draw=none,fill=royalblue8762253] (axis cs:2.825,0) rectangle (axis cs:3.175,0.231);
\draw[draw=none,fill=royalblue8762253] (axis cs:3.825,0) rectangle (axis cs:4.175,0.277);
\draw[draw=none,fill=royalblue8762253] (axis cs:4.825,0) rectangle (axis cs:5.175,0.308);
\draw[draw=none,fill=royalblue8762253] (axis cs:5.825,0) rectangle (axis cs:6.175,0.323);
\draw[draw=none,fill=royalblue8762253] (axis cs:6.825,0) rectangle (axis cs:7.175,0.323);
\draw[draw=none,fill=royalblue8762253] (axis cs:7.825,0) rectangle (axis cs:8.175,0.323);
\draw[draw=none,fill=dodgerblue45123246] (axis cs:-0.175,0) rectangle (axis cs:0.175,0.092);
\addlegendimage{ybar,ybar legend,draw=none,fill=dodgerblue45123246}
\addlegendentry{P@5}

\draw[draw=none,fill=dodgerblue45123246] (axis cs:0.825,0) rectangle (axis cs:1.175,0.169);
\draw[draw=none,fill=dodgerblue45123246] (axis cs:1.825,0) rectangle (axis cs:2.175,0.215);
\draw[draw=none,fill=dodgerblue45123246] (axis cs:2.825,0) rectangle (axis cs:3.175,0.231);
\draw[draw=none,fill=dodgerblue45123246] (axis cs:3.825,0) rectangle (axis cs:4.175,0.277);
\draw[draw=none,fill=dodgerblue45123246] (axis cs:4.825,0) rectangle (axis cs:5.175,0.308);
\draw[draw=none,fill=dodgerblue45123246] (axis cs:5.825,0) rectangle (axis cs:6.175,0.323);
\draw[draw=none,fill=dodgerblue45123246] (axis cs:6.825,0) rectangle (axis cs:7.175,0.323);
\draw[draw=none,fill=dodgerblue45123246] (axis cs:7.825,0) rectangle (axis cs:8.175,0.323);
\draw[draw=none,fill=deepskyblue3176236] (axis cs:-0.175,0) rectangle (axis cs:0.175,0.077);
\addlegendimage{ybar,ybar legend,draw=none,fill=deepskyblue3176236}
\addlegendentry{P@1}

\draw[draw=none,fill=deepskyblue3176236] (axis cs:0.825,0) rectangle (axis cs:1.175,0.138);
\draw[draw=none,fill=deepskyblue3176236] (axis cs:1.825,0) rectangle (axis cs:2.175,0.185);
\draw[draw=none,fill=deepskyblue3176236] (axis cs:2.825,0) rectangle (axis cs:3.175,0.2);
\draw[draw=none,fill=deepskyblue3176236] (axis cs:3.825,0) rectangle (axis cs:4.175,0.246);
\draw[draw=none,fill=deepskyblue3176236] (axis cs:4.825,0) rectangle (axis cs:5.175,0.277);
\draw[draw=none,fill=deepskyblue3176236] (axis cs:5.825,0) rectangle (axis cs:6.175,0.292);
\draw[draw=none,fill=deepskyblue3176236] (axis cs:6.825,0) rectangle (axis cs:7.175,0.292);
\draw[draw=none,fill=deepskyblue3176236] (axis cs:7.825,0) rectangle (axis cs:8.175,0.292);
\addplot [red, thick] coordinates {(-0.8925,0.10) (14.8925,0.10)};
\end{axis}

\end{tikzpicture}
}
%\resizebox{.3\textwidth}{!}{% This file was created with tikzplotlib v0.10.1.
\begin{tikzpicture}

\definecolor{darkgray176}{RGB}{176,176,176}
\definecolor{darkviolet1270255}{RGB}{127,0,255}
\definecolor{deepskyblue3176236}{RGB}{3,176,236}
\definecolor{dodgerblue45123246}{RGB}{45,123,246}
\definecolor{lightgray204}{RGB}{204,204,204}
\definecolor{royalblue8762253}{RGB}{87,62,253}

\begin{axis}[
legend cell align={left},
legend style={fill opacity=0.8, draw opacity=1, text opacity=1, draw=lightgray204, legend columns=-1, legend pos=north west},
tick align=outside,
tick pos=left,
title={$\Delta=1$ Repair Precision},
x grid style={darkgray176},
xlabel={Seconds},
xmin=-0.4925, xmax=6.4925,
xtick style={color=black},
xtick={0,1,2,3,4,5,6},
xticklabels={20,60,100,140,180,220,260},
y grid style={darkgray176},
ylabel={\phantom{Precision@k}},
ymin=0, ymax=0.77595,
ytick style={color=black}
]
\addlegendimage{empty legend}
\addlegendentry{P@}
\draw[draw=none,fill=darkviolet1270255] (axis cs:-0.175,0) rectangle (axis cs:0.175,0.145);
\addlegendimage{ybar,ybar legend,draw=none,fill=darkviolet1270255}
\addlegendentry{All}

\draw[draw=none,fill=darkviolet1270255] (axis cs:0.825,0) rectangle (axis cs:1.175,0.304);
\draw[draw=none,fill=darkviolet1270255] (axis cs:1.825,0) rectangle (axis cs:2.175,0.42);
\draw[draw=none,fill=darkviolet1270255] (axis cs:2.825,0) rectangle (axis cs:3.175,0.507);
\draw[draw=none,fill=darkviolet1270255] (axis cs:3.825,0) rectangle (axis cs:4.175,0.594);
\draw[draw=none,fill=darkviolet1270255] (axis cs:4.825,0) rectangle (axis cs:5.175,0.71);
\draw[draw=none,fill=darkviolet1270255] (axis cs:5.825,0) rectangle (axis cs:6.175,0.739);
\draw[draw=none,fill=royalblue8762253] (axis cs:-0.175,0) rectangle (axis cs:0.175,0.145);
\addlegendimage{ybar,ybar legend,draw=none,fill=royalblue8762253}
\addlegendentry{10}

\draw[draw=none,fill=royalblue8762253] (axis cs:0.825,0) rectangle (axis cs:1.175,0.304);
\draw[draw=none,fill=royalblue8762253] (axis cs:1.825,0) rectangle (axis cs:2.175,0.42);
\draw[draw=none,fill=royalblue8762253] (axis cs:2.825,0) rectangle (axis cs:3.175,0.507);
\draw[draw=none,fill=royalblue8762253] (axis cs:3.825,0) rectangle (axis cs:4.175,0.594);
\draw[draw=none,fill=royalblue8762253] (axis cs:4.825,0) rectangle (axis cs:5.175,0.703);
\draw[draw=none,fill=royalblue8762253] (axis cs:5.825,0) rectangle (axis cs:6.175,0.732);
\draw[draw=none,fill=dodgerblue45123246] (axis cs:-0.175,0) rectangle (axis cs:0.175,0.145);
\addlegendimage{ybar,ybar legend,draw=none,fill=dodgerblue45123246}
\addlegendentry{5}

\draw[draw=none,fill=dodgerblue45123246] (axis cs:0.825,0) rectangle (axis cs:1.175,0.304);
\draw[draw=none,fill=dodgerblue45123246] (axis cs:1.825,0) rectangle (axis cs:2.175,0.42);
\draw[draw=none,fill=dodgerblue45123246] (axis cs:2.825,0) rectangle (axis cs:3.175,0.493);
\draw[draw=none,fill=dodgerblue45123246] (axis cs:3.825,0) rectangle (axis cs:4.175,0.565);
\draw[draw=none,fill=dodgerblue45123246] (axis cs:4.825,0) rectangle (axis cs:5.175,0.638);
\draw[draw=none,fill=dodgerblue45123246] (axis cs:5.825,0) rectangle (axis cs:6.175,0.667);
\draw[draw=none,fill=deepskyblue3176236] (axis cs:-0.175,0) rectangle (axis cs:0.175,0.116);
\addlegendimage{ybar,ybar legend,draw=none,fill=deepskyblue3176236}
\addlegendentry{1}

\draw[draw=none,fill=deepskyblue3176236] (axis cs:0.825,0) rectangle (axis cs:1.175,0.203);
\draw[draw=none,fill=deepskyblue3176236] (axis cs:1.825,0) rectangle (axis cs:2.175,0.268);
\draw[draw=none,fill=deepskyblue3176236] (axis cs:2.825,0) rectangle (axis cs:3.175,0.312);
\draw[draw=none,fill=deepskyblue3176236] (axis cs:3.825,0) rectangle (axis cs:4.175,0.355);
\draw[draw=none,fill=deepskyblue3176236] (axis cs:4.825,0) rectangle (axis cs:5.175,0.428);
\draw[draw=none,fill=deepskyblue3176236] (axis cs:5.825,0) rectangle (axis cs:6.175,0.442);
\end{axis}

\end{tikzpicture}
}
%\resizebox{.307\textwidth}{!}{% This file was created with tikzplotlib v0.10.1.
\begin{tikzpicture}

\definecolor{darkgray176}{RGB}{176,176,176}
\definecolor{darkviolet1270255}{RGB}{127,0,255}
\definecolor{deepskyblue3176236}{RGB}{3,176,236}
\definecolor{dodgerblue45123246}{RGB}{45,123,246}
\definecolor{lightgray204}{RGB}{204,204,204}
\definecolor{royalblue8762253}{RGB}{87,62,253}

\begin{axis}[
legend cell align={left},
legend style={fill opacity=0.8, draw opacity=1, text opacity=1, draw=lightgray204, legend columns=-1, legend pos=north west},
tick align=outside,
tick pos=left,
title={$\Delta=2$ Repair Precision},
x grid style={darkgray176},
xlabel={Seconds},
xmin=-0.4925, xmax=6.4925,
xtick style={color=black},
xtick={0,1,2,3,4,5,6},
xticklabels={20,60,100,140,180,220,260},
y grid style={darkgray176},
ylabel={\phantom{Precision@k}},
ymin=0, ymax=0.40635,
ytick style={color=black}
]
\addlegendimage{empty legend}
\addlegendentry{P@}
\draw[draw=none,fill=darkviolet1270255] (axis cs:-0.175,0) rectangle (axis cs:0.175,0.048);
\addlegendimage{ybar,ybar legend,draw=none,fill=darkviolet1270255}
\addlegendentry{All}

\draw[draw=none,fill=darkviolet1270255] (axis cs:0.825,0) rectangle (axis cs:1.175,0.081);
\draw[draw=none,fill=darkviolet1270255] (axis cs:1.825,0) rectangle (axis cs:2.175,0.177);
\draw[draw=none,fill=darkviolet1270255] (axis cs:2.825,0) rectangle (axis cs:3.175,0.274);
\draw[draw=none,fill=darkviolet1270255] (axis cs:3.825,0) rectangle (axis cs:4.175,0.306);
\draw[draw=none,fill=darkviolet1270255] (axis cs:4.825,0) rectangle (axis cs:5.175,0.355);
\draw[draw=none,fill=darkviolet1270255] (axis cs:5.825,0) rectangle (axis cs:6.175,0.387);
\draw[draw=none,fill=royalblue8762253] (axis cs:-0.175,0) rectangle (axis cs:0.175,0.048);
\addlegendimage{ybar,ybar legend,draw=none,fill=royalblue8762253}
\addlegendentry{10}

\draw[draw=none,fill=royalblue8762253] (axis cs:0.825,0) rectangle (axis cs:1.175,0.081);
\draw[draw=none,fill=royalblue8762253] (axis cs:1.825,0) rectangle (axis cs:2.175,0.113);
\draw[draw=none,fill=royalblue8762253] (axis cs:2.825,0) rectangle (axis cs:3.175,0.21);
\draw[draw=none,fill=royalblue8762253] (axis cs:3.825,0) rectangle (axis cs:4.175,0.226);
\draw[draw=none,fill=royalblue8762253] (axis cs:4.825,0) rectangle (axis cs:5.175,0.258);
\draw[draw=none,fill=royalblue8762253] (axis cs:5.825,0) rectangle (axis cs:6.175,0.242);
\draw[draw=none,fill=dodgerblue45123246] (axis cs:-0.175,0) rectangle (axis cs:0.175,0.048);
\addlegendimage{ybar,ybar legend,draw=none,fill=dodgerblue45123246}
\addlegendentry{5}

\draw[draw=none,fill=dodgerblue45123246] (axis cs:0.825,0) rectangle (axis cs:1.175,0.032);
\draw[draw=none,fill=dodgerblue45123246] (axis cs:1.825,0) rectangle (axis cs:2.175,0.065);
\draw[draw=none,fill=dodgerblue45123246] (axis cs:2.825,0) rectangle (axis cs:3.175,0.129);
\draw[draw=none,fill=dodgerblue45123246] (axis cs:3.825,0) rectangle (axis cs:4.175,0.113);
\draw[draw=none,fill=dodgerblue45123246] (axis cs:4.825,0) rectangle (axis cs:5.175,0.129);
\draw[draw=none,fill=dodgerblue45123246] (axis cs:5.825,0) rectangle (axis cs:6.175,0.113);
\draw[draw=none,fill=deepskyblue3176236] (axis cs:-0.175,0) rectangle (axis cs:0.175,0.016);
\addlegendimage{ybar,ybar legend,draw=none,fill=deepskyblue3176236}
\addlegendentry{1}

\draw[draw=none,fill=deepskyblue3176236] (axis cs:0.825,0) rectangle (axis cs:1.175,0);
\draw[draw=none,fill=deepskyblue3176236] (axis cs:1.825,0) rectangle (axis cs:2.175,0.016);
\draw[draw=none,fill=deepskyblue3176236] (axis cs:2.825,0) rectangle (axis cs:3.175,0.065);
\draw[draw=none,fill=deepskyblue3176236] (axis cs:3.825,0) rectangle (axis cs:4.175,0.048);
\draw[draw=none,fill=deepskyblue3176236] (axis cs:4.825,0) rectangle (axis cs:5.175,0.032);
\draw[draw=none,fill=deepskyblue3176236] (axis cs:5.825,0) rectangle (axis cs:6.175,0.032);
\end{axis}

\end{tikzpicture}
}
\caption{Human repair benchmarks. Note the y-axis across different edit distance plots has varying ranges. The red line indicates Seq2Parse and the orange line indicates BIFI's Precision@1 on the same repairs.}\label{fig:human}
\end{figure}

\begin{wrapfigure}{r}{0.38\textwidth}
\vspace{-1cm}
\hspace{-0.4cm}
\resizebox{.44\textwidth}{!}{% This file was created with matplot2tikz v0.3.3.
\begin{tikzpicture}

\definecolor{darkgray176}{RGB}{176,176,176}
\definecolor{teal9147104}{RGB}{9,147,104}

\begin{axis}[
hide x axis,
hide y axis,
tick align=outside,
tick pos=left,
x grid style={darkgray176},
xmin=-0.65, xmax=2.10714285714286,
xtick style={color=black},
y grid style={darkgray176},
ymin=-1.22685803692431, ymax=1.22218305326489,
ytick style={color=black}
]
\path [draw=teal9147104, fill=teal9147104]
(axis cs:-0.25,0.315857142857143)
--(axis cs:-0.125,0.315857142857143)
.. controls (axis cs:-0.03125,0.315857142857143) and (axis cs:0.03125,0.315857142857143) .. (axis cs:0.125,0.315857142857143)
--(axis cs:0.25,0.315857142857143)
--(axis cs:0.4,0.315857142857143)
.. controls (axis cs:0.4265114773,0.315857142857143) and (axis cs:0.4519642327,0.326400019357143) .. (axis cs:0.4707106781,0.345146464757143)
.. controls (axis cs:0.4894571235,0.363892910157143) and (axis cs:0.5,0.389345665557143) .. (axis cs:0.5,0.415857142857143)
--(axis cs:0.5,0.565857142857143)
--(axis cs:0.47,0.565857142857143)
--(axis cs:0.596714285714286,0.672183053264892)
--(axis cs:0.723428571428572,0.565857142857143)
--(axis cs:0.693428571428572,0.565857142857143)
--(axis cs:0.693428571428572,0.415857142857143)
.. controls (axis cs:0.693428571428572,0.338064893751143) and (axis cs:0.662492759527143,0.263379237191714) .. (axis cs:0.607485332596286,0.208371810260857)
.. controls (axis cs:0.552477905665429,0.15336438333) and (axis cs:0.477792249106,0.122428571428571) .. (axis cs:0.4,0.122428571428571)
--(axis cs:0.693428571428572,0.122428571428571)
--(axis cs:0.843428571428572,0.122428571428571)
.. controls (axis cs:0.869940048728572,0.122428571428571) and (axis cs:0.895392804128571,0.132971447928571) .. (axis cs:0.914139249528572,0.151717893328571)
.. controls (axis cs:0.932885694928572,0.170464338728571) and (axis cs:0.943428571428572,0.195917094128571) .. (axis cs:0.943428571428572,0.222428571428571)
--(axis cs:0.943428571428572,0.565857142857143)
--(axis cs:0.913428571428571,0.565857142857143)
--(axis cs:0.992571428571429,0.63226588509603)
--(axis cs:1.07171428571429,0.565857142857143)
--(axis cs:1.04171428571429,0.565857142857143)
--(axis cs:1.04171428571429,0.222428571428571)
.. controls (axis cs:1.04171428571429,0.169860099296571) and (axis cs:1.02080926774,0.119390921446286) .. (axis cs:0.983637744575429,0.0822193982817143)
.. controls (axis cs:0.946466221410857,0.0450478751171429) and (axis cs:0.895997043560572,0.0241428571428572) .. (axis cs:0.843428571428572,0.0241428571428572)
--(axis cs:1.04171428571429,0.0241428571428571)
--(axis cs:1.19171428571429,0.0241428571428571)
.. controls (axis cs:1.21822576301429,0.0241428571428571) and (axis cs:1.24367851841429,0.0346857336428571) .. (axis cs:1.26242496381429,0.0534321790428571)
.. controls (axis cs:1.28117140921429,0.0721786244428572) and (axis cs:1.29171428571429,0.0976313798428571) .. (axis cs:1.29171428571429,0.124142857142857)
--(axis cs:1.29171428571429,0.565857142857143)
--(axis cs:1.26171428571429,0.565857142857143)
--(axis cs:1.30557142857143,0.602657655253061)
--(axis cs:1.34942857142857,0.565857142857143)
--(axis cs:1.31942857142857,0.565857142857143)
--(axis cs:1.31942857142857,0.124142857142857)
.. controls (axis cs:1.31942857142857,0.0902839132768571) and (axis cs:1.30596381201286,0.0577771085231429) .. (axis cs:1.28202192317343,0.0338352196837143)
.. controls (axis cs:1.258080034334,0.00989333084428572) and (axis cs:1.22557322958029,-0.00357142857142857) .. (axis cs:1.19171428571429,-0.00357142857142857)
--(axis cs:1.31942857142857,-0.00357142857142857)
--(axis cs:1.46942857142857,-0.00357142857142857)
.. controls (axis cs:1.49594004872857,-0.00357142857142857) and (axis cs:1.52139280412857,0.00697144792857142) .. (axis cs:1.54013924952857,0.0257178933285714)
.. controls (axis cs:1.55888569492857,0.0444643387285714) and (axis cs:1.56942857142857,0.0699170941285714) .. (axis cs:1.56942857142857,0.0964285714285714)
--(axis cs:1.56942857142857,0.565857142857143)
--(axis cs:1.53942857142857,0.565857142857143)
--(axis cs:1.62328571428571,0.636221640500152)
--(axis cs:1.70714285714286,0.565857142857143)
--(axis cs:1.67714285714286,0.565857142857143)
--(axis cs:1.67714285714286,0.0964285714285714)
.. controls (axis cs:1.67714285714286,0.0413604457225714) and (axis cs:1.65524379652714,-0.0115085633511429) .. (axis cs:1.61630475136771,-0.0504476085105714)
.. controls (axis cs:1.57736570620829,-0.08938665367) and (axis cs:1.52449669713457,-0.111285714285714) .. (axis cs:1.46942857142857,-0.111285714285714)
--(axis cs:0.4,-0.111285714285714)
.. controls (axis cs:0.480746385148,-0.111285714285714) and (axis cs:0.558268205880571,-0.143396303854286) .. (axis cs:0.615364522441714,-0.200492620415429)
.. controls (axis cs:0.672460839002857,-0.257588936976571) and (axis cs:0.704571428571429,-0.335110757709143) .. (axis cs:0.704571428571429,-0.415857142857143)
--(axis cs:0.704571428571429,-0.565857142857143)
--(axis cs:0.734571428571429,-0.565857142857143)
--(axis cs:0.602285714285714,-0.676858036924309)
--(axis cs:0.47,-0.565857142857143)
--(axis cs:0.5,-0.565857142857143)
--(axis cs:0.5,-0.415857142857143)
.. controls (axis cs:0.5,-0.389345665557143) and (axis cs:0.4894571235,-0.363892910157143) .. (axis cs:0.4707106781,-0.345146464757143)
.. controls (axis cs:0.4519642327,-0.326400019357143) and (axis cs:0.4265114773,-0.315857142857143) .. (axis cs:0.4,-0.315857142857143)
--(axis cs:0.25,-0.315857142857143)
--(axis cs:0.25,-0.315857142857143)
--(axis cs:0.125,-0.315857142857143)
.. controls (axis cs:0.03125,-0.315857142857143) and (axis cs:-0.03125,-0.315857142857143) .. (axis cs:-0.125,-0.315857142857143)
--(axis cs:-0.25,-0.315857142857143)
--(axis cs:-0.25,-0.315857142857143)
--(axis cs:0.015035612076138,0)
--(axis cs:-0.25,0.315857142857143)
--(axis cs:0,0.315857142857143)
--(axis cs:-0.25,0.315857142857143)
--cycle;
\draw (axis cs:-0.134964387923862,0) node[
  scale=0.5,
  text=black,
  rotate=0.0,
  align=center
]{Total\\
2211};
\draw (axis cs:0.596714285714286,0.822183053264892) node[
  scale=0.5,
  text=black,
  rotate=0.0,
  align=center
]{Top-1\\677};
\draw (axis cs:0.992571428571429,0.78226588509603) node[
  scale=0.5,
  text=black,
  rotate=0.0,
  align=center
]{[2-10]\\344};
\draw (axis cs:1.30557142857143,0.752657655253061) node[
  scale=0.5,
  text=black,
  rotate=0.0,
  align=center
]{[11-99]\\97};
\draw (axis cs:1.62328571428571,0.786221640500152) node[
  scale=0.5,
  text=black,
  rotate=0.0,
  align=center
]{Top-100+\\377};
\draw (axis cs:0.602285714285714,-0.826858036924309) node[
  scale=0.5,
  text=black,
  rotate=0.0,
  align=center
]{NR\\716};
\end{axis}

\end{tikzpicture}
}
\vspace{-1.1cm}
\caption{Outcomes in the repair pipeline.}
\label{fig:sankey}
\end{wrapfigure}

%We believe that rewriting the sampler in CUDA or using a more informed prior could significantly improve the latency-precision tradeoff.

We present a Sankey diagram of our repair pipeline in Fig.~\ref{fig:sankey}. We drew 2247 total repairs from the StackOverflow dataset balanced evenly across lengths and edit distances ($\lfloor|\err\sigma| / 10\rfloor \in [0, 8], \Delta(\err\sigma, \sigma') < 4$) with a timeout of 30s and tracked individual outcomes. In 101 cases, the intersection grammar was too large to construct and threw an out-of-memory (OOM) error, in 45 cases the human repair was not recognized, in 253 cases the sampler timed out before drawing the human repair, in 1226 cases the human repair was drawn but not ranked first, and in the remaining 622 cases the first prediction matched the human repair.

\clearpage The remaining experiments in this section were run on a 10-core ARM64 M1 with 16 GB of memory. We balance the StackOverflow dataset across Levenshtein distances, then measure the number of samples required to draw the exact human repair across varying Levenshtein radii. This tells us of how many samples are required on average to saturate the admissible set.

\begin{figure}[h!]
% This file was created with tikzplotlib v0.10.1.
\begin{tikzpicture}[scale=0.75]
\begin{axis}[
legend cell align={left},
legend style={fill opacity=0.8, draw opacity=1, text opacity=1, draw=lightgray204, legend columns=-1, legend pos=north west},
width=\textwidth,
height=.4\textwidth,
log basis x={10},
tick align=outside,
tick pos=left,
axis lines*=left,
title={\textbf{Uniform Sampling Efficiency}},
x grid style={darkgray176},
xlabel={Samples drawn (log scale)},
xmin=0.34892211545542, xmax=1001,
xmode=log,
%xtick style={color=black},
xtick={0.01,0.1,1,10,100,1000,10000},
xticklabels={
  \(\displaystyle 10^-2\),
  \(\displaystyle 10^-1\),
  \(\displaystyle 10^2\),
  \(\displaystyle 10^3\),
  \(\displaystyle 10^4\),
  \(\displaystyle 10^5\),
  \(\displaystyle 10^6\)
},
y grid style={darkgray176},
ylabel={Precision@All},
ymin=0, ymax=101,
%ytick style={color=black}
ytick={0,20,40,60,80,100}
]
\draw[draw=none,fill=green!80] (axis cs:-0.5,0) rectangle (axis cs:0.9,100);
\addlegendimage{empty legend}
\addlegendentry{$\Delta(\err\sigma, \sigma')=$}
\addlegendimage{ybar,ybar legend,draw=none,fill=green!80}
\addlegendentry{\footnotesize{$1,$}}
\addlegendimage{ybar,ybar legend,draw=none,fill=blue!80}
\addlegendentry{\footnotesize{$2,$}}
\addlegendimage{ybar,ybar legend,draw=none,fill=orange!80}
\addlegendentry{\footnotesize{$3$}}

\draw[draw=none,fill=green!80] (axis cs:0.1,0) rectangle (axis cs:1.9,100);
\draw[draw=none,fill=green!80] (axis cs:1.1,0) rectangle (axis cs:2.9,100);
\draw[draw=none,fill=green!80] (axis cs:2.1,0) rectangle (axis cs:3.9,100);
\draw[draw=none,fill=green!80] (axis cs:3.1,0) rectangle (axis cs:4.9,100);
\draw[draw=none,fill=green!80] (axis cs:4.1,0) rectangle (axis cs:5.9,100);
\draw[draw=none,fill=green!80] (axis cs:5.1,0) rectangle (axis cs:6.9,100);
\draw[draw=none,fill=green!80] (axis cs:6.1,0) rectangle (axis cs:7.9,100);
\draw[draw=none,fill=green!80] (axis cs:7.1,0) rectangle (axis cs:8.9,100);
\draw[draw=none,fill=green!80] (axis cs:8.1,0) rectangle (axis cs:9.9,100);
\draw[draw=none,fill=green!80] (axis cs:9.1,0) rectangle (axis cs:10.9,100);
\draw[draw=none,fill=green!80] (axis cs:10.1,0) rectangle (axis cs:11.9,100);
\draw[draw=none,fill=green!80] (axis cs:11.1,0) rectangle (axis cs:12.9,100);
\draw[draw=none,fill=green!80] (axis cs:12.1,0) rectangle (axis cs:13.9,100);
\draw[draw=none,fill=green!80] (axis cs:13.1,0) rectangle (axis cs:14.9,100);
\draw[draw=none,fill=green!80] (axis cs:14.1,0) rectangle (axis cs:15.9,100);
\draw[draw=none,fill=green!80] (axis cs:15.1,0) rectangle (axis cs:16.9,100);
\draw[draw=none,fill=green!80] (axis cs:16.1,0) rectangle (axis cs:17.9,100);
\draw[draw=none,fill=green!80] (axis cs:17.1,0) rectangle (axis cs:18.9,100);
\draw[draw=none,fill=green!80] (axis cs:18.1,0) rectangle (axis cs:19.9,100);
\draw[draw=none,fill=green!80] (axis cs:19.1,0) rectangle (axis cs:20.9,100);
\draw[draw=none,fill=green!80] (axis cs:20.1,0) rectangle (axis cs:21.9,100);
\draw[draw=none,fill=green!80] (axis cs:21.1,0) rectangle (axis cs:22.9,100);
\draw[draw=none,fill=green!80] (axis cs:22.1,0) rectangle (axis cs:23.9,100);
\draw[draw=none,fill=green!80] (axis cs:23.1,0) rectangle (axis cs:24.9,100);
\draw[draw=none,fill=green!80] (axis cs:24.1,0) rectangle (axis cs:25.9,100);
\draw[draw=none,fill=green!80] (axis cs:25.1,0) rectangle (axis cs:26.9,100);
\draw[draw=none,fill=green!80] (axis cs:26.1,0) rectangle (axis cs:27.9,100);
\draw[draw=none,fill=green!80] (axis cs:27.1,0) rectangle (axis cs:28.9,100);
\draw[draw=none,fill=green!80] (axis cs:28.1,0) rectangle (axis cs:29.9,100);
\draw[draw=none,fill=green!80] (axis cs:29.1,0) rectangle (axis cs:30.9,100);
\draw[draw=none,fill=green!80] (axis cs:30.1,0) rectangle (axis cs:31.9,100);
\draw[draw=none,fill=green!80] (axis cs:31.1,0) rectangle (axis cs:32.9,100);
\draw[draw=none,fill=green!80] (axis cs:32.1,0) rectangle (axis cs:33.9,100);
\draw[draw=none,fill=green!80] (axis cs:33.1,0) rectangle (axis cs:34.9,100);
\draw[draw=none,fill=green!80] (axis cs:34.1,0) rectangle (axis cs:35.9,100);
\draw[draw=none,fill=green!80] (axis cs:35.1,0) rectangle (axis cs:36.9,100);
\draw[draw=none,fill=green!80] (axis cs:36.1,0) rectangle (axis cs:37.9,100);
\draw[draw=none,fill=green!80] (axis cs:37.1,0) rectangle (axis cs:38.9,100);
\draw[draw=none,fill=green!80] (axis cs:38.1,0) rectangle (axis cs:39.9,100);
\draw[draw=none,fill=green!80] (axis cs:39.1,0) rectangle (axis cs:40.9,100);
\draw[draw=none,fill=green!80] (axis cs:40.1,0) rectangle (axis cs:41.9,100);
\draw[draw=none,fill=green!80] (axis cs:41.1,0) rectangle (axis cs:42.9,100);
\draw[draw=none,fill=green!80] (axis cs:42.1,0) rectangle (axis cs:43.9,100);
\draw[draw=none,fill=green!80] (axis cs:43.1,0) rectangle (axis cs:44.9,100);
\draw[draw=none,fill=green!80] (axis cs:44.1,0) rectangle (axis cs:45.9,100);
\draw[draw=none,fill=green!80] (axis cs:45.1,0) rectangle (axis cs:46.9,100);
\draw[draw=none,fill=green!80] (axis cs:46.1,0) rectangle (axis cs:47.9,100);
\draw[draw=none,fill=green!80] (axis cs:47.1,0) rectangle (axis cs:48.9,100);
\draw[draw=none,fill=green!80] (axis cs:48.1,0) rectangle (axis cs:49.9,100);
\draw[draw=none,fill=green!80] (axis cs:49.1,0) rectangle (axis cs:50.9,100);
\draw[draw=none,fill=green!80] (axis cs:50.1,0) rectangle (axis cs:51.9,100);
\draw[draw=none,fill=green!80] (axis cs:51.1,0) rectangle (axis cs:52.9,100);
\draw[draw=none,fill=green!80] (axis cs:52.1,0) rectangle (axis cs:53.9,100);
\draw[draw=none,fill=green!80] (axis cs:53.1,0) rectangle (axis cs:54.9,100);
\draw[draw=none,fill=green!80] (axis cs:54.1,0) rectangle (axis cs:55.9,100);
\draw[draw=none,fill=green!80] (axis cs:55.1,0) rectangle (axis cs:56.9,100);
\draw[draw=none,fill=green!80] (axis cs:56.1,0) rectangle (axis cs:57.9,100);
\draw[draw=none,fill=green!80] (axis cs:57.1,0) rectangle (axis cs:58.9,100);
\draw[draw=none,fill=green!80] (axis cs:58.1,0) rectangle (axis cs:59.9,100);
\draw[draw=none,fill=green!80] (axis cs:59.1,0) rectangle (axis cs:60.9,100);
\draw[draw=none,fill=green!80] (axis cs:60.1,0) rectangle (axis cs:61.9,100);
\draw[draw=none,fill=green!80] (axis cs:61.1,0) rectangle (axis cs:62.9,100);
\draw[draw=none,fill=green!80] (axis cs:62.1,0) rectangle (axis cs:63.9,100);
\draw[draw=none,fill=green!80] (axis cs:63.1,0) rectangle (axis cs:64.9,100);
\draw[draw=none,fill=green!80] (axis cs:64.1,0) rectangle (axis cs:65.9,100);
\draw[draw=none,fill=green!80] (axis cs:65.1,0) rectangle (axis cs:66.9,100);
\draw[draw=none,fill=green!80] (axis cs:66.1,0) rectangle (axis cs:67.9,100);
\draw[draw=none,fill=green!80] (axis cs:67.1,0) rectangle (axis cs:68.9,100);
\draw[draw=none,fill=green!80] (axis cs:68.1,0) rectangle (axis cs:69.9,100);
\draw[draw=none,fill=green!80] (axis cs:69.1,0) rectangle (axis cs:70.9,100);
\draw[draw=none,fill=green!80] (axis cs:70.1,0) rectangle (axis cs:71.9,100);
\draw[draw=none,fill=green!80] (axis cs:71.1,0) rectangle (axis cs:72.9,100);
\draw[draw=none,fill=green!80] (axis cs:72.1,0) rectangle (axis cs:73.9,100);
\draw[draw=none,fill=green!80] (axis cs:73.1,0) rectangle (axis cs:74.9,100);
\draw[draw=none,fill=green!80] (axis cs:74.1,0) rectangle (axis cs:75.9,100);
\draw[draw=none,fill=green!80] (axis cs:75.1,0) rectangle (axis cs:76.9,100);
\draw[draw=none,fill=green!80] (axis cs:76.1,0) rectangle (axis cs:77.9,100);
\draw[draw=none,fill=green!80] (axis cs:77.1,0) rectangle (axis cs:78.9,100);
\draw[draw=none,fill=green!80] (axis cs:78.1,0) rectangle (axis cs:79.9,100);
\draw[draw=none,fill=green!80] (axis cs:79.1,0) rectangle (axis cs:80.9,100);
\draw[draw=none,fill=green!80] (axis cs:80.1,0) rectangle (axis cs:81.9,100);
\draw[draw=none,fill=green!80] (axis cs:81.1,0) rectangle (axis cs:82.9,100);
\draw[draw=none,fill=green!80] (axis cs:82.1,0) rectangle (axis cs:83.9,100);

\draw[draw=none,fill=blue!80] (axis cs:0.1,0) rectangle (axis cs:1.9,19.7452229299363);
\draw[draw=none,fill=blue!80] (axis cs:1.1,0) rectangle (axis cs:2.9,26.7515923566879);
\draw[draw=none,fill=blue!80] (axis cs:2.1,0) rectangle (axis cs:3.9,31.5286624203822);
\draw[draw=none,fill=blue!80] (axis cs:3.1,0) rectangle (axis cs:4.9,35.3503184713376);
\draw[draw=none,fill=blue!80] (axis cs:4.1,0) rectangle (axis cs:5.9,38.2165605095541);
\draw[draw=none,fill=blue!80] (axis cs:5.1,0) rectangle (axis cs:6.9,40.4458598726115);
\draw[draw=none,fill=blue!80] (axis cs:6.1,0) rectangle (axis cs:7.9,44.9044585987261);
\draw[draw=none,fill=blue!80] (axis cs:7.1,0) rectangle (axis cs:8.9,48.0891719745223);
\draw[draw=none,fill=blue!80] (axis cs:8.1,0) rectangle (axis cs:9.9,51.5923566878981);
\draw[draw=none,fill=blue!80] (axis cs:9.1,0) rectangle (axis cs:10.9,54.7770700636943);
\draw[draw=none,fill=blue!80] (axis cs:10.1,0) rectangle (axis cs:11.9,57.6433121019108);
\draw[draw=none,fill=blue!80] (axis cs:11.1,0) rectangle (axis cs:12.9,58.5987261146497);
\draw[draw=none,fill=blue!80] (axis cs:12.1,2) rectangle (axis cs:13.9,61.1464968152866);
\draw[draw=none,fill=blue!80] (axis cs:13.1,2) rectangle (axis cs:14.9,63.3757961783439);
\draw[draw=none,fill=blue!80] (axis cs:14.1,4) rectangle (axis cs:15.9,66.8789808917197);
\draw[draw=none,fill=blue!80] (axis cs:15.1,4) rectangle (axis cs:16.9,69.7452229299363);
\draw[draw=none,fill=blue!80] (axis cs:16.1,6) rectangle (axis cs:17.9,70.3821656050955);
\draw[draw=none,fill=blue!80] (axis cs:17.1,6) rectangle (axis cs:18.9,72.9299363057325);
\draw[draw=none,fill=blue!80] (axis cs:18.1,6) rectangle (axis cs:19.9,75.1592356687898);
\draw[draw=none,fill=blue!80] (axis cs:19.1,6) rectangle (axis cs:20.9,76.1146496815287);
\draw[draw=none,fill=blue!80] (axis cs:20.1,8) rectangle (axis cs:21.9,78.343949044586);
\draw[draw=none,fill=blue!80] (axis cs:21.1,10) rectangle (axis cs:22.9,79.2993630573248);
\draw[draw=none,fill=blue!80] (axis cs:22.1,10) rectangle (axis cs:23.9,80.2547770700637);
\draw[draw=none,fill=blue!80] (axis cs:23.1,12) rectangle (axis cs:24.9,80.5732484076433);
\draw[draw=none,fill=blue!80] (axis cs:24.1,12) rectangle (axis cs:25.9,82.1656050955414);
\draw[draw=none,fill=blue!80] (axis cs:25.1,12) rectangle (axis cs:26.9,82.484076433121);
\draw[draw=none,fill=blue!80] (axis cs:26.1,12) rectangle (axis cs:27.9,82.8025477707006);
\draw[draw=none,fill=blue!80] (axis cs:27.1,14) rectangle (axis cs:28.9,84.0764331210191);
\draw[draw=none,fill=blue!80] (axis cs:28.1,14) rectangle (axis cs:29.9,85.6687898089172);
\draw[draw=none,fill=blue!80] (axis cs:29.1,14) rectangle (axis cs:30.9,86.3057324840764);
\draw[draw=none,fill=blue!80] (axis cs:30.1,16) rectangle (axis cs:31.9,86.3057324840764);
\draw[draw=none,fill=blue!80] (axis cs:31.1,16) rectangle (axis cs:32.9,87.5796178343949);
\draw[draw=none,fill=blue!80] (axis cs:32.1,16) rectangle (axis cs:33.9,89.171974522293);
\draw[draw=none,fill=blue!80] (axis cs:33.1,16) rectangle (axis cs:34.9,90.1273885350319);
\draw[draw=none,fill=blue!80] (axis cs:34.1,18) rectangle (axis cs:35.9,90.4458598726115);
\draw[draw=none,fill=blue!80] (axis cs:35.1,18) rectangle (axis cs:36.9,90.7643312101911);
\draw[draw=none,fill=blue!80] (axis cs:36.1,18) rectangle (axis cs:37.9,91.7197452229299);
\draw[draw=none,fill=blue!80] (axis cs:37.1,18) rectangle (axis cs:38.9,91.7197452229299);
\draw[draw=none,fill=blue!80] (axis cs:38.1,18) rectangle (axis cs:39.9,92.0382165605096);
\draw[draw=none,fill=blue!80] (axis cs:39.1,18) rectangle (axis cs:40.9,92.9936305732484);
\draw[draw=none,fill=blue!80] (axis cs:40.1,18) rectangle (axis cs:41.9,93.312101910828);
\draw[draw=none,fill=blue!80] (axis cs:41.1,18) rectangle (axis cs:42.9,93.312101910828);
\draw[draw=none,fill=blue!80] (axis cs:42.1,18) rectangle (axis cs:43.9,93.312101910828);
\draw[draw=none,fill=blue!80] (axis cs:43.1,20) rectangle (axis cs:44.9,93.6305732484076);
\draw[draw=none,fill=blue!80] (axis cs:44.1,20) rectangle (axis cs:45.9,95.2229299363057);
\draw[draw=none,fill=blue!80] (axis cs:45.1,20) rectangle (axis cs:46.9,95.2229299363057);
\draw[draw=none,fill=blue!80] (axis cs:46.1,22) rectangle (axis cs:47.9,95.2229299363057);
\draw[draw=none,fill=blue!80] (axis cs:47.1,22) rectangle (axis cs:48.9,95.2229299363057);
\draw[draw=none,fill=blue!80] (axis cs:48.1,22) rectangle (axis cs:49.9,95.2229299363057);
\draw[draw=none,fill=blue!80] (axis cs:49.1,22) rectangle (axis cs:50.9,95.5414012738854);
\draw[draw=none,fill=blue!80] (axis cs:50.1,22) rectangle (axis cs:51.9,95.5414012738854);
\draw[draw=none,fill=blue!80] (axis cs:51.1,24) rectangle (axis cs:52.9,96.1783439490446);
\draw[draw=none,fill=blue!80] (axis cs:52.1,26) rectangle (axis cs:53.9,96.1783439490446);
\draw[draw=none,fill=blue!80] (axis cs:53.1,26) rectangle (axis cs:54.9,96.1783439490446);
\draw[draw=none,fill=blue!80] (axis cs:54.1,26) rectangle (axis cs:55.9,96.1783439490446);
\draw[draw=none,fill=blue!80] (axis cs:55.1,26) rectangle (axis cs:56.9,96.1783439490446);
\draw[draw=none,fill=blue!80] (axis cs:56.1,26) rectangle (axis cs:57.9,96.8152866242038);
\draw[draw=none,fill=blue!80] (axis cs:57.1,26) rectangle (axis cs:58.9,97.1337579617834);
\draw[draw=none,fill=blue!80] (axis cs:58.1,26) rectangle (axis cs:59.9,97.1337579617834);
\draw[draw=none,fill=blue!80] (axis cs:59.1,26) rectangle (axis cs:60.9,97.1337579617834);
\draw[draw=none,fill=blue!80] (axis cs:60.1,26) rectangle (axis cs:61.9,97.1337579617834);
\draw[draw=none,fill=blue!80] (axis cs:61.1,26) rectangle (axis cs:62.9,97.4522292993631);
\draw[draw=none,fill=blue!80] (axis cs:62.1,26) rectangle (axis cs:63.9,98.0891719745223);
\draw[draw=none,fill=blue!80] (axis cs:63.1,26) rectangle (axis cs:64.9,98.0891719745223);
\draw[draw=none,fill=blue!80] (axis cs:64.1,26) rectangle (axis cs:65.9,98.4076433121019);
\draw[draw=none,fill=blue!80] (axis cs:65.1,26) rectangle (axis cs:66.9,98.7261146496815);
\draw[draw=none,fill=blue!80] (axis cs:66.1,26) rectangle (axis cs:67.9,98.7261146496815);
\draw[draw=none,fill=blue!80] (axis cs:67.1,26) rectangle (axis cs:68.9,98.7261146496815);
\draw[draw=none,fill=blue!80] (axis cs:68.1,26) rectangle (axis cs:69.9,99.0445859872611);
\draw[draw=none,fill=blue!80] (axis cs:69.1,26) rectangle (axis cs:70.9,99.0445859872611);
\draw[draw=none,fill=blue!80] (axis cs:70.1,26) rectangle (axis cs:71.9,99.0445859872611);
\draw[draw=none,fill=blue!80] (axis cs:71.1,26) rectangle (axis cs:72.9,99.0445859872611);
\draw[draw=none,fill=blue!80] (axis cs:72.1,26) rectangle (axis cs:73.9,99.3630573248408);
\draw[draw=none,fill=blue!80] (axis cs:73.1,26) rectangle (axis cs:74.9,99.3630573248408);
\draw[draw=none,fill=blue!80] (axis cs:74.1,26) rectangle (axis cs:75.9,99.3630573248408);
\draw[draw=none,fill=blue!80] (axis cs:75.1,28) rectangle (axis cs:76.9,99.3630573248408);
\draw[draw=none,fill=blue!80] (axis cs:76.1,28) rectangle (axis cs:77.9,99.3630573248408);
\draw[draw=none,fill=blue!80] (axis cs:77.1,28) rectangle (axis cs:78.9,99.6815286624204);
\draw[draw=none,fill=blue!80] (axis cs:78.1,30) rectangle (axis cs:786.9,100);

\draw[draw=none,fill=orange!80] (axis cs:12.0,0) rectangle (axis cs:14.99,2);
\draw[draw=none,fill=orange!80] (axis cs:14.0,0) rectangle (axis cs:15.99,4);
\draw[draw=none,fill=orange!80] (axis cs:15.0,0) rectangle (axis cs:16.99,4);
\draw[draw=none,fill=orange!80] (axis cs:16.0,0) rectangle (axis cs:17.99,6);
\draw[draw=none,fill=orange!80] (axis cs:17.0,0) rectangle (axis cs:18.99,6);
\draw[draw=none,fill=orange!80] (axis cs:18.0,0) rectangle (axis cs:19.99,6);
\draw[draw=none,fill=orange!80] (axis cs:19.0,0) rectangle (axis cs:20.99,6);
\draw[draw=none,fill=orange!80] (axis cs:20.0,0) rectangle (axis cs:21.99,8);
\draw[draw=none,fill=orange!80] (axis cs:21.0,0) rectangle (axis cs:22.99,10);
\draw[draw=none,fill=orange!80] (axis cs:22.0,0) rectangle (axis cs:23.99,10);
\draw[draw=none,fill=orange!80] (axis cs:23.0,0) rectangle (axis cs:24.99,12);
\draw[draw=none,fill=orange!80] (axis cs:24.0,0) rectangle (axis cs:25.99,12);
\draw[draw=none,fill=orange!80] (axis cs:25.0,0) rectangle (axis cs:26.99,12);
\draw[draw=none,fill=orange!80] (axis cs:26.0,0) rectangle (axis cs:27.99,12);
\draw[draw=none,fill=orange!80] (axis cs:27.0,0) rectangle (axis cs:28.99,14);
\draw[draw=none,fill=orange!80] (axis cs:28.0,0) rectangle (axis cs:29.99,14);
\draw[draw=none,fill=orange!80] (axis cs:29.0,0) rectangle (axis cs:30.99,14);
\draw[draw=none,fill=orange!80] (axis cs:30.0,0) rectangle (axis cs:31.99,16);
\draw[draw=none,fill=orange!80] (axis cs:31.0,0) rectangle (axis cs:32.99,16);
\draw[draw=none,fill=orange!80] (axis cs:32.0,0) rectangle (axis cs:33.99,16);
\draw[draw=none,fill=orange!80] (axis cs:33.0,0) rectangle (axis cs:34.99,16);
\draw[draw=none,fill=orange!80] (axis cs:34.0,0) rectangle (axis cs:43.99,18);
\draw[draw=none,fill=orange!80] (axis cs:43.0,0) rectangle (axis cs:44.99,20);
\draw[draw=none,fill=orange!80] (axis cs:44.0,0) rectangle (axis cs:45.99,20);
\draw[draw=none,fill=orange!80] (axis cs:45.0,0) rectangle (axis cs:46.99,20);
\draw[draw=none,fill=orange!80] (axis cs:46.0,0) rectangle (axis cs:47.99,22);
\draw[draw=none,fill=orange!80] (axis cs:47.0,0) rectangle (axis cs:48.99,22);
\draw[draw=none,fill=orange!80] (axis cs:48.0,0) rectangle (axis cs:49.99,22);
\draw[draw=none,fill=orange!80] (axis cs:49.0,0) rectangle (axis cs:50.99,22);
\draw[draw=none,fill=orange!80] (axis cs:50.0,0) rectangle (axis cs:51.99,22);
\draw[draw=none,fill=orange!80] (axis cs:51.0,0) rectangle (axis cs:52.99,24);
\draw[draw=none,fill=orange!80] (axis cs:52.0,0) rectangle (axis cs:75.99,26);
\draw[draw=none,fill=orange!80] (axis cs:75.0,0) rectangle (axis cs:76.99,28);
\draw[draw=none,fill=orange!80] (axis cs:76.0,0) rectangle (axis cs:77.99,28);
\draw[draw=none,fill=orange!80] (axis cs:77.0,0) rectangle (axis cs:78.99,28);
\draw[draw=none,fill=orange!80] (axis cs:78.0,0) rectangle (axis cs:79.99,30);
\draw[draw=none,fill=orange!80] (axis cs:79.0,0) rectangle (axis cs:80.99,30);
\draw[draw=none,fill=orange!80] (axis cs:80.0,0) rectangle (axis cs:81.99,30);
\draw[draw=none,fill=orange!80] (axis cs:81.0,0) rectangle (axis cs:82.99,30);
\draw[draw=none,fill=orange!80] (axis cs:82.0,0) rectangle (axis cs:83.99,30);
\draw[draw=none,fill=orange!80] (axis cs:83.0,0) rectangle (axis cs:84.99,30);
\draw[draw=none,fill=orange!80] (axis cs:84.0,0) rectangle (axis cs:85.99,30);
\draw[draw=none,fill=orange!80] (axis cs:85.0,0) rectangle (axis cs:86.99,32);
\draw[draw=none,fill=orange!80] (axis cs:86.0,0) rectangle (axis cs:87.99,32);
\draw[draw=none,fill=orange!80] (axis cs:87.0,0) rectangle (axis cs:88.99,32);
\draw[draw=none,fill=orange!80] (axis cs:88.0,0) rectangle (axis cs:89.99,32);
\draw[draw=none,fill=orange!80] (axis cs:89.0,0) rectangle (axis cs:90.99,32);
\draw[draw=none,fill=orange!80] (axis cs:90.0,0) rectangle (axis cs:91.99,32);
\draw[draw=none,fill=orange!80] (axis cs:91.0,0) rectangle (axis cs:116.99,36);
\draw[draw=none,fill=orange!80] (axis cs:116.0,0) rectangle (axis cs:130.99,38);
\draw[draw=none,fill=orange!80] (axis cs:130.0,0) rectangle (axis cs:138.99,40);
\draw[draw=none,fill=orange!80] (axis cs:138.0,0) rectangle (axis cs:139.99,42);
\draw[draw=none,fill=orange!80] (axis cs:139.0,0) rectangle (axis cs:140.99,42);
\draw[draw=none,fill=orange!80] (axis cs:140.0,0) rectangle (axis cs:141.99,42);
\draw[draw=none,fill=orange!80] (axis cs:141.0,0) rectangle (axis cs:142.99,42);
\draw[draw=none,fill=orange!80] (axis cs:142.0,0) rectangle (axis cs:143.99,42);
\draw[draw=none,fill=orange!80] (axis cs:143.0,0) rectangle (axis cs:144.99,42);
\draw[draw=none,fill=orange!80] (axis cs:144.0,0) rectangle (axis cs:145.99,42);
\draw[draw=none,fill=orange!80] (axis cs:145.0,0) rectangle (axis cs:170.99,44);
\draw[draw=none,fill=orange!80] (axis cs:170.0,0) rectangle (axis cs:191.99,46);
\draw[draw=none,fill=orange!80] (axis cs:191.0,0) rectangle (axis cs:192.99,48);
\draw[draw=none,fill=orange!80] (axis cs:192.0,0) rectangle (axis cs:193.99,48);
\draw[draw=none,fill=orange!80] (axis cs:193.0,0) rectangle (axis cs:194.99,48);
\draw[draw=none,fill=orange!80] (axis cs:194.0,0) rectangle (axis cs:195.99,48);
\draw[draw=none,fill=orange!80] (axis cs:195.0,0) rectangle (axis cs:196.99,48);
\draw[draw=none,fill=orange!80] (axis cs:196.0,0) rectangle (axis cs:197.99,48);
\draw[draw=none,fill=orange!80] (axis cs:197.0,0) rectangle (axis cs:211.99,50);
\draw[draw=none,fill=orange!80] (axis cs:211.0,0) rectangle (axis cs:212.99,52);
\draw[draw=none,fill=orange!80] (axis cs:212.0,0) rectangle (axis cs:213.99,52);
\draw[draw=none,fill=orange!80] (axis cs:213.0,0) rectangle (axis cs:214.99,52);
\draw[draw=none,fill=orange!80] (axis cs:214.0,0) rectangle (axis cs:215.99,56);
\draw[draw=none,fill=orange!80] (axis cs:215.0,0) rectangle (axis cs:216.99,56);
\draw[draw=none,fill=orange!80] (axis cs:216.0,0) rectangle (axis cs:217.99,56);
\draw[draw=none,fill=orange!80] (axis cs:217.0,0) rectangle (axis cs:218.99,56);
\draw[draw=none,fill=orange!80] (axis cs:218.0,0) rectangle (axis cs:219.99,56);
\draw[draw=none,fill=orange!80] (axis cs:219.0,0) rectangle (axis cs:220.99,56);
\draw[draw=none,fill=orange!80] (axis cs:220.0,0) rectangle (axis cs:221.99,56);
\draw[draw=none,fill=orange!80] (axis cs:221.0,0) rectangle (axis cs:222.99,56);
\draw[draw=none,fill=orange!80] (axis cs:222.0,0) rectangle (axis cs:223.99,56);
\draw[draw=none,fill=orange!80] (axis cs:223.0,0) rectangle (axis cs:224.99,56);
\draw[draw=none,fill=orange!80] (axis cs:224.0,0) rectangle (axis cs:236.99,58);
\draw[draw=none,fill=orange!80] (axis cs:236.0,0) rectangle (axis cs:247.99,60);
\draw[draw=none,fill=orange!80] (axis cs:247.0,0) rectangle (axis cs:288.99,64);
\draw[draw=none,fill=orange!80] (axis cs:288.0,0) rectangle (axis cs:289.99,68);
\draw[draw=none,fill=orange!80] (axis cs:289.0,0) rectangle (axis cs:290.99,68);
\draw[draw=none,fill=orange!80] (axis cs:290.0,0) rectangle (axis cs:304.99,70);
\draw[draw=none,fill=orange!80] (axis cs:304.0,0) rectangle (axis cs:305.99,72);
\draw[draw=none,fill=orange!80] (axis cs:305.0,0) rectangle (axis cs:306.99,72);
\draw[draw=none,fill=orange!80] (axis cs:306.0,0) rectangle (axis cs:307.99,72);
\draw[draw=none,fill=orange!80] (axis cs:307.0,0) rectangle (axis cs:308.99,72);
\draw[draw=none,fill=orange!80] (axis cs:308.0,0) rectangle (axis cs:309.99,74);
\draw[draw=none,fill=orange!80] (axis cs:309.0,0) rectangle (axis cs:310.99,74);
\draw[draw=none,fill=orange!80] (axis cs:310.0,0) rectangle (axis cs:311.99,74);
\draw[draw=none,fill=orange!80] (axis cs:311.0,0) rectangle (axis cs:366.99,76);
\draw[draw=none,fill=orange!80] (axis cs:366.0,0) rectangle (axis cs:394.99,78);
\draw[draw=none,fill=orange!80] (axis cs:394.0,0) rectangle (axis cs:415.99,80);
\draw[draw=none,fill=orange!80] (axis cs:415.0,0) rectangle (axis cs:416.99,82);
\draw[draw=none,fill=orange!80] (axis cs:416.0,0) rectangle (axis cs:440.99,84);
\draw[draw=none,fill=orange!80] (axis cs:440.0,0) rectangle (axis cs:490.99,86);
\draw[draw=none,fill=orange!80] (axis cs:490.0,0) rectangle (axis cs:513.99,88);
\draw[draw=none,fill=orange!80] (axis cs:513.0,0) rectangle (axis cs:522.99,90);
\draw[draw=none,fill=orange!80] (axis cs:522.0,0) rectangle (axis cs:554.99,92);
\draw[draw=none,fill=orange!80] (axis cs:554.0,0) rectangle (axis cs:583.99,94);
\draw[draw=none,fill=orange!80] (axis cs:583.0,0) rectangle (axis cs:597.99,96);
\draw[draw=none,fill=orange!80] (axis cs:597.0,0) rectangle (axis cs:786.99,98);
\end{axis}

\end{tikzpicture}
\caption{Sample efficiency of Tidyparse at varying Levenshtein radii. After drawing up to $\sim10^5$ samples without replacement we can usually retrieve the human repair for almost all repairs fewer than four edits.}\label{fig:sample_efficiency}
\end{figure}

\begin{wrapfigure}{r}{0.35\textwidth}
\vspace{-0.4cm}
\resizebox{.35\textwidth}{!}{% This file was created with tikzplotlib v0.10.1.
\begin{tikzpicture}
\begin{axis}[
  legend cell align={left},
  legend style={fill opacity=0.8, draw opacity=1, text opacity=1, legend columns=-1, legend pos=north east},
  tick align=outside,
  tick pos=left,
  axis lines*=left,
  title={\textbf{Average throughput}},
  x grid style={darkgray176},
  xlabel={Lexical tokens},
  xmin=-0.95, xmax=19.95,
  xtick style={color=black},
  xtick={0,1,2,3,4,5,6,7,8,9,10,11,12,13,14,15,16,17,18,19},
  xticklabels={4, , 8, , 12, , 16, , 20, , 24, , 28, , 32, , 36, , 40,},
  y grid style={darkgray176},
  ylabel={Total repairs discovered},
  ymin=0, ymax=200000,
  ymode=log,
  legend pos=north east,
  ytick style={color=black}
]
\addlegendimage{empty legend}
\addlegendentry{$\Delta=$}
\addlegendentry{\footnotesize{$1,$}}
\addlegendentry{\footnotesize{$2,$}}
\addlegendentry{\footnotesize{$3,$}}
\addlegendentry{\footnotesize{$4$}}
\addplot [semithick, green, mark=*, mark size=3, mark options={solid}]
table {%
  0 7.14285714285714
  1 12.1666666666667
  2 9.27272727272727
  3 10.6041666666667
  4 7.22857142857143
  5 6.4
  6 10.3956043956044
  7 7.86075949367089
  8 9.81818181818182
  9 8.28089887640449
  10 8.08695652173913
  11 10.0430107526882
  12 6.90476190476191
  13 6.94029850746269
  14 8.74698795180723
  15 7.70422535211268
  16 12.1414141414141
  17 4.1
};
\addplot [semithick, blue, mark=*, mark size=3, mark options={solid}]
table {%
0 46.625
1 135.653846153846
2 175.676923076923
3 79.0090909090909
4 87.6589147286822
5 69.9683544303797
6 83.9896373056995
7 85.9777777777778
8 73.3684210526316
9 67.7715517241379
10 92.1878172588832
11 186.568527918782
12 116.195945945946
13 87.8901734104046
14 99.6514285714286
15 154.44
16 123.658064516129
17 36.6111111111111
};
\addplot [semithick, orange, mark=*, mark size=3, mark options={solid}]
table {%
0 1506.8
1 2076.57142857143
2 5194.91304347826
3 413.939393939394
4 3749.34693877551
5 1130.83076923077
6 431.141025641026
7 601.172839506173
8 2135.26666666667
9 1233.85106382979
10 960.263888888889
11 932.225352112676
12 612.833333333333
13 880.2625
14 965.863636363636
15 725.139784946237
16 1199.05882352941
17 114.5
};
\addplot [semithick, red, mark=*, mark size=3, mark options={solid}]
table {%
0 nan
1 73461
2 24965
3 4220.4
4 35145.8181818182
5 2613.28571428571
6 68402.25
7 3300.5
8 10010
9 24536.5217391304
10 17497.6315789474
11 8883.58333333333
12 1948.23076923077
13 4175.27272727273
14 5145.45454545455
15 27785.8095238095
16 2468
17 4027.25
};
\end{axis}

\end{tikzpicture}}
\vspace{-0.6cm}
\caption{Distinct repairs found in 10s.}
\label{fig:throughput}
\vspace{-0.3cm}
\end{wrapfigure}

End-to-end throughput varies significantly with the edit distance of the repair. Some errors are trivial to fix, while others require a large number of edits to be sampled before the ground truth is discovered. We evaluate throughput by sampling patches across invalid strings $|\err\sigma| \leq 40$ from the StackOverflow dataset balanced across length and distance, and measure the total number of unique valid patches discovered, as a function of string length and edit distance $\Delta\in[1, 4]$. Each trial is terminated after 10 seconds, and the experiment is repeated across 7.3k total repairs. Note the y-axis is log-scaled, as the number of admissible repairs increases sharply with edit distance. Our approach discovers a large number of syntactic repairs in a relatively short amount of time, and is able to quickly saturate the admissible set for $\Delta(\err\sigma, \sigma') \in [1, 4]$ before timeout.

\begin{wrapfigure}{l}{0.35\textwidth}
\vspace{-0.4cm}
\resizebox{.35\textwidth}{!}{% This file was created with matplot2tikz v0.3.3.
\tikzset{mark size=0.5}

\begin{tikzpicture}

\begin{axis}[
log basis y={10},
tick align=outside,
tick pos=left,
x grid style={darkdarkgray176},
xlabel={Snippet length},
xmin=0, xmax=80,
title={\textbf{Latency across repair size}},
xtick style={color=black},
y grid style={darkdarkgray176},
log basis y={10},
axis lines*=left,
ylabel={Repair latency (ms)},
ymin=6.99556128596625, ymax=147694.796423657,
ymode=log,
mark size=1.2pt,
ytick style={color=black},
yticklabels={
  \(\displaystyle {10^{0}}\),
  \(\displaystyle {10^{1}}\),
  \(\displaystyle {10^{2}}\),
  \(\displaystyle {10^{3}}\),
  \(\displaystyle {10^{4}}\),
  \(\displaystyle {10^{5}}\),
}
]
\addplot [draw=darkgray, fill=darkgray, mark=*, only marks]
table{%
x  y
9 1549
12 116
16 1976
18 288
39 1303
63 2641
11 56
73 4245
45 1416
34 706
27 1489
37 918
29 542
29 389
15 142
39 968
29 434
45 1260
19 879
61 2273
7 120
23 924
51 1570
49 1441
61 2300
42 1175
68 3944
60 2896
72 4049
53 2217
8 113
13 265
62 17742
33 3818
47 9889
41 1394
60 4760
77 4309
72 4271
27 484
75 3934
16 96
32 677
21 339
49 1836
45 1435
58 3188
18 120
30 816
15 463
19 714
49 1691
43 1476
41 831
14 192
21 182
19 10979
78 4436
52 1445
23 652
21 1557
33 974
17 118
30 933
30 374
24 541
66 25769
50 1829
47 1638
46 1360
35 4446
51 1482
59 2163
13 150
41 947
33 1189
26 312
17 182
17 178
20 253
27 1030
34 20964
26 315
8 25
20 152
53 2119
59 2414
41 26227
19 259
63 2519
63 10671
41 1815
32 423
65 4079
74 4863
36 986
9 37
46 1446
78 5210
61 4684
69 4292
14 79
45 1274
14 232
33 540
38 953
16 103
31 908
69 31228
33 709
31 13133
15 253
48 2581
66 19478
36 629
33 495
30 396
40 941
71 4825
10 983
28 426
69 3913
25 994
44 2016
7 21
50 2080
53 2054
29 382
34 763
63 2470
34 955
9 60
16 119
22 10030
40 1040
22 512
77 5203
30 616
25 242
39 958
69 5508
59 2896
21 1549
51 40442
10 134
30 722
54 29715
36 990
40 988
52 1745
10 340
48 1874
29 574
52 2219
39 794
34 732
33 516
23 470
50 15557
30 641
16 496
23 328
22 459
33 742
31 765
7 102
48 2084
45 1484
25 258
30 691
71 3900
27 548
26 353
66 3080
63 4977
40 1250
18 289
14 1780
27 2357
29 623
40 1008
22 213
64 4137
24 398
36 862
46 1424
13 113
68 4167
10 54
53 6662
40 1054
9 66
26 611
68 3685
50 4591
19 798
30 485
52 1583
12 74
34 644
77 6474
76 85265
21 429
11 170
23 287
25 496
34 1099
6 21
28 882
13 149
25 397
35 1044
23 4948
25 1657
15 243
36 848
23 237
27 1713
19 217
25 520
37 919
24 965
39 1717
15 792
49 1469
77 5034
38 3093
40 1097
51 1957
6 63
10 135
17 597
39 753
63 3249
25 447
41 2468
56 2072
74 11972
49 1742
25 494
34 93928
67 6025
64 3035
13 79
56 1714
67 3955
66 3282
63 3261
16 105
5 85
26 545
28 413
8 90
13 106
37 805
69 5137
28 369
29 673
10 44
42 1239
67 4648
27 792
51 7387
20 1096
30 585
37 790
30 462
28 986
9 44
50 1843
70 3327
52 1525
11 79
30 515
16 210
16 196
27 554
11 184
53 1947
8 67
37 929
26 971
26 367
33 648
32 616
14 134
47 1524
5 11
24 913
32 418
75 12513
35 1148
59 2608
11 157
50 2808
15 191
14 184
41 1019
10 157
71 4695
69 4455
34 540
69 4740
51 3258
44 1029
51 1498
50 1420
46 1850
40 1080
17 149
59 2862
21 14398
37 681
37 2124
11 222
23 463
51 2020
33 530
42 17922
4 13
39 698
52 2063
21 328
78 7145
28 429
19 250
7 97
43 1210
59 38172
22 362
42 2745
47 1206
28 379
62 2811
55 2179
75 4071
45 1063
37 1899
12 112
40 745
21 884
9 40
70 3981
36 888
28 392
56 5825
27 393
23 223
24 10408
18 174
23 2177
53 1662
39 1384
41 989
71 3464
60 2194
35 977
67 4303
50 9881
21 379
55 2905
58 2756
45 1769
20 609
43 38706
40 3024
32 664
23 217
18 135
65 5228
67 3550
51 1385
14 107
19 212
33 1746
75 3857
72 4455
44 1118
68 4834
8 55
39 29045
10 43
22 336
48 7343
72 19347
24 269
49 2791
19 785
39 1120
49 1322
21 760
28 521
63 62239
49 1629
23 6078
37 27981
41 1375
51 2274
5 65
17 191
62 3688
27 1016
41 1326
34 2739
33 2785
35 916
63 2554
40 1149
11 56
72 4190
18 145
62 3069
16 379
65 2780
46 1193
50 1621
76 4978
45 1264
16 382
23 655
20 292
27 705
58 2380
26 539
42 7325
15 264
55 2162
76 25267
49 2954
46 1565
55 2499
35 568
24 303
36 783
36 725
29 392
53 6744
19 185
52 1495
11 149
16 583
26 471
76 5789
51 1362
48 1514
54 2037
23 306
8 145
31 623
12 105
70 3810
21 167
11 186
43 11621
15 897
77 30152
8 219
14 2669
15 198
29 907
11 46
38 708
31 894
41 1293
30 24580
60 31750
64 3403
26 527
27 783
21 189
34 980
15 138
9 64
9 32
8 90
59 2109
15 94
33 695
17 14201
36 733
75 5222
36 1902
18 535
67 4110
29 950
35 1798
42 1463
33 766
41 1037
33 503
30 1323
21 252
27 757
32 916
37 1450
54 2323
49 1980
44 9744
26 226
40 11847
22 919
21 4868
42 1916
27 326
53 2114
50 1959
13 178
24 2361
8 92
19 298
29 454
56 3157
27 466
55 2120
71 4493
12 97
25 544
39 1089
38 1161
36 580
53 2653
16 203
10 3282
12 57
30 644
55 1814
34 17457
14 70
25 365
19 265
38 1074
16 203
23 626
46 1762
20 538
13 385
53 2187
37 1069
33 702
9 104
62 4072
32 632
47 1670
50 2049
65 3868
38 810
16 262
62 2518
51 1981
11 109
12 2936
28 520
76 11739
71 3836
49 2243
};
\addplot [draw=darkgray, fill=darkgray, mark=*, only marks]
table{%
x  y
37 967
13 576
31 725
13 82
62 3570
27 286
11 65
55 1667
29 342
18 467
27 300
16 272
57 2199
75 4119
20 205
17 366
16 125
63 4737
45 1024
70 3483
26 608
21 563
30 1598
38 1097
12 62
33 613
61 2243
19 280
27 902
35 556
63 3421
26 714
18 110
47 5579
47 1203
30 955
55 2276
36 1207
46 1206
58 2320
41 855
73 3738
30 646
73 4086
34 805
26 254
48 1192
33 562
14 286
50 1452
64 2511
31 576
28 290
73 4351
45 3090
28 614
36 667
52 2867
77 5575
31 489
50 1848
35 820
42 1054
28 397
76 4091
37 736
64 2432
52 1436
10 169
41 775
9 54
67 3335
27 451
28 303
31 427
57 1809
42 1335
18 216
45 978
39 846
12 157
42 872
20 163
17 398
19 2462
23 198
19 367
26 310
69 3317
15 249
77 4157
42 856
17 121
20 758
36 558
18 149
22 496
62 2332
24 569
29 376
17 939
40 993
24 297
8 70
18 285
57 2095
23 670
29 336
25 1004
30 396
72 3449
38 1492
58 2403
54 2802
43 1186
20 423
26 1536
15 76
5 19
78 5718
39 670
26 281
43 874
44 1863
13 67
17 471
43 1152
24 266
43 956
29 363
48 1478
39 992
16 153
36 693
57 1936
14 390
66 5587
52 2022
31 1252
30 384
20 248
20 150
42 1961
27 464
11 106
21 581
11 99
33 720
29 536
56 6622
61 2275
12 99
78 4706
9 158
78 5312
64 3852
43 935
36 573
33 1219
21 1063
13 59
19 243
23 199
12 185
29 1062
59 2021
25 290
34 585
5 19
39 974
20 186
25 260
74 4492
25 468
46 1182
28 310
44 913
38 735
23 361
31 388
25 252
77 4451
53 3401
10 67
10 110
55 2453
61 2455
71 3342
43 921
63 3298
78 4901
60 13008
30 599
68 2986
32 559
13 310
22 216
20 368
51 2238
29 691
37 608
28 320
35 797
53 3644
23 263
58 2283
24 1170
50 4202
11 130
60 2883
34 2120
71 3378
22 180
36 858
19 217
5 47
18 384
22 392
26 1020
30 430
9 75
62 2291
8 143
39 753
73 3607
57 2080
35 2589
47 1105
20 591
38 1445
38 718
31 728
60 11098
34 496
35 1681
41 781
14 208
65 5798
61 2345
18 116
17 124
21 178
8 93
39 736
49 1481
46 1083
25 554
24 347
63 2938
16 180
15 202
43 1072
11 54
22 216
36 572
27 451
18 292
27 333
37 923
54 1697
26 745
18 223
17 564
48 1236
13 102
24 313
47 1151
24 418
73 4691
22 186
50 1315
72 3757
32 846
29 851
43 1050
34 480
31 419
30 762
17 394
33 557
30 1017
32 422
20 162
16 94
33 690
70 4054
27 626
44 2331
52 1935
22 749
71 3332
20 1017
34 2256
14 107
29 559
18 146
38 761
39 1008
18 226
48 2323
60 2418
38 653
26 273
27 336
33 455
32 416
13 214
77 4543
49 2068
20 355
61 2236
55 4802
11 163
20 179
65 2701
48 6149
77 4137
51 4601
55 1865
26 273
35 4314
67 2978
25 303
24 699
76 4592
23 236
25 265
7 36
34 499
62 2235
22 344
27 787
70 3060
30 2236
20 457
36 1361
72 5525
23 1335
13 373
16 263
30 1738
33 901
9 61
71 3235
40 775
33 638
30 366
16 401
33 3066
40 807
69 3141
9 60
28 398
59 2206
72 5128
36 1173
38 755
30 396
59 9951
73 4086
18 244
19 287
31 629
58 1929
48 1854
29 543
15 574
76 4735
37 3785
28 3065
33 3925
20 315
13 128
21 779
35 586
27 461
38 728
49 2232
15 578
24 237
24 234
39 831
17 264
34 499
14 560
40 799
31 403
24 227
21 418
50 1515
61 2860
48 1200
62 2411
75 4749
35 511
39 687
51 2613
15 99
40 784
4 21
24 2146
37 665
46 1132
24 275
22 396
31 403
33 456
31 615
23 324
16 216
54 1671
34 1328
23 467
56 1767
9 181
36 560
20 191
63 2774
27 371
8 215
27 346
33 3668
41 6073
16 109
56 2557
42 854
29 923
66 7498
28 328
50 2110
17 183
73 4756
39 907
47 1142
74 4066
17 114
18 134
46 1133
24 243
45 1212
35 1622
19 718
50 1286
74 3944
52 1596
22 417
32 1180
24 807
20 175
45 1019
23 203
25 255
65 2716
37 612
40 799
9 75
36 668
16 181
39 3541
16 211
49 2661
25 495
59 1979
37 595
16 403
16 280
53 2048
62 2206
28 308
54 1583
47 1330
21 195
15 394
55 2754
37 1058
32 818
14 95
41 811
15 91
34 561
15 517
13 298
48 1138
28 458
44 1152
24 223
30 376
79 4556
35 814
43 2771
53 1486
21 202
42 848
45 995
18 115
12 170
37 616
19 546
57 1925
26 322
21 173
60 2626
51 1515
27 2046
19 366
18 332
71 3346
31 656
73 3772
44 944
70 3714
43 1159
68 3472
19 336
55 1778
24 251
45 1963
18 154
58 2220
10 220
45 1072
43 1188
33 517
40 876
16 646
48 1214
22 561
37 930
63 4137
33 1414
13 397
30 469
25 306
15 130
52 1676
51 1391
42 1326
29 596
65 2703
60 3376
71 3470
18 109
40 737
35 827
30 444
31 495
59 2695
79 4538
37 619
31 538
20 376
57 2979
23 410
29 908
28 376
36 955
13 225
43 1404
29 472
35 604
23 220
16 168
17 151
13 151
13 77
12 396
57 1895
41 860
57 2269
75 3797
22 176
21 658
27 286
19 172
54 1513
10 138
17 319
44 987
27 286
25 473
17 119
64 2514
28 447
42 1606
53 1810
14 160
33 1491
53 2038
58 2610
21 236
36 578
27 473
24 275
74 3879
79 5334
67 3619
47 2079
41 833
15 254
53 1662
26 265
20 142
22 415
52 1511
11 74
31 505
50 1314
9 219
20 178
53 1688
55 1727
25 714
21 380
45 1511
73 3864
20 169
14 178
35 2191
63 2616
29 603
49 2332
47 1248
26 793
7 19
28 335
18 416
36 1161
20 182
24 714
14 88
49 1390
37 637
66 3852
29 337
12 64
32 468
34 1461
17 104
42 912
17 406
28 464
20 661
34 1508
32 501
41 785
74 3767
17 125
40 1750
30 468
11 256
53 1588
35 716
28 523
20 132
38 1072
62 2652
45 1865
42 888
71 3634
55 1915
66 2838
39 826
66 3678
62 2941
33 862
32 1045
20 463
26 407
14 88
51 1526
32 1004
32 810
27 396
5 34
15 458
31 530
11 116
36 1260
21 204
46 1272
64 3805
53 1946
56 2260
44 981
29 394
34 764
29 503
24 267
20 357
70 5314
21 212
19 271
25 1602
36 668
54 1692
41 879
27 314
29 469
43 1248
26 288
78 7398
46 1522
42 1005
19 162
20 659
16 352
29 369
41 1444
35 584
29 493
72 4540
36 565
21 948
16 111
34 1065
59 2490
22 253
16 195
51 1527
49 1373
53 1549
36 677
64 2722
55 1864
25 269
24 230
37 748
32 447
16 151
15 122
16 872
65 3000
17 100
12 296
29 466
32 674
29 337
15 230
45 1084
35 960
79 5158
34 563
21 394
33 633
24 330
27 295
27 329
36 565
34 1367
41 802
71 4002
35 548
47 1487
18 198
29 448
57 3080
56 2030
39 706
37 669
11 104
41 877
27 318
36 1874
22 231
42 1185
8 40
42 894
32 855
64 3359
33 603
41 823
22 306
29 448
15 94
34 824
6 15
41 860
12 153
4 28
35 906
72 3895
64 2509
16 427
15 113
31 737
18 138
42 1059
11 63
12 149
5 37
65 2822
17 278
19 303
17 335
18 175
17 562
22 605
23 240
35 3819
26 259
45 997
52 1586
16 429
34 490
29 1283
22 555
33 991
34 520
70 4662
13 213
13 97
52 1514
22 210
50 1817
29 383
33 675
18 422
18 117
53 2349
24 2137
39 1467
18 157
78 5258
45 1250
29 342
10 82
28 322
24 292
71 4662
16 211
27 294
15 271
42 1225
33 532
57 1970
34 566
36 1764
34 592
32 499
9 81
72 5270
31 1433
31 725
61 2268
50 2300
19 925
32 446
8 73
71 4323
25 763
18 310
31 442
54 1689
22 248
17 298
19 148
39 715
46 1237
24 378
33 620
55 1928
33 448
54 2785
28 2556
16 114
17 143
29 414
6 15
25 820
76 4052
35 846
27 347
44 1173
73 3785
34 486
15 148
69 4745
31 1132
12 130
29 553
77 4168
47 1263
54 3108
40 2104
25 370
32 1279
52 12972
36 1281
29 417
20 184
38 1157
42 889
11 164
46 1070
47 2268
5 46
13 530
9 55
30 898
45 1442
20 179
34 2244
11 49
14 85
5 29
12 48
26 402
25 586
28 415
38 708
40 1013
24 470
26 771
28 380
36 1000
19 609
48 1182
52 1710
26 289
36 605
13 390
23 376
28 3022
20 985
26 405
42 853
20 159
11 89
15 99
60 2482
38 870
35 743
20 890
19 1640
14 92
25 295
25 913
5 41
32 472
25 233
22 446
15 232
68 3026
19 514
37 2185
78 4881
39 773
26 295
41 909
30 432
12 304
13 88
32 454
76 4302
26 819
74 6097
35 603
48 4513
21 378
40 834
29 380
66 2766
71 3677
23 1391
24 261
38 747
8 146
26 1750
39 732
22 588
9 155
33 485
23 1342
15 459
44 1375
76 4937
44 1381
15 372
34 1294
22 195
29 333
49 6540
18 163
57 2083
44 963
34 504
19 229
59 2052
34 624
35 551
59 2096
23 195
46 1312
74 3939
29 628
26 633
27 306
20 651
18 122
50 1552
43 2398
27 559
12 49
34 648
65 2751
27 443
33 518
56 1772
12 56
37 1870
36 600
29 2453
19 321
22 622
29 1666
62 2886
43 976
15 110
54 2059
53 1487
16 124
24 383
37 602
35 531
34 716
58 2505
14 164
22 323
35 664
23 400
78 5086
};
\addplot [draw=darkgray, fill=darkgray, mark=*, only marks]
table{%
x  y
19 524
42 1166
45 3390
69 18648
51 3254
32 10954
36 680
76 7230
40 2763
64 3123
35 970
75 26775
60 4154
37 3369
28 406
48 36005
70 5503
54 4443
43 4796
35 4535
43 2163
11 255
30 1756
31 701
61 13931
37 2268
76 30644
44 925
43 2764
15 1076
50 6609
60 12517
30 2429
29 1154
34 694
9 655
29 1101
31 3358
33 620
12 3279
32 5493
65 15497
26 507
29 1146
65 5376
24 858
36 1494
40 1621
26 787
71 4837
36 1460
39 1047
28 485
29 1114
25 3603
67 3174
};
\end{axis}

\end{tikzpicture}
}
\vspace{-0.7cm}
\caption{End-to-end repair timings.}
\label{fig:timings}
\vspace{-0.3cm}
\end{wrapfigure}

In Fig.~\ref{fig:timings}, we plot the end-to-end repair timings by collecting 1,645 samples balanced across length and edit distance, then measure the wallclock runtime of the single core implementation of the repair algorithm (\S~\ref{sec:implementation}). This wall-clock runtime corroborates the complexity analysis (\S~\ref{sec:method}) showing a distinct lower bound. While short repairs finish quickly, latency is positively correlated with length and edit distance. Our method is typically able to saturate the admissible set for 1- and 2-edit repairs before timeout, while 4+-edit throughput starts becoming constrained by compute around 10s, when Python's admissible set approaches a volume of $10^5$ valid edits. This bottleneck can be relaxed with a longer timeout or additional processor cores. We anticipate that a much longer delay will begin to tax the patience of most users, and so we consider 10s a reasonable upper bound for repair latency.

%  In the following benchmark, we measure the Precision@k of our repair procedure against human repairs of varying edit distances and latency cutoffs, using an adaptive resampling procedure described in \S\ref{sec:adaptive}. This sampler maintains a buffer of successful repairs ranked by perplexity and uses stochastic local search to resample edits within a neighborhood. Initially, edits are sampled uniformly at random. Over time and as the admissible set grows, it prioritizes edits nearby low-perplexity repairs. This technique offers a significant advantage in the low-latency setting.

\clearpage\subsection{Subcomponent ablation}\label{sec:rq3}

Originally, we used an adaptive rejection-based sampler, which did not sample directly from the admissible set, but the entire Levenshtein ball, and then rejected invalid samples. Although rejection sampling has a much lower minimum latency threshold to return admissible repairs, i.e., a few hundred milliseconds, the average time required to attain a desired precision on human repairs is much higher. We present the results from the rejection-based evaluation for comparison below.

\begin{figure}[H]
\resizebox{.24\textwidth}{!}{% This file was created with tikzplotlib v0.10.1.
\begin{tikzpicture}

\definecolor{darkgray176}{RGB}{176,176,176}
\definecolor{darkviolet1270255}{RGB}{127,0,255}
\definecolor{deepskyblue3176236}{RGB}{3,176,236}
\definecolor{dodgerblue45123246}{RGB}{45,123,246}
\definecolor{lightgray204}{RGB}{204,204,204}
\definecolor{royalblue8762253}{RGB}{87,62,253}

\begin{axis}[
legend cell align={left},
legend style={fill opacity=0.8, draw opacity=1, text opacity=1, draw=lightgray204, legend columns=-1, legend pos=north west},
tick align=outside,
tick pos=left,
title={\(\displaystyle \Delta\in[1,3]\) Repair Precision},
x grid style={darkgray176},
xlabel={Seconds},
xmin=-0.5925, xmax=8.5925,
ymin=0.0, ymax=0.7,
xtick style={color=black},
xtick={0,1,2,3,4,5,6,7,8},
xticklabels={0.1,0.3,0.5,0.7,0.9,1.1,1.3,1.5,1.7},
y grid style={darkgray176},
ylabel={Precision@k},
ytick style={color=black},
width=0.5\textwidth,
height=0.48\textwidth,
]
\addlegendimage{empty legend}
\addlegendentry{P@}
\draw[draw=none,fill=darkviolet1270255] (axis cs:-0.175,0) rectangle (axis cs:0.175,0.245);
\addlegendimage{ybar,ybar legend,draw=none,fill=darkviolet1270255}
\addlegendentry{All}

\draw[draw=none,fill=darkviolet1270255] (axis cs:0.825,0) rectangle (axis cs:1.175,0.417);
\draw[draw=none,fill=darkviolet1270255] (axis cs:1.825,0) rectangle (axis cs:2.175,0.496);
\draw[draw=none,fill=darkviolet1270255] (axis cs:2.825,0) rectangle (axis cs:3.175,0.535);
\draw[draw=none,fill=darkviolet1270255] (axis cs:3.825,0) rectangle (axis cs:4.175,0.562);
\draw[draw=none,fill=darkviolet1270255] (axis cs:4.825,0) rectangle (axis cs:5.175,0.584);
\draw[draw=none,fill=darkviolet1270255] (axis cs:5.825,0) rectangle (axis cs:6.175,0.603);
\draw[draw=none,fill=darkviolet1270255] (axis cs:6.825,0) rectangle (axis cs:7.175,0.619);
\draw[draw=none,fill=darkviolet1270255] (axis cs:7.825,0) rectangle (axis cs:8.175,0.629);
\draw[draw=none,fill=royalblue8762253] (axis cs:-0.175,0) rectangle (axis cs:0.175,0.234);
\addlegendimage{ybar,ybar legend,draw=none,fill=royalblue8762253}
\addlegendentry{10}

\draw[draw=none,fill=royalblue8762253] (axis cs:0.825,0) rectangle (axis cs:1.175,0.387);
\draw[draw=none,fill=royalblue8762253] (axis cs:1.825,0) rectangle (axis cs:2.175,0.451);
\draw[draw=none,fill=royalblue8762253] (axis cs:2.825,0) rectangle (axis cs:3.175,0.481);
\draw[draw=none,fill=royalblue8762253] (axis cs:3.825,0) rectangle (axis cs:4.175,0.502);
\draw[draw=none,fill=royalblue8762253] (axis cs:4.825,0) rectangle (axis cs:5.175,0.517);
\draw[draw=none,fill=royalblue8762253] (axis cs:5.825,0) rectangle (axis cs:6.175,0.53);
\draw[draw=none,fill=royalblue8762253] (axis cs:6.825,0) rectangle (axis cs:7.175,0.54);
\draw[draw=none,fill=royalblue8762253] (axis cs:7.825,0) rectangle (axis cs:8.175,0.546);
\draw[draw=none,fill=dodgerblue45123246] (axis cs:-0.175,0) rectangle (axis cs:0.175,0.218);
\addlegendimage{ybar,ybar legend,draw=none,fill=dodgerblue45123246}
\addlegendentry{5}

\draw[draw=none,fill=dodgerblue45123246] (axis cs:0.825,0) rectangle (axis cs:1.175,0.353);
\draw[draw=none,fill=dodgerblue45123246] (axis cs:1.825,0) rectangle (axis cs:2.175,0.406);
\draw[draw=none,fill=dodgerblue45123246] (axis cs:2.825,0) rectangle (axis cs:3.175,0.43);
\draw[draw=none,fill=dodgerblue45123246] (axis cs:3.825,0) rectangle (axis cs:4.175,0.444);
\draw[draw=none,fill=dodgerblue45123246] (axis cs:4.825,0) rectangle (axis cs:5.175,0.455);
\draw[draw=none,fill=dodgerblue45123246] (axis cs:5.825,0) rectangle (axis cs:6.175,0.467);
\draw[draw=none,fill=dodgerblue45123246] (axis cs:6.825,0) rectangle (axis cs:7.175,0.474);
\draw[draw=none,fill=dodgerblue45123246] (axis cs:7.825,0) rectangle (axis cs:8.175,0.479);
\draw[draw=none,fill=deepskyblue3176236] (axis cs:-0.175,0) rectangle (axis cs:0.175,0.117);
\addlegendimage{ybar,ybar legend,draw=none,fill=deepskyblue3176236}
\addlegendentry{1}

\draw[draw=none,fill=deepskyblue3176236] (axis cs:0.825,0) rectangle (axis cs:1.175,0.187);
\draw[draw=none,fill=deepskyblue3176236] (axis cs:1.825,0) rectangle (axis cs:2.175,0.219);
\draw[draw=none,fill=deepskyblue3176236] (axis cs:2.825,0) rectangle (axis cs:3.175,0.233);
\draw[draw=none,fill=deepskyblue3176236] (axis cs:3.825,0) rectangle (axis cs:4.175,0.244);
\draw[draw=none,fill=deepskyblue3176236] (axis cs:4.825,0) rectangle (axis cs:5.175,0.25);
\draw[draw=none,fill=deepskyblue3176236] (axis cs:5.825,0) rectangle (axis cs:6.175,0.258);
\draw[draw=none,fill=deepskyblue3176236] (axis cs:6.825,0) rectangle (axis cs:7.175,0.262);
\draw[draw=none,fill=deepskyblue3176236] (axis cs:7.825,0) rectangle (axis cs:8.175,0.265);
\addplot [red, thick] coordinates {(-0.8925,0.35) (14.8925,0.35)};
\end{axis}

\end{tikzpicture}
}
\resizebox{.25\textwidth}{!}{% This file was created with tikzplotlib v0.10.1.
\begin{tikzpicture}

\definecolor{darkgray176}{RGB}{176,176,176}
\definecolor{darkviolet1270255}{RGB}{127,0,255}
\definecolor{deepskyblue3176236}{RGB}{3,176,236}
\definecolor{dodgerblue45123246}{RGB}{45,123,246}
\definecolor{lightgray204}{RGB}{204,204,204}
\definecolor{royalblue8762253}{RGB}{87,62,253}

\begin{axis}[
legend cell align={left},
legend style={fill opacity=0.8, draw opacity=1, text opacity=1, draw=lightgray204, legend columns=-1, legend pos=north west},
tick align=outside,
tick pos=left,
title={$\Delta=1$ Repair Precision},
x grid style={darkgray176},
xlabel={Seconds},
xmin=-0.5925, xmax=8.5925,
xtick style={color=black},
xtick={0,1,2,3,4,5,6,7,8},
xticklabels={0.1,0.3,0.5,0.7,0.9,1.1,1.3,1.5,1.7},
y grid style={darkgray176},
ymin=0, ymax=1.0,
ytick style={color=black}
width=0.43\textwidth,
height=0.49\textwidth,
]
\addlegendimage{empty legend}
\addlegendentry{P@}
\draw[draw=none,fill=darkviolet1270255] (axis cs:-0.175,0) rectangle (axis cs:0.175,0.416);
\addlegendimage{ybar,ybar legend,draw=none,fill=darkviolet1270255}
\addlegendentry{All}

\draw[draw=none,fill=darkviolet1270255] (axis cs:0.825,0) rectangle (axis cs:1.175,0.672);
\draw[draw=none,fill=darkviolet1270255] (axis cs:1.825,0) rectangle (axis cs:2.175,0.773);
\draw[draw=none,fill=darkviolet1270255] (axis cs:2.825,0) rectangle (axis cs:3.175,0.822);
\draw[draw=none,fill=darkviolet1270255] (axis cs:3.825,0) rectangle (axis cs:4.175,0.849);
\draw[draw=none,fill=darkviolet1270255] (axis cs:4.825,0) rectangle (axis cs:5.175,0.873);
\draw[draw=none,fill=darkviolet1270255] (axis cs:5.825,0) rectangle (axis cs:6.175,0.891);
\draw[draw=none,fill=darkviolet1270255] (axis cs:6.825,0) rectangle (axis cs:7.175,0.904);
\draw[draw=none,fill=darkviolet1270255] (axis cs:7.825,0) rectangle (axis cs:8.175,0.912);
\draw[draw=none,fill=royalblue8762253] (axis cs:-0.175,0) rectangle (axis cs:0.175,0.402);
\addlegendimage{ybar,ybar legend,draw=none,fill=royalblue8762253}
\addlegendentry{10}

\draw[draw=none,fill=royalblue8762253] (axis cs:0.825,0) rectangle (axis cs:1.175,0.645);
\draw[draw=none,fill=royalblue8762253] (axis cs:1.825,0) rectangle (axis cs:2.175,0.741);
\draw[draw=none,fill=royalblue8762253] (axis cs:2.825,0) rectangle (axis cs:3.175,0.786);
\draw[draw=none,fill=royalblue8762253] (axis cs:3.825,0) rectangle (axis cs:4.175,0.812);
\draw[draw=none,fill=royalblue8762253] (axis cs:4.825,0) rectangle (axis cs:5.175,0.834);
\draw[draw=none,fill=royalblue8762253] (axis cs:5.825,0) rectangle (axis cs:6.175,0.851);
\draw[draw=none,fill=royalblue8762253] (axis cs:6.825,0) rectangle (axis cs:7.175,0.862);
\draw[draw=none,fill=royalblue8762253] (axis cs:7.825,0) rectangle (axis cs:8.175,0.87);
\draw[draw=none,fill=dodgerblue45123246] (axis cs:-0.175,0) rectangle (axis cs:0.175,0.379);
\addlegendimage{ybar,ybar legend,draw=none,fill=dodgerblue45123246}
\addlegendentry{5}

\draw[draw=none,fill=dodgerblue45123246] (axis cs:0.825,0) rectangle (axis cs:1.175,0.601);
\draw[draw=none,fill=dodgerblue45123246] (axis cs:1.825,0) rectangle (axis cs:2.175,0.686);
\draw[draw=none,fill=dodgerblue45123246] (axis cs:2.825,0) rectangle (axis cs:3.175,0.724);
\draw[draw=none,fill=dodgerblue45123246] (axis cs:3.825,0) rectangle (axis cs:4.175,0.743);
\draw[draw=none,fill=dodgerblue45123246] (axis cs:4.825,0) rectangle (axis cs:5.175,0.762);
\draw[draw=none,fill=dodgerblue45123246] (axis cs:5.825,0) rectangle (axis cs:6.175,0.777);
\draw[draw=none,fill=dodgerblue45123246] (axis cs:6.825,0) rectangle (axis cs:7.175,0.785);
\draw[draw=none,fill=dodgerblue45123246] (axis cs:7.825,0) rectangle (axis cs:8.175,0.791);
\draw[draw=none,fill=deepskyblue3176236] (axis cs:-0.175,0) rectangle (axis cs:0.175,0.211);
\addlegendimage{ybar,ybar legend,draw=none,fill=deepskyblue3176236}
\addlegendentry{1}

\draw[draw=none,fill=deepskyblue3176236] (axis cs:0.825,0) rectangle (axis cs:1.175,0.327);
\draw[draw=none,fill=deepskyblue3176236] (axis cs:1.825,0) rectangle (axis cs:2.175,0.378);
\draw[draw=none,fill=deepskyblue3176236] (axis cs:2.825,0) rectangle (axis cs:3.175,0.401);
\draw[draw=none,fill=deepskyblue3176236] (axis cs:3.825,0) rectangle (axis cs:4.175,0.418);
\draw[draw=none,fill=deepskyblue3176236] (axis cs:4.825,0) rectangle (axis cs:5.175,0.428);
\draw[draw=none,fill=deepskyblue3176236] (axis cs:5.825,0) rectangle (axis cs:6.175,0.437);
\draw[draw=none,fill=deepskyblue3176236] (axis cs:6.825,0) rectangle (axis cs:7.175,0.442);
\draw[draw=none,fill=deepskyblue3176236] (axis cs:7.825,0) rectangle (axis cs:8.175,0.446);
\addplot [red, thick] coordinates {(-0.8925,0.45) (14.8925,0.45)};
\end{axis}

\end{tikzpicture}
}
\resizebox{.24\textwidth}{!}{% This file was created with tikzplotlib v0.10.1.
\begin{tikzpicture}

\definecolor{darkgray176}{RGB}{176,176,176}
\definecolor{darkviolet1270255}{RGB}{127,0,255}
\definecolor{deepskyblue3176236}{RGB}{3,176,236}
\definecolor{dodgerblue45123246}{RGB}{45,123,246}
\definecolor{lightgray204}{RGB}{204,204,204}
\definecolor{royalblue8762253}{RGB}{87,62,253}

\begin{axis}[
legend cell align={left},
legend style={fill opacity=0.8, draw opacity=1, text opacity=1, draw=lightgray204, legend columns=-1, legend pos=north west},
tick align=outside,
tick pos=left,
title={$\Delta=2$ Repair Precision},
x grid style={darkgray176},
xlabel={Seconds},
xmin=-0.5925, xmax=8.5925,
xtick style={color=black},
xtick={0,1,2,3,4,5,6,7,8},
xticklabels={0.1,0.3,0.5,0.7,0.9,1.1,1.3,1.5,1.7},
y grid style={darkgray176},
ymin=0, ymax=0.3843,
ytick style={color=black},
width=0.55\textwidth,
height=0.49\textwidth,
]
\addlegendimage{empty legend}
\addlegendentry{P@}
\draw[draw=none,fill=darkviolet1270255] (axis cs:-0.175,0) rectangle (axis cs:0.175,0.067);
\addlegendimage{ybar,ybar legend,draw=none,fill=darkviolet1270255}
\addlegendentry{All}

\draw[draw=none,fill=darkviolet1270255] (axis cs:0.825,0) rectangle (axis cs:1.175,0.165);
\draw[draw=none,fill=darkviolet1270255] (axis cs:1.825,0) rectangle (axis cs:2.175,0.225);
\draw[draw=none,fill=darkviolet1270255] (axis cs:2.825,0) rectangle (axis cs:3.175,0.261);
\draw[draw=none,fill=darkviolet1270255] (axis cs:3.825,0) rectangle (axis cs:4.175,0.292);
\draw[draw=none,fill=darkviolet1270255] (axis cs:4.825,0) rectangle (axis cs:5.175,0.309);
\draw[draw=none,fill=darkviolet1270255] (axis cs:5.825,0) rectangle (axis cs:6.175,0.335);
\draw[draw=none,fill=darkviolet1270255] (axis cs:6.825,0) rectangle (axis cs:7.175,0.353);
\draw[draw=none,fill=darkviolet1270255] (axis cs:7.825,0) rectangle (axis cs:8.175,0.366);
\draw[draw=none,fill=royalblue8762253] (axis cs:-0.175,0) rectangle (axis cs:0.175,0.06);
\addlegendimage{ybar,ybar legend,draw=none,fill=royalblue8762253}
\addlegendentry{10}

\draw[draw=none,fill=royalblue8762253] (axis cs:0.825,0) rectangle (axis cs:1.175,0.128);
\draw[draw=none,fill=royalblue8762253] (axis cs:1.825,0) rectangle (axis cs:2.175,0.164);
\draw[draw=none,fill=royalblue8762253] (axis cs:2.825,0) rectangle (axis cs:3.175,0.182);
\draw[draw=none,fill=royalblue8762253] (axis cs:3.825,0) rectangle (axis cs:4.175,0.2);
\draw[draw=none,fill=royalblue8762253] (axis cs:4.825,0) rectangle (axis cs:5.175,0.208);
\draw[draw=none,fill=royalblue8762253] (axis cs:5.825,0) rectangle (axis cs:6.175,0.221);
\draw[draw=none,fill=royalblue8762253] (axis cs:6.825,0) rectangle (axis cs:7.175,0.23);
\draw[draw=none,fill=royalblue8762253] (axis cs:7.825,0) rectangle (axis cs:8.175,0.238);
\draw[draw=none,fill=dodgerblue45123246] (axis cs:-0.175,0) rectangle (axis cs:0.175,0.05);
\addlegendimage{ybar,ybar legend,draw=none,fill=dodgerblue45123246}
\addlegendentry{5}

\draw[draw=none,fill=dodgerblue45123246] (axis cs:0.825,0) rectangle (axis cs:1.175,0.1);
\draw[draw=none,fill=dodgerblue45123246] (axis cs:1.825,0) rectangle (axis cs:2.175,0.123);
\draw[draw=none,fill=dodgerblue45123246] (axis cs:2.825,0) rectangle (axis cs:3.175,0.132);
\draw[draw=none,fill=dodgerblue45123246] (axis cs:3.825,0) rectangle (axis cs:4.175,0.142);
\draw[draw=none,fill=dodgerblue45123246] (axis cs:4.825,0) rectangle (axis cs:5.175,0.144);
\draw[draw=none,fill=dodgerblue45123246] (axis cs:5.825,0) rectangle (axis cs:6.175,0.155);
\draw[draw=none,fill=dodgerblue45123246] (axis cs:6.825,0) rectangle (axis cs:7.175,0.163);
\draw[draw=none,fill=dodgerblue45123246] (axis cs:7.825,0) rectangle (axis cs:8.175,0.167);
\draw[draw=none,fill=deepskyblue3176236] (axis cs:-0.175,0) rectangle (axis cs:0.175,0.016);
\addlegendimage{ybar,ybar legend,draw=none,fill=deepskyblue3176236}
\addlegendentry{1}

\draw[draw=none,fill=deepskyblue3176236] (axis cs:0.825,0) rectangle (axis cs:1.175,0.038);
\draw[draw=none,fill=deepskyblue3176236] (axis cs:1.825,0) rectangle (axis cs:2.175,0.052);
\draw[draw=none,fill=deepskyblue3176236] (axis cs:2.825,0) rectangle (axis cs:3.175,0.057);
\draw[draw=none,fill=deepskyblue3176236] (axis cs:3.825,0) rectangle (axis cs:4.175,0.061);
\draw[draw=none,fill=deepskyblue3176236] (axis cs:4.825,0) rectangle (axis cs:5.175,0.064);
\draw[draw=none,fill=deepskyblue3176236] (axis cs:5.825,0) rectangle (axis cs:6.175,0.072);
\draw[draw=none,fill=deepskyblue3176236] (axis cs:6.825,0) rectangle (axis cs:7.175,0.076);
\draw[draw=none,fill=deepskyblue3176236] (axis cs:7.825,0) rectangle (axis cs:8.175,0.078);
\addplot [red, thick] coordinates {(-0.8925,0.16) (14.8925,0.16)};
\end{axis}

\end{tikzpicture}
}
\resizebox{.24\textwidth}{!}{% This file was created with tikzplotlib v0.10.1.
\begin{tikzpicture}
\begin{axis}[
legend cell align={left},
legend style={fill opacity=0.8, draw opacity=1, text opacity=1, draw=lightgray204, legend columns=-1, legend pos=north west},
tick align=outside,
tick pos=left,
axis lines*=left,
title={$\Delta=3$ Repair Precision},
x grid style={darkgray176},
xlabel={Seconds},
xmin=-0.5925, xmax=8.5925,
xtick style={color=black},
xtick={0,1,2,3,4,5,6,7,8},
xticklabels={0.1,0.3,0.5,0.7,0.9,1.1,1.3,1.5,1.7},
yticklabels={0, 0.05, 0.10, 0.15, 0.20},
y grid style={darkgray176},
ymin=0, ymax=0.15225,
ytick style={color=black},
width=0.53\textwidth,
height=0.48\textwidth,
]
\draw[draw=none,fill=darkviolet1270255] (axis cs:-0.175,0) rectangle (axis cs:0.175,0.01);

\draw[draw=none,fill=darkviolet1270255] (axis cs:0.825,0) rectangle (axis cs:1.175,0.035);
\draw[draw=none,fill=darkviolet1270255] (axis cs:1.825,0) rectangle (axis cs:2.175,0.064);
\draw[draw=none,fill=darkviolet1270255] (axis cs:2.825,0) rectangle (axis cs:3.175,0.072);
\draw[draw=none,fill=darkviolet1270255] (axis cs:3.825,0) rectangle (axis cs:4.175,0.09);
\draw[draw=none,fill=darkviolet1270255] (axis cs:4.825,0) rectangle (axis cs:5.175,0.109);
\draw[draw=none,fill=darkviolet1270255] (axis cs:5.825,0) rectangle (axis cs:6.175,0.116);
\draw[draw=none,fill=darkviolet1270255] (axis cs:6.825,0) rectangle (axis cs:7.175,0.133);
\draw[draw=none,fill=darkviolet1270255] (axis cs:7.825,0) rectangle (axis cs:8.175,0.145);
\draw[draw=none,fill=royalblue8762253] (axis cs:-0.175,0) rectangle (axis cs:0.175,0.003);

\draw[draw=none,fill=royalblue8762253] (axis cs:0.825,0) rectangle (axis cs:1.175,0.01);
\draw[draw=none,fill=royalblue8762253] (axis cs:1.825,0) rectangle (axis cs:2.175,0.01);
\draw[draw=none,fill=royalblue8762253] (axis cs:2.825,0) rectangle (axis cs:3.175,0.014);
\draw[draw=none,fill=royalblue8762253] (axis cs:3.825,0) rectangle (axis cs:4.175,0.018);
\draw[draw=none,fill=royalblue8762253] (axis cs:4.825,0) rectangle (axis cs:5.175,0.019);
\draw[draw=none,fill=royalblue8762253] (axis cs:5.825,0) rectangle (axis cs:6.175,0.018);
\draw[draw=none,fill=royalblue8762253] (axis cs:6.825,0) rectangle (axis cs:7.175,0.023);
\draw[draw=none,fill=royalblue8762253] (axis cs:7.825,0) rectangle (axis cs:8.175,0.023);
\draw[draw=none,fill=dodgerblue45123246] (axis cs:-0.175,0) rectangle (axis cs:0.175,0.002);

\draw[draw=none,fill=dodgerblue45123246] (axis cs:0.825,0) rectangle (axis cs:1.175,0.005);
\draw[draw=none,fill=dodgerblue45123246] (axis cs:1.825,0) rectangle (axis cs:2.175,0.008);
\draw[draw=none,fill=dodgerblue45123246] (axis cs:2.825,0) rectangle (axis cs:3.175,0.01);
\draw[draw=none,fill=dodgerblue45123246] (axis cs:3.825,0) rectangle (axis cs:4.175,0.011);
\draw[draw=none,fill=dodgerblue45123246] (axis cs:4.825,0) rectangle (axis cs:5.175,0.011);
\draw[draw=none,fill=dodgerblue45123246] (axis cs:5.825,0) rectangle (axis cs:6.175,0.011);
\draw[draw=none,fill=dodgerblue45123246] (axis cs:6.825,0) rectangle (axis cs:7.175,0.014);
\draw[draw=none,fill=dodgerblue45123246] (axis cs:7.825,0) rectangle (axis cs:8.175,0.018);
\draw[draw=none,fill=deepskyblue3176236] (axis cs:-0.175,0) rectangle (axis cs:0.175,0);

\draw[draw=none,fill=deepskyblue3176236] (axis cs:0.825,0) rectangle (axis cs:1.175,0.003);
\draw[draw=none,fill=deepskyblue3176236] (axis cs:1.825,0) rectangle (axis cs:2.175,0.008);
\draw[draw=none,fill=deepskyblue3176236] (axis cs:2.825,0) rectangle (axis cs:3.175,0.008);
\draw[draw=none,fill=deepskyblue3176236] (axis cs:3.825,0) rectangle (axis cs:4.175,0.011);
\draw[draw=none,fill=deepskyblue3176236] (axis cs:4.825,0) rectangle (axis cs:5.175,0.011);
\draw[draw=none,fill=deepskyblue3176236] (axis cs:5.825,0) rectangle (axis cs:6.175,0.011);
\draw[draw=none,fill=deepskyblue3176236] (axis cs:6.825,0) rectangle (axis cs:7.175,0.013);
\draw[draw=none,fill=deepskyblue3176236] (axis cs:7.825,0) rectangle (axis cs:8.175,0.013);
\addplot [red, thick] coordinates {(-0.8925,0.096) (14.8925,0.096)};
\addplot [orange, thick] coordinates {(-0.8925,0.02) (14.8925,0.02)};
\end{axis}

\end{tikzpicture}
}
\caption{Adaptive sampling repairs. The red line indicates Seq2Parse Precision@1, and the orange indicates BIFI's precision at single-shot repair, all three of which were evaluated on the exact same repairs.}\label{fig:adaptive}
\end{figure}

We also evaluate Seq2Parse on the same dataset. Seq2Parse only supports Precision@1 repairs, and so we only report Seq2parse Precision@1 from the StackOverflow benchmark for comparison. Unlike our approach, which only produces syntactically correct repairs, Seq2Parse and BIFI also produce syntactically incorrect repairs in practice. The overall latency of Seq2Parse varies depending on the length of the repair, averaging 1.5s for $\Delta=1$ to 2.7s for $\Delta=3$, across the entire StackOverflow dataset, while BIFI consistently achieves subsecond latency across all repairs and distances.

%Next, we conduct an ablation study across three decoding strategies to compare their relative effectiveness. In each experiment, we balance the StackOverflow dataset across edit distances and run the candidate sampler for up to 30 seconds. In Alg.~\ref{alg:enum_pcfg}, we use the enumerative sampler and rank all repairs by PCFG score, in Alg.~\ref{alg:enum_ngram}, we use the same approach but rank the repairs by c-gram log-likelihood, and in Alg.~\ref{alg:dfa_walk}, we translate the BH intersection grammar into a DFA then sample trajectories according to a c-gram transition probability, as described in \S~\ref{sec:decoding}. We compare the Precision@1 of each method at recovering the ground truth human repair.
%
%\begin{figure}[h]
%\begin{tikzpicture}[scale=0.9]
  \begin{axis}[
  ybar,
  xlabel={Edit distance, $\Delta(\err\sigma, \sigma)$},
  ylabel={Precision@1},
  title={Enumeration + PCFG (Alg. 1)},
  axis x line*=bottom,
  axis y line*=left,
  ymin=0,
  ymax=1,
  xtick=data,
  width=5cm,
  height=4cm,
  xticklabels={1, 2, 3, 4},
  enlarge x limits=0.15,
  legend style={at={(0.5,-0.15)},
  anchor=north,legend columns=-1},
  ]
  \addplot[fill=black!30] table[x=Lev, y=P@1] {
    Lev P@1
    1 0.41
    2 0.13
    3 0.02
    4 0.00
  };
  \end{axis}
\end{tikzpicture}
%\begin{tikzpicture}[scale=0.9]
  \begin{axis}[
  ybar,
  xlabel={Edit distance, $\Delta(\err\sigma, \sigma)$},
  ylabel={Precision@1},
  title={Enumeration + Markov (Alg. 2)},
  axis x line*=bottom,
  axis y line*=left,
  ymin=0,
  ymax=1,
  xtick=data,
  xticklabels={1, 2, 3, 4},
  width=5cm,
  height=4cm,
  enlarge x limits=0.15,
  legend style={at={(0.5,-0.15)},
  anchor=north,legend columns=-1},
  ]
  \addplot[fill=black!30] table[x=Lev, y=P@1] {
    Lev P@1
    1 0.52
    2 0.25
    3 0.14
    4 0.11
  };
  \end{axis}
\end{tikzpicture}
%\begin{tikzpicture}[scale=0.9]
  \begin{axis}[
  ybar,
  xlabel={Edit distance, $\Delta(\err\sigma, \sigma)$},
  ylabel={Precision@1},
  title={DFA Walker (Alg. 3)},
  axis x line*=bottom,
  axis y line*=left,
  ymin=0,
  ymax=1,
  xtick=data,
  xticklabels={1, 2, 3, 4},
  width=5cm,
  height=4cm,
  enlarge x limits=0.15,
  legend style={at={(0.5,-0.15)},
  anchor=north,legend columns=-1},
  ]
  \addplot[fill=black!30] table[x=Lev, y=P@1] {
    Lev P@1
    1 0.55
    2 0.25
    3 0.13
    4 0.07
  };
  \end{axis}
\end{tikzpicture}

% P@1
%Δ(1)= Top-1/total: 176 / 318 = 0.5534591194968553
%Δ(2)= Top-1/total: 81 / 325 = 0.24923076923076923
%Δ(3)= Top-1/total: 39 / 297 = 0.13131313131313133
%Δ(4)= Top-1/total: 22 / 292 = 0.07534246575342465

% P@All
%Δ(1)= Top-1/total: 147 / 148 = 0.99
%Δ(2)= Top-1/total: 166 / 168 = 0.99
%Δ(3)= Top-1/total: 96 / 150 = 0.64
%Δ(4)= Top-1/total: 26 / 152 = 0.17
%\end{figure}
%
%In general, c-gram likelihood appears to have a significant advantage over PCFG scoring, however this margin may decrease with PCFG models that consider higher-order nonterminal dependencies. Alg.~\ref{alg:enum_pcfg} is efficient, but also the least precise, being a poor model for lexical alignment. Alg.~\ref{alg:enum_ngram} offers competitive precision for Python, but can produce duplicate samples in highly ambiguous CFGs. Alg.~\ref{alg:dfa_walk} has the best performance across all edit distances and languages, but is also the most computationally expensive, requiring a determinization and minimization preprocessing step.

Finally, we evaluate the impact of increased parallelism on repair throughput. We balance the StackOverflow dataset across edit distances and run DFA sampler for up to 30 seconds, then measure the total number of unique valid repairs discovered as a function of the number of additional CPU cores assigned, which we exercise to both construct the intersection grammar and sample from it.

\begin{wrapfigure}{r}{0.4\textwidth}
\begin{tikzpicture}[scale=0.47]
  \begin{axis}[
    ybar,
    enlargelimits=0.15,
    legend style={at={(0.03,0.97)},anchor=north west},
    title={\textbf{Relative Total Distinct Solutions Found vs. Single Core}},
    x=1cm,
    axis lines*=left,
    ylabel={Relative improvement},
    xlabel={Number of assigned cores},
    symbolic x coords={2,3,4,5,6,7,8,9,10},
    xtick=data,
    bar width=4pt,
  ]
    \addplot[green, fill=green!50] coordinates { (2,0.1343) (3,0.1955) (4,0.2249) (5,0.2475) (6,0.2760) (7,0.2994) (8,0.3073) (9,0.3151) (10,0.3229) };
    \addplot[blue, fill=blue!50] coordinates { (2,0.2655) (3,0.4353) (4,0.5614) (5,0.6644) (6,0.7462) (7,0.7783) (8,0.8347) (9,0.8005) (10,0.7798) };
    \addplot[orange, fill=orange!50] coordinates { (2,0.3972) (3,0.6928) (4,0.9327) (5,1.1834) (6,1.3138) (7,1.3988) (8,1.6039) (9,1.5500) (10,1.5691) };
    \addplot[red, fill=red!50] coordinates { (2,0.4863) (3,0.8326) (4,1.1368) (5,1.3879) (6,1.5873) (7,1.7494) (8,1.8802) (9,1.9059) (10,1.9625) };
    \legend{Δ=1, Δ=2, Δ=3, Δ=4}
  \end{axis}
\end{tikzpicture}
\caption{Observed improvement in throughput relative to total CPU cores assigned.}
\label{fig:speedup}
\end{wrapfigure}

We measure the relative improvement in throughput (measured by the number of distinct repairs found after 30s) as a function of the number of additional CPU cores, averaged across 1000 trials. We observe from Fig.~\ref{fig:speedup} the relative throughput increases logarithmically with the number of additional CPU cores, with at least four CPU cores needed to offset the parallelization overhead. Generally, increasing parallelism only helps when the size of the admissible set is large enough to absorb the additional computation, which is seldom the case for small-radii Levenshtein balls.

\clearpage\section{Discussion}\label{sec:discussion}

The main lesson we draw from our experiments is that it is possible to leverage compute to compete with large language models on practical program repair tasks. Though sample-efficient, their size comes at the cost of expensive training, and domain adaptation requires fine-tuning or retraining on pairwise repairs. Our approach uses a small grammar and a relatively cheap ranking metric to achieve significantly higher precision. This allows us to repair errors in languages with little to no training data and provides far more flexibility and controllability during the repair process.

Our primary insight leading to state-of-the-art precision is that repairs are typically concentrated near the center of a small Levenshtein ball, and by enumerating or sampling it carefully, then reranking repairs by naturalness, one can achieve significantly higher precision than one-shot neural repair. This is especially true for small-radii Levenshtein balls, where the admissible set is small enough to be completely enumerated and ranked. For larger radii, we can still achieve competitive precision by using an efficient decoder to sample the admissible set.

There is a clear tradeoff between latency and precision for any repair model. While existing neural syntax repair models scale poorly with additional time, Tidyparse is highly effective at exchanging more time for higher precision. We find that the Precision@1 of our method is competitive with BIFI's Precision@20k, while requiring only a fraction of the data and compute for training and inference. As Tidyparse uses its own grammar, it can sample directly from the formal language specification and does not require a stochastic language model to suggest nearby valid repairs, only to rank them by naturalness. The emphasis on completeness is especially useful for discovering small or contextually unlikely repairs, which may be overlooked by neural models.

Although latency and precision are ultimately the deciding usability factors, repair throughput is a crucial intermediate factor to consider when evaluating the performance of a repair system. Even with a perfectly accurate scoring function, if the correct repair is never retrieved, it will be for naught. By maximizing the total number of unique valid repairs, we increase the probability of retrieving natural repairs to give the scoring function the best chance of ranking them successfully. For this reason, we prioritize throughput heavily in our design and evaluation (Fig.~\ref{fig:throughput}).

\subsection{Limitations and future work}

%  We identify three broad categories of limitations in evaluating Tidyparse and suggest directions for future work: naturalness, complexity, and semantics.

\subsubsection{Naturalness}

Firstly, Tidyparse does not currently support intersections between weighted CFGs and weighted finite automata, a la Pasti et al.~\cite{pasti2023intersection}. This feature would allow us to put transition probabilities on the Levenshtein automaton corresponding to edit likelihood then construct a weighted intersection grammar. With this, one could preemptively discard unlikely productions from $G_\cap$ to reduce the complexity in exchange for relaxed completeness. We also hope to explore more incremental sampling strategies such as sequential Monte-Carlo~\cite{lew2023sequential}.

The scoring function is currently computed over lexical tokens. We expect that a more precise scoring function could be constructed by splicing candidate repairs back into the original source code and then scoring plaintext, however this would require special handling for insertions and substitutions of names, numbers and identifiers that were absent from the original source code. For this reason, we currently perform the scoring in lexical space, which discards a useful signal, but even this coarse approximation is sufficient to achieve state-of-the-art precision.

Furthermore, the scoring function only considers each candidate repair $P_\theta(\sigma')$ in isolation, returning the most plausible candidate independent of the original error. One way to improve this would be to incorporate the broken sequence ($\err\sigma$), parser error message ($m$), original source ($s$), and possibly other contextual priors to inform the scoring function. This would require a more expressive probabilistic language model to faithfully model the joint distribution $P_\theta(\sigma' \mid \err\sigma, m, s, \ldots)$, but would significantly improve the precision of the generated repairs.

\subsubsection{Complexity}

Latency can vary depending on several factors including string length, grammar size, and critically the Levenshtein edit distance. This can be an advantage because, without any contextual or statistical information, syntax and minimal Levenshtein edits are often sufficiently constrained to identify a small number of valid repairs. It is also a limitation because the admissible set expands rapidly with edit distance and the Levenshtein metric diminishes in usefulness without a very precise metric to discriminate natural solutions in the cosmos of equidistant repairs.

Space complexity increases sharply with edit distance and to a lesser extent with length. This can be partly alleviated with more precise criteria to avoid creating superfluous productions, but the memory overhead is still considerable. Memory pressure can be attributed to engineering factors such as the grammar encoding, but is also an inherent challenge of language intersection. Therefore, managing the size of the intersection grammar by preprocessing the syntax and automaton, then eliminating unnecessary synthetic productions is a critical factor in scaling up our technique.

\subsubsection{Toolchain integration}

Lastly and perhaps most significantly, Tidyparse does not incorporate semantic constraints, so its repairs whilst syntactically admissible, are not guaranteed to be type safe. It may be possible to add a type-based semantic refinement to our language intersection, however this would require a more expressive grammatical formalism than CFGs naturally provide.

Program slicing is an important preprocessing consideration that has so far gone unmentioned. The current implementation expects pre-sliced code fragments, however in a more practical scenario, it would be necessary to leverage editor information to identify the boundaries of the repairable fragment. This could be achieved by analyzing historical editor states or via ad hoc slicing techniques.

Additionally, the generated repairs must be spliced back into the surrounding context, which requires careful editor integration. One approach would be to filter all repairs through an incremental compiler or linter, however, the latency necessary to check every repair may be non-negligible.

%We leave this aspect for future work.

We envision a few primary use cases for Tidyparse: (1) helping novice programmers become more quickly familiar with a new programming language, (2) autocorrecting common typos among proficient but forgetful programmers, (3) as a prototyping tool for PL designers and educators, and (4) as a pluggable library or service for parser-generators and language servers.

\section{Related Work}\label{sec:related}

Three important questions arise when repairing syntax errors: (1) is the program broken in the first place? (2) if so, where are the errors located? (3) how should those locations then be altered? Those questions are addressed by three theoretical areas, (1)~parsing, (2)~language equations and (3)~syntax repair. We survey each of those areas, then turn our attention to more engineering-oriented research, including (4) string solving, (5) error-correction, (6) decoding and finally (7) neural program repair.

\subsection{Parsing}

Context-free language (CFL) parsing is the well-studied problem of how to turn a string into a unique tree, with many different algorithms and implementations (e.g., shift-reduce, recursive-descent, LR). Many of those algorithms expect grammars to be expressed in a certain form (e.g., left- or right- recursive) or are optimized for a narrow class of grammars (e.g., regular, linear).

General CFL parsing allows ambiguity (non-unique trees) and can be formulated as a dynamic programming problem, as shown by Cocke-Younger-Kasami (CYK)~\cite{sakai1961syntax}, Earley~\cite{earley1970efficient} and others. These parsers have roughly cubic complexity with respect to the length of the input string.

As shown by Valiant~\cite{valiant1975general}, Lee~\cite{lee2002fast} and others, general CFL recognition is in some sense equivalent to binary matrix multiplication, another well-studied combinatorial problem with broad applications, known to be at worst subcubic. This reduction opens the door to a range of complexity-theoretic speedups to CFL recognition, however large constants tend to limit their practical utility.

%  Okhotin~\cite{okhotin2001conjunctive} extends CFGs with language conjunction in \textit{conjunctive grammars}, followed by Zhang \& Su~\cite{zhang2017context} who apply conjunctive language reachability to dataflow analysis.

From a more applied perspective, parsers are ubiquitous in present-day software engineering, but none are designed to handle arbitrary CFGs or recover from arbitrary errors. Parr and Quong introduce ANTLR~\cite{parr1995antlr} which can handle LL(k) grammars and offers an IDE plugin with limited support for error recovery. Scott and Johnstone~\cite{scott2010gll} introduce GLL parsing, which supports linear-time parsing for LL grammars and cubic for arbitrary CFGs, but does not support error correction. Inspired by their work, we introduce a method for repairing small syntax errors in arbitrary CFLs.

\subsection{Language equations}

Language equations are a powerful tool for reasoning about formal languages and their inhabitants. First proposed by Ginsburg et al.~\cite{ginsburg1962two} for the ALGOL language, language equations are essentially systems of inequalities with variables representing \textit{holes}, i.e., unknown values, in the language or grammar. Solutions to these equations can be obtained using various fixpoint techniques, yielding members of the language. This insight reveals the true algebraic nature of CFLs and their cousins.

Being an algebraic formalism, language equations naturally give rise to a kind of calculus, vaguely reminiscent of Leibniz' and Newton's. First studied by Brzozowski~\cite{brzozowski1964derivatives, brzozowski1980equations} and Antimirov~\cite{antimirov1996partial}, one can take the derivative of a language equation, which can be interpreted as a kind of continuation or language quotient, revealing the suffixes that complete a given prefix. This technique leads to an elegant family of algorithms for incremental parsing~\cite{might2011parsing, adams2016complexity} and automata minimization~\cite{brzozowski1962canonical}.

Bar-Hillel~\cite{bar1961formal} establishes the closure of CFLs under intersection with regular languages, but does not elaborate on how to construct the corresponding grammar in order to recognize it. Beigel~\cite{beigelproof} and Pasti et al.~\cite{pasti2023intersection} provide helpful insights into the construction of the intersection grammar, and Nederhof and Satta~\cite{nederhof2004language} specifically consider finite CFL intersections, but neither considers Levenshtein intersections. Our work specializes Bar-Hillel intersections to Levenshtein automata in particular, and more generally acyclic automata using a refinement of Salomaa's construction~\cite{salomaa1973formal}.

More concretely, we restrict our attention to language equations over CFLs whose variables coincide with edit locations in the source code of a computer program, and solutions correspond to syntax repairs. While prior work has studied the use of language equations for parsing~\cite{might2011parsing}, to our knowledge they were never specifically applied to code completion or syntax error correction.

\subsection{Syntax repair}

In finite languages, syntax repair corresponds to spelling correction, a more restrictive and largely solved problem. Schulz and Stoyan~\cite{schulz2002fast} construct a finite automaton that returns the nearest dictionary entry by Levenshtein edit distance. Though considerably simpler than syntax correction, their work shares similar challenges and offers insights for handling more general repair scenarios.

When a sentence is grammatically invalid, parsing grows more challenging. Like spelling, the problem is to find the minimum number of edits required to transform an arbitrary string into a syntactically valid one, where validity is defined as containment in a (typically) context-free language. Early work, including Irons~\cite{irons1963error} and Aho~\cite{aho1972minimum} propose a dynamic programming algorithm to compute the minimum number of edits required to fix an invalid string. Prior work on error correcting parsing only considers the nearest edit(s), and does not study edits of varying distance in the Levenshtein ball. Furthermore, the problem of repair is not generally well-posed, as there can be many valid solutions. We instead focus on maximum likelihood Levenshtein-CFL reachability, which attempts to find the most natural repair within a fixed Levenshtein distance.

%Bar-Hillel~\cite{bar1961formal} establishes the closure of CFLs under intersection with regular languages, and can be used to construct the corresponding minimum-cost CFG in a straightforward manner. However while generally quite expressive, CFLs are themselves not closed under intersection and have other practical limitations, i.e., are unable to express indentation or variable binding. These limitations motivate us towards more expressive yet still efficiently parsable formalisms.

%Conveniently, Okhotin~\cite{okhotin2001conjunctive} offers exactly the formalism needed: language equations augmented with logical operators like conjunction or disjunction. These operators afford us the flexibility to encode both language union and intersection, which are difficult or impossible to express using a single grammar, as well as incorporate extra-syntactic side-constraints during solving.

%Naively, one might think to enumerate all edits within a certain Levenshtein radius and reject invalid strings. However, this approach is intractable for long strings. Another approach is to compute the \textit{edit distance} between the string and the language as proposed by Aho. This approach is also intractable depending on the size of the grammar. A third approach is to use a language equation to compute the intersection between the Levenshtein hypersphere and the language.

%The literature on parsers is vast and deep, covering far more ground than we could possibly hope to survey. We take inspiration from their findings, but restrict ourselves to a dozen most closely related papers, in four major research areas: (1) formal language theory, (2) constraint satisfaction (3) program synthesis, and (4) error correction. We survey each of these areas in turn.
%
%It was Noam Chomsky who first developed the algebraic theory of \textit{context-free grammars} (CFGs) in 1959~\cite{chomsky1959algebraic}, and since then, CFGs have been the subject of extensive research in formal language theory and program analysis. In particular, we take inspiration from Leslie Valiant~\cite{valiant1975general} who first discovered the connection to matrix multiplication in 1975 and Alexander Okhotin~\cite{okhotin2001conjunctive} who later introduced the idea of \textit{conjunctive grammars} in 2001. More recently, Azimov \& Grigorev~\cite{azimov2018cfpq} and Qirun Zhang shed light into the practical applicability of these ideas and made the theory of CFG reachability more accessible to the general public. In our work, we show how to compile their ideas onto a SAT solver to fix syntax errors, which seems like a perfectly natural extension, but was heretofore previously never considered to the best of our knowledge.

\subsection{String solving}

There is related work on string constraints in the constraint programing literature, featuring solvers like CFGAnalyzer and HAMPI~\cite{kiezun2009hampi}, which consider bounded context free grammars and intersections thereof. Boja{\'n}czyk et al. (2014)~\cite{bojanczyk2014automata} introduce the theory of nominal automata. Around the same time, D'Antoni et al. (2014) introduce \textit{symbolic automata}~\cite{dantoni2014minimization}, a generalization of finite automata which allow infinite alphabets and symbolic expressions over them. Hague et al. (2024)~\cite{hague2024parikh} use Parikh's theorem in the context of symbolic automata to speed up string constraint solving, from which we draw partial inspiration for the Levenshtein-Bar-Hillel construction in \S~\ref{sec:lev_bh}. In none of the constraint programming literature we surveyed do any of the approaches specifically consider the problem of syntax error correction, which is the main focus of our work.

%\subsection{Error correcting codes}
%
%Our work focuses on errors arising from human factors in computer programming, in particular \textit{syntax error correction}, which is the problem of fixing partially corrupted programs. Modern research on error correction, however, can be traced back to the early days of coding theory when researchers designed \textit{error-correcting codes} (ECCs) to denoise transmission errors induced by external interference, e.g., collision with a high-energy proton, manipulation by an adversary or even typographical mistake. In this context, \textit{code} can be any logical representation for communicating information between two parties (such as a human and a computer), and an ECC is a carefully-designed scheme which ensures that even if some portion of the message should become corrupted, one can still recover the original message by solving a linear system of equations. When designing ECCs, one typically assumes a noise model over a certain sample space, such as the Hamming~\cite{titsias2017hamming, dong2023number} or Levenshtein~\cite{levenshtein1966binary, becerra2008learning, barlev2021levenshtein} balls, from which we draw inspiration for this work.

\subsection{Decoding}

Decoding is a key problem in machine translation, speech recognition, and other sequence-to-sequence tasks. Given a compressed encoding of some finite distribution, the goal is find the maximum likelihood samples. A classic example is Viterbi decoding, which is used to find the most likely sequence of transitions in a hidden Markov model (HMM) and is closely related to probabilistic parsing. For PCFGs, the problem is more challenging, as the solution space can be exponentially larger than HMMs relative to the number of transitions.

In particular, we care about the problem of \textit{top-k decoding}, which attempts to find the exact or approximate $k$-most likely samples in order of decreasing likelihood. This is closely related to the $k$-best enumeration~\cite{eppstein2014k} problem, a carefully studied problem in graph theory and combinatorial optimization. An exact solution to this problem for large acyclic PCFGs is often intractable, but we can approximate it using a beam search or cube-pruning technique.

A popular solution to k-best decoding in the NLP literature is a technique called cube-pruning~\cite{huang2005better, huang2007forest, chiang2007hierarchical}, which samples maximum likelihood paths through a hypergraph. We take inspiration from this technique, and adapt it to the setting of constrained decoding from finite CFGs. Our approach is also complementary to work by Zhang and McDonald~\cite{zhang2012generalized}, but specialized to language intersections.

An alternate line of work originates from combinatorics~\cite{hickey1983uniform, gore1997quasi} and Boltzmann sampling~\cite{duchon2004boltzmann}, which constructs a generating function for the language and samples it uniformly. Unlike our method, distinctness or convergence guarantees for arbitrary finite CFLs are not provided.

Another approach would be to use MCMC or sequential Monte Carlo (SMC) to steer a transformer-based LLM, as proposed by Lew et al.~\cite{lew2023sequential}. This technique shows promise for constrained sampling from LLMs, and could be adapted to improve sample efficiency. The downside is that distinctly sampling an LLM is unclear how to do properly, being a fundamentally non-Markovian process. One solution proposed by Shi and Bieber~\cite{shi2020incremental} assumes trace injectivity and constructs a trie, however their solution is not stateless and can introduce a significant latency overhead.

%Our approach is complementary to existing work in constrained decoding. The bijection proposed in Eq.~\ref{eq:pairing} guarantees that all repairs are well-formed and converge linearly to the exact top-k maximum likelihood samples. This method is completely stateless and can be used to enumerate a bounded Levenshtein ball with linear parallelization speedup. Alternately, in the case of approximate ranked repair over a very large sample space, this technique can be adapted to sample with high probability a representative subset of the most likely sentences in a finite but large PCFG.

\subsection{Neural program repair}

More recently, probabilistic repair techniques have been introduced using neural models to predict the most likely correction~\cite{allamanis2021self, chirkova2021empirical, drain2021generating}. These approaches typically employ large language models (LLMs) and treat the problem as a sequence-to-sequence transformation. While capable of generating natural repairs, these models are susceptible to misgeneralization, costly to train, and challenging to customize thereafter. Furthermore, the generated repairs are not necessarily sound without additional filtering, and we observe the released models often hallucinate false positive repairs.

In particular, two papers stand out being closely related to our own: Break-It-Fix-It (BIFI)~\cite{yasunaga2021break} and Seq2Parse~\cite{sakkas2022seq2parse}. BIFI adapts techniques from semi-supervised learning to generate synthetic errors in clean code and fixes them. This reduces the need for pairwise training data, but tends to generalize poorly to lengthy or out-of-distribution repairs. Seq2Parse combines a transformer-based model with an augmented version of the Early parser to suggest error rules, but only suggests a single repair. Our work differs from both in that we suggest multiple repairs at much higher precision, do not require a pairwise repair dataset, and can fix syntax errors in any language with a well-defined grammar. We note our approach is complementary to existing work in neural program repair, and may be used to generate synthetic repairs for training or employ an LLM for ranking.

Recent work by Merrill et al.~\cite{merrill2022saturated} and Chiang et al.~\cite{chiang2023tighter} suggest that the issue with generalization may be more foundational: transformer-based language models, a popular class of neural language models used in probabilistic program repair, are fundamentally less expressive than context-free grammars, which formally describe the syntax of most programming languages. This suggests such models, despite their useful approximation properties, are ill-suited for the task of end-to-end syntax repair. Yet, they may still be useful for resolving ambiguity between valid repairs of differing likelihood or searching a large sample space for the most likely repair.

\section{Conclusion}\label{sec:conclusion}

Our work, while a case study on syntax repair, is part of a broader line of inquiry in program synthesis that investigates how to weave formal language theory and machine learning into helpful programming tools for everyday developers. In some ways, syntax repair serves as a test bench for integrating learning and language theory, as it lacks the intricacies of type-checking and semantic analysis, but is still rich enough to be an interesting challenge. By starting with syntax repair, we hope to lay the foundation for more organic hybrid approaches to program synthesis.

Two high-level codesign patterns have emerged to combine the naturalness of neural language models with the precision of formal methods. One seeks to filter the outputs of a generative language model to satisfy a formal specification, typically by some form of rejection sampling. Alternatively, some attempt to use language models to steer an incremental search for valid programs via a reinforcement learning or hybrid neurosymbolic approach. However, implementing these strategies is often painstaking and their generalization behavior can be difficult to analyze.

In our work, we take a more pragmatic tack - by incorporating the distance metric into a formal language, we attempt to exhaustively enumerate repairs by increasing distance, then use the stochastic language model to sort the resulting solutions by naturalness. The more constraints we can incorporate into formal language, the more efficient sampling becomes, and the more precise control we have over the output. This reduces the need for training a large, expensive language model to relearn syntax, and allows us to leverage compute for more efficient search and ranking.

%  The great compromise in program synthesis is that of efficiency versus expressiveness. The more expressive a language, the more concise and varied the programs it can represent, but the harder those programs are to synthesize without resorting to domain-specific heuristics. Likewise, the simpler a language is to synthesize, the weaker its concision and expressive power. A large body of work focuses on general $\lambda$-calculi, or narrow languages such as finite sets or regular expressions. The former are too expressive to be efficiently synthesized or verified, whilst the latter are too restrictive to be useful for syntax repair. %Our work, we focus on context-free languages, which are expressive enough to capture a variety of practical synthesis tasks while retaining the benefits of compositionality and reusability across domains.

There is a delicate balance in formal methods between soundness and completeness. Often these two seem at odds because the target language is too expressive to achieve them both simultaneously. In syntax repair, we also care about \textit{naturalness}. Fortunately, syntax repair is tractable enough to achieve all three by modeling the problem using language intersection. Completeness helps us to avoid missing simple repairs that might be easily overlooked, soundness guarantees all repairs will be valid, and naturalness ensures the most likely repairs receive the highest priority.

%  The second great compromise in program synthesis is that of reusability versus specialization. In programming, as in human communications, there is a vast constellation of languages, each requiring specialized generators and interpreters. Are these languages truly irreconcilable? Or, as Noam Chomsky argues, are these merely dialects of a universal language? \textit{Synthesis} then, might be a misnomer, and more aptly called \textit{recognition}, in the analytic tradition.

%  In our work, we argue these two compromises are not mutually exclusive, but complementary and reciprocal. Programs and the languages they inhabit are indeed synthetic, but can be analyzed and reused in the metalanguage of algebraic language theory. Not only does this admit an efficient synthesis algorithm, but allows users to introduce additional constraints without breaking compositionality, one of the most sacred tenets in programming language design.

%  Over the last few years, there has been a surge of progress in applying language models to write programs. That work is primarily based on methods from differential calculus and continuous optimization, leading to the so-called \textit{naturalness hypothesis}, which suggests programming languages are not so different from natural ones. In contrast, programming language theory takes the view that languages are essentially discrete and finitely-generated sets governed by logical calculi. Programming, thus viewed, is more like a mathematical exercise in constraint satisfaction. These constraints naturally arise at various stages of syntax validation, type-checking and runtime verification, and help to ensure programs fulfill their intended purpose.

%  As our work shows, not only is linear algebra over finite fields an expressive language for probabilistic inference, but also an efficient framework for inference on languages themselves. Borrowing analysis techniques from multilinear algebra and tensor completion in the machine learning setting, we develop an equational theory that allows us to translate various decision problems on formal languages into a system of inequalities over finite fields. We demonstrate the effectiveness of our approach for syntax repair in context-free languages, and show that our approach is competitive with state-of-the-art methods in terms of both accuracy and efficiency. In future work, we hope to extend our method to more natural grammars like conjunctive languages, TAG, LCFRS and other mildly context-sensitive languages.

%From a usability standpoint, syntax repair tools should be as user-friendly and widely accessible as autocorrection tools in word processors. We argue it is possible to reduce disruption from manual syntax repair and improve the efficiency of working programmers by driving down the latency needed to synthesize an acceptable repair. In contrast with program synthesizers that require intermediate editor states to be well-formed, our synthesizer does not impose any constraints on the code itself being written and is possible to use in an interactive programming setting.

%  The design of the tool itself is relatively simple. Tidyparse accepts a context-free language and a string. If the string is valid, it returns the parse forest, otherwise, it returns a set of repairs, ordered by likelihood. This approach has many advantages, enabling us to repair broken syntax, correct typos and recover from small errors, while being provably sound and complete with respect to the grammatical specification and a Levenshtein bound. It is also compatible with neural program synthesis and repair techniques, which can be used to score and rank the generated repairs.

We have implemented our approach and demonstrated its viability as a tool for syntax assistance in real-world programming languages. Tidyparse is capable of generating repairs for invalid source code in a range of practical languages with little to no data required. We plan to continue expanding the prototype's autocorrection functionality to cover a broader range of languages and hope to conduct a more thorough user study to validate its effectiveness in practical programming scenarios.

\section*{Data-Availability Statement}

An artifact for Tidyparse is currently available as a browser application~\footnote{\url{https://tidyparse.github.io}}. The data and source code for the experiments contained in this paper will be made available upon publication.

\clearpage\bibliography{../bib/acmart}

\pagebreak\appendix

\section{Levenshtein Automata Matrices}

These are useful for visually checking different implementations.

\begin{figure}[H]
  \begin{center}
    \resizebox{0.45\textwidth}{!}{%
      \begin{tikzpicture}[x=0.3cm, y=0.3cm, draw=gray, very thin]
  \path[fill=white] (0,13) rectangle ++(1,1);
  \path[fill=black] (1,13) rectangle ++(1,1);
  \path[fill=black] (2,13) rectangle ++(1,1);
  \path[fill=black] (3,13) rectangle ++(1,1);
  \path[fill=white] (4,13) rectangle ++(1,1);
  \path[fill=black] (5,13) rectangle ++(1,1);
  \path[fill=white] (6,13) rectangle ++(1,1);
  \path[fill=white] (7,13) rectangle ++(1,1);
  \path[fill=white] (8,13) rectangle ++(1,1);
  \path[fill=white] (9,13) rectangle ++(1,1);
  \path[fill=white] (10,13) rectangle ++(1,1);
  \path[fill=white] (11,13) rectangle ++(1,1);
  \path[fill=white] (12,13) rectangle ++(1,1);
  \path[fill=white] (13,13) rectangle ++(1,1);
  \path[fill=white] (0,12) rectangle ++(1,1);
  \path[fill=white] (1,12) rectangle ++(1,1);
  \path[fill=white] (2,12) rectangle ++(1,1);
  \path[fill=black] (3,12) rectangle ++(1,1);
  \path[fill=white] (4,12) rectangle ++(1,1);
  \path[fill=white] (5,12) rectangle ++(1,1);
  \path[fill=white] (6,12) rectangle ++(1,1);
  \path[fill=white] (7,12) rectangle ++(1,1);
  \path[fill=white] (8,12) rectangle ++(1,1);
  \path[fill=white] (9,12) rectangle ++(1,1);
  \path[fill=white] (10,12) rectangle ++(1,1);
  \path[fill=white] (11,12) rectangle ++(1,1);
  \path[fill=white] (12,12) rectangle ++(1,1);
  \path[fill=white] (13,12) rectangle ++(1,1);
  \path[fill=white] (0,11) rectangle ++(1,1);
  \path[fill=white] (1,11) rectangle ++(1,1);
  \path[fill=white] (2,11) rectangle ++(1,1);
  \path[fill=black] (3,11) rectangle ++(1,1);
  \path[fill=black] (4,11) rectangle ++(1,1);
  \path[fill=black] (5,11) rectangle ++(1,1);
  \path[fill=white] (6,11) rectangle ++(1,1);
  \path[fill=black] (7,11) rectangle ++(1,1);
  \path[fill=white] (8,11) rectangle ++(1,1);
  \path[fill=white] (9,11) rectangle ++(1,1);
  \path[fill=white] (10,11) rectangle ++(1,1);
  \path[fill=white] (11,11) rectangle ++(1,1);
  \path[fill=white] (12,11) rectangle ++(1,1);
  \path[fill=white] (13,11) rectangle ++(1,1);
  \path[fill=white] (0,10) rectangle ++(1,1);
  \path[fill=white] (1,10) rectangle ++(1,1);
  \path[fill=white] (2,10) rectangle ++(1,1);
  \path[fill=white] (3,10) rectangle ++(1,1);
  \path[fill=white] (4,10) rectangle ++(1,1);
  \path[fill=black] (5,10) rectangle ++(1,1);
  \path[fill=white] (6,10) rectangle ++(1,1);
  \path[fill=white] (7,10) rectangle ++(1,1);
  \path[fill=white] (8,10) rectangle ++(1,1);
  \path[fill=white] (9,10) rectangle ++(1,1);
  \path[fill=white] (10,10) rectangle ++(1,1);
  \path[fill=white] (11,10) rectangle ++(1,1);
  \path[fill=white] (12,10) rectangle ++(1,1);
  \path[fill=white] (13,10) rectangle ++(1,1);
  \path[fill=white] (0,9) rectangle ++(1,1);
  \path[fill=white] (1,9) rectangle ++(1,1);
  \path[fill=white] (2,9) rectangle ++(1,1);
  \path[fill=white] (3,9) rectangle ++(1,1);
  \path[fill=white] (4,9) rectangle ++(1,1);
  \path[fill=black] (5,9) rectangle ++(1,1);
  \path[fill=black] (6,9) rectangle ++(1,1);
  \path[fill=black] (7,9) rectangle ++(1,1);
  \path[fill=white] (8,9) rectangle ++(1,1);
  \path[fill=black] (9,9) rectangle ++(1,1);
  \path[fill=white] (10,9) rectangle ++(1,1);
  \path[fill=white] (11,9) rectangle ++(1,1);
  \path[fill=white] (12,9) rectangle ++(1,1);
  \path[fill=white] (13,9) rectangle ++(1,1);
  \path[fill=white] (0,8) rectangle ++(1,1);
  \path[fill=white] (1,8) rectangle ++(1,1);
  \path[fill=white] (2,8) rectangle ++(1,1);
  \path[fill=white] (3,8) rectangle ++(1,1);
  \path[fill=white] (4,8) rectangle ++(1,1);
  \path[fill=white] (5,8) rectangle ++(1,1);
  \path[fill=white] (6,8) rectangle ++(1,1);
  \path[fill=black] (7,8) rectangle ++(1,1);
  \path[fill=white] (8,8) rectangle ++(1,1);
  \path[fill=white] (9,8) rectangle ++(1,1);
  \path[fill=white] (10,8) rectangle ++(1,1);
  \path[fill=white] (11,8) rectangle ++(1,1);
  \path[fill=white] (12,8) rectangle ++(1,1);
  \path[fill=white] (13,8) rectangle ++(1,1);
  \path[fill=white] (0,7) rectangle ++(1,1);
  \path[fill=white] (1,7) rectangle ++(1,1);
  \path[fill=white] (2,7) rectangle ++(1,1);
  \path[fill=white] (3,7) rectangle ++(1,1);
  \path[fill=white] (4,7) rectangle ++(1,1);
  \path[fill=white] (5,7) rectangle ++(1,1);
  \path[fill=white] (6,7) rectangle ++(1,1);
  \path[fill=black] (7,7) rectangle ++(1,1);
  \path[fill=black] (8,7) rectangle ++(1,1);
  \path[fill=black] (9,7) rectangle ++(1,1);
  \path[fill=white] (10,7) rectangle ++(1,1);
  \path[fill=black] (11,7) rectangle ++(1,1);
  \path[fill=white] (12,7) rectangle ++(1,1);
  \path[fill=white] (13,7) rectangle ++(1,1);
  \path[fill=white] (0,6) rectangle ++(1,1);
  \path[fill=white] (1,6) rectangle ++(1,1);
  \path[fill=white] (2,6) rectangle ++(1,1);
  \path[fill=white] (3,6) rectangle ++(1,1);
  \path[fill=white] (4,6) rectangle ++(1,1);
  \path[fill=white] (5,6) rectangle ++(1,1);
  \path[fill=white] (6,6) rectangle ++(1,1);
  \path[fill=white] (7,6) rectangle ++(1,1);
  \path[fill=white] (8,6) rectangle ++(1,1);
  \path[fill=black] (9,6) rectangle ++(1,1);
  \path[fill=white] (10,6) rectangle ++(1,1);
  \path[fill=white] (11,6) rectangle ++(1,1);
  \path[fill=white] (12,6) rectangle ++(1,1);
  \path[fill=white] (13,6) rectangle ++(1,1);
  \path[fill=white] (0,5) rectangle ++(1,1);
  \path[fill=white] (1,5) rectangle ++(1,1);
  \path[fill=white] (2,5) rectangle ++(1,1);
  \path[fill=white] (3,5) rectangle ++(1,1);
  \path[fill=white] (4,5) rectangle ++(1,1);
  \path[fill=white] (5,5) rectangle ++(1,1);
  \path[fill=white] (6,5) rectangle ++(1,1);
  \path[fill=white] (7,5) rectangle ++(1,1);
  \path[fill=white] (8,5) rectangle ++(1,1);
  \path[fill=black] (9,5) rectangle ++(1,1);
  \path[fill=black] (10,5) rectangle ++(1,1);
  \path[fill=black] (11,5) rectangle ++(1,1);
  \path[fill=white] (12,5) rectangle ++(1,1);
  \path[fill=black] (13,5) rectangle ++(1,1);
  \path[fill=white] (0,4) rectangle ++(1,1);
  \path[fill=white] (1,4) rectangle ++(1,1);
  \path[fill=white] (2,4) rectangle ++(1,1);
  \path[fill=white] (3,4) rectangle ++(1,1);
  \path[fill=white] (4,4) rectangle ++(1,1);
  \path[fill=white] (5,4) rectangle ++(1,1);
  \path[fill=white] (6,4) rectangle ++(1,1);
  \path[fill=white] (7,4) rectangle ++(1,1);
  \path[fill=white] (8,4) rectangle ++(1,1);
  \path[fill=white] (9,4) rectangle ++(1,1);
  \path[fill=white] (10,4) rectangle ++(1,1);
  \path[fill=black] (11,4) rectangle ++(1,1);
  \path[fill=white] (12,4) rectangle ++(1,1);
  \path[fill=white] (13,4) rectangle ++(1,1);
  \path[fill=white] (0,3) rectangle ++(1,1);
  \path[fill=white] (1,3) rectangle ++(1,1);
  \path[fill=white] (2,3) rectangle ++(1,1);
  \path[fill=white] (3,3) rectangle ++(1,1);
  \path[fill=white] (4,3) rectangle ++(1,1);
  \path[fill=white] (5,3) rectangle ++(1,1);
  \path[fill=white] (6,3) rectangle ++(1,1);
  \path[fill=white] (7,3) rectangle ++(1,1);
  \path[fill=white] (8,3) rectangle ++(1,1);
  \path[fill=white] (9,3) rectangle ++(1,1);
  \path[fill=white] (10,3) rectangle ++(1,1);
  \path[fill=black] (11,3) rectangle ++(1,1);
  \path[fill=black] (12,3) rectangle ++(1,1);
  \path[fill=black] (13,3) rectangle ++(1,1);
  \path[fill=white] (0,2) rectangle ++(1,1);
  \path[fill=white] (1,2) rectangle ++(1,1);
  \path[fill=white] (2,2) rectangle ++(1,1);
  \path[fill=white] (3,2) rectangle ++(1,1);
  \path[fill=white] (4,2) rectangle ++(1,1);
  \path[fill=white] (5,2) rectangle ++(1,1);
  \path[fill=white] (6,2) rectangle ++(1,1);
  \path[fill=white] (7,2) rectangle ++(1,1);
  \path[fill=white] (8,2) rectangle ++(1,1);
  \path[fill=white] (9,2) rectangle ++(1,1);
  \path[fill=white] (10,2) rectangle ++(1,1);
  \path[fill=white] (11,2) rectangle ++(1,1);
  \path[fill=white] (12,2) rectangle ++(1,1);
  \path[fill=black] (13,2) rectangle ++(1,1);
  \path[fill=white] (0,1) rectangle ++(1,1);
  \path[fill=white] (1,1) rectangle ++(1,1);
  \path[fill=white] (2,1) rectangle ++(1,1);
  \path[fill=white] (3,1) rectangle ++(1,1);
  \path[fill=white] (4,1) rectangle ++(1,1);
  \path[fill=white] (5,1) rectangle ++(1,1);
  \path[fill=white] (6,1) rectangle ++(1,1);
  \path[fill=white] (7,1) rectangle ++(1,1);
  \path[fill=white] (8,1) rectangle ++(1,1);
  \path[fill=white] (9,1) rectangle ++(1,1);
  \path[fill=white] (10,1) rectangle ++(1,1);
  \path[fill=white] (11,1) rectangle ++(1,1);
  \path[fill=white] (12,1) rectangle ++(1,1);
  \path[fill=black] (13,1) rectangle ++(1,1);
  \path[fill=white] (0,0) rectangle ++(1,1);
  \path[fill=white] (1,0) rectangle ++(1,1);
  \path[fill=white] (2,0) rectangle ++(1,1);
  \path[fill=white] (3,0) rectangle ++(1,1);
  \path[fill=white] (4,0) rectangle ++(1,1);
  \path[fill=white] (5,0) rectangle ++(1,1);
  \path[fill=white] (6,0) rectangle ++(1,1);
  \path[fill=white] (7,0) rectangle ++(1,1);
  \path[fill=white] (8,0) rectangle ++(1,1);
  \path[fill=white] (9,0) rectangle ++(1,1);
  \path[fill=white] (10,0) rectangle ++(1,1);
  \path[fill=white] (11,0) rectangle ++(1,1);
  \path[fill=white] (12,0) rectangle ++(1,1);
  \path[fill=white] (13,0) rectangle ++(1,1);
\end{tikzpicture}

\begin{tikzpicture}[x=0.3cm, y=0.3cm, draw=gray, very thin]
  \path[fill=white] (0,13) rectangle ++(1,1);
  \path[fill=black] (1,13) rectangle ++(1,1);
  \path[fill=black] (2,13) rectangle ++(1,1);
  \path[fill=black] (3,13) rectangle ++(1,1);
  \path[fill=black] (4,13) rectangle ++(1,1);
  \path[fill=black] (5,13) rectangle ++(1,1);
  \path[fill=black] (6,13) rectangle ++(1,1);
  \path[fill=black] (7,13) rectangle ++(1,1);
  \path[fill=black] (8,13) rectangle ++(1,1);
  \path[fill=black] (9,13) rectangle ++(1,1);
  \path[fill=black] (10,13) rectangle ++(1,1);
  \path[fill=black] (11,13) rectangle ++(1,1);
  \path[fill=black] (12,13) rectangle ++(1,1);
  \path[fill=black] (13,13) rectangle ++(1,1);
  \path[fill=white] (0,12) rectangle ++(1,1);
  \path[fill=white] (1,12) rectangle ++(1,1);
  \path[fill=white] (2,12) rectangle ++(1,1);
  \path[fill=black] (3,12) rectangle ++(1,1);
  \path[fill=white] (4,12) rectangle ++(1,1);
  \path[fill=black] (5,12) rectangle ++(1,1);
  \path[fill=white] (6,12) rectangle ++(1,1);
  \path[fill=black] (7,12) rectangle ++(1,1);
  \path[fill=white] (8,12) rectangle ++(1,1);
  \path[fill=black] (9,12) rectangle ++(1,1);
  \path[fill=white] (10,12) rectangle ++(1,1);
  \path[fill=black] (11,12) rectangle ++(1,1);
  \path[fill=white] (12,12) rectangle ++(1,1);
  \path[fill=black] (13,12) rectangle ++(1,1);
  \path[fill=white] (0,11) rectangle ++(1,1);
  \path[fill=white] (1,11) rectangle ++(1,1);
  \path[fill=white] (2,11) rectangle ++(1,1);
  \path[fill=black] (3,11) rectangle ++(1,1);
  \path[fill=black] (4,11) rectangle ++(1,1);
  \path[fill=black] (5,11) rectangle ++(1,1);
  \path[fill=black] (6,11) rectangle ++(1,1);
  \path[fill=black] (7,11) rectangle ++(1,1);
  \path[fill=black] (8,11) rectangle ++(1,1);
  \path[fill=black] (9,11) rectangle ++(1,1);
  \path[fill=black] (10,11) rectangle ++(1,1);
  \path[fill=black] (11,11) rectangle ++(1,1);
  \path[fill=black] (12,11) rectangle ++(1,1);
  \path[fill=black] (13,11) rectangle ++(1,1);
  \path[fill=white] (0,10) rectangle ++(1,1);
  \path[fill=white] (1,10) rectangle ++(1,1);
  \path[fill=white] (2,10) rectangle ++(1,1);
  \path[fill=white] (3,10) rectangle ++(1,1);
  \path[fill=white] (4,10) rectangle ++(1,1);
  \path[fill=black] (5,10) rectangle ++(1,1);
  \path[fill=white] (6,10) rectangle ++(1,1);
  \path[fill=black] (7,10) rectangle ++(1,1);
  \path[fill=white] (8,10) rectangle ++(1,1);
  \path[fill=black] (9,10) rectangle ++(1,1);
  \path[fill=white] (10,10) rectangle ++(1,1);
  \path[fill=black] (11,10) rectangle ++(1,1);
  \path[fill=white] (12,10) rectangle ++(1,1);
  \path[fill=black] (13,10) rectangle ++(1,1);
  \path[fill=white] (0,9) rectangle ++(1,1);
  \path[fill=white] (1,9) rectangle ++(1,1);
  \path[fill=white] (2,9) rectangle ++(1,1);
  \path[fill=white] (3,9) rectangle ++(1,1);
  \path[fill=white] (4,9) rectangle ++(1,1);
  \path[fill=black] (5,9) rectangle ++(1,1);
  \path[fill=black] (6,9) rectangle ++(1,1);
  \path[fill=black] (7,9) rectangle ++(1,1);
  \path[fill=black] (8,9) rectangle ++(1,1);
  \path[fill=black] (9,9) rectangle ++(1,1);
  \path[fill=black] (10,9) rectangle ++(1,1);
  \path[fill=black] (11,9) rectangle ++(1,1);
  \path[fill=black] (12,9) rectangle ++(1,1);
  \path[fill=black] (13,9) rectangle ++(1,1);
  \path[fill=white] (0,8) rectangle ++(1,1);
  \path[fill=white] (1,8) rectangle ++(1,1);
  \path[fill=white] (2,8) rectangle ++(1,1);
  \path[fill=white] (3,8) rectangle ++(1,1);
  \path[fill=white] (4,8) rectangle ++(1,1);
  \path[fill=white] (5,8) rectangle ++(1,1);
  \path[fill=white] (6,8) rectangle ++(1,1);
  \path[fill=black] (7,8) rectangle ++(1,1);
  \path[fill=white] (8,8) rectangle ++(1,1);
  \path[fill=black] (9,8) rectangle ++(1,1);
  \path[fill=white] (10,8) rectangle ++(1,1);
  \path[fill=black] (11,8) rectangle ++(1,1);
  \path[fill=white] (12,8) rectangle ++(1,1);
  \path[fill=black] (13,8) rectangle ++(1,1);
  \path[fill=white] (0,7) rectangle ++(1,1);
  \path[fill=white] (1,7) rectangle ++(1,1);
  \path[fill=white] (2,7) rectangle ++(1,1);
  \path[fill=white] (3,7) rectangle ++(1,1);
  \path[fill=white] (4,7) rectangle ++(1,1);
  \path[fill=white] (5,7) rectangle ++(1,1);
  \path[fill=white] (6,7) rectangle ++(1,1);
  \path[fill=black] (7,7) rectangle ++(1,1);
  \path[fill=black] (8,7) rectangle ++(1,1);
  \path[fill=black] (9,7) rectangle ++(1,1);
  \path[fill=black] (10,7) rectangle ++(1,1);
  \path[fill=black] (11,7) rectangle ++(1,1);
  \path[fill=black] (12,7) rectangle ++(1,1);
  \path[fill=black] (13,7) rectangle ++(1,1);
  \path[fill=white] (0,6) rectangle ++(1,1);
  \path[fill=white] (1,6) rectangle ++(1,1);
  \path[fill=white] (2,6) rectangle ++(1,1);
  \path[fill=white] (3,6) rectangle ++(1,1);
  \path[fill=white] (4,6) rectangle ++(1,1);
  \path[fill=white] (5,6) rectangle ++(1,1);
  \path[fill=white] (6,6) rectangle ++(1,1);
  \path[fill=white] (7,6) rectangle ++(1,1);
  \path[fill=white] (8,6) rectangle ++(1,1);
  \path[fill=black] (9,6) rectangle ++(1,1);
  \path[fill=white] (10,6) rectangle ++(1,1);
  \path[fill=black] (11,6) rectangle ++(1,1);
  \path[fill=white] (12,6) rectangle ++(1,1);
  \path[fill=black] (13,6) rectangle ++(1,1);
  \path[fill=white] (0,5) rectangle ++(1,1);
  \path[fill=white] (1,5) rectangle ++(1,1);
  \path[fill=white] (2,5) rectangle ++(1,1);
  \path[fill=white] (3,5) rectangle ++(1,1);
  \path[fill=white] (4,5) rectangle ++(1,1);
  \path[fill=white] (5,5) rectangle ++(1,1);
  \path[fill=white] (6,5) rectangle ++(1,1);
  \path[fill=white] (7,5) rectangle ++(1,1);
  \path[fill=white] (8,5) rectangle ++(1,1);
  \path[fill=black] (9,5) rectangle ++(1,1);
  \path[fill=black] (10,5) rectangle ++(1,1);
  \path[fill=black] (11,5) rectangle ++(1,1);
  \path[fill=black] (12,5) rectangle ++(1,1);
  \path[fill=black] (13,5) rectangle ++(1,1);
  \path[fill=white] (0,4) rectangle ++(1,1);
  \path[fill=white] (1,4) rectangle ++(1,1);
  \path[fill=white] (2,4) rectangle ++(1,1);
  \path[fill=white] (3,4) rectangle ++(1,1);
  \path[fill=white] (4,4) rectangle ++(1,1);
  \path[fill=white] (5,4) rectangle ++(1,1);
  \path[fill=white] (6,4) rectangle ++(1,1);
  \path[fill=white] (7,4) rectangle ++(1,1);
  \path[fill=white] (8,4) rectangle ++(1,1);
  \path[fill=white] (9,4) rectangle ++(1,1);
  \path[fill=white] (10,4) rectangle ++(1,1);
  \path[fill=black] (11,4) rectangle ++(1,1);
  \path[fill=white] (12,4) rectangle ++(1,1);
  \path[fill=black] (13,4) rectangle ++(1,1);
  \path[fill=white] (0,3) rectangle ++(1,1);
  \path[fill=white] (1,3) rectangle ++(1,1);
  \path[fill=white] (2,3) rectangle ++(1,1);
  \path[fill=white] (3,3) rectangle ++(1,1);
  \path[fill=white] (4,3) rectangle ++(1,1);
  \path[fill=white] (5,3) rectangle ++(1,1);
  \path[fill=white] (6,3) rectangle ++(1,1);
  \path[fill=white] (7,3) rectangle ++(1,1);
  \path[fill=white] (8,3) rectangle ++(1,1);
  \path[fill=white] (9,3) rectangle ++(1,1);
  \path[fill=white] (10,3) rectangle ++(1,1);
  \path[fill=black] (11,3) rectangle ++(1,1);
  \path[fill=black] (12,3) rectangle ++(1,1);
  \path[fill=black] (13,3) rectangle ++(1,1);
  \path[fill=white] (0,2) rectangle ++(1,1);
  \path[fill=white] (1,2) rectangle ++(1,1);
  \path[fill=white] (2,2) rectangle ++(1,1);
  \path[fill=white] (3,2) rectangle ++(1,1);
  \path[fill=white] (4,2) rectangle ++(1,1);
  \path[fill=white] (5,2) rectangle ++(1,1);
  \path[fill=white] (6,2) rectangle ++(1,1);
  \path[fill=white] (7,2) rectangle ++(1,1);
  \path[fill=white] (8,2) rectangle ++(1,1);
  \path[fill=white] (9,2) rectangle ++(1,1);
  \path[fill=white] (10,2) rectangle ++(1,1);
  \path[fill=white] (11,2) rectangle ++(1,1);
  \path[fill=white] (12,2) rectangle ++(1,1);
  \path[fill=black] (13,2) rectangle ++(1,1);
  \path[fill=white] (0,1) rectangle ++(1,1);
  \path[fill=white] (1,1) rectangle ++(1,1);
  \path[fill=white] (2,1) rectangle ++(1,1);
  \path[fill=white] (3,1) rectangle ++(1,1);
  \path[fill=white] (4,1) rectangle ++(1,1);
  \path[fill=white] (5,1) rectangle ++(1,1);
  \path[fill=white] (6,1) rectangle ++(1,1);
  \path[fill=white] (7,1) rectangle ++(1,1);
  \path[fill=white] (8,1) rectangle ++(1,1);
  \path[fill=white] (9,1) rectangle ++(1,1);
  \path[fill=white] (10,1) rectangle ++(1,1);
  \path[fill=white] (11,1) rectangle ++(1,1);
  \path[fill=white] (12,1) rectangle ++(1,1);
  \path[fill=black] (13,1) rectangle ++(1,1);
  \path[fill=white] (0,0) rectangle ++(1,1);
  \path[fill=white] (1,0) rectangle ++(1,1);
  \path[fill=white] (2,0) rectangle ++(1,1);
  \path[fill=white] (3,0) rectangle ++(1,1);
  \path[fill=white] (4,0) rectangle ++(1,1);
  \path[fill=white] (5,0) rectangle ++(1,1);
  \path[fill=white] (6,0) rectangle ++(1,1);
  \path[fill=white] (7,0) rectangle ++(1,1);
  \path[fill=white] (8,0) rectangle ++(1,1);
  \path[fill=white] (9,0) rectangle ++(1,1);
  \path[fill=white] (10,0) rectangle ++(1,1);
  \path[fill=white] (11,0) rectangle ++(1,1);
  \path[fill=white] (12,0) rectangle ++(1,1);
  \path[fill=white] (13,0) rectangle ++(1,1);
\end{tikzpicture}
    }
  \end{center}
  \caption{Lev(|\sigma|=6, \Delta=1) adjacency and reachability matrices.}
\end{figure}

\begin{figure}[H]
  \begin{center}
    \resizebox{0.45\textwidth}{!}{%
      \input{figures/lev_nfa_6x2}
    }
  \end{center}
  \caption{Lev(|\sigma|=6, \Delta=2) adjacency and reachability matrices.}
\end{figure}

\begin{figure}[H]
\begin{center}
  \resizebox{0.45\textwidth}{!}{%
    \begin{tikzpicture}[x=0.3cm, y=0.3cm, draw=gray, very thin]
\path[fill=white] (0,27) rectangle ++(1,1);
\path[fill=black] (1,27) rectangle ++(1,1);
\path[fill=black] (2,27) rectangle ++(1,1);
\path[fill=white] (3,27) rectangle ++(1,1);
\path[fill=black] (4,27) rectangle ++(1,1);
\path[fill=white] (5,27) rectangle ++(1,1);
\path[fill=white] (6,27) rectangle ++(1,1);
\path[fill=white] (7,27) rectangle ++(1,1);
\path[fill=black] (8,27) rectangle ++(1,1);
\path[fill=white] (9,27) rectangle ++(1,1);
\path[fill=white] (10,27) rectangle ++(1,1);
\path[fill=white] (11,27) rectangle ++(1,1);
\path[fill=white] (12,27) rectangle ++(1,1);
\path[fill=white] (13,27) rectangle ++(1,1);
\path[fill=white] (14,27) rectangle ++(1,1);
\path[fill=black] (15,27) rectangle ++(1,1);
\path[fill=white] (16,27) rectangle ++(1,1);
\path[fill=white] (17,27) rectangle ++(1,1);
\path[fill=white] (18,27) rectangle ++(1,1);
\path[fill=white] (19,27) rectangle ++(1,1);
\path[fill=white] (20,27) rectangle ++(1,1);
\path[fill=white] (21,27) rectangle ++(1,1);
\path[fill=black] (22,27) rectangle ++(1,1);
\path[fill=white] (23,27) rectangle ++(1,1);
\path[fill=white] (24,27) rectangle ++(1,1);
\path[fill=white] (25,27) rectangle ++(1,1);
\path[fill=white] (26,27) rectangle ++(1,1);
\path[fill=white] (27,27) rectangle ++(1,1);
\path[fill=white] (0,26) rectangle ++(1,1);
\path[fill=white] (1,26) rectangle ++(1,1);
\path[fill=white] (2,26) rectangle ++(1,1);
\path[fill=black] (3,26) rectangle ++(1,1);
\path[fill=black] (4,26) rectangle ++(1,1);
\path[fill=white] (5,26) rectangle ++(1,1);
\path[fill=white] (6,26) rectangle ++(1,1);
\path[fill=black] (7,26) rectangle ++(1,1);
\path[fill=white] (8,26) rectangle ++(1,1);
\path[fill=white] (9,26) rectangle ++(1,1);
\path[fill=white] (10,26) rectangle ++(1,1);
\path[fill=black] (11,26) rectangle ++(1,1);
\path[fill=white] (12,26) rectangle ++(1,1);
\path[fill=white] (13,26) rectangle ++(1,1);
\path[fill=white] (14,26) rectangle ++(1,1);
\path[fill=white] (15,26) rectangle ++(1,1);
\path[fill=white] (16,26) rectangle ++(1,1);
\path[fill=white] (17,26) rectangle ++(1,1);
\path[fill=black] (18,26) rectangle ++(1,1);
\path[fill=white] (19,26) rectangle ++(1,1);
\path[fill=white] (20,26) rectangle ++(1,1);
\path[fill=white] (21,26) rectangle ++(1,1);
\path[fill=white] (22,26) rectangle ++(1,1);
\path[fill=white] (23,26) rectangle ++(1,1);
\path[fill=white] (24,26) rectangle ++(1,1);
\path[fill=white] (25,26) rectangle ++(1,1);
\path[fill=white] (26,26) rectangle ++(1,1);
\path[fill=white] (27,26) rectangle ++(1,1);
\path[fill=white] (0,25) rectangle ++(1,1);
\path[fill=white] (1,25) rectangle ++(1,1);
\path[fill=white] (2,25) rectangle ++(1,1);
\path[fill=white] (3,25) rectangle ++(1,1);
\path[fill=black] (4,25) rectangle ++(1,1);
\path[fill=black] (5,25) rectangle ++(1,1);
\path[fill=white] (6,25) rectangle ++(1,1);
\path[fill=white] (7,25) rectangle ++(1,1);
\path[fill=black] (8,25) rectangle ++(1,1);
\path[fill=white] (9,25) rectangle ++(1,1);
\path[fill=white] (10,25) rectangle ++(1,1);
\path[fill=white] (11,25) rectangle ++(1,1);
\path[fill=black] (12,25) rectangle ++(1,1);
\path[fill=white] (13,25) rectangle ++(1,1);
\path[fill=white] (14,25) rectangle ++(1,1);
\path[fill=white] (15,25) rectangle ++(1,1);
\path[fill=white] (16,25) rectangle ++(1,1);
\path[fill=white] (17,25) rectangle ++(1,1);
\path[fill=white] (18,25) rectangle ++(1,1);
\path[fill=black] (19,25) rectangle ++(1,1);
\path[fill=white] (20,25) rectangle ++(1,1);
\path[fill=white] (21,25) rectangle ++(1,1);
\path[fill=white] (22,25) rectangle ++(1,1);
\path[fill=white] (23,25) rectangle ++(1,1);
\path[fill=white] (24,25) rectangle ++(1,1);
\path[fill=black] (25,25) rectangle ++(1,1);
\path[fill=white] (26,25) rectangle ++(1,1);
\path[fill=white] (27,25) rectangle ++(1,1);
\path[fill=white] (0,24) rectangle ++(1,1);
\path[fill=white] (1,24) rectangle ++(1,1);
\path[fill=white] (2,24) rectangle ++(1,1);
\path[fill=white] (3,24) rectangle ++(1,1);
\path[fill=white] (4,24) rectangle ++(1,1);
\path[fill=white] (5,24) rectangle ++(1,1);
\path[fill=black] (6,24) rectangle ++(1,1);
\path[fill=black] (7,24) rectangle ++(1,1);
\path[fill=white] (8,24) rectangle ++(1,1);
\path[fill=white] (9,24) rectangle ++(1,1);
\path[fill=black] (10,24) rectangle ++(1,1);
\path[fill=white] (11,24) rectangle ++(1,1);
\path[fill=white] (12,24) rectangle ++(1,1);
\path[fill=white] (13,24) rectangle ++(1,1);
\path[fill=black] (14,24) rectangle ++(1,1);
\path[fill=white] (15,24) rectangle ++(1,1);
\path[fill=white] (16,24) rectangle ++(1,1);
\path[fill=white] (17,24) rectangle ++(1,1);
\path[fill=white] (18,24) rectangle ++(1,1);
\path[fill=white] (19,24) rectangle ++(1,1);
\path[fill=white] (20,24) rectangle ++(1,1);
\path[fill=white] (21,24) rectangle ++(1,1);
\path[fill=white] (22,24) rectangle ++(1,1);
\path[fill=white] (23,24) rectangle ++(1,1);
\path[fill=white] (24,24) rectangle ++(1,1);
\path[fill=white] (25,24) rectangle ++(1,1);
\path[fill=white] (26,24) rectangle ++(1,1);
\path[fill=white] (27,24) rectangle ++(1,1);
\path[fill=white] (0,23) rectangle ++(1,1);
\path[fill=white] (1,23) rectangle ++(1,1);
\path[fill=white] (2,23) rectangle ++(1,1);
\path[fill=white] (3,23) rectangle ++(1,1);
\path[fill=white] (4,23) rectangle ++(1,1);
\path[fill=white] (5,23) rectangle ++(1,1);
\path[fill=white] (6,23) rectangle ++(1,1);
\path[fill=black] (7,23) rectangle ++(1,1);
\path[fill=black] (8,23) rectangle ++(1,1);
\path[fill=white] (9,23) rectangle ++(1,1);
\path[fill=white] (10,23) rectangle ++(1,1);
\path[fill=black] (11,23) rectangle ++(1,1);
\path[fill=white] (12,23) rectangle ++(1,1);
\path[fill=white] (13,23) rectangle ++(1,1);
\path[fill=white] (14,23) rectangle ++(1,1);
\path[fill=black] (15,23) rectangle ++(1,1);
\path[fill=white] (16,23) rectangle ++(1,1);
\path[fill=white] (17,23) rectangle ++(1,1);
\path[fill=white] (18,23) rectangle ++(1,1);
\path[fill=white] (19,23) rectangle ++(1,1);
\path[fill=white] (20,23) rectangle ++(1,1);
\path[fill=white] (21,23) rectangle ++(1,1);
\path[fill=black] (22,23) rectangle ++(1,1);
\path[fill=white] (23,23) rectangle ++(1,1);
\path[fill=white] (24,23) rectangle ++(1,1);
\path[fill=white] (25,23) rectangle ++(1,1);
\path[fill=white] (26,23) rectangle ++(1,1);
\path[fill=white] (27,23) rectangle ++(1,1);
\path[fill=white] (0,22) rectangle ++(1,1);
\path[fill=white] (1,22) rectangle ++(1,1);
\path[fill=white] (2,22) rectangle ++(1,1);
\path[fill=white] (3,22) rectangle ++(1,1);
\path[fill=white] (4,22) rectangle ++(1,1);
\path[fill=white] (5,22) rectangle ++(1,1);
\path[fill=white] (6,22) rectangle ++(1,1);
\path[fill=white] (7,22) rectangle ++(1,1);
\path[fill=black] (8,22) rectangle ++(1,1);
\path[fill=black] (9,22) rectangle ++(1,1);
\path[fill=white] (10,22) rectangle ++(1,1);
\path[fill=white] (11,22) rectangle ++(1,1);
\path[fill=black] (12,22) rectangle ++(1,1);
\path[fill=white] (13,22) rectangle ++(1,1);
\path[fill=white] (14,22) rectangle ++(1,1);
\path[fill=white] (15,22) rectangle ++(1,1);
\path[fill=black] (16,22) rectangle ++(1,1);
\path[fill=white] (17,22) rectangle ++(1,1);
\path[fill=white] (18,22) rectangle ++(1,1);
\path[fill=white] (19,22) rectangle ++(1,1);
\path[fill=white] (20,22) rectangle ++(1,1);
\path[fill=white] (21,22) rectangle ++(1,1);
\path[fill=white] (22,22) rectangle ++(1,1);
\path[fill=black] (23,22) rectangle ++(1,1);
\path[fill=white] (24,22) rectangle ++(1,1);
\path[fill=white] (25,22) rectangle ++(1,1);
\path[fill=white] (26,22) rectangle ++(1,1);
\path[fill=black] (27,22) rectangle ++(1,1);
\path[fill=white] (0,21) rectangle ++(1,1);
\path[fill=white] (1,21) rectangle ++(1,1);
\path[fill=white] (2,21) rectangle ++(1,1);
\path[fill=white] (3,21) rectangle ++(1,1);
\path[fill=white] (4,21) rectangle ++(1,1);
\path[fill=white] (5,21) rectangle ++(1,1);
\path[fill=white] (6,21) rectangle ++(1,1);
\path[fill=white] (7,21) rectangle ++(1,1);
\path[fill=white] (8,21) rectangle ++(1,1);
\path[fill=white] (9,21) rectangle ++(1,1);
\path[fill=black] (10,21) rectangle ++(1,1);
\path[fill=white] (11,21) rectangle ++(1,1);
\path[fill=white] (12,21) rectangle ++(1,1);
\path[fill=white] (13,21) rectangle ++(1,1);
\path[fill=white] (14,21) rectangle ++(1,1);
\path[fill=white] (15,21) rectangle ++(1,1);
\path[fill=white] (16,21) rectangle ++(1,1);
\path[fill=white] (17,21) rectangle ++(1,1);
\path[fill=white] (18,21) rectangle ++(1,1);
\path[fill=white] (19,21) rectangle ++(1,1);
\path[fill=white] (20,21) rectangle ++(1,1);
\path[fill=white] (21,21) rectangle ++(1,1);
\path[fill=white] (22,21) rectangle ++(1,1);
\path[fill=white] (23,21) rectangle ++(1,1);
\path[fill=white] (24,21) rectangle ++(1,1);
\path[fill=white] (25,21) rectangle ++(1,1);
\path[fill=white] (26,21) rectangle ++(1,1);
\path[fill=white] (27,21) rectangle ++(1,1);
\path[fill=white] (0,20) rectangle ++(1,1);
\path[fill=white] (1,20) rectangle ++(1,1);
\path[fill=white] (2,20) rectangle ++(1,1);
\path[fill=white] (3,20) rectangle ++(1,1);
\path[fill=white] (4,20) rectangle ++(1,1);
\path[fill=white] (5,20) rectangle ++(1,1);
\path[fill=white] (6,20) rectangle ++(1,1);
\path[fill=white] (7,20) rectangle ++(1,1);
\path[fill=white] (8,20) rectangle ++(1,1);
\path[fill=white] (9,20) rectangle ++(1,1);
\path[fill=black] (10,20) rectangle ++(1,1);
\path[fill=black] (11,20) rectangle ++(1,1);
\path[fill=white] (12,20) rectangle ++(1,1);
\path[fill=white] (13,20) rectangle ++(1,1);
\path[fill=black] (14,20) rectangle ++(1,1);
\path[fill=white] (15,20) rectangle ++(1,1);
\path[fill=white] (16,20) rectangle ++(1,1);
\path[fill=white] (17,20) rectangle ++(1,1);
\path[fill=black] (18,20) rectangle ++(1,1);
\path[fill=white] (19,20) rectangle ++(1,1);
\path[fill=white] (20,20) rectangle ++(1,1);
\path[fill=white] (21,20) rectangle ++(1,1);
\path[fill=white] (22,20) rectangle ++(1,1);
\path[fill=white] (23,20) rectangle ++(1,1);
\path[fill=white] (24,20) rectangle ++(1,1);
\path[fill=white] (25,20) rectangle ++(1,1);
\path[fill=white] (26,20) rectangle ++(1,1);
\path[fill=white] (27,20) rectangle ++(1,1);
\path[fill=white] (0,19) rectangle ++(1,1);
\path[fill=white] (1,19) rectangle ++(1,1);
\path[fill=white] (2,19) rectangle ++(1,1);
\path[fill=white] (3,19) rectangle ++(1,1);
\path[fill=white] (4,19) rectangle ++(1,1);
\path[fill=white] (5,19) rectangle ++(1,1);
\path[fill=white] (6,19) rectangle ++(1,1);
\path[fill=white] (7,19) rectangle ++(1,1);
\path[fill=white] (8,19) rectangle ++(1,1);
\path[fill=white] (9,19) rectangle ++(1,1);
\path[fill=white] (10,19) rectangle ++(1,1);
\path[fill=black] (11,19) rectangle ++(1,1);
\path[fill=black] (12,19) rectangle ++(1,1);
\path[fill=white] (13,19) rectangle ++(1,1);
\path[fill=white] (14,19) rectangle ++(1,1);
\path[fill=black] (15,19) rectangle ++(1,1);
\path[fill=white] (16,19) rectangle ++(1,1);
\path[fill=white] (17,19) rectangle ++(1,1);
\path[fill=white] (18,19) rectangle ++(1,1);
\path[fill=black] (19,19) rectangle ++(1,1);
\path[fill=white] (20,19) rectangle ++(1,1);
\path[fill=white] (21,19) rectangle ++(1,1);
\path[fill=white] (22,19) rectangle ++(1,1);
\path[fill=white] (23,19) rectangle ++(1,1);
\path[fill=white] (24,19) rectangle ++(1,1);
\path[fill=black] (25,19) rectangle ++(1,1);
\path[fill=white] (26,19) rectangle ++(1,1);
\path[fill=white] (27,19) rectangle ++(1,1);
\path[fill=white] (0,18) rectangle ++(1,1);
\path[fill=white] (1,18) rectangle ++(1,1);
\path[fill=white] (2,18) rectangle ++(1,1);
\path[fill=white] (3,18) rectangle ++(1,1);
\path[fill=white] (4,18) rectangle ++(1,1);
\path[fill=white] (5,18) rectangle ++(1,1);
\path[fill=white] (6,18) rectangle ++(1,1);
\path[fill=white] (7,18) rectangle ++(1,1);
\path[fill=white] (8,18) rectangle ++(1,1);
\path[fill=white] (9,18) rectangle ++(1,1);
\path[fill=white] (10,18) rectangle ++(1,1);
\path[fill=white] (11,18) rectangle ++(1,1);
\path[fill=black] (12,18) rectangle ++(1,1);
\path[fill=black] (13,18) rectangle ++(1,1);
\path[fill=white] (14,18) rectangle ++(1,1);
\path[fill=white] (15,18) rectangle ++(1,1);
\path[fill=black] (16,18) rectangle ++(1,1);
\path[fill=white] (17,18) rectangle ++(1,1);
\path[fill=white] (18,18) rectangle ++(1,1);
\path[fill=white] (19,18) rectangle ++(1,1);
\path[fill=black] (20,18) rectangle ++(1,1);
\path[fill=white] (21,18) rectangle ++(1,1);
\path[fill=white] (22,18) rectangle ++(1,1);
\path[fill=white] (23,18) rectangle ++(1,1);
\path[fill=white] (24,18) rectangle ++(1,1);
\path[fill=white] (25,18) rectangle ++(1,1);
\path[fill=black] (26,18) rectangle ++(1,1);
\path[fill=white] (27,18) rectangle ++(1,1);
\path[fill=white] (0,17) rectangle ++(1,1);
\path[fill=white] (1,17) rectangle ++(1,1);
\path[fill=white] (2,17) rectangle ++(1,1);
\path[fill=white] (3,17) rectangle ++(1,1);
\path[fill=white] (4,17) rectangle ++(1,1);
\path[fill=white] (5,17) rectangle ++(1,1);
\path[fill=white] (6,17) rectangle ++(1,1);
\path[fill=white] (7,17) rectangle ++(1,1);
\path[fill=white] (8,17) rectangle ++(1,1);
\path[fill=white] (9,17) rectangle ++(1,1);
\path[fill=white] (10,17) rectangle ++(1,1);
\path[fill=white] (11,17) rectangle ++(1,1);
\path[fill=white] (12,17) rectangle ++(1,1);
\path[fill=white] (13,17) rectangle ++(1,1);
\path[fill=black] (14,17) rectangle ++(1,1);
\path[fill=white] (15,17) rectangle ++(1,1);
\path[fill=white] (16,17) rectangle ++(1,1);
\path[fill=white] (17,17) rectangle ++(1,1);
\path[fill=white] (18,17) rectangle ++(1,1);
\path[fill=white] (19,17) rectangle ++(1,1);
\path[fill=white] (20,17) rectangle ++(1,1);
\path[fill=white] (21,17) rectangle ++(1,1);
\path[fill=white] (22,17) rectangle ++(1,1);
\path[fill=white] (23,17) rectangle ++(1,1);
\path[fill=white] (24,17) rectangle ++(1,1);
\path[fill=white] (25,17) rectangle ++(1,1);
\path[fill=white] (26,17) rectangle ++(1,1);
\path[fill=white] (27,17) rectangle ++(1,1);
\path[fill=white] (0,16) rectangle ++(1,1);
\path[fill=white] (1,16) rectangle ++(1,1);
\path[fill=white] (2,16) rectangle ++(1,1);
\path[fill=white] (3,16) rectangle ++(1,1);
\path[fill=white] (4,16) rectangle ++(1,1);
\path[fill=white] (5,16) rectangle ++(1,1);
\path[fill=white] (6,16) rectangle ++(1,1);
\path[fill=white] (7,16) rectangle ++(1,1);
\path[fill=white] (8,16) rectangle ++(1,1);
\path[fill=white] (9,16) rectangle ++(1,1);
\path[fill=white] (10,16) rectangle ++(1,1);
\path[fill=white] (11,16) rectangle ++(1,1);
\path[fill=white] (12,16) rectangle ++(1,1);
\path[fill=white] (13,16) rectangle ++(1,1);
\path[fill=black] (14,16) rectangle ++(1,1);
\path[fill=black] (15,16) rectangle ++(1,1);
\path[fill=white] (16,16) rectangle ++(1,1);
\path[fill=white] (17,16) rectangle ++(1,1);
\path[fill=black] (18,16) rectangle ++(1,1);
\path[fill=white] (19,16) rectangle ++(1,1);
\path[fill=white] (20,16) rectangle ++(1,1);
\path[fill=white] (21,16) rectangle ++(1,1);
\path[fill=black] (22,16) rectangle ++(1,1);
\path[fill=white] (23,16) rectangle ++(1,1);
\path[fill=white] (24,16) rectangle ++(1,1);
\path[fill=white] (25,16) rectangle ++(1,1);
\path[fill=white] (26,16) rectangle ++(1,1);
\path[fill=white] (27,16) rectangle ++(1,1);
\path[fill=white] (0,15) rectangle ++(1,1);
\path[fill=white] (1,15) rectangle ++(1,1);
\path[fill=white] (2,15) rectangle ++(1,1);
\path[fill=white] (3,15) rectangle ++(1,1);
\path[fill=white] (4,15) rectangle ++(1,1);
\path[fill=white] (5,15) rectangle ++(1,1);
\path[fill=white] (6,15) rectangle ++(1,1);
\path[fill=white] (7,15) rectangle ++(1,1);
\path[fill=white] (8,15) rectangle ++(1,1);
\path[fill=white] (9,15) rectangle ++(1,1);
\path[fill=white] (10,15) rectangle ++(1,1);
\path[fill=white] (11,15) rectangle ++(1,1);
\path[fill=white] (12,15) rectangle ++(1,1);
\path[fill=white] (13,15) rectangle ++(1,1);
\path[fill=white] (14,15) rectangle ++(1,1);
\path[fill=black] (15,15) rectangle ++(1,1);
\path[fill=black] (16,15) rectangle ++(1,1);
\path[fill=white] (17,15) rectangle ++(1,1);
\path[fill=white] (18,15) rectangle ++(1,1);
\path[fill=black] (19,15) rectangle ++(1,1);
\path[fill=white] (20,15) rectangle ++(1,1);
\path[fill=white] (21,15) rectangle ++(1,1);
\path[fill=white] (22,15) rectangle ++(1,1);
\path[fill=black] (23,15) rectangle ++(1,1);
\path[fill=white] (24,15) rectangle ++(1,1);
\path[fill=white] (25,15) rectangle ++(1,1);
\path[fill=white] (26,15) rectangle ++(1,1);
\path[fill=black] (27,15) rectangle ++(1,1);
\path[fill=white] (0,14) rectangle ++(1,1);
\path[fill=white] (1,14) rectangle ++(1,1);
\path[fill=white] (2,14) rectangle ++(1,1);
\path[fill=white] (3,14) rectangle ++(1,1);
\path[fill=white] (4,14) rectangle ++(1,1);
\path[fill=white] (5,14) rectangle ++(1,1);
\path[fill=white] (6,14) rectangle ++(1,1);
\path[fill=white] (7,14) rectangle ++(1,1);
\path[fill=white] (8,14) rectangle ++(1,1);
\path[fill=white] (9,14) rectangle ++(1,1);
\path[fill=white] (10,14) rectangle ++(1,1);
\path[fill=white] (11,14) rectangle ++(1,1);
\path[fill=white] (12,14) rectangle ++(1,1);
\path[fill=white] (13,14) rectangle ++(1,1);
\path[fill=white] (14,14) rectangle ++(1,1);
\path[fill=white] (15,14) rectangle ++(1,1);
\path[fill=black] (16,14) rectangle ++(1,1);
\path[fill=black] (17,14) rectangle ++(1,1);
\path[fill=white] (18,14) rectangle ++(1,1);
\path[fill=white] (19,14) rectangle ++(1,1);
\path[fill=black] (20,14) rectangle ++(1,1);
\path[fill=white] (21,14) rectangle ++(1,1);
\path[fill=white] (22,14) rectangle ++(1,1);
\path[fill=white] (23,14) rectangle ++(1,1);
\path[fill=black] (24,14) rectangle ++(1,1);
\path[fill=white] (25,14) rectangle ++(1,1);
\path[fill=white] (26,14) rectangle ++(1,1);
\path[fill=white] (27,14) rectangle ++(1,1);
\path[fill=white] (0,13) rectangle ++(1,1);
\path[fill=white] (1,13) rectangle ++(1,1);
\path[fill=white] (2,13) rectangle ++(1,1);
\path[fill=white] (3,13) rectangle ++(1,1);
\path[fill=white] (4,13) rectangle ++(1,1);
\path[fill=white] (5,13) rectangle ++(1,1);
\path[fill=white] (6,13) rectangle ++(1,1);
\path[fill=white] (7,13) rectangle ++(1,1);
\path[fill=white] (8,13) rectangle ++(1,1);
\path[fill=white] (9,13) rectangle ++(1,1);
\path[fill=white] (10,13) rectangle ++(1,1);
\path[fill=white] (11,13) rectangle ++(1,1);
\path[fill=white] (12,13) rectangle ++(1,1);
\path[fill=white] (13,13) rectangle ++(1,1);
\path[fill=white] (14,13) rectangle ++(1,1);
\path[fill=white] (15,13) rectangle ++(1,1);
\path[fill=white] (16,13) rectangle ++(1,1);
\path[fill=white] (17,13) rectangle ++(1,1);
\path[fill=black] (18,13) rectangle ++(1,1);
\path[fill=white] (19,13) rectangle ++(1,1);
\path[fill=white] (20,13) rectangle ++(1,1);
\path[fill=white] (21,13) rectangle ++(1,1);
\path[fill=white] (22,13) rectangle ++(1,1);
\path[fill=white] (23,13) rectangle ++(1,1);
\path[fill=white] (24,13) rectangle ++(1,1);
\path[fill=white] (25,13) rectangle ++(1,1);
\path[fill=white] (26,13) rectangle ++(1,1);
\path[fill=white] (27,13) rectangle ++(1,1);
\path[fill=white] (0,12) rectangle ++(1,1);
\path[fill=white] (1,12) rectangle ++(1,1);
\path[fill=white] (2,12) rectangle ++(1,1);
\path[fill=white] (3,12) rectangle ++(1,1);
\path[fill=white] (4,12) rectangle ++(1,1);
\path[fill=white] (5,12) rectangle ++(1,1);
\path[fill=white] (6,12) rectangle ++(1,1);
\path[fill=white] (7,12) rectangle ++(1,1);
\path[fill=white] (8,12) rectangle ++(1,1);
\path[fill=white] (9,12) rectangle ++(1,1);
\path[fill=white] (10,12) rectangle ++(1,1);
\path[fill=white] (11,12) rectangle ++(1,1);
\path[fill=white] (12,12) rectangle ++(1,1);
\path[fill=white] (13,12) rectangle ++(1,1);
\path[fill=white] (14,12) rectangle ++(1,1);
\path[fill=white] (15,12) rectangle ++(1,1);
\path[fill=white] (16,12) rectangle ++(1,1);
\path[fill=white] (17,12) rectangle ++(1,1);
\path[fill=black] (18,12) rectangle ++(1,1);
\path[fill=black] (19,12) rectangle ++(1,1);
\path[fill=white] (20,12) rectangle ++(1,1);
\path[fill=white] (21,12) rectangle ++(1,1);
\path[fill=black] (22,12) rectangle ++(1,1);
\path[fill=white] (23,12) rectangle ++(1,1);
\path[fill=white] (24,12) rectangle ++(1,1);
\path[fill=black] (25,12) rectangle ++(1,1);
\path[fill=white] (26,12) rectangle ++(1,1);
\path[fill=white] (27,12) rectangle ++(1,1);
\path[fill=white] (0,11) rectangle ++(1,1);
\path[fill=white] (1,11) rectangle ++(1,1);
\path[fill=white] (2,11) rectangle ++(1,1);
\path[fill=white] (3,11) rectangle ++(1,1);
\path[fill=white] (4,11) rectangle ++(1,1);
\path[fill=white] (5,11) rectangle ++(1,1);
\path[fill=white] (6,11) rectangle ++(1,1);
\path[fill=white] (7,11) rectangle ++(1,1);
\path[fill=white] (8,11) rectangle ++(1,1);
\path[fill=white] (9,11) rectangle ++(1,1);
\path[fill=white] (10,11) rectangle ++(1,1);
\path[fill=white] (11,11) rectangle ++(1,1);
\path[fill=white] (12,11) rectangle ++(1,1);
\path[fill=white] (13,11) rectangle ++(1,1);
\path[fill=white] (14,11) rectangle ++(1,1);
\path[fill=white] (15,11) rectangle ++(1,1);
\path[fill=white] (16,11) rectangle ++(1,1);
\path[fill=white] (17,11) rectangle ++(1,1);
\path[fill=white] (18,11) rectangle ++(1,1);
\path[fill=black] (19,11) rectangle ++(1,1);
\path[fill=black] (20,11) rectangle ++(1,1);
\path[fill=white] (21,11) rectangle ++(1,1);
\path[fill=white] (22,11) rectangle ++(1,1);
\path[fill=black] (23,11) rectangle ++(1,1);
\path[fill=white] (24,11) rectangle ++(1,1);
\path[fill=white] (25,11) rectangle ++(1,1);
\path[fill=black] (26,11) rectangle ++(1,1);
\path[fill=white] (27,11) rectangle ++(1,1);
\path[fill=white] (0,10) rectangle ++(1,1);
\path[fill=white] (1,10) rectangle ++(1,1);
\path[fill=white] (2,10) rectangle ++(1,1);
\path[fill=white] (3,10) rectangle ++(1,1);
\path[fill=white] (4,10) rectangle ++(1,1);
\path[fill=white] (5,10) rectangle ++(1,1);
\path[fill=white] (6,10) rectangle ++(1,1);
\path[fill=white] (7,10) rectangle ++(1,1);
\path[fill=white] (8,10) rectangle ++(1,1);
\path[fill=white] (9,10) rectangle ++(1,1);
\path[fill=white] (10,10) rectangle ++(1,1);
\path[fill=white] (11,10) rectangle ++(1,1);
\path[fill=white] (12,10) rectangle ++(1,1);
\path[fill=white] (13,10) rectangle ++(1,1);
\path[fill=white] (14,10) rectangle ++(1,1);
\path[fill=white] (15,10) rectangle ++(1,1);
\path[fill=white] (16,10) rectangle ++(1,1);
\path[fill=white] (17,10) rectangle ++(1,1);
\path[fill=white] (18,10) rectangle ++(1,1);
\path[fill=white] (19,10) rectangle ++(1,1);
\path[fill=black] (20,10) rectangle ++(1,1);
\path[fill=black] (21,10) rectangle ++(1,1);
\path[fill=white] (22,10) rectangle ++(1,1);
\path[fill=white] (23,10) rectangle ++(1,1);
\path[fill=black] (24,10) rectangle ++(1,1);
\path[fill=white] (25,10) rectangle ++(1,1);
\path[fill=white] (26,10) rectangle ++(1,1);
\path[fill=white] (27,10) rectangle ++(1,1);
\path[fill=white] (0,9) rectangle ++(1,1);
\path[fill=white] (1,9) rectangle ++(1,1);
\path[fill=white] (2,9) rectangle ++(1,1);
\path[fill=white] (3,9) rectangle ++(1,1);
\path[fill=white] (4,9) rectangle ++(1,1);
\path[fill=white] (5,9) rectangle ++(1,1);
\path[fill=white] (6,9) rectangle ++(1,1);
\path[fill=white] (7,9) rectangle ++(1,1);
\path[fill=white] (8,9) rectangle ++(1,1);
\path[fill=white] (9,9) rectangle ++(1,1);
\path[fill=white] (10,9) rectangle ++(1,1);
\path[fill=white] (11,9) rectangle ++(1,1);
\path[fill=white] (12,9) rectangle ++(1,1);
\path[fill=white] (13,9) rectangle ++(1,1);
\path[fill=white] (14,9) rectangle ++(1,1);
\path[fill=white] (15,9) rectangle ++(1,1);
\path[fill=white] (16,9) rectangle ++(1,1);
\path[fill=white] (17,9) rectangle ++(1,1);
\path[fill=white] (18,9) rectangle ++(1,1);
\path[fill=white] (19,9) rectangle ++(1,1);
\path[fill=white] (20,9) rectangle ++(1,1);
\path[fill=white] (21,9) rectangle ++(1,1);
\path[fill=black] (22,9) rectangle ++(1,1);
\path[fill=white] (23,9) rectangle ++(1,1);
\path[fill=white] (24,9) rectangle ++(1,1);
\path[fill=white] (25,9) rectangle ++(1,1);
\path[fill=white] (26,9) rectangle ++(1,1);
\path[fill=white] (27,9) rectangle ++(1,1);
\path[fill=white] (0,8) rectangle ++(1,1);
\path[fill=white] (1,8) rectangle ++(1,1);
\path[fill=white] (2,8) rectangle ++(1,1);
\path[fill=white] (3,8) rectangle ++(1,1);
\path[fill=white] (4,8) rectangle ++(1,1);
\path[fill=white] (5,8) rectangle ++(1,1);
\path[fill=white] (6,8) rectangle ++(1,1);
\path[fill=white] (7,8) rectangle ++(1,1);
\path[fill=white] (8,8) rectangle ++(1,1);
\path[fill=white] (9,8) rectangle ++(1,1);
\path[fill=white] (10,8) rectangle ++(1,1);
\path[fill=white] (11,8) rectangle ++(1,1);
\path[fill=white] (12,8) rectangle ++(1,1);
\path[fill=white] (13,8) rectangle ++(1,1);
\path[fill=white] (14,8) rectangle ++(1,1);
\path[fill=white] (15,8) rectangle ++(1,1);
\path[fill=white] (16,8) rectangle ++(1,1);
\path[fill=white] (17,8) rectangle ++(1,1);
\path[fill=white] (18,8) rectangle ++(1,1);
\path[fill=white] (19,8) rectangle ++(1,1);
\path[fill=white] (20,8) rectangle ++(1,1);
\path[fill=white] (21,8) rectangle ++(1,1);
\path[fill=black] (22,8) rectangle ++(1,1);
\path[fill=black] (23,8) rectangle ++(1,1);
\path[fill=white] (24,8) rectangle ++(1,1);
\path[fill=black] (25,8) rectangle ++(1,1);
\path[fill=white] (26,8) rectangle ++(1,1);
\path[fill=black] (27,8) rectangle ++(1,1);
\path[fill=white] (0,7) rectangle ++(1,1);
\path[fill=white] (1,7) rectangle ++(1,1);
\path[fill=white] (2,7) rectangle ++(1,1);
\path[fill=white] (3,7) rectangle ++(1,1);
\path[fill=white] (4,7) rectangle ++(1,1);
\path[fill=white] (5,7) rectangle ++(1,1);
\path[fill=white] (6,7) rectangle ++(1,1);
\path[fill=white] (7,7) rectangle ++(1,1);
\path[fill=white] (8,7) rectangle ++(1,1);
\path[fill=white] (9,7) rectangle ++(1,1);
\path[fill=white] (10,7) rectangle ++(1,1);
\path[fill=white] (11,7) rectangle ++(1,1);
\path[fill=white] (12,7) rectangle ++(1,1);
\path[fill=white] (13,7) rectangle ++(1,1);
\path[fill=white] (14,7) rectangle ++(1,1);
\path[fill=white] (15,7) rectangle ++(1,1);
\path[fill=white] (16,7) rectangle ++(1,1);
\path[fill=white] (17,7) rectangle ++(1,1);
\path[fill=white] (18,7) rectangle ++(1,1);
\path[fill=white] (19,7) rectangle ++(1,1);
\path[fill=white] (20,7) rectangle ++(1,1);
\path[fill=white] (21,7) rectangle ++(1,1);
\path[fill=white] (22,7) rectangle ++(1,1);
\path[fill=black] (23,7) rectangle ++(1,1);
\path[fill=black] (24,7) rectangle ++(1,1);
\path[fill=white] (25,7) rectangle ++(1,1);
\path[fill=black] (26,7) rectangle ++(1,1);
\path[fill=white] (27,7) rectangle ++(1,1);
\path[fill=white] (0,6) rectangle ++(1,1);
\path[fill=white] (1,6) rectangle ++(1,1);
\path[fill=white] (2,6) rectangle ++(1,1);
\path[fill=white] (3,6) rectangle ++(1,1);
\path[fill=white] (4,6) rectangle ++(1,1);
\path[fill=white] (5,6) rectangle ++(1,1);
\path[fill=white] (6,6) rectangle ++(1,1);
\path[fill=white] (7,6) rectangle ++(1,1);
\path[fill=white] (8,6) rectangle ++(1,1);
\path[fill=white] (9,6) rectangle ++(1,1);
\path[fill=white] (10,6) rectangle ++(1,1);
\path[fill=white] (11,6) rectangle ++(1,1);
\path[fill=white] (12,6) rectangle ++(1,1);
\path[fill=white] (13,6) rectangle ++(1,1);
\path[fill=white] (14,6) rectangle ++(1,1);
\path[fill=white] (15,6) rectangle ++(1,1);
\path[fill=white] (16,6) rectangle ++(1,1);
\path[fill=white] (17,6) rectangle ++(1,1);
\path[fill=white] (18,6) rectangle ++(1,1);
\path[fill=white] (19,6) rectangle ++(1,1);
\path[fill=white] (20,6) rectangle ++(1,1);
\path[fill=white] (21,6) rectangle ++(1,1);
\path[fill=white] (22,6) rectangle ++(1,1);
\path[fill=white] (23,6) rectangle ++(1,1);
\path[fill=black] (24,6) rectangle ++(1,1);
\path[fill=white] (25,6) rectangle ++(1,1);
\path[fill=white] (26,6) rectangle ++(1,1);
\path[fill=white] (27,6) rectangle ++(1,1);
\path[fill=white] (0,5) rectangle ++(1,1);
\path[fill=white] (1,5) rectangle ++(1,1);
\path[fill=white] (2,5) rectangle ++(1,1);
\path[fill=white] (3,5) rectangle ++(1,1);
\path[fill=white] (4,5) rectangle ++(1,1);
\path[fill=white] (5,5) rectangle ++(1,1);
\path[fill=white] (6,5) rectangle ++(1,1);
\path[fill=white] (7,5) rectangle ++(1,1);
\path[fill=white] (8,5) rectangle ++(1,1);
\path[fill=white] (9,5) rectangle ++(1,1);
\path[fill=white] (10,5) rectangle ++(1,1);
\path[fill=white] (11,5) rectangle ++(1,1);
\path[fill=white] (12,5) rectangle ++(1,1);
\path[fill=white] (13,5) rectangle ++(1,1);
\path[fill=white] (14,5) rectangle ++(1,1);
\path[fill=white] (15,5) rectangle ++(1,1);
\path[fill=white] (16,5) rectangle ++(1,1);
\path[fill=white] (17,5) rectangle ++(1,1);
\path[fill=white] (18,5) rectangle ++(1,1);
\path[fill=white] (19,5) rectangle ++(1,1);
\path[fill=white] (20,5) rectangle ++(1,1);
\path[fill=white] (21,5) rectangle ++(1,1);
\path[fill=white] (22,5) rectangle ++(1,1);
\path[fill=white] (23,5) rectangle ++(1,1);
\path[fill=white] (24,5) rectangle ++(1,1);
\path[fill=black] (25,5) rectangle ++(1,1);
\path[fill=white] (26,5) rectangle ++(1,1);
\path[fill=white] (27,5) rectangle ++(1,1);
\path[fill=white] (0,4) rectangle ++(1,1);
\path[fill=white] (1,4) rectangle ++(1,1);
\path[fill=white] (2,4) rectangle ++(1,1);
\path[fill=white] (3,4) rectangle ++(1,1);
\path[fill=white] (4,4) rectangle ++(1,1);
\path[fill=white] (5,4) rectangle ++(1,1);
\path[fill=white] (6,4) rectangle ++(1,1);
\path[fill=white] (7,4) rectangle ++(1,1);
\path[fill=white] (8,4) rectangle ++(1,1);
\path[fill=white] (9,4) rectangle ++(1,1);
\path[fill=white] (10,4) rectangle ++(1,1);
\path[fill=white] (11,4) rectangle ++(1,1);
\path[fill=white] (12,4) rectangle ++(1,1);
\path[fill=white] (13,4) rectangle ++(1,1);
\path[fill=white] (14,4) rectangle ++(1,1);
\path[fill=white] (15,4) rectangle ++(1,1);
\path[fill=white] (16,4) rectangle ++(1,1);
\path[fill=white] (17,4) rectangle ++(1,1);
\path[fill=white] (18,4) rectangle ++(1,1);
\path[fill=white] (19,4) rectangle ++(1,1);
\path[fill=white] (20,4) rectangle ++(1,1);
\path[fill=white] (21,4) rectangle ++(1,1);
\path[fill=white] (22,4) rectangle ++(1,1);
\path[fill=white] (23,4) rectangle ++(1,1);
\path[fill=white] (24,4) rectangle ++(1,1);
\path[fill=black] (25,4) rectangle ++(1,1);
\path[fill=black] (26,4) rectangle ++(1,1);
\path[fill=black] (27,4) rectangle ++(1,1);
\path[fill=white] (0,3) rectangle ++(1,1);
\path[fill=white] (1,3) rectangle ++(1,1);
\path[fill=white] (2,3) rectangle ++(1,1);
\path[fill=white] (3,3) rectangle ++(1,1);
\path[fill=white] (4,3) rectangle ++(1,1);
\path[fill=white] (5,3) rectangle ++(1,1);
\path[fill=white] (6,3) rectangle ++(1,1);
\path[fill=white] (7,3) rectangle ++(1,1);
\path[fill=white] (8,3) rectangle ++(1,1);
\path[fill=white] (9,3) rectangle ++(1,1);
\path[fill=white] (10,3) rectangle ++(1,1);
\path[fill=white] (11,3) rectangle ++(1,1);
\path[fill=white] (12,3) rectangle ++(1,1);
\path[fill=white] (13,3) rectangle ++(1,1);
\path[fill=white] (14,3) rectangle ++(1,1);
\path[fill=white] (15,3) rectangle ++(1,1);
\path[fill=white] (16,3) rectangle ++(1,1);
\path[fill=white] (17,3) rectangle ++(1,1);
\path[fill=white] (18,3) rectangle ++(1,1);
\path[fill=white] (19,3) rectangle ++(1,1);
\path[fill=white] (20,3) rectangle ++(1,1);
\path[fill=white] (21,3) rectangle ++(1,1);
\path[fill=white] (22,3) rectangle ++(1,1);
\path[fill=white] (23,3) rectangle ++(1,1);
\path[fill=white] (24,3) rectangle ++(1,1);
\path[fill=white] (25,3) rectangle ++(1,1);
\path[fill=black] (26,3) rectangle ++(1,1);
\path[fill=white] (27,3) rectangle ++(1,1);
\path[fill=white] (0,2) rectangle ++(1,1);
\path[fill=white] (1,2) rectangle ++(1,1);
\path[fill=white] (2,2) rectangle ++(1,1);
\path[fill=white] (3,2) rectangle ++(1,1);
\path[fill=white] (4,2) rectangle ++(1,1);
\path[fill=white] (5,2) rectangle ++(1,1);
\path[fill=white] (6,2) rectangle ++(1,1);
\path[fill=white] (7,2) rectangle ++(1,1);
\path[fill=white] (8,2) rectangle ++(1,1);
\path[fill=white] (9,2) rectangle ++(1,1);
\path[fill=white] (10,2) rectangle ++(1,1);
\path[fill=white] (11,2) rectangle ++(1,1);
\path[fill=white] (12,2) rectangle ++(1,1);
\path[fill=white] (13,2) rectangle ++(1,1);
\path[fill=white] (14,2) rectangle ++(1,1);
\path[fill=white] (15,2) rectangle ++(1,1);
\path[fill=white] (16,2) rectangle ++(1,1);
\path[fill=white] (17,2) rectangle ++(1,1);
\path[fill=white] (18,2) rectangle ++(1,1);
\path[fill=white] (19,2) rectangle ++(1,1);
\path[fill=white] (20,2) rectangle ++(1,1);
\path[fill=white] (21,2) rectangle ++(1,1);
\path[fill=white] (22,2) rectangle ++(1,1);
\path[fill=white] (23,2) rectangle ++(1,1);
\path[fill=white] (24,2) rectangle ++(1,1);
\path[fill=white] (25,2) rectangle ++(1,1);
\path[fill=white] (26,2) rectangle ++(1,1);
\path[fill=black] (27,2) rectangle ++(1,1);
\path[fill=white] (0,1) rectangle ++(1,1);
\path[fill=white] (1,1) rectangle ++(1,1);
\path[fill=white] (2,1) rectangle ++(1,1);
\path[fill=white] (3,1) rectangle ++(1,1);
\path[fill=white] (4,1) rectangle ++(1,1);
\path[fill=white] (5,1) rectangle ++(1,1);
\path[fill=white] (6,1) rectangle ++(1,1);
\path[fill=white] (7,1) rectangle ++(1,1);
\path[fill=white] (8,1) rectangle ++(1,1);
\path[fill=white] (9,1) rectangle ++(1,1);
\path[fill=white] (10,1) rectangle ++(1,1);
\path[fill=white] (11,1) rectangle ++(1,1);
\path[fill=white] (12,1) rectangle ++(1,1);
\path[fill=white] (13,1) rectangle ++(1,1);
\path[fill=white] (14,1) rectangle ++(1,1);
\path[fill=white] (15,1) rectangle ++(1,1);
\path[fill=white] (16,1) rectangle ++(1,1);
\path[fill=white] (17,1) rectangle ++(1,1);
\path[fill=white] (18,1) rectangle ++(1,1);
\path[fill=white] (19,1) rectangle ++(1,1);
\path[fill=white] (20,1) rectangle ++(1,1);
\path[fill=white] (21,1) rectangle ++(1,1);
\path[fill=white] (22,1) rectangle ++(1,1);
\path[fill=white] (23,1) rectangle ++(1,1);
\path[fill=white] (24,1) rectangle ++(1,1);
\path[fill=white] (25,1) rectangle ++(1,1);
\path[fill=white] (26,1) rectangle ++(1,1);
\path[fill=black] (27,1) rectangle ++(1,1);
\path[fill=white] (0,0) rectangle ++(1,1);
\path[fill=white] (1,0) rectangle ++(1,1);
\path[fill=white] (2,0) rectangle ++(1,1);
\path[fill=white] (3,0) rectangle ++(1,1);
\path[fill=white] (4,0) rectangle ++(1,1);
\path[fill=white] (5,0) rectangle ++(1,1);
\path[fill=white] (6,0) rectangle ++(1,1);
\path[fill=white] (7,0) rectangle ++(1,1);
\path[fill=white] (8,0) rectangle ++(1,1);
\path[fill=white] (9,0) rectangle ++(1,1);
\path[fill=white] (10,0) rectangle ++(1,1);
\path[fill=white] (11,0) rectangle ++(1,1);
\path[fill=white] (12,0) rectangle ++(1,1);
\path[fill=white] (13,0) rectangle ++(1,1);
\path[fill=white] (14,0) rectangle ++(1,1);
\path[fill=white] (15,0) rectangle ++(1,1);
\path[fill=white] (16,0) rectangle ++(1,1);
\path[fill=white] (17,0) rectangle ++(1,1);
\path[fill=white] (18,0) rectangle ++(1,1);
\path[fill=white] (19,0) rectangle ++(1,1);
\path[fill=white] (20,0) rectangle ++(1,1);
\path[fill=white] (21,0) rectangle ++(1,1);
\path[fill=white] (22,0) rectangle ++(1,1);
\path[fill=white] (23,0) rectangle ++(1,1);
\path[fill=white] (24,0) rectangle ++(1,1);
\path[fill=white] (25,0) rectangle ++(1,1);
\path[fill=white] (26,0) rectangle ++(1,1);
\path[fill=white] (27,0) rectangle ++(1,1);
\end{tikzpicture}

\begin{tikzpicture}[x=0.3cm, y=0.3cm, draw=gray, very thin]
\path[fill=white] (0,27) rectangle ++(1,1);
\path[fill=black] (1,27) rectangle ++(1,1);
\path[fill=black] (2,27) rectangle ++(1,1);
\path[fill=black] (3,27) rectangle ++(1,1);
\path[fill=black] (4,27) rectangle ++(1,1);
\path[fill=black] (5,27) rectangle ++(1,1);
\path[fill=black] (6,27) rectangle ++(1,1);
\path[fill=black] (7,27) rectangle ++(1,1);
\path[fill=black] (8,27) rectangle ++(1,1);
\path[fill=black] (9,27) rectangle ++(1,1);
\path[fill=black] (10,27) rectangle ++(1,1);
\path[fill=black] (11,27) rectangle ++(1,1);
\path[fill=black] (12,27) rectangle ++(1,1);
\path[fill=black] (13,27) rectangle ++(1,1);
\path[fill=black] (14,27) rectangle ++(1,1);
\path[fill=black] (15,27) rectangle ++(1,1);
\path[fill=black] (16,27) rectangle ++(1,1);
\path[fill=black] (17,27) rectangle ++(1,1);
\path[fill=black] (18,27) rectangle ++(1,1);
\path[fill=black] (19,27) rectangle ++(1,1);
\path[fill=black] (20,27) rectangle ++(1,1);
\path[fill=black] (21,27) rectangle ++(1,1);
\path[fill=black] (22,27) rectangle ++(1,1);
\path[fill=black] (23,27) rectangle ++(1,1);
\path[fill=black] (24,27) rectangle ++(1,1);
\path[fill=black] (25,27) rectangle ++(1,1);
\path[fill=black] (26,27) rectangle ++(1,1);
\path[fill=black] (27,27) rectangle ++(1,1);
\path[fill=white] (0,26) rectangle ++(1,1);
\path[fill=white] (1,26) rectangle ++(1,1);
\path[fill=white] (2,26) rectangle ++(1,1);
\path[fill=black] (3,26) rectangle ++(1,1);
\path[fill=black] (4,26) rectangle ++(1,1);
\path[fill=white] (5,26) rectangle ++(1,1);
\path[fill=black] (6,26) rectangle ++(1,1);
\path[fill=black] (7,26) rectangle ++(1,1);
\path[fill=black] (8,26) rectangle ++(1,1);
\path[fill=white] (9,26) rectangle ++(1,1);
\path[fill=black] (10,26) rectangle ++(1,1);
\path[fill=black] (11,26) rectangle ++(1,1);
\path[fill=black] (12,26) rectangle ++(1,1);
\path[fill=white] (13,26) rectangle ++(1,1);
\path[fill=black] (14,26) rectangle ++(1,1);
\path[fill=black] (15,26) rectangle ++(1,1);
\path[fill=black] (16,26) rectangle ++(1,1);
\path[fill=white] (17,26) rectangle ++(1,1);
\path[fill=black] (18,26) rectangle ++(1,1);
\path[fill=black] (19,26) rectangle ++(1,1);
\path[fill=black] (20,26) rectangle ++(1,1);
\path[fill=white] (21,26) rectangle ++(1,1);
\path[fill=black] (22,26) rectangle ++(1,1);
\path[fill=black] (23,26) rectangle ++(1,1);
\path[fill=black] (24,26) rectangle ++(1,1);
\path[fill=black] (25,26) rectangle ++(1,1);
\path[fill=black] (26,26) rectangle ++(1,1);
\path[fill=black] (27,26) rectangle ++(1,1);
\path[fill=white] (0,25) rectangle ++(1,1);
\path[fill=white] (1,25) rectangle ++(1,1);
\path[fill=white] (2,25) rectangle ++(1,1);
\path[fill=white] (3,25) rectangle ++(1,1);
\path[fill=black] (4,25) rectangle ++(1,1);
\path[fill=black] (5,25) rectangle ++(1,1);
\path[fill=white] (6,25) rectangle ++(1,1);
\path[fill=black] (7,25) rectangle ++(1,1);
\path[fill=black] (8,25) rectangle ++(1,1);
\path[fill=black] (9,25) rectangle ++(1,1);
\path[fill=black] (10,25) rectangle ++(1,1);
\path[fill=black] (11,25) rectangle ++(1,1);
\path[fill=black] (12,25) rectangle ++(1,1);
\path[fill=black] (13,25) rectangle ++(1,1);
\path[fill=black] (14,25) rectangle ++(1,1);
\path[fill=black] (15,25) rectangle ++(1,1);
\path[fill=black] (16,25) rectangle ++(1,1);
\path[fill=black] (17,25) rectangle ++(1,1);
\path[fill=black] (18,25) rectangle ++(1,1);
\path[fill=black] (19,25) rectangle ++(1,1);
\path[fill=black] (20,25) rectangle ++(1,1);
\path[fill=black] (21,25) rectangle ++(1,1);
\path[fill=black] (22,25) rectangle ++(1,1);
\path[fill=black] (23,25) rectangle ++(1,1);
\path[fill=black] (24,25) rectangle ++(1,1);
\path[fill=black] (25,25) rectangle ++(1,1);
\path[fill=black] (26,25) rectangle ++(1,1);
\path[fill=black] (27,25) rectangle ++(1,1);
\path[fill=white] (0,24) rectangle ++(1,1);
\path[fill=white] (1,24) rectangle ++(1,1);
\path[fill=white] (2,24) rectangle ++(1,1);
\path[fill=white] (3,24) rectangle ++(1,1);
\path[fill=white] (4,24) rectangle ++(1,1);
\path[fill=white] (5,24) rectangle ++(1,1);
\path[fill=black] (6,24) rectangle ++(1,1);
\path[fill=black] (7,24) rectangle ++(1,1);
\path[fill=white] (8,24) rectangle ++(1,1);
\path[fill=white] (9,24) rectangle ++(1,1);
\path[fill=black] (10,24) rectangle ++(1,1);
\path[fill=black] (11,24) rectangle ++(1,1);
\path[fill=white] (12,24) rectangle ++(1,1);
\path[fill=white] (13,24) rectangle ++(1,1);
\path[fill=black] (14,24) rectangle ++(1,1);
\path[fill=black] (15,24) rectangle ++(1,1);
\path[fill=white] (16,24) rectangle ++(1,1);
\path[fill=white] (17,24) rectangle ++(1,1);
\path[fill=black] (18,24) rectangle ++(1,1);
\path[fill=black] (19,24) rectangle ++(1,1);
\path[fill=white] (20,24) rectangle ++(1,1);
\path[fill=white] (21,24) rectangle ++(1,1);
\path[fill=black] (22,24) rectangle ++(1,1);
\path[fill=black] (23,24) rectangle ++(1,1);
\path[fill=white] (24,24) rectangle ++(1,1);
\path[fill=black] (25,24) rectangle ++(1,1);
\path[fill=black] (26,24) rectangle ++(1,1);
\path[fill=black] (27,24) rectangle ++(1,1);
\path[fill=white] (0,23) rectangle ++(1,1);
\path[fill=white] (1,23) rectangle ++(1,1);
\path[fill=white] (2,23) rectangle ++(1,1);
\path[fill=white] (3,23) rectangle ++(1,1);
\path[fill=white] (4,23) rectangle ++(1,1);
\path[fill=white] (5,23) rectangle ++(1,1);
\path[fill=white] (6,23) rectangle ++(1,1);
\path[fill=black] (7,23) rectangle ++(1,1);
\path[fill=black] (8,23) rectangle ++(1,1);
\path[fill=white] (9,23) rectangle ++(1,1);
\path[fill=black] (10,23) rectangle ++(1,1);
\path[fill=black] (11,23) rectangle ++(1,1);
\path[fill=black] (12,23) rectangle ++(1,1);
\path[fill=white] (13,23) rectangle ++(1,1);
\path[fill=black] (14,23) rectangle ++(1,1);
\path[fill=black] (15,23) rectangle ++(1,1);
\path[fill=black] (16,23) rectangle ++(1,1);
\path[fill=white] (17,23) rectangle ++(1,1);
\path[fill=black] (18,23) rectangle ++(1,1);
\path[fill=black] (19,23) rectangle ++(1,1);
\path[fill=black] (20,23) rectangle ++(1,1);
\path[fill=white] (21,23) rectangle ++(1,1);
\path[fill=black] (22,23) rectangle ++(1,1);
\path[fill=black] (23,23) rectangle ++(1,1);
\path[fill=black] (24,23) rectangle ++(1,1);
\path[fill=black] (25,23) rectangle ++(1,1);
\path[fill=black] (26,23) rectangle ++(1,1);
\path[fill=black] (27,23) rectangle ++(1,1);
\path[fill=white] (0,22) rectangle ++(1,1);
\path[fill=white] (1,22) rectangle ++(1,1);
\path[fill=white] (2,22) rectangle ++(1,1);
\path[fill=white] (3,22) rectangle ++(1,1);
\path[fill=white] (4,22) rectangle ++(1,1);
\path[fill=white] (5,22) rectangle ++(1,1);
\path[fill=white] (6,22) rectangle ++(1,1);
\path[fill=white] (7,22) rectangle ++(1,1);
\path[fill=black] (8,22) rectangle ++(1,1);
\path[fill=black] (9,22) rectangle ++(1,1);
\path[fill=white] (10,22) rectangle ++(1,1);
\path[fill=black] (11,22) rectangle ++(1,1);
\path[fill=black] (12,22) rectangle ++(1,1);
\path[fill=black] (13,22) rectangle ++(1,1);
\path[fill=black] (14,22) rectangle ++(1,1);
\path[fill=black] (15,22) rectangle ++(1,1);
\path[fill=black] (16,22) rectangle ++(1,1);
\path[fill=black] (17,22) rectangle ++(1,1);
\path[fill=black] (18,22) rectangle ++(1,1);
\path[fill=black] (19,22) rectangle ++(1,1);
\path[fill=black] (20,22) rectangle ++(1,1);
\path[fill=black] (21,22) rectangle ++(1,1);
\path[fill=black] (22,22) rectangle ++(1,1);
\path[fill=black] (23,22) rectangle ++(1,1);
\path[fill=black] (24,22) rectangle ++(1,1);
\path[fill=black] (25,22) rectangle ++(1,1);
\path[fill=black] (26,22) rectangle ++(1,1);
\path[fill=black] (27,22) rectangle ++(1,1);
\path[fill=white] (0,21) rectangle ++(1,1);
\path[fill=white] (1,21) rectangle ++(1,1);
\path[fill=white] (2,21) rectangle ++(1,1);
\path[fill=white] (3,21) rectangle ++(1,1);
\path[fill=white] (4,21) rectangle ++(1,1);
\path[fill=white] (5,21) rectangle ++(1,1);
\path[fill=white] (6,21) rectangle ++(1,1);
\path[fill=white] (7,21) rectangle ++(1,1);
\path[fill=white] (8,21) rectangle ++(1,1);
\path[fill=white] (9,21) rectangle ++(1,1);
\path[fill=black] (10,21) rectangle ++(1,1);
\path[fill=white] (11,21) rectangle ++(1,1);
\path[fill=white] (12,21) rectangle ++(1,1);
\path[fill=white] (13,21) rectangle ++(1,1);
\path[fill=black] (14,21) rectangle ++(1,1);
\path[fill=white] (15,21) rectangle ++(1,1);
\path[fill=white] (16,21) rectangle ++(1,1);
\path[fill=white] (17,21) rectangle ++(1,1);
\path[fill=black] (18,21) rectangle ++(1,1);
\path[fill=white] (19,21) rectangle ++(1,1);
\path[fill=white] (20,21) rectangle ++(1,1);
\path[fill=white] (21,21) rectangle ++(1,1);
\path[fill=black] (22,21) rectangle ++(1,1);
\path[fill=white] (23,21) rectangle ++(1,1);
\path[fill=white] (24,21) rectangle ++(1,1);
\path[fill=black] (25,21) rectangle ++(1,1);
\path[fill=white] (26,21) rectangle ++(1,1);
\path[fill=black] (27,21) rectangle ++(1,1);
\path[fill=white] (0,20) rectangle ++(1,1);
\path[fill=white] (1,20) rectangle ++(1,1);
\path[fill=white] (2,20) rectangle ++(1,1);
\path[fill=white] (3,20) rectangle ++(1,1);
\path[fill=white] (4,20) rectangle ++(1,1);
\path[fill=white] (5,20) rectangle ++(1,1);
\path[fill=white] (6,20) rectangle ++(1,1);
\path[fill=white] (7,20) rectangle ++(1,1);
\path[fill=white] (8,20) rectangle ++(1,1);
\path[fill=white] (9,20) rectangle ++(1,1);
\path[fill=black] (10,20) rectangle ++(1,1);
\path[fill=black] (11,20) rectangle ++(1,1);
\path[fill=white] (12,20) rectangle ++(1,1);
\path[fill=white] (13,20) rectangle ++(1,1);
\path[fill=black] (14,20) rectangle ++(1,1);
\path[fill=black] (15,20) rectangle ++(1,1);
\path[fill=white] (16,20) rectangle ++(1,1);
\path[fill=white] (17,20) rectangle ++(1,1);
\path[fill=black] (18,20) rectangle ++(1,1);
\path[fill=black] (19,20) rectangle ++(1,1);
\path[fill=white] (20,20) rectangle ++(1,1);
\path[fill=white] (21,20) rectangle ++(1,1);
\path[fill=black] (22,20) rectangle ++(1,1);
\path[fill=black] (23,20) rectangle ++(1,1);
\path[fill=white] (24,20) rectangle ++(1,1);
\path[fill=black] (25,20) rectangle ++(1,1);
\path[fill=black] (26,20) rectangle ++(1,1);
\path[fill=black] (27,20) rectangle ++(1,1);
\path[fill=white] (0,19) rectangle ++(1,1);
\path[fill=white] (1,19) rectangle ++(1,1);
\path[fill=white] (2,19) rectangle ++(1,1);
\path[fill=white] (3,19) rectangle ++(1,1);
\path[fill=white] (4,19) rectangle ++(1,1);
\path[fill=white] (5,19) rectangle ++(1,1);
\path[fill=white] (6,19) rectangle ++(1,1);
\path[fill=white] (7,19) rectangle ++(1,1);
\path[fill=white] (8,19) rectangle ++(1,1);
\path[fill=white] (9,19) rectangle ++(1,1);
\path[fill=white] (10,19) rectangle ++(1,1);
\path[fill=black] (11,19) rectangle ++(1,1);
\path[fill=black] (12,19) rectangle ++(1,1);
\path[fill=white] (13,19) rectangle ++(1,1);
\path[fill=black] (14,19) rectangle ++(1,1);
\path[fill=black] (15,19) rectangle ++(1,1);
\path[fill=black] (16,19) rectangle ++(1,1);
\path[fill=white] (17,19) rectangle ++(1,1);
\path[fill=black] (18,19) rectangle ++(1,1);
\path[fill=black] (19,19) rectangle ++(1,1);
\path[fill=black] (20,19) rectangle ++(1,1);
\path[fill=white] (21,19) rectangle ++(1,1);
\path[fill=black] (22,19) rectangle ++(1,1);
\path[fill=black] (23,19) rectangle ++(1,1);
\path[fill=black] (24,19) rectangle ++(1,1);
\path[fill=black] (25,19) rectangle ++(1,1);
\path[fill=black] (26,19) rectangle ++(1,1);
\path[fill=black] (27,19) rectangle ++(1,1);
\path[fill=white] (0,18) rectangle ++(1,1);
\path[fill=white] (1,18) rectangle ++(1,1);
\path[fill=white] (2,18) rectangle ++(1,1);
\path[fill=white] (3,18) rectangle ++(1,1);
\path[fill=white] (4,18) rectangle ++(1,1);
\path[fill=white] (5,18) rectangle ++(1,1);
\path[fill=white] (6,18) rectangle ++(1,1);
\path[fill=white] (7,18) rectangle ++(1,1);
\path[fill=white] (8,18) rectangle ++(1,1);
\path[fill=white] (9,18) rectangle ++(1,1);
\path[fill=white] (10,18) rectangle ++(1,1);
\path[fill=white] (11,18) rectangle ++(1,1);
\path[fill=black] (12,18) rectangle ++(1,1);
\path[fill=black] (13,18) rectangle ++(1,1);
\path[fill=white] (14,18) rectangle ++(1,1);
\path[fill=black] (15,18) rectangle ++(1,1);
\path[fill=black] (16,18) rectangle ++(1,1);
\path[fill=black] (17,18) rectangle ++(1,1);
\path[fill=black] (18,18) rectangle ++(1,1);
\path[fill=black] (19,18) rectangle ++(1,1);
\path[fill=black] (20,18) rectangle ++(1,1);
\path[fill=black] (21,18) rectangle ++(1,1);
\path[fill=black] (22,18) rectangle ++(1,1);
\path[fill=black] (23,18) rectangle ++(1,1);
\path[fill=black] (24,18) rectangle ++(1,1);
\path[fill=black] (25,18) rectangle ++(1,1);
\path[fill=black] (26,18) rectangle ++(1,1);
\path[fill=black] (27,18) rectangle ++(1,1);
\path[fill=white] (0,17) rectangle ++(1,1);
\path[fill=white] (1,17) rectangle ++(1,1);
\path[fill=white] (2,17) rectangle ++(1,1);
\path[fill=white] (3,17) rectangle ++(1,1);
\path[fill=white] (4,17) rectangle ++(1,1);
\path[fill=white] (5,17) rectangle ++(1,1);
\path[fill=white] (6,17) rectangle ++(1,1);
\path[fill=white] (7,17) rectangle ++(1,1);
\path[fill=white] (8,17) rectangle ++(1,1);
\path[fill=white] (9,17) rectangle ++(1,1);
\path[fill=white] (10,17) rectangle ++(1,1);
\path[fill=white] (11,17) rectangle ++(1,1);
\path[fill=white] (12,17) rectangle ++(1,1);
\path[fill=white] (13,17) rectangle ++(1,1);
\path[fill=black] (14,17) rectangle ++(1,1);
\path[fill=white] (15,17) rectangle ++(1,1);
\path[fill=white] (16,17) rectangle ++(1,1);
\path[fill=white] (17,17) rectangle ++(1,1);
\path[fill=black] (18,17) rectangle ++(1,1);
\path[fill=white] (19,17) rectangle ++(1,1);
\path[fill=white] (20,17) rectangle ++(1,1);
\path[fill=white] (21,17) rectangle ++(1,1);
\path[fill=black] (22,17) rectangle ++(1,1);
\path[fill=white] (23,17) rectangle ++(1,1);
\path[fill=white] (24,17) rectangle ++(1,1);
\path[fill=black] (25,17) rectangle ++(1,1);
\path[fill=white] (26,17) rectangle ++(1,1);
\path[fill=black] (27,17) rectangle ++(1,1);
\path[fill=white] (0,16) rectangle ++(1,1);
\path[fill=white] (1,16) rectangle ++(1,1);
\path[fill=white] (2,16) rectangle ++(1,1);
\path[fill=white] (3,16) rectangle ++(1,1);
\path[fill=white] (4,16) rectangle ++(1,1);
\path[fill=white] (5,16) rectangle ++(1,1);
\path[fill=white] (6,16) rectangle ++(1,1);
\path[fill=white] (7,16) rectangle ++(1,1);
\path[fill=white] (8,16) rectangle ++(1,1);
\path[fill=white] (9,16) rectangle ++(1,1);
\path[fill=white] (10,16) rectangle ++(1,1);
\path[fill=white] (11,16) rectangle ++(1,1);
\path[fill=white] (12,16) rectangle ++(1,1);
\path[fill=white] (13,16) rectangle ++(1,1);
\path[fill=black] (14,16) rectangle ++(1,1);
\path[fill=black] (15,16) rectangle ++(1,1);
\path[fill=white] (16,16) rectangle ++(1,1);
\path[fill=white] (17,16) rectangle ++(1,1);
\path[fill=black] (18,16) rectangle ++(1,1);
\path[fill=black] (19,16) rectangle ++(1,1);
\path[fill=white] (20,16) rectangle ++(1,1);
\path[fill=white] (21,16) rectangle ++(1,1);
\path[fill=black] (22,16) rectangle ++(1,1);
\path[fill=black] (23,16) rectangle ++(1,1);
\path[fill=white] (24,16) rectangle ++(1,1);
\path[fill=black] (25,16) rectangle ++(1,1);
\path[fill=black] (26,16) rectangle ++(1,1);
\path[fill=black] (27,16) rectangle ++(1,1);
\path[fill=white] (0,15) rectangle ++(1,1);
\path[fill=white] (1,15) rectangle ++(1,1);
\path[fill=white] (2,15) rectangle ++(1,1);
\path[fill=white] (3,15) rectangle ++(1,1);
\path[fill=white] (4,15) rectangle ++(1,1);
\path[fill=white] (5,15) rectangle ++(1,1);
\path[fill=white] (6,15) rectangle ++(1,1);
\path[fill=white] (7,15) rectangle ++(1,1);
\path[fill=white] (8,15) rectangle ++(1,1);
\path[fill=white] (9,15) rectangle ++(1,1);
\path[fill=white] (10,15) rectangle ++(1,1);
\path[fill=white] (11,15) rectangle ++(1,1);
\path[fill=white] (12,15) rectangle ++(1,1);
\path[fill=white] (13,15) rectangle ++(1,1);
\path[fill=white] (14,15) rectangle ++(1,1);
\path[fill=black] (15,15) rectangle ++(1,1);
\path[fill=black] (16,15) rectangle ++(1,1);
\path[fill=white] (17,15) rectangle ++(1,1);
\path[fill=black] (18,15) rectangle ++(1,1);
\path[fill=black] (19,15) rectangle ++(1,1);
\path[fill=black] (20,15) rectangle ++(1,1);
\path[fill=white] (21,15) rectangle ++(1,1);
\path[fill=black] (22,15) rectangle ++(1,1);
\path[fill=black] (23,15) rectangle ++(1,1);
\path[fill=black] (24,15) rectangle ++(1,1);
\path[fill=black] (25,15) rectangle ++(1,1);
\path[fill=black] (26,15) rectangle ++(1,1);
\path[fill=black] (27,15) rectangle ++(1,1);
\path[fill=white] (0,14) rectangle ++(1,1);
\path[fill=white] (1,14) rectangle ++(1,1);
\path[fill=white] (2,14) rectangle ++(1,1);
\path[fill=white] (3,14) rectangle ++(1,1);
\path[fill=white] (4,14) rectangle ++(1,1);
\path[fill=white] (5,14) rectangle ++(1,1);
\path[fill=white] (6,14) rectangle ++(1,1);
\path[fill=white] (7,14) rectangle ++(1,1);
\path[fill=white] (8,14) rectangle ++(1,1);
\path[fill=white] (9,14) rectangle ++(1,1);
\path[fill=white] (10,14) rectangle ++(1,1);
\path[fill=white] (11,14) rectangle ++(1,1);
\path[fill=white] (12,14) rectangle ++(1,1);
\path[fill=white] (13,14) rectangle ++(1,1);
\path[fill=white] (14,14) rectangle ++(1,1);
\path[fill=white] (15,14) rectangle ++(1,1);
\path[fill=black] (16,14) rectangle ++(1,1);
\path[fill=black] (17,14) rectangle ++(1,1);
\path[fill=white] (18,14) rectangle ++(1,1);
\path[fill=black] (19,14) rectangle ++(1,1);
\path[fill=black] (20,14) rectangle ++(1,1);
\path[fill=black] (21,14) rectangle ++(1,1);
\path[fill=black] (22,14) rectangle ++(1,1);
\path[fill=black] (23,14) rectangle ++(1,1);
\path[fill=black] (24,14) rectangle ++(1,1);
\path[fill=black] (25,14) rectangle ++(1,1);
\path[fill=black] (26,14) rectangle ++(1,1);
\path[fill=black] (27,14) rectangle ++(1,1);
\path[fill=white] (0,13) rectangle ++(1,1);
\path[fill=white] (1,13) rectangle ++(1,1);
\path[fill=white] (2,13) rectangle ++(1,1);
\path[fill=white] (3,13) rectangle ++(1,1);
\path[fill=white] (4,13) rectangle ++(1,1);
\path[fill=white] (5,13) rectangle ++(1,1);
\path[fill=white] (6,13) rectangle ++(1,1);
\path[fill=white] (7,13) rectangle ++(1,1);
\path[fill=white] (8,13) rectangle ++(1,1);
\path[fill=white] (9,13) rectangle ++(1,1);
\path[fill=white] (10,13) rectangle ++(1,1);
\path[fill=white] (11,13) rectangle ++(1,1);
\path[fill=white] (12,13) rectangle ++(1,1);
\path[fill=white] (13,13) rectangle ++(1,1);
\path[fill=white] (14,13) rectangle ++(1,1);
\path[fill=white] (15,13) rectangle ++(1,1);
\path[fill=white] (16,13) rectangle ++(1,1);
\path[fill=white] (17,13) rectangle ++(1,1);
\path[fill=black] (18,13) rectangle ++(1,1);
\path[fill=white] (19,13) rectangle ++(1,1);
\path[fill=white] (20,13) rectangle ++(1,1);
\path[fill=white] (21,13) rectangle ++(1,1);
\path[fill=black] (22,13) rectangle ++(1,1);
\path[fill=white] (23,13) rectangle ++(1,1);
\path[fill=white] (24,13) rectangle ++(1,1);
\path[fill=black] (25,13) rectangle ++(1,1);
\path[fill=white] (26,13) rectangle ++(1,1);
\path[fill=black] (27,13) rectangle ++(1,1);
\path[fill=white] (0,12) rectangle ++(1,1);
\path[fill=white] (1,12) rectangle ++(1,1);
\path[fill=white] (2,12) rectangle ++(1,1);
\path[fill=white] (3,12) rectangle ++(1,1);
\path[fill=white] (4,12) rectangle ++(1,1);
\path[fill=white] (5,12) rectangle ++(1,1);
\path[fill=white] (6,12) rectangle ++(1,1);
\path[fill=white] (7,12) rectangle ++(1,1);
\path[fill=white] (8,12) rectangle ++(1,1);
\path[fill=white] (9,12) rectangle ++(1,1);
\path[fill=white] (10,12) rectangle ++(1,1);
\path[fill=white] (11,12) rectangle ++(1,1);
\path[fill=white] (12,12) rectangle ++(1,1);
\path[fill=white] (13,12) rectangle ++(1,1);
\path[fill=white] (14,12) rectangle ++(1,1);
\path[fill=white] (15,12) rectangle ++(1,1);
\path[fill=white] (16,12) rectangle ++(1,1);
\path[fill=white] (17,12) rectangle ++(1,1);
\path[fill=black] (18,12) rectangle ++(1,1);
\path[fill=black] (19,12) rectangle ++(1,1);
\path[fill=white] (20,12) rectangle ++(1,1);
\path[fill=white] (21,12) rectangle ++(1,1);
\path[fill=black] (22,12) rectangle ++(1,1);
\path[fill=black] (23,12) rectangle ++(1,1);
\path[fill=white] (24,12) rectangle ++(1,1);
\path[fill=black] (25,12) rectangle ++(1,1);
\path[fill=black] (26,12) rectangle ++(1,1);
\path[fill=black] (27,12) rectangle ++(1,1);
\path[fill=white] (0,11) rectangle ++(1,1);
\path[fill=white] (1,11) rectangle ++(1,1);
\path[fill=white] (2,11) rectangle ++(1,1);
\path[fill=white] (3,11) rectangle ++(1,1);
\path[fill=white] (4,11) rectangle ++(1,1);
\path[fill=white] (5,11) rectangle ++(1,1);
\path[fill=white] (6,11) rectangle ++(1,1);
\path[fill=white] (7,11) rectangle ++(1,1);
\path[fill=white] (8,11) rectangle ++(1,1);
\path[fill=white] (9,11) rectangle ++(1,1);
\path[fill=white] (10,11) rectangle ++(1,1);
\path[fill=white] (11,11) rectangle ++(1,1);
\path[fill=white] (12,11) rectangle ++(1,1);
\path[fill=white] (13,11) rectangle ++(1,1);
\path[fill=white] (14,11) rectangle ++(1,1);
\path[fill=white] (15,11) rectangle ++(1,1);
\path[fill=white] (16,11) rectangle ++(1,1);
\path[fill=white] (17,11) rectangle ++(1,1);
\path[fill=white] (18,11) rectangle ++(1,1);
\path[fill=black] (19,11) rectangle ++(1,1);
\path[fill=black] (20,11) rectangle ++(1,1);
\path[fill=white] (21,11) rectangle ++(1,1);
\path[fill=black] (22,11) rectangle ++(1,1);
\path[fill=black] (23,11) rectangle ++(1,1);
\path[fill=black] (24,11) rectangle ++(1,1);
\path[fill=black] (25,11) rectangle ++(1,1);
\path[fill=black] (26,11) rectangle ++(1,1);
\path[fill=black] (27,11) rectangle ++(1,1);
\path[fill=white] (0,10) rectangle ++(1,1);
\path[fill=white] (1,10) rectangle ++(1,1);
\path[fill=white] (2,10) rectangle ++(1,1);
\path[fill=white] (3,10) rectangle ++(1,1);
\path[fill=white] (4,10) rectangle ++(1,1);
\path[fill=white] (5,10) rectangle ++(1,1);
\path[fill=white] (6,10) rectangle ++(1,1);
\path[fill=white] (7,10) rectangle ++(1,1);
\path[fill=white] (8,10) rectangle ++(1,1);
\path[fill=white] (9,10) rectangle ++(1,1);
\path[fill=white] (10,10) rectangle ++(1,1);
\path[fill=white] (11,10) rectangle ++(1,1);
\path[fill=white] (12,10) rectangle ++(1,1);
\path[fill=white] (13,10) rectangle ++(1,1);
\path[fill=white] (14,10) rectangle ++(1,1);
\path[fill=white] (15,10) rectangle ++(1,1);
\path[fill=white] (16,10) rectangle ++(1,1);
\path[fill=white] (17,10) rectangle ++(1,1);
\path[fill=white] (18,10) rectangle ++(1,1);
\path[fill=white] (19,10) rectangle ++(1,1);
\path[fill=black] (20,10) rectangle ++(1,1);
\path[fill=black] (21,10) rectangle ++(1,1);
\path[fill=white] (22,10) rectangle ++(1,1);
\path[fill=black] (23,10) rectangle ++(1,1);
\path[fill=black] (24,10) rectangle ++(1,1);
\path[fill=black] (25,10) rectangle ++(1,1);
\path[fill=black] (26,10) rectangle ++(1,1);
\path[fill=black] (27,10) rectangle ++(1,1);
\path[fill=white] (0,9) rectangle ++(1,1);
\path[fill=white] (1,9) rectangle ++(1,1);
\path[fill=white] (2,9) rectangle ++(1,1);
\path[fill=white] (3,9) rectangle ++(1,1);
\path[fill=white] (4,9) rectangle ++(1,1);
\path[fill=white] (5,9) rectangle ++(1,1);
\path[fill=white] (6,9) rectangle ++(1,1);
\path[fill=white] (7,9) rectangle ++(1,1);
\path[fill=white] (8,9) rectangle ++(1,1);
\path[fill=white] (9,9) rectangle ++(1,1);
\path[fill=white] (10,9) rectangle ++(1,1);
\path[fill=white] (11,9) rectangle ++(1,1);
\path[fill=white] (12,9) rectangle ++(1,1);
\path[fill=white] (13,9) rectangle ++(1,1);
\path[fill=white] (14,9) rectangle ++(1,1);
\path[fill=white] (15,9) rectangle ++(1,1);
\path[fill=white] (16,9) rectangle ++(1,1);
\path[fill=white] (17,9) rectangle ++(1,1);
\path[fill=white] (18,9) rectangle ++(1,1);
\path[fill=white] (19,9) rectangle ++(1,1);
\path[fill=white] (20,9) rectangle ++(1,1);
\path[fill=white] (21,9) rectangle ++(1,1);
\path[fill=black] (22,9) rectangle ++(1,1);
\path[fill=white] (23,9) rectangle ++(1,1);
\path[fill=white] (24,9) rectangle ++(1,1);
\path[fill=black] (25,9) rectangle ++(1,1);
\path[fill=white] (26,9) rectangle ++(1,1);
\path[fill=black] (27,9) rectangle ++(1,1);
\path[fill=white] (0,8) rectangle ++(1,1);
\path[fill=white] (1,8) rectangle ++(1,1);
\path[fill=white] (2,8) rectangle ++(1,1);
\path[fill=white] (3,8) rectangle ++(1,1);
\path[fill=white] (4,8) rectangle ++(1,1);
\path[fill=white] (5,8) rectangle ++(1,1);
\path[fill=white] (6,8) rectangle ++(1,1);
\path[fill=white] (7,8) rectangle ++(1,1);
\path[fill=white] (8,8) rectangle ++(1,1);
\path[fill=white] (9,8) rectangle ++(1,1);
\path[fill=white] (10,8) rectangle ++(1,1);
\path[fill=white] (11,8) rectangle ++(1,1);
\path[fill=white] (12,8) rectangle ++(1,1);
\path[fill=white] (13,8) rectangle ++(1,1);
\path[fill=white] (14,8) rectangle ++(1,1);
\path[fill=white] (15,8) rectangle ++(1,1);
\path[fill=white] (16,8) rectangle ++(1,1);
\path[fill=white] (17,8) rectangle ++(1,1);
\path[fill=white] (18,8) rectangle ++(1,1);
\path[fill=white] (19,8) rectangle ++(1,1);
\path[fill=white] (20,8) rectangle ++(1,1);
\path[fill=white] (21,8) rectangle ++(1,1);
\path[fill=black] (22,8) rectangle ++(1,1);
\path[fill=black] (23,8) rectangle ++(1,1);
\path[fill=white] (24,8) rectangle ++(1,1);
\path[fill=black] (25,8) rectangle ++(1,1);
\path[fill=black] (26,8) rectangle ++(1,1);
\path[fill=black] (27,8) rectangle ++(1,1);
\path[fill=white] (0,7) rectangle ++(1,1);
\path[fill=white] (1,7) rectangle ++(1,1);
\path[fill=white] (2,7) rectangle ++(1,1);
\path[fill=white] (3,7) rectangle ++(1,1);
\path[fill=white] (4,7) rectangle ++(1,1);
\path[fill=white] (5,7) rectangle ++(1,1);
\path[fill=white] (6,7) rectangle ++(1,1);
\path[fill=white] (7,7) rectangle ++(1,1);
\path[fill=white] (8,7) rectangle ++(1,1);
\path[fill=white] (9,7) rectangle ++(1,1);
\path[fill=white] (10,7) rectangle ++(1,1);
\path[fill=white] (11,7) rectangle ++(1,1);
\path[fill=white] (12,7) rectangle ++(1,1);
\path[fill=white] (13,7) rectangle ++(1,1);
\path[fill=white] (14,7) rectangle ++(1,1);
\path[fill=white] (15,7) rectangle ++(1,1);
\path[fill=white] (16,7) rectangle ++(1,1);
\path[fill=white] (17,7) rectangle ++(1,1);
\path[fill=white] (18,7) rectangle ++(1,1);
\path[fill=white] (19,7) rectangle ++(1,1);
\path[fill=white] (20,7) rectangle ++(1,1);
\path[fill=white] (21,7) rectangle ++(1,1);
\path[fill=white] (22,7) rectangle ++(1,1);
\path[fill=black] (23,7) rectangle ++(1,1);
\path[fill=black] (24,7) rectangle ++(1,1);
\path[fill=black] (25,7) rectangle ++(1,1);
\path[fill=black] (26,7) rectangle ++(1,1);
\path[fill=black] (27,7) rectangle ++(1,1);
\path[fill=white] (0,6) rectangle ++(1,1);
\path[fill=white] (1,6) rectangle ++(1,1);
\path[fill=white] (2,6) rectangle ++(1,1);
\path[fill=white] (3,6) rectangle ++(1,1);
\path[fill=white] (4,6) rectangle ++(1,1);
\path[fill=white] (5,6) rectangle ++(1,1);
\path[fill=white] (6,6) rectangle ++(1,1);
\path[fill=white] (7,6) rectangle ++(1,1);
\path[fill=white] (8,6) rectangle ++(1,1);
\path[fill=white] (9,6) rectangle ++(1,1);
\path[fill=white] (10,6) rectangle ++(1,1);
\path[fill=white] (11,6) rectangle ++(1,1);
\path[fill=white] (12,6) rectangle ++(1,1);
\path[fill=white] (13,6) rectangle ++(1,1);
\path[fill=white] (14,6) rectangle ++(1,1);
\path[fill=white] (15,6) rectangle ++(1,1);
\path[fill=white] (16,6) rectangle ++(1,1);
\path[fill=white] (17,6) rectangle ++(1,1);
\path[fill=white] (18,6) rectangle ++(1,1);
\path[fill=white] (19,6) rectangle ++(1,1);
\path[fill=white] (20,6) rectangle ++(1,1);
\path[fill=white] (21,6) rectangle ++(1,1);
\path[fill=white] (22,6) rectangle ++(1,1);
\path[fill=white] (23,6) rectangle ++(1,1);
\path[fill=black] (24,6) rectangle ++(1,1);
\path[fill=white] (25,6) rectangle ++(1,1);
\path[fill=black] (26,6) rectangle ++(1,1);
\path[fill=black] (27,6) rectangle ++(1,1);
\path[fill=white] (0,5) rectangle ++(1,1);
\path[fill=white] (1,5) rectangle ++(1,1);
\path[fill=white] (2,5) rectangle ++(1,1);
\path[fill=white] (3,5) rectangle ++(1,1);
\path[fill=white] (4,5) rectangle ++(1,1);
\path[fill=white] (5,5) rectangle ++(1,1);
\path[fill=white] (6,5) rectangle ++(1,1);
\path[fill=white] (7,5) rectangle ++(1,1);
\path[fill=white] (8,5) rectangle ++(1,1);
\path[fill=white] (9,5) rectangle ++(1,1);
\path[fill=white] (10,5) rectangle ++(1,1);
\path[fill=white] (11,5) rectangle ++(1,1);
\path[fill=white] (12,5) rectangle ++(1,1);
\path[fill=white] (13,5) rectangle ++(1,1);
\path[fill=white] (14,5) rectangle ++(1,1);
\path[fill=white] (15,5) rectangle ++(1,1);
\path[fill=white] (16,5) rectangle ++(1,1);
\path[fill=white] (17,5) rectangle ++(1,1);
\path[fill=white] (18,5) rectangle ++(1,1);
\path[fill=white] (19,5) rectangle ++(1,1);
\path[fill=white] (20,5) rectangle ++(1,1);
\path[fill=white] (21,5) rectangle ++(1,1);
\path[fill=white] (22,5) rectangle ++(1,1);
\path[fill=white] (23,5) rectangle ++(1,1);
\path[fill=white] (24,5) rectangle ++(1,1);
\path[fill=black] (25,5) rectangle ++(1,1);
\path[fill=white] (26,5) rectangle ++(1,1);
\path[fill=black] (27,5) rectangle ++(1,1);
\path[fill=white] (0,4) rectangle ++(1,1);
\path[fill=white] (1,4) rectangle ++(1,1);
\path[fill=white] (2,4) rectangle ++(1,1);
\path[fill=white] (3,4) rectangle ++(1,1);
\path[fill=white] (4,4) rectangle ++(1,1);
\path[fill=white] (5,4) rectangle ++(1,1);
\path[fill=white] (6,4) rectangle ++(1,1);
\path[fill=white] (7,4) rectangle ++(1,1);
\path[fill=white] (8,4) rectangle ++(1,1);
\path[fill=white] (9,4) rectangle ++(1,1);
\path[fill=white] (10,4) rectangle ++(1,1);
\path[fill=white] (11,4) rectangle ++(1,1);
\path[fill=white] (12,4) rectangle ++(1,1);
\path[fill=white] (13,4) rectangle ++(1,1);
\path[fill=white] (14,4) rectangle ++(1,1);
\path[fill=white] (15,4) rectangle ++(1,1);
\path[fill=white] (16,4) rectangle ++(1,1);
\path[fill=white] (17,4) rectangle ++(1,1);
\path[fill=white] (18,4) rectangle ++(1,1);
\path[fill=white] (19,4) rectangle ++(1,1);
\path[fill=white] (20,4) rectangle ++(1,1);
\path[fill=white] (21,4) rectangle ++(1,1);
\path[fill=white] (22,4) rectangle ++(1,1);
\path[fill=white] (23,4) rectangle ++(1,1);
\path[fill=white] (24,4) rectangle ++(1,1);
\path[fill=black] (25,4) rectangle ++(1,1);
\path[fill=black] (26,4) rectangle ++(1,1);
\path[fill=black] (27,4) rectangle ++(1,1);
\path[fill=white] (0,3) rectangle ++(1,1);
\path[fill=white] (1,3) rectangle ++(1,1);
\path[fill=white] (2,3) rectangle ++(1,1);
\path[fill=white] (3,3) rectangle ++(1,1);
\path[fill=white] (4,3) rectangle ++(1,1);
\path[fill=white] (5,3) rectangle ++(1,1);
\path[fill=white] (6,3) rectangle ++(1,1);
\path[fill=white] (7,3) rectangle ++(1,1);
\path[fill=white] (8,3) rectangle ++(1,1);
\path[fill=white] (9,3) rectangle ++(1,1);
\path[fill=white] (10,3) rectangle ++(1,1);
\path[fill=white] (11,3) rectangle ++(1,1);
\path[fill=white] (12,3) rectangle ++(1,1);
\path[fill=white] (13,3) rectangle ++(1,1);
\path[fill=white] (14,3) rectangle ++(1,1);
\path[fill=white] (15,3) rectangle ++(1,1);
\path[fill=white] (16,3) rectangle ++(1,1);
\path[fill=white] (17,3) rectangle ++(1,1);
\path[fill=white] (18,3) rectangle ++(1,1);
\path[fill=white] (19,3) rectangle ++(1,1);
\path[fill=white] (20,3) rectangle ++(1,1);
\path[fill=white] (21,3) rectangle ++(1,1);
\path[fill=white] (22,3) rectangle ++(1,1);
\path[fill=white] (23,3) rectangle ++(1,1);
\path[fill=white] (24,3) rectangle ++(1,1);
\path[fill=white] (25,3) rectangle ++(1,1);
\path[fill=black] (26,3) rectangle ++(1,1);
\path[fill=black] (27,3) rectangle ++(1,1);
\path[fill=white] (0,2) rectangle ++(1,1);
\path[fill=white] (1,2) rectangle ++(1,1);
\path[fill=white] (2,2) rectangle ++(1,1);
\path[fill=white] (3,2) rectangle ++(1,1);
\path[fill=white] (4,2) rectangle ++(1,1);
\path[fill=white] (5,2) rectangle ++(1,1);
\path[fill=white] (6,2) rectangle ++(1,1);
\path[fill=white] (7,2) rectangle ++(1,1);
\path[fill=white] (8,2) rectangle ++(1,1);
\path[fill=white] (9,2) rectangle ++(1,1);
\path[fill=white] (10,2) rectangle ++(1,1);
\path[fill=white] (11,2) rectangle ++(1,1);
\path[fill=white] (12,2) rectangle ++(1,1);
\path[fill=white] (13,2) rectangle ++(1,1);
\path[fill=white] (14,2) rectangle ++(1,1);
\path[fill=white] (15,2) rectangle ++(1,1);
\path[fill=white] (16,2) rectangle ++(1,1);
\path[fill=white] (17,2) rectangle ++(1,1);
\path[fill=white] (18,2) rectangle ++(1,1);
\path[fill=white] (19,2) rectangle ++(1,1);
\path[fill=white] (20,2) rectangle ++(1,1);
\path[fill=white] (21,2) rectangle ++(1,1);
\path[fill=white] (22,2) rectangle ++(1,1);
\path[fill=white] (23,2) rectangle ++(1,1);
\path[fill=white] (24,2) rectangle ++(1,1);
\path[fill=white] (25,2) rectangle ++(1,1);
\path[fill=white] (26,2) rectangle ++(1,1);
\path[fill=black] (27,2) rectangle ++(1,1);
\path[fill=white] (0,1) rectangle ++(1,1);
\path[fill=white] (1,1) rectangle ++(1,1);
\path[fill=white] (2,1) rectangle ++(1,1);
\path[fill=white] (3,1) rectangle ++(1,1);
\path[fill=white] (4,1) rectangle ++(1,1);
\path[fill=white] (5,1) rectangle ++(1,1);
\path[fill=white] (6,1) rectangle ++(1,1);
\path[fill=white] (7,1) rectangle ++(1,1);
\path[fill=white] (8,1) rectangle ++(1,1);
\path[fill=white] (9,1) rectangle ++(1,1);
\path[fill=white] (10,1) rectangle ++(1,1);
\path[fill=white] (11,1) rectangle ++(1,1);
\path[fill=white] (12,1) rectangle ++(1,1);
\path[fill=white] (13,1) rectangle ++(1,1);
\path[fill=white] (14,1) rectangle ++(1,1);
\path[fill=white] (15,1) rectangle ++(1,1);
\path[fill=white] (16,1) rectangle ++(1,1);
\path[fill=white] (17,1) rectangle ++(1,1);
\path[fill=white] (18,1) rectangle ++(1,1);
\path[fill=white] (19,1) rectangle ++(1,1);
\path[fill=white] (20,1) rectangle ++(1,1);
\path[fill=white] (21,1) rectangle ++(1,1);
\path[fill=white] (22,1) rectangle ++(1,1);
\path[fill=white] (23,1) rectangle ++(1,1);
\path[fill=white] (24,1) rectangle ++(1,1);
\path[fill=white] (25,1) rectangle ++(1,1);
\path[fill=white] (26,1) rectangle ++(1,1);
\path[fill=black] (27,1) rectangle ++(1,1);
\path[fill=white] (0,0) rectangle ++(1,1);
\path[fill=white] (1,0) rectangle ++(1,1);
\path[fill=white] (2,0) rectangle ++(1,1);
\path[fill=white] (3,0) rectangle ++(1,1);
\path[fill=white] (4,0) rectangle ++(1,1);
\path[fill=white] (5,0) rectangle ++(1,1);
\path[fill=white] (6,0) rectangle ++(1,1);
\path[fill=white] (7,0) rectangle ++(1,1);
\path[fill=white] (8,0) rectangle ++(1,1);
\path[fill=white] (9,0) rectangle ++(1,1);
\path[fill=white] (10,0) rectangle ++(1,1);
\path[fill=white] (11,0) rectangle ++(1,1);
\path[fill=white] (12,0) rectangle ++(1,1);
\path[fill=white] (13,0) rectangle ++(1,1);
\path[fill=white] (14,0) rectangle ++(1,1);
\path[fill=white] (15,0) rectangle ++(1,1);
\path[fill=white] (16,0) rectangle ++(1,1);
\path[fill=white] (17,0) rectangle ++(1,1);
\path[fill=white] (18,0) rectangle ++(1,1);
\path[fill=white] (19,0) rectangle ++(1,1);
\path[fill=white] (20,0) rectangle ++(1,1);
\path[fill=white] (21,0) rectangle ++(1,1);
\path[fill=white] (22,0) rectangle ++(1,1);
\path[fill=white] (23,0) rectangle ++(1,1);
\path[fill=white] (24,0) rectangle ++(1,1);
\path[fill=white] (25,0) rectangle ++(1,1);
\path[fill=white] (26,0) rectangle ++(1,1);
\path[fill=white] (27,0) rectangle ++(1,1);
\end{tikzpicture}
  }
\end{center}
  \caption{Lev(|\sigma|=6, \Delta=3) adjacency and reachability matrices.}
\end{figure}

\begin{figure}[H]
  \begin{center}
    \resizebox{0.45\textwidth}{!}{%
      \begin{tikzpicture}[x=0.3cm, y=0.3cm, draw=gray, very thin]
\path[fill=white] (0,34) rectangle ++(1,1);
\path[fill=black] (1,34) rectangle ++(1,1);
\path[fill=black] (2,34) rectangle ++(1,1);
\path[fill=white] (3,34) rectangle ++(1,1);
\path[fill=black] (4,34) rectangle ++(1,1);
\path[fill=white] (5,34) rectangle ++(1,1);
\path[fill=white] (6,34) rectangle ++(1,1);
\path[fill=white] (7,34) rectangle ++(1,1);
\path[fill=black] (8,34) rectangle ++(1,1);
\path[fill=white] (9,34) rectangle ++(1,1);
\path[fill=white] (10,34) rectangle ++(1,1);
\path[fill=white] (11,34) rectangle ++(1,1);
\path[fill=white] (12,34) rectangle ++(1,1);
\path[fill=white] (13,34) rectangle ++(1,1);
\path[fill=white] (14,34) rectangle ++(1,1);
\path[fill=white] (15,34) rectangle ++(1,1);
\path[fill=white] (16,34) rectangle ++(1,1);
\path[fill=black] (17,34) rectangle ++(1,1);
\path[fill=white] (18,34) rectangle ++(1,1);
\path[fill=white] (19,34) rectangle ++(1,1);
\path[fill=white] (20,34) rectangle ++(1,1);
\path[fill=white] (21,34) rectangle ++(1,1);
\path[fill=white] (22,34) rectangle ++(1,1);
\path[fill=white] (23,34) rectangle ++(1,1);
\path[fill=white] (24,34) rectangle ++(1,1);
\path[fill=white] (25,34) rectangle ++(1,1);
\path[fill=black] (26,34) rectangle ++(1,1);
\path[fill=white] (27,34) rectangle ++(1,1);
\path[fill=white] (28,34) rectangle ++(1,1);
\path[fill=white] (29,34) rectangle ++(1,1);
\path[fill=white] (30,34) rectangle ++(1,1);
\path[fill=white] (31,34) rectangle ++(1,1);
\path[fill=black] (32,34) rectangle ++(1,1);
\path[fill=white] (33,34) rectangle ++(1,1);
\path[fill=white] (34,34) rectangle ++(1,1);
\path[fill=white] (0,33) rectangle ++(1,1);
\path[fill=white] (1,33) rectangle ++(1,1);
\path[fill=white] (2,33) rectangle ++(1,1);
\path[fill=black] (3,33) rectangle ++(1,1);
\path[fill=black] (4,33) rectangle ++(1,1);
\path[fill=white] (5,33) rectangle ++(1,1);
\path[fill=white] (6,33) rectangle ++(1,1);
\path[fill=black] (7,33) rectangle ++(1,1);
\path[fill=white] (8,33) rectangle ++(1,1);
\path[fill=white] (9,33) rectangle ++(1,1);
\path[fill=white] (10,33) rectangle ++(1,1);
\path[fill=white] (11,33) rectangle ++(1,1);
\path[fill=black] (12,33) rectangle ++(1,1);
\path[fill=white] (13,33) rectangle ++(1,1);
\path[fill=white] (14,33) rectangle ++(1,1);
\path[fill=white] (15,33) rectangle ++(1,1);
\path[fill=white] (16,33) rectangle ++(1,1);
\path[fill=white] (17,33) rectangle ++(1,1);
\path[fill=white] (18,33) rectangle ++(1,1);
\path[fill=white] (19,33) rectangle ++(1,1);
\path[fill=white] (20,33) rectangle ++(1,1);
\path[fill=black] (21,33) rectangle ++(1,1);
\path[fill=white] (22,33) rectangle ++(1,1);
\path[fill=white] (23,33) rectangle ++(1,1);
\path[fill=white] (24,33) rectangle ++(1,1);
\path[fill=white] (25,33) rectangle ++(1,1);
\path[fill=white] (26,33) rectangle ++(1,1);
\path[fill=white] (27,33) rectangle ++(1,1);
\path[fill=white] (28,33) rectangle ++(1,1);
\path[fill=black] (29,33) rectangle ++(1,1);
\path[fill=white] (30,33) rectangle ++(1,1);
\path[fill=white] (31,33) rectangle ++(1,1);
\path[fill=white] (32,33) rectangle ++(1,1);
\path[fill=white] (33,33) rectangle ++(1,1);
\path[fill=white] (34,33) rectangle ++(1,1);
\path[fill=white] (0,32) rectangle ++(1,1);
\path[fill=white] (1,32) rectangle ++(1,1);
\path[fill=white] (2,32) rectangle ++(1,1);
\path[fill=white] (3,32) rectangle ++(1,1);
\path[fill=black] (4,32) rectangle ++(1,1);
\path[fill=black] (5,32) rectangle ++(1,1);
\path[fill=white] (6,32) rectangle ++(1,1);
\path[fill=white] (7,32) rectangle ++(1,1);
\path[fill=black] (8,32) rectangle ++(1,1);
\path[fill=white] (9,32) rectangle ++(1,1);
\path[fill=white] (10,32) rectangle ++(1,1);
\path[fill=white] (11,32) rectangle ++(1,1);
\path[fill=white] (12,32) rectangle ++(1,1);
\path[fill=black] (13,32) rectangle ++(1,1);
\path[fill=white] (14,32) rectangle ++(1,1);
\path[fill=white] (15,32) rectangle ++(1,1);
\path[fill=white] (16,32) rectangle ++(1,1);
\path[fill=white] (17,32) rectangle ++(1,1);
\path[fill=white] (18,32) rectangle ++(1,1);
\path[fill=white] (19,32) rectangle ++(1,1);
\path[fill=white] (20,32) rectangle ++(1,1);
\path[fill=white] (21,32) rectangle ++(1,1);
\path[fill=black] (22,32) rectangle ++(1,1);
\path[fill=white] (23,32) rectangle ++(1,1);
\path[fill=white] (24,32) rectangle ++(1,1);
\path[fill=white] (25,32) rectangle ++(1,1);
\path[fill=white] (26,32) rectangle ++(1,1);
\path[fill=white] (27,32) rectangle ++(1,1);
\path[fill=white] (28,32) rectangle ++(1,1);
\path[fill=white] (29,32) rectangle ++(1,1);
\path[fill=black] (30,32) rectangle ++(1,1);
\path[fill=white] (31,32) rectangle ++(1,1);
\path[fill=white] (32,32) rectangle ++(1,1);
\path[fill=white] (33,32) rectangle ++(1,1);
\path[fill=black] (34,32) rectangle ++(1,1);
\path[fill=white] (0,31) rectangle ++(1,1);
\path[fill=white] (1,31) rectangle ++(1,1);
\path[fill=white] (2,31) rectangle ++(1,1);
\path[fill=white] (3,31) rectangle ++(1,1);
\path[fill=white] (4,31) rectangle ++(1,1);
\path[fill=white] (5,31) rectangle ++(1,1);
\path[fill=black] (6,31) rectangle ++(1,1);
\path[fill=black] (7,31) rectangle ++(1,1);
\path[fill=white] (8,31) rectangle ++(1,1);
\path[fill=white] (9,31) rectangle ++(1,1);
\path[fill=white] (10,31) rectangle ++(1,1);
\path[fill=black] (11,31) rectangle ++(1,1);
\path[fill=white] (12,31) rectangle ++(1,1);
\path[fill=white] (13,31) rectangle ++(1,1);
\path[fill=white] (14,31) rectangle ++(1,1);
\path[fill=white] (15,31) rectangle ++(1,1);
\path[fill=black] (16,31) rectangle ++(1,1);
\path[fill=white] (17,31) rectangle ++(1,1);
\path[fill=white] (18,31) rectangle ++(1,1);
\path[fill=white] (19,31) rectangle ++(1,1);
\path[fill=white] (20,31) rectangle ++(1,1);
\path[fill=white] (21,31) rectangle ++(1,1);
\path[fill=white] (22,31) rectangle ++(1,1);
\path[fill=white] (23,31) rectangle ++(1,1);
\path[fill=white] (24,31) rectangle ++(1,1);
\path[fill=black] (25,31) rectangle ++(1,1);
\path[fill=white] (26,31) rectangle ++(1,1);
\path[fill=white] (27,31) rectangle ++(1,1);
\path[fill=white] (28,31) rectangle ++(1,1);
\path[fill=white] (29,31) rectangle ++(1,1);
\path[fill=white] (30,31) rectangle ++(1,1);
\path[fill=white] (31,31) rectangle ++(1,1);
\path[fill=white] (32,31) rectangle ++(1,1);
\path[fill=white] (33,31) rectangle ++(1,1);
\path[fill=white] (34,31) rectangle ++(1,1);
\path[fill=white] (0,30) rectangle ++(1,1);
\path[fill=white] (1,30) rectangle ++(1,1);
\path[fill=white] (2,30) rectangle ++(1,1);
\path[fill=white] (3,30) rectangle ++(1,1);
\path[fill=white] (4,30) rectangle ++(1,1);
\path[fill=white] (5,30) rectangle ++(1,1);
\path[fill=white] (6,30) rectangle ++(1,1);
\path[fill=black] (7,30) rectangle ++(1,1);
\path[fill=black] (8,30) rectangle ++(1,1);
\path[fill=white] (9,30) rectangle ++(1,1);
\path[fill=white] (10,30) rectangle ++(1,1);
\path[fill=white] (11,30) rectangle ++(1,1);
\path[fill=black] (12,30) rectangle ++(1,1);
\path[fill=white] (13,30) rectangle ++(1,1);
\path[fill=white] (14,30) rectangle ++(1,1);
\path[fill=white] (15,30) rectangle ++(1,1);
\path[fill=white] (16,30) rectangle ++(1,1);
\path[fill=black] (17,30) rectangle ++(1,1);
\path[fill=white] (18,30) rectangle ++(1,1);
\path[fill=white] (19,30) rectangle ++(1,1);
\path[fill=white] (20,30) rectangle ++(1,1);
\path[fill=white] (21,30) rectangle ++(1,1);
\path[fill=white] (22,30) rectangle ++(1,1);
\path[fill=white] (23,30) rectangle ++(1,1);
\path[fill=white] (24,30) rectangle ++(1,1);
\path[fill=white] (25,30) rectangle ++(1,1);
\path[fill=black] (26,30) rectangle ++(1,1);
\path[fill=white] (27,30) rectangle ++(1,1);
\path[fill=white] (28,30) rectangle ++(1,1);
\path[fill=white] (29,30) rectangle ++(1,1);
\path[fill=white] (30,30) rectangle ++(1,1);
\path[fill=white] (31,30) rectangle ++(1,1);
\path[fill=black] (32,30) rectangle ++(1,1);
\path[fill=white] (33,30) rectangle ++(1,1);
\path[fill=white] (34,30) rectangle ++(1,1);
\path[fill=white] (0,29) rectangle ++(1,1);
\path[fill=white] (1,29) rectangle ++(1,1);
\path[fill=white] (2,29) rectangle ++(1,1);
\path[fill=white] (3,29) rectangle ++(1,1);
\path[fill=white] (4,29) rectangle ++(1,1);
\path[fill=white] (5,29) rectangle ++(1,1);
\path[fill=white] (6,29) rectangle ++(1,1);
\path[fill=white] (7,29) rectangle ++(1,1);
\path[fill=black] (8,29) rectangle ++(1,1);
\path[fill=black] (9,29) rectangle ++(1,1);
\path[fill=white] (10,29) rectangle ++(1,1);
\path[fill=white] (11,29) rectangle ++(1,1);
\path[fill=white] (12,29) rectangle ++(1,1);
\path[fill=black] (13,29) rectangle ++(1,1);
\path[fill=white] (14,29) rectangle ++(1,1);
\path[fill=white] (15,29) rectangle ++(1,1);
\path[fill=white] (16,29) rectangle ++(1,1);
\path[fill=white] (17,29) rectangle ++(1,1);
\path[fill=black] (18,29) rectangle ++(1,1);
\path[fill=white] (19,29) rectangle ++(1,1);
\path[fill=white] (20,29) rectangle ++(1,1);
\path[fill=white] (21,29) rectangle ++(1,1);
\path[fill=white] (22,29) rectangle ++(1,1);
\path[fill=white] (23,29) rectangle ++(1,1);
\path[fill=white] (24,29) rectangle ++(1,1);
\path[fill=white] (25,29) rectangle ++(1,1);
\path[fill=white] (26,29) rectangle ++(1,1);
\path[fill=black] (27,29) rectangle ++(1,1);
\path[fill=white] (28,29) rectangle ++(1,1);
\path[fill=white] (29,29) rectangle ++(1,1);
\path[fill=white] (30,29) rectangle ++(1,1);
\path[fill=white] (31,29) rectangle ++(1,1);
\path[fill=white] (32,29) rectangle ++(1,1);
\path[fill=black] (33,29) rectangle ++(1,1);
\path[fill=white] (34,29) rectangle ++(1,1);
\path[fill=white] (0,28) rectangle ++(1,1);
\path[fill=white] (1,28) rectangle ++(1,1);
\path[fill=white] (2,28) rectangle ++(1,1);
\path[fill=white] (3,28) rectangle ++(1,1);
\path[fill=white] (4,28) rectangle ++(1,1);
\path[fill=white] (5,28) rectangle ++(1,1);
\path[fill=white] (6,28) rectangle ++(1,1);
\path[fill=white] (7,28) rectangle ++(1,1);
\path[fill=white] (8,28) rectangle ++(1,1);
\path[fill=white] (9,28) rectangle ++(1,1);
\path[fill=black] (10,28) rectangle ++(1,1);
\path[fill=black] (11,28) rectangle ++(1,1);
\path[fill=white] (12,28) rectangle ++(1,1);
\path[fill=white] (13,28) rectangle ++(1,1);
\path[fill=white] (14,28) rectangle ++(1,1);
\path[fill=black] (15,28) rectangle ++(1,1);
\path[fill=white] (16,28) rectangle ++(1,1);
\path[fill=white] (17,28) rectangle ++(1,1);
\path[fill=white] (18,28) rectangle ++(1,1);
\path[fill=white] (19,28) rectangle ++(1,1);
\path[fill=black] (20,28) rectangle ++(1,1);
\path[fill=white] (21,28) rectangle ++(1,1);
\path[fill=white] (22,28) rectangle ++(1,1);
\path[fill=white] (23,28) rectangle ++(1,1);
\path[fill=white] (24,28) rectangle ++(1,1);
\path[fill=white] (25,28) rectangle ++(1,1);
\path[fill=white] (26,28) rectangle ++(1,1);
\path[fill=white] (27,28) rectangle ++(1,1);
\path[fill=white] (28,28) rectangle ++(1,1);
\path[fill=white] (29,28) rectangle ++(1,1);
\path[fill=white] (30,28) rectangle ++(1,1);
\path[fill=white] (31,28) rectangle ++(1,1);
\path[fill=white] (32,28) rectangle ++(1,1);
\path[fill=white] (33,28) rectangle ++(1,1);
\path[fill=white] (34,28) rectangle ++(1,1);
\path[fill=white] (0,27) rectangle ++(1,1);
\path[fill=white] (1,27) rectangle ++(1,1);
\path[fill=white] (2,27) rectangle ++(1,1);
\path[fill=white] (3,27) rectangle ++(1,1);
\path[fill=white] (4,27) rectangle ++(1,1);
\path[fill=white] (5,27) rectangle ++(1,1);
\path[fill=white] (6,27) rectangle ++(1,1);
\path[fill=white] (7,27) rectangle ++(1,1);
\path[fill=white] (8,27) rectangle ++(1,1);
\path[fill=white] (9,27) rectangle ++(1,1);
\path[fill=white] (10,27) rectangle ++(1,1);
\path[fill=black] (11,27) rectangle ++(1,1);
\path[fill=black] (12,27) rectangle ++(1,1);
\path[fill=white] (13,27) rectangle ++(1,1);
\path[fill=white] (14,27) rectangle ++(1,1);
\path[fill=white] (15,27) rectangle ++(1,1);
\path[fill=black] (16,27) rectangle ++(1,1);
\path[fill=white] (17,27) rectangle ++(1,1);
\path[fill=white] (18,27) rectangle ++(1,1);
\path[fill=white] (19,27) rectangle ++(1,1);
\path[fill=white] (20,27) rectangle ++(1,1);
\path[fill=black] (21,27) rectangle ++(1,1);
\path[fill=white] (22,27) rectangle ++(1,1);
\path[fill=white] (23,27) rectangle ++(1,1);
\path[fill=white] (24,27) rectangle ++(1,1);
\path[fill=white] (25,27) rectangle ++(1,1);
\path[fill=white] (26,27) rectangle ++(1,1);
\path[fill=white] (27,27) rectangle ++(1,1);
\path[fill=white] (28,27) rectangle ++(1,1);
\path[fill=black] (29,27) rectangle ++(1,1);
\path[fill=white] (30,27) rectangle ++(1,1);
\path[fill=white] (31,27) rectangle ++(1,1);
\path[fill=white] (32,27) rectangle ++(1,1);
\path[fill=white] (33,27) rectangle ++(1,1);
\path[fill=white] (34,27) rectangle ++(1,1);
\path[fill=white] (0,26) rectangle ++(1,1);
\path[fill=white] (1,26) rectangle ++(1,1);
\path[fill=white] (2,26) rectangle ++(1,1);
\path[fill=white] (3,26) rectangle ++(1,1);
\path[fill=white] (4,26) rectangle ++(1,1);
\path[fill=white] (5,26) rectangle ++(1,1);
\path[fill=white] (6,26) rectangle ++(1,1);
\path[fill=white] (7,26) rectangle ++(1,1);
\path[fill=white] (8,26) rectangle ++(1,1);
\path[fill=white] (9,26) rectangle ++(1,1);
\path[fill=white] (10,26) rectangle ++(1,1);
\path[fill=white] (11,26) rectangle ++(1,1);
\path[fill=black] (12,26) rectangle ++(1,1);
\path[fill=black] (13,26) rectangle ++(1,1);
\path[fill=white] (14,26) rectangle ++(1,1);
\path[fill=white] (15,26) rectangle ++(1,1);
\path[fill=white] (16,26) rectangle ++(1,1);
\path[fill=black] (17,26) rectangle ++(1,1);
\path[fill=white] (18,26) rectangle ++(1,1);
\path[fill=white] (19,26) rectangle ++(1,1);
\path[fill=white] (20,26) rectangle ++(1,1);
\path[fill=white] (21,26) rectangle ++(1,1);
\path[fill=black] (22,26) rectangle ++(1,1);
\path[fill=white] (23,26) rectangle ++(1,1);
\path[fill=white] (24,26) rectangle ++(1,1);
\path[fill=white] (25,26) rectangle ++(1,1);
\path[fill=white] (26,26) rectangle ++(1,1);
\path[fill=white] (27,26) rectangle ++(1,1);
\path[fill=white] (28,26) rectangle ++(1,1);
\path[fill=white] (29,26) rectangle ++(1,1);
\path[fill=black] (30,26) rectangle ++(1,1);
\path[fill=white] (31,26) rectangle ++(1,1);
\path[fill=white] (32,26) rectangle ++(1,1);
\path[fill=white] (33,26) rectangle ++(1,1);
\path[fill=black] (34,26) rectangle ++(1,1);
\path[fill=white] (0,25) rectangle ++(1,1);
\path[fill=white] (1,25) rectangle ++(1,1);
\path[fill=white] (2,25) rectangle ++(1,1);
\path[fill=white] (3,25) rectangle ++(1,1);
\path[fill=white] (4,25) rectangle ++(1,1);
\path[fill=white] (5,25) rectangle ++(1,1);
\path[fill=white] (6,25) rectangle ++(1,1);
\path[fill=white] (7,25) rectangle ++(1,1);
\path[fill=white] (8,25) rectangle ++(1,1);
\path[fill=white] (9,25) rectangle ++(1,1);
\path[fill=white] (10,25) rectangle ++(1,1);
\path[fill=white] (11,25) rectangle ++(1,1);
\path[fill=white] (12,25) rectangle ++(1,1);
\path[fill=black] (13,25) rectangle ++(1,1);
\path[fill=black] (14,25) rectangle ++(1,1);
\path[fill=white] (15,25) rectangle ++(1,1);
\path[fill=white] (16,25) rectangle ++(1,1);
\path[fill=white] (17,25) rectangle ++(1,1);
\path[fill=black] (18,25) rectangle ++(1,1);
\path[fill=white] (19,25) rectangle ++(1,1);
\path[fill=white] (20,25) rectangle ++(1,1);
\path[fill=white] (21,25) rectangle ++(1,1);
\path[fill=white] (22,25) rectangle ++(1,1);
\path[fill=black] (23,25) rectangle ++(1,1);
\path[fill=white] (24,25) rectangle ++(1,1);
\path[fill=white] (25,25) rectangle ++(1,1);
\path[fill=white] (26,25) rectangle ++(1,1);
\path[fill=white] (27,25) rectangle ++(1,1);
\path[fill=white] (28,25) rectangle ++(1,1);
\path[fill=white] (29,25) rectangle ++(1,1);
\path[fill=white] (30,25) rectangle ++(1,1);
\path[fill=black] (31,25) rectangle ++(1,1);
\path[fill=white] (32,25) rectangle ++(1,1);
\path[fill=white] (33,25) rectangle ++(1,1);
\path[fill=white] (34,25) rectangle ++(1,1);
\path[fill=white] (0,24) rectangle ++(1,1);
\path[fill=white] (1,24) rectangle ++(1,1);
\path[fill=white] (2,24) rectangle ++(1,1);
\path[fill=white] (3,24) rectangle ++(1,1);
\path[fill=white] (4,24) rectangle ++(1,1);
\path[fill=white] (5,24) rectangle ++(1,1);
\path[fill=white] (6,24) rectangle ++(1,1);
\path[fill=white] (7,24) rectangle ++(1,1);
\path[fill=white] (8,24) rectangle ++(1,1);
\path[fill=white] (9,24) rectangle ++(1,1);
\path[fill=white] (10,24) rectangle ++(1,1);
\path[fill=white] (11,24) rectangle ++(1,1);
\path[fill=white] (12,24) rectangle ++(1,1);
\path[fill=white] (13,24) rectangle ++(1,1);
\path[fill=white] (14,24) rectangle ++(1,1);
\path[fill=black] (15,24) rectangle ++(1,1);
\path[fill=white] (16,24) rectangle ++(1,1);
\path[fill=white] (17,24) rectangle ++(1,1);
\path[fill=white] (18,24) rectangle ++(1,1);
\path[fill=white] (19,24) rectangle ++(1,1);
\path[fill=white] (20,24) rectangle ++(1,1);
\path[fill=white] (21,24) rectangle ++(1,1);
\path[fill=white] (22,24) rectangle ++(1,1);
\path[fill=white] (23,24) rectangle ++(1,1);
\path[fill=white] (24,24) rectangle ++(1,1);
\path[fill=white] (25,24) rectangle ++(1,1);
\path[fill=white] (26,24) rectangle ++(1,1);
\path[fill=white] (27,24) rectangle ++(1,1);
\path[fill=white] (28,24) rectangle ++(1,1);
\path[fill=white] (29,24) rectangle ++(1,1);
\path[fill=white] (30,24) rectangle ++(1,1);
\path[fill=white] (31,24) rectangle ++(1,1);
\path[fill=white] (32,24) rectangle ++(1,1);
\path[fill=white] (33,24) rectangle ++(1,1);
\path[fill=white] (34,24) rectangle ++(1,1);
\path[fill=white] (0,23) rectangle ++(1,1);
\path[fill=white] (1,23) rectangle ++(1,1);
\path[fill=white] (2,23) rectangle ++(1,1);
\path[fill=white] (3,23) rectangle ++(1,1);
\path[fill=white] (4,23) rectangle ++(1,1);
\path[fill=white] (5,23) rectangle ++(1,1);
\path[fill=white] (6,23) rectangle ++(1,1);
\path[fill=white] (7,23) rectangle ++(1,1);
\path[fill=white] (8,23) rectangle ++(1,1);
\path[fill=white] (9,23) rectangle ++(1,1);
\path[fill=white] (10,23) rectangle ++(1,1);
\path[fill=white] (11,23) rectangle ++(1,1);
\path[fill=white] (12,23) rectangle ++(1,1);
\path[fill=white] (13,23) rectangle ++(1,1);
\path[fill=white] (14,23) rectangle ++(1,1);
\path[fill=black] (15,23) rectangle ++(1,1);
\path[fill=black] (16,23) rectangle ++(1,1);
\path[fill=white] (17,23) rectangle ++(1,1);
\path[fill=white] (18,23) rectangle ++(1,1);
\path[fill=white] (19,23) rectangle ++(1,1);
\path[fill=black] (20,23) rectangle ++(1,1);
\path[fill=white] (21,23) rectangle ++(1,1);
\path[fill=white] (22,23) rectangle ++(1,1);
\path[fill=white] (23,23) rectangle ++(1,1);
\path[fill=white] (24,23) rectangle ++(1,1);
\path[fill=black] (25,23) rectangle ++(1,1);
\path[fill=white] (26,23) rectangle ++(1,1);
\path[fill=white] (27,23) rectangle ++(1,1);
\path[fill=white] (28,23) rectangle ++(1,1);
\path[fill=white] (29,23) rectangle ++(1,1);
\path[fill=white] (30,23) rectangle ++(1,1);
\path[fill=white] (31,23) rectangle ++(1,1);
\path[fill=white] (32,23) rectangle ++(1,1);
\path[fill=white] (33,23) rectangle ++(1,1);
\path[fill=white] (34,23) rectangle ++(1,1);
\path[fill=white] (0,22) rectangle ++(1,1);
\path[fill=white] (1,22) rectangle ++(1,1);
\path[fill=white] (2,22) rectangle ++(1,1);
\path[fill=white] (3,22) rectangle ++(1,1);
\path[fill=white] (4,22) rectangle ++(1,1);
\path[fill=white] (5,22) rectangle ++(1,1);
\path[fill=white] (6,22) rectangle ++(1,1);
\path[fill=white] (7,22) rectangle ++(1,1);
\path[fill=white] (8,22) rectangle ++(1,1);
\path[fill=white] (9,22) rectangle ++(1,1);
\path[fill=white] (10,22) rectangle ++(1,1);
\path[fill=white] (11,22) rectangle ++(1,1);
\path[fill=white] (12,22) rectangle ++(1,1);
\path[fill=white] (13,22) rectangle ++(1,1);
\path[fill=white] (14,22) rectangle ++(1,1);
\path[fill=white] (15,22) rectangle ++(1,1);
\path[fill=black] (16,22) rectangle ++(1,1);
\path[fill=black] (17,22) rectangle ++(1,1);
\path[fill=white] (18,22) rectangle ++(1,1);
\path[fill=white] (19,22) rectangle ++(1,1);
\path[fill=white] (20,22) rectangle ++(1,1);
\path[fill=black] (21,22) rectangle ++(1,1);
\path[fill=white] (22,22) rectangle ++(1,1);
\path[fill=white] (23,22) rectangle ++(1,1);
\path[fill=white] (24,22) rectangle ++(1,1);
\path[fill=white] (25,22) rectangle ++(1,1);
\path[fill=black] (26,22) rectangle ++(1,1);
\path[fill=white] (27,22) rectangle ++(1,1);
\path[fill=white] (28,22) rectangle ++(1,1);
\path[fill=white] (29,22) rectangle ++(1,1);
\path[fill=white] (30,22) rectangle ++(1,1);
\path[fill=white] (31,22) rectangle ++(1,1);
\path[fill=black] (32,22) rectangle ++(1,1);
\path[fill=white] (33,22) rectangle ++(1,1);
\path[fill=white] (34,22) rectangle ++(1,1);
\path[fill=white] (0,21) rectangle ++(1,1);
\path[fill=white] (1,21) rectangle ++(1,1);
\path[fill=white] (2,21) rectangle ++(1,1);
\path[fill=white] (3,21) rectangle ++(1,1);
\path[fill=white] (4,21) rectangle ++(1,1);
\path[fill=white] (5,21) rectangle ++(1,1);
\path[fill=white] (6,21) rectangle ++(1,1);
\path[fill=white] (7,21) rectangle ++(1,1);
\path[fill=white] (8,21) rectangle ++(1,1);
\path[fill=white] (9,21) rectangle ++(1,1);
\path[fill=white] (10,21) rectangle ++(1,1);
\path[fill=white] (11,21) rectangle ++(1,1);
\path[fill=white] (12,21) rectangle ++(1,1);
\path[fill=white] (13,21) rectangle ++(1,1);
\path[fill=white] (14,21) rectangle ++(1,1);
\path[fill=white] (15,21) rectangle ++(1,1);
\path[fill=white] (16,21) rectangle ++(1,1);
\path[fill=black] (17,21) rectangle ++(1,1);
\path[fill=black] (18,21) rectangle ++(1,1);
\path[fill=white] (19,21) rectangle ++(1,1);
\path[fill=white] (20,21) rectangle ++(1,1);
\path[fill=white] (21,21) rectangle ++(1,1);
\path[fill=black] (22,21) rectangle ++(1,1);
\path[fill=white] (23,21) rectangle ++(1,1);
\path[fill=white] (24,21) rectangle ++(1,1);
\path[fill=white] (25,21) rectangle ++(1,1);
\path[fill=white] (26,21) rectangle ++(1,1);
\path[fill=black] (27,21) rectangle ++(1,1);
\path[fill=white] (28,21) rectangle ++(1,1);
\path[fill=white] (29,21) rectangle ++(1,1);
\path[fill=white] (30,21) rectangle ++(1,1);
\path[fill=white] (31,21) rectangle ++(1,1);
\path[fill=white] (32,21) rectangle ++(1,1);
\path[fill=black] (33,21) rectangle ++(1,1);
\path[fill=white] (34,21) rectangle ++(1,1);
\path[fill=white] (0,20) rectangle ++(1,1);
\path[fill=white] (1,20) rectangle ++(1,1);
\path[fill=white] (2,20) rectangle ++(1,1);
\path[fill=white] (3,20) rectangle ++(1,1);
\path[fill=white] (4,20) rectangle ++(1,1);
\path[fill=white] (5,20) rectangle ++(1,1);
\path[fill=white] (6,20) rectangle ++(1,1);
\path[fill=white] (7,20) rectangle ++(1,1);
\path[fill=white] (8,20) rectangle ++(1,1);
\path[fill=white] (9,20) rectangle ++(1,1);
\path[fill=white] (10,20) rectangle ++(1,1);
\path[fill=white] (11,20) rectangle ++(1,1);
\path[fill=white] (12,20) rectangle ++(1,1);
\path[fill=white] (13,20) rectangle ++(1,1);
\path[fill=white] (14,20) rectangle ++(1,1);
\path[fill=white] (15,20) rectangle ++(1,1);
\path[fill=white] (16,20) rectangle ++(1,1);
\path[fill=white] (17,20) rectangle ++(1,1);
\path[fill=black] (18,20) rectangle ++(1,1);
\path[fill=black] (19,20) rectangle ++(1,1);
\path[fill=white] (20,20) rectangle ++(1,1);
\path[fill=white] (21,20) rectangle ++(1,1);
\path[fill=white] (22,20) rectangle ++(1,1);
\path[fill=black] (23,20) rectangle ++(1,1);
\path[fill=white] (24,20) rectangle ++(1,1);
\path[fill=white] (25,20) rectangle ++(1,1);
\path[fill=white] (26,20) rectangle ++(1,1);
\path[fill=white] (27,20) rectangle ++(1,1);
\path[fill=black] (28,20) rectangle ++(1,1);
\path[fill=white] (29,20) rectangle ++(1,1);
\path[fill=white] (30,20) rectangle ++(1,1);
\path[fill=white] (31,20) rectangle ++(1,1);
\path[fill=white] (32,20) rectangle ++(1,1);
\path[fill=white] (33,20) rectangle ++(1,1);
\path[fill=white] (34,20) rectangle ++(1,1);
\path[fill=white] (0,19) rectangle ++(1,1);
\path[fill=white] (1,19) rectangle ++(1,1);
\path[fill=white] (2,19) rectangle ++(1,1);
\path[fill=white] (3,19) rectangle ++(1,1);
\path[fill=white] (4,19) rectangle ++(1,1);
\path[fill=white] (5,19) rectangle ++(1,1);
\path[fill=white] (6,19) rectangle ++(1,1);
\path[fill=white] (7,19) rectangle ++(1,1);
\path[fill=white] (8,19) rectangle ++(1,1);
\path[fill=white] (9,19) rectangle ++(1,1);
\path[fill=white] (10,19) rectangle ++(1,1);
\path[fill=white] (11,19) rectangle ++(1,1);
\path[fill=white] (12,19) rectangle ++(1,1);
\path[fill=white] (13,19) rectangle ++(1,1);
\path[fill=white] (14,19) rectangle ++(1,1);
\path[fill=white] (15,19) rectangle ++(1,1);
\path[fill=white] (16,19) rectangle ++(1,1);
\path[fill=white] (17,19) rectangle ++(1,1);
\path[fill=white] (18,19) rectangle ++(1,1);
\path[fill=white] (19,19) rectangle ++(1,1);
\path[fill=black] (20,19) rectangle ++(1,1);
\path[fill=white] (21,19) rectangle ++(1,1);
\path[fill=white] (22,19) rectangle ++(1,1);
\path[fill=white] (23,19) rectangle ++(1,1);
\path[fill=white] (24,19) rectangle ++(1,1);
\path[fill=white] (25,19) rectangle ++(1,1);
\path[fill=white] (26,19) rectangle ++(1,1);
\path[fill=white] (27,19) rectangle ++(1,1);
\path[fill=white] (28,19) rectangle ++(1,1);
\path[fill=white] (29,19) rectangle ++(1,1);
\path[fill=white] (30,19) rectangle ++(1,1);
\path[fill=white] (31,19) rectangle ++(1,1);
\path[fill=white] (32,19) rectangle ++(1,1);
\path[fill=white] (33,19) rectangle ++(1,1);
\path[fill=white] (34,19) rectangle ++(1,1);
\path[fill=white] (0,18) rectangle ++(1,1);
\path[fill=white] (1,18) rectangle ++(1,1);
\path[fill=white] (2,18) rectangle ++(1,1);
\path[fill=white] (3,18) rectangle ++(1,1);
\path[fill=white] (4,18) rectangle ++(1,1);
\path[fill=white] (5,18) rectangle ++(1,1);
\path[fill=white] (6,18) rectangle ++(1,1);
\path[fill=white] (7,18) rectangle ++(1,1);
\path[fill=white] (8,18) rectangle ++(1,1);
\path[fill=white] (9,18) rectangle ++(1,1);
\path[fill=white] (10,18) rectangle ++(1,1);
\path[fill=white] (11,18) rectangle ++(1,1);
\path[fill=white] (12,18) rectangle ++(1,1);
\path[fill=white] (13,18) rectangle ++(1,1);
\path[fill=white] (14,18) rectangle ++(1,1);
\path[fill=white] (15,18) rectangle ++(1,1);
\path[fill=white] (16,18) rectangle ++(1,1);
\path[fill=white] (17,18) rectangle ++(1,1);
\path[fill=white] (18,18) rectangle ++(1,1);
\path[fill=white] (19,18) rectangle ++(1,1);
\path[fill=black] (20,18) rectangle ++(1,1);
\path[fill=black] (21,18) rectangle ++(1,1);
\path[fill=white] (22,18) rectangle ++(1,1);
\path[fill=white] (23,18) rectangle ++(1,1);
\path[fill=white] (24,18) rectangle ++(1,1);
\path[fill=black] (25,18) rectangle ++(1,1);
\path[fill=white] (26,18) rectangle ++(1,1);
\path[fill=white] (27,18) rectangle ++(1,1);
\path[fill=white] (28,18) rectangle ++(1,1);
\path[fill=black] (29,18) rectangle ++(1,1);
\path[fill=white] (30,18) rectangle ++(1,1);
\path[fill=white] (31,18) rectangle ++(1,1);
\path[fill=white] (32,18) rectangle ++(1,1);
\path[fill=white] (33,18) rectangle ++(1,1);
\path[fill=white] (34,18) rectangle ++(1,1);
\path[fill=white] (0,17) rectangle ++(1,1);
\path[fill=white] (1,17) rectangle ++(1,1);
\path[fill=white] (2,17) rectangle ++(1,1);
\path[fill=white] (3,17) rectangle ++(1,1);
\path[fill=white] (4,17) rectangle ++(1,1);
\path[fill=white] (5,17) rectangle ++(1,1);
\path[fill=white] (6,17) rectangle ++(1,1);
\path[fill=white] (7,17) rectangle ++(1,1);
\path[fill=white] (8,17) rectangle ++(1,1);
\path[fill=white] (9,17) rectangle ++(1,1);
\path[fill=white] (10,17) rectangle ++(1,1);
\path[fill=white] (11,17) rectangle ++(1,1);
\path[fill=white] (12,17) rectangle ++(1,1);
\path[fill=white] (13,17) rectangle ++(1,1);
\path[fill=white] (14,17) rectangle ++(1,1);
\path[fill=white] (15,17) rectangle ++(1,1);
\path[fill=white] (16,17) rectangle ++(1,1);
\path[fill=white] (17,17) rectangle ++(1,1);
\path[fill=white] (18,17) rectangle ++(1,1);
\path[fill=white] (19,17) rectangle ++(1,1);
\path[fill=white] (20,17) rectangle ++(1,1);
\path[fill=black] (21,17) rectangle ++(1,1);
\path[fill=black] (22,17) rectangle ++(1,1);
\path[fill=white] (23,17) rectangle ++(1,1);
\path[fill=white] (24,17) rectangle ++(1,1);
\path[fill=white] (25,17) rectangle ++(1,1);
\path[fill=black] (26,17) rectangle ++(1,1);
\path[fill=white] (27,17) rectangle ++(1,1);
\path[fill=white] (28,17) rectangle ++(1,1);
\path[fill=white] (29,17) rectangle ++(1,1);
\path[fill=black] (30,17) rectangle ++(1,1);
\path[fill=white] (31,17) rectangle ++(1,1);
\path[fill=white] (32,17) rectangle ++(1,1);
\path[fill=white] (33,17) rectangle ++(1,1);
\path[fill=black] (34,17) rectangle ++(1,1);
\path[fill=white] (0,16) rectangle ++(1,1);
\path[fill=white] (1,16) rectangle ++(1,1);
\path[fill=white] (2,16) rectangle ++(1,1);
\path[fill=white] (3,16) rectangle ++(1,1);
\path[fill=white] (4,16) rectangle ++(1,1);
\path[fill=white] (5,16) rectangle ++(1,1);
\path[fill=white] (6,16) rectangle ++(1,1);
\path[fill=white] (7,16) rectangle ++(1,1);
\path[fill=white] (8,16) rectangle ++(1,1);
\path[fill=white] (9,16) rectangle ++(1,1);
\path[fill=white] (10,16) rectangle ++(1,1);
\path[fill=white] (11,16) rectangle ++(1,1);
\path[fill=white] (12,16) rectangle ++(1,1);
\path[fill=white] (13,16) rectangle ++(1,1);
\path[fill=white] (14,16) rectangle ++(1,1);
\path[fill=white] (15,16) rectangle ++(1,1);
\path[fill=white] (16,16) rectangle ++(1,1);
\path[fill=white] (17,16) rectangle ++(1,1);
\path[fill=white] (18,16) rectangle ++(1,1);
\path[fill=white] (19,16) rectangle ++(1,1);
\path[fill=white] (20,16) rectangle ++(1,1);
\path[fill=white] (21,16) rectangle ++(1,1);
\path[fill=black] (22,16) rectangle ++(1,1);
\path[fill=black] (23,16) rectangle ++(1,1);
\path[fill=white] (24,16) rectangle ++(1,1);
\path[fill=white] (25,16) rectangle ++(1,1);
\path[fill=white] (26,16) rectangle ++(1,1);
\path[fill=black] (27,16) rectangle ++(1,1);
\path[fill=white] (28,16) rectangle ++(1,1);
\path[fill=white] (29,16) rectangle ++(1,1);
\path[fill=white] (30,16) rectangle ++(1,1);
\path[fill=black] (31,16) rectangle ++(1,1);
\path[fill=white] (32,16) rectangle ++(1,1);
\path[fill=white] (33,16) rectangle ++(1,1);
\path[fill=white] (34,16) rectangle ++(1,1);
\path[fill=white] (0,15) rectangle ++(1,1);
\path[fill=white] (1,15) rectangle ++(1,1);
\path[fill=white] (2,15) rectangle ++(1,1);
\path[fill=white] (3,15) rectangle ++(1,1);
\path[fill=white] (4,15) rectangle ++(1,1);
\path[fill=white] (5,15) rectangle ++(1,1);
\path[fill=white] (6,15) rectangle ++(1,1);
\path[fill=white] (7,15) rectangle ++(1,1);
\path[fill=white] (8,15) rectangle ++(1,1);
\path[fill=white] (9,15) rectangle ++(1,1);
\path[fill=white] (10,15) rectangle ++(1,1);
\path[fill=white] (11,15) rectangle ++(1,1);
\path[fill=white] (12,15) rectangle ++(1,1);
\path[fill=white] (13,15) rectangle ++(1,1);
\path[fill=white] (14,15) rectangle ++(1,1);
\path[fill=white] (15,15) rectangle ++(1,1);
\path[fill=white] (16,15) rectangle ++(1,1);
\path[fill=white] (17,15) rectangle ++(1,1);
\path[fill=white] (18,15) rectangle ++(1,1);
\path[fill=white] (19,15) rectangle ++(1,1);
\path[fill=white] (20,15) rectangle ++(1,1);
\path[fill=white] (21,15) rectangle ++(1,1);
\path[fill=white] (22,15) rectangle ++(1,1);
\path[fill=black] (23,15) rectangle ++(1,1);
\path[fill=black] (24,15) rectangle ++(1,1);
\path[fill=white] (25,15) rectangle ++(1,1);
\path[fill=white] (26,15) rectangle ++(1,1);
\path[fill=white] (27,15) rectangle ++(1,1);
\path[fill=black] (28,15) rectangle ++(1,1);
\path[fill=white] (29,15) rectangle ++(1,1);
\path[fill=white] (30,15) rectangle ++(1,1);
\path[fill=white] (31,15) rectangle ++(1,1);
\path[fill=white] (32,15) rectangle ++(1,1);
\path[fill=white] (33,15) rectangle ++(1,1);
\path[fill=white] (34,15) rectangle ++(1,1);
\path[fill=white] (0,14) rectangle ++(1,1);
\path[fill=white] (1,14) rectangle ++(1,1);
\path[fill=white] (2,14) rectangle ++(1,1);
\path[fill=white] (3,14) rectangle ++(1,1);
\path[fill=white] (4,14) rectangle ++(1,1);
\path[fill=white] (5,14) rectangle ++(1,1);
\path[fill=white] (6,14) rectangle ++(1,1);
\path[fill=white] (7,14) rectangle ++(1,1);
\path[fill=white] (8,14) rectangle ++(1,1);
\path[fill=white] (9,14) rectangle ++(1,1);
\path[fill=white] (10,14) rectangle ++(1,1);
\path[fill=white] (11,14) rectangle ++(1,1);
\path[fill=white] (12,14) rectangle ++(1,1);
\path[fill=white] (13,14) rectangle ++(1,1);
\path[fill=white] (14,14) rectangle ++(1,1);
\path[fill=white] (15,14) rectangle ++(1,1);
\path[fill=white] (16,14) rectangle ++(1,1);
\path[fill=white] (17,14) rectangle ++(1,1);
\path[fill=white] (18,14) rectangle ++(1,1);
\path[fill=white] (19,14) rectangle ++(1,1);
\path[fill=white] (20,14) rectangle ++(1,1);
\path[fill=white] (21,14) rectangle ++(1,1);
\path[fill=white] (22,14) rectangle ++(1,1);
\path[fill=white] (23,14) rectangle ++(1,1);
\path[fill=white] (24,14) rectangle ++(1,1);
\path[fill=black] (25,14) rectangle ++(1,1);
\path[fill=white] (26,14) rectangle ++(1,1);
\path[fill=white] (27,14) rectangle ++(1,1);
\path[fill=white] (28,14) rectangle ++(1,1);
\path[fill=white] (29,14) rectangle ++(1,1);
\path[fill=white] (30,14) rectangle ++(1,1);
\path[fill=white] (31,14) rectangle ++(1,1);
\path[fill=white] (32,14) rectangle ++(1,1);
\path[fill=white] (33,14) rectangle ++(1,1);
\path[fill=white] (34,14) rectangle ++(1,1);
\path[fill=white] (0,13) rectangle ++(1,1);
\path[fill=white] (1,13) rectangle ++(1,1);
\path[fill=white] (2,13) rectangle ++(1,1);
\path[fill=white] (3,13) rectangle ++(1,1);
\path[fill=white] (4,13) rectangle ++(1,1);
\path[fill=white] (5,13) rectangle ++(1,1);
\path[fill=white] (6,13) rectangle ++(1,1);
\path[fill=white] (7,13) rectangle ++(1,1);
\path[fill=white] (8,13) rectangle ++(1,1);
\path[fill=white] (9,13) rectangle ++(1,1);
\path[fill=white] (10,13) rectangle ++(1,1);
\path[fill=white] (11,13) rectangle ++(1,1);
\path[fill=white] (12,13) rectangle ++(1,1);
\path[fill=white] (13,13) rectangle ++(1,1);
\path[fill=white] (14,13) rectangle ++(1,1);
\path[fill=white] (15,13) rectangle ++(1,1);
\path[fill=white] (16,13) rectangle ++(1,1);
\path[fill=white] (17,13) rectangle ++(1,1);
\path[fill=white] (18,13) rectangle ++(1,1);
\path[fill=white] (19,13) rectangle ++(1,1);
\path[fill=white] (20,13) rectangle ++(1,1);
\path[fill=white] (21,13) rectangle ++(1,1);
\path[fill=white] (22,13) rectangle ++(1,1);
\path[fill=white] (23,13) rectangle ++(1,1);
\path[fill=white] (24,13) rectangle ++(1,1);
\path[fill=black] (25,13) rectangle ++(1,1);
\path[fill=black] (26,13) rectangle ++(1,1);
\path[fill=white] (27,13) rectangle ++(1,1);
\path[fill=white] (28,13) rectangle ++(1,1);
\path[fill=black] (29,13) rectangle ++(1,1);
\path[fill=white] (30,13) rectangle ++(1,1);
\path[fill=white] (31,13) rectangle ++(1,1);
\path[fill=black] (32,13) rectangle ++(1,1);
\path[fill=white] (33,13) rectangle ++(1,1);
\path[fill=white] (34,13) rectangle ++(1,1);
\path[fill=white] (0,12) rectangle ++(1,1);
\path[fill=white] (1,12) rectangle ++(1,1);
\path[fill=white] (2,12) rectangle ++(1,1);
\path[fill=white] (3,12) rectangle ++(1,1);
\path[fill=white] (4,12) rectangle ++(1,1);
\path[fill=white] (5,12) rectangle ++(1,1);
\path[fill=white] (6,12) rectangle ++(1,1);
\path[fill=white] (7,12) rectangle ++(1,1);
\path[fill=white] (8,12) rectangle ++(1,1);
\path[fill=white] (9,12) rectangle ++(1,1);
\path[fill=white] (10,12) rectangle ++(1,1);
\path[fill=white] (11,12) rectangle ++(1,1);
\path[fill=white] (12,12) rectangle ++(1,1);
\path[fill=white] (13,12) rectangle ++(1,1);
\path[fill=white] (14,12) rectangle ++(1,1);
\path[fill=white] (15,12) rectangle ++(1,1);
\path[fill=white] (16,12) rectangle ++(1,1);
\path[fill=white] (17,12) rectangle ++(1,1);
\path[fill=white] (18,12) rectangle ++(1,1);
\path[fill=white] (19,12) rectangle ++(1,1);
\path[fill=white] (20,12) rectangle ++(1,1);
\path[fill=white] (21,12) rectangle ++(1,1);
\path[fill=white] (22,12) rectangle ++(1,1);
\path[fill=white] (23,12) rectangle ++(1,1);
\path[fill=white] (24,12) rectangle ++(1,1);
\path[fill=white] (25,12) rectangle ++(1,1);
\path[fill=black] (26,12) rectangle ++(1,1);
\path[fill=black] (27,12) rectangle ++(1,1);
\path[fill=white] (28,12) rectangle ++(1,1);
\path[fill=white] (29,12) rectangle ++(1,1);
\path[fill=black] (30,12) rectangle ++(1,1);
\path[fill=white] (31,12) rectangle ++(1,1);
\path[fill=white] (32,12) rectangle ++(1,1);
\path[fill=black] (33,12) rectangle ++(1,1);
\path[fill=white] (34,12) rectangle ++(1,1);
\path[fill=white] (0,11) rectangle ++(1,1);
\path[fill=white] (1,11) rectangle ++(1,1);
\path[fill=white] (2,11) rectangle ++(1,1);
\path[fill=white] (3,11) rectangle ++(1,1);
\path[fill=white] (4,11) rectangle ++(1,1);
\path[fill=white] (5,11) rectangle ++(1,1);
\path[fill=white] (6,11) rectangle ++(1,1);
\path[fill=white] (7,11) rectangle ++(1,1);
\path[fill=white] (8,11) rectangle ++(1,1);
\path[fill=white] (9,11) rectangle ++(1,1);
\path[fill=white] (10,11) rectangle ++(1,1);
\path[fill=white] (11,11) rectangle ++(1,1);
\path[fill=white] (12,11) rectangle ++(1,1);
\path[fill=white] (13,11) rectangle ++(1,1);
\path[fill=white] (14,11) rectangle ++(1,1);
\path[fill=white] (15,11) rectangle ++(1,1);
\path[fill=white] (16,11) rectangle ++(1,1);
\path[fill=white] (17,11) rectangle ++(1,1);
\path[fill=white] (18,11) rectangle ++(1,1);
\path[fill=white] (19,11) rectangle ++(1,1);
\path[fill=white] (20,11) rectangle ++(1,1);
\path[fill=white] (21,11) rectangle ++(1,1);
\path[fill=white] (22,11) rectangle ++(1,1);
\path[fill=white] (23,11) rectangle ++(1,1);
\path[fill=white] (24,11) rectangle ++(1,1);
\path[fill=white] (25,11) rectangle ++(1,1);
\path[fill=white] (26,11) rectangle ++(1,1);
\path[fill=black] (27,11) rectangle ++(1,1);
\path[fill=black] (28,11) rectangle ++(1,1);
\path[fill=white] (29,11) rectangle ++(1,1);
\path[fill=white] (30,11) rectangle ++(1,1);
\path[fill=black] (31,11) rectangle ++(1,1);
\path[fill=white] (32,11) rectangle ++(1,1);
\path[fill=white] (33,11) rectangle ++(1,1);
\path[fill=white] (34,11) rectangle ++(1,1);
\path[fill=white] (0,10) rectangle ++(1,1);
\path[fill=white] (1,10) rectangle ++(1,1);
\path[fill=white] (2,10) rectangle ++(1,1);
\path[fill=white] (3,10) rectangle ++(1,1);
\path[fill=white] (4,10) rectangle ++(1,1);
\path[fill=white] (5,10) rectangle ++(1,1);
\path[fill=white] (6,10) rectangle ++(1,1);
\path[fill=white] (7,10) rectangle ++(1,1);
\path[fill=white] (8,10) rectangle ++(1,1);
\path[fill=white] (9,10) rectangle ++(1,1);
\path[fill=white] (10,10) rectangle ++(1,1);
\path[fill=white] (11,10) rectangle ++(1,1);
\path[fill=white] (12,10) rectangle ++(1,1);
\path[fill=white] (13,10) rectangle ++(1,1);
\path[fill=white] (14,10) rectangle ++(1,1);
\path[fill=white] (15,10) rectangle ++(1,1);
\path[fill=white] (16,10) rectangle ++(1,1);
\path[fill=white] (17,10) rectangle ++(1,1);
\path[fill=white] (18,10) rectangle ++(1,1);
\path[fill=white] (19,10) rectangle ++(1,1);
\path[fill=white] (20,10) rectangle ++(1,1);
\path[fill=white] (21,10) rectangle ++(1,1);
\path[fill=white] (22,10) rectangle ++(1,1);
\path[fill=white] (23,10) rectangle ++(1,1);
\path[fill=white] (24,10) rectangle ++(1,1);
\path[fill=white] (25,10) rectangle ++(1,1);
\path[fill=white] (26,10) rectangle ++(1,1);
\path[fill=white] (27,10) rectangle ++(1,1);
\path[fill=black] (28,10) rectangle ++(1,1);
\path[fill=white] (29,10) rectangle ++(1,1);
\path[fill=white] (30,10) rectangle ++(1,1);
\path[fill=white] (31,10) rectangle ++(1,1);
\path[fill=white] (32,10) rectangle ++(1,1);
\path[fill=white] (33,10) rectangle ++(1,1);
\path[fill=white] (34,10) rectangle ++(1,1);
\path[fill=white] (0,9) rectangle ++(1,1);
\path[fill=white] (1,9) rectangle ++(1,1);
\path[fill=white] (2,9) rectangle ++(1,1);
\path[fill=white] (3,9) rectangle ++(1,1);
\path[fill=white] (4,9) rectangle ++(1,1);
\path[fill=white] (5,9) rectangle ++(1,1);
\path[fill=white] (6,9) rectangle ++(1,1);
\path[fill=white] (7,9) rectangle ++(1,1);
\path[fill=white] (8,9) rectangle ++(1,1);
\path[fill=white] (9,9) rectangle ++(1,1);
\path[fill=white] (10,9) rectangle ++(1,1);
\path[fill=white] (11,9) rectangle ++(1,1);
\path[fill=white] (12,9) rectangle ++(1,1);
\path[fill=white] (13,9) rectangle ++(1,1);
\path[fill=white] (14,9) rectangle ++(1,1);
\path[fill=white] (15,9) rectangle ++(1,1);
\path[fill=white] (16,9) rectangle ++(1,1);
\path[fill=white] (17,9) rectangle ++(1,1);
\path[fill=white] (18,9) rectangle ++(1,1);
\path[fill=white] (19,9) rectangle ++(1,1);
\path[fill=white] (20,9) rectangle ++(1,1);
\path[fill=white] (21,9) rectangle ++(1,1);
\path[fill=white] (22,9) rectangle ++(1,1);
\path[fill=white] (23,9) rectangle ++(1,1);
\path[fill=white] (24,9) rectangle ++(1,1);
\path[fill=white] (25,9) rectangle ++(1,1);
\path[fill=white] (26,9) rectangle ++(1,1);
\path[fill=white] (27,9) rectangle ++(1,1);
\path[fill=white] (28,9) rectangle ++(1,1);
\path[fill=black] (29,9) rectangle ++(1,1);
\path[fill=white] (30,9) rectangle ++(1,1);
\path[fill=white] (31,9) rectangle ++(1,1);
\path[fill=white] (32,9) rectangle ++(1,1);
\path[fill=white] (33,9) rectangle ++(1,1);
\path[fill=white] (34,9) rectangle ++(1,1);
\path[fill=white] (0,8) rectangle ++(1,1);
\path[fill=white] (1,8) rectangle ++(1,1);
\path[fill=white] (2,8) rectangle ++(1,1);
\path[fill=white] (3,8) rectangle ++(1,1);
\path[fill=white] (4,8) rectangle ++(1,1);
\path[fill=white] (5,8) rectangle ++(1,1);
\path[fill=white] (6,8) rectangle ++(1,1);
\path[fill=white] (7,8) rectangle ++(1,1);
\path[fill=white] (8,8) rectangle ++(1,1);
\path[fill=white] (9,8) rectangle ++(1,1);
\path[fill=white] (10,8) rectangle ++(1,1);
\path[fill=white] (11,8) rectangle ++(1,1);
\path[fill=white] (12,8) rectangle ++(1,1);
\path[fill=white] (13,8) rectangle ++(1,1);
\path[fill=white] (14,8) rectangle ++(1,1);
\path[fill=white] (15,8) rectangle ++(1,1);
\path[fill=white] (16,8) rectangle ++(1,1);
\path[fill=white] (17,8) rectangle ++(1,1);
\path[fill=white] (18,8) rectangle ++(1,1);
\path[fill=white] (19,8) rectangle ++(1,1);
\path[fill=white] (20,8) rectangle ++(1,1);
\path[fill=white] (21,8) rectangle ++(1,1);
\path[fill=white] (22,8) rectangle ++(1,1);
\path[fill=white] (23,8) rectangle ++(1,1);
\path[fill=white] (24,8) rectangle ++(1,1);
\path[fill=white] (25,8) rectangle ++(1,1);
\path[fill=white] (26,8) rectangle ++(1,1);
\path[fill=white] (27,8) rectangle ++(1,1);
\path[fill=white] (28,8) rectangle ++(1,1);
\path[fill=black] (29,8) rectangle ++(1,1);
\path[fill=black] (30,8) rectangle ++(1,1);
\path[fill=white] (31,8) rectangle ++(1,1);
\path[fill=black] (32,8) rectangle ++(1,1);
\path[fill=white] (33,8) rectangle ++(1,1);
\path[fill=black] (34,8) rectangle ++(1,1);
\path[fill=white] (0,7) rectangle ++(1,1);
\path[fill=white] (1,7) rectangle ++(1,1);
\path[fill=white] (2,7) rectangle ++(1,1);
\path[fill=white] (3,7) rectangle ++(1,1);
\path[fill=white] (4,7) rectangle ++(1,1);
\path[fill=white] (5,7) rectangle ++(1,1);
\path[fill=white] (6,7) rectangle ++(1,1);
\path[fill=white] (7,7) rectangle ++(1,1);
\path[fill=white] (8,7) rectangle ++(1,1);
\path[fill=white] (9,7) rectangle ++(1,1);
\path[fill=white] (10,7) rectangle ++(1,1);
\path[fill=white] (11,7) rectangle ++(1,1);
\path[fill=white] (12,7) rectangle ++(1,1);
\path[fill=white] (13,7) rectangle ++(1,1);
\path[fill=white] (14,7) rectangle ++(1,1);
\path[fill=white] (15,7) rectangle ++(1,1);
\path[fill=white] (16,7) rectangle ++(1,1);
\path[fill=white] (17,7) rectangle ++(1,1);
\path[fill=white] (18,7) rectangle ++(1,1);
\path[fill=white] (19,7) rectangle ++(1,1);
\path[fill=white] (20,7) rectangle ++(1,1);
\path[fill=white] (21,7) rectangle ++(1,1);
\path[fill=white] (22,7) rectangle ++(1,1);
\path[fill=white] (23,7) rectangle ++(1,1);
\path[fill=white] (24,7) rectangle ++(1,1);
\path[fill=white] (25,7) rectangle ++(1,1);
\path[fill=white] (26,7) rectangle ++(1,1);
\path[fill=white] (27,7) rectangle ++(1,1);
\path[fill=white] (28,7) rectangle ++(1,1);
\path[fill=white] (29,7) rectangle ++(1,1);
\path[fill=black] (30,7) rectangle ++(1,1);
\path[fill=black] (31,7) rectangle ++(1,1);
\path[fill=white] (32,7) rectangle ++(1,1);
\path[fill=black] (33,7) rectangle ++(1,1);
\path[fill=white] (34,7) rectangle ++(1,1);
\path[fill=white] (0,6) rectangle ++(1,1);
\path[fill=white] (1,6) rectangle ++(1,1);
\path[fill=white] (2,6) rectangle ++(1,1);
\path[fill=white] (3,6) rectangle ++(1,1);
\path[fill=white] (4,6) rectangle ++(1,1);
\path[fill=white] (5,6) rectangle ++(1,1);
\path[fill=white] (6,6) rectangle ++(1,1);
\path[fill=white] (7,6) rectangle ++(1,1);
\path[fill=white] (8,6) rectangle ++(1,1);
\path[fill=white] (9,6) rectangle ++(1,1);
\path[fill=white] (10,6) rectangle ++(1,1);
\path[fill=white] (11,6) rectangle ++(1,1);
\path[fill=white] (12,6) rectangle ++(1,1);
\path[fill=white] (13,6) rectangle ++(1,1);
\path[fill=white] (14,6) rectangle ++(1,1);
\path[fill=white] (15,6) rectangle ++(1,1);
\path[fill=white] (16,6) rectangle ++(1,1);
\path[fill=white] (17,6) rectangle ++(1,1);
\path[fill=white] (18,6) rectangle ++(1,1);
\path[fill=white] (19,6) rectangle ++(1,1);
\path[fill=white] (20,6) rectangle ++(1,1);
\path[fill=white] (21,6) rectangle ++(1,1);
\path[fill=white] (22,6) rectangle ++(1,1);
\path[fill=white] (23,6) rectangle ++(1,1);
\path[fill=white] (24,6) rectangle ++(1,1);
\path[fill=white] (25,6) rectangle ++(1,1);
\path[fill=white] (26,6) rectangle ++(1,1);
\path[fill=white] (27,6) rectangle ++(1,1);
\path[fill=white] (28,6) rectangle ++(1,1);
\path[fill=white] (29,6) rectangle ++(1,1);
\path[fill=white] (30,6) rectangle ++(1,1);
\path[fill=black] (31,6) rectangle ++(1,1);
\path[fill=white] (32,6) rectangle ++(1,1);
\path[fill=white] (33,6) rectangle ++(1,1);
\path[fill=white] (34,6) rectangle ++(1,1);
\path[fill=white] (0,5) rectangle ++(1,1);
\path[fill=white] (1,5) rectangle ++(1,1);
\path[fill=white] (2,5) rectangle ++(1,1);
\path[fill=white] (3,5) rectangle ++(1,1);
\path[fill=white] (4,5) rectangle ++(1,1);
\path[fill=white] (5,5) rectangle ++(1,1);
\path[fill=white] (6,5) rectangle ++(1,1);
\path[fill=white] (7,5) rectangle ++(1,1);
\path[fill=white] (8,5) rectangle ++(1,1);
\path[fill=white] (9,5) rectangle ++(1,1);
\path[fill=white] (10,5) rectangle ++(1,1);
\path[fill=white] (11,5) rectangle ++(1,1);
\path[fill=white] (12,5) rectangle ++(1,1);
\path[fill=white] (13,5) rectangle ++(1,1);
\path[fill=white] (14,5) rectangle ++(1,1);
\path[fill=white] (15,5) rectangle ++(1,1);
\path[fill=white] (16,5) rectangle ++(1,1);
\path[fill=white] (17,5) rectangle ++(1,1);
\path[fill=white] (18,5) rectangle ++(1,1);
\path[fill=white] (19,5) rectangle ++(1,1);
\path[fill=white] (20,5) rectangle ++(1,1);
\path[fill=white] (21,5) rectangle ++(1,1);
\path[fill=white] (22,5) rectangle ++(1,1);
\path[fill=white] (23,5) rectangle ++(1,1);
\path[fill=white] (24,5) rectangle ++(1,1);
\path[fill=white] (25,5) rectangle ++(1,1);
\path[fill=white] (26,5) rectangle ++(1,1);
\path[fill=white] (27,5) rectangle ++(1,1);
\path[fill=white] (28,5) rectangle ++(1,1);
\path[fill=white] (29,5) rectangle ++(1,1);
\path[fill=white] (30,5) rectangle ++(1,1);
\path[fill=white] (31,5) rectangle ++(1,1);
\path[fill=black] (32,5) rectangle ++(1,1);
\path[fill=white] (33,5) rectangle ++(1,1);
\path[fill=white] (34,5) rectangle ++(1,1);
\path[fill=white] (0,4) rectangle ++(1,1);
\path[fill=white] (1,4) rectangle ++(1,1);
\path[fill=white] (2,4) rectangle ++(1,1);
\path[fill=white] (3,4) rectangle ++(1,1);
\path[fill=white] (4,4) rectangle ++(1,1);
\path[fill=white] (5,4) rectangle ++(1,1);
\path[fill=white] (6,4) rectangle ++(1,1);
\path[fill=white] (7,4) rectangle ++(1,1);
\path[fill=white] (8,4) rectangle ++(1,1);
\path[fill=white] (9,4) rectangle ++(1,1);
\path[fill=white] (10,4) rectangle ++(1,1);
\path[fill=white] (11,4) rectangle ++(1,1);
\path[fill=white] (12,4) rectangle ++(1,1);
\path[fill=white] (13,4) rectangle ++(1,1);
\path[fill=white] (14,4) rectangle ++(1,1);
\path[fill=white] (15,4) rectangle ++(1,1);
\path[fill=white] (16,4) rectangle ++(1,1);
\path[fill=white] (17,4) rectangle ++(1,1);
\path[fill=white] (18,4) rectangle ++(1,1);
\path[fill=white] (19,4) rectangle ++(1,1);
\path[fill=white] (20,4) rectangle ++(1,1);
\path[fill=white] (21,4) rectangle ++(1,1);
\path[fill=white] (22,4) rectangle ++(1,1);
\path[fill=white] (23,4) rectangle ++(1,1);
\path[fill=white] (24,4) rectangle ++(1,1);
\path[fill=white] (25,4) rectangle ++(1,1);
\path[fill=white] (26,4) rectangle ++(1,1);
\path[fill=white] (27,4) rectangle ++(1,1);
\path[fill=white] (28,4) rectangle ++(1,1);
\path[fill=white] (29,4) rectangle ++(1,1);
\path[fill=white] (30,4) rectangle ++(1,1);
\path[fill=white] (31,4) rectangle ++(1,1);
\path[fill=black] (32,4) rectangle ++(1,1);
\path[fill=black] (33,4) rectangle ++(1,1);
\path[fill=black] (34,4) rectangle ++(1,1);
\path[fill=white] (0,3) rectangle ++(1,1);
\path[fill=white] (1,3) rectangle ++(1,1);
\path[fill=white] (2,3) rectangle ++(1,1);
\path[fill=white] (3,3) rectangle ++(1,1);
\path[fill=white] (4,3) rectangle ++(1,1);
\path[fill=white] (5,3) rectangle ++(1,1);
\path[fill=white] (6,3) rectangle ++(1,1);
\path[fill=white] (7,3) rectangle ++(1,1);
\path[fill=white] (8,3) rectangle ++(1,1);
\path[fill=white] (9,3) rectangle ++(1,1);
\path[fill=white] (10,3) rectangle ++(1,1);
\path[fill=white] (11,3) rectangle ++(1,1);
\path[fill=white] (12,3) rectangle ++(1,1);
\path[fill=white] (13,3) rectangle ++(1,1);
\path[fill=white] (14,3) rectangle ++(1,1);
\path[fill=white] (15,3) rectangle ++(1,1);
\path[fill=white] (16,3) rectangle ++(1,1);
\path[fill=white] (17,3) rectangle ++(1,1);
\path[fill=white] (18,3) rectangle ++(1,1);
\path[fill=white] (19,3) rectangle ++(1,1);
\path[fill=white] (20,3) rectangle ++(1,1);
\path[fill=white] (21,3) rectangle ++(1,1);
\path[fill=white] (22,3) rectangle ++(1,1);
\path[fill=white] (23,3) rectangle ++(1,1);
\path[fill=white] (24,3) rectangle ++(1,1);
\path[fill=white] (25,3) rectangle ++(1,1);
\path[fill=white] (26,3) rectangle ++(1,1);
\path[fill=white] (27,3) rectangle ++(1,1);
\path[fill=white] (28,3) rectangle ++(1,1);
\path[fill=white] (29,3) rectangle ++(1,1);
\path[fill=white] (30,3) rectangle ++(1,1);
\path[fill=white] (31,3) rectangle ++(1,1);
\path[fill=white] (32,3) rectangle ++(1,1);
\path[fill=black] (33,3) rectangle ++(1,1);
\path[fill=white] (34,3) rectangle ++(1,1);
\path[fill=white] (0,2) rectangle ++(1,1);
\path[fill=white] (1,2) rectangle ++(1,1);
\path[fill=white] (2,2) rectangle ++(1,1);
\path[fill=white] (3,2) rectangle ++(1,1);
\path[fill=white] (4,2) rectangle ++(1,1);
\path[fill=white] (5,2) rectangle ++(1,1);
\path[fill=white] (6,2) rectangle ++(1,1);
\path[fill=white] (7,2) rectangle ++(1,1);
\path[fill=white] (8,2) rectangle ++(1,1);
\path[fill=white] (9,2) rectangle ++(1,1);
\path[fill=white] (10,2) rectangle ++(1,1);
\path[fill=white] (11,2) rectangle ++(1,1);
\path[fill=white] (12,2) rectangle ++(1,1);
\path[fill=white] (13,2) rectangle ++(1,1);
\path[fill=white] (14,2) rectangle ++(1,1);
\path[fill=white] (15,2) rectangle ++(1,1);
\path[fill=white] (16,2) rectangle ++(1,1);
\path[fill=white] (17,2) rectangle ++(1,1);
\path[fill=white] (18,2) rectangle ++(1,1);
\path[fill=white] (19,2) rectangle ++(1,1);
\path[fill=white] (20,2) rectangle ++(1,1);
\path[fill=white] (21,2) rectangle ++(1,1);
\path[fill=white] (22,2) rectangle ++(1,1);
\path[fill=white] (23,2) rectangle ++(1,1);
\path[fill=white] (24,2) rectangle ++(1,1);
\path[fill=white] (25,2) rectangle ++(1,1);
\path[fill=white] (26,2) rectangle ++(1,1);
\path[fill=white] (27,2) rectangle ++(1,1);
\path[fill=white] (28,2) rectangle ++(1,1);
\path[fill=white] (29,2) rectangle ++(1,1);
\path[fill=white] (30,2) rectangle ++(1,1);
\path[fill=white] (31,2) rectangle ++(1,1);
\path[fill=white] (32,2) rectangle ++(1,1);
\path[fill=white] (33,2) rectangle ++(1,1);
\path[fill=black] (34,2) rectangle ++(1,1);
\path[fill=white] (0,1) rectangle ++(1,1);
\path[fill=white] (1,1) rectangle ++(1,1);
\path[fill=white] (2,1) rectangle ++(1,1);
\path[fill=white] (3,1) rectangle ++(1,1);
\path[fill=white] (4,1) rectangle ++(1,1);
\path[fill=white] (5,1) rectangle ++(1,1);
\path[fill=white] (6,1) rectangle ++(1,1);
\path[fill=white] (7,1) rectangle ++(1,1);
\path[fill=white] (8,1) rectangle ++(1,1);
\path[fill=white] (9,1) rectangle ++(1,1);
\path[fill=white] (10,1) rectangle ++(1,1);
\path[fill=white] (11,1) rectangle ++(1,1);
\path[fill=white] (12,1) rectangle ++(1,1);
\path[fill=white] (13,1) rectangle ++(1,1);
\path[fill=white] (14,1) rectangle ++(1,1);
\path[fill=white] (15,1) rectangle ++(1,1);
\path[fill=white] (16,1) rectangle ++(1,1);
\path[fill=white] (17,1) rectangle ++(1,1);
\path[fill=white] (18,1) rectangle ++(1,1);
\path[fill=white] (19,1) rectangle ++(1,1);
\path[fill=white] (20,1) rectangle ++(1,1);
\path[fill=white] (21,1) rectangle ++(1,1);
\path[fill=white] (22,1) rectangle ++(1,1);
\path[fill=white] (23,1) rectangle ++(1,1);
\path[fill=white] (24,1) rectangle ++(1,1);
\path[fill=white] (25,1) rectangle ++(1,1);
\path[fill=white] (26,1) rectangle ++(1,1);
\path[fill=white] (27,1) rectangle ++(1,1);
\path[fill=white] (28,1) rectangle ++(1,1);
\path[fill=white] (29,1) rectangle ++(1,1);
\path[fill=white] (30,1) rectangle ++(1,1);
\path[fill=white] (31,1) rectangle ++(1,1);
\path[fill=white] (32,1) rectangle ++(1,1);
\path[fill=white] (33,1) rectangle ++(1,1);
\path[fill=black] (34,1) rectangle ++(1,1);
\path[fill=white] (0,0) rectangle ++(1,1);
\path[fill=white] (1,0) rectangle ++(1,1);
\path[fill=white] (2,0) rectangle ++(1,1);
\path[fill=white] (3,0) rectangle ++(1,1);
\path[fill=white] (4,0) rectangle ++(1,1);
\path[fill=white] (5,0) rectangle ++(1,1);
\path[fill=white] (6,0) rectangle ++(1,1);
\path[fill=white] (7,0) rectangle ++(1,1);
\path[fill=white] (8,0) rectangle ++(1,1);
\path[fill=white] (9,0) rectangle ++(1,1);
\path[fill=white] (10,0) rectangle ++(1,1);
\path[fill=white] (11,0) rectangle ++(1,1);
\path[fill=white] (12,0) rectangle ++(1,1);
\path[fill=white] (13,0) rectangle ++(1,1);
\path[fill=white] (14,0) rectangle ++(1,1);
\path[fill=white] (15,0) rectangle ++(1,1);
\path[fill=white] (16,0) rectangle ++(1,1);
\path[fill=white] (17,0) rectangle ++(1,1);
\path[fill=white] (18,0) rectangle ++(1,1);
\path[fill=white] (19,0) rectangle ++(1,1);
\path[fill=white] (20,0) rectangle ++(1,1);
\path[fill=white] (21,0) rectangle ++(1,1);
\path[fill=white] (22,0) rectangle ++(1,1);
\path[fill=white] (23,0) rectangle ++(1,1);
\path[fill=white] (24,0) rectangle ++(1,1);
\path[fill=white] (25,0) rectangle ++(1,1);
\path[fill=white] (26,0) rectangle ++(1,1);
\path[fill=white] (27,0) rectangle ++(1,1);
\path[fill=white] (28,0) rectangle ++(1,1);
\path[fill=white] (29,0) rectangle ++(1,1);
\path[fill=white] (30,0) rectangle ++(1,1);
\path[fill=white] (31,0) rectangle ++(1,1);
\path[fill=white] (32,0) rectangle ++(1,1);
\path[fill=white] (33,0) rectangle ++(1,1);
\path[fill=white] (34,0) rectangle ++(1,1);
\end{tikzpicture}

\begin{tikzpicture}[x=0.3cm, y=0.3cm, draw=gray, very thin]
\path[fill=white] (0,34) rectangle ++(1,1);
\path[fill=black] (1,34) rectangle ++(1,1);
\path[fill=black] (2,34) rectangle ++(1,1);
\path[fill=black] (3,34) rectangle ++(1,1);
\path[fill=black] (4,34) rectangle ++(1,1);
\path[fill=black] (5,34) rectangle ++(1,1);
\path[fill=black] (6,34) rectangle ++(1,1);
\path[fill=black] (7,34) rectangle ++(1,1);
\path[fill=black] (8,34) rectangle ++(1,1);
\path[fill=black] (9,34) rectangle ++(1,1);
\path[fill=black] (10,34) rectangle ++(1,1);
\path[fill=black] (11,34) rectangle ++(1,1);
\path[fill=black] (12,34) rectangle ++(1,1);
\path[fill=black] (13,34) rectangle ++(1,1);
\path[fill=black] (14,34) rectangle ++(1,1);
\path[fill=black] (15,34) rectangle ++(1,1);
\path[fill=black] (16,34) rectangle ++(1,1);
\path[fill=black] (17,34) rectangle ++(1,1);
\path[fill=black] (18,34) rectangle ++(1,1);
\path[fill=black] (19,34) rectangle ++(1,1);
\path[fill=black] (20,34) rectangle ++(1,1);
\path[fill=black] (21,34) rectangle ++(1,1);
\path[fill=black] (22,34) rectangle ++(1,1);
\path[fill=black] (23,34) rectangle ++(1,1);
\path[fill=black] (24,34) rectangle ++(1,1);
\path[fill=black] (25,34) rectangle ++(1,1);
\path[fill=black] (26,34) rectangle ++(1,1);
\path[fill=black] (27,34) rectangle ++(1,1);
\path[fill=black] (28,34) rectangle ++(1,1);
\path[fill=black] (29,34) rectangle ++(1,1);
\path[fill=black] (30,34) rectangle ++(1,1);
\path[fill=black] (31,34) rectangle ++(1,1);
\path[fill=black] (32,34) rectangle ++(1,1);
\path[fill=black] (33,34) rectangle ++(1,1);
\path[fill=black] (34,34) rectangle ++(1,1);
\path[fill=white] (0,33) rectangle ++(1,1);
\path[fill=white] (1,33) rectangle ++(1,1);
\path[fill=white] (2,33) rectangle ++(1,1);
\path[fill=black] (3,33) rectangle ++(1,1);
\path[fill=black] (4,33) rectangle ++(1,1);
\path[fill=white] (5,33) rectangle ++(1,1);
\path[fill=black] (6,33) rectangle ++(1,1);
\path[fill=black] (7,33) rectangle ++(1,1);
\path[fill=black] (8,33) rectangle ++(1,1);
\path[fill=white] (9,33) rectangle ++(1,1);
\path[fill=black] (10,33) rectangle ++(1,1);
\path[fill=black] (11,33) rectangle ++(1,1);
\path[fill=black] (12,33) rectangle ++(1,1);
\path[fill=black] (13,33) rectangle ++(1,1);
\path[fill=white] (14,33) rectangle ++(1,1);
\path[fill=black] (15,33) rectangle ++(1,1);
\path[fill=black] (16,33) rectangle ++(1,1);
\path[fill=black] (17,33) rectangle ++(1,1);
\path[fill=black] (18,33) rectangle ++(1,1);
\path[fill=white] (19,33) rectangle ++(1,1);
\path[fill=black] (20,33) rectangle ++(1,1);
\path[fill=black] (21,33) rectangle ++(1,1);
\path[fill=black] (22,33) rectangle ++(1,1);
\path[fill=black] (23,33) rectangle ++(1,1);
\path[fill=white] (24,33) rectangle ++(1,1);
\path[fill=black] (25,33) rectangle ++(1,1);
\path[fill=black] (26,33) rectangle ++(1,1);
\path[fill=black] (27,33) rectangle ++(1,1);
\path[fill=black] (28,33) rectangle ++(1,1);
\path[fill=black] (29,33) rectangle ++(1,1);
\path[fill=black] (30,33) rectangle ++(1,1);
\path[fill=black] (31,33) rectangle ++(1,1);
\path[fill=black] (32,33) rectangle ++(1,1);
\path[fill=black] (33,33) rectangle ++(1,1);
\path[fill=black] (34,33) rectangle ++(1,1);
\path[fill=white] (0,32) rectangle ++(1,1);
\path[fill=white] (1,32) rectangle ++(1,1);
\path[fill=white] (2,32) rectangle ++(1,1);
\path[fill=white] (3,32) rectangle ++(1,1);
\path[fill=black] (4,32) rectangle ++(1,1);
\path[fill=black] (5,32) rectangle ++(1,1);
\path[fill=white] (6,32) rectangle ++(1,1);
\path[fill=black] (7,32) rectangle ++(1,1);
\path[fill=black] (8,32) rectangle ++(1,1);
\path[fill=black] (9,32) rectangle ++(1,1);
\path[fill=white] (10,32) rectangle ++(1,1);
\path[fill=black] (11,32) rectangle ++(1,1);
\path[fill=black] (12,32) rectangle ++(1,1);
\path[fill=black] (13,32) rectangle ++(1,1);
\path[fill=black] (14,32) rectangle ++(1,1);
\path[fill=black] (15,32) rectangle ++(1,1);
\path[fill=black] (16,32) rectangle ++(1,1);
\path[fill=black] (17,32) rectangle ++(1,1);
\path[fill=black] (18,32) rectangle ++(1,1);
\path[fill=black] (19,32) rectangle ++(1,1);
\path[fill=black] (20,32) rectangle ++(1,1);
\path[fill=black] (21,32) rectangle ++(1,1);
\path[fill=black] (22,32) rectangle ++(1,1);
\path[fill=black] (23,32) rectangle ++(1,1);
\path[fill=black] (24,32) rectangle ++(1,1);
\path[fill=black] (25,32) rectangle ++(1,1);
\path[fill=black] (26,32) rectangle ++(1,1);
\path[fill=black] (27,32) rectangle ++(1,1);
\path[fill=black] (28,32) rectangle ++(1,1);
\path[fill=black] (29,32) rectangle ++(1,1);
\path[fill=black] (30,32) rectangle ++(1,1);
\path[fill=black] (31,32) rectangle ++(1,1);
\path[fill=black] (32,32) rectangle ++(1,1);
\path[fill=black] (33,32) rectangle ++(1,1);
\path[fill=black] (34,32) rectangle ++(1,1);
\path[fill=white] (0,31) rectangle ++(1,1);
\path[fill=white] (1,31) rectangle ++(1,1);
\path[fill=white] (2,31) rectangle ++(1,1);
\path[fill=white] (3,31) rectangle ++(1,1);
\path[fill=white] (4,31) rectangle ++(1,1);
\path[fill=white] (5,31) rectangle ++(1,1);
\path[fill=black] (6,31) rectangle ++(1,1);
\path[fill=black] (7,31) rectangle ++(1,1);
\path[fill=white] (8,31) rectangle ++(1,1);
\path[fill=white] (9,31) rectangle ++(1,1);
\path[fill=black] (10,31) rectangle ++(1,1);
\path[fill=black] (11,31) rectangle ++(1,1);
\path[fill=black] (12,31) rectangle ++(1,1);
\path[fill=white] (13,31) rectangle ++(1,1);
\path[fill=white] (14,31) rectangle ++(1,1);
\path[fill=black] (15,31) rectangle ++(1,1);
\path[fill=black] (16,31) rectangle ++(1,1);
\path[fill=black] (17,31) rectangle ++(1,1);
\path[fill=white] (18,31) rectangle ++(1,1);
\path[fill=white] (19,31) rectangle ++(1,1);
\path[fill=black] (20,31) rectangle ++(1,1);
\path[fill=black] (21,31) rectangle ++(1,1);
\path[fill=black] (22,31) rectangle ++(1,1);
\path[fill=white] (23,31) rectangle ++(1,1);
\path[fill=white] (24,31) rectangle ++(1,1);
\path[fill=black] (25,31) rectangle ++(1,1);
\path[fill=black] (26,31) rectangle ++(1,1);
\path[fill=black] (27,31) rectangle ++(1,1);
\path[fill=white] (28,31) rectangle ++(1,1);
\path[fill=black] (29,31) rectangle ++(1,1);
\path[fill=black] (30,31) rectangle ++(1,1);
\path[fill=black] (31,31) rectangle ++(1,1);
\path[fill=black] (32,31) rectangle ++(1,1);
\path[fill=black] (33,31) rectangle ++(1,1);
\path[fill=black] (34,31) rectangle ++(1,1);
\path[fill=white] (0,30) rectangle ++(1,1);
\path[fill=white] (1,30) rectangle ++(1,1);
\path[fill=white] (2,30) rectangle ++(1,1);
\path[fill=white] (3,30) rectangle ++(1,1);
\path[fill=white] (4,30) rectangle ++(1,1);
\path[fill=white] (5,30) rectangle ++(1,1);
\path[fill=white] (6,30) rectangle ++(1,1);
\path[fill=black] (7,30) rectangle ++(1,1);
\path[fill=black] (8,30) rectangle ++(1,1);
\path[fill=white] (9,30) rectangle ++(1,1);
\path[fill=white] (10,30) rectangle ++(1,1);
\path[fill=black] (11,30) rectangle ++(1,1);
\path[fill=black] (12,30) rectangle ++(1,1);
\path[fill=black] (13,30) rectangle ++(1,1);
\path[fill=white] (14,30) rectangle ++(1,1);
\path[fill=black] (15,30) rectangle ++(1,1);
\path[fill=black] (16,30) rectangle ++(1,1);
\path[fill=black] (17,30) rectangle ++(1,1);
\path[fill=black] (18,30) rectangle ++(1,1);
\path[fill=white] (19,30) rectangle ++(1,1);
\path[fill=black] (20,30) rectangle ++(1,1);
\path[fill=black] (21,30) rectangle ++(1,1);
\path[fill=black] (22,30) rectangle ++(1,1);
\path[fill=black] (23,30) rectangle ++(1,1);
\path[fill=white] (24,30) rectangle ++(1,1);
\path[fill=black] (25,30) rectangle ++(1,1);
\path[fill=black] (26,30) rectangle ++(1,1);
\path[fill=black] (27,30) rectangle ++(1,1);
\path[fill=black] (28,30) rectangle ++(1,1);
\path[fill=black] (29,30) rectangle ++(1,1);
\path[fill=black] (30,30) rectangle ++(1,1);
\path[fill=black] (31,30) rectangle ++(1,1);
\path[fill=black] (32,30) rectangle ++(1,1);
\path[fill=black] (33,30) rectangle ++(1,1);
\path[fill=black] (34,30) rectangle ++(1,1);
\path[fill=white] (0,29) rectangle ++(1,1);
\path[fill=white] (1,29) rectangle ++(1,1);
\path[fill=white] (2,29) rectangle ++(1,1);
\path[fill=white] (3,29) rectangle ++(1,1);
\path[fill=white] (4,29) rectangle ++(1,1);
\path[fill=white] (5,29) rectangle ++(1,1);
\path[fill=white] (6,29) rectangle ++(1,1);
\path[fill=white] (7,29) rectangle ++(1,1);
\path[fill=black] (8,29) rectangle ++(1,1);
\path[fill=black] (9,29) rectangle ++(1,1);
\path[fill=white] (10,29) rectangle ++(1,1);
\path[fill=white] (11,29) rectangle ++(1,1);
\path[fill=black] (12,29) rectangle ++(1,1);
\path[fill=black] (13,29) rectangle ++(1,1);
\path[fill=black] (14,29) rectangle ++(1,1);
\path[fill=white] (15,29) rectangle ++(1,1);
\path[fill=black] (16,29) rectangle ++(1,1);
\path[fill=black] (17,29) rectangle ++(1,1);
\path[fill=black] (18,29) rectangle ++(1,1);
\path[fill=black] (19,29) rectangle ++(1,1);
\path[fill=black] (20,29) rectangle ++(1,1);
\path[fill=black] (21,29) rectangle ++(1,1);
\path[fill=black] (22,29) rectangle ++(1,1);
\path[fill=black] (23,29) rectangle ++(1,1);
\path[fill=black] (24,29) rectangle ++(1,1);
\path[fill=black] (25,29) rectangle ++(1,1);
\path[fill=black] (26,29) rectangle ++(1,1);
\path[fill=black] (27,29) rectangle ++(1,1);
\path[fill=black] (28,29) rectangle ++(1,1);
\path[fill=black] (29,29) rectangle ++(1,1);
\path[fill=black] (30,29) rectangle ++(1,1);
\path[fill=black] (31,29) rectangle ++(1,1);
\path[fill=black] (32,29) rectangle ++(1,1);
\path[fill=black] (33,29) rectangle ++(1,1);
\path[fill=black] (34,29) rectangle ++(1,1);
\path[fill=white] (0,28) rectangle ++(1,1);
\path[fill=white] (1,28) rectangle ++(1,1);
\path[fill=white] (2,28) rectangle ++(1,1);
\path[fill=white] (3,28) rectangle ++(1,1);
\path[fill=white] (4,28) rectangle ++(1,1);
\path[fill=white] (5,28) rectangle ++(1,1);
\path[fill=white] (6,28) rectangle ++(1,1);
\path[fill=white] (7,28) rectangle ++(1,1);
\path[fill=white] (8,28) rectangle ++(1,1);
\path[fill=white] (9,28) rectangle ++(1,1);
\path[fill=black] (10,28) rectangle ++(1,1);
\path[fill=black] (11,28) rectangle ++(1,1);
\path[fill=white] (12,28) rectangle ++(1,1);
\path[fill=white] (13,28) rectangle ++(1,1);
\path[fill=white] (14,28) rectangle ++(1,1);
\path[fill=black] (15,28) rectangle ++(1,1);
\path[fill=black] (16,28) rectangle ++(1,1);
\path[fill=white] (17,28) rectangle ++(1,1);
\path[fill=white] (18,28) rectangle ++(1,1);
\path[fill=white] (19,28) rectangle ++(1,1);
\path[fill=black] (20,28) rectangle ++(1,1);
\path[fill=black] (21,28) rectangle ++(1,1);
\path[fill=white] (22,28) rectangle ++(1,1);
\path[fill=white] (23,28) rectangle ++(1,1);
\path[fill=white] (24,28) rectangle ++(1,1);
\path[fill=black] (25,28) rectangle ++(1,1);
\path[fill=black] (26,28) rectangle ++(1,1);
\path[fill=white] (27,28) rectangle ++(1,1);
\path[fill=white] (28,28) rectangle ++(1,1);
\path[fill=black] (29,28) rectangle ++(1,1);
\path[fill=black] (30,28) rectangle ++(1,1);
\path[fill=white] (31,28) rectangle ++(1,1);
\path[fill=black] (32,28) rectangle ++(1,1);
\path[fill=black] (33,28) rectangle ++(1,1);
\path[fill=black] (34,28) rectangle ++(1,1);
\path[fill=white] (0,27) rectangle ++(1,1);
\path[fill=white] (1,27) rectangle ++(1,1);
\path[fill=white] (2,27) rectangle ++(1,1);
\path[fill=white] (3,27) rectangle ++(1,1);
\path[fill=white] (4,27) rectangle ++(1,1);
\path[fill=white] (5,27) rectangle ++(1,1);
\path[fill=white] (6,27) rectangle ++(1,1);
\path[fill=white] (7,27) rectangle ++(1,1);
\path[fill=white] (8,27) rectangle ++(1,1);
\path[fill=white] (9,27) rectangle ++(1,1);
\path[fill=white] (10,27) rectangle ++(1,1);
\path[fill=black] (11,27) rectangle ++(1,1);
\path[fill=black] (12,27) rectangle ++(1,1);
\path[fill=white] (13,27) rectangle ++(1,1);
\path[fill=white] (14,27) rectangle ++(1,1);
\path[fill=black] (15,27) rectangle ++(1,1);
\path[fill=black] (16,27) rectangle ++(1,1);
\path[fill=black] (17,27) rectangle ++(1,1);
\path[fill=white] (18,27) rectangle ++(1,1);
\path[fill=white] (19,27) rectangle ++(1,1);
\path[fill=black] (20,27) rectangle ++(1,1);
\path[fill=black] (21,27) rectangle ++(1,1);
\path[fill=black] (22,27) rectangle ++(1,1);
\path[fill=white] (23,27) rectangle ++(1,1);
\path[fill=white] (24,27) rectangle ++(1,1);
\path[fill=black] (25,27) rectangle ++(1,1);
\path[fill=black] (26,27) rectangle ++(1,1);
\path[fill=black] (27,27) rectangle ++(1,1);
\path[fill=white] (28,27) rectangle ++(1,1);
\path[fill=black] (29,27) rectangle ++(1,1);
\path[fill=black] (30,27) rectangle ++(1,1);
\path[fill=black] (31,27) rectangle ++(1,1);
\path[fill=black] (32,27) rectangle ++(1,1);
\path[fill=black] (33,27) rectangle ++(1,1);
\path[fill=black] (34,27) rectangle ++(1,1);
\path[fill=white] (0,26) rectangle ++(1,1);
\path[fill=white] (1,26) rectangle ++(1,1);
\path[fill=white] (2,26) rectangle ++(1,1);
\path[fill=white] (3,26) rectangle ++(1,1);
\path[fill=white] (4,26) rectangle ++(1,1);
\path[fill=white] (5,26) rectangle ++(1,1);
\path[fill=white] (6,26) rectangle ++(1,1);
\path[fill=white] (7,26) rectangle ++(1,1);
\path[fill=white] (8,26) rectangle ++(1,1);
\path[fill=white] (9,26) rectangle ++(1,1);
\path[fill=white] (10,26) rectangle ++(1,1);
\path[fill=white] (11,26) rectangle ++(1,1);
\path[fill=black] (12,26) rectangle ++(1,1);
\path[fill=black] (13,26) rectangle ++(1,1);
\path[fill=white] (14,26) rectangle ++(1,1);
\path[fill=white] (15,26) rectangle ++(1,1);
\path[fill=black] (16,26) rectangle ++(1,1);
\path[fill=black] (17,26) rectangle ++(1,1);
\path[fill=black] (18,26) rectangle ++(1,1);
\path[fill=white] (19,26) rectangle ++(1,1);
\path[fill=black] (20,26) rectangle ++(1,1);
\path[fill=black] (21,26) rectangle ++(1,1);
\path[fill=black] (22,26) rectangle ++(1,1);
\path[fill=black] (23,26) rectangle ++(1,1);
\path[fill=white] (24,26) rectangle ++(1,1);
\path[fill=black] (25,26) rectangle ++(1,1);
\path[fill=black] (26,26) rectangle ++(1,1);
\path[fill=black] (27,26) rectangle ++(1,1);
\path[fill=black] (28,26) rectangle ++(1,1);
\path[fill=black] (29,26) rectangle ++(1,1);
\path[fill=black] (30,26) rectangle ++(1,1);
\path[fill=black] (31,26) rectangle ++(1,1);
\path[fill=black] (32,26) rectangle ++(1,1);
\path[fill=black] (33,26) rectangle ++(1,1);
\path[fill=black] (34,26) rectangle ++(1,1);
\path[fill=white] (0,25) rectangle ++(1,1);
\path[fill=white] (1,25) rectangle ++(1,1);
\path[fill=white] (2,25) rectangle ++(1,1);
\path[fill=white] (3,25) rectangle ++(1,1);
\path[fill=white] (4,25) rectangle ++(1,1);
\path[fill=white] (5,25) rectangle ++(1,1);
\path[fill=white] (6,25) rectangle ++(1,1);
\path[fill=white] (7,25) rectangle ++(1,1);
\path[fill=white] (8,25) rectangle ++(1,1);
\path[fill=white] (9,25) rectangle ++(1,1);
\path[fill=white] (10,25) rectangle ++(1,1);
\path[fill=white] (11,25) rectangle ++(1,1);
\path[fill=white] (12,25) rectangle ++(1,1);
\path[fill=black] (13,25) rectangle ++(1,1);
\path[fill=black] (14,25) rectangle ++(1,1);
\path[fill=white] (15,25) rectangle ++(1,1);
\path[fill=white] (16,25) rectangle ++(1,1);
\path[fill=black] (17,25) rectangle ++(1,1);
\path[fill=black] (18,25) rectangle ++(1,1);
\path[fill=black] (19,25) rectangle ++(1,1);
\path[fill=white] (20,25) rectangle ++(1,1);
\path[fill=black] (21,25) rectangle ++(1,1);
\path[fill=black] (22,25) rectangle ++(1,1);
\path[fill=black] (23,25) rectangle ++(1,1);
\path[fill=black] (24,25) rectangle ++(1,1);
\path[fill=black] (25,25) rectangle ++(1,1);
\path[fill=black] (26,25) rectangle ++(1,1);
\path[fill=black] (27,25) rectangle ++(1,1);
\path[fill=black] (28,25) rectangle ++(1,1);
\path[fill=black] (29,25) rectangle ++(1,1);
\path[fill=black] (30,25) rectangle ++(1,1);
\path[fill=black] (31,25) rectangle ++(1,1);
\path[fill=black] (32,25) rectangle ++(1,1);
\path[fill=black] (33,25) rectangle ++(1,1);
\path[fill=black] (34,25) rectangle ++(1,1);
\path[fill=white] (0,24) rectangle ++(1,1);
\path[fill=white] (1,24) rectangle ++(1,1);
\path[fill=white] (2,24) rectangle ++(1,1);
\path[fill=white] (3,24) rectangle ++(1,1);
\path[fill=white] (4,24) rectangle ++(1,1);
\path[fill=white] (5,24) rectangle ++(1,1);
\path[fill=white] (6,24) rectangle ++(1,1);
\path[fill=white] (7,24) rectangle ++(1,1);
\path[fill=white] (8,24) rectangle ++(1,1);
\path[fill=white] (9,24) rectangle ++(1,1);
\path[fill=white] (10,24) rectangle ++(1,1);
\path[fill=white] (11,24) rectangle ++(1,1);
\path[fill=white] (12,24) rectangle ++(1,1);
\path[fill=white] (13,24) rectangle ++(1,1);
\path[fill=white] (14,24) rectangle ++(1,1);
\path[fill=black] (15,24) rectangle ++(1,1);
\path[fill=white] (16,24) rectangle ++(1,1);
\path[fill=white] (17,24) rectangle ++(1,1);
\path[fill=white] (18,24) rectangle ++(1,1);
\path[fill=white] (19,24) rectangle ++(1,1);
\path[fill=black] (20,24) rectangle ++(1,1);
\path[fill=white] (21,24) rectangle ++(1,1);
\path[fill=white] (22,24) rectangle ++(1,1);
\path[fill=white] (23,24) rectangle ++(1,1);
\path[fill=white] (24,24) rectangle ++(1,1);
\path[fill=black] (25,24) rectangle ++(1,1);
\path[fill=white] (26,24) rectangle ++(1,1);
\path[fill=white] (27,24) rectangle ++(1,1);
\path[fill=white] (28,24) rectangle ++(1,1);
\path[fill=black] (29,24) rectangle ++(1,1);
\path[fill=white] (30,24) rectangle ++(1,1);
\path[fill=white] (31,24) rectangle ++(1,1);
\path[fill=black] (32,24) rectangle ++(1,1);
\path[fill=white] (33,24) rectangle ++(1,1);
\path[fill=black] (34,24) rectangle ++(1,1);
\path[fill=white] (0,23) rectangle ++(1,1);
\path[fill=white] (1,23) rectangle ++(1,1);
\path[fill=white] (2,23) rectangle ++(1,1);
\path[fill=white] (3,23) rectangle ++(1,1);
\path[fill=white] (4,23) rectangle ++(1,1);
\path[fill=white] (5,23) rectangle ++(1,1);
\path[fill=white] (6,23) rectangle ++(1,1);
\path[fill=white] (7,23) rectangle ++(1,1);
\path[fill=white] (8,23) rectangle ++(1,1);
\path[fill=white] (9,23) rectangle ++(1,1);
\path[fill=white] (10,23) rectangle ++(1,1);
\path[fill=white] (11,23) rectangle ++(1,1);
\path[fill=white] (12,23) rectangle ++(1,1);
\path[fill=white] (13,23) rectangle ++(1,1);
\path[fill=white] (14,23) rectangle ++(1,1);
\path[fill=black] (15,23) rectangle ++(1,1);
\path[fill=black] (16,23) rectangle ++(1,1);
\path[fill=white] (17,23) rectangle ++(1,1);
\path[fill=white] (18,23) rectangle ++(1,1);
\path[fill=white] (19,23) rectangle ++(1,1);
\path[fill=black] (20,23) rectangle ++(1,1);
\path[fill=black] (21,23) rectangle ++(1,1);
\path[fill=white] (22,23) rectangle ++(1,1);
\path[fill=white] (23,23) rectangle ++(1,1);
\path[fill=white] (24,23) rectangle ++(1,1);
\path[fill=black] (25,23) rectangle ++(1,1);
\path[fill=black] (26,23) rectangle ++(1,1);
\path[fill=white] (27,23) rectangle ++(1,1);
\path[fill=white] (28,23) rectangle ++(1,1);
\path[fill=black] (29,23) rectangle ++(1,1);
\path[fill=black] (30,23) rectangle ++(1,1);
\path[fill=white] (31,23) rectangle ++(1,1);
\path[fill=black] (32,23) rectangle ++(1,1);
\path[fill=black] (33,23) rectangle ++(1,1);
\path[fill=black] (34,23) rectangle ++(1,1);
\path[fill=white] (0,22) rectangle ++(1,1);
\path[fill=white] (1,22) rectangle ++(1,1);
\path[fill=white] (2,22) rectangle ++(1,1);
\path[fill=white] (3,22) rectangle ++(1,1);
\path[fill=white] (4,22) rectangle ++(1,1);
\path[fill=white] (5,22) rectangle ++(1,1);
\path[fill=white] (6,22) rectangle ++(1,1);
\path[fill=white] (7,22) rectangle ++(1,1);
\path[fill=white] (8,22) rectangle ++(1,1);
\path[fill=white] (9,22) rectangle ++(1,1);
\path[fill=white] (10,22) rectangle ++(1,1);
\path[fill=white] (11,22) rectangle ++(1,1);
\path[fill=white] (12,22) rectangle ++(1,1);
\path[fill=white] (13,22) rectangle ++(1,1);
\path[fill=white] (14,22) rectangle ++(1,1);
\path[fill=white] (15,22) rectangle ++(1,1);
\path[fill=black] (16,22) rectangle ++(1,1);
\path[fill=black] (17,22) rectangle ++(1,1);
\path[fill=white] (18,22) rectangle ++(1,1);
\path[fill=white] (19,22) rectangle ++(1,1);
\path[fill=black] (20,22) rectangle ++(1,1);
\path[fill=black] (21,22) rectangle ++(1,1);
\path[fill=black] (22,22) rectangle ++(1,1);
\path[fill=white] (23,22) rectangle ++(1,1);
\path[fill=white] (24,22) rectangle ++(1,1);
\path[fill=black] (25,22) rectangle ++(1,1);
\path[fill=black] (26,22) rectangle ++(1,1);
\path[fill=black] (27,22) rectangle ++(1,1);
\path[fill=white] (28,22) rectangle ++(1,1);
\path[fill=black] (29,22) rectangle ++(1,1);
\path[fill=black] (30,22) rectangle ++(1,1);
\path[fill=black] (31,22) rectangle ++(1,1);
\path[fill=black] (32,22) rectangle ++(1,1);
\path[fill=black] (33,22) rectangle ++(1,1);
\path[fill=black] (34,22) rectangle ++(1,1);
\path[fill=white] (0,21) rectangle ++(1,1);
\path[fill=white] (1,21) rectangle ++(1,1);
\path[fill=white] (2,21) rectangle ++(1,1);
\path[fill=white] (3,21) rectangle ++(1,1);
\path[fill=white] (4,21) rectangle ++(1,1);
\path[fill=white] (5,21) rectangle ++(1,1);
\path[fill=white] (6,21) rectangle ++(1,1);
\path[fill=white] (7,21) rectangle ++(1,1);
\path[fill=white] (8,21) rectangle ++(1,1);
\path[fill=white] (9,21) rectangle ++(1,1);
\path[fill=white] (10,21) rectangle ++(1,1);
\path[fill=white] (11,21) rectangle ++(1,1);
\path[fill=white] (12,21) rectangle ++(1,1);
\path[fill=white] (13,21) rectangle ++(1,1);
\path[fill=white] (14,21) rectangle ++(1,1);
\path[fill=white] (15,21) rectangle ++(1,1);
\path[fill=white] (16,21) rectangle ++(1,1);
\path[fill=black] (17,21) rectangle ++(1,1);
\path[fill=black] (18,21) rectangle ++(1,1);
\path[fill=white] (19,21) rectangle ++(1,1);
\path[fill=white] (20,21) rectangle ++(1,1);
\path[fill=black] (21,21) rectangle ++(1,1);
\path[fill=black] (22,21) rectangle ++(1,1);
\path[fill=black] (23,21) rectangle ++(1,1);
\path[fill=white] (24,21) rectangle ++(1,1);
\path[fill=black] (25,21) rectangle ++(1,1);
\path[fill=black] (26,21) rectangle ++(1,1);
\path[fill=black] (27,21) rectangle ++(1,1);
\path[fill=black] (28,21) rectangle ++(1,1);
\path[fill=black] (29,21) rectangle ++(1,1);
\path[fill=black] (30,21) rectangle ++(1,1);
\path[fill=black] (31,21) rectangle ++(1,1);
\path[fill=black] (32,21) rectangle ++(1,1);
\path[fill=black] (33,21) rectangle ++(1,1);
\path[fill=black] (34,21) rectangle ++(1,1);
\path[fill=white] (0,20) rectangle ++(1,1);
\path[fill=white] (1,20) rectangle ++(1,1);
\path[fill=white] (2,20) rectangle ++(1,1);
\path[fill=white] (3,20) rectangle ++(1,1);
\path[fill=white] (4,20) rectangle ++(1,1);
\path[fill=white] (5,20) rectangle ++(1,1);
\path[fill=white] (6,20) rectangle ++(1,1);
\path[fill=white] (7,20) rectangle ++(1,1);
\path[fill=white] (8,20) rectangle ++(1,1);
\path[fill=white] (9,20) rectangle ++(1,1);
\path[fill=white] (10,20) rectangle ++(1,1);
\path[fill=white] (11,20) rectangle ++(1,1);
\path[fill=white] (12,20) rectangle ++(1,1);
\path[fill=white] (13,20) rectangle ++(1,1);
\path[fill=white] (14,20) rectangle ++(1,1);
\path[fill=white] (15,20) rectangle ++(1,1);
\path[fill=white] (16,20) rectangle ++(1,1);
\path[fill=white] (17,20) rectangle ++(1,1);
\path[fill=black] (18,20) rectangle ++(1,1);
\path[fill=black] (19,20) rectangle ++(1,1);
\path[fill=white] (20,20) rectangle ++(1,1);
\path[fill=white] (21,20) rectangle ++(1,1);
\path[fill=black] (22,20) rectangle ++(1,1);
\path[fill=black] (23,20) rectangle ++(1,1);
\path[fill=black] (24,20) rectangle ++(1,1);
\path[fill=white] (25,20) rectangle ++(1,1);
\path[fill=black] (26,20) rectangle ++(1,1);
\path[fill=black] (27,20) rectangle ++(1,1);
\path[fill=black] (28,20) rectangle ++(1,1);
\path[fill=black] (29,20) rectangle ++(1,1);
\path[fill=black] (30,20) rectangle ++(1,1);
\path[fill=black] (31,20) rectangle ++(1,1);
\path[fill=black] (32,20) rectangle ++(1,1);
\path[fill=black] (33,20) rectangle ++(1,1);
\path[fill=black] (34,20) rectangle ++(1,1);
\path[fill=white] (0,19) rectangle ++(1,1);
\path[fill=white] (1,19) rectangle ++(1,1);
\path[fill=white] (2,19) rectangle ++(1,1);
\path[fill=white] (3,19) rectangle ++(1,1);
\path[fill=white] (4,19) rectangle ++(1,1);
\path[fill=white] (5,19) rectangle ++(1,1);
\path[fill=white] (6,19) rectangle ++(1,1);
\path[fill=white] (7,19) rectangle ++(1,1);
\path[fill=white] (8,19) rectangle ++(1,1);
\path[fill=white] (9,19) rectangle ++(1,1);
\path[fill=white] (10,19) rectangle ++(1,1);
\path[fill=white] (11,19) rectangle ++(1,1);
\path[fill=white] (12,19) rectangle ++(1,1);
\path[fill=white] (13,19) rectangle ++(1,1);
\path[fill=white] (14,19) rectangle ++(1,1);
\path[fill=white] (15,19) rectangle ++(1,1);
\path[fill=white] (16,19) rectangle ++(1,1);
\path[fill=white] (17,19) rectangle ++(1,1);
\path[fill=white] (18,19) rectangle ++(1,1);
\path[fill=white] (19,19) rectangle ++(1,1);
\path[fill=black] (20,19) rectangle ++(1,1);
\path[fill=white] (21,19) rectangle ++(1,1);
\path[fill=white] (22,19) rectangle ++(1,1);
\path[fill=white] (23,19) rectangle ++(1,1);
\path[fill=white] (24,19) rectangle ++(1,1);
\path[fill=black] (25,19) rectangle ++(1,1);
\path[fill=white] (26,19) rectangle ++(1,1);
\path[fill=white] (27,19) rectangle ++(1,1);
\path[fill=white] (28,19) rectangle ++(1,1);
\path[fill=black] (29,19) rectangle ++(1,1);
\path[fill=white] (30,19) rectangle ++(1,1);
\path[fill=white] (31,19) rectangle ++(1,1);
\path[fill=black] (32,19) rectangle ++(1,1);
\path[fill=white] (33,19) rectangle ++(1,1);
\path[fill=black] (34,19) rectangle ++(1,1);
\path[fill=white] (0,18) rectangle ++(1,1);
\path[fill=white] (1,18) rectangle ++(1,1);
\path[fill=white] (2,18) rectangle ++(1,1);
\path[fill=white] (3,18) rectangle ++(1,1);
\path[fill=white] (4,18) rectangle ++(1,1);
\path[fill=white] (5,18) rectangle ++(1,1);
\path[fill=white] (6,18) rectangle ++(1,1);
\path[fill=white] (7,18) rectangle ++(1,1);
\path[fill=white] (8,18) rectangle ++(1,1);
\path[fill=white] (9,18) rectangle ++(1,1);
\path[fill=white] (10,18) rectangle ++(1,1);
\path[fill=white] (11,18) rectangle ++(1,1);
\path[fill=white] (12,18) rectangle ++(1,1);
\path[fill=white] (13,18) rectangle ++(1,1);
\path[fill=white] (14,18) rectangle ++(1,1);
\path[fill=white] (15,18) rectangle ++(1,1);
\path[fill=white] (16,18) rectangle ++(1,1);
\path[fill=white] (17,18) rectangle ++(1,1);
\path[fill=white] (18,18) rectangle ++(1,1);
\path[fill=white] (19,18) rectangle ++(1,1);
\path[fill=black] (20,18) rectangle ++(1,1);
\path[fill=black] (21,18) rectangle ++(1,1);
\path[fill=white] (22,18) rectangle ++(1,1);
\path[fill=white] (23,18) rectangle ++(1,1);
\path[fill=white] (24,18) rectangle ++(1,1);
\path[fill=black] (25,18) rectangle ++(1,1);
\path[fill=black] (26,18) rectangle ++(1,1);
\path[fill=white] (27,18) rectangle ++(1,1);
\path[fill=white] (28,18) rectangle ++(1,1);
\path[fill=black] (29,18) rectangle ++(1,1);
\path[fill=black] (30,18) rectangle ++(1,1);
\path[fill=white] (31,18) rectangle ++(1,1);
\path[fill=black] (32,18) rectangle ++(1,1);
\path[fill=black] (33,18) rectangle ++(1,1);
\path[fill=black] (34,18) rectangle ++(1,1);
\path[fill=white] (0,17) rectangle ++(1,1);
\path[fill=white] (1,17) rectangle ++(1,1);
\path[fill=white] (2,17) rectangle ++(1,1);
\path[fill=white] (3,17) rectangle ++(1,1);
\path[fill=white] (4,17) rectangle ++(1,1);
\path[fill=white] (5,17) rectangle ++(1,1);
\path[fill=white] (6,17) rectangle ++(1,1);
\path[fill=white] (7,17) rectangle ++(1,1);
\path[fill=white] (8,17) rectangle ++(1,1);
\path[fill=white] (9,17) rectangle ++(1,1);
\path[fill=white] (10,17) rectangle ++(1,1);
\path[fill=white] (11,17) rectangle ++(1,1);
\path[fill=white] (12,17) rectangle ++(1,1);
\path[fill=white] (13,17) rectangle ++(1,1);
\path[fill=white] (14,17) rectangle ++(1,1);
\path[fill=white] (15,17) rectangle ++(1,1);
\path[fill=white] (16,17) rectangle ++(1,1);
\path[fill=white] (17,17) rectangle ++(1,1);
\path[fill=white] (18,17) rectangle ++(1,1);
\path[fill=white] (19,17) rectangle ++(1,1);
\path[fill=white] (20,17) rectangle ++(1,1);
\path[fill=black] (21,17) rectangle ++(1,1);
\path[fill=black] (22,17) rectangle ++(1,1);
\path[fill=white] (23,17) rectangle ++(1,1);
\path[fill=white] (24,17) rectangle ++(1,1);
\path[fill=black] (25,17) rectangle ++(1,1);
\path[fill=black] (26,17) rectangle ++(1,1);
\path[fill=black] (27,17) rectangle ++(1,1);
\path[fill=white] (28,17) rectangle ++(1,1);
\path[fill=black] (29,17) rectangle ++(1,1);
\path[fill=black] (30,17) rectangle ++(1,1);
\path[fill=black] (31,17) rectangle ++(1,1);
\path[fill=black] (32,17) rectangle ++(1,1);
\path[fill=black] (33,17) rectangle ++(1,1);
\path[fill=black] (34,17) rectangle ++(1,1);
\path[fill=white] (0,16) rectangle ++(1,1);
\path[fill=white] (1,16) rectangle ++(1,1);
\path[fill=white] (2,16) rectangle ++(1,1);
\path[fill=white] (3,16) rectangle ++(1,1);
\path[fill=white] (4,16) rectangle ++(1,1);
\path[fill=white] (5,16) rectangle ++(1,1);
\path[fill=white] (6,16) rectangle ++(1,1);
\path[fill=white] (7,16) rectangle ++(1,1);
\path[fill=white] (8,16) rectangle ++(1,1);
\path[fill=white] (9,16) rectangle ++(1,1);
\path[fill=white] (10,16) rectangle ++(1,1);
\path[fill=white] (11,16) rectangle ++(1,1);
\path[fill=white] (12,16) rectangle ++(1,1);
\path[fill=white] (13,16) rectangle ++(1,1);
\path[fill=white] (14,16) rectangle ++(1,1);
\path[fill=white] (15,16) rectangle ++(1,1);
\path[fill=white] (16,16) rectangle ++(1,1);
\path[fill=white] (17,16) rectangle ++(1,1);
\path[fill=white] (18,16) rectangle ++(1,1);
\path[fill=white] (19,16) rectangle ++(1,1);
\path[fill=white] (20,16) rectangle ++(1,1);
\path[fill=white] (21,16) rectangle ++(1,1);
\path[fill=black] (22,16) rectangle ++(1,1);
\path[fill=black] (23,16) rectangle ++(1,1);
\path[fill=white] (24,16) rectangle ++(1,1);
\path[fill=white] (25,16) rectangle ++(1,1);
\path[fill=black] (26,16) rectangle ++(1,1);
\path[fill=black] (27,16) rectangle ++(1,1);
\path[fill=black] (28,16) rectangle ++(1,1);
\path[fill=black] (29,16) rectangle ++(1,1);
\path[fill=black] (30,16) rectangle ++(1,1);
\path[fill=black] (31,16) rectangle ++(1,1);
\path[fill=black] (32,16) rectangle ++(1,1);
\path[fill=black] (33,16) rectangle ++(1,1);
\path[fill=black] (34,16) rectangle ++(1,1);
\path[fill=white] (0,15) rectangle ++(1,1);
\path[fill=white] (1,15) rectangle ++(1,1);
\path[fill=white] (2,15) rectangle ++(1,1);
\path[fill=white] (3,15) rectangle ++(1,1);
\path[fill=white] (4,15) rectangle ++(1,1);
\path[fill=white] (5,15) rectangle ++(1,1);
\path[fill=white] (6,15) rectangle ++(1,1);
\path[fill=white] (7,15) rectangle ++(1,1);
\path[fill=white] (8,15) rectangle ++(1,1);
\path[fill=white] (9,15) rectangle ++(1,1);
\path[fill=white] (10,15) rectangle ++(1,1);
\path[fill=white] (11,15) rectangle ++(1,1);
\path[fill=white] (12,15) rectangle ++(1,1);
\path[fill=white] (13,15) rectangle ++(1,1);
\path[fill=white] (14,15) rectangle ++(1,1);
\path[fill=white] (15,15) rectangle ++(1,1);
\path[fill=white] (16,15) rectangle ++(1,1);
\path[fill=white] (17,15) rectangle ++(1,1);
\path[fill=white] (18,15) rectangle ++(1,1);
\path[fill=white] (19,15) rectangle ++(1,1);
\path[fill=white] (20,15) rectangle ++(1,1);
\path[fill=white] (21,15) rectangle ++(1,1);
\path[fill=white] (22,15) rectangle ++(1,1);
\path[fill=black] (23,15) rectangle ++(1,1);
\path[fill=black] (24,15) rectangle ++(1,1);
\path[fill=white] (25,15) rectangle ++(1,1);
\path[fill=white] (26,15) rectangle ++(1,1);
\path[fill=black] (27,15) rectangle ++(1,1);
\path[fill=black] (28,15) rectangle ++(1,1);
\path[fill=white] (29,15) rectangle ++(1,1);
\path[fill=black] (30,15) rectangle ++(1,1);
\path[fill=black] (31,15) rectangle ++(1,1);
\path[fill=black] (32,15) rectangle ++(1,1);
\path[fill=black] (33,15) rectangle ++(1,1);
\path[fill=black] (34,15) rectangle ++(1,1);
\path[fill=white] (0,14) rectangle ++(1,1);
\path[fill=white] (1,14) rectangle ++(1,1);
\path[fill=white] (2,14) rectangle ++(1,1);
\path[fill=white] (3,14) rectangle ++(1,1);
\path[fill=white] (4,14) rectangle ++(1,1);
\path[fill=white] (5,14) rectangle ++(1,1);
\path[fill=white] (6,14) rectangle ++(1,1);
\path[fill=white] (7,14) rectangle ++(1,1);
\path[fill=white] (8,14) rectangle ++(1,1);
\path[fill=white] (9,14) rectangle ++(1,1);
\path[fill=white] (10,14) rectangle ++(1,1);
\path[fill=white] (11,14) rectangle ++(1,1);
\path[fill=white] (12,14) rectangle ++(1,1);
\path[fill=white] (13,14) rectangle ++(1,1);
\path[fill=white] (14,14) rectangle ++(1,1);
\path[fill=white] (15,14) rectangle ++(1,1);
\path[fill=white] (16,14) rectangle ++(1,1);
\path[fill=white] (17,14) rectangle ++(1,1);
\path[fill=white] (18,14) rectangle ++(1,1);
\path[fill=white] (19,14) rectangle ++(1,1);
\path[fill=white] (20,14) rectangle ++(1,1);
\path[fill=white] (21,14) rectangle ++(1,1);
\path[fill=white] (22,14) rectangle ++(1,1);
\path[fill=white] (23,14) rectangle ++(1,1);
\path[fill=white] (24,14) rectangle ++(1,1);
\path[fill=black] (25,14) rectangle ++(1,1);
\path[fill=white] (26,14) rectangle ++(1,1);
\path[fill=white] (27,14) rectangle ++(1,1);
\path[fill=white] (28,14) rectangle ++(1,1);
\path[fill=black] (29,14) rectangle ++(1,1);
\path[fill=white] (30,14) rectangle ++(1,1);
\path[fill=white] (31,14) rectangle ++(1,1);
\path[fill=black] (32,14) rectangle ++(1,1);
\path[fill=white] (33,14) rectangle ++(1,1);
\path[fill=black] (34,14) rectangle ++(1,1);
\path[fill=white] (0,13) rectangle ++(1,1);
\path[fill=white] (1,13) rectangle ++(1,1);
\path[fill=white] (2,13) rectangle ++(1,1);
\path[fill=white] (3,13) rectangle ++(1,1);
\path[fill=white] (4,13) rectangle ++(1,1);
\path[fill=white] (5,13) rectangle ++(1,1);
\path[fill=white] (6,13) rectangle ++(1,1);
\path[fill=white] (7,13) rectangle ++(1,1);
\path[fill=white] (8,13) rectangle ++(1,1);
\path[fill=white] (9,13) rectangle ++(1,1);
\path[fill=white] (10,13) rectangle ++(1,1);
\path[fill=white] (11,13) rectangle ++(1,1);
\path[fill=white] (12,13) rectangle ++(1,1);
\path[fill=white] (13,13) rectangle ++(1,1);
\path[fill=white] (14,13) rectangle ++(1,1);
\path[fill=white] (15,13) rectangle ++(1,1);
\path[fill=white] (16,13) rectangle ++(1,1);
\path[fill=white] (17,13) rectangle ++(1,1);
\path[fill=white] (18,13) rectangle ++(1,1);
\path[fill=white] (19,13) rectangle ++(1,1);
\path[fill=white] (20,13) rectangle ++(1,1);
\path[fill=white] (21,13) rectangle ++(1,1);
\path[fill=white] (22,13) rectangle ++(1,1);
\path[fill=white] (23,13) rectangle ++(1,1);
\path[fill=white] (24,13) rectangle ++(1,1);
\path[fill=black] (25,13) rectangle ++(1,1);
\path[fill=black] (26,13) rectangle ++(1,1);
\path[fill=white] (27,13) rectangle ++(1,1);
\path[fill=white] (28,13) rectangle ++(1,1);
\path[fill=black] (29,13) rectangle ++(1,1);
\path[fill=black] (30,13) rectangle ++(1,1);
\path[fill=white] (31,13) rectangle ++(1,1);
\path[fill=black] (32,13) rectangle ++(1,1);
\path[fill=black] (33,13) rectangle ++(1,1);
\path[fill=black] (34,13) rectangle ++(1,1);
\path[fill=white] (0,12) rectangle ++(1,1);
\path[fill=white] (1,12) rectangle ++(1,1);
\path[fill=white] (2,12) rectangle ++(1,1);
\path[fill=white] (3,12) rectangle ++(1,1);
\path[fill=white] (4,12) rectangle ++(1,1);
\path[fill=white] (5,12) rectangle ++(1,1);
\path[fill=white] (6,12) rectangle ++(1,1);
\path[fill=white] (7,12) rectangle ++(1,1);
\path[fill=white] (8,12) rectangle ++(1,1);
\path[fill=white] (9,12) rectangle ++(1,1);
\path[fill=white] (10,12) rectangle ++(1,1);
\path[fill=white] (11,12) rectangle ++(1,1);
\path[fill=white] (12,12) rectangle ++(1,1);
\path[fill=white] (13,12) rectangle ++(1,1);
\path[fill=white] (14,12) rectangle ++(1,1);
\path[fill=white] (15,12) rectangle ++(1,1);
\path[fill=white] (16,12) rectangle ++(1,1);
\path[fill=white] (17,12) rectangle ++(1,1);
\path[fill=white] (18,12) rectangle ++(1,1);
\path[fill=white] (19,12) rectangle ++(1,1);
\path[fill=white] (20,12) rectangle ++(1,1);
\path[fill=white] (21,12) rectangle ++(1,1);
\path[fill=white] (22,12) rectangle ++(1,1);
\path[fill=white] (23,12) rectangle ++(1,1);
\path[fill=white] (24,12) rectangle ++(1,1);
\path[fill=white] (25,12) rectangle ++(1,1);
\path[fill=black] (26,12) rectangle ++(1,1);
\path[fill=black] (27,12) rectangle ++(1,1);
\path[fill=white] (28,12) rectangle ++(1,1);
\path[fill=black] (29,12) rectangle ++(1,1);
\path[fill=black] (30,12) rectangle ++(1,1);
\path[fill=black] (31,12) rectangle ++(1,1);
\path[fill=black] (32,12) rectangle ++(1,1);
\path[fill=black] (33,12) rectangle ++(1,1);
\path[fill=black] (34,12) rectangle ++(1,1);
\path[fill=white] (0,11) rectangle ++(1,1);
\path[fill=white] (1,11) rectangle ++(1,1);
\path[fill=white] (2,11) rectangle ++(1,1);
\path[fill=white] (3,11) rectangle ++(1,1);
\path[fill=white] (4,11) rectangle ++(1,1);
\path[fill=white] (5,11) rectangle ++(1,1);
\path[fill=white] (6,11) rectangle ++(1,1);
\path[fill=white] (7,11) rectangle ++(1,1);
\path[fill=white] (8,11) rectangle ++(1,1);
\path[fill=white] (9,11) rectangle ++(1,1);
\path[fill=white] (10,11) rectangle ++(1,1);
\path[fill=white] (11,11) rectangle ++(1,1);
\path[fill=white] (12,11) rectangle ++(1,1);
\path[fill=white] (13,11) rectangle ++(1,1);
\path[fill=white] (14,11) rectangle ++(1,1);
\path[fill=white] (15,11) rectangle ++(1,1);
\path[fill=white] (16,11) rectangle ++(1,1);
\path[fill=white] (17,11) rectangle ++(1,1);
\path[fill=white] (18,11) rectangle ++(1,1);
\path[fill=white] (19,11) rectangle ++(1,1);
\path[fill=white] (20,11) rectangle ++(1,1);
\path[fill=white] (21,11) rectangle ++(1,1);
\path[fill=white] (22,11) rectangle ++(1,1);
\path[fill=white] (23,11) rectangle ++(1,1);
\path[fill=white] (24,11) rectangle ++(1,1);
\path[fill=white] (25,11) rectangle ++(1,1);
\path[fill=white] (26,11) rectangle ++(1,1);
\path[fill=black] (27,11) rectangle ++(1,1);
\path[fill=black] (28,11) rectangle ++(1,1);
\path[fill=white] (29,11) rectangle ++(1,1);
\path[fill=black] (30,11) rectangle ++(1,1);
\path[fill=black] (31,11) rectangle ++(1,1);
\path[fill=black] (32,11) rectangle ++(1,1);
\path[fill=black] (33,11) rectangle ++(1,1);
\path[fill=black] (34,11) rectangle ++(1,1);
\path[fill=white] (0,10) rectangle ++(1,1);
\path[fill=white] (1,10) rectangle ++(1,1);
\path[fill=white] (2,10) rectangle ++(1,1);
\path[fill=white] (3,10) rectangle ++(1,1);
\path[fill=white] (4,10) rectangle ++(1,1);
\path[fill=white] (5,10) rectangle ++(1,1);
\path[fill=white] (6,10) rectangle ++(1,1);
\path[fill=white] (7,10) rectangle ++(1,1);
\path[fill=white] (8,10) rectangle ++(1,1);
\path[fill=white] (9,10) rectangle ++(1,1);
\path[fill=white] (10,10) rectangle ++(1,1);
\path[fill=white] (11,10) rectangle ++(1,1);
\path[fill=white] (12,10) rectangle ++(1,1);
\path[fill=white] (13,10) rectangle ++(1,1);
\path[fill=white] (14,10) rectangle ++(1,1);
\path[fill=white] (15,10) rectangle ++(1,1);
\path[fill=white] (16,10) rectangle ++(1,1);
\path[fill=white] (17,10) rectangle ++(1,1);
\path[fill=white] (18,10) rectangle ++(1,1);
\path[fill=white] (19,10) rectangle ++(1,1);
\path[fill=white] (20,10) rectangle ++(1,1);
\path[fill=white] (21,10) rectangle ++(1,1);
\path[fill=white] (22,10) rectangle ++(1,1);
\path[fill=white] (23,10) rectangle ++(1,1);
\path[fill=white] (24,10) rectangle ++(1,1);
\path[fill=white] (25,10) rectangle ++(1,1);
\path[fill=white] (26,10) rectangle ++(1,1);
\path[fill=white] (27,10) rectangle ++(1,1);
\path[fill=black] (28,10) rectangle ++(1,1);
\path[fill=white] (29,10) rectangle ++(1,1);
\path[fill=white] (30,10) rectangle ++(1,1);
\path[fill=black] (31,10) rectangle ++(1,1);
\path[fill=white] (32,10) rectangle ++(1,1);
\path[fill=black] (33,10) rectangle ++(1,1);
\path[fill=black] (34,10) rectangle ++(1,1);
\path[fill=white] (0,9) rectangle ++(1,1);
\path[fill=white] (1,9) rectangle ++(1,1);
\path[fill=white] (2,9) rectangle ++(1,1);
\path[fill=white] (3,9) rectangle ++(1,1);
\path[fill=white] (4,9) rectangle ++(1,1);
\path[fill=white] (5,9) rectangle ++(1,1);
\path[fill=white] (6,9) rectangle ++(1,1);
\path[fill=white] (7,9) rectangle ++(1,1);
\path[fill=white] (8,9) rectangle ++(1,1);
\path[fill=white] (9,9) rectangle ++(1,1);
\path[fill=white] (10,9) rectangle ++(1,1);
\path[fill=white] (11,9) rectangle ++(1,1);
\path[fill=white] (12,9) rectangle ++(1,1);
\path[fill=white] (13,9) rectangle ++(1,1);
\path[fill=white] (14,9) rectangle ++(1,1);
\path[fill=white] (15,9) rectangle ++(1,1);
\path[fill=white] (16,9) rectangle ++(1,1);
\path[fill=white] (17,9) rectangle ++(1,1);
\path[fill=white] (18,9) rectangle ++(1,1);
\path[fill=white] (19,9) rectangle ++(1,1);
\path[fill=white] (20,9) rectangle ++(1,1);
\path[fill=white] (21,9) rectangle ++(1,1);
\path[fill=white] (22,9) rectangle ++(1,1);
\path[fill=white] (23,9) rectangle ++(1,1);
\path[fill=white] (24,9) rectangle ++(1,1);
\path[fill=white] (25,9) rectangle ++(1,1);
\path[fill=white] (26,9) rectangle ++(1,1);
\path[fill=white] (27,9) rectangle ++(1,1);
\path[fill=white] (28,9) rectangle ++(1,1);
\path[fill=black] (29,9) rectangle ++(1,1);
\path[fill=white] (30,9) rectangle ++(1,1);
\path[fill=white] (31,9) rectangle ++(1,1);
\path[fill=black] (32,9) rectangle ++(1,1);
\path[fill=white] (33,9) rectangle ++(1,1);
\path[fill=black] (34,9) rectangle ++(1,1);
\path[fill=white] (0,8) rectangle ++(1,1);
\path[fill=white] (1,8) rectangle ++(1,1);
\path[fill=white] (2,8) rectangle ++(1,1);
\path[fill=white] (3,8) rectangle ++(1,1);
\path[fill=white] (4,8) rectangle ++(1,1);
\path[fill=white] (5,8) rectangle ++(1,1);
\path[fill=white] (6,8) rectangle ++(1,1);
\path[fill=white] (7,8) rectangle ++(1,1);
\path[fill=white] (8,8) rectangle ++(1,1);
\path[fill=white] (9,8) rectangle ++(1,1);
\path[fill=white] (10,8) rectangle ++(1,1);
\path[fill=white] (11,8) rectangle ++(1,1);
\path[fill=white] (12,8) rectangle ++(1,1);
\path[fill=white] (13,8) rectangle ++(1,1);
\path[fill=white] (14,8) rectangle ++(1,1);
\path[fill=white] (15,8) rectangle ++(1,1);
\path[fill=white] (16,8) rectangle ++(1,1);
\path[fill=white] (17,8) rectangle ++(1,1);
\path[fill=white] (18,8) rectangle ++(1,1);
\path[fill=white] (19,8) rectangle ++(1,1);
\path[fill=white] (20,8) rectangle ++(1,1);
\path[fill=white] (21,8) rectangle ++(1,1);
\path[fill=white] (22,8) rectangle ++(1,1);
\path[fill=white] (23,8) rectangle ++(1,1);
\path[fill=white] (24,8) rectangle ++(1,1);
\path[fill=white] (25,8) rectangle ++(1,1);
\path[fill=white] (26,8) rectangle ++(1,1);
\path[fill=white] (27,8) rectangle ++(1,1);
\path[fill=white] (28,8) rectangle ++(1,1);
\path[fill=black] (29,8) rectangle ++(1,1);
\path[fill=black] (30,8) rectangle ++(1,1);
\path[fill=white] (31,8) rectangle ++(1,1);
\path[fill=black] (32,8) rectangle ++(1,1);
\path[fill=black] (33,8) rectangle ++(1,1);
\path[fill=black] (34,8) rectangle ++(1,1);
\path[fill=white] (0,7) rectangle ++(1,1);
\path[fill=white] (1,7) rectangle ++(1,1);
\path[fill=white] (2,7) rectangle ++(1,1);
\path[fill=white] (3,7) rectangle ++(1,1);
\path[fill=white] (4,7) rectangle ++(1,1);
\path[fill=white] (5,7) rectangle ++(1,1);
\path[fill=white] (6,7) rectangle ++(1,1);
\path[fill=white] (7,7) rectangle ++(1,1);
\path[fill=white] (8,7) rectangle ++(1,1);
\path[fill=white] (9,7) rectangle ++(1,1);
\path[fill=white] (10,7) rectangle ++(1,1);
\path[fill=white] (11,7) rectangle ++(1,1);
\path[fill=white] (12,7) rectangle ++(1,1);
\path[fill=white] (13,7) rectangle ++(1,1);
\path[fill=white] (14,7) rectangle ++(1,1);
\path[fill=white] (15,7) rectangle ++(1,1);
\path[fill=white] (16,7) rectangle ++(1,1);
\path[fill=white] (17,7) rectangle ++(1,1);
\path[fill=white] (18,7) rectangle ++(1,1);
\path[fill=white] (19,7) rectangle ++(1,1);
\path[fill=white] (20,7) rectangle ++(1,1);
\path[fill=white] (21,7) rectangle ++(1,1);
\path[fill=white] (22,7) rectangle ++(1,1);
\path[fill=white] (23,7) rectangle ++(1,1);
\path[fill=white] (24,7) rectangle ++(1,1);
\path[fill=white] (25,7) rectangle ++(1,1);
\path[fill=white] (26,7) rectangle ++(1,1);
\path[fill=white] (27,7) rectangle ++(1,1);
\path[fill=white] (28,7) rectangle ++(1,1);
\path[fill=white] (29,7) rectangle ++(1,1);
\path[fill=black] (30,7) rectangle ++(1,1);
\path[fill=black] (31,7) rectangle ++(1,1);
\path[fill=black] (32,7) rectangle ++(1,1);
\path[fill=black] (33,7) rectangle ++(1,1);
\path[fill=black] (34,7) rectangle ++(1,1);
\path[fill=white] (0,6) rectangle ++(1,1);
\path[fill=white] (1,6) rectangle ++(1,1);
\path[fill=white] (2,6) rectangle ++(1,1);
\path[fill=white] (3,6) rectangle ++(1,1);
\path[fill=white] (4,6) rectangle ++(1,1);
\path[fill=white] (5,6) rectangle ++(1,1);
\path[fill=white] (6,6) rectangle ++(1,1);
\path[fill=white] (7,6) rectangle ++(1,1);
\path[fill=white] (8,6) rectangle ++(1,1);
\path[fill=white] (9,6) rectangle ++(1,1);
\path[fill=white] (10,6) rectangle ++(1,1);
\path[fill=white] (11,6) rectangle ++(1,1);
\path[fill=white] (12,6) rectangle ++(1,1);
\path[fill=white] (13,6) rectangle ++(1,1);
\path[fill=white] (14,6) rectangle ++(1,1);
\path[fill=white] (15,6) rectangle ++(1,1);
\path[fill=white] (16,6) rectangle ++(1,1);
\path[fill=white] (17,6) rectangle ++(1,1);
\path[fill=white] (18,6) rectangle ++(1,1);
\path[fill=white] (19,6) rectangle ++(1,1);
\path[fill=white] (20,6) rectangle ++(1,1);
\path[fill=white] (21,6) rectangle ++(1,1);
\path[fill=white] (22,6) rectangle ++(1,1);
\path[fill=white] (23,6) rectangle ++(1,1);
\path[fill=white] (24,6) rectangle ++(1,1);
\path[fill=white] (25,6) rectangle ++(1,1);
\path[fill=white] (26,6) rectangle ++(1,1);
\path[fill=white] (27,6) rectangle ++(1,1);
\path[fill=white] (28,6) rectangle ++(1,1);
\path[fill=white] (29,6) rectangle ++(1,1);
\path[fill=white] (30,6) rectangle ++(1,1);
\path[fill=black] (31,6) rectangle ++(1,1);
\path[fill=white] (32,6) rectangle ++(1,1);
\path[fill=black] (33,6) rectangle ++(1,1);
\path[fill=black] (34,6) rectangle ++(1,1);
\path[fill=white] (0,5) rectangle ++(1,1);
\path[fill=white] (1,5) rectangle ++(1,1);
\path[fill=white] (2,5) rectangle ++(1,1);
\path[fill=white] (3,5) rectangle ++(1,1);
\path[fill=white] (4,5) rectangle ++(1,1);
\path[fill=white] (5,5) rectangle ++(1,1);
\path[fill=white] (6,5) rectangle ++(1,1);
\path[fill=white] (7,5) rectangle ++(1,1);
\path[fill=white] (8,5) rectangle ++(1,1);
\path[fill=white] (9,5) rectangle ++(1,1);
\path[fill=white] (10,5) rectangle ++(1,1);
\path[fill=white] (11,5) rectangle ++(1,1);
\path[fill=white] (12,5) rectangle ++(1,1);
\path[fill=white] (13,5) rectangle ++(1,1);
\path[fill=white] (14,5) rectangle ++(1,1);
\path[fill=white] (15,5) rectangle ++(1,1);
\path[fill=white] (16,5) rectangle ++(1,1);
\path[fill=white] (17,5) rectangle ++(1,1);
\path[fill=white] (18,5) rectangle ++(1,1);
\path[fill=white] (19,5) rectangle ++(1,1);
\path[fill=white] (20,5) rectangle ++(1,1);
\path[fill=white] (21,5) rectangle ++(1,1);
\path[fill=white] (22,5) rectangle ++(1,1);
\path[fill=white] (23,5) rectangle ++(1,1);
\path[fill=white] (24,5) rectangle ++(1,1);
\path[fill=white] (25,5) rectangle ++(1,1);
\path[fill=white] (26,5) rectangle ++(1,1);
\path[fill=white] (27,5) rectangle ++(1,1);
\path[fill=white] (28,5) rectangle ++(1,1);
\path[fill=white] (29,5) rectangle ++(1,1);
\path[fill=white] (30,5) rectangle ++(1,1);
\path[fill=white] (31,5) rectangle ++(1,1);
\path[fill=black] (32,5) rectangle ++(1,1);
\path[fill=white] (33,5) rectangle ++(1,1);
\path[fill=black] (34,5) rectangle ++(1,1);
\path[fill=white] (0,4) rectangle ++(1,1);
\path[fill=white] (1,4) rectangle ++(1,1);
\path[fill=white] (2,4) rectangle ++(1,1);
\path[fill=white] (3,4) rectangle ++(1,1);
\path[fill=white] (4,4) rectangle ++(1,1);
\path[fill=white] (5,4) rectangle ++(1,1);
\path[fill=white] (6,4) rectangle ++(1,1);
\path[fill=white] (7,4) rectangle ++(1,1);
\path[fill=white] (8,4) rectangle ++(1,1);
\path[fill=white] (9,4) rectangle ++(1,1);
\path[fill=white] (10,4) rectangle ++(1,1);
\path[fill=white] (11,4) rectangle ++(1,1);
\path[fill=white] (12,4) rectangle ++(1,1);
\path[fill=white] (13,4) rectangle ++(1,1);
\path[fill=white] (14,4) rectangle ++(1,1);
\path[fill=white] (15,4) rectangle ++(1,1);
\path[fill=white] (16,4) rectangle ++(1,1);
\path[fill=white] (17,4) rectangle ++(1,1);
\path[fill=white] (18,4) rectangle ++(1,1);
\path[fill=white] (19,4) rectangle ++(1,1);
\path[fill=white] (20,4) rectangle ++(1,1);
\path[fill=white] (21,4) rectangle ++(1,1);
\path[fill=white] (22,4) rectangle ++(1,1);
\path[fill=white] (23,4) rectangle ++(1,1);
\path[fill=white] (24,4) rectangle ++(1,1);
\path[fill=white] (25,4) rectangle ++(1,1);
\path[fill=white] (26,4) rectangle ++(1,1);
\path[fill=white] (27,4) rectangle ++(1,1);
\path[fill=white] (28,4) rectangle ++(1,1);
\path[fill=white] (29,4) rectangle ++(1,1);
\path[fill=white] (30,4) rectangle ++(1,1);
\path[fill=white] (31,4) rectangle ++(1,1);
\path[fill=black] (32,4) rectangle ++(1,1);
\path[fill=black] (33,4) rectangle ++(1,1);
\path[fill=black] (34,4) rectangle ++(1,1);
\path[fill=white] (0,3) rectangle ++(1,1);
\path[fill=white] (1,3) rectangle ++(1,1);
\path[fill=white] (2,3) rectangle ++(1,1);
\path[fill=white] (3,3) rectangle ++(1,1);
\path[fill=white] (4,3) rectangle ++(1,1);
\path[fill=white] (5,3) rectangle ++(1,1);
\path[fill=white] (6,3) rectangle ++(1,1);
\path[fill=white] (7,3) rectangle ++(1,1);
\path[fill=white] (8,3) rectangle ++(1,1);
\path[fill=white] (9,3) rectangle ++(1,1);
\path[fill=white] (10,3) rectangle ++(1,1);
\path[fill=white] (11,3) rectangle ++(1,1);
\path[fill=white] (12,3) rectangle ++(1,1);
\path[fill=white] (13,3) rectangle ++(1,1);
\path[fill=white] (14,3) rectangle ++(1,1);
\path[fill=white] (15,3) rectangle ++(1,1);
\path[fill=white] (16,3) rectangle ++(1,1);
\path[fill=white] (17,3) rectangle ++(1,1);
\path[fill=white] (18,3) rectangle ++(1,1);
\path[fill=white] (19,3) rectangle ++(1,1);
\path[fill=white] (20,3) rectangle ++(1,1);
\path[fill=white] (21,3) rectangle ++(1,1);
\path[fill=white] (22,3) rectangle ++(1,1);
\path[fill=white] (23,3) rectangle ++(1,1);
\path[fill=white] (24,3) rectangle ++(1,1);
\path[fill=white] (25,3) rectangle ++(1,1);
\path[fill=white] (26,3) rectangle ++(1,1);
\path[fill=white] (27,3) rectangle ++(1,1);
\path[fill=white] (28,3) rectangle ++(1,1);
\path[fill=white] (29,3) rectangle ++(1,1);
\path[fill=white] (30,3) rectangle ++(1,1);
\path[fill=white] (31,3) rectangle ++(1,1);
\path[fill=white] (32,3) rectangle ++(1,1);
\path[fill=black] (33,3) rectangle ++(1,1);
\path[fill=black] (34,3) rectangle ++(1,1);
\path[fill=white] (0,2) rectangle ++(1,1);
\path[fill=white] (1,2) rectangle ++(1,1);
\path[fill=white] (2,2) rectangle ++(1,1);
\path[fill=white] (3,2) rectangle ++(1,1);
\path[fill=white] (4,2) rectangle ++(1,1);
\path[fill=white] (5,2) rectangle ++(1,1);
\path[fill=white] (6,2) rectangle ++(1,1);
\path[fill=white] (7,2) rectangle ++(1,1);
\path[fill=white] (8,2) rectangle ++(1,1);
\path[fill=white] (9,2) rectangle ++(1,1);
\path[fill=white] (10,2) rectangle ++(1,1);
\path[fill=white] (11,2) rectangle ++(1,1);
\path[fill=white] (12,2) rectangle ++(1,1);
\path[fill=white] (13,2) rectangle ++(1,1);
\path[fill=white] (14,2) rectangle ++(1,1);
\path[fill=white] (15,2) rectangle ++(1,1);
\path[fill=white] (16,2) rectangle ++(1,1);
\path[fill=white] (17,2) rectangle ++(1,1);
\path[fill=white] (18,2) rectangle ++(1,1);
\path[fill=white] (19,2) rectangle ++(1,1);
\path[fill=white] (20,2) rectangle ++(1,1);
\path[fill=white] (21,2) rectangle ++(1,1);
\path[fill=white] (22,2) rectangle ++(1,1);
\path[fill=white] (23,2) rectangle ++(1,1);
\path[fill=white] (24,2) rectangle ++(1,1);
\path[fill=white] (25,2) rectangle ++(1,1);
\path[fill=white] (26,2) rectangle ++(1,1);
\path[fill=white] (27,2) rectangle ++(1,1);
\path[fill=white] (28,2) rectangle ++(1,1);
\path[fill=white] (29,2) rectangle ++(1,1);
\path[fill=white] (30,2) rectangle ++(1,1);
\path[fill=white] (31,2) rectangle ++(1,1);
\path[fill=white] (32,2) rectangle ++(1,1);
\path[fill=white] (33,2) rectangle ++(1,1);
\path[fill=black] (34,2) rectangle ++(1,1);
\path[fill=white] (0,1) rectangle ++(1,1);
\path[fill=white] (1,1) rectangle ++(1,1);
\path[fill=white] (2,1) rectangle ++(1,1);
\path[fill=white] (3,1) rectangle ++(1,1);
\path[fill=white] (4,1) rectangle ++(1,1);
\path[fill=white] (5,1) rectangle ++(1,1);
\path[fill=white] (6,1) rectangle ++(1,1);
\path[fill=white] (7,1) rectangle ++(1,1);
\path[fill=white] (8,1) rectangle ++(1,1);
\path[fill=white] (9,1) rectangle ++(1,1);
\path[fill=white] (10,1) rectangle ++(1,1);
\path[fill=white] (11,1) rectangle ++(1,1);
\path[fill=white] (12,1) rectangle ++(1,1);
\path[fill=white] (13,1) rectangle ++(1,1);
\path[fill=white] (14,1) rectangle ++(1,1);
\path[fill=white] (15,1) rectangle ++(1,1);
\path[fill=white] (16,1) rectangle ++(1,1);
\path[fill=white] (17,1) rectangle ++(1,1);
\path[fill=white] (18,1) rectangle ++(1,1);
\path[fill=white] (19,1) rectangle ++(1,1);
\path[fill=white] (20,1) rectangle ++(1,1);
\path[fill=white] (21,1) rectangle ++(1,1);
\path[fill=white] (22,1) rectangle ++(1,1);
\path[fill=white] (23,1) rectangle ++(1,1);
\path[fill=white] (24,1) rectangle ++(1,1);
\path[fill=white] (25,1) rectangle ++(1,1);
\path[fill=white] (26,1) rectangle ++(1,1);
\path[fill=white] (27,1) rectangle ++(1,1);
\path[fill=white] (28,1) rectangle ++(1,1);
\path[fill=white] (29,1) rectangle ++(1,1);
\path[fill=white] (30,1) rectangle ++(1,1);
\path[fill=white] (31,1) rectangle ++(1,1);
\path[fill=white] (32,1) rectangle ++(1,1);
\path[fill=white] (33,1) rectangle ++(1,1);
\path[fill=black] (34,1) rectangle ++(1,1);
\path[fill=white] (0,0) rectangle ++(1,1);
\path[fill=white] (1,0) rectangle ++(1,1);
\path[fill=white] (2,0) rectangle ++(1,1);
\path[fill=white] (3,0) rectangle ++(1,1);
\path[fill=white] (4,0) rectangle ++(1,1);
\path[fill=white] (5,0) rectangle ++(1,1);
\path[fill=white] (6,0) rectangle ++(1,1);
\path[fill=white] (7,0) rectangle ++(1,1);
\path[fill=white] (8,0) rectangle ++(1,1);
\path[fill=white] (9,0) rectangle ++(1,1);
\path[fill=white] (10,0) rectangle ++(1,1);
\path[fill=white] (11,0) rectangle ++(1,1);
\path[fill=white] (12,0) rectangle ++(1,1);
\path[fill=white] (13,0) rectangle ++(1,1);
\path[fill=white] (14,0) rectangle ++(1,1);
\path[fill=white] (15,0) rectangle ++(1,1);
\path[fill=white] (16,0) rectangle ++(1,1);
\path[fill=white] (17,0) rectangle ++(1,1);
\path[fill=white] (18,0) rectangle ++(1,1);
\path[fill=white] (19,0) rectangle ++(1,1);
\path[fill=white] (20,0) rectangle ++(1,1);
\path[fill=white] (21,0) rectangle ++(1,1);
\path[fill=white] (22,0) rectangle ++(1,1);
\path[fill=white] (23,0) rectangle ++(1,1);
\path[fill=white] (24,0) rectangle ++(1,1);
\path[fill=white] (25,0) rectangle ++(1,1);
\path[fill=white] (26,0) rectangle ++(1,1);
\path[fill=white] (27,0) rectangle ++(1,1);
\path[fill=white] (28,0) rectangle ++(1,1);
\path[fill=white] (29,0) rectangle ++(1,1);
\path[fill=white] (30,0) rectangle ++(1,1);
\path[fill=white] (31,0) rectangle ++(1,1);
\path[fill=white] (32,0) rectangle ++(1,1);
\path[fill=white] (33,0) rectangle ++(1,1);
\path[fill=white] (34,0) rectangle ++(1,1);
\end{tikzpicture}
    }
  \end{center}
  \caption{Lev(|\sigma|=6, \Delta=4) adjacency and reachability matrices.}
\end{figure}


\section{Levenshtein Automata Minimality}

It is reasonable to ask whether the Levenshtein automaton defined is minimal, in the sense of whether there exists an automaton with fewer states than $A$ yet still generates $\mathcal{L}(G_\cap)$ when intersected with $\mathcal{L}(G)$. In other words, given $G$ and $\err\sigma$, is there an $A'$ such that $|Q_{A'}| < |Q_{A}|$ yet $\mathcal{L}(G) \cap \mathcal{L}(A') = \mathcal{L}(G) \cap \mathcal{L}(A)$ still holds? In fact, there is a trivial example:

\begin{theorem}
  Let $Q_{A'}$ be defined as $Q_A \setminus \{q_{n, 0}\}$.
\end{theorem}

Since $q_{n, 0}$ accepts the original string $\err\sigma: \bar\ell \cap \Sigma^n$ which is by definition outside $\mathcal{L}(G)$, we can immediately rule out this state. Moreover, we can define a family of automata with strictly fewer states than the full LBH construction by making the following observation: if we can prove one edit must occur before the last $s$ tokens, we can rule out the last $s$ states absorbing editless trajectories.

\begin{theorem}
  $\varnothing = \mathcal{L}(\err\sigma_{1 \ldots (n-s)}\cdot\Sigma^s)\cap \mathcal{L}(G)$ implies the states $[q_{n-i, 0}]_{i \in 1\ldots s}$ are unnecessary.
\end{theorem}

Likewise, if we expend our entire edit budget in the first $p$ tokens, we will be unable to recover in a string where at least one repair must occur after the first $p$ tokens.

\begin{theorem}
  $\varnothing = \mathcal{L}(\Sigma^p\cdot\err\sigma_{p\ldots n})\cap \mathcal{L}(G)$ implies the states $[q_{i, d_{\max}}]_{i \in 0\ldots p}$ are unnecessary.
\end{theorem}

Therefor, we can eliminate $p+s$ states from $A$ by proving emptiness of $\mathcal{L}(\Sigma^p\cdot\err\sigma_{p\ldots (n-s)}\cdot\Sigma^s) \cap \mathcal{L}(G)$, without affecting $\mathcal{L}(G_\cap)$. Pictorially,

\begin{figure}[H]
  \resizebox{0.47\textwidth}{!}{
    \input{figures/original_nfa}
  }
  \resizebox{0.47\textwidth}{!}{
    \begin{tikzpicture}[
%->, % makes the edges directed
>=stealth',
node distance=2.5cm, % specifies the minimum distance between two nodes. Change if necessary.
%  every state/.style={thick, fill=gray!10}, % sets the properties for each ’state’ node
initial text=$ $, % sets the text that appears on the start arrow
]
\node[state, initial]                (00) {$q_{0,0}$};
\node[state, right of=00]            (10) {$q_{1,0}$};
\node[accepting, state, right of=10] (20) {$q_{2,0}$};
\phantom{\node[accepting, state, right of=20] (30) {$q_{3,0}$};}
\phantom{\node[accepting, state, right of=30] (40) {$q_{4,0}$};}
\phantom{\node[accepting, state, right of=40] (50) {$q_{5,0}$};}

\node[state, above of=00, shift={(-2cm,0cm)}] (01) {$q_{0,1}$};
\node[state, right of=01]                          (11) {$q_{1,1}$};
\node[state, right of=11]                          (21) {$q_{2,1}$};
\node[accepting, state, right of=21]               (31) {$q_{3,1}$};
\node[accepting, state, right of=31]               (41) {$q_{4,1}$};
\node[accepting, state, right of=41]               (51) {$q_{5,1}$};

\node[state, above of=01, shift={(-2cm,0cm)}] (0j) {$q_{0,2}$};
\node[state, right of=0j]                          (1j) {$q_{1,2}$};
\node[state, right of=1j]                          (2j) {$q_{2,2}$};
\node[state, right of=2j]                          (3j) {$q_{3,2}$};
\node[accepting, state, right of=3j]               (4j) {$q_{4,2}$};
\node[accepting, state, right of=4j]               (5j) {$q_{5,2}$};

\phantom{\node[state, above of=0j, shift={(-2cm,0cm)}] (0k) {$q_{0,3}$};}
\phantom{\node[state, right of=0k]                         (1k) {$q_{1,3}$};}
\phantom{\node[state, right of=1k]                         (2k) {$q_{2,3}$};}
\node[state, right of=2k]                         (3k) {$q_{3,3}$};
\node[state, right of=3k]                         (4k) {$q_{4,3}$};
\node[accepting, state, right of=4k]              (5k) {$q_{5,3}$};

\draw [->] (00) edge[below] node{$\sigma_1$} (10);
\draw [->] (10) edge[below] node{$\sigma_2$} (20);
%        \draw [->] (20) edge[below] node{$\sigma_3$} (30);
%        \draw [->] (30) edge[below] node{$\sigma_4$} (40);
%        \draw [->] (40) edge[below] node{$\sigma_5$} (50);

\draw [->] (01) edge[below] node{$\sigma_1$} (11);
\draw [->] (11) edge[below] node[shift={(-0.2cm,0cm)}]{$\sigma_2$} (21);
\draw [->] (21) edge[below] node[shift={(-0.2cm,0cm)}]{$\sigma_3$} (31);
\draw [->] (31) edge[below] node[shift={(-0.2cm,0cm)}]{$\sigma_4$} (41);
\draw [->] (41) edge[below] node{$\sigma_5$} (51);

\draw [->] (0j) edge[below] node{$\sigma_1$} (1j);
\draw [->] (1j) edge[below] node{$\sigma_2$} (2j);
\draw [->] (2j) edge[below] node{$\sigma_3$} (3j);
\draw [->] (3j) edge[below] node{$\sigma_4$} (4j);
\draw [->] (4j) edge[below] node{$\sigma_5$} (5j);

%        \draw [->] (0k) edge[below] node{$\sigma_1$} (1k);
%        \draw [->] (1k) edge[below] node{$\sigma_2$} (2k);
%        \draw [->] (2k) edge[below] node{$\sigma_3$} (3k);
\draw [->] (3k) edge[below] node{$\sigma_4$} (4k);
\draw [->] (4k) edge[below] node{$\sigma_5$} (5k);

\draw [->] (00) edge[left] node{$\phantom{\cdot}$} (11);
\draw [->] (10) edge[left] node{$\phantom{\cdot}$} (21);
\draw [->] (20) edge[left] node{$\phantom{\cdot}$} (31);
%        \draw [->] (30) edge[left] node{$\phantom{\cdot}$} (41);
%        \draw [->] (40) edge[left] node{$\phantom{\cdot}$} (51);

% Super-knight arcs
\draw [->, red] (00) edge[bend right=8] node[east, shift={(-0.2cm,-0.7cm)}]{$\color{red}\sigma_3$}         (3j);
\draw [->, red] (10) edge[bend right=8] node[east, shift={(-0.2cm,-0.7cm)}]{$\color{red}\sigma_4$}         (4j);
\draw [->, red] (20) edge[bend right=8] node[east, shift={(-0.2cm,-0.7cm)}]{$\color{red}\sigma_5$}         (5j);

\draw [->, red] (01) edge[bend left=8] node[east, shift={(-0.2cm,-0.7cm)}]{$\color{red}\sigma_3$}         (3k);
\draw [->, red] (11) edge[bend left=8] node[east, shift={(-0.2cm,-0.7cm)}]{$\color{red}\sigma_4$}         (4k);
\draw [->, red] (21) edge[bend left=8] node[east, shift={(-0.2cm,-0.7cm)}]{$\color{red}\sigma_5$}         (5k);

\draw [->, violet] (00) edge node[east, shift={(-0.1cm,-0.8cm)}]{$\color{violet}\sigma_4$}  (4k);
\draw [->, violet] (10) edge node[east, shift={(-0.1cm,-0.8cm)}]{$\color{violet}\sigma_5$}  (5k);

\draw [->] (01) edge[left] node{$\phantom{\cdot}$} (1j);
\draw [->] (11) edge[left] node{$\phantom{\cdot}$} (2j);
\draw [->] (21) edge[left] node{$\phantom{\cdot}$} (3j);
\draw [->] (31) edge[left] node{$\phantom{\cdot}$} (4j);
\draw [->] (41) edge[left] node{$\phantom{\cdot}$} (5j);

%        \draw [->] (0j) edge[left] node{$\phantom{\cdot}$} (1k);
%        \draw [->] (1j) edge[left] node{$\phantom{\cdot}$} (2k);
\draw [->] (2j) edge[left] node{$\phantom{\cdot}$} (3k);
\draw [->] (3j) edge[left] node{$\phantom{\cdot}$} (4k);
\draw [->] (4j) edge[left] node{$\phantom{\cdot}$} (5k);

\draw [->] (00) edge[bend left=10, left] node{$\phantom{\cdot}$} (01);
\draw [->] (10) edge[bend left=10, left] node{$\phantom{\cdot}$} (11);
\draw [->] (20) edge[bend left=10, left] node{$\phantom{\cdot}$} (21);
%        \draw [->] (30) edge[bend left=10, left] node{$\phantom{\cdot}$} (31);
%        \draw [->] (40) edge[bend left=10, left] node{$\phantom{\cdot}$} (41);
%        \draw [->] (50) edge[bend left=10, left] node{$\phantom{\cdot}$} (51);

\draw [->] (01) edge[bend left=10, left] node{$\phantom{\cdot}$} (0j);
\draw [->] (11) edge[bend left=10, left] node{$\phantom{\cdot}$} (1j);
\draw [->] (21) edge[bend left=10, left] node{$\phantom{\cdot}$} (2j);
\draw [->] (31) edge[bend left=10, left] node{$\phantom{\cdot}$} (3j);
\draw [->] (41) edge[bend left=10, left] node{$\phantom{\cdot}$} (4j);
\draw [->] (51) edge[bend left=10, left] node{$\phantom{\cdot}$} (5j);

%        \draw [->] (0j) edge[bend left=10, left] node{$\phantom{\cdot}$} (0k);
%        \draw [->] (1j) edge[bend left=10, left] node{$\phantom{\cdot}$} (1k);
%        \draw [->] (2j) edge[bend left=10, left] node{$\phantom{\cdot}$} (2k);
\draw [->] (3j) edge[bend left=10, left] node{$\phantom{\cdot}$} (3k);
\draw [->] (4j) edge[bend left=10, left] node{$\phantom{\cdot}$} (4k);
\draw [->] (5j) edge[bend left=10, left] node{$\phantom{\cdot}$} (5k);

\draw [->, blue] (00) edge[bend right=11,below] node[shift={(0.5cm,0.3cm)}]{$\color{blue}\sigma_2$}    (21);
\draw [->, blue] (10) edge[bend right=11,below] node[shift={(0.5cm,0.3cm)}]{$\color{blue}\sigma_3$}    (31);
\draw [->, blue] (20) edge[bend right=11,below] node[shift={(0.5cm,0.3cm)}]{$\color{blue}\sigma_4$}    (41);
%        \draw [->, blue] (30) edge[bend right=11,below] node[shift={(0.5cm,0.3cm)}]{$\color{blue}\sigma_5$}    (51);

\draw [->, blue] (01) edge[bend right=3,below] node[shift={(0.3cm,0.2cm)}]{$\color{blue}\sigma_2$}    (2j);
\draw [->, blue] (11) edge[bend right=3,below] node[shift={(0.3cm,0.2cm)}]{$\color{blue}\sigma_3$}    (3j);
\draw [->, blue] (21) edge[bend right=3,below] node[shift={(0.3cm,0.2cm)}]{$\color{blue}\sigma_4$}    (4j);
\draw [->, blue] (31) edge[bend right=3,below] node[shift={(0.3cm,0.2cm)}]{$\color{blue}\sigma_4$}    (5j);

%        \draw [->, blue] (0j) edge[bend left=8,below] node[shift={(-0.45cm,-0.55cm)}]{$\color{blue}\sigma_2$}    (2k);
\draw [->, blue] (1j) edge[bend left=8,below] node[shift={(-0.45cm,-0.55cm)}]{$\color{blue}\sigma_3$}    (3k);
\draw [->, blue] (2j) edge[bend left=8,below] node[shift={(-0.45cm,-0.55cm)}]{$\color{blue}\sigma_4$}    (4k);
\draw [->, blue] (3j) edge[bend left=8,below] node[shift={(-0.45cm,-0.55cm)}]{$\color{blue}\sigma_5$}    (5k);

%https://tex.stackexchange.com/a/20986/139648
%        \draw [decorate,decoration={brace,amplitude=10pt,raise=10pt,mirror}] (00.south west) -- (50.south east) node[midway,yshift=-3em]{\textbf{String length}};
%        \draw [decorate,decoration={brace,amplitude=10pt,raise=20pt}] (00.south west) -- (0k.north west) node[midway,xshift=-1cm,yshift=-1cm,rotate=-54]{\textbf{Edit distance}};
\end{tikzpicture}
  }
  \caption{Levenshtein NFA before and after pruning.}
\end{figure}\vspace{-0.175cm}
Pruned L-NFA for the broken string $\err\sigma = \texttt{[ ( + ) ]}$ with $G = \{S \rightarrow ( S ) \mid [ S ] \mid S + S \mid 1\}$.

\noindent\phantom{$\sigma$: }\texttt{\_ \_ + ) ]}\phantom{...}\emoji{cross-mark}\phantom{...} $\land$ \phantom{...}\texttt{\_ \_ \_ ) ]}\phantom{...}\emoji{check-mark-button}\\
\phantom{$\sigma$: }\texttt{[ ( + \_ \_}\phantom{...}\emoji{cross-mark}\phantom{...} $\land$ \phantom{...}\texttt{[ ( \_ \_ \_}\phantom{...}\emoji{check-mark-button}

\clearpage\section{Example Repairs}\label{sec:exaple_repairs}

Below, we provide a few representative examples of broken code snippets and the corresponding human repairs that were successfully ranked first by our method. On the left is a complete snippet fed to the model and on the right, the corresponding human repair that was correctly predicted.

\begin{figure}[H]
\begin{tabular}{|m{6.6cm}|m{6.6cm}|}
\hline \rule{0pt}{2.5ex}\textbf{Original broken code}\rule[-1ex]{0pt}{2ex} &  \rule{0pt}{2.5ex}\textbf{First predicted repair}\rule[-1ex]{0pt}{2ex} \\\hline
\begin{smallpy}

(*@\hlorange{form}@*) sympy import *
x = Symbol('x', real=True)
x, re(x), im(x)

\end{smallpy} & \begin{smallpy}

(*@\hlorange{\textbf{from}}@*) sympy import *
x = Symbol('x', real=True)
x, re(x), im(x)

\end{smallpy} \\\hline
\begin{smallpy}

result = (*@\hlorange{yeald}@*) From(item.create())
raise Return(result)

\end{smallpy} & \begin{smallpy}

result = (*@\hlorange{\textbf{yield}}@*) From(item.create())
raise Return(result)

\end{smallpy} \\\hline
\begin{smallpy}

df.apply(lambda row: list(set(row['ids'(*@\hlorange{)}@*))))

\end{smallpy} & \begin{smallpy}

df.apply(lambda row: list(set(row['ids'(*@\hlorange{]}@*))))

\end{smallpy} \\\hline
%        \begin{lstlisting}[basicstyle=\ttfamily\lst@ifdisplaystyle\footnotesize\fi, language=python]
%
%  import numpy (*@\hlorange{ad}@*) np
%  A_concate = np.array([a_0, a_1, a_2,..., a_n])
%
%        \end{lstlisting} & \begin{lstlisting}[basicstyle=\ttfamily\lst@ifdisplaystyle\footnotesize\fi, language=python]
%
%  import numpy (*@\hlorange{\textbf{as}}@*) np
%  A_concate = np.array([a_0, a_1, a_2,..., a_n])
%
%        \end{lstlisting} \\\hline
\begin{smallpy}

sum(len(v) for v items.values())(*@\hlred{)}@*)

\end{smallpy} & \begin{smallpy}

sum(len(v) for v (*@\hlgreen{\textbf{in}}@*) items.values())

\end{smallpy} \\\hline
\begin{smallpy}

def average(values):
  if values == (1,2,3):
    return (1+2+3)/3
  (*@\hlorange{else}@*) (*@\hlred{if}@*) values == (-3,2,8,-1):
    return (-3+2+8-1)/4

\end{smallpy} & \begin{smallpy}

def average(values):
  if values == (1,2,3):
    return (1+2+3)/3
  (*@\hlorange{elif}@*) values == (-3,2,8,-1):
    return (-3+2+8-1)/4

\end{smallpy} \\\hline
\begin{smallpy}

dict = {
  "Jan": 1
  "January": 1
  "Feb": 2 # and so on
}

\end{smallpy} & \begin{smallpy}

dict = {
  "Jan": 1(*@\hlgreen{,}@*)
  "January": 1(*@\hlgreen{,}@*)
  "Feb": 2 # and so on
}

\end{smallpy} \\\hline
\begin{smallpy}

class MixIn(object)
  def m():
    pass

class classA(MixIn):

class classB(MixIn):

\end{smallpy} & \begin{smallpy}

class MixIn(object)(*@\hlgreen{:}@*)
  def m():
    pass

class classA(MixIn): (*@\hlgreen{\textbf{pass}}@*)

class classB(MixIn): (*@\hlgreen{\textbf{pass}}@*)

\end{smallpy} \\\hline
\end{tabular}
\end{figure}

\clearpage\section{Raw data}\label{sec:raw_prec_data}

Raw data from Precision@k experiments across snippet length and Levenshtein distance from \S~\ref{sec:rq2}. $|\err\sigma|$ indicates the snippet length and $\Delta$ indicates the Levenshtein distance between the broken and code and human fix computed over lexical tokens. For Tidyparse, we sample until exhausting the admissible set or a timeout of 30s is reached, whichever happens first, then rank the results. For the other models Precision@1, we sample one repair and report the percentage of repairs matching the human repair. For Precision@All, we report the percentage of repairs matching the human repair within the top 20000 samples.

\begin{table}[!h]
\centering
\begin{tabular}{c|c|cccccccc}
\hline\hline
& $\Delta$ & \multicolumn{8}{c}{Precision@1} \\ \hline
$|\err\sigma|$ &  & $(0,10)$ & $[10,20)$ & $[20,30)$ & $[30, 40)$ & $[40,50)$ & $[50, 60)$ & $[60,70)$ & $[70, 80)$ \\ \hline
Tidyparse
& 1 & 0.56 & 0.44 & 0.43 & 0.49 & 0.55 & 0.55 & 0.53 & 0.57 \\
& 2 & 0.37 & 0.28 & 0.26 & 0.24 & 0.19 & 0.25 & 0.23 & 0.18 \\
& 3 & 0.18 & 0.20 & 0.19 & 0.15 & 0.10 & 0.09 & 0.11 & 0.11 \\ \hline
Seq2Parse
& 1 & 0.35 & 0.41 & 0.40 & 0.37 & 0.31 & 0.29 & 0.27 & 0.21 \\
& 2 & 0.12 & 0.13 & 0.14 & 0.12 & 0.11 & 0.11 & 0.10 & 0.12 \\
& 3 & 0.03 & 0.07 & 0.08 & 0.09 & 0.09 & 0.02 & 0.07 & 0.06 \\ \hline
BIFI
& 1 & 0.20 & 0.33 & 0.32 & 0.27 & 0.21 & 0.21 & 0.25 & 0.18 \\
& 2 & 0.18 & 0.18 & 0.21 & 0.19 & 0.19 & 0.18 & 0.11 & 0.11 \\
& 3 & 0.02 & 0.02 & 0.03 & 0.02 & 0.03 & 0.05 & 0.03 & 0.02 \\ \hline
& & \multicolumn{8}{c}{Precision@All} \\ \hline
Tidyparse
& 1 & 1.00 & 1.00 & 1.00 & 0.99 & 0.99 & 1.00 & 0.97 & 0.97 \\
& 2 & 1.00 & 0.99 & 0.98 & 1.00 & 1.00 & 1.00 & 0.94 & 0.90 \\
& 3 & 1.00 & 0.98 & 0.80 & 0.70 & 0.55 & 0.42 & 0.42 & 0.31 \\ \hline
BIFI
& 1 & 0.65 & 0.67 & 0.70 & 0.65 & 0.60 & 0.62 & 0.60 & 0.64 \\
& 2 & 0.52 & 0.41 & 0.37 & 0.32 & 0.27 & 0.27 & 0.21 & 0.24 \\
& 3 & 0.20 & 0.13 & 0.08 & 0.17 & 0.15 & 0.18 & 0.17 & 0.07 \\ \hline\hline
\end{tabular}
\end{table}

%Precision@1
%===========
%(|σ|∈[0, 10), Δ=1): Top-1/total: 56 / 100 = 0.56
%(|σ|∈[0, 10), Δ=2): Top-1/total: 37 / 100 = 0.37
%(|σ|∈[0, 10), Δ=3): Top-1/total: 9 / 50 = 0.18
%
%(|σ|∈[10, 20), Δ=1): Top-1/total: 44 / 100 = 0.44
%(|σ|∈[10, 20), Δ=2): Top-1/total: 28 / 100 = 0.28
%(|σ|∈[10, 20), Δ=3): Top-1/total: 20 / 100 = 0.2
%
%(|σ|∈[20, 30), Δ=1): Top-1/total: 43 / 100 = 0.43
%(|σ|∈[20, 30), Δ=2): Top-1/total: 26 / 100 = 0.26
%(|σ|∈[20, 30), Δ=3): Top-1/total: 19 / 99 = 0.1919191919191919
%
%(|σ|∈[30, 40), Δ=1): Top-1/total: 49 / 100 = 0.49
%(|σ|∈[30, 40), Δ=2): Top-1/total: 24 / 100 = 0.24
%(|σ|∈[30, 40), Δ=3): Top-1/total: 15 / 100 = 0.15
%
%(|σ|∈[40, 50), Δ=1): Top-1/total: 55 / 100 = 0.55
%(|σ|∈[40, 50), Δ=2): Top-1/total: 19 / 100 = 0.19
%(|σ|∈[40, 50), Δ=3): Top-1/total: 10 / 100 = 0.1
%
%(|σ|∈[50, 60), Δ=1): Top-1/total: 55 / 100 = 0.55
%(|σ|∈[50, 60), Δ=2): Top-1/total: 25 / 100 = 0.25
%(|σ|∈[50, 60), Δ=3): Top-1/total: 9 / 100 = 0.09
%
%(|σ|∈[60, 70), Δ=1): Top-1/total: 53 / 100 = 0.53
%(|σ|∈[60, 70), Δ=2): Top-1/total: 23 / 100 = 0.23
%(|σ|∈[60, 70), Δ=3): Top-1/total: 11 / 100 = 0.11
%
%(|σ|∈[70, 80), Δ=1): Top-1/total: 57 / 100 = 0.57
%(|σ|∈[70, 80), Δ=2): Top-1/total: 18 / 100 = 0.18
%(|σ|∈[70, 80), Δ=3): Top-1/total: 9 / 81 = 0.1111111111111111
%
%Precision@All
%=============
%(|σ|∈[0, 10), Δ=1): Top-1/total: 100 / 100 = 1.0
%(|σ|∈[0, 10), Δ=2): Top-1/total: 100 / 100 = 1.0
%(|σ|∈[0, 10), Δ=3): Top-1/total: 50 / 50 = 1.0
%
%(|σ|∈[10, 20), Δ=1): Top-1/total: 100 / 100 = 1.0
%(|σ|∈[10, 20), Δ=2): Top-1/total: 99 / 100 = 0.99
%(|σ|∈[10, 20), Δ=3): Top-1/total: 98 / 100 = 0.98
%
%(|σ|∈[20, 30), Δ=1): Top-1/total: 100 / 100 = 1.0
%(|σ|∈[20, 30), Δ=2): Top-1/total: 98 / 100 = 0.98
%(|σ|∈[20, 30), Δ=3): Top-1/total: 79 / 99 = 0.797979797979798
%
%(|σ|∈[30, 40), Δ=1): Top-1/total: 99 / 100 = 0.99
%(|σ|∈[30, 40), Δ=2): Top-1/total: 100 / 100 = 1.0
%(|σ|∈[30, 40), Δ=3): Top-1/total: 70 / 100 = 0.7
%
%(|σ|∈[40, 50), Δ=1): Top-1/total: 99 / 100 = 0.99
%(|σ|∈[40, 50), Δ=2): Top-1/total: 100 / 100 = 1.0
%(|σ|∈[40, 50), Δ=3): Top-1/total: 55 / 100 = 0.55
%
%(|σ|∈[50, 60), Δ=1): Top-1/total: 100 / 100 = 1.0
%(|σ|∈[50, 60), Δ=2): Top-1/total: 100 / 100 = 1.0
%(|σ|∈[50, 60), Δ=3): Top-1/total: 42 / 100 = 0.42
%
%(|σ|∈[60, 70), Δ=1): Top-1/total: 97 / 100 = 0.97
%(|σ|∈[60, 70), Δ=2): Top-1/total: 94 / 100 = 0.94
%(|σ|∈[60, 70), Δ=3): Top-1/total: 42 / 100 = 0.42
%
%(|σ|∈[70, 80), Δ=1): Top-1/total: 97 / 100 = 0.97
%(|σ|∈[70, 80), Δ=2): Top-1/total: 90 / 100 = 0.9
%(|σ|∈[70, 80), Δ=3): Top-1/total: 25 / 81 = 0.30864197530864196


%  Synthetic evaluation
%
%  \begin{table}[!h]
%    \centering
%    \begin{tabular}{c|c|cccccccc}
%      \hline\hline
%      & $\Delta$ & \multicolumn{8}{c}{Precision@1} \\ \hline
%      $|\err\sigma|$ &  & $(0,10)$ & $[10,20)$ & $[20,30)$ & $[30, 40)$ & $[40,50)$ & $[50, 60)$ & $[60,70)$ & $[70, 80)$ \\ \hline
%      Tidyparse
%      & 1 & 1.00 & 1.00 & 1.00 & 0.99 & 1.00 & 1.00 & 1.00 & 0.98 \\
%      & 2 & 0.45 & 0.63 & 0.66 & 0.68 & 0.65 & 0.81 & 0.64 & 0.62 \\
%      & 3 & 0.06 & 0.20 & 0.29 & 0.36 & 0.29 & 0.39 & 0.38 & 0.32 \\ \hline
%      & & \multicolumn{8}{c}{Precision@All} \\ \hline
%      Tidyparse
%      & 1 & 1.00 & 1.00 & 1.00 & 0.99 & 1.00 & 1.00 & 0.98 & 1.00 \\
%      & 2 & 0.98 & 0.98 & 0.94 & 0.94 & 0.98 & 0.97 & 0.89 & 0.90 \\
%      & 3 & 1.00 & 0.97 & 0.92 & 0.84 & 0.87 & 0.90 & 0.84 & 0.72 \\ \hline\hline
%    \end{tabular}
%  \end{table}

\clearpage\section{Supplemental Proofs}

The problem of syntax error correction under a finite number of typographic errors is reducible to the bounded Levenshtein-CFL reachability problem, which can be formally stated as follows:

\begin{definition}
The language edit distance (LED) is the minimum number of edits required to transform an invalid string into a valid one, where validity is defined as containment in a context-free language, $\ell$, i.e., $\Delta^*(\err{\sigma}, \ell) \coloneqq \min_{\sigma \in \ell}\Delta(\err{\sigma}, \sigma)$, and $\Delta$ is the Levenshtein distance.
\end{definition}

We seek to find the set of strings $S$ such that $\forall \sigma'\in S, \Delta(\err{\sigma}, \sigma') \leq q$, where $q$ is greater than or equal to the language edit distance. We call this set the \textit{Levenshtein ball} of $\err{\sigma}$ and denote it $\Delta_q(\err{\sigma})$. Since $1 \leq \Delta^*(\err{\sigma}, \ell) \leq q$, we have $1 \leq q$. We now consider an upper bound on $\Delta^*(\err{\sigma}, \ell)$, i.e., the greatest lower bound on $q$ such that $\Delta_q(\err{\sigma}) \cap \ell \neq \varnothing$.

\begin{lemma}\label{lemma:upper-bound}
For any nonempty language $\ell$ and invalid string $\err{\sigma}: \Sigma^n \cap \bar\ell$, there exists an $(\sigma', m)$ such that $\sigma' \in \ell\cap\Sigma^m$ and $0 < \Delta(\err{\sigma}, \ell) \leq \max(m, n) < \infty$.
\end{lemma}

\begin{proof}
Since $\ell$ is nonempty, it must have at least one inhabitant $\sigma \in \ell$. Let $\sigma'$ be the smallest such member. Since $\sigma'$ is a valid sentence in $\ell$, by definition it must be that $|\sigma'|<\infty$. Let $m\coloneqq|\sigma'|$. Since we know $\err{\sigma} \notin \ell$, it follows that $0 < \Delta(\err{\sigma}, \ell)$. Let us consider two cases, either $\sigma' = \varepsilon$, or $0 < |\sigma'|$:

\begin{itemize}
\item If $\sigma' = \varepsilon$, then $\Delta(\err{\sigma}, \sigma') = n$ by full erasure of $\err{\sigma}$, or
\item If $0 < m$, then $\Delta(\err{\sigma}, \sigma') \leq \max(m, n)$ by overwriting.
\end{itemize}

In either case, it follows $\Delta(\err{\sigma}, \ell) \leq \max(m, n)$ and $\ell$ is always reachable via a finite nonempty set of Levenshtein edits, i.e., $0 < \Delta(\err{\sigma}, \ell) < \infty$.
\end{proof}

Let us now consider the maximum growth rate of the \textit{admissible set}, $\ell_\cap \coloneqq \Delta_q(\err{\sigma}) \cap \ell$, as a function of $q$ and $n$. Let $\bar\ell \coloneqq \{\err{\sigma}\}$. Since $\bar\ell$ is finite and thus regular, $\ell = \Sigma^* \setminus \{\err{\sigma}\}$ is regular by the closure of regular languages under complementation, and thus context-free a fortiori. Since $\ell$ accepts every string except $\err{\sigma}$, it represents the worst CFL in terms of asymptotic growth of $\ell_\cap$.

\begin{lemma}\label{lemma:interleaving}
The complexity of enumerating $\ell_\cap$ is upper bounded by $\mathcal{O}\left(\sum_{c=1}^q{{cn + n + c} \choose c}(|\Sigma| + 1)^c\right)$.
\end{lemma}

\begin{proof}
We can overestimate the size of $\ell_\cap$ by considering the number of unique ways to insert, delete, or substitute $c$ terminals into a string $\err{\sigma}$ of length $n$. This can be overaproximated by interleaving $\varepsilon^c$ around every token, i.e., $\err{\sigma}_\varepsilon\coloneqq \left(\varepsilon^c\err{\sigma}_i\right)_{i=1}^n\varepsilon^c$, where $|\err{\sigma}_\varepsilon| = cn + n + c$, and only considering substitution. We augment $\Sigma_\varepsilon \coloneqq \Sigma \cup \{\varepsilon\}$ so that deletions and insertions may be treated as special cases of substitution. Thus, we have $cn + n + c$ positions to substitute $(|\Sigma_\varepsilon|)$ tokens, i.e., ${{cn + n + c} \choose c}|\Sigma_\varepsilon|^c$ ways to edit $\err{\sigma}_\varepsilon$ for each $c \in [1, q]$. This upper bound is not tight, as overcounts many identical edits w.r.t. $\err{\sigma}$. Nonetheless, it is sufficient to show $|A| < \sum_{c=1}^q{{cn + n + c} \choose c}|\Sigma_\varepsilon|^c$.
\end{proof}

We note that the above bound applies to all strings and languages, and relates to the Hamming bound on $H_q(\err{\sigma}_\varepsilon)$, which only considers substitutions. In practice, much tighter bounds may be obtained by considering the structure of $\ell$ and $\err{\sigma}$. For example, based on an empirical evaluation from a dataset of human errors and repairs in Python code snippets ($|\Sigma| = 50, |\err{\sigma}| < 40, \Delta(\err{\sigma}, \ell) \in [1, 3]$), we estimate the \textit{filtration rate}, i.e., the density of the admissible set relative to the Levenshtein ball, $D=|A|/|\Delta_q(\err{\sigma})|$ to have empirical mean $E_\sigma[D] \approx 2.6\times 10^{-4}$, and variance $\mathrm{Var}_\sigma[D] \approx 3.8\times10^{-7}$.

%  In practice, this problem is ill-posed even when $q = \Delta^*(\err{\sigma}, \ell) \approx 1$. For example, consider the language of ursine dietary preferences. Although $\err{\sigma}\coloneqq$ ``Bears like to eat plastic'' is not a valid sentence, e.g., $\sigma'\coloneqq$``Bears like to eat'' is $(\Delta^*=1)$, however there are many others with roughly the same edit distance, e.g., ``Bears like to eat \{\hlorange{berries}, \hlorange{honey}, \hlorange{fish}\}'', or ``\{\hlgreen{Polar}, \hlgreen{Panda}\} bears like to eat \{\hlgreen{seals}, \hlgreen{bamboo}\}''. In general, there are usually many strings nearby $\err{\sigma}$, and we seek to find those among them which are both syntactically valid and semantically plausible as quickly as possible.

\end{document}