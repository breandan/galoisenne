%! suppress = LineBreak
%% For double-blind review submission, w/o CCS and ACM Reference (max submission space)
%\documentclass[sigplan,10pt,review,anonymous]{acmart}
%\settopmatter{printfolios=false,printccs=false,printacmref=false}
%% For double-blind review submission, w/ CCS and ACM Reference
%\documentclass[sigplan,review,anonymous]{acmart}\settopmatter{printfolios=true}
%% For single-blind review submission, w/o CCS and ACM Reference (max submission space)
%\documentclass[sigplan,review]{acmart}\settopmatter{printfolios=true,printccs=false,printacmref=false}
%% For single-blind review submission, w/ CCS and ACM Reference
%\documentclass[sigplan,review]{acmart}\settopmatter{printfolios=true}
%% For final camera-ready submission, w/ required CCS and ACM Reference
%\documentclass[sigplan,nonacm]{acmart}
\documentclass[sigplan,review,acmsmall,nonacm,screen,anonymous]{acmart}\settopmatter{printfolios=false,printccs=false,printacmref=false}

%% Conference information
%% Supplied to authors by publisher for camera-ready submission;
%% use defaults for review submission.
%\acmConference[SPLASH'24]{ACM SIGPLAN conference on Systems, Programming, Languages, and Applications: Software for Humanity}{October 22-27, 2024}{Pasadena, California, United States}
%\acmConference{}{}{}
%\acmYear{2018}
%\acmISBN{} % \acmISBN{978-x-xxxx-xxxx-x/YY/MM}
%\acmDOI{} % \acmDOI{10.1145/nnnnnnn.nnnnnnn}
%\startPage{1}

%% Copyright information
%% Supplied to authors (based on authors' rights management selection;
%% see authors.acm.org) by publisher for camera-ready submission;
%% use 'none' for review submission.
\setcopyright{none}
%\setcopyright{acmcopyright}
%\setcopyright{acmlicensed}
%\setcopyright{rightsretained}
%\copyrightyear{2018}           %% If different from \acmYear

%% Bibliography style
\bibliographystyle{acmart}

\usepackage{dsfont}
\usepackage{stmaryrd}
\usepackage{colortbl}
\usepackage{hyperref}

\usepackage{amsmath}
\DeclareMathOperator*{\argmax}{argmax}
\DeclareMathOperator*{\argmin}{argmin}
\usepackage{amssymb}

\usepackage[dvipsnames, table]{xcolor}
\usepackage{textcomp}

% Packages
\usepackage[pdf]{graphviz}
\usepackage{mathrsfs}

\newcommand*\circled[1]{\tikz[baseline=-0.1cm]{
  \node[shape=circle,draw,inner sep=0.48pt] (char) {\fontsize{7}{12}\textsf{#1}};}}

\DeclareMathAlphabet{\mathcal}{OMS}{cmsy}{m}{n}
\usepackage{cancel}
\newcommand\ccancel[2][red]{\renewcommand\CancelColor{\color{#1}}\cancel{#2}}
\newcommand{\nDownarrow}{\ensuremath{\text{ }\cancel{\Downarrow}\text{ }}}
\usepackage{centernot}

\usepackage{pgfplots, pgfplotstable}
\pgfplotsset{compat=1.7}
\usepgfplotslibrary{fillbetween}
\usetikzlibrary{patterns}
\pgfmathdeclarefunction{gauss}{2}{\pgfmathparse{1/(#2*sqrt(2*pi))*exp(-((x-#1)^2)/(2*#2^2))}}
\pgfmathdeclarefunction{nil}{1}{\pgfmathparse{0.001}}

\usepackage{arydshln}
\usepackage{adjustbox}
\usepackage{enumerate}
\usepackage{enumitem}
\usepackage{tikz-cd}
\usetikzlibrary{calc}
\usepackage{amsfonts}
%\usepackage{prooftrees}
\usepackage{bussproofs}
\renewcommand{\sectionautorefname}{\S}
\renewcommand{\subsectionautorefname}{\S}
\usepackage{float}

\usepackage{tikz-3dplot}
\usetikzlibrary{3d}
\usetikzlibrary{calligraphy}
\newif\ifshowcellnumber
\showcellnumbertrue

\usepackage{algorithm}
\usepackage{algpseudocode}
\usepackage{algorithmicx}
\usepackage{sourcecodepro}
\usepackage{tikz-qtree}
\usepackage{amsthm}
\usepackage{bm}
\usetikzlibrary{bayesnet}
\usetikzlibrary{arrows}
\usepackage{subcaption}
\usetikzlibrary{backgrounds}
\usetikzlibrary{tikzmark}

\newcommand{\E}{\mathbb{E}}
\newcommand{\Var}{\mathrm{Var}}
\newcommand{\Cov}{\mathrm{Cov}}

\newcommand{\CompOrder}{\mathcal{O}}
\def\graphspace{\mathbf{G}}
\def\Uniform{\mbox{\rm Uniform}}
\def\Gaussian{\mbox{\rm Gaussian}}
\def\Bernoulli{\mbox{\rm Bernoulli}}
\def\Dirichlet{\mbox{\rm Dirichlet}}

\usepackage{mathtools}% superior to amsmath
\usepackage{tikz}
% Packages
\usepackage{listings}
\DeclareRobustCommand{\hlred}[1]{{\sethlcolor{pink}\hl{#1}}}
\usepackage{fontspec}

\setmonofont[Scale=0.8]{JetBrainsMono}[
  Contextuals={Alternate},
  Path=./font/,
  Extension = .ttf,
  UprightFont=*-Regular,
  BoldFont=*-Bold,
  ItalicFont=*-Italic,
  BoldItalicFont=*-BoldItalic
]

\usepackage[skins,breakable,listings]{tcolorbox}

\lstdefinelanguage{kotlin}{
  comment=[l]{//},
  commentstyle={\color{gray}\ttfamily},
  emph={delegate, filter, firstOrNull, forEach, it, lazy, mapNotNull, println, repeat, assert, with, head, tail, len, return@},
  numberstyle=\noncopyable,
  identifierstyle=\color{black},
  keywords={abstract, actual, as, as?, break, by, class, companion, continue, data, do, dynamic, else, enum, expect, false, final, for, fun, get, if, import, in, infix, interface, internal, is, null, object, open, operator, override, package, private, public, return, sealed, set, super, suspend, this, throw, true, try, catch, typealias, val, var, vararg, when, where, while, tailrec, reified},
  keywordstyle={\bfseries},
  morecomment=[s]{/*}{*/},
  morestring=[b]",
  morestring=[s]{"""*}{*"""},
  ndkeywords={@Deprecated, @JvmField, @JvmName, @JvmOverloads, @JvmStatic, @JvmSynthetic, Array, Byte, Double, Float, Boolean, Int, Integer, Iterable, Long, Runnable, Short, String, int},
  ndkeywordstyle={\bfseries},
  sensitive=true,
  stringstyle={\ttfamily},
  literate={`}{{\char0}}1,
  escapeinside={(*@}{@*)}
}
\lstdefinelanguage{tidy}{
  comment=[l]{//},
  commentstyle={\color{gray}\ttfamily},
  emph={|, ->, ---},
  emphstyle={\color{red}},
  identifierstyle=\color{black},
  keywords={\|, ->, ---},
  otherkeywords={|,->},
  morekeywords={|,->},
  keywordstyle={\color{blue}\bfseries},
  morecomment=[s]{/*}{*/},
  morestring=[b]",
  morestring=[s]{"""*}{*"""},
  ndkeywords={@Deprecated, @JvmField, @JvmName, @JvmOverloads, @JvmStatic, @JvmSynthetic, Array, Byte, Double, Float, Int, Integer, Iterable, Long, Runnable, Short, String},
  ndkeywordstyle={\color{orange}\bfseries},
  sensitive=true,
  stringstyle={\color{green}\ttfamily},
  literate={`}{{\char0}}1
}

%%%%%%%%%%%%%%%%%%%%%%%%%%%%%%%%%%%%%%%%%%%
%
% Color boxes
%
%%%%%%%%%%%%%%%%%%%%%%%%%%%%%%%%%%%%%%%%%%%

\tcbset{
  enhanced jigsaw,
  breakable,
  listing only,
%  boxsep=-1pt,
%  top=-1pt,
  bottom=0.1cm,
  right=0.5cm,
  overlay first={
    \node[black!50] (S) at (frame.south) {\Large\ding{34}};
    \draw[dashed,black!50] (frame.south west) -- (S) -- (frame.south east);
  },
  overlay middle={
    \node[black!50] (S) at (frame.south) {\Large\ding{34}};
    \draw[dashed,black!50] (frame.south west) -- (S) -- (frame.south east);
    \node[black!50] (S) at (frame.north) {\Large\ding{34}};
    \draw[dashed,black!50] (frame.north west) -- (S) -- (frame.north east);
  },
  overlay last={
    \node[black!50] (S) at (frame.north) {\Large\ding{34}};
    \draw[dashed,black!50] (frame.north west) -- (S) -- (frame.north east);
  },
  before={\par\vspace{5pt}},
  after={\par\vspace{\parskip}\noindent}
}

\definecolor{slightgray}{rgb}{0.90, 0.90, 0.90}

\usepackage{soul}
\makeatletter
\def\SOUL@hlpreamble{%
  \setul{}{3.0ex}%
  \let\SOUL@stcolor\SOUL@hlcolor%
  \SOUL@stpreamble%
}
\makeatother

\newcommand{\inline}[1]{%
  \begingroup%
  \sethlcolor{slightgray}%
  \hl{\ttfamily\footnotesize #1}%
  \endgroup
}

\newcommand{\tinline}[1]{%
  \begingroup%
  \sethlcolor{slightgray}%
  \hl{\ttfamily\tiny #1}%
  \endgroup
}

\newtcblisting{halftidyinput}[1][]{%
  left skip=0.7cm,
  width=6cm,
%  left=-0.01cm,
  top=-0.1cm,
  bottom=-0.35cm,
  listing options={
    language=tidy,
    basicstyle=\ttfamily\small,
%numberstyle=\footnotesize,
    showstringspaces=false,
    tabsize=2,
    breaklines=true,
    numbers=none,
    inputencoding=utf8,
    escapeinside={(*@}{@*)},
    #1
  },
  underlay unbroken and first={%
    \path[draw=none] (interior.north west) rectangle node[white]{\includegraphics[width=4mm]{../figures/tidyparse_logo.png}} ([xshift=-10mm,yshift=-7mm]interior.north west);
  }
}

\newtcblisting{wholetidyinput}[1][]{%
  left skip=0.7cm,
  top=0.1cm,
  middle=0mm,
  boxsep=0mm,
  listing options={
    language=tidy,
    basicstyle=\ttfamily\small,
%numberstyle=\footnotesize,
    showstringspaces=false,
    tabsize=2,
    breaklines=true,
    numbers=none,
    inputencoding=utf8,
    escapeinside={(*@}{@*)},
    #1
  },
  underlay unbroken and first={%
      \path[draw=none] (interior.north west) rectangle node[white]{\includegraphics[width=4mm]{../figures/tidyparse_logo.png}} ([xshift=-10mm,yshift=-9mm]interior.north west);
  }
}

\definecolor{A}{RGB}{6,150,104}
\definecolor{B}{RGB}{196,74,137}
\definecolor{C}{RGB}{117,237,133}
\definecolor{D}{RGB}{246,46,243}
\definecolor{E}{RGB}{89,162,12}
\definecolor{F}{RGB}{113,12,158}
\definecolor{G}{RGB}{191,205,142}
\definecolor{H}{RGB}{51,58,158}
\definecolor{I}{RGB}{244,212,3}
\definecolor{J}{RGB}{37,36,249}
\definecolor{K}{RGB}{253,165,71}
\definecolor{L}{RGB}{27,81,29}
\colorlet{LA}{A!30}
\colorlet{LB}{B!30}
\colorlet{LC}{C!30}
\colorlet{LD}{D!30}
\colorlet{LE}{E!30}
\colorlet{LF}{F!30}
\colorlet{LG}{G!30}
\colorlet{LH}{H!30}
\colorlet{LI}{I!30}
\colorlet{LJ}{J!30}
\colorlet{LK}{K!30}
\colorlet{LL}{L!30}
\newcommand{\hiliA}[1]{%
  \colorbox{LA}{$#1$}}
\newcommand{\hiliB}[1]{%
  \colorbox{LB}{$#1$}}
\newcommand{\hiliC}[1]{%
  \colorbox{LC}{$#1$}}
\newcommand{\hiliD}[1]{%
  \colorbox{LD}{$#1$}}
\newcommand{\hiliE}[1]{%
  \colorbox{LE}{$#1$}}
\newcommand{\hiliF}[1]{%
  \colorbox{LF}{$#1$}}
\newcommand{\hiliG}[1]{%
  \colorbox{LG}{$#1$}}
\newcommand{\hiliH}[1]{%
  \colorbox{LH}{$#1$}}
\newcommand{\hiliI}[1]{%
  \colorbox{LI}{$#1$}}
\newcommand{\hiliJ}[1]{%
  \colorbox{LJ}{$#1$}}
\newcommand{\hiliK}[1]{%
  \colorbox{LK}{$#1$}}
\newcommand{\hiliL}[1]{%
  \colorbox{LL}{$#1$}}
\newcommand{\highlight}[1]{%
  \colorbox{lgray}{$#1$}}
\colorlet{lred}{red!30}
\colorlet{lorange}{orange!30}
\colorlet{lgreen}{green!30}
\colorlet{lgray}{black!15}
\colorlet{dgray}{black!75}
\DeclareRobustCommand{\hlred}[1]{{\sethlcolor{lred}\hl{#1}}}
\DeclareRobustCommand{\hlorange}[1]{{\sethlcolor{lorange}\hl{#1}}}
\DeclareRobustCommand{\hlgreen}[1]{{\sethlcolor{lgreen}\hl{#1}}}
\DeclareRobustCommand{\hlgray}[1]{{\sethlcolor{lgray}\hl{#1}}}
\DeclareRobustCommand{\caret}[1]{{\sethlcolor{dgray}\textcolor{white}{\hl{#1}}}}

\usepackage{url}
\usepackage{qtree}

\usepackage{filecontents}
\usepackage{pstricks-add}
\usepackage{emoji}
\usepackage{alltt}
\usepackage{nicematrix}
\usepackage{graphicx}
\usepackage{ulem}
\usepackage{upquote}
\tikzstyle{every picture}+=[remember picture]
\usepackage{menukeys}
\pgfplotstableread[col sep=comma,]{timings_loc.csv}\loctimings
\pgfplotstableread[col sep=comma,]{timings_unloc.csv}\unloctimings

\makeatletter
\DeclareRobustCommand{\cev}[1]{%
  {\mathpalette\do@cev{#1}}%
}
\newcommand{\do@cev}[2]{%
  \vbox{\offinterlineskip
  \sbox\z@{$\m@th#1 x$}%
  \ialign{##\cr
  \hidewidth\reflectbox{$\m@th#1\vec{}\mkern4mu$}\hidewidth\cr
  \noalign{\kern-\ht\z@}
    $\m@th#1#2$\cr
  }%
  }%
}
\makeatother

\makeatletter
\DeclareRobustCommand{\pder}[1]{%
  \@ifnextchar\bgroup{\@pder{#1}}{\@pder{}{#1}}}
\newcommand{\@pder}[2]{\frac{\partial#1}{\partial#2}}
\makeatother

\newcommand{\shup}{\shortuparrow}
\newcommand{\shri}{\shortrightarrow}
\newcommand{\shur}{\shup\hspace{-5pt}\shri}

\makeatletter
\def\squiggly{\bgroup \markoverwith{\textcolor{red}{\lower3\p@\hbox{\sixly \char58}}}\ULon}
\makeatother

\newcommand{\err}[1]{\smash{\squiggly{#1}{}}}
\newcommand{\stirlingii}{\genfrac{\{}{\}}{0pt}{}}

%======== Arrows =========
\newcommand{\knightarrow}{
  \tikz{
    \fill (0pt,0pt) circle [radius = 1pt];
    \fill (0pt,6pt) circle [radius = 1pt];
    \fill (6pt,0pt) circle [radius = 1pt];
    \fill (6pt,6pt) circle [radius = 1pt];
    \fill (12pt,0pt) circle [radius = 1pt];
    \fill (12pt,6pt) circle [radius = 1pt];
    \fill (6pt,0pt) circle [radius = 1pt];
    \fill (12pt,0pt) circle [radius = 1pt];
    \draw [-to] (0pt,0pt) -- (12pt,6pt);
  }
}

\newcommand{\kingarrow}{
  \tikz{
    \fill (0pt,0pt) circle [radius = 1pt];
    \fill (6pt,0pt) circle [radius = 1pt];
    \fill (0pt,6pt) circle [radius = 1pt];
    \fill (6pt,6pt) circle [radius = 1pt];
    \draw [-to] (0pt,0pt) -- (6pt,6pt);
    \draw [-to] (0pt,0pt) -- (0pt,6pt);
    \draw [-to] (0pt,0pt) -- (6pt,0pt);
  }
}

\newcommand{\knightkingarrow}{
  \tikz{
    \fill (0pt,0pt) circle [radius = 1pt];
    \fill (0pt,6pt) circle [radius = 1pt];
    \fill (6pt,0pt) circle [radius = 1pt];
    \fill (6pt,6pt) circle [radius = 1pt];
    \fill (12pt,0pt) circle [radius = 1pt];
    \fill (12pt,6pt) circle [radius = 1pt];
    \draw [-to] (0pt,0pt) -- (6pt,6pt);
    \draw [-to] (0pt,0pt) -- (0pt,6pt);
    \draw [-to] (0pt,0pt) -- (6pt,0pt);
    \draw [-to] (0pt,0pt) -- (12pt,6pt);
  }
}

%======== Arrows =========

\usetikzlibrary{decorations.pathreplacing,automata,calc,positioning,matrix,fit}
\usepackage{wrapfig}

\newcommand{\mkTrellis}[1]{
  \begin{tikzpicture}
    \def\dx{20pt}
    \def\dy{30pt}
    \newcounter{i}
    \stepcounter{i}
    \node[circle, draw, fill=black!30] (\arabic{i}) at (0,0){};
    \foreach [count=\i] \x in {2,...,#1}{
      \pgfmathsetmacro{\lox}{\x-1}%
      \pgfmathsetmacro{\loxt}{\x-3}%
      \foreach [count=\j] \xx in {-\lox,-\loxt,...,\lox}{
        \pgfmathsetmacro{\jj}{\j-1}%
        \stepcounter{i}
        \pgfmathsetmacro{\kk}{\xx-2}%
        \pgfmathsetmacro{\lbl}{\lox!/(\jj!*(\lox-\jj)!)}
        \ifnum\x<\kk
        \pgfmath\node[circle, draw]  (\arabic{i}) at (\xx*\dx, -\lox*\dy) {};
        \else
        \pgfmath\node[circle, draw, fill=black!30]  (\arabic{i}) at (\xx*\dx, -\lox*\dy) {};
        \fi
      }
    }
    \newcounter{z}
    \newcounter{xn}
    \newcounter{xnn}
    \pgfmathsetmacro{\maxx}{#1 - 1}
    \foreach \x in {1,...,\maxx}{
      \foreach \xx in {1,...,\x}{
        \stepcounter{z}
        \setcounter{xn}{\arabic{z}}
        \addtocounter{xn}{\x}
        \setcounter{xnn}{\arabic{xn}}
        \stepcounter{xnn}
        \draw [<-] (\arabic{z}) -- (\arabic{xn});
        \draw [<-] (\arabic{z}) -- (\arabic{xnn});
      }
    }
  \end{tikzpicture}
}

\newcommand{\dx}{20pt}
\newcommand{\dy}{30pt}
\newcounter{i}
\newcounter{z}
\newcounter{xn}
\newcounter{xnn}
\newcommand{\mkTrellisAppend}[1]{
  \begin{tikzpicture}
    \setcounter{i}{0}
    \setcounter{z}{0}
    \setcounter{xn}{0}
    \setcounter{xnn}{0}
    \stepcounter{i}
    \node[circle, draw] (\arabic{i}) at (0,0){};
    \foreach [count=\i] \x in {2,...,#1}{
      \pgfmathsetmacro{\lox}{\x-1}%
      \pgfmathsetmacro{\loxt}{\x-3}%
      \foreach [count=\j] \xx in {-\lox,-\loxt,...,\lox}{
        \pgfmathsetmacro{\jj}{\j-1}%
        \stepcounter{i}
        \pgfmathsetmacro{\kk}{\xx+2}%
        \pgfmathsetmacro{\lbl}{\lox!/(\jj!*(\lox-\jj)!)}
        \ifnum\x>\kk
        \pgfmath\node[circle, draw, fill=black!30]  (\arabic{i}) at (\xx*\dx, -\lox*\dy) {};
        \else
        \pgfmath\node[circle, draw]  (\arabic{i}) at (\xx*\dx, -\lox*\dy) {};
        \fi
      }
    }
    \pgfmathsetmacro{\maxx}{#1 - 1}
    \foreach \x in {1,...,\maxx}{
      \foreach \xx in {1,...,\x}{
        \stepcounter{z}
        \setcounter{xn}{\arabic{z}}
        \addtocounter{xn}{\x}
        \setcounter{xnn}{\arabic{xn}}
        \stepcounter{xnn}
        \draw [<-] (\arabic{z}) -- (\arabic{xn});
        \draw [<-] (\arabic{z}) -- (\arabic{xnn});
      }
    }
  \end{tikzpicture}
}

\newcommand{\mkTrellisInsert}[1]{
  \begin{tikzpicture}
    \setcounter{i}{0}
    \setcounter{z}{0}
    \setcounter{xn}{0}
    \setcounter{xnn}{0}
    \stepcounter{i}
    \node[circle, draw] (\arabic{i}) at (0,0){};
    \foreach [count=\i] \x in {2,...,#1}{
      \pgfmathsetmacro{\lox}{\x-1}%
      \pgfmathsetmacro{\loxt}{\x-3}%
      \foreach [count=\j] \xx in {-\lox,-\loxt,...,\lox}{
        \pgfmathsetmacro{\jj}{\j-1}%
        \stepcounter{i}
        \pgfmathsetmacro{\mp}{\xx+#1}%
        \pgfmathsetmacro{\mq}{\xx+\x}%
        \pgfmathsetmacro{\lbl}{\lox!/(\jj!*(\lox-\jj)!)}
        \ifnum\x>\mp
        \pgfmath\node[circle, draw, fill=black!30]  (\arabic{i}) at (\xx*\dx, -\lox*\dy) {};
        \else
        \ifnum#1<\mq
        \pgfmath\node[circle, draw, fill=black!30]  (\arabic{i}) at (\xx*\dx, -\lox*\dy) {};
        \else
        \pgfmath\node[circle, draw]  (\arabic{i}) at (\xx*\dx, -\lox*\dy) {};
        \fi
        \fi

      }
    }
    \pgfmathsetmacro{\maxx}{#1 - 1}
    \foreach \x in {1,...,\maxx}{
      \foreach \xx in {1,...,\x}{
        \stepcounter{z}
        \setcounter{xn}{\arabic{z}}
        \addtocounter{xn}{\x}
        \setcounter{xnn}{\arabic{xn}}
        \stepcounter{xnn}
        \draw [<-] (\arabic{z}) -- (\arabic{xn});
        \draw [<-] (\arabic{z}) -- (\arabic{xnn});
      }
    }
  \end{tikzpicture}
}

\usetikzlibrary{automata, positioning, arrows}

\newcommand{\nobarfrac}{\genfrac{}{}{0pt}{}}
\pgfplotstableread[col sep=comma,]{timings_loc.csv}\loctimings
\pgfplotstableread[col sep=comma,]{timings_unloc.csv}\unloctimings
\pgfplotstableread[col sep=comma,]{natural_errors.csv}\naturalerrors
\pgfplotstableread[col sep=comma,]{synthetic_errors.csv}\syntheticerrors


%\usepackage{draftwatermark}
%\SetWatermarkLightness{0.75}
%\SetWatermarkText{DRAFT}
%\makeatletter
%\let\@authorsaddresses\@empty
%\makeatother

\begin{document}
%
\title{Syntax Repair as Language Intersection}
%
\begin{abstract}
We introduce a new technique for repairing syntax errors in arbitrary context-free languages. This technique models syntax repair as a language intersection problem by defining a finite language that provably generates every syntactically valid repair within a given edit distance. Leveraging a theoretical connection between the Bar-Hillel construction from formal language theory and CFL reachability from program analysis, we show that repairability in a finite number of typographic edits is polylogarithmic parallel time decidable and provide an enumeration algorithm based on the Brzozowski derivative. Finally, we evaluate this algorithm and its implementation, demonstrating state-of-the-art results on a Python syntax repair benchmark.\keywords{Error correction \and CFL reachability \and Language games.}
\end{abstract}

%\titlerunning{Abbreviated paper title}
% If the paper title is too long for the running head, you can set
% an abbreviated paper title here
\author{Breandan Considine}
\email{bre@ndan.co}

\maketitle

\section{Introduction}\label{sec:intro}

When programming, one invariably encounters a recurring scenario in which the editor occupies an unparseable state. Faced with this predicament, programmers must spend time to locate and repair the error before proceeding. In the following paper, we propose to solve this problem automatically by generating a list of candidate repairs that contains with high probability the true repair, assuming this repair differs by no more than a few edits from the broken source code.

Prior research on syntax repair can be classified into exact and approximate methods. In the former, specialized parsers with error recovery are used to propose an alternative. While appealing for their interpretability and well-understood algorithmic properties, these methods are too weak to model the full distribution of natural source code and must rely on relatively brittle heuristics.

In the latter case, the set of all strings is typically used as the sample space for a distribution whose parameters are learned from a dataset of pairwise errors and fixes. Though statistically more robust, these methods typically use some form of approximate inference and thus require expensive postprocessing or rejection sampling to ensure the generated results conform to the grammar.

The primary shortcoming with both approaches is \textit{they generate far too few repairs}. As we will show, even if the repair model guarantees correctness or has good statistical generalization properties, it is likely to miss the intended repair in ambiguous scenarios or when there are many candidates from which to choose. Most syntax errors, however, require only a few typographic modifications to repair, of which there are only a finite number of possibilities to consider.

Thus we arrive at the core problem this paper aims to solve: how can we quickly recover the most probable repairs in proximity to a syntactically broken code snippet? To address this problem, we propose to extensively evaluate the probability of every repair within a fixed edit distance. At first, this might seem to take much longer than generating a single repair, but if we intend to quickly generate probable repairs and not just valid ones, extensive search becomes highly advantageous. To ensure the search space is well-defined, we will construct and decode a regular expression that generates all and only valid repairs within a fixed edit distance, thereby avoiding rejection sampling entirely without skipping any nearby valid repairs. This construction is shown in Fig.~\ref{fig:arch_simp}.

\begin{figure}[H]
  \includegraphics[width=\textwidth]{flow}\vspace{-0.1cm}
  \caption{
    Our algorithm first constructs an automaton representing all strings within a certain edit distance. This automaton is parsed into a matrix denoting all valid repairs in the programming language and edit distance. We construct a regular expression (RE) from the matrix, and finally decode the RE using an n-gram model to produce a finite list of samples, then rerank and truncate this list to obtain our final repairs.
%    Given a grammar and broken code fragment, our method creates an automaton generating the language of local edits, then constructs a regular expression. This regular expression can be decoded to produce a list which is then reranked and truncated to obtain the most probable repairs.
  }\label{fig:arch_simp}\vspace{-0.2cm}
\end{figure}

To operationalize this technique, we design, develop and benchmark a new developer tool for syntax repair which is readily executable on off-the-shelf GPUs. We provide a reference implementation of our tool on the WebGPU platform and show these computational resources, which typically sit idle during text editing, can be profitably used to accelerate real-time program repair.

Finally, we show the efficacy of this technique for locating and repairing syntax errors of up to three edits and eighty lexical tokens in under ten seconds, practical for a few lines of source code in realistic programming languages. Our work shows this technique is highly effective at predicting the human repair across a dataset of Python source code, up to 5x more accurately than previous state-of-the-art methods at comparable latency and compute thresholds.

\section{Background}

Recall that a CFG, $\mathcal{G} = \langle \Sigma, V, P, S\rangle$, is a quadruple consisting of terminals $(\Sigma)$, nonterminals $(V)$, productions $\big(P\colon V \rightarrow (V \mid \Sigma)^+\big)$, and a start symbol, $(S)$. Every CFG is reducible to so-called \textit{Chomsky Normal Form}~\cite{chomsky1959certain}, $P'\colon V \rightarrow (V^2 \mid \Sigma)$, where every production is either (1) a binary production $w \rightarrow xz$, or (2) a unit production $w \rightarrow t$, where $w, x, z: V$ and $t: \Sigma$. For example:\vspace{-3pt}

\begin{table}[H]
  \begin{tabular}{llll}
    $G = \big\{\;S \rightarrow S\:S \mid (\:S\:) \mid (\:)\;\big\} \Longrightarrow G' = \big\{\;S\rightarrow Q\:R \mid S\:S \mid L\:R,$ & $R \rightarrow\:),$ & $L \rightarrow (,$ & $Q\rightarrow L\:S\;\big\}$
  \end{tabular}
\end{table}\vspace{-8pt}

Likewise, a finite state automaton (FSA) is a quintuple $\mathcal{A} = \langle Q, \Sigma, \delta, q_\alpha, F\rangle$, where $Q$ is a finite set of states, $\Sigma$ is a finite alphabet, $\delta \subseteq Q \times \Sigma \times Q$ is the transition function, $q_\alpha$ is the initial state, and $F \subseteq Q$ are the accepting states. These generally come in two varieties, deterministic and nondeterministic depending on whether or not $\delta$ maps each pair $\langle q, s \rangle$ to a unique $q'$.

There is an equivalent characterization of the regular languages via an inductively defined datatype, which is often more elegant than FSAs to work with. Consider the generalized regular expression (GRE) fragment containing concatenation, conjunction and disjunction:

\begin{definition}[Star-free GRE fragment]
  Let \( e: E \) be an expression defined by the grammar:
  \[
    e \rightarrow \varnothing \mid \varepsilon \mid \Sigma \mid e \cdot e \mid e \lor e \mid e \land e
  \]

where $\varepsilon$ is the empty symbol. Semantically, we interpret these expressions as denoting languages:\vspace{-0.8cm}

  \setlength{\columnseprule}{0pt}
  \setlength{\columnsep}{-3cm}
  \begin{multicols}{2}
    \begin{eqnarray*}
      \mathcal{L}(&\hspace{-0.35cm} \varnothing \hspace{-0.35cm}&) = \varnothing \\
      \mathcal{L}(&\hspace{-0.35cm} \varepsilon \hspace{-0.35cm}&) = \{\varepsilon\} \\
      \mathcal{L}(&\hspace{-0.35cm} a           \hspace{-0.35cm}&) = \{a\}
    \end{eqnarray*} \break\vspace{-0.45cm}
    \begin{eqnarray*}
      \mathcal{L}(&\hspace{-0.35cm} x\cdot z \hspace{-0.35cm}&) = \mathcal{L}(x) \circ \mathcal{L}(z)\text{\footnotemark}\\
      \mathcal{L}(&\hspace{-0.35cm} x\vee  z \hspace{-0.35cm}&) = \mathcal{L}(x) \cup  \mathcal{L}(z)\\
      \mathcal{L}(&\hspace{-0.35cm} x\land z \hspace{-0.35cm}&) = \mathcal{L}(x) \cap  \mathcal{L}(z)
    \end{eqnarray*}
  \end{multicols}
  \footnotetext{Where $\mathcal{L}(x)\circ\mathcal{L}(z)$ is defined as $\big\{a \cdot b \mid a \in \mathcal{L}(x) \land b \in \mathcal{L}(z) \big\}$.}
\end{definition}\vspace{-0.2cm}

\noindent Brzozowski~\cite{brzozowski1964derivatives} introduces an operator, $\partial: E \times \Sigma \rightarrow E$, which quotients a language by some prefix,

\begin{definition}[Brzozowski, 1964]
  To compute the quotient \(\partial_a(L) = \{b \mid ab \in L\}\), we:

  \vspace{-0.8cm}
  \begin{multicols}{2}
    \begin{eqnarray*}
      \phantom{--}\partial_a(&\hspace{-0.35cm} \varnothing \hspace{-0.35cm}&) = \varnothing                                           \\
      \phantom{--}\partial_a(&\hspace{-0.35cm} \varepsilon \hspace{-0.35cm}&) = \varnothing                                           \\
      \phantom{--}\partial_a(&\hspace{-0.35cm} b           \hspace{-0.35cm}&) = \begin{cases}\varepsilon &\text{ if } a = b\\ \varnothing &\text{ if } a \neq b \end{cases}\\
      \phantom{--}\partial_a(&\hspace{-0.35cm} x\cdot z    \hspace{-0.35cm}&) = (\partial_a x)\cdot z \vee \delta(x)\cdot\partial_a z \\
      \phantom{--}\partial_a(&\hspace{-0.35cm} x\vee  z    \hspace{-0.35cm}&) =  \partial_a x \vee  \partial_a z                       \\
      \phantom{--}\partial_a(&\hspace{-0.35cm} x\land z    \hspace{-0.35cm}&) =  \partial_a x \land \partial_a z
    \end{eqnarray*} \break\vspace{-0.45cm}
    \begin{eqnarray*}
      \delta(&\hspace{-0.35cm} \varnothing \hspace{-0.35cm}&) = \varnothing                                      \\
      \delta(&\hspace{-0.35cm} \varepsilon \hspace{-0.35cm}&) = \varepsilon                                      \\
      \delta(&\hspace{-0.35cm} a           \hspace{-0.35cm}&) = \varnothing\phantom{\begin{cases}\varepsilon\\\varnothing\end{cases}}\\
      \delta(&\hspace{-0.35cm} x\cdot z    \hspace{-0.35cm}&) = \delta(x) \land \delta(z)                        \\
      \delta(&\hspace{-0.35cm} x\vee  z    \hspace{-0.35cm}&) = \delta(x) \vee  \delta(z)                        \\
      \delta(&\hspace{-0.35cm} x\land z    \hspace{-0.35cm}&) = \delta(x) \land \delta(z)
    \end{eqnarray*}
  \end{multicols}
\end{definition}

Primarily, this gadget was designed to handle membership queries, for which purpose it has  found a number of applications~\cite{might2011parsing,stanford2021symbolic,varatalu2025re} in recent years:

\begin{theorem}[Recognition]
  For any regex \(e\) and \(\sigma: \Sigma^*\), \(\sigma \in \mathcal{L}(e) \Longleftrightarrow \varepsilon \in \mathcal{L}(\partial_\sigma e)\), where:

  \[
    \partial_\sigma (e): E \rightarrow E = \begin{cases}e &\text{ if } \sigma = \varepsilon\\\partial_b(\partial_a e) &\text{ if } \sigma = a \cdot b, a \in \Sigma, b \in \Sigma^* \end{cases}
  \]
\end{theorem}

Variations on this basic procedure can also be used for functional parsing and regular expression tasks. Less well known, perhaps, is that Brzozowski's derivative can also be used to decode witnesses. We will first focus on the nonempty disjunctive fragment, and define this process in two steps:

\begin{theorem}[Generation]\label{thm:generation}
  For any nonempty $(\varepsilon, \land)$-free regex, \(e\), to witness $\sigma \in \mathcal{L}(e)$:\\

  \hspace{1.6cm}$\texttt{follow}(e): E \rightarrow 2^\Sigma$ = \begin{cases}
   \{e\} &\text{ if } e \in \Sigma \\
   \texttt{follow}(x) &\text{ if } e = x \cdot z\\
   \texttt{follow}(x)\cup\texttt{follow}(z) &\text{ if } e = x \lor z
  \end{cases}\\\\

  \hspace{1.6cm}$\texttt{choose}(e): E \rightarrow \Sigma^+$ = \begin{cases}
   e &\text{ if } e \in \Sigma \\
   \big(s \stackrel{\$}{\gets} \texttt{follow}(e)\big)\cdot \texttt{choose}(\partial_s e) &\text{ if } e = x \cdot z\\
   \texttt{choose}\big(e' \stackrel{\$}{\gets} \{x, z\}\big) &\text{ if } e = x \lor z
  \end{cases}
\end{theorem}

Here, we use the $\stackrel{\$}{\gets}$ operator to denote probabilistic choice, however, any deterministic choice function will also suffice to generate a witness. Now we are equipped to handle conjunction.

Recall that every regular language is also context-free a fortiori. So, given an $(\varepsilon, \land)$-free regular expression, we can construct an equivalent CFG with productions $P(e)$ as follows:

\begin{equation}
P(e): E \rightarrow \big(V \rightarrow (\Sigma \mid V \mid V^2)\big) = \begin{cases}
 \{ S_e \rightarrow e \} & \text{if } e \in \Sigma \\
 P(x) \cup P(z) \cup \{ S_e \rightarrow S_x S_z \} & \text{if } e = x \cdot z \\
 P(x) \cup P(z) \cup \{ S_e \rightarrow S_x, S_e \rightarrow S_z \} & \text{if } e = x \lor z \\
\end{cases}
\end{equation}\vspace{0.2cm}

\noindent where the CFG is $G(e) = \langle V, \Sigma, P(e), S_e\rangle$ with $V$ being nonterminals in $P(e)$. Therefore, to intersect two regular languages, we can treat one of them as a CFL. Alternatively, we can take the intersection between some truly non-regular CFL (say, a programming language syntax) and a regular language.

\begin{theorem}[Bar-Hillel, 1961]
  For any CFG, $G = \langle V, \Sigma, P, S\rangle$, and nondeterministic finite automata (NFA), $A = \langle Q, \Sigma, \delta, q_\alpha, F\rangle$, there is a CFG, \(G_\cap=\langle V_\cap, \Sigma_\cap, P_\cap, S_\cap\rangle\) s.t. $\mathcal{L}(G_\cap) = \mathcal{L}(G)\cap\mathcal{L}(A)$.
\end{theorem}

\noindent Salomaa~\cite{salomaa1973formal} introduces a direct, but inefficient construction for the intersection grammar:

\begin{definition}[Salomaa, 1973]
  One could construct $G_\cap$ like so,

  \noindent\begin{prooftree}
      \hskip -0.5em
      \AxiomC{$q_\omega \in F\vphantom{\overset{a}{\rightarrow}}$}
      \RightLabel{$\mathcal{S}$}
      \UnaryInfC{$\big(S\rightarrow q_\alpha S q_\omega\big) \in P_\cap$}
      \DisplayProof
      \hskip 1em
      \AxiomC{$(w \rightarrow a) \in P$}
      \AxiomC{$(q\overset{a}{\rightarrow}r) \in \delta$}
      \RightLabel{$\uparrow$}
      \BinaryInfC{$\big(qwr\rightarrow a\big)\in P_\cap$}
      \DisplayProof
      \hskip 1em
      \AxiomC{$(w \rightarrow xz) \in P$}
      \AxiomC{$\vphantom{(}p,q,r \in Q\vphantom{\overset{a}{\rightarrow}}$}
      \RightLabel{$\Join$}
      \BinaryInfC{$\big(pwr\rightarrow (pxq)(qzr)\big) \in P_\cap$}
  \end{prooftree}
\end{definition}\vspace{0.2cm}

\noindent however, most synthetic productions in $P_\cap$ will be non-generating or unreachable. This method will construct a synthetic production for state pairs that are not even connected by any path, which is clearly excessive. In \S~\ref{sec:method}, we will present a far more efficient construction for the special case when the intersection is finite. But first, let us return to the broader question of syntax repair.

\subsection{Informal statement}

Assume there exists a transducer from Unicode tokens to grammatical tokens, $t: \Sigma_U^* \rightarrow \Sigma_G^*$. In the compiler nomenclature, $t$ is called a \textit{lexer} and would typically be regular under mild conditions. In this paper, we do not consider $t$ and strictly deal with languages over $\Sigma_G^*$, or simply $\Sigma^*$ for brevity.

Now suppose we have a syntax, $\ell \subset \Sigma^*$, containing every acceptable program. A syntax error is an unacceptable string, $\err\sigma \notin \ell$, that we wish to repair. We can model syntax repair as a language intersection between a context-free language (CFL) and a regular language. Henceforth, $\err\sigma$ will always and only be used to denote a syntactically invalid string whose intended language is known.

\begin{wrapfigure}{r}{0.4\textwidth}
\vspace{-0.3cm}
\resizebox{0.42\textwidth}{!}{
  \def\secondcirclepath{(1.15,0) coordinate (e) circle (2cm)}
  \begin{tikzpicture}[
    dot/.style = {circle, inner sep=0pt, minimum size=1mm, fill,
    node contents={}}
  ]
    \def\firstcircle{(-2.1,0) coordinate (a) circle (2.4cm)}
    \def\firstcirclea{(-2.1,0) coordinate (b) circle (0.6cm)}
    \def\firstcircleb{(-2.1,0) coordinate (c) circle (1.2cm)}
    \def\firstcirclec{(-2.1,0) coordinate (d) circle (1.8cm)}
    \def\secondcircle{(1.2,0) coordinate (e) circle (1.5cm)}

    \begin{scope}
      \clip[decorate, decoration={snake, amplitude=0.6mm, segment length=5.01mm}] \secondcirclepath;
      \fill[black!35] \firstcircle;
    \end{scope}

    \draw \firstcircle node[dot,label=$\err{\sigma}$](z0);
    \draw [dashed] \firstcirclea;
    \draw [dashed] \firstcircleb;
    \draw [dashed] \firstcirclec;
    \draw[-stealth] (-2.1,0) -- (-1.5, 0) node[midway,below]{$d_1$};
    \draw[-stealth] (-1.5,0) -- (-0.9, 0) node[midway,below]{$d_2$};
    \draw[-stealth] (-0.9,0) -- (-0.3, 0) node[midway,below]{$d_3$};
    \draw[-stealth] (-0.3,0) -- (0.3, 0) node[midway,above]{$\tilde{\sigma}$};
    \draw[-stealth] (-0.3,0) -- (0.3, 0) node[midway,below]{$d_4$};

    \draw[decorate, decoration={snake, amplitude=0.6mm, segment length=5.01mm}] \secondcirclepath;
    \node [above] at (current bounding box.north -| a) {$\mathcal{L}\bigl(L(\err\sigma, d^*)\bigr)$};
    \node [above,yshift=2.1cm] at (e) {$\mathcal{L}(G)$};
    \node [above,yshift=1.8cm, xshift=-1.4cm] at (e) {$\ell_\cap$};
  \end{tikzpicture}
}
\vspace{-0.3cm}
\caption{CFL intersection with the local edit region around a broken code snippet, where $d^*=3$ is the language edit distance (LED).}
\vspace{-0.3cm}
\end{wrapfigure}

Given a lexical representation of a broken computer program $\err\sigma$ and a grammar $G$, our goal is to find every valid string $\sigma$ consistent with the grammar $G$ and within a certain edit distance, $d$. Consider the language of nearby strings: if intersected with the language of grammatically valid programs, $\mathcal{L}(G)$, the result ($\ell_\cap$) will contain every possible repair within the given edit distance, a subset of which will be natural or statistically probable. If we can locate these repairs, then we can map them back into Unicode, adding placeholders for fresh names, numbers, and string literals, then finally apply an off-the-shelf code formatter to display them. Both the preprocessing and the cosmetic postprocessing steps are tangential to this work, in which we confine ourselves to a lexical alphabet.

\subsection{Formal statement}\label{sec:problem}

Let us now restate our informal description of the syntax repair problem in more formal terms.

\begin{definition}[Bounded Levenshtein-CFL reachability]\label{def:bcflr}
Given a CFL, $\ell$, and an invalid string, $\err{\sigma}: \bar\ell$, find every valid string reachable within $d$ edits of $\err{\sigma}$, i.e., letting $\Delta$ be the Levenshtein metric and $\mathcal{L}\big(L(\err\sigma, d)\big) = \{\sigma' \mid \Delta(\err{\sigma}, \sigma') \leq d\}$ be the Levenshtein $d$-ball, we seek to find $\ell_\cap = \mathcal{L}\big(L(\err\sigma, d)\big) \cap \ell$.
\end{definition}

As the admissible set $\ell_\cap$ is typically under-constrained, we want a procedure that surfaces natural and valid repairs over unnatural but valid repairs:

\begin{definition}[Ranked repair]\label{def:ranked-repair}
Given a finite language $\ell_\cap = \mathcal{L}\big(L(\err\sigma, d)\big) \cap \ell$ and a probabilistic language model $\text{P}_\theta: \Sigma^* \rightarrow [0, 1] \subset \mathbb{R}$, find the top-$k$ maximum probability repairs. That is,
\begin{equation}
R(\ell_\cap, P_\theta): 2^{\Sigma^*} \times (\Sigma^* \rightarrow \mathbb{R}) \rightarrow (\Sigma^*)^{\leq k} = \argmax_{\bm{\sigma} \subseteq \ell_\cap, |\bm{\sigma}| \leq k} \sum_{\sigma \in \bm{\sigma}}\text{P}_\theta(\sigma)
\end{equation}
\end{definition}

A popular approach to ranked repair involves learning a distribution over strings, however, this is highly sample-inefficient and generalizes poorly to new languages. Approximating a distribution over $\Sigma^*$ forces the model to jointly learn syntax and stylometry. Furthermore, even with an extremely efficient approximate sampler for $\sigma \sim \ell_\cap$, due to the size of the languages involved, it would be intractable to sample either $\ell$ or $\mathcal{L}\big(L(\err\sigma, d)\big)$, reject duplicates, then reject unreachable or invalid edits, and completely out of the question to sample $\sigma \sim \Sigma^*$ as do most neural language models.

As we will demonstrate, the ranked repair problem can be factorized into three subproblems: (1) exact representation, (2) decoding and (3) reranking. Instead of working with strings, we will explicitly construct a grammar which soundly and completely generates $\ell \cap \mathcal{L}\big(L(\err\sigma, d)\big)$, then decode repairs from its language. By ensuring decoding is sufficiently precise and extensive, ensuring the retrieved set contains the true repair can be achieved with a much simpler, syntax-oblivious model. Finally, we will train a language model to rerank the repair candidates and take the top-$k$ results.

\clearpage\section{Method}\label{sec:method}

The key to solving this problem is to treat finite language intersections as matrix exponentiation, exploiting a correspondence between the Bar-Hillel construction and CFL reachability. We show that if one of the participants in the language intersection is presented as an acyclic FSA, the finite intersection nonemptiness problem is polylogarithmic parallel time decidable. Formally,

\begin{theorem}\label{thm:parallel_decision_complexity}
  For any CFG, $G = \langle V, \Sigma, P, S\rangle$, and acyclic NFA (ANFA), $A = \langle Q, \Sigma, \delta, q_\alpha: Q, F \subseteq Q\rangle$, there exists a decision procedure $\Psi: \text{CFG} \rightarrow \text{ANFA} \rightarrow \mathbb{B}$ such that $\Psi(G, A) \models [\mathcal{L}(G)\cap\mathcal{L}(A) \neq \varnothing]$ requiring $\mathcal{O}\big(\log^2|Q|+\log|Q||V|\big)$ time using $\mathcal{O}\big(|Q|^2|V|\big)$ parallel random access (PRAM) processors.
\end{theorem}

\begin{proof}[Proof]
  To prove nonemptiness, we must show there exists a path $q_\alpha \rightsquigarrow q_\omega$ in $A$ such that $q_\omega: F$ where $q_\alpha \rightsquigarrow q_\omega \vdash S$. At least one of two cases must hold for $w \in V$ to parse a given $p \rightsquigarrow r$ pair:

  \begin{enumerate}
    \item $p$ steps directly to $r$ in which case it suffices to check $\exists a.\big((p \overset{a}{\rightarrow} r)\in \delta \land (w \rightarrow a) \in P\big)$, or,
    \item there is some midpoint $q \in Q$, $p \rightsquigarrow q \rightsquigarrow r$ such that $\exists x, z.\big((w \rightarrow xz) \in P\land\overbrace{\underbrace{p \rightsquigarrow q}_x, \underbrace{q \rightsquigarrow r}_z}^w\big)$.
  \end{enumerate}

  \noindent This decomposition immediately suggests a dynamic programming solution. Let M be a matrix of type $E^{|Q|\times|Q|\times|V|}$ indexed by $Q$. Since we assumed $\delta$ is acyclic, there exists a topological sort of $\delta$ imposing a total order on $Q$ such that $M$ is strictly upper triangular (SUT). Note $Q$ can be ordered topologically in $\mathcal{O}(\log^2 |Q|)$ time~\cite{dekel1981parallel} using matrix multiplication. We initialize $M$ thusly:
  \begin{align}
    M_0[r, c, w] = \bigvee_{a\in\Sigma} \big\{a \mid (w \rightarrow a) \in P \land (q_r \overset{a}{\rightarrow} q_c)\in \delta\big\}
  \end{align}

  Now, our goal will be to find $M=M^2$ such that $\big[M_0[r, c, w] \neq \varnothing\big] \implies \big[M[r, c, w] \neq \varnothing\big]$ under a certain near-semiring. The algebraic operations $\oplus, \otimes: E^{2|V|} \rightarrow E^{|V|}$ we will define elementwise:
  \begin{equation}
    [\ell \oplus r]_w  = [\ell_w \lor r_w]\hspace{0.5cm}\text{and}\hspace{0.5cm}
    [\ell \otimes r]_w = \bigvee_{\mathclap{x, z\:\in\:V}}\big\{\ell_x \cdot r_z \mid (w \rightarrow xz) \in P\big\}.
  \end{equation}
  By slight abuse of notation,\footnote{Customarily, there is a $\frac{1}{k!}$ factor to modulate exploding values, but alas this domain has no multiplicative inverse.} we will redefine the matrix exponential over this domain as,
  \begin{align}
    \exp(M) &= \sum_{i = 0}^\infty M_0^i = \sum_{i = 0}^{\mathclap{|Q||V|}} M_0^i \text { (since $M_0$ is SUT and thus nilpotent).}
  \end{align}
  While $|Q||V|$ is an upper-bound and $\exp(M)$ may converge sooner, incremental evaluation grows expensive even with unbounded parallelism. Instead, we will use exponentiation-by-squaring:
  \begin{align}
    \sum_{i = 0}^{2n} M_0^i = T(2n) \;=\; \begin{cases}
       M_0, & \text{if } n = 1,\\
       T(n) + T(n)^2 & \text{otherwise}.
    \end{cases}
  \end{align}
  Therefore, the complexity can be reduced to at most $\lceil\log_2 |Q||V|\rceil$ sequential steps in the limit. Finally, we will union all the languages of every state pair deriving $S$ into a new nonterminal, $S_\cap$.
  \begin{align}
    S_\cap = \bigvee_{\mathclap{\:q_\omega \in F}}\exp(M)[q_\alpha, q_\omega, S] \text{, and } \Psi = [S_\cap \neq \varnothing].
  \end{align}
  Note that it is possible to check $\Psi$ before each recurrence of $T$ and escape immediately thereafter in the positive case. Optimistically, this can occur in $\Omega(\log_2 p^*)$ time, where $p^*$ is the length of the shortest path through $\delta$, $p^*=\min_{q_\omega \in F}|q_\alpha\rightsquigarrow q_\omega|$. In case of nonemptiness, one may simply $\texttt{choose}(S_\cap)$ (see Theorem~\ref{thm:generation}) to decode a witness $\sigma \in \mathcal{L}(G)\cap\mathcal{L}(A)$. In either case, the algorithm terminates in $\mathcal{O}(\log^2 |Q| + \log |Q||V|)$ parallel time with $\mathcal{O}(|Q|^2|V|)$ processors.
\end{proof}\clearpage

\section{Examples}

In this section, we will consider three examples of intersections with finite languages. First, parsing can be viewed as a special case of intersection with a singleton language. Second, we will introduce \textit{completion} as an intersection that admits terminal wildcards in fixed locations. Thirdly, we consider syntax repair, where we will intersect a language representing all possible edit paths within a certain distance to determine the location(s) and fill them with the appropriate terminal(s).

\subsection{Recognition as intersection}

In the case of ordinary CFL recognition, the automaton forms a single row and accepts one word:

\begin{figure}[H]
\resizebox{0.5\textwidth}{!}{
  \begin{tikzpicture}[>=stealth', node distance=2.5cm, initial text=$ $]
    \node[state, initial]         (00) {$q_{0,0}$};
    \node[state, right of=00]     (10) {$q_{1,0}$};
    \node[state, right of=10, draw=none]     (20) {$\ldots$};
    \node[state, accepting, right of=20] (30) {$q_{n,0}$};

    \draw [->] (00) edge[below] node{$\sigma_1$} (10);
    \draw [->] (10) edge[below] node{$\sigma_2$} (20);
    \draw [->] (20) edge[below] node{$\sigma_n$} (30);
  \end{tikzpicture}
}
\end{figure}

Since the word is predetermined, we just need to keep track of nonterminal subsets for each substring. So, given a CFG, $G' : \mathcal{G}$ in Chomsky Normal Form (CNF), we can construct a recognizer for strings $\sigma: \Sigma^n$ as follows. Let $2^V$ be our domain, $0$ be $\varnothing$, $\oplus$ be $\cup$, and $\otimes$ be defined as:\vspace{-10pt}

\begin{align}
  X \otimes Z = \big\{\;w \mid \langle x, z\rangle \in X \times Z, (w\rightarrow xz) \in P\;\big\}
\end{align}

\noindent If we define $\hat\sigma_r = \{w \mid (w \rightarrow \sigma_r) \in P\}$, then construct a matrix with nonterminals on the superdiagonal representing each token, $M_0[r+1=c](G', \sigma) = \;\hat\sigma_r$, the fixpoint $M_{i+1} = M_i + M_i^2$ is uniquely determined by the superdiagonal entries. Omitting the exponentiation-by-squaring detail, the ordinary fixedpoint iteration simply fills successive diagonals:\vspace{-10pt}

\begin{align*}
  M_0=
  \begin{pNiceMatrix}[nullify-dots,xdots/line-style=loosely dotted]
    \varnothing & \hat\sigma_1 & \varnothing & \Cdots & \varnothing  \\
    \Vdots      & \Ddots       & \Ddots      & \Ddots & \Vdots       \\
                &              &             &        & \varnothing  \\
                &              &             &        & \hat\sigma_n \\
    \varnothing & \Cdots       &             &        & \varnothing
  \end{pNiceMatrix} & \,,\, M_1=
  \begin{pNiceMatrix}[nullify-dots,xdots/line-style=loosely dotted]
    \varnothing & \hat\sigma_1 & \Lambda     & \Cdots & \varnothing  \\
    \Vdots      & \Ddots       & \Ddots      & \Ddots & \Vdots       \\
                &              &             &        & \Lambda      \\
                &              &             &        & \hat\sigma_n \\
    \varnothing & \Cdots       &             &        & \varnothing
  \end{pNiceMatrix}\,,\,\ldots & \hspace{-0.6cm}\,,\,M_\infty =
  \begin{pNiceMatrix}[nullify-dots,xdots/line-style=loosely dotted]
    \varnothing & \hat\sigma_1 & \Lambda     & \Cdots & \Lambda^*_\sigma \\
    \Vdots      & \Ddots       & \Ddots      & \Ddots & \Vdots           \\
                &              &             &        & \Lambda          \\
                &              &             &        & \hat\sigma_n     \\
    \varnothing & \Cdots       &             &        & \varnothing
  \end{pNiceMatrix}
\end{align*}

Once the fixpoint $M_\infty$ is attained, the proposition $[S \in \Lambda^*_\sigma]$~\footnote{Hereinafter, we use Iverson brackets to denote the indicator function of a predicate with free variables, i.e., $[P] \Leftrightarrow \mathds{1}(P)$.} decides language membership, i.e., $[\sigma \in \mathcal{L}(G)]$. So far, this procedure is essentially the textbook CYK algorithm in a linear algebraic notation~\cite{goodman1999semiring} and a well-established technique in the parsing literature~\cite{Grune2008}.

\subsection{Completion as intersection}

Let us now consider a problem of intermediate difficulty, wherein we are given a string template admitting edits at fixed locations, which can be filled by any terminal. When intersected with a CFL, this specifies a finite language whose contents are the set of all words consistent with the template. This problem we call \textit{completion}. Formally,

\begin{definition}[Completion]
  Let $\underline\Sigma = \Sigma \cup \{\_\}$, where $\_$ denotes a hole. We denote $\sqsubseteq: \Sigma^n \times \underline\Sigma^n$ as the relation $\{\langle\sigma', \sigma\rangle \mid \sigma_i \in \Sigma \implies \sigma_i' = \sigma_i\}$ and the set of all inhabitants $\{\sigma': \Sigma^+ \mid \sigma' \sqsubseteq \sigma\}$ as $\text{H}(\sigma)$. Given a \textit{porous string}, $\sigma: \underline\Sigma^*$ we seek all syntactically valid inhabitants, i.e., $A(\sigma)=\text{H}(\sigma)\cap\ell$.
\end{definition}

Here, the FSA takes a similar shape but can have multiple arcs between adjacent states, e.g.:

\begin{figure}[H]
  \resizebox{0.5\textwidth}{!}{
    \begin{tikzpicture}[>=stealth', node distance=2.5cm, initial text=$ $]
      \node[state, initial]                (00) {$q_{0,0}$};
      \node[state, right of=00]            (10) {$q_{1,0}$};
      \node[state, right of=10]            (20) {$q_{2,0}$};
      \node[state, accepting, right of=20] (30) {$q_{3,0}$};

      \draw [->] (00) edge[below]             node{$\sigma_1$} (10);
      \draw [->] (10) edge[below]             node{$\ldots$}   (20);
      \draw [->] (10) edge[below, bend left]  node{$\Sigma_1$} (20);
      \draw [->] (10) edge[below, bend right] node{$\Sigma_n$} (20);
      \draw [->] (20) edge[below]             node{$\ldots$}   (30);
      \draw [->] (20) edge[below, bend left]  node{$\Sigma_1$} (30);
      \draw [->] (20) edge[below, bend right] node{$\Sigma_n$} (30);
    \end{tikzpicture}
  }
\end{figure}

\noindent This corresponds to a template with two holes, $\sigma = 1$ \_ \_. Suppose the context-free grammar is $G=\{S\rightarrow N O N, O \rightarrow + \mid \times, N \rightarrow 0 \mid 1\}$. This grammar will first be rewritten into CNF as $G'= \{S \rightarrow N L, N \rightarrow 0 \mid 1, O \rightarrow \times \mid +, L \rightarrow O N\}$. Using the powerset algebra we just defined, the matrix fixpoint $M' = M + M^2$ can be computed as follows, shown in the leftmost column below:\vspace{0.3cm}

\begin{small}
{\renewcommand{\arraystretch}{1.2}
\noindent\phantom{...}\begin{tabular}{|c|c|c|c|}
  \hline
  & $2^V$ & $\mathbb{Z}_2^{|V|}$ & GRE$^{|V|}$\\\hline
  $M_0$ & \begin{pmatrix}
            \phantom{V} & \tiny{\{N\}} &              &              \\
                        &              & \{N,O\}      &              \\
                        &              &              & \{N,O\}      \\
                        &              &              &
  \end{pmatrix} & \begin{pmatrix}
            \phantom{V} & \overset{L}{\ws}\overset{N}{\bs}\overset{O}{\ws}\overset{S}{\ws} &              &              \\
                        &              & \ws\bs\bs\ws &              \\
                        &              &              & \ws\bs\bs\ws \\
                        &              &              &
  \end{pmatrix} & \begin{pmatrix}
            \phantom{V} & E_{0, 1}     &              &              \\
                        &              & E_{1, 2}     &              \\
                        &              &              & E_{2, 3}     \\
                        &              &              &
  \end{pmatrix} \\\hline
  $M_1$ & \begin{pmatrix}
            \phantom{V} & \tiny{\{N\}} & \varnothing  &              \\
                        &              & \{N,O\}      & \{L\}        \\
                        &              &              & \{N,O\}      \\
                        &              &              &
  \end{pmatrix} & \begin{pmatrix}
            \phantom{V} & \ws\bs\ws\ws & \ws\ws\ws\ws &              \\
                        &              & \ws\bs\bs\ws & \bs\ws\ws\ws \\
                        &              &              & \ws\bs\bs\ws \\
                        &              &              &
  \end{pmatrix} & \begin{pmatrix}
            \phantom{V} & E_{0, 1}     & E_{0, 2}     &              \\
                        &              & E_{1, 2}     & E_{1, 3}     \\
                        &              &              & E_{2, 3}     \\
                        &              &              &
  \end{pmatrix} \\\hline
  \begin{tabular}{@{}c@{}}$M_2$\\$=$\\$M_\infty$\end{tabular} & \begin{pmatrix}
            \phantom{V} & \tiny{\{N\}} & \varnothing  & \{S\}        \\
                        &              & \{N,O\}      & \{L\}        \\
                        &              &              & \{N,O\}      \\
                        &              &              &
  \end{pmatrix} & \begin{pmatrix}
            \phantom{V} & \ws\bs\ws\ws & \ws\ws\ws\ws & \ws\ws\ws\bs \\
                        &              & \ws\bs\bs\ws & \bs\ws\ws\ws \\
                        &              &              & \ws\bs\bs\ws \\
                        &              &              &
  \end{pmatrix} & \begin{pmatrix}
            \phantom{V} & E_{0, 1}     & E_{0, 2}     & E_{0, 3}     \\
                        &              & E_{1, 2}     & E_{1, 3}     \\
                        &              &              & E_{2, 3}     \\
                        &              &              &
  \end{pmatrix} \\\hline
\end{tabular}\\
}
\end{small}

\vspace{8pt}The same procedure can be translated, without loss of generality, into the bit domain ($\mathbb{Z}_2^{|V|}$) using a lexicographic nonterminal ordering, however $M_\infty$ in both $2^V$ and $\mathbb{Z}_2^{|V|}$ represents a decision procedure, i.e., $\big[S\in M_\infty[0, 3]\big]\Leftrightarrow \big[M_\infty[0, 3, 3]=\bs\big] \Leftrightarrow \big[A(\sigma) \neq \varnothing\big]$. Since $M_\infty[0, 3] = \{S\}$, we know there is at least one $\sigma' \in A(\sigma)$, but neither $M_\infty$ in $2^V$ or $\mathbb{Z}_2^V$ lets us recover a witness.

%$\{\text{xor}, \land, \top\}$ is a functionally complete set is equivalent to $\mathbb{Z}_2$ $\top := 1, \land := \times, \text{xor} := +$. We can define $=$ as $(a = b) \Leftrightarrow (a \text{ xor } b) \text{ xor } \top \Leftrightarrow (a + b) + \top$.

To witness $\sigma' \in A(\sigma)$, we can translate the matrix exponential to the GRE domain. We first define $X \otimes Z = [X_2 \cdot Z_1, \varnothing, \varnothing, X_1 \cdot Z_0]$ and $X \oplus Z = [X_i \lor Z_i]_{i \in [0, |V|)}$, mirroring $\oplus, \otimes$ from the powerset domain. Since the unit nonterminals $O, N$ can only occur on the superdiagonal, they may be safely ignored by $\otimes$. To solve for $M_\infty$, we proceed by first computing $E_{0, 2}, E_{1, 3}$:\vspace{-8pt}

\begin{small}
\begin{align*}
  E_{0, 2} &= E_{0, j} \cdot E_{j, 2} = E_{0, 1} \otimes E_{1, 2}                         &  E_{1, 3} &= E_{1, j} \cdot E_{j, 3} = E_{1, 2} \otimes E_{2, 3}\\
  &= [L \in E_{0, 2}, \varnothing, \varnothing, S \in E_{0, 2}]                                           &  &= [L \in E_{1, 3}, \varnothing, \varnothing, S \in E_{1, 3}]\\
  &= [O \in E_{0, 1} \cdot N \in E_{1, 2}, \varnothing, \varnothing, N \in E_{0, 1} \cdot L \in E_{1, 2}] &  &= [O \in E_{1, 2} \cdot N \in E_{2, 3}, \varnothing, \varnothing, N \in E_{1, 2} \cdot L \in E_{2, 3}]\\
  &= [E_{0, 1, 2} \cdot E_{1, 2, 1}, \varnothing, \varnothing, E_{0, 1, 1} \cdot E_{1, 2, 0}]             &  &= [E_{1, 2, 2} \cdot E_{2, 3, 1}, \varnothing, \varnothing, E_{1, 2, 1} \cdot E_{2, 3, 0}]
\end{align*}
\end{small}\vspace{-8pt}

\noindent Now we solve for the corner entry $E_{0, 3}$ by dotting the first row and last column, which yields:\vspace{-8pt}

\begin{align*}
  E_{0, 3} &= E_{0, j} \cdot E_{j, 3} = (E_{0, 1} \otimes E_{1, 3}) \oplus (E_{0, 2} \otimes E_{2, 3})\\
%  &= [E_{0, 1, 2} \cdot E_{1, 3, 1}, \varnothing, \varnothing, E_{0, 1, 1} \cdot E_{1, 3, 0}] + [E_{0, 2, 2} \cdot E_{2, 3, 1}, \varnothing, \varnothing, E_{0, 2, 1} \cdot E_{2, 3, 0}]\\
  &= [E_{0, 1, 2} \cdot E_{1, 3, 1} \lor E_{0, 2, 2} \cdot E_{2, 3, 1}, \varnothing, \varnothing, E_{0, 1, 1} \cdot E_{1, 3, 0} \lor E_{0, 2, 1} \cdot E_{2, 3, 0}]
\end{align*}

\noindent Since we only care about $E_{0, 3, 3} \Leftrightarrow [S \in E_{0, 3}]$, we can ignore the first three entries and solve for:\vspace{-8pt}

\begin{align*}
  E_{0, 3, 3} &= E_{0, 1, 1} \cdot E_{1, 3, 0} \lor E_{0, 2, 1} \cdot E_{2, 3, 0}\\
  &= E_{0, 1, 1} \cdot (E_{1, 2, 2} \cdot E_{2, 3, 1}) \lor E_{0, 2, 1} \cdot \varnothing\\
  &= E_{0, 1, 1} \cdot E_{1, 2, 2} \cdot E_{2, 3, 1} \big(= [N \in E_{0, 1}] \cdot [O \in E_{1, 2}] \cdot [N \in E_{2, 3}]\big)\\
  &= 1 \cdot \{+, \times\} \cdot \{0, 1\}
\end{align*}

\noindent Finally, to recover a witness, we can simply $\texttt{choose}\big(1 \cdot \{+, \times\} \cdot \{0, 1\}\big)$.

\subsection{Repair as intersection}\label{sec:repair_ex}

Now, we are ready to consider the general case of syntax repair, in which case the edit locations are not localized but can occur anywhere inside the snippet. In this case, we construct a lattice of all possible edit paths up to a fixed distance. This structure is called a Levenshtein automaton.

\begin{wrapfigure}{r}{0.5\textwidth}
  \vspace{-0.3cm}
  \begin{center}
    \begin{minipage}[c]{0.45\textwidth}
  \centering
  \underline{NIA}\vspace{10pt}
  \resizebox{\textwidth}{!}{
  \begin{tikzpicture}[
%->, % makes the edges directed
  >=stealth',
  node distance=2.5cm, % specifies the minimum distance between two nodes. Change if necessary.
%  every state/.style={thick, fill=gray!10}, % sets the properties for each ’state’ node
  initial text=$ $, % sets the text that appears on the start arrow
  ]
  \node[state, initial]                (00) {$q_{0,0}$};
  \node[state, right of=00]            (10) {$q_{1,0}$};
  \node[state, right of=10]            (20) {$q_{2,0}$};
  \node[state, right of=20]            (30) {$q_{3,0}$};
  \node[right of=30]                   (40) {$\vphantom{\vdots}\cdots$};
  \node[accepting, state, right of=40] (n0) {$q_{n,0}$};

  \node[state, above of=00]            (01) {$q_{0,1}$};
  \node[state, right of=01]            (11) {$q_{1,1}$};
  \node[state, right of=11]            (21) {$q_{2,1}$};
  \node[state, right of=21]            (31) {$q_{3,1}$};
  \node[right of=31]                   (41) {$\vphantom{\vdots}\cdots$};
  \node[accepting, state, right of=41] (n1) {$q_{n,1}$};

  \node[above of=01]                   (0j) {$\mathmakebox[\widthof{$\cdots$}]{\vdots}$};
\node[right of=0j]                   (1j) {$\mathmakebox[\widthof{$\cdots$}]{\vdots}$};
\node[right of=1j]                   (2j) {$\mathmakebox[\widthof{$\cdots$}]{\vdots}$};
\node[right of=2j]                   (3j) {$\mathmakebox[\widthof{$\cdots$}]{\vdots}$};
\node[right of=3j]                   (4j) {$\iddots$};
\node[accepting, right of=4j]        (nj) {$\mathmakebox[\widthof{$\cdots$}]{\vdots}$};

\node[state, above of=0j]            (0k) {$q_{0,k}$};
\node[state, right of=0k]            (1k) {$q_{1,k}$};
\node[state, right of=1k]            (2k) {$q_{2,k}$};
\node[state, right of=2k]            (3k) {$q_{3,k}$};
\node[right of=3k]                   (4k) {$\vphantom{\vdots}\cdots$};
\node[accepting, state, right of=4k] (nk) {$q_{n,k}$};

\draw [->] (00) edge[below] node{$\sigma_1$} (10);
\draw [->] (10) edge[below] node{$\sigma_2$} (20);
\draw [->] (20) edge[below] node{$\sigma_3$} (30);
\draw [->] (30) edge[below] node{$\sigma_4$} (40);
\draw [->] (40) edge[below] node{$\sigma_n$} (n0);

\draw [->] (01) edge[below] node{$\sigma_1$} (11);
\draw [->] (11) edge[below] node{$\sigma_2$} (21);
\draw [->] (21) edge[below] node{$\sigma_3$} (31);
\draw [->] (31) edge[below] node{$\sigma_4$} (41);
\draw [->] (41) edge[below] node{$\sigma_n$} (n1);

\draw [->] (0j) edge[below] node{$\sigma_1$} (1j);
\draw [->] (1j) edge[below] node{$\sigma_2$} (2j);
\draw [->] (2j) edge[below] node{$\sigma_3$} (3j);
\draw [->] (3j) edge[below] node{$\sigma_4$} (4j);
\draw [->] (4j) edge[below] node{$\sigma_n$} (nj);

\draw [->] (0k) edge[below] node{$\sigma_1$} (1k);
\draw [->] (1k) edge[below] node{$\sigma_2$} (2k);
\draw [->] (2k) edge[below] node{$\sigma_3$} (3k);
\draw [->] (3k) edge[below] node{$\sigma_4$} (4k);
\draw [->] (4k) edge[below] node{$\sigma_n$} (nk);

\draw [->] (00) edge[left] node{$*$}         (11);
\draw [->] (10) edge[left] node{$*$}         (21);
\draw [->] (20) edge[left] node{$*$}         (31);
\draw [->] (30) edge[left] node{$*$}         (41);
\draw [->] (30) edge[bend right, below] node{$\sigma_5$} (41);
\draw [->] (40) edge[           right] node{$\sigma_n$}  (n1);
\draw [->] (40) edge[bend right, left] node{$*$}         (n1);

\draw [->] (01) edge[left] node{$*$}                     (1j);
\draw [->] (11) edge[left] node{$*$}                     (2j);
\draw [->] (21) edge[left] node{$*$}                     (3j);
\draw [->] (31) edge[left] node{$*$}                     (4j);
\draw [->] (31) edge[bend right, below] node{$\sigma_5$} (4j);
\draw [->] (41) edge[           right] node{$\sigma_n$}  (nj);
\draw [->] (41) edge[bend right, left] node{$*$}         (nj);

\draw [->] (0j) edge[left] node{$*$}                     (1k);
\draw [->] (1j) edge[left] node{$*$}                     (2k);
\draw [->] (2j) edge[left] node{$*$}                     (3k);
\draw [->] (3j) edge[left] node{$*$}                     (4k);
\draw [->] (3j) edge[bend right, below] node{$\sigma_5$} (4k);
\draw [->] (4j) edge[           right] node{$\sigma_n$}  (nk);
\draw [->] (4j) edge[bend right, left] node{$*$}         (nk);

\draw [->] (00) edge[bend left, left] node{$*$}   (01);
\draw [->] (10) edge[bend left, left] node{$*$}   (11);
\draw [->] (20) edge[bend left, left] node{$*$}   (21);
\draw [->] (30) edge[bend left, left] node{$*$}   (31);
\draw [->] (40) edge[right] node{$*$}             (41);
\draw [->] (n0) edge[bend right, right] node{$*$} (n1);

\draw [->] (01) edge[bend left, left] node{$*$}   (0j);
\draw [->] (11) edge[bend left, left] node{$*$}   (1j);
\draw [->] (21) edge[bend left, left] node{$*$}   (2j);
\draw [->] (31) edge[bend left, left] node{$*$}   (3j);
\draw [->] (41) edge[right] node{$*$}             (4j);
\draw [->] (n1) edge[bend right, right] node{$*$} (nj);

\draw [->] (0j) edge[bend left, left] node{$*$}   (0k);
\draw [->] (1j) edge[bend left, left] node{$*$}   (1k);
\draw [->] (2j) edge[bend left, left] node{$*$}   (2k);
\draw [->] (3j) edge[bend left, left] node{$*$}   (3k);
\draw [->] (4j) edge[right] node{$*$}             (4k);
\draw [->] (nj) edge[bend right, right] node{$*$} (nk);

\draw [->] (00) edge[below] node{$\sigma_2$}    (21);
\draw [->] (10) edge[below] node{$\sigma_3$}    (31);
\draw [->] (20) edge[below] node{$\sigma_4$}    (41);

\draw [->] (01) edge[below] node{$\sigma_2$}    (2j);
\draw [->] (11) edge[below] node{$\sigma_3$}    (3j);
\draw [->] (21) edge[below] node{$\sigma_4$}    (4j);

\draw [->] (0j) edge[below] node{$\sigma_2$}    (2k);
\draw [->] (1j) edge[below] node{$\sigma_3$}    (3k);
\draw [->] (2j) edge[below] node{$\sigma_4$}    (4k);

%https://tex.stackexchange.com/a/20986/139648
\draw [decorate,decoration={brace,amplitude=10pt,raise=10pt,mirror}] (00.south west) -- (n0.south east) node[midway,yshift=-3em]{\textbf{String length}};
\draw [decorate,decoration={brace,amplitude=10pt,raise=20pt}] (00.south west) -- (0k.north west) node[midway,xshift=-40pt,rotate=90]{\textbf{Levenshtein edit distance}};
\end{tikzpicture}
}
\end{minipage}
\hfill
\begin{minipage}[l]{5 cm}
\centering
\underline{CFG}\vspace{7pt}
\begin{align*}
S &\Rightarrow \{\cdot \in Q \mid \delta(\cdot, q_{n,0}) \leq k\}\\
* &\Rightarrow \{\cdot \in \Sigma\}\\
\big\{q_{i, j} &\Rightarrow \{q_{i, j-1}*\} \mid i, j \in [1, n]\times[1, k]\big\}\\
\big\{q_{i, j} &\Rightarrow \{q_{i-1, j-1}*\}\mid i, j\in[1, n]\times [1, k]\big\}\\
\big\{q_{i, j} &\Rightarrow \{q_{i-1, j} \sigma_i \}\mid i, j \in [1, n]\times[0, k]\big\} \\
\big\{q_{i, j} &\Rightarrow \{q_{i-2, j-1} \sigma_i\} \mid i, j \in [2, n]\times[1, k] \big\}\\
\end{align*}
\end{minipage}
  \end{center}
  \caption{Levenshtein NFA recognizing $\mathcal{L}\big(L(\sigma: \Sigma^5, 3)\big)$.}\label{fig:lev_nfa}
  \vspace{-0.5cm}
\end{wrapfigure}

As the original construction defined by Schultz and Mihov~\cite{schulz2002fast} contains cycles and $\varepsilon$-transitions, we propose a variant which is $\varepsilon$-free and acyclic. Furthermore, we adopt a nominal form that supports infinite alphabets and simplifies the description to follow. Illustrated in Fig.~\ref{fig:lev_nfa} is an example of a small Levenshtein automaton recognizing $\mathcal{L}\big(L(\sigma: \Sigma^5, 3)\big)$. Unlabeled arcs accept any terminal from the alphabet, $\Sigma$. Equivalently, this transition system can be viewed as a kind of proof system within an unlabeled lattice. The following construction is equivalent to Schultz and Mihov's original Levenshtein automaton, but is more amenable to our purposes as it does not contain any $\varepsilon$-arcs, and instead uses skip connections to recognize consecutive deletions of varying lengths.

\begin{prooftree}
  \AxiomC{$s\in\Sigma \phantom{\land} i \in [0, n] \phantom{\land} j \in [1, d_{\max}]$}
  \RightLabel{$\duparrow$}
  \UnaryInfC{$(q_{i, j-1} \overset{s}{\rightarrow} q_{i,j}) \in \delta$}
  \DisplayProof
  \hskip 1.5em
  \AxiomC{$s\in\Sigma \phantom{\land} i \in [1, n] \phantom{\land} j \in [1, d_{\max}]$}
  \RightLabel{$\ddiagarrow$}
  \UnaryInfC{$(q_{i-1, j-1} \overset{s}{\rightarrow} q_{i,j}) \in \delta$}
\end{prooftree}
\begin{prooftree}
  \AxiomC{$i \in [1, n] \phantom{\land} j \in [0, d_{\max}]$}
  \RightLabel{$\drightarrow$}
  \UnaryInfC{$(q_{i-1, j} \overset{\sigma_i}{\rightarrow} q_{i,j}) \in \delta$}
  \DisplayProof
  \hskip 1.5em
  \AxiomC{$d \in [1, d_{\max}] \phantom{\land} i \in [d + 1, n] \phantom{\land} j \in [d, d_{\max}]$}
  \RightLabel{$\knightarrow$}
  \UnaryInfC{$(q_{i-d-1, j-d} \overset{\sigma_i}{\rightarrow} q_{i,j}) \in \delta$}
\end{prooftree}
\begin{prooftree}
  \AxiomC{$\vphantom{|}$}
  \RightLabel{$\textsc{Init}$}
  \UnaryInfC{$q_{0,0} \in I$}
  \DisplayProof
  \hskip 1.5em
  \AxiomC{$q_{i, j} \in Q$}
  \AxiomC{$|n-i+j| \leq d_{\max}$}
  \RightLabel{$\textsc{Done}$}
  \BinaryInfC{$q_{i, j}\in F$}
\end{prooftree}

\newcommand{\substitutionExample}{
  \tikz{
    \foreach \x in {0,8,16,24,32,40}{
      \fill (\x pt,0pt) circle [radius = 1pt];
      \fill (\x pt,8pt) circle [radius = 1pt];
    }
    \phantom{\fill (0pt,-8pt) circle [radius = 1pt];}
    \draw [-to] (0pt,0pt) -- (8pt,0pt);
    \draw [-to] (8pt,0pt) -- (16pt,0pt);
    \draw [-to] (16pt,0pt) -- (24pt,8pt);
    \draw [-to] (24pt,8pt) -- (32pt,8pt);
    \draw [-to] (32pt,8pt) -- (40pt,8pt);
  }
}

\newcommand{\insertionExample}{
  \tikz{
    \foreach \x in {0,8,16,24,32,40}{
      \fill (\x pt,0pt) circle [radius = 1pt];
      \fill (\x pt,8pt) circle [radius = 1pt];
    }
    \phantom{\fill (0pt,-8pt) circle [radius = 1pt];}
    \fill[white] (16pt,0pt) circle [radius = 1.2pt];
    \fill[white] (24pt,8pt) circle [radius = 1.2pt];
    \draw [-to] (0pt,0pt) -- (8pt,0pt);
    \draw [-to] (8pt,0pt) -- (24pt,0pt);
    \draw [-to] (24pt,0pt) -- (16pt,8pt);
    \draw [-to] (16pt,8pt) -- (32pt,8pt);
    \draw [-to] (32pt,8pt) -- (40pt,8pt);
  }
}

\newcommand{\deletionExample}{
  \tikz{
    \foreach \x in {0,8,16,24,32,40}{
      \fill (\x pt,0pt) circle [radius = 1pt];
      \fill (\x pt,8pt) circle [radius = 1pt];
    }
    \phantom{\fill (0pt,-8pt) circle [radius = 1pt];}
    \draw [-to] (0pt,0pt) -- (8pt,0pt);
    \draw [-to] (8pt,0pt) -- (16pt,0pt);
    \draw [-to] (16pt,0pt) -- (24pt,0pt);
    \draw [-to] (24pt,0pt) -- (40pt,8pt);
  }
}

\newcommand{\doubleDeletionExample}{
  \tikz{
    \foreach \x in {0,8,16,24,32,40}{
      \fill (\x pt,0pt) circle [radius = 1pt];
      \fill (\x pt,8pt) circle [radius = 1pt];
      \fill (\x pt,16pt) circle [radius = 1pt];
    }
    \draw [-to] (0pt,0pt) -- (24pt,16pt);
    \draw [-to] (24pt,16pt) -- (32pt,16pt);
    \draw [-to] (32pt,16pt) -- (40pt,16pt);
  }
}

\newcommand{\subDelExample}{
  \tikz{
    \foreach \x in {0,8,16,24,32,40}{
      \fill (\x pt,0pt) circle [radius = 1pt];
      \fill (\x pt,8pt) circle [radius = 1pt];
      \fill (\x pt,16pt) circle [radius = 1pt];
    }
    \draw [-to] (0pt,0pt) -- (8pt,0pt);
    \draw [-to] (8pt,0pt) -- (16pt,8pt);
    \draw [-to] (16pt,8pt) -- (32pt,16pt);
    \draw [-to] (32pt,16pt) -- (40pt,16pt);
  }
}

\newcommand{\subSubExample}{
  \tikz{
    \foreach \x in {0,8,16,24,32,40}{
      \fill (\x pt,0pt) circle [radius = 1pt];
      \fill (\x pt,8pt) circle [radius = 1pt];
      \fill (\x pt,16pt) circle [radius = 1pt];
    }
    \draw [-to] (0pt,0pt) -- (8pt,0pt);
    \draw [-to] (8pt,0pt) -- (16pt,8pt);
    \draw [-to] (16pt,8pt) -- (24pt,16pt);
    \draw [-to] (24pt,16pt) -- (32pt,16pt);
    \draw [-to] (32pt,16pt) -- (40pt,16pt);
  }
}

\newcommand{\insertDeleteExample}{
  \tikz{
    \foreach \x in {0,8,16,24,32,40,48}{
      \fill (\x pt,0pt) circle [radius = 1pt];
      \fill (\x pt,8pt) circle [radius = 1pt];
      \fill (\x pt,16pt) circle [radius = 1pt];
    }
    \fill[white] (16pt,16pt) circle [radius = 1.2pt];
    \fill[white] (8pt,0pt) circle [radius = 1.2pt];
    \fill[white] (16pt,8pt) circle [radius = 1.2pt];
    \draw [-to] (0pt,0pt) -- (16pt,0pt);
    \draw [-to] (16pt,0pt) -- (8pt,8pt);
    \draw [-to] (8pt,8pt) -- (24pt,8pt);
    \draw [-to] (24pt,8pt) -- (40pt,16pt);
    \draw [-to] (40pt,16pt) -- (48pt,16pt);
  }
}

Each type of arc plays a specific role. $\duparrow$ handles insertions, $\ddiagarrow$ handles substitutions and $\knightarrow$ handles deletions of one or more terminals. Let us consider some illustrative cases.

\begin{table}[h!]
  \begin{tabular}{ccccccc}

    \texttt{f\hspace{3pt}.\hspace{3pt}\hlorange{[}\hspace{3pt}x\hspace{3pt})} &
    \texttt{f\hspace{3pt}.\hspace{3pt}\phantom{(}\hspace{3pt}x\hspace{3pt})} &
    \texttt{f\hspace{3pt}.\hspace{3pt}(\hspace{3pt}\hlred{x}\hspace{3pt})} &
    \texttt{\hlred{.}\hspace{3pt}\hlred{+}\hspace{3pt}(\hspace{3pt}x\hspace{3pt})} &
    \texttt{f\hspace{3pt}\hlorange{.}\hspace{3pt}\hlred{(}\hspace{3pt}x\hspace{3pt};} &
    \texttt{[\hspace{3pt}\hlorange{,}\hspace{3pt}\hlorange{x}\hspace{3pt}y\hspace{3pt}]} &
    \texttt{[\hspace{3pt}\phantom{,}\hspace{3pt},\hspace{3pt}\hlred{x}\hspace{3pt}y\hspace{3pt}]} \\

    \texttt{f\hspace{3pt}.\hspace{3pt}\hlorange{(}\hspace{3pt}x\hspace{3pt})} &
    \texttt{f\hspace{3pt}.\hspace{3pt}\hlgreen{(}\hspace{3pt}x\hspace{3pt})} &
    \texttt{f\hspace{3pt}.\hspace{3pt}(\hspace{3pt}\phantom{x}\hspace{3pt})} &
    \texttt{\phantom{f}\hspace{3pt}\phantom{.}\hspace{3pt}(\hspace{3pt}x\hspace{3pt})} &
    \texttt{f\hspace{3pt}\hlorange{*}\hspace{3pt}\phantom{(}\hspace{3pt}x\hspace{3pt};} &
    \texttt{[\hspace{3pt}\hlorange{x}\hspace{3pt}\hlorange{,}\hspace{3pt}y\hspace{3pt}]} &
    \texttt{[\hspace{3pt}\hlgreen{x}\hspace{3pt},\hspace{3pt}\phantom{x}\hspace{3pt}y\hspace{3pt}]} \\

    \substitutionExample & \insertionExample & \deletionExample & \doubleDeletionExample & \subDelExample & \subSubExample & \insertDeleteExample
  \end{tabular}
\end{table}\vspace{-0.3cm}

Note that the same patch can have multiple Levenshtein alignments. $\textsc{Done}$ constructs the final states, which are all states accepting strings $\sigma'$ whose Levenshtein distance $\Delta(\sigma, \sigma') \leq d_\max$.

To avoid creating a parallel bundle of arcs for each insertion and substitution point, we instead decorate each arc with a nominal predicate, accepting or rejecting $\sigma_i$. To distinguish this nominal variant from the original construction, we highlight the modified rules in orange below.

\begin{prooftree}
  \AxiomC{$i \in [0, n] \phantom{\land} j \in [1, d_{\max}]$}
  \RightLabel{$\duparrow$}
  \UnaryInfC{$(q_{i, j-1} \overset{{\color{orange}[\neq \sigma_{i+1}]}}{\rightarrow} q_{i,j}) \in \delta$}
  \DisplayProof
  \hskip 1.5em
  \AxiomC{$i \in [1, n] \phantom{\land} j \in [1, d_{\max}]$}
  \RightLabel{$\ddiagarrow$}
  \UnaryInfC{$(q_{i-1, j-1} \overset{{\color{orange}[\neq \sigma_i]}}{\rightarrow} q_{i,j}) \in \delta$}
\end{prooftree}
\begin{prooftree}
  \AxiomC{$i \in [1, n] \phantom{\land} j \in [0, d_{\max}]$}
  \RightLabel{$\drightarrow$}
  \UnaryInfC{$(q_{i-1, j} \overset{{\color{orange}[=\sigma_i]}}{\rightarrow} q_{i,j}) \in \delta$}
  \DisplayProof
  \hskip 1.5em
  \AxiomC{$d \in [1, d_{\max}] \phantom{\land} i \in [d + 1, n] \phantom{\land} j \in [d, d_{\max}]$}
  \RightLabel{$\knightarrow$}
  \UnaryInfC{$(q_{i-d-1, j-d} \overset{{\color{orange}[=\sigma_i]}}{\rightarrow} q_{i,j}) \in \delta$}
\end{prooftree}

Nominalizing the NFA eliminates the creation of $2(|\Sigma| - 1)\cdot|\sigma|\cdot d_\max$ unnecessary arcs and drastically reduces the representation size of the Levenshtein automaton, but does not affect the underlying semantics. Thus, it is important to first nominalize the automaton before proceeding.

\begin{wrapfigure}{r}{0.40\textwidth}
\resizebox{0.4\textwidth}{!}{%
\begin{tikzpicture}[
%->, % makes the edges directed
  >=stealth',
  node distance=2.5cm, % specifies the minimum distance between two nodes. Change if necessary.
%  every state/.style={thick, fill=gray!10}, % sets the properties for each ’state’ node
  initial text=$ $, % sets the text that appears on the start arrow
]
  \node[state, initial]                (00) {$q_{0,0}$};
  \node[state, right of=00]            (10) {$q_{1,0}$};
  \node[accepting, state, right of=10] (20) {$q_{2,0}$};
  \node[accepting, state, right of=20] (30) {$q_{3,0}$};

  \node[state, above of=00, shift={(-2cm,0cm)}] (01) {$q_{0,1}$};
  \node[state, right of=01]                     (11) {$q_{1,1}$};
  \node[state, right of=11]                     (21) {$q_{2,1}$};
  \node[accepting, state, right of=21]          (31) {$q_{3,1}$};

  \draw [->] (00) edge[below] node{\tiny{$[= \texttt{(}]$}} (10);
  \draw [->] (10) edge[below] node{\tiny{$[= \texttt{)}]$}} (20);
  \draw [->] (20) edge[below] node{\tiny{$[= \texttt{)}]$}} (30);

  \draw [->] (01) edge[below] node{\tiny{$[= \texttt{(}]$}}                       (11);
  \draw [->] (11) edge[below] node[shift={(-0.2cm,0cm)}]{\tiny{$[= \texttt{)}]$}} (21);
  \draw [->] (21) edge[below] node[shift={(-0.2cm,0cm)}]{\tiny{$[= \texttt{)}]$}} (31);

  \draw [->] (00) edge[left] node{\tiny{$[\neq \texttt{(}]$}} (11);
  \draw [->] (10) edge[left] node{\tiny{$[\neq \texttt{)}]$}} (21);
  \draw [->] (20) edge[left] node{\tiny{$[\neq \texttt{)}]$}} (31);

  \draw [->] (00) edge[bend left=10, left] node{\tiny{$[\neq \texttt{(}]$}} (01);
  \draw [->] (10) edge[bend left=10, left] node{\tiny{$[\neq \texttt{)}]$}} (11);
  \draw [->] (20) edge[bend left=10, left] node{\tiny{$[\neq \texttt{)}]$}} (21);
  \draw [->] (30) edge[bend left=10, left] node{\tiny{$[=.]$}} (31);


  \draw [->, blue] (00) edge[bend right=11,below] node[shift={(0.2cm,0.8cm)}]{\tiny{$[= \texttt{)}]$}}    (21);
  \draw [->, blue] (10) edge[bend right=11,below] node[shift={(0.2cm,0.8cm)}]{\tiny{$[= \texttt{)}]$}}    (31);
\end{tikzpicture}

}
\caption{Simple Levenshtein automaton.}\label{fig:ex_atm}

\vspace{0.3cm}
\resizebox{0.4\textwidth}{!}{%
\[
  \begin{tikzcd}[row sep=1.7em, column sep=1.7em]
  (0,3) \arrow[r]  & (1,3) \arrow[dr] & (2,3) \arrow[r]  & (3,3) \arrow[dr] & (4,3) \\
  (0,2) \arrow[dr] & (1,2) \arrow[ul] & (2,2) \arrow[dr] & (3,2) \arrow[ul] & (4,2) \arrow[u] \\
  (0,1) \arrow[u]  & (1,1) \arrow[dr] & (2,1) \arrow[ul] & (3,1) \arrow[dr] & (4,1) \arrow[ul] \\
  (0,0) \arrow[r]  & (1,0) \arrow[ul] & (2,0) \arrow[r]  & (3,0) \arrow[ul] & (4,0) \arrow[u]
  \end{tikzcd}
\]
}
\caption{Pairing function over $\mathcal{L}\big(L(\sigma: \Sigma^3, 1)\big)$.}\label{fig:pairing_fun}

\vspace{0.3cm}
\begin{center}
\resizebox{0.35\textwidth}{!}{%
\begin{tikzpicture}[x=0.3cm, y=0.3cm, draw=gray, very thin]
  \path[fill=white] (0,19) rectangle ++(1,1);
  \path[fill=black] (1,19) rectangle ++(1,1);
  \path[fill=black] (2,19) rectangle ++(1,1);
  \path[fill=white] (3,19) rectangle ++(1,1);
  \path[fill=black] (4,19) rectangle ++(1,1);
  \path[fill=white] (5,19) rectangle ++(1,1);
  \path[fill=white] (6,19) rectangle ++(1,1);
  \path[fill=white] (7,19) rectangle ++(1,1);
  \path[fill=black] (8,19) rectangle ++(1,1);
  \path[fill=white] (9,19) rectangle ++(1,1);
  \path[fill=white] (10,19) rectangle ++(1,1);
  \path[fill=white] (11,19) rectangle ++(1,1);
  \path[fill=white] (12,19) rectangle ++(1,1);
  \path[fill=white] (13,19) rectangle ++(1,1);
  \path[fill=white] (14,19) rectangle ++(1,1);
  \path[fill=black] (15,19) rectangle ++(1,1);
  \path[fill=white] (16,19) rectangle ++(1,1);
  \path[fill=white] (17,19) rectangle ++(1,1);
  \path[fill=white] (18,19) rectangle ++(1,1);
  \path[fill=black] (19,19) rectangle ++(1,1);
  \path[fill=white] (0,18) rectangle ++(1,1);
  \path[fill=white] (1,18) rectangle ++(1,1);
  \path[fill=white] (2,18) rectangle ++(1,1);
  \path[fill=black] (3,18) rectangle ++(1,1);
  \path[fill=black] (4,18) rectangle ++(1,1);
  \path[fill=white] (5,18) rectangle ++(1,1);
  \path[fill=white] (6,18) rectangle ++(1,1);
  \path[fill=black] (7,18) rectangle ++(1,1);
  \path[fill=white] (8,18) rectangle ++(1,1);
  \path[fill=white] (9,18) rectangle ++(1,1);
  \path[fill=white] (10,18) rectangle ++(1,1);
  \path[fill=black] (11,18) rectangle ++(1,1);
  \path[fill=white] (12,18) rectangle ++(1,1);
  \path[fill=white] (13,18) rectangle ++(1,1);
  \path[fill=white] (14,18) rectangle ++(1,1);
  \path[fill=white] (15,18) rectangle ++(1,1);
  \path[fill=white] (16,18) rectangle ++(1,1);
  \path[fill=black] (17,18) rectangle ++(1,1);
  \path[fill=white] (18,18) rectangle ++(1,1);
  \path[fill=white] (19,18) rectangle ++(1,1);
  \path[fill=white] (0,17) rectangle ++(1,1);
  \path[fill=white] (1,17) rectangle ++(1,1);
  \path[fill=white] (2,17) rectangle ++(1,1);
  \path[fill=white] (3,17) rectangle ++(1,1);
  \path[fill=black] (4,17) rectangle ++(1,1);
  \path[fill=black] (5,17) rectangle ++(1,1);
  \path[fill=white] (6,17) rectangle ++(1,1);
  \path[fill=white] (7,17) rectangle ++(1,1);
  \path[fill=black] (8,17) rectangle ++(1,1);
  \path[fill=white] (9,17) rectangle ++(1,1);
  \path[fill=white] (10,17) rectangle ++(1,1);
  \path[fill=white] (11,17) rectangle ++(1,1);
  \path[fill=black] (12,17) rectangle ++(1,1);
  \path[fill=white] (13,17) rectangle ++(1,1);
  \path[fill=white] (14,17) rectangle ++(1,1);
  \path[fill=white] (15,17) rectangle ++(1,1);
  \path[fill=white] (16,17) rectangle ++(1,1);
  \path[fill=white] (17,17) rectangle ++(1,1);
  \path[fill=black] (18,17) rectangle ++(1,1);
  \path[fill=white] (19,17) rectangle ++(1,1);
  \path[fill=white] (0,16) rectangle ++(1,1);
  \path[fill=white] (1,16) rectangle ++(1,1);
  \path[fill=white] (2,16) rectangle ++(1,1);
  \path[fill=white] (3,16) rectangle ++(1,1);
  \path[fill=white] (4,16) rectangle ++(1,1);
  \path[fill=white] (5,16) rectangle ++(1,1);
  \path[fill=black] (6,16) rectangle ++(1,1);
  \path[fill=black] (7,16) rectangle ++(1,1);
  \path[fill=white] (8,16) rectangle ++(1,1);
  \path[fill=white] (9,16) rectangle ++(1,1);
  \path[fill=black] (10,16) rectangle ++(1,1);
  \path[fill=white] (11,16) rectangle ++(1,1);
  \path[fill=white] (12,16) rectangle ++(1,1);
  \path[fill=white] (13,16) rectangle ++(1,1);
  \path[fill=black] (14,16) rectangle ++(1,1);
  \path[fill=white] (15,16) rectangle ++(1,1);
  \path[fill=white] (16,16) rectangle ++(1,1);
  \path[fill=white] (17,16) rectangle ++(1,1);
  \path[fill=white] (18,16) rectangle ++(1,1);
  \path[fill=white] (19,16) rectangle ++(1,1);
  \path[fill=white] (0,15) rectangle ++(1,1);
  \path[fill=white] (1,15) rectangle ++(1,1);
  \path[fill=white] (2,15) rectangle ++(1,1);
  \path[fill=white] (3,15) rectangle ++(1,1);
  \path[fill=white] (4,15) rectangle ++(1,1);
  \path[fill=white] (5,15) rectangle ++(1,1);
  \path[fill=white] (6,15) rectangle ++(1,1);
  \path[fill=black] (7,15) rectangle ++(1,1);
  \path[fill=black] (8,15) rectangle ++(1,1);
  \path[fill=white] (9,15) rectangle ++(1,1);
  \path[fill=white] (10,15) rectangle ++(1,1);
  \path[fill=black] (11,15) rectangle ++(1,1);
  \path[fill=white] (12,15) rectangle ++(1,1);
  \path[fill=white] (13,15) rectangle ++(1,1);
  \path[fill=white] (14,15) rectangle ++(1,1);
  \path[fill=black] (15,15) rectangle ++(1,1);
  \path[fill=white] (16,15) rectangle ++(1,1);
  \path[fill=white] (17,15) rectangle ++(1,1);
  \path[fill=white] (18,15) rectangle ++(1,1);
  \path[fill=black] (19,15) rectangle ++(1,1);
  \path[fill=white] (0,14) rectangle ++(1,1);
  \path[fill=white] (1,14) rectangle ++(1,1);
  \path[fill=white] (2,14) rectangle ++(1,1);
  \path[fill=white] (3,14) rectangle ++(1,1);
  \path[fill=white] (4,14) rectangle ++(1,1);
  \path[fill=white] (5,14) rectangle ++(1,1);
  \path[fill=white] (6,14) rectangle ++(1,1);
  \path[fill=white] (7,14) rectangle ++(1,1);
  \path[fill=black] (8,14) rectangle ++(1,1);
  \path[fill=black] (9,14) rectangle ++(1,1);
  \path[fill=white] (10,14) rectangle ++(1,1);
  \path[fill=white] (11,14) rectangle ++(1,1);
  \path[fill=black] (12,14) rectangle ++(1,1);
  \path[fill=white] (13,14) rectangle ++(1,1);
  \path[fill=white] (14,14) rectangle ++(1,1);
  \path[fill=white] (15,14) rectangle ++(1,1);
  \path[fill=black] (16,14) rectangle ++(1,1);
  \path[fill=white] (17,14) rectangle ++(1,1);
  \path[fill=white] (18,14) rectangle ++(1,1);
  \path[fill=white] (19,14) rectangle ++(1,1);
  \path[fill=white] (0,13) rectangle ++(1,1);
  \path[fill=white] (1,13) rectangle ++(1,1);
  \path[fill=white] (2,13) rectangle ++(1,1);
  \path[fill=white] (3,13) rectangle ++(1,1);
  \path[fill=white] (4,13) rectangle ++(1,1);
  \path[fill=white] (5,13) rectangle ++(1,1);
  \path[fill=white] (6,13) rectangle ++(1,1);
  \path[fill=white] (7,13) rectangle ++(1,1);
  \path[fill=white] (8,13) rectangle ++(1,1);
  \path[fill=white] (9,13) rectangle ++(1,1);
  \path[fill=black] (10,13) rectangle ++(1,1);
  \path[fill=white] (11,13) rectangle ++(1,1);
  \path[fill=white] (12,13) rectangle ++(1,1);
  \path[fill=white] (13,13) rectangle ++(1,1);
  \path[fill=white] (14,13) rectangle ++(1,1);
  \path[fill=white] (15,13) rectangle ++(1,1);
  \path[fill=white] (16,13) rectangle ++(1,1);
  \path[fill=white] (17,13) rectangle ++(1,1);
  \path[fill=white] (18,13) rectangle ++(1,1);
  \path[fill=white] (19,13) rectangle ++(1,1);
  \path[fill=white] (0,12) rectangle ++(1,1);
  \path[fill=white] (1,12) rectangle ++(1,1);
  \path[fill=white] (2,12) rectangle ++(1,1);
  \path[fill=white] (3,12) rectangle ++(1,1);
  \path[fill=white] (4,12) rectangle ++(1,1);
  \path[fill=white] (5,12) rectangle ++(1,1);
  \path[fill=white] (6,12) rectangle ++(1,1);
  \path[fill=white] (7,12) rectangle ++(1,1);
  \path[fill=white] (8,12) rectangle ++(1,1);
  \path[fill=white] (9,12) rectangle ++(1,1);
  \path[fill=black] (10,12) rectangle ++(1,1);
  \path[fill=black] (11,12) rectangle ++(1,1);
  \path[fill=white] (12,12) rectangle ++(1,1);
  \path[fill=white] (13,12) rectangle ++(1,1);
  \path[fill=black] (14,12) rectangle ++(1,1);
  \path[fill=white] (15,12) rectangle ++(1,1);
  \path[fill=white] (16,12) rectangle ++(1,1);
  \path[fill=black] (17,12) rectangle ++(1,1);
  \path[fill=white] (18,12) rectangle ++(1,1);
  \path[fill=white] (19,12) rectangle ++(1,1);
  \path[fill=white] (0,11) rectangle ++(1,1);
  \path[fill=white] (1,11) rectangle ++(1,1);
  \path[fill=white] (2,11) rectangle ++(1,1);
  \path[fill=white] (3,11) rectangle ++(1,1);
  \path[fill=white] (4,11) rectangle ++(1,1);
  \path[fill=white] (5,11) rectangle ++(1,1);
  \path[fill=white] (6,11) rectangle ++(1,1);
  \path[fill=white] (7,11) rectangle ++(1,1);
  \path[fill=white] (8,11) rectangle ++(1,1);
  \path[fill=white] (9,11) rectangle ++(1,1);
  \path[fill=white] (10,11) rectangle ++(1,1);
  \path[fill=black] (11,11) rectangle ++(1,1);
  \path[fill=black] (12,11) rectangle ++(1,1);
  \path[fill=white] (13,11) rectangle ++(1,1);
  \path[fill=white] (14,11) rectangle ++(1,1);
  \path[fill=black] (15,11) rectangle ++(1,1);
  \path[fill=white] (16,11) rectangle ++(1,1);
  \path[fill=white] (17,11) rectangle ++(1,1);
  \path[fill=black] (18,11) rectangle ++(1,1);
  \path[fill=white] (19,11) rectangle ++(1,1);
  \path[fill=white] (0,10) rectangle ++(1,1);
  \path[fill=white] (1,10) rectangle ++(1,1);
  \path[fill=white] (2,10) rectangle ++(1,1);
  \path[fill=white] (3,10) rectangle ++(1,1);
  \path[fill=white] (4,10) rectangle ++(1,1);
  \path[fill=white] (5,10) rectangle ++(1,1);
  \path[fill=white] (6,10) rectangle ++(1,1);
  \path[fill=white] (7,10) rectangle ++(1,1);
  \path[fill=white] (8,10) rectangle ++(1,1);
  \path[fill=white] (9,10) rectangle ++(1,1);
  \path[fill=white] (10,10) rectangle ++(1,1);
  \path[fill=white] (11,10) rectangle ++(1,1);
  \path[fill=black] (12,10) rectangle ++(1,1);
  \path[fill=black] (13,10) rectangle ++(1,1);
  \path[fill=white] (14,10) rectangle ++(1,1);
  \path[fill=white] (15,10) rectangle ++(1,1);
  \path[fill=black] (16,10) rectangle ++(1,1);
  \path[fill=white] (17,10) rectangle ++(1,1);
  \path[fill=white] (18,10) rectangle ++(1,1);
  \path[fill=white] (19,10) rectangle ++(1,1);
  \path[fill=white] (0,9) rectangle ++(1,1);
  \path[fill=white] (1,9) rectangle ++(1,1);
  \path[fill=white] (2,9) rectangle ++(1,1);
  \path[fill=white] (3,9) rectangle ++(1,1);
  \path[fill=white] (4,9) rectangle ++(1,1);
  \path[fill=white] (5,9) rectangle ++(1,1);
  \path[fill=white] (6,9) rectangle ++(1,1);
  \path[fill=white] (7,9) rectangle ++(1,1);
  \path[fill=white] (8,9) rectangle ++(1,1);
  \path[fill=white] (9,9) rectangle ++(1,1);
  \path[fill=white] (10,9) rectangle ++(1,1);
  \path[fill=white] (11,9) rectangle ++(1,1);
  \path[fill=white] (12,9) rectangle ++(1,1);
  \path[fill=white] (13,9) rectangle ++(1,1);
  \path[fill=black] (14,9) rectangle ++(1,1);
  \path[fill=white] (15,9) rectangle ++(1,1);
  \path[fill=white] (16,9) rectangle ++(1,1);
  \path[fill=white] (17,9) rectangle ++(1,1);
  \path[fill=white] (18,9) rectangle ++(1,1);
  \path[fill=white] (19,9) rectangle ++(1,1);
  \path[fill=white] (0,8) rectangle ++(1,1);
  \path[fill=white] (1,8) rectangle ++(1,1);
  \path[fill=white] (2,8) rectangle ++(1,1);
  \path[fill=white] (3,8) rectangle ++(1,1);
  \path[fill=white] (4,8) rectangle ++(1,1);
  \path[fill=white] (5,8) rectangle ++(1,1);
  \path[fill=white] (6,8) rectangle ++(1,1);
  \path[fill=white] (7,8) rectangle ++(1,1);
  \path[fill=white] (8,8) rectangle ++(1,1);
  \path[fill=white] (9,8) rectangle ++(1,1);
  \path[fill=white] (10,8) rectangle ++(1,1);
  \path[fill=white] (11,8) rectangle ++(1,1);
  \path[fill=white] (12,8) rectangle ++(1,1);
  \path[fill=white] (13,8) rectangle ++(1,1);
  \path[fill=black] (14,8) rectangle ++(1,1);
  \path[fill=black] (15,8) rectangle ++(1,1);
  \path[fill=white] (16,8) rectangle ++(1,1);
  \path[fill=black] (17,8) rectangle ++(1,1);
  \path[fill=white] (18,8) rectangle ++(1,1);
  \path[fill=black] (19,8) rectangle ++(1,1);
  \path[fill=white] (0,7) rectangle ++(1,1);
  \path[fill=white] (1,7) rectangle ++(1,1);
  \path[fill=white] (2,7) rectangle ++(1,1);
  \path[fill=white] (3,7) rectangle ++(1,1);
  \path[fill=white] (4,7) rectangle ++(1,1);
  \path[fill=white] (5,7) rectangle ++(1,1);
  \path[fill=white] (6,7) rectangle ++(1,1);
  \path[fill=white] (7,7) rectangle ++(1,1);
  \path[fill=white] (8,7) rectangle ++(1,1);
  \path[fill=white] (9,7) rectangle ++(1,1);
  \path[fill=white] (10,7) rectangle ++(1,1);
  \path[fill=white] (11,7) rectangle ++(1,1);
  \path[fill=white] (12,7) rectangle ++(1,1);
  \path[fill=white] (13,7) rectangle ++(1,1);
  \path[fill=white] (14,7) rectangle ++(1,1);
  \path[fill=black] (15,7) rectangle ++(1,1);
  \path[fill=black] (16,7) rectangle ++(1,1);
  \path[fill=white] (17,7) rectangle ++(1,1);
  \path[fill=black] (18,7) rectangle ++(1,1);
  \path[fill=white] (19,7) rectangle ++(1,1);
  \path[fill=white] (0,6) rectangle ++(1,1);
  \path[fill=white] (1,6) rectangle ++(1,1);
  \path[fill=white] (2,6) rectangle ++(1,1);
  \path[fill=white] (3,6) rectangle ++(1,1);
  \path[fill=white] (4,6) rectangle ++(1,1);
  \path[fill=white] (5,6) rectangle ++(1,1);
  \path[fill=white] (6,6) rectangle ++(1,1);
  \path[fill=white] (7,6) rectangle ++(1,1);
  \path[fill=white] (8,6) rectangle ++(1,1);
  \path[fill=white] (9,6) rectangle ++(1,1);
  \path[fill=white] (10,6) rectangle ++(1,1);
  \path[fill=white] (11,6) rectangle ++(1,1);
  \path[fill=white] (12,6) rectangle ++(1,1);
  \path[fill=white] (13,6) rectangle ++(1,1);
  \path[fill=white] (14,6) rectangle ++(1,1);
  \path[fill=white] (15,6) rectangle ++(1,1);
  \path[fill=black] (16,6) rectangle ++(1,1);
  \path[fill=white] (17,6) rectangle ++(1,1);
  \path[fill=white] (18,6) rectangle ++(1,1);
  \path[fill=white] (19,6) rectangle ++(1,1);
  \path[fill=white] (0,5) rectangle ++(1,1);
  \path[fill=white] (1,5) rectangle ++(1,1);
  \path[fill=white] (2,5) rectangle ++(1,1);
  \path[fill=white] (3,5) rectangle ++(1,1);
  \path[fill=white] (4,5) rectangle ++(1,1);
  \path[fill=white] (5,5) rectangle ++(1,1);
  \path[fill=white] (6,5) rectangle ++(1,1);
  \path[fill=white] (7,5) rectangle ++(1,1);
  \path[fill=white] (8,5) rectangle ++(1,1);
  \path[fill=white] (9,5) rectangle ++(1,1);
  \path[fill=white] (10,5) rectangle ++(1,1);
  \path[fill=white] (11,5) rectangle ++(1,1);
  \path[fill=white] (12,5) rectangle ++(1,1);
  \path[fill=white] (13,5) rectangle ++(1,1);
  \path[fill=white] (14,5) rectangle ++(1,1);
  \path[fill=white] (15,5) rectangle ++(1,1);
  \path[fill=white] (16,5) rectangle ++(1,1);
  \path[fill=black] (17,5) rectangle ++(1,1);
  \path[fill=white] (18,5) rectangle ++(1,1);
  \path[fill=white] (19,5) rectangle ++(1,1);
  \path[fill=white] (0,4) rectangle ++(1,1);
  \path[fill=white] (1,4) rectangle ++(1,1);
  \path[fill=white] (2,4) rectangle ++(1,1);
  \path[fill=white] (3,4) rectangle ++(1,1);
  \path[fill=white] (4,4) rectangle ++(1,1);
  \path[fill=white] (5,4) rectangle ++(1,1);
  \path[fill=white] (6,4) rectangle ++(1,1);
  \path[fill=white] (7,4) rectangle ++(1,1);
  \path[fill=white] (8,4) rectangle ++(1,1);
  \path[fill=white] (9,4) rectangle ++(1,1);
  \path[fill=white] (10,4) rectangle ++(1,1);
  \path[fill=white] (11,4) rectangle ++(1,1);
  \path[fill=white] (12,4) rectangle ++(1,1);
  \path[fill=white] (13,4) rectangle ++(1,1);
  \path[fill=white] (14,4) rectangle ++(1,1);
  \path[fill=white] (15,4) rectangle ++(1,1);
  \path[fill=white] (16,4) rectangle ++(1,1);
  \path[fill=black] (17,4) rectangle ++(1,1);
  \path[fill=black] (18,4) rectangle ++(1,1);
  \path[fill=black] (19,4) rectangle ++(1,1);
  \path[fill=white] (0,3) rectangle ++(1,1);
  \path[fill=white] (1,3) rectangle ++(1,1);
  \path[fill=white] (2,3) rectangle ++(1,1);
  \path[fill=white] (3,3) rectangle ++(1,1);
  \path[fill=white] (4,3) rectangle ++(1,1);
  \path[fill=white] (5,3) rectangle ++(1,1);
  \path[fill=white] (6,3) rectangle ++(1,1);
  \path[fill=white] (7,3) rectangle ++(1,1);
  \path[fill=white] (8,3) rectangle ++(1,1);
  \path[fill=white] (9,3) rectangle ++(1,1);
  \path[fill=white] (10,3) rectangle ++(1,1);
  \path[fill=white] (11,3) rectangle ++(1,1);
  \path[fill=white] (12,3) rectangle ++(1,1);
  \path[fill=white] (13,3) rectangle ++(1,1);
  \path[fill=white] (14,3) rectangle ++(1,1);
  \path[fill=white] (15,3) rectangle ++(1,1);
  \path[fill=white] (16,3) rectangle ++(1,1);
  \path[fill=white] (17,3) rectangle ++(1,1);
  \path[fill=black] (18,3) rectangle ++(1,1);
  \path[fill=white] (19,3) rectangle ++(1,1);
  \path[fill=white] (0,2) rectangle ++(1,1);
  \path[fill=white] (1,2) rectangle ++(1,1);
  \path[fill=white] (2,2) rectangle ++(1,1);
  \path[fill=white] (3,2) rectangle ++(1,1);
  \path[fill=white] (4,2) rectangle ++(1,1);
  \path[fill=white] (5,2) rectangle ++(1,1);
  \path[fill=white] (6,2) rectangle ++(1,1);
  \path[fill=white] (7,2) rectangle ++(1,1);
  \path[fill=white] (8,2) rectangle ++(1,1);
  \path[fill=white] (9,2) rectangle ++(1,1);
  \path[fill=white] (10,2) rectangle ++(1,1);
  \path[fill=white] (11,2) rectangle ++(1,1);
  \path[fill=white] (12,2) rectangle ++(1,1);
  \path[fill=white] (13,2) rectangle ++(1,1);
  \path[fill=white] (14,2) rectangle ++(1,1);
  \path[fill=white] (15,2) rectangle ++(1,1);
  \path[fill=white] (16,2) rectangle ++(1,1);
  \path[fill=white] (17,2) rectangle ++(1,1);
  \path[fill=white] (18,2) rectangle ++(1,1);
  \path[fill=black] (19,2) rectangle ++(1,1);
  \path[fill=white] (0,1) rectangle ++(1,1);
  \path[fill=white] (1,1) rectangle ++(1,1);
  \path[fill=white] (2,1) rectangle ++(1,1);
  \path[fill=white] (3,1) rectangle ++(1,1);
  \path[fill=white] (4,1) rectangle ++(1,1);
  \path[fill=white] (5,1) rectangle ++(1,1);
  \path[fill=white] (6,1) rectangle ++(1,1);
  \path[fill=white] (7,1) rectangle ++(1,1);
  \path[fill=white] (8,1) rectangle ++(1,1);
  \path[fill=white] (9,1) rectangle ++(1,1);
  \path[fill=white] (10,1) rectangle ++(1,1);
  \path[fill=white] (11,1) rectangle ++(1,1);
  \path[fill=white] (12,1) rectangle ++(1,1);
  \path[fill=white] (13,1) rectangle ++(1,1);
  \path[fill=white] (14,1) rectangle ++(1,1);
  \path[fill=white] (15,1) rectangle ++(1,1);
  \path[fill=white] (16,1) rectangle ++(1,1);
  \path[fill=white] (17,1) rectangle ++(1,1);
  \path[fill=white] (18,1) rectangle ++(1,1);
  \path[fill=black] (19,1) rectangle ++(1,1);
  \path[fill=white] (0,0) rectangle ++(1,1);
  \path[fill=white] (1,0) rectangle ++(1,1);
  \path[fill=white] (2,0) rectangle ++(1,1);
  \path[fill=white] (3,0) rectangle ++(1,1);
  \path[fill=white] (4,0) rectangle ++(1,1);
  \path[fill=white] (5,0) rectangle ++(1,1);
  \path[fill=white] (6,0) rectangle ++(1,1);
  \path[fill=white] (7,0) rectangle ++(1,1);
  \path[fill=white] (8,0) rectangle ++(1,1);
  \path[fill=white] (9,0) rectangle ++(1,1);
  \path[fill=white] (10,0) rectangle ++(1,1);
  \path[fill=white] (11,0) rectangle ++(1,1);
  \path[fill=white] (12,0) rectangle ++(1,1);
  \path[fill=white] (13,0) rectangle ++(1,1);
  \path[fill=white] (14,0) rectangle ++(1,1);
  \path[fill=white] (15,0) rectangle ++(1,1);
  \path[fill=white] (16,0) rectangle ++(1,1);
  \path[fill=white] (17,0) rectangle ++(1,1);
  \path[fill=white] (18,0) rectangle ++(1,1);
  \path[fill=white] (19,0) rectangle ++(1,1);
\end{tikzpicture}

\begin{tikzpicture}[x=0.3cm, y=0.3cm, draw=gray, very thin]
  \path[fill=white] (0,19) rectangle ++(1,1);
  \path[fill=black] (1,19) rectangle ++(1,1);
  \path[fill=black] (2,19) rectangle ++(1,1);
  \path[fill=black] (3,19) rectangle ++(1,1);
  \path[fill=black] (4,19) rectangle ++(1,1);
  \path[fill=white] (5,19) rectangle ++(1,1);
  \path[fill=white] (6,19) rectangle ++(1,1);
  \path[fill=black] (7,19) rectangle ++(1,1);
  \path[fill=black] (8,19) rectangle ++(1,1);
  \path[fill=white] (9,19) rectangle ++(1,1);
  \path[fill=black] (10,19) rectangle ++(1,1);
  \path[fill=black] (11,19) rectangle ++(1,1);
  \path[fill=white] (12,19) rectangle ++(1,1);
  \path[fill=white] (13,19) rectangle ++(1,1);
  \path[fill=black] (14,19) rectangle ++(1,1);
  \path[fill=black] (15,19) rectangle ++(1,1);
  \path[fill=white] (16,19) rectangle ++(1,1);
  \path[fill=black] (17,19) rectangle ++(1,1);
  \path[fill=black] (18,19) rectangle ++(1,1);
  \path[fill=black] (19,19) rectangle ++(1,1);
  \path[fill=white] (0,18) rectangle ++(1,1);
  \path[fill=white] (1,18) rectangle ++(1,1);
  \path[fill=white] (2,18) rectangle ++(1,1);
  \path[fill=black] (3,18) rectangle ++(1,1);
  \path[fill=black] (4,18) rectangle ++(1,1);
  \path[fill=white] (5,18) rectangle ++(1,1);
  \path[fill=black] (6,18) rectangle ++(1,1);
  \path[fill=black] (7,18) rectangle ++(1,1);
  \path[fill=white] (8,18) rectangle ++(1,1);
  \path[fill=white] (9,18) rectangle ++(1,1);
  \path[fill=black] (10,18) rectangle ++(1,1);
  \path[fill=black] (11,18) rectangle ++(1,1);
  \path[fill=white] (12,18) rectangle ++(1,1);
  \path[fill=white] (13,18) rectangle ++(1,1);
  \path[fill=black] (14,18) rectangle ++(1,1);
  \path[fill=white] (15,18) rectangle ++(1,1);
  \path[fill=white] (16,18) rectangle ++(1,1);
  \path[fill=black] (17,18) rectangle ++(1,1);
  \path[fill=white] (18,18) rectangle ++(1,1);
  \path[fill=black] (19,18) rectangle ++(1,1);
  \path[fill=white] (0,17) rectangle ++(1,1);
  \path[fill=white] (1,17) rectangle ++(1,1);
  \path[fill=white] (2,17) rectangle ++(1,1);
  \path[fill=white] (3,17) rectangle ++(1,1);
  \path[fill=black] (4,17) rectangle ++(1,1);
  \path[fill=black] (5,17) rectangle ++(1,1);
  \path[fill=white] (6,17) rectangle ++(1,1);
  \path[fill=black] (7,17) rectangle ++(1,1);
  \path[fill=black] (8,17) rectangle ++(1,1);
  \path[fill=white] (9,17) rectangle ++(1,1);
  \path[fill=white] (10,17) rectangle ++(1,1);
  \path[fill=black] (11,17) rectangle ++(1,1);
  \path[fill=black] (12,17) rectangle ++(1,1);
  \path[fill=white] (13,17) rectangle ++(1,1);
  \path[fill=black] (14,17) rectangle ++(1,1);
  \path[fill=black] (15,17) rectangle ++(1,1);
  \path[fill=white] (16,17) rectangle ++(1,1);
  \path[fill=black] (17,17) rectangle ++(1,1);
  \path[fill=black] (18,17) rectangle ++(1,1);
  \path[fill=black] (19,17) rectangle ++(1,1);
  \path[fill=white] (0,16) rectangle ++(1,1);
  \path[fill=white] (1,16) rectangle ++(1,1);
  \path[fill=white] (2,16) rectangle ++(1,1);
  \path[fill=white] (3,16) rectangle ++(1,1);
  \path[fill=white] (4,16) rectangle ++(1,1);
  \path[fill=white] (5,16) rectangle ++(1,1);
  \path[fill=black] (6,16) rectangle ++(1,1);
  \path[fill=black] (7,16) rectangle ++(1,1);
  \path[fill=white] (8,16) rectangle ++(1,1);
  \path[fill=white] (9,16) rectangle ++(1,1);
  \path[fill=black] (10,16) rectangle ++(1,1);
  \path[fill=white] (11,16) rectangle ++(1,1);
  \path[fill=white] (12,16) rectangle ++(1,1);
  \path[fill=white] (13,16) rectangle ++(1,1);
  \path[fill=black] (14,16) rectangle ++(1,1);
  \path[fill=white] (15,16) rectangle ++(1,1);
  \path[fill=white] (16,16) rectangle ++(1,1);
  \path[fill=white] (17,16) rectangle ++(1,1);
  \path[fill=white] (18,16) rectangle ++(1,1);
  \path[fill=white] (19,16) rectangle ++(1,1);
  \path[fill=white] (0,15) rectangle ++(1,1);
  \path[fill=white] (1,15) rectangle ++(1,1);
  \path[fill=white] (2,15) rectangle ++(1,1);
  \path[fill=white] (3,15) rectangle ++(1,1);
  \path[fill=white] (4,15) rectangle ++(1,1);
  \path[fill=white] (5,15) rectangle ++(1,1);
  \path[fill=white] (6,15) rectangle ++(1,1);
  \path[fill=black] (7,15) rectangle ++(1,1);
  \path[fill=black] (8,15) rectangle ++(1,1);
  \path[fill=white] (9,15) rectangle ++(1,1);
  \path[fill=black] (10,15) rectangle ++(1,1);
  \path[fill=black] (11,15) rectangle ++(1,1);
  \path[fill=white] (12,15) rectangle ++(1,1);
  \path[fill=white] (13,15) rectangle ++(1,1);
  \path[fill=black] (14,15) rectangle ++(1,1);
  \path[fill=black] (15,15) rectangle ++(1,1);
  \path[fill=white] (16,15) rectangle ++(1,1);
  \path[fill=black] (17,15) rectangle ++(1,1);
  \path[fill=black] (18,15) rectangle ++(1,1);
  \path[fill=black] (19,15) rectangle ++(1,1);
  \path[fill=white] (0,14) rectangle ++(1,1);
  \path[fill=white] (1,14) rectangle ++(1,1);
  \path[fill=white] (2,14) rectangle ++(1,1);
  \path[fill=white] (3,14) rectangle ++(1,1);
  \path[fill=white] (4,14) rectangle ++(1,1);
  \path[fill=white] (5,14) rectangle ++(1,1);
  \path[fill=white] (6,14) rectangle ++(1,1);
  \path[fill=white] (7,14) rectangle ++(1,1);
  \path[fill=black] (8,14) rectangle ++(1,1);
  \path[fill=black] (9,14) rectangle ++(1,1);
  \path[fill=white] (10,14) rectangle ++(1,1);
  \path[fill=black] (11,14) rectangle ++(1,1);
  \path[fill=black] (12,14) rectangle ++(1,1);
  \path[fill=white] (13,14) rectangle ++(1,1);
  \path[fill=white] (14,14) rectangle ++(1,1);
  \path[fill=black] (15,14) rectangle ++(1,1);
  \path[fill=black] (16,14) rectangle ++(1,1);
  \path[fill=black] (17,14) rectangle ++(1,1);
  \path[fill=black] (18,14) rectangle ++(1,1);
  \path[fill=black] (19,14) rectangle ++(1,1);
  \path[fill=white] (0,13) rectangle ++(1,1);
  \path[fill=white] (1,13) rectangle ++(1,1);
  \path[fill=white] (2,13) rectangle ++(1,1);
  \path[fill=white] (3,13) rectangle ++(1,1);
  \path[fill=white] (4,13) rectangle ++(1,1);
  \path[fill=white] (5,13) rectangle ++(1,1);
  \path[fill=white] (6,13) rectangle ++(1,1);
  \path[fill=white] (7,13) rectangle ++(1,1);
  \path[fill=white] (8,13) rectangle ++(1,1);
  \path[fill=white] (9,13) rectangle ++(1,1);
  \path[fill=black] (10,13) rectangle ++(1,1);
  \path[fill=white] (11,13) rectangle ++(1,1);
  \path[fill=white] (12,13) rectangle ++(1,1);
  \path[fill=white] (13,13) rectangle ++(1,1);
  \path[fill=white] (14,13) rectangle ++(1,1);
  \path[fill=white] (15,13) rectangle ++(1,1);
  \path[fill=white] (16,13) rectangle ++(1,1);
  \path[fill=white] (17,13) rectangle ++(1,1);
  \path[fill=white] (18,13) rectangle ++(1,1);
  \path[fill=white] (19,13) rectangle ++(1,1);
  \path[fill=white] (0,12) rectangle ++(1,1);
  \path[fill=white] (1,12) rectangle ++(1,1);
  \path[fill=white] (2,12) rectangle ++(1,1);
  \path[fill=white] (3,12) rectangle ++(1,1);
  \path[fill=white] (4,12) rectangle ++(1,1);
  \path[fill=white] (5,12) rectangle ++(1,1);
  \path[fill=white] (6,12) rectangle ++(1,1);
  \path[fill=white] (7,12) rectangle ++(1,1);
  \path[fill=white] (8,12) rectangle ++(1,1);
  \path[fill=white] (9,12) rectangle ++(1,1);
  \path[fill=black] (10,12) rectangle ++(1,1);
  \path[fill=black] (11,12) rectangle ++(1,1);
  \path[fill=white] (12,12) rectangle ++(1,1);
  \path[fill=white] (13,12) rectangle ++(1,1);
  \path[fill=black] (14,12) rectangle ++(1,1);
  \path[fill=white] (15,12) rectangle ++(1,1);
  \path[fill=white] (16,12) rectangle ++(1,1);
  \path[fill=black] (17,12) rectangle ++(1,1);
  \path[fill=white] (18,12) rectangle ++(1,1);
  \path[fill=white] (19,12) rectangle ++(1,1);
  \path[fill=white] (0,11) rectangle ++(1,1);
  \path[fill=white] (1,11) rectangle ++(1,1);
  \path[fill=white] (2,11) rectangle ++(1,1);
  \path[fill=white] (3,11) rectangle ++(1,1);
  \path[fill=white] (4,11) rectangle ++(1,1);
  \path[fill=white] (5,11) rectangle ++(1,1);
  \path[fill=white] (6,11) rectangle ++(1,1);
  \path[fill=white] (7,11) rectangle ++(1,1);
  \path[fill=white] (8,11) rectangle ++(1,1);
  \path[fill=white] (9,11) rectangle ++(1,1);
  \path[fill=white] (10,11) rectangle ++(1,1);
  \path[fill=black] (11,11) rectangle ++(1,1);
  \path[fill=black] (12,11) rectangle ++(1,1);
  \path[fill=white] (13,11) rectangle ++(1,1);
  \path[fill=black] (14,11) rectangle ++(1,1);
  \path[fill=black] (15,11) rectangle ++(1,1);
  \path[fill=white] (16,11) rectangle ++(1,1);
  \path[fill=black] (17,11) rectangle ++(1,1);
  \path[fill=black] (18,11) rectangle ++(1,1);
  \path[fill=black] (19,11) rectangle ++(1,1);
  \path[fill=white] (0,10) rectangle ++(1,1);
  \path[fill=white] (1,10) rectangle ++(1,1);
  \path[fill=white] (2,10) rectangle ++(1,1);
  \path[fill=white] (3,10) rectangle ++(1,1);
  \path[fill=white] (4,10) rectangle ++(1,1);
  \path[fill=white] (5,10) rectangle ++(1,1);
  \path[fill=white] (6,10) rectangle ++(1,1);
  \path[fill=white] (7,10) rectangle ++(1,1);
  \path[fill=white] (8,10) rectangle ++(1,1);
  \path[fill=white] (9,10) rectangle ++(1,1);
  \path[fill=white] (10,10) rectangle ++(1,1);
  \path[fill=white] (11,10) rectangle ++(1,1);
  \path[fill=black] (12,10) rectangle ++(1,1);
  \path[fill=black] (13,10) rectangle ++(1,1);
  \path[fill=white] (14,10) rectangle ++(1,1);
  \path[fill=black] (15,10) rectangle ++(1,1);
  \path[fill=black] (16,10) rectangle ++(1,1);
  \path[fill=white] (17,10) rectangle ++(1,1);
  \path[fill=black] (18,10) rectangle ++(1,1);
  \path[fill=black] (19,10) rectangle ++(1,1);
  \path[fill=white] (0,9) rectangle ++(1,1);
  \path[fill=white] (1,9) rectangle ++(1,1);
  \path[fill=white] (2,9) rectangle ++(1,1);
  \path[fill=white] (3,9) rectangle ++(1,1);
  \path[fill=white] (4,9) rectangle ++(1,1);
  \path[fill=white] (5,9) rectangle ++(1,1);
  \path[fill=white] (6,9) rectangle ++(1,1);
  \path[fill=white] (7,9) rectangle ++(1,1);
  \path[fill=white] (8,9) rectangle ++(1,1);
  \path[fill=white] (9,9) rectangle ++(1,1);
  \path[fill=white] (10,9) rectangle ++(1,1);
  \path[fill=white] (11,9) rectangle ++(1,1);
  \path[fill=white] (12,9) rectangle ++(1,1);
  \path[fill=white] (13,9) rectangle ++(1,1);
  \path[fill=black] (14,9) rectangle ++(1,1);
  \path[fill=white] (15,9) rectangle ++(1,1);
  \path[fill=white] (16,9) rectangle ++(1,1);
  \path[fill=white] (17,9) rectangle ++(1,1);
  \path[fill=white] (18,9) rectangle ++(1,1);
  \path[fill=white] (19,9) rectangle ++(1,1);
  \path[fill=white] (0,8) rectangle ++(1,1);
  \path[fill=white] (1,8) rectangle ++(1,1);
  \path[fill=white] (2,8) rectangle ++(1,1);
  \path[fill=white] (3,8) rectangle ++(1,1);
  \path[fill=white] (4,8) rectangle ++(1,1);
  \path[fill=white] (5,8) rectangle ++(1,1);
  \path[fill=white] (6,8) rectangle ++(1,1);
  \path[fill=white] (7,8) rectangle ++(1,1);
  \path[fill=white] (8,8) rectangle ++(1,1);
  \path[fill=white] (9,8) rectangle ++(1,1);
  \path[fill=white] (10,8) rectangle ++(1,1);
  \path[fill=white] (11,8) rectangle ++(1,1);
  \path[fill=white] (12,8) rectangle ++(1,1);
  \path[fill=white] (13,8) rectangle ++(1,1);
  \path[fill=black] (14,8) rectangle ++(1,1);
  \path[fill=black] (15,8) rectangle ++(1,1);
  \path[fill=white] (16,8) rectangle ++(1,1);
  \path[fill=black] (17,8) rectangle ++(1,1);
  \path[fill=white] (18,8) rectangle ++(1,1);
  \path[fill=black] (19,8) rectangle ++(1,1);
  \path[fill=white] (0,7) rectangle ++(1,1);
  \path[fill=white] (1,7) rectangle ++(1,1);
  \path[fill=white] (2,7) rectangle ++(1,1);
  \path[fill=white] (3,7) rectangle ++(1,1);
  \path[fill=white] (4,7) rectangle ++(1,1);
  \path[fill=white] (5,7) rectangle ++(1,1);
  \path[fill=white] (6,7) rectangle ++(1,1);
  \path[fill=white] (7,7) rectangle ++(1,1);
  \path[fill=white] (8,7) rectangle ++(1,1);
  \path[fill=white] (9,7) rectangle ++(1,1);
  \path[fill=white] (10,7) rectangle ++(1,1);
  \path[fill=white] (11,7) rectangle ++(1,1);
  \path[fill=white] (12,7) rectangle ++(1,1);
  \path[fill=white] (13,7) rectangle ++(1,1);
  \path[fill=white] (14,7) rectangle ++(1,1);
  \path[fill=black] (15,7) rectangle ++(1,1);
  \path[fill=black] (16,7) rectangle ++(1,1);
  \path[fill=black] (17,7) rectangle ++(1,1);
  \path[fill=black] (18,7) rectangle ++(1,1);
  \path[fill=black] (19,7) rectangle ++(1,1);
  \path[fill=white] (0,6) rectangle ++(1,1);
  \path[fill=white] (1,6) rectangle ++(1,1);
  \path[fill=white] (2,6) rectangle ++(1,1);
  \path[fill=white] (3,6) rectangle ++(1,1);
  \path[fill=white] (4,6) rectangle ++(1,1);
  \path[fill=white] (5,6) rectangle ++(1,1);
  \path[fill=white] (6,6) rectangle ++(1,1);
  \path[fill=white] (7,6) rectangle ++(1,1);
  \path[fill=white] (8,6) rectangle ++(1,1);
  \path[fill=white] (9,6) rectangle ++(1,1);
  \path[fill=white] (10,6) rectangle ++(1,1);
  \path[fill=white] (11,6) rectangle ++(1,1);
  \path[fill=white] (12,6) rectangle ++(1,1);
  \path[fill=white] (13,6) rectangle ++(1,1);
  \path[fill=white] (14,6) rectangle ++(1,1);
  \path[fill=white] (15,6) rectangle ++(1,1);
  \path[fill=black] (16,6) rectangle ++(1,1);
  \path[fill=white] (17,6) rectangle ++(1,1);
  \path[fill=black] (18,6) rectangle ++(1,1);
  \path[fill=white] (19,6) rectangle ++(1,1);
  \path[fill=white] (0,5) rectangle ++(1,1);
  \path[fill=white] (1,5) rectangle ++(1,1);
  \path[fill=white] (2,5) rectangle ++(1,1);
  \path[fill=white] (3,5) rectangle ++(1,1);
  \path[fill=white] (4,5) rectangle ++(1,1);
  \path[fill=white] (5,5) rectangle ++(1,1);
  \path[fill=white] (6,5) rectangle ++(1,1);
  \path[fill=white] (7,5) rectangle ++(1,1);
  \path[fill=white] (8,5) rectangle ++(1,1);
  \path[fill=white] (9,5) rectangle ++(1,1);
  \path[fill=white] (10,5) rectangle ++(1,1);
  \path[fill=white] (11,5) rectangle ++(1,1);
  \path[fill=white] (12,5) rectangle ++(1,1);
  \path[fill=white] (13,5) rectangle ++(1,1);
  \path[fill=white] (14,5) rectangle ++(1,1);
  \path[fill=white] (15,5) rectangle ++(1,1);
  \path[fill=white] (16,5) rectangle ++(1,1);
  \path[fill=black] (17,5) rectangle ++(1,1);
  \path[fill=white] (18,5) rectangle ++(1,1);
  \path[fill=white] (19,5) rectangle ++(1,1);
  \path[fill=white] (0,4) rectangle ++(1,1);
  \path[fill=white] (1,4) rectangle ++(1,1);
  \path[fill=white] (2,4) rectangle ++(1,1);
  \path[fill=white] (3,4) rectangle ++(1,1);
  \path[fill=white] (4,4) rectangle ++(1,1);
  \path[fill=white] (5,4) rectangle ++(1,1);
  \path[fill=white] (6,4) rectangle ++(1,1);
  \path[fill=white] (7,4) rectangle ++(1,1);
  \path[fill=white] (8,4) rectangle ++(1,1);
  \path[fill=white] (9,4) rectangle ++(1,1);
  \path[fill=white] (10,4) rectangle ++(1,1);
  \path[fill=white] (11,4) rectangle ++(1,1);
  \path[fill=white] (12,4) rectangle ++(1,1);
  \path[fill=white] (13,4) rectangle ++(1,1);
  \path[fill=white] (14,4) rectangle ++(1,1);
  \path[fill=white] (15,4) rectangle ++(1,1);
  \path[fill=white] (16,4) rectangle ++(1,1);
  \path[fill=black] (17,4) rectangle ++(1,1);
  \path[fill=black] (18,4) rectangle ++(1,1);
  \path[fill=black] (19,4) rectangle ++(1,1);
  \path[fill=white] (0,3) rectangle ++(1,1);
  \path[fill=white] (1,3) rectangle ++(1,1);
  \path[fill=white] (2,3) rectangle ++(1,1);
  \path[fill=white] (3,3) rectangle ++(1,1);
  \path[fill=white] (4,3) rectangle ++(1,1);
  \path[fill=white] (5,3) rectangle ++(1,1);
  \path[fill=white] (6,3) rectangle ++(1,1);
  \path[fill=white] (7,3) rectangle ++(1,1);
  \path[fill=white] (8,3) rectangle ++(1,1);
  \path[fill=white] (9,3) rectangle ++(1,1);
  \path[fill=white] (10,3) rectangle ++(1,1);
  \path[fill=white] (11,3) rectangle ++(1,1);
  \path[fill=white] (12,3) rectangle ++(1,1);
  \path[fill=white] (13,3) rectangle ++(1,1);
  \path[fill=white] (14,3) rectangle ++(1,1);
  \path[fill=white] (15,3) rectangle ++(1,1);
  \path[fill=white] (16,3) rectangle ++(1,1);
  \path[fill=white] (17,3) rectangle ++(1,1);
  \path[fill=black] (18,3) rectangle ++(1,1);
  \path[fill=black] (19,3) rectangle ++(1,1);
  \path[fill=white] (0,2) rectangle ++(1,1);
  \path[fill=white] (1,2) rectangle ++(1,1);
  \path[fill=white] (2,2) rectangle ++(1,1);
  \path[fill=white] (3,2) rectangle ++(1,1);
  \path[fill=white] (4,2) rectangle ++(1,1);
  \path[fill=white] (5,2) rectangle ++(1,1);
  \path[fill=white] (6,2) rectangle ++(1,1);
  \path[fill=white] (7,2) rectangle ++(1,1);
  \path[fill=white] (8,2) rectangle ++(1,1);
  \path[fill=white] (9,2) rectangle ++(1,1);
  \path[fill=white] (10,2) rectangle ++(1,1);
  \path[fill=white] (11,2) rectangle ++(1,1);
  \path[fill=white] (12,2) rectangle ++(1,1);
  \path[fill=white] (13,2) rectangle ++(1,1);
  \path[fill=white] (14,2) rectangle ++(1,1);
  \path[fill=white] (15,2) rectangle ++(1,1);
  \path[fill=white] (16,2) rectangle ++(1,1);
  \path[fill=white] (17,2) rectangle ++(1,1);
  \path[fill=white] (18,2) rectangle ++(1,1);
  \path[fill=black] (19,2) rectangle ++(1,1);
  \path[fill=white] (0,1) rectangle ++(1,1);
  \path[fill=white] (1,1) rectangle ++(1,1);
  \path[fill=white] (2,1) rectangle ++(1,1);
  \path[fill=white] (3,1) rectangle ++(1,1);
  \path[fill=white] (4,1) rectangle ++(1,1);
  \path[fill=white] (5,1) rectangle ++(1,1);
  \path[fill=white] (6,1) rectangle ++(1,1);
  \path[fill=white] (7,1) rectangle ++(1,1);
  \path[fill=white] (8,1) rectangle ++(1,1);
  \path[fill=white] (9,1) rectangle ++(1,1);
  \path[fill=white] (10,1) rectangle ++(1,1);
  \path[fill=white] (11,1) rectangle ++(1,1);
  \path[fill=white] (12,1) rectangle ++(1,1);
  \path[fill=white] (13,1) rectangle ++(1,1);
  \path[fill=white] (14,1) rectangle ++(1,1);
  \path[fill=white] (15,1) rectangle ++(1,1);
  \path[fill=white] (16,1) rectangle ++(1,1);
  \path[fill=white] (17,1) rectangle ++(1,1);
  \path[fill=white] (18,1) rectangle ++(1,1);
  \path[fill=black] (19,1) rectangle ++(1,1);
  \path[fill=white] (0,0) rectangle ++(1,1);
  \path[fill=white] (1,0) rectangle ++(1,1);
  \path[fill=white] (2,0) rectangle ++(1,1);
  \path[fill=white] (3,0) rectangle ++(1,1);
  \path[fill=white] (4,0) rectangle ++(1,1);
  \path[fill=white] (5,0) rectangle ++(1,1);
  \path[fill=white] (6,0) rectangle ++(1,1);
  \path[fill=white] (7,0) rectangle ++(1,1);
  \path[fill=white] (8,0) rectangle ++(1,1);
  \path[fill=white] (9,0) rectangle ++(1,1);
  \path[fill=white] (10,0) rectangle ++(1,1);
  \path[fill=white] (11,0) rectangle ++(1,1);
  \path[fill=white] (12,0) rectangle ++(1,1);
  \path[fill=white] (13,0) rectangle ++(1,1);
  \path[fill=white] (14,0) rectangle ++(1,1);
  \path[fill=white] (15,0) rectangle ++(1,1);
  \path[fill=white] (16,0) rectangle ++(1,1);
  \path[fill=white] (17,0) rectangle ++(1,1);
  \path[fill=white] (18,0) rectangle ++(1,1);
  \path[fill=white] (19,0) rectangle ++(1,1);
\end{tikzpicture}

\begin{tikzpicture}[x=0.3cm, y=0.3cm, draw=gray, very thin]
  \path[fill=white] (0,19) rectangle ++(1,1);
  \path[fill=black] (1,19) rectangle ++(1,1);
  \path[fill=black] (2,19) rectangle ++(1,1);
  \path[fill=black] (3,19) rectangle ++(1,1);
  \path[fill=black] (4,19) rectangle ++(1,1);
  \path[fill=white] (5,19) rectangle ++(1,1);
  \path[fill=black] (6,19) rectangle ++(1,1);
  \path[fill=black] (7,19) rectangle ++(1,1);
  \path[fill=black] (8,19) rectangle ++(1,1);
  \path[fill=white] (9,19) rectangle ++(1,1);
  \path[fill=black] (10,19) rectangle ++(1,1);
  \path[fill=black] (11,19) rectangle ++(1,1);
  \path[fill=black] (12,19) rectangle ++(1,1);
  \path[fill=white] (13,19) rectangle ++(1,1);
  \path[fill=black] (14,19) rectangle ++(1,1);
  \path[fill=black] (15,19) rectangle ++(1,1);
  \path[fill=white] (16,19) rectangle ++(1,1);
  \path[fill=black] (17,19) rectangle ++(1,1);
  \path[fill=black] (18,19) rectangle ++(1,1);
  \path[fill=black] (19,19) rectangle ++(1,1);
  \path[fill=white] (0,18) rectangle ++(1,1);
  \path[fill=white] (1,18) rectangle ++(1,1);
  \path[fill=white] (2,18) rectangle ++(1,1);
  \path[fill=black] (3,18) rectangle ++(1,1);
  \path[fill=black] (4,18) rectangle ++(1,1);
  \path[fill=white] (5,18) rectangle ++(1,1);
  \path[fill=black] (6,18) rectangle ++(1,1);
  \path[fill=black] (7,18) rectangle ++(1,1);
  \path[fill=white] (8,18) rectangle ++(1,1);
  \path[fill=white] (9,18) rectangle ++(1,1);
  \path[fill=black] (10,18) rectangle ++(1,1);
  \path[fill=black] (11,18) rectangle ++(1,1);
  \path[fill=white] (12,18) rectangle ++(1,1);
  \path[fill=white] (13,18) rectangle ++(1,1);
  \path[fill=black] (14,18) rectangle ++(1,1);
  \path[fill=black] (15,18) rectangle ++(1,1);
  \path[fill=white] (16,18) rectangle ++(1,1);
  \path[fill=black] (17,18) rectangle ++(1,1);
  \path[fill=white] (18,18) rectangle ++(1,1);
  \path[fill=black] (19,18) rectangle ++(1,1);
  \path[fill=white] (0,17) rectangle ++(1,1);
  \path[fill=white] (1,17) rectangle ++(1,1);
  \path[fill=white] (2,17) rectangle ++(1,1);
  \path[fill=white] (3,17) rectangle ++(1,1);
  \path[fill=black] (4,17) rectangle ++(1,1);
  \path[fill=black] (5,17) rectangle ++(1,1);
  \path[fill=white] (6,17) rectangle ++(1,1);
  \path[fill=black] (7,17) rectangle ++(1,1);
  \path[fill=black] (8,17) rectangle ++(1,1);
  \path[fill=white] (9,17) rectangle ++(1,1);
  \path[fill=black] (10,17) rectangle ++(1,1);
  \path[fill=black] (11,17) rectangle ++(1,1);
  \path[fill=black] (12,17) rectangle ++(1,1);
  \path[fill=white] (13,17) rectangle ++(1,1);
  \path[fill=black] (14,17) rectangle ++(1,1);
  \path[fill=black] (15,17) rectangle ++(1,1);
  \path[fill=black] (16,17) rectangle ++(1,1);
  \path[fill=black] (17,17) rectangle ++(1,1);
  \path[fill=black] (18,17) rectangle ++(1,1);
  \path[fill=black] (19,17) rectangle ++(1,1);
  \path[fill=white] (0,16) rectangle ++(1,1);
  \path[fill=white] (1,16) rectangle ++(1,1);
  \path[fill=white] (2,16) rectangle ++(1,1);
  \path[fill=white] (3,16) rectangle ++(1,1);
  \path[fill=white] (4,16) rectangle ++(1,1);
  \path[fill=white] (5,16) rectangle ++(1,1);
  \path[fill=black] (6,16) rectangle ++(1,1);
  \path[fill=black] (7,16) rectangle ++(1,1);
  \path[fill=white] (8,16) rectangle ++(1,1);
  \path[fill=white] (9,16) rectangle ++(1,1);
  \path[fill=black] (10,16) rectangle ++(1,1);
  \path[fill=white] (11,16) rectangle ++(1,1);
  \path[fill=white] (12,16) rectangle ++(1,1);
  \path[fill=white] (13,16) rectangle ++(1,1);
  \path[fill=black] (14,16) rectangle ++(1,1);
  \path[fill=white] (15,16) rectangle ++(1,1);
  \path[fill=white] (16,16) rectangle ++(1,1);
  \path[fill=black] (17,16) rectangle ++(1,1);
  \path[fill=white] (18,16) rectangle ++(1,1);
  \path[fill=white] (19,16) rectangle ++(1,1);
  \path[fill=white] (0,15) rectangle ++(1,1);
  \path[fill=white] (1,15) rectangle ++(1,1);
  \path[fill=white] (2,15) rectangle ++(1,1);
  \path[fill=white] (3,15) rectangle ++(1,1);
  \path[fill=white] (4,15) rectangle ++(1,1);
  \path[fill=white] (5,15) rectangle ++(1,1);
  \path[fill=white] (6,15) rectangle ++(1,1);
  \path[fill=black] (7,15) rectangle ++(1,1);
  \path[fill=black] (8,15) rectangle ++(1,1);
  \path[fill=white] (9,15) rectangle ++(1,1);
  \path[fill=black] (10,15) rectangle ++(1,1);
  \path[fill=black] (11,15) rectangle ++(1,1);
  \path[fill=white] (12,15) rectangle ++(1,1);
  \path[fill=white] (13,15) rectangle ++(1,1);
  \path[fill=black] (14,15) rectangle ++(1,1);
  \path[fill=black] (15,15) rectangle ++(1,1);
  \path[fill=white] (16,15) rectangle ++(1,1);
  \path[fill=black] (17,15) rectangle ++(1,1);
  \path[fill=black] (18,15) rectangle ++(1,1);
  \path[fill=black] (19,15) rectangle ++(1,1);
  \path[fill=white] (0,14) rectangle ++(1,1);
  \path[fill=white] (1,14) rectangle ++(1,1);
  \path[fill=white] (2,14) rectangle ++(1,1);
  \path[fill=white] (3,14) rectangle ++(1,1);
  \path[fill=white] (4,14) rectangle ++(1,1);
  \path[fill=white] (5,14) rectangle ++(1,1);
  \path[fill=white] (6,14) rectangle ++(1,1);
  \path[fill=white] (7,14) rectangle ++(1,1);
  \path[fill=black] (8,14) rectangle ++(1,1);
  \path[fill=black] (9,14) rectangle ++(1,1);
  \path[fill=white] (10,14) rectangle ++(1,1);
  \path[fill=black] (11,14) rectangle ++(1,1);
  \path[fill=black] (12,14) rectangle ++(1,1);
  \path[fill=white] (13,14) rectangle ++(1,1);
  \path[fill=black] (14,14) rectangle ++(1,1);
  \path[fill=black] (15,14) rectangle ++(1,1);
  \path[fill=black] (16,14) rectangle ++(1,1);
  \path[fill=black] (17,14) rectangle ++(1,1);
  \path[fill=black] (18,14) rectangle ++(1,1);
  \path[fill=black] (19,14) rectangle ++(1,1);
  \path[fill=white] (0,13) rectangle ++(1,1);
  \path[fill=white] (1,13) rectangle ++(1,1);
  \path[fill=white] (2,13) rectangle ++(1,1);
  \path[fill=white] (3,13) rectangle ++(1,1);
  \path[fill=white] (4,13) rectangle ++(1,1);
  \path[fill=white] (5,13) rectangle ++(1,1);
  \path[fill=white] (6,13) rectangle ++(1,1);
  \path[fill=white] (7,13) rectangle ++(1,1);
  \path[fill=white] (8,13) rectangle ++(1,1);
  \path[fill=white] (9,13) rectangle ++(1,1);
  \path[fill=black] (10,13) rectangle ++(1,1);
  \path[fill=white] (11,13) rectangle ++(1,1);
  \path[fill=white] (12,13) rectangle ++(1,1);
  \path[fill=white] (13,13) rectangle ++(1,1);
  \path[fill=white] (14,13) rectangle ++(1,1);
  \path[fill=white] (15,13) rectangle ++(1,1);
  \path[fill=white] (16,13) rectangle ++(1,1);
  \path[fill=white] (17,13) rectangle ++(1,1);
  \path[fill=white] (18,13) rectangle ++(1,1);
  \path[fill=white] (19,13) rectangle ++(1,1);
  \path[fill=white] (0,12) rectangle ++(1,1);
  \path[fill=white] (1,12) rectangle ++(1,1);
  \path[fill=white] (2,12) rectangle ++(1,1);
  \path[fill=white] (3,12) rectangle ++(1,1);
  \path[fill=white] (4,12) rectangle ++(1,1);
  \path[fill=white] (5,12) rectangle ++(1,1);
  \path[fill=white] (6,12) rectangle ++(1,1);
  \path[fill=white] (7,12) rectangle ++(1,1);
  \path[fill=white] (8,12) rectangle ++(1,1);
  \path[fill=white] (9,12) rectangle ++(1,1);
  \path[fill=black] (10,12) rectangle ++(1,1);
  \path[fill=black] (11,12) rectangle ++(1,1);
  \path[fill=white] (12,12) rectangle ++(1,1);
  \path[fill=white] (13,12) rectangle ++(1,1);
  \path[fill=black] (14,12) rectangle ++(1,1);
  \path[fill=white] (15,12) rectangle ++(1,1);
  \path[fill=white] (16,12) rectangle ++(1,1);
  \path[fill=black] (17,12) rectangle ++(1,1);
  \path[fill=white] (18,12) rectangle ++(1,1);
  \path[fill=black] (19,12) rectangle ++(1,1);
  \path[fill=white] (0,11) rectangle ++(1,1);
  \path[fill=white] (1,11) rectangle ++(1,1);
  \path[fill=white] (2,11) rectangle ++(1,1);
  \path[fill=white] (3,11) rectangle ++(1,1);
  \path[fill=white] (4,11) rectangle ++(1,1);
  \path[fill=white] (5,11) rectangle ++(1,1);
  \path[fill=white] (6,11) rectangle ++(1,1);
  \path[fill=white] (7,11) rectangle ++(1,1);
  \path[fill=white] (8,11) rectangle ++(1,1);
  \path[fill=white] (9,11) rectangle ++(1,1);
  \path[fill=white] (10,11) rectangle ++(1,1);
  \path[fill=black] (11,11) rectangle ++(1,1);
  \path[fill=black] (12,11) rectangle ++(1,1);
  \path[fill=white] (13,11) rectangle ++(1,1);
  \path[fill=black] (14,11) rectangle ++(1,1);
  \path[fill=black] (15,11) rectangle ++(1,1);
  \path[fill=white] (16,11) rectangle ++(1,1);
  \path[fill=black] (17,11) rectangle ++(1,1);
  \path[fill=black] (18,11) rectangle ++(1,1);
  \path[fill=black] (19,11) rectangle ++(1,1);
  \path[fill=white] (0,10) rectangle ++(1,1);
  \path[fill=white] (1,10) rectangle ++(1,1);
  \path[fill=white] (2,10) rectangle ++(1,1);
  \path[fill=white] (3,10) rectangle ++(1,1);
  \path[fill=white] (4,10) rectangle ++(1,1);
  \path[fill=white] (5,10) rectangle ++(1,1);
  \path[fill=white] (6,10) rectangle ++(1,1);
  \path[fill=white] (7,10) rectangle ++(1,1);
  \path[fill=white] (8,10) rectangle ++(1,1);
  \path[fill=white] (9,10) rectangle ++(1,1);
  \path[fill=white] (10,10) rectangle ++(1,1);
  \path[fill=white] (11,10) rectangle ++(1,1);
  \path[fill=black] (12,10) rectangle ++(1,1);
  \path[fill=black] (13,10) rectangle ++(1,1);
  \path[fill=white] (14,10) rectangle ++(1,1);
  \path[fill=black] (15,10) rectangle ++(1,1);
  \path[fill=black] (16,10) rectangle ++(1,1);
  \path[fill=black] (17,10) rectangle ++(1,1);
  \path[fill=black] (18,10) rectangle ++(1,1);
  \path[fill=black] (19,10) rectangle ++(1,1);
  \path[fill=white] (0,9) rectangle ++(1,1);
  \path[fill=white] (1,9) rectangle ++(1,1);
  \path[fill=white] (2,9) rectangle ++(1,1);
  \path[fill=white] (3,9) rectangle ++(1,1);
  \path[fill=white] (4,9) rectangle ++(1,1);
  \path[fill=white] (5,9) rectangle ++(1,1);
  \path[fill=white] (6,9) rectangle ++(1,1);
  \path[fill=white] (7,9) rectangle ++(1,1);
  \path[fill=white] (8,9) rectangle ++(1,1);
  \path[fill=white] (9,9) rectangle ++(1,1);
  \path[fill=white] (10,9) rectangle ++(1,1);
  \path[fill=white] (11,9) rectangle ++(1,1);
  \path[fill=white] (12,9) rectangle ++(1,1);
  \path[fill=white] (13,9) rectangle ++(1,1);
  \path[fill=black] (14,9) rectangle ++(1,1);
  \path[fill=white] (15,9) rectangle ++(1,1);
  \path[fill=white] (16,9) rectangle ++(1,1);
  \path[fill=white] (17,9) rectangle ++(1,1);
  \path[fill=white] (18,9) rectangle ++(1,1);
  \path[fill=white] (19,9) rectangle ++(1,1);
  \path[fill=white] (0,8) rectangle ++(1,1);
  \path[fill=white] (1,8) rectangle ++(1,1);
  \path[fill=white] (2,8) rectangle ++(1,1);
  \path[fill=white] (3,8) rectangle ++(1,1);
  \path[fill=white] (4,8) rectangle ++(1,1);
  \path[fill=white] (5,8) rectangle ++(1,1);
  \path[fill=white] (6,8) rectangle ++(1,1);
  \path[fill=white] (7,8) rectangle ++(1,1);
  \path[fill=white] (8,8) rectangle ++(1,1);
  \path[fill=white] (9,8) rectangle ++(1,1);
  \path[fill=white] (10,8) rectangle ++(1,1);
  \path[fill=white] (11,8) rectangle ++(1,1);
  \path[fill=white] (12,8) rectangle ++(1,1);
  \path[fill=white] (13,8) rectangle ++(1,1);
  \path[fill=black] (14,8) rectangle ++(1,1);
  \path[fill=black] (15,8) rectangle ++(1,1);
  \path[fill=white] (16,8) rectangle ++(1,1);
  \path[fill=black] (17,8) rectangle ++(1,1);
  \path[fill=white] (18,8) rectangle ++(1,1);
  \path[fill=black] (19,8) rectangle ++(1,1);
  \path[fill=white] (0,7) rectangle ++(1,1);
  \path[fill=white] (1,7) rectangle ++(1,1);
  \path[fill=white] (2,7) rectangle ++(1,1);
  \path[fill=white] (3,7) rectangle ++(1,1);
  \path[fill=white] (4,7) rectangle ++(1,1);
  \path[fill=white] (5,7) rectangle ++(1,1);
  \path[fill=white] (6,7) rectangle ++(1,1);
  \path[fill=white] (7,7) rectangle ++(1,1);
  \path[fill=white] (8,7) rectangle ++(1,1);
  \path[fill=white] (9,7) rectangle ++(1,1);
  \path[fill=white] (10,7) rectangle ++(1,1);
  \path[fill=white] (11,7) rectangle ++(1,1);
  \path[fill=white] (12,7) rectangle ++(1,1);
  \path[fill=white] (13,7) rectangle ++(1,1);
  \path[fill=white] (14,7) rectangle ++(1,1);
  \path[fill=black] (15,7) rectangle ++(1,1);
  \path[fill=black] (16,7) rectangle ++(1,1);
  \path[fill=black] (17,7) rectangle ++(1,1);
  \path[fill=black] (18,7) rectangle ++(1,1);
  \path[fill=black] (19,7) rectangle ++(1,1);
  \path[fill=white] (0,6) rectangle ++(1,1);
  \path[fill=white] (1,6) rectangle ++(1,1);
  \path[fill=white] (2,6) rectangle ++(1,1);
  \path[fill=white] (3,6) rectangle ++(1,1);
  \path[fill=white] (4,6) rectangle ++(1,1);
  \path[fill=white] (5,6) rectangle ++(1,1);
  \path[fill=white] (6,6) rectangle ++(1,1);
  \path[fill=white] (7,6) rectangle ++(1,1);
  \path[fill=white] (8,6) rectangle ++(1,1);
  \path[fill=white] (9,6) rectangle ++(1,1);
  \path[fill=white] (10,6) rectangle ++(1,1);
  \path[fill=white] (11,6) rectangle ++(1,1);
  \path[fill=white] (12,6) rectangle ++(1,1);
  \path[fill=white] (13,6) rectangle ++(1,1);
  \path[fill=white] (14,6) rectangle ++(1,1);
  \path[fill=white] (15,6) rectangle ++(1,1);
  \path[fill=black] (16,6) rectangle ++(1,1);
  \path[fill=white] (17,6) rectangle ++(1,1);
  \path[fill=black] (18,6) rectangle ++(1,1);
  \path[fill=black] (19,6) rectangle ++(1,1);
  \path[fill=white] (0,5) rectangle ++(1,1);
  \path[fill=white] (1,5) rectangle ++(1,1);
  \path[fill=white] (2,5) rectangle ++(1,1);
  \path[fill=white] (3,5) rectangle ++(1,1);
  \path[fill=white] (4,5) rectangle ++(1,1);
  \path[fill=white] (5,5) rectangle ++(1,1);
  \path[fill=white] (6,5) rectangle ++(1,1);
  \path[fill=white] (7,5) rectangle ++(1,1);
  \path[fill=white] (8,5) rectangle ++(1,1);
  \path[fill=white] (9,5) rectangle ++(1,1);
  \path[fill=white] (10,5) rectangle ++(1,1);
  \path[fill=white] (11,5) rectangle ++(1,1);
  \path[fill=white] (12,5) rectangle ++(1,1);
  \path[fill=white] (13,5) rectangle ++(1,1);
  \path[fill=white] (14,5) rectangle ++(1,1);
  \path[fill=white] (15,5) rectangle ++(1,1);
  \path[fill=white] (16,5) rectangle ++(1,1);
  \path[fill=black] (17,5) rectangle ++(1,1);
  \path[fill=white] (18,5) rectangle ++(1,1);
  \path[fill=white] (19,5) rectangle ++(1,1);
  \path[fill=white] (0,4) rectangle ++(1,1);
  \path[fill=white] (1,4) rectangle ++(1,1);
  \path[fill=white] (2,4) rectangle ++(1,1);
  \path[fill=white] (3,4) rectangle ++(1,1);
  \path[fill=white] (4,4) rectangle ++(1,1);
  \path[fill=white] (5,4) rectangle ++(1,1);
  \path[fill=white] (6,4) rectangle ++(1,1);
  \path[fill=white] (7,4) rectangle ++(1,1);
  \path[fill=white] (8,4) rectangle ++(1,1);
  \path[fill=white] (9,4) rectangle ++(1,1);
  \path[fill=white] (10,4) rectangle ++(1,1);
  \path[fill=white] (11,4) rectangle ++(1,1);
  \path[fill=white] (12,4) rectangle ++(1,1);
  \path[fill=white] (13,4) rectangle ++(1,1);
  \path[fill=white] (14,4) rectangle ++(1,1);
  \path[fill=white] (15,4) rectangle ++(1,1);
  \path[fill=white] (16,4) rectangle ++(1,1);
  \path[fill=black] (17,4) rectangle ++(1,1);
  \path[fill=black] (18,4) rectangle ++(1,1);
  \path[fill=black] (19,4) rectangle ++(1,1);
  \path[fill=white] (0,3) rectangle ++(1,1);
  \path[fill=white] (1,3) rectangle ++(1,1);
  \path[fill=white] (2,3) rectangle ++(1,1);
  \path[fill=white] (3,3) rectangle ++(1,1);
  \path[fill=white] (4,3) rectangle ++(1,1);
  \path[fill=white] (5,3) rectangle ++(1,1);
  \path[fill=white] (6,3) rectangle ++(1,1);
  \path[fill=white] (7,3) rectangle ++(1,1);
  \path[fill=white] (8,3) rectangle ++(1,1);
  \path[fill=white] (9,3) rectangle ++(1,1);
  \path[fill=white] (10,3) rectangle ++(1,1);
  \path[fill=white] (11,3) rectangle ++(1,1);
  \path[fill=white] (12,3) rectangle ++(1,1);
  \path[fill=white] (13,3) rectangle ++(1,1);
  \path[fill=white] (14,3) rectangle ++(1,1);
  \path[fill=white] (15,3) rectangle ++(1,1);
  \path[fill=white] (16,3) rectangle ++(1,1);
  \path[fill=white] (17,3) rectangle ++(1,1);
  \path[fill=black] (18,3) rectangle ++(1,1);
  \path[fill=black] (19,3) rectangle ++(1,1);
  \path[fill=white] (0,2) rectangle ++(1,1);
  \path[fill=white] (1,2) rectangle ++(1,1);
  \path[fill=white] (2,2) rectangle ++(1,1);
  \path[fill=white] (3,2) rectangle ++(1,1);
  \path[fill=white] (4,2) rectangle ++(1,1);
  \path[fill=white] (5,2) rectangle ++(1,1);
  \path[fill=white] (6,2) rectangle ++(1,1);
  \path[fill=white] (7,2) rectangle ++(1,1);
  \path[fill=white] (8,2) rectangle ++(1,1);
  \path[fill=white] (9,2) rectangle ++(1,1);
  \path[fill=white] (10,2) rectangle ++(1,1);
  \path[fill=white] (11,2) rectangle ++(1,1);
  \path[fill=white] (12,2) rectangle ++(1,1);
  \path[fill=white] (13,2) rectangle ++(1,1);
  \path[fill=white] (14,2) rectangle ++(1,1);
  \path[fill=white] (15,2) rectangle ++(1,1);
  \path[fill=white] (16,2) rectangle ++(1,1);
  \path[fill=white] (17,2) rectangle ++(1,1);
  \path[fill=white] (18,2) rectangle ++(1,1);
  \path[fill=black] (19,2) rectangle ++(1,1);
  \path[fill=white] (0,1) rectangle ++(1,1);
  \path[fill=white] (1,1) rectangle ++(1,1);
  \path[fill=white] (2,1) rectangle ++(1,1);
  \path[fill=white] (3,1) rectangle ++(1,1);
  \path[fill=white] (4,1) rectangle ++(1,1);
  \path[fill=white] (5,1) rectangle ++(1,1);
  \path[fill=white] (6,1) rectangle ++(1,1);
  \path[fill=white] (7,1) rectangle ++(1,1);
  \path[fill=white] (8,1) rectangle ++(1,1);
  \path[fill=white] (9,1) rectangle ++(1,1);
  \path[fill=white] (10,1) rectangle ++(1,1);
  \path[fill=white] (11,1) rectangle ++(1,1);
  \path[fill=white] (12,1) rectangle ++(1,1);
  \path[fill=white] (13,1) rectangle ++(1,1);
  \path[fill=white] (14,1) rectangle ++(1,1);
  \path[fill=white] (15,1) rectangle ++(1,1);
  \path[fill=white] (16,1) rectangle ++(1,1);
  \path[fill=white] (17,1) rectangle ++(1,1);
  \path[fill=white] (18,1) rectangle ++(1,1);
  \path[fill=black] (19,1) rectangle ++(1,1);
  \path[fill=white] (0,0) rectangle ++(1,1);
  \path[fill=white] (1,0) rectangle ++(1,1);
  \path[fill=white] (2,0) rectangle ++(1,1);
  \path[fill=white] (3,0) rectangle ++(1,1);
  \path[fill=white] (4,0) rectangle ++(1,1);
  \path[fill=white] (5,0) rectangle ++(1,1);
  \path[fill=white] (6,0) rectangle ++(1,1);
  \path[fill=white] (7,0) rectangle ++(1,1);
  \path[fill=white] (8,0) rectangle ++(1,1);
  \path[fill=white] (9,0) rectangle ++(1,1);
  \path[fill=white] (10,0) rectangle ++(1,1);
  \path[fill=white] (11,0) rectangle ++(1,1);
  \path[fill=white] (12,0) rectangle ++(1,1);
  \path[fill=white] (13,0) rectangle ++(1,1);
  \path[fill=white] (14,0) rectangle ++(1,1);
  \path[fill=white] (15,0) rectangle ++(1,1);
  \path[fill=white] (16,0) rectangle ++(1,1);
  \path[fill=white] (17,0) rectangle ++(1,1);
  \path[fill=white] (18,0) rectangle ++(1,1);
  \path[fill=white] (19,0) rectangle ++(1,1);
\end{tikzpicture}
}
\end{center}
\vspace{-0.3cm}
\caption{Adjacency and reachability matrix.}\label{fig:reach_matr}

\vspace{0.3cm}
\resizebox{0.4\textwidth}{!}{
\[
  \begin{array}{c|cccccccc}
    M_0 & q_{00} & q_{01} & q_{10} & q_{11} & q_{20} & q_{21} & q_{30} & q_{31} \\ \hline
    q_{00}       &        &
    \cellcolor{lightgray}{\overset{S}{\ws}\overset{F}{\ws}\overset{L}{\ws}\overset{R}{\bs}} &
    \cellcolor{lightgray}{\overset{S}{\ws}\overset{F}{\ws}\overset{L}{\bs}\overset{R}{\ws}} &
    \cellcolor{lightgray}{\overset{S}{\ws}\overset{F}{\ws}\overset{L}{\ws}\overset{R}{\bs}} &
    \overset{S}{\ws}\overset{F}{\ws}\overset{L}{\ws}\overset{R}{\ws} &
    \cellcolor{lightgray}{\overset{S}{\ws}\overset{F}{\ws}\overset{L}{\ws}\overset{R}{\bs}} &
    \overset{S}{\ws}\overset{F}{\ws}\overset{L}{\ws}\overset{R}{\ws} &
    \overset{S}{\ws}\overset{F}{\ws}\overset{L}{\ws}\overset{R}{\ws} \\ [6pt]
%     q_{00}       &        & \ws\ws\ws\bs & \ws\ws\bs\ws & \ws\ws\ws\bs & \ws\ws\ws\ws & \ws\ws\ws\bs & \ws\ws\ws\ws & \ws\ws\ws\ws \\ [6pt]
    q_{01}       &        &              & \ws\ws\ws\ws & \cellcolor{lightgray}{\ws\ws\bs\ws} & \ws\ws\ws\ws & \ws\ws\ws\ws & \ws\ws\ws\ws & \ws\ws\ws\ws \\ [6pt]
    q_{10}       &        &              &              & \cellcolor{lightgray}{\ws\ws\bs\ws} & \cellcolor{lightgray}{\ws\ws\ws\bs} & \cellcolor{lightgray}{\ws\ws\bs\ws} & \ws\ws\ws\ws & \cellcolor{lightgray}{\ws\ws\ws\bs} \\ [6pt]
    q_{11}       &        &              &              &              & \ws\ws\ws\ws & \cellcolor{lightgray}{\ws\ws\ws\bs} & \ws\ws\ws\ws & \ws\ws\ws\ws \\ [6pt]
    q_{20}       &        &              &              &              &              & \cellcolor{lightgray}{\ws\ws\bs\ws} & \cellcolor{lightgray}{\ws\ws\ws\bs} & \cellcolor{lightgray}{\ws\ws\bs\ws} \\ [6pt]
    q_{21}       &        &              &              &              &              &              & \ws\ws\ws\ws & \cellcolor{lightgray}{\ws\ws\ws\bs} \\ [6pt]
    q_{30}       &        &              &              &              &              &              &              & \cellcolor{lightgray}{\ws\ws\bs\bs} \\ [6pt]
    q_{31}       &        &              &              &              &              &              &              &              \\ [6pt]
  \end{array}
\]
}
\caption{Initial parse chart configuration.}\label{fig:initial_pc}

\vspace{0.3cm}
\resizebox{0.4\textwidth}{!}{
\input{figures/pc_final}
}

\caption{Final parse chart configuration.}\label{fig:final_pc}

\begin{center}
\resizebox{0.4\textwidth}{!}{
  \includegraphics{figures/gre}
}
\end{center}
\vspace{-0.3cm}
\caption{Regular expression denoting $\mathcal{L}(G_\cap)$.}\label{fig:re_tree}
\end{wrapfigure}

As a concrete example, suppose we have the string, $\err\sigma=\texttt{())}$ and wish to balance the parentheses. We will initially have the Levenshtein automaton, $A$, depicted in Fig.~\ref{fig:ex_atm}. To check for non-emptiness, we will perform the following procedure. Suppose we have a CNF CFG, $G'= \big\{S \rightarrow L R, S \rightarrow L F, S \rightarrow S S, F \rightarrow S R, L \rightarrow \hspace{-0.05cm}\texttt{(}, R \rightarrow\hspace{-0.05cm}\texttt{)}\big\}$ and let us assume an ordering of $S, F, L, R$ on $V$.

First, we need to order the automata states by increasing longest-path distance from $q_0$. One approach would be to topologically sort the adjacency matrix. While some form of sorting is unavoidable for arbitrary ANFAs, if we know ahead of time that our structure is a Levenshtein automaton, we can simply enumerate its state space by increasing Manhattan distance from the origin. % using, e.g., the Cantor pairing function to construct a valid ordering. This ordering will form the row and column indices of our intersection matrix, and each entry will represent the existence of some path between a two states yielding a given nonterminal.
So, a valid ordering on $Q$ would be $q_{00}, q_{01}, q_{10}, q_{11}, q_{20}, q_{21}, q_{30}, q_{31}$. Now, we want to compute whether $[\mathcal{L}(G')\cap \mathcal{L}(A) \neq \varnothing]$.

Under such an ordering, the adjacency matrix takes an upper triangular form and becomes the template for the initial parse chart, $M_0$ (Fig.~\ref{fig:initial_pc}). Each entry of this chart corresponds to a vector of expressions $E^{|V|}$ with at least one expression denoting a nonempty language. Likewise, the reachability matrix signifies a subset of state pairs which can participate in the language intersection. The adjacency and reachability matrices will always cover the expression vectors of the initial and final parse charts, respectively. In other words, we may safely ignore absent $\langle q, q'\rangle$ pairs in the reachability matrix, as these state pairs definitely cannot participate in the intersection.

From the reachability matrix we can construct the parse chart via matrix exponentiation. We note that n-step reachability constrains n-step parseability, i.e., $\sum_{i=0}^n A^i[q, q'] = \ws \vdash M_n[q, q', v] = \ws$, thus we can avoid substantial work via memoization. In this example, since $M_\infty[q_{00}, q_{31}, S] = \bs$, this implies that $\mathcal{L}(A)\cap \mathcal{L}(G') \neq \varnothing$, hence $\text{LED}(\sigma, G) = 1$. Using the same matrix, we will then perform a second pass to construct regular expressions representing finite languages for each nonempty constituent. Once again, we can skip $\langle q, q', v\rangle$ entries when $M_\infty[q, q', v] = \ws$ to hasten convergence.

\enlargethispage{4\baselineskip}
Just as before, we will define $\oplus, \otimes$ over GRE vectors, where $X \otimes Z = [X_x\cdot Z_z \mid (w\rightarrow xz) \in P]_{w\in V}$ and $X \oplus Z= [ X_w\vee Z_w ]_{w\in V}$. Finally, we will repeat the matrix exponential, using $M_\infty$ in the binary domain as a guide. This allows us to construct the regular expression tree for $S_\cap = q_{00}Sq_{20}\vee q_{00}Sq_{31}$ shown in Fig.~\ref{fig:re_tree}. Once this regex is constructed, decoding becomes simply a matter of invoking \texttt{choose}$(S_\cap)$. In this case there are only a few choices, but in general, there can be a vast multitude.

\clearpage

\section{Measuring the language intersection}\label{sec:measurement}

We will now attempt to put a probability distribution over the language intersection. We shall start with a few cursory but illuminative approaches, then proceed towards a more refined solution.

\subsection{Mode collapse}

Ordinarily, one might think to train a top-down PCFG sampler using a treebank of well-formed code snippets, however this method is highly degenerate in the finite case, exhibiting poor sample diversity. Consider an illustrative pathological case for top-down ancestral (TDA) sampling:
$$
G=\left\{ S \rightarrow A\:B \: \left(\frac{10^5 - 1}{10^5}\right), \hspace{2pt}
     S \rightarrow C\:C \: \left(\frac{1}{10^5}\right), \hspace{2pt}
     A \rightarrow a \: (1), \hspace{2pt}
     B  \rightarrow b \: (1), \hspace{2pt}
     C  \rightarrow a \: \left(\frac{1}{26}\right) \mid \ldots \mid z \: \left(\frac{1}{26}\right)\right\}
$$
Such a sampler will almost always yield $a b$, but most of $\mathcal{L}(G)$ is concealed in the hidden branch, $S \rightarrow C C$. Though a contrived example, it illustrates why TDA sampling is unviable: our sampler should match the true distribution over the finite CFL, not the PCFG's local approximation thereof.

\subsection{Exact enumeration}

To correct for mode collapse, a brute force solution would be to simply generate every tree. While the whole set can be materialized in some cases when the intersection language is small, this strategy is clearly suboptimal due to its worst-case complexity. Nevertheless, it is useful for checking completeness. To enumerate trees, we first need the total number of trees, which is denoted $|e|$.

\begin{definition}[Cardinality]
  $|e|: E \rightarrow \mathbb{N} =$ \begin{cases}
    1           & \text{if } e \in \Sigma \\
    x \times z  & \text{if } e = x \cdot z \\
    x + z       & \text{if } e = x \vee z
  \end{cases}\\
\end{definition}

\begin{theorem}[Enumeration]
  To enumerate, we can invoke $\bigcup_{i = 0}^{|R|}\{\texttt{enum}(R, i)\}$:\\

  $\texttt{enum}(e, n): E \times \mathbb{N} \rightarrow \Sigma^*$ = \begin{cases}
       e &\text{if } e \in \Sigma \\
       \texttt{enum}\big(x, \lfloor \frac{n}{|z|} \rfloor\big) \cdot \texttt{enum}\big(z,\, n \bmod |z|\big)  &\text{if } e = x \cdot z \\
       \texttt{enum}\big((x, z)_{\min(1, \lfloor\frac{n}{|x|}\rfloor)}, n-|x|\min(1, \lfloor\frac{n}{|x|}\rfloor)\big) &\text{if } e = x \vee z
  \end{cases}
\end{theorem}

This can be converted to a uniform sampler by drawing integers without replacement using a pseudorandom number generator, however, if $|e|$ is very large, \texttt{enum} can fail to capture modes.

\subsection{The problem with ambiguity}

The main problem with the previous approach is that it counts distinct trees, which overcounts the total number of words, $|\mathcal{L}(G_\cap)|$. Since the Levenshtein automaton can be ambiguous, this causes certain repairs to be overrepresented, resulting in a pernicious bias. Consider, for example,

\begin{lemma}\label{lemma:ambiguity}
If the FSA, $\alpha$, is ambiguous, then the intersection grammar, $G_\cap$, can be ambiguous.
\end{lemma}

\begin{proof}
Let $\ell$ be the language defined by $G=\{S\rightarrow LR, L \rightarrow\texttt{(}, R \rightarrow\texttt{)}\}$, where $\alpha=L(\err\sigma, 2)$, the broken string $\err\sigma$ is $\texttt{)(}$, and $\mathcal{L}(G_\cap) = \ell \cap \mathcal{L}(\alpha)$. Then, $\mathcal{L}(G_\cap)$ contains the following two identical repairs: \texttt{\hlred{)}(\hlgreen{)}} with the parse $S \rightarrow q_{00}Lq_{21}\phantom{.}q_{21}Rq_{22}$, and \texttt{\hlorange{(}\hlorange{)}} with the parse $S \rightarrow q_{00}Lq_{11}\phantom{.}q_{11}Rq_{22}$.
\end{proof}

\noindent We would expect the underlying sample space to be a proper set, \textit{not} a multiset.

\subsection{Disambiguation}\label{sec:transfer_method}

To count the number of distinct repairs, we will need to convert $G_\cap$ to an automaton. Since $\mathcal{L}(G_\cap)$ is finite, it must be regular a fortiori. Recalling the definition for an NFA, $\langle Q, \Sigma, \delta, q_\alpha: Q, F \subseteq Q \rangle$, and star-free regex, $e \rightarrow \Sigma \mid e \lor e \mid e \land e$, we will proceed by structural induction on the regex:

\begin{equation*}
N(e) =
\begin{cases}
  \begin{alignedat}{8}
    &\big\langle \{q_\alpha, q_\omega\}
    &&\,,\, \{ q_\alpha \overset{e}{\rightarrow} q_\omega \}
    &&\,,\, q_\alpha\,\,, \{q_{\omega}\}
    &&\big\rangle
    &&\:\: \text{if } e \in \Sigma \\[0.5em]

    &\big\langle Q_{x} \cup Q_{z}
    &&\,,\, \{ q \overset{s}{\rightarrow} q_{\alpha z} \mid (q \overset{s}{\rightarrow} q_{\omega}^{\in F_x}) \in \delta_x \} \cup \delta_{x} \cup \delta_{z}
    &&\,,\, q_{\alpha x}\,, F_{z}
    &&\big\rangle
    &&\:\: \text{if } e = x \cdot z \\[0.5em]

    &\Bigg\langle   \begin{matrix} Q_{x}\cup \{q_{\alpha e}\}\:\cup\\ Q_{z} \cup \{q_{\omega e}\}\phantom{\:\cup} \end{matrix}
    &&\,,\, \begin{matrix}\{ q_{\alpha e} \overset{s}{\rightarrow} q \mid (q_{\alpha x, \alpha z} \overset{s}{\rightarrow} q)\in \delta_{x, z} \} \cup \delta_{x}\:\cup\\
    \{ q \overset{s}{\rightarrow} q_{\omega e} \mid (q \overset{s}{\rightarrow} q_{\omega}^{\in F_{x,z}})\in \delta_{x, z} \}\cup \delta_{z}\phantom{\,\:\cup}\end{matrix}
    &&\,,\, q_{\alpha e}\,, \{q_{\omega e}\}
    &&\Bigg\rangle
    &&\:\:\text{if } e = x \lor z\hspace{-0.5cm}\\[-0.5em]
    \multicolumn{9}{c}{\tiny{\text{- - - - - - - - - - - - or - - - - - - - - - - - -}}} \\[-0.5em]
    &\big\langle Q_{x} \cup Q_{z} \cup \{q_{\alpha e}\}
    &&\,,\, \{ q_{\alpha e} \overset{s}{\rightarrow} q \mid (q_{\alpha x, \alpha z} \overset{s}{\rightarrow} q) \in \delta_{x, z} \}  \cup \delta_{x} \cup \delta_{z}
    &&\,,\, q_{\alpha e}\,, F_{x} \cup F_{z}
    &&\big\rangle
    &&\:\: \text{if } e = x \lor z
  \end{alignedat}
\end{cases}\vspace{0.2cm}
\end{equation*}

\noindent Though less conventional than Thompson's construction, $N(e)$ avoids the creation of unnecessary $\varepsilon$ arcs. And while slightly more verbose, we find the topology induced by the first version of the $\lor$ case to be more favorable for minimization. Continuing with our running example from \S~\ref{sec:repair_ex}, we will use Brzozowski's algorithm~\cite{brzozowski1962canonical} to construct the unique minimal DFA, $D^*_\cap \equiv_\mathcal{L} G_\cap$: \vspace{-0.4cm}

\begin{figure}[H]
  \centering
  \includegraphics[width=0.27\textwidth]{figures/dyck_nfa}
  \includegraphics[width=0.27\textwidth]{figures/dyck_nfa_orig}
  \includegraphics[width=0.30\textwidth]{figures/dyck_dfa}
  \vspace{-0.2cm}
  \caption{FSA for $\mathcal{L}\big(L(\texttt{())}, 1)\big)\cap\mathcal{L}(G')$ (a) with or (b) without $\lor$-merging, and then (c) post-minimization.}\label{fig:fsas_for_re}
  \vspace{-0.2cm}
\end{figure}
Since $\mathcal{L}(G_\cap)$ is necessarily finite, we can infer that the corresponding DFA is acyclic and thus representable as an upper triangular adjacency matrix under a topological ordering of $\delta$. For any such DFA, we can ascertain the size of its language by counting walks from $q_\alpha$ to $q_\omega \in F$. Letting $A$ be the adjacency matrix for $D_\cap^*$, i.e., $A[q, q'] = \big[1 \text{ if } \exists s: \Sigma \text{ s.t. } (q \overset{s}{\rightarrow} q') \in \delta \text{ else } 0\big]$, the number of words it recognizes is given via the transfer matrix method~\cite{flajolet2009analytic}, that is,
\begin{align}
  C(A, q_\alpha, F): \mathbb{N}^{|Q|\times|Q|} \times Q \times 2^Q \rightarrow \mathbb{N} = \sum_{\mathclap{q_\omega \in F}}(I-A)^{-1}[q_\alpha, q_\omega] &= \sum_{\mathclap{q_\omega \in F}}\sum_{i = 0}^{\mathclap{|Q|-1}}A^i[q_\alpha, q_\omega]
\end{align}
\noindent Plugging in powers of the adjacency matrix for the DFA shown in Fig.~\ref{fig:fsas_for_re}.(c), we arrive at the total:
\begin{align}
(I-A)^{-1} &= \hspace{1cm}I + A \hspace{1cm}+\hspace{1.05cm} A^2 \hspace{1.07cm}+\hspace{1cm} A^3 \hspace{0.9cm}+\hspace{0.9cm} A^4\\
  &=\begin{tiny}\begin{pmatrix}
       1 & 1 &   &   &   &   \\
        & 1  & 1 & 1 &   &   \\
        &   & 1  &   & 1 &   \\
        &   &   & 1  & 1 &   \\
        &   &   &   & 1  & 1 \\
        &   &   &   &   & 1  \\
  \end{pmatrix} + \begin{pmatrix}
              &   & 1 & 1 &   &   \\
              &   &   &   & 2 &   \\
              &   &   &   &   & 1 \\
              &   &   &   &   & 1 \\
              &   &   &   &   &   \\
              &   &   &   &   &   \\
  \end{pmatrix} + \begin{pmatrix}
              &   &   &   & 2 &   \\
              &   &   &   &   & 2 \\
              &   &   &   &   &   \\
              &   &   &   &   &   \\
              &   &   &   &   &   \\
              &   &   &   &   &   \\
  \end{pmatrix} + \begin{pmatrix}
              &   &   &   &   & 2 \\
              &   &   &   &   &   \\
              &   &   &   &   &   \\
              &   &   &   &   &   \\
              &   &   &   &   &   \\
              &   &   &   &   &   \\
  \end{pmatrix}\end{tiny}\\
  &= \begin{tiny}\begin{pmatrix}
          1   & 1  & 1  & \underline{1} & 2 & \underline{2} \\
              & 1  & 1  & 1 & 2 & 2 \\
              &    & 1  &   & 1 & 1 \\
              &    &    & 1 & 1 & 1 \\
              &    &    &   & 1 & 1 \\
              &    &    &   &   & 1 \\
\end{pmatrix}\end{tiny} \text{ therefore, } \big|\mathcal{L}(D_\cap^*)\big| = C\big(A, q_0, \{q_3, q_5\}\big) = \underline{1} + \underline{2} = 3.
\end{align}
Note the model counting problem for arbitrary GREs is strictly harder than deciding intersection nonemptiness as it requires determinization, however, weak bounds may be obtained by applying $C$ to the FSA generated by $N(e)$ or by direct analysis of $e$. While the inequality $C_{D_\cap^*} \leq C_{N(e)} \leq |e|$ will hold, the bounds provided by the latter approximations may be vacuous, whereas $C_{D_\cap^*}$ is exact.

\clearpage\section{Implementation}\label{sec:implementation}

The implementation essentially consists of four stages, each dependent on its predecessor.

\begin{enumerate}
  \item $\texttt{lev\_build}: \Sigma^{|Q|-1} \times \mathbb{N}^{3} \rightarrow \text{NFA}$ -- constructs a Levenshtein NFA from the broken string.
  \item $\texttt{cfl\_fixpt}: \text{NFA} \times \text{CFG} \rightarrow \mathbb{B}^{|Q|\times |Q| \times |V|}$ -- computes the matrix exponential.
  \item $\texttt{reg\_build}: \mathbb{B}^{|Q|\times |Q| \times |V|} \times \text{CFG} \rightarrow \text{GRE}$ -- constructs the regular expression for $G_\cap$.
  \item $\texttt{reg\_dcode}: \text{GRE} \times \mathbb{N}^{|\Sigma|^{c\approx 3}} \hspace{-0.05cm}\times \mathbb{N} \rightarrow\hspace{-0.02cm} (\Sigma^+)^{k\approx 10}$ -- returns a small set of the most probable repairs.
\end{enumerate}

\noindent We will now explore the imperative pseudocode for each stage, starting with the Levenshtein automata constructor, which is a straightforward translation of the inference rules in \S~\ref{sec:repair_ex}.

\begin{algorithm}[H]
\caption{\texttt{lev\_build} pseudocode}
\label{alg:lev_build}
\begin{algorithmic}[1]
  \Procedure{\texttt{lev\_build}$(\sigma: \Sigma^n, d_{\max}: \mathbb{N})$}{} \Comment{Takes a string and maximum edit distance.}
  \State $Q, \delta \gets \varnothing$
  \For{$\langle h, j, i, k \rangle \textbf{ in } [0, n]^2\times[0, d_{\max}]^2$\vspace{1.34cm}}
    \State \vspace{-1.65cm}\[\hspace{1.15cm}\delta\,\gets \delta\,\cup\:\!\left\{
        \begin{alignedat}{9}
          &\;& q_{h,i} &\hspace{-0.1cm}\overset{{\color{orange}[\neq\sigma_{j+1}]}}{\rightarrow} &q_{j,k} &\qquad& \text{if}\;& h = j   &\:\land\:& i = k-1  &\qquad& \duparrow\\[-2pt]
          && q_{h,i}   &\overset{{\color{orange}[\neq\sigma_j]}}{\rightarrow} &q_{j,k} &&       \text{if}\;& h = j-1 &\:\land\:& i = k-1  &&       \ddiagarrow\\[-2pt]
          && q_{h,i}   &\overset{{\color{orange}[=\sigma_j]}}{\rightarrow}    &q_{j,k} &&       \text{if}\;& h = j-1 &\:\land\:& i = k    &&       \drightarrow\\[-2pt]
          && q_{h,i}   &\overset{{\color{orange}[=\sigma_j]}}{\rightarrow}    &q_{j,k} &&       \text{if}\;& 1 \leq j - h - 1 \leq d_{\max} &\:\land\:& 1 \leq k - i \leq d_{\max}   && \knightarrow\;
        \end{alignedat}
      \right\}\]
    \State $Q \gets Q \cup \{q_{h,i}, q_{j,k}\}$
  \EndFor
  \State $I \gets \{q_{0,0}\}, F \gets \{q_{i, j} \mid n - i + j \leq d_{\max}\}$
  \State \Return $\langle Q, \Sigma, \delta = Q \times (\Sigma\:{\color{orange}\rightarrow \mathbb{B}}) \times Q, q_\alpha, F\rangle$  \Comment{Returns a [nominal] Levenshtein automaton.}
\end{algorithmic}
\end{algorithm}\vspace{-0.2cm}

\noindent Next, the chart parser expects an acyclic NFA, a CNF grammar and returns a Boolean 3-tensor.

\begin{algorithm}[H]
\caption{\texttt{cfl\_fixpt} pseudocode}
\label{alg:cfl_fixpt}
\begin{algorithmic}[1]
\Require CFG must be in CNF and the NFA must be $\varepsilon$-free and acyclic (i.e., denote a finite language).
\Procedure{\texttt{cfl\_fixpt}$\big(\langle \Sigma, V, P, S\rangle: \text{CFG}, \langle Q, \Sigma, \delta, q_\alpha, F\rangle: \text{NFA}\big)$}{}
\State $R: \mathbb{B}^{|Q|\times |Q|} \gets \big[\bs \textbf{ if } \exists \sigma \in \Sigma^+ \mid q \overset{\sigma}{\rightsquigarrow} q' \textbf{ else } \ws\big]_{q,\,q'\,:\, Q}$ \Comment{Solve for reachability matrix.}
\State $M: \mathbb{B}^{|Q|\times |Q| \times |V|} \gets \big[\bs \textbf{ if } \exists s: \Sigma \mid (v \rightarrow s) \in P \land (q \overset{{\color{orange}\varphi}}{\rightarrow} q') \in \delta \land {\color{orange}\varphi(}s{\color{orange})} \textbf{ else } \ws\big]_{q,\,q'\,:\,Q,\,v\,:\,V}$
\For{$i \textbf{ in } \big[0, \lceil\log_2(|Q||V|)\rceil\big]$} \Comment{Solves matrix exponential, $\exp(M_0)$.}
\State $\textsc{done} \gets \bs$
\For{$\langle p, r, w \rangle \textbf{ in } Q^2\times V$} \Comment{Iterates one step of $M_{i+1} = M_i + M_i^2$.}
  \State $\textbf{if } M[p, r, w] \textbf{ or not } R[p, r] \textbf{ then continue}$
  \State $Q_{pr} \gets \big\{q: Q \mid R[p, q] \land R[q, r]\big\}$ \Comment{Consider reachable states between p and r.}
  \State $M[p, r, w] \gets \bs \textbf{ if } \exists q: Q_{pr}, x, z: V \mid M[p, q, x] \land M[q, r, z] \land (w \rightarrow x z) \in P \textbf { else } \ws$
  \State $\textbf{if } M[p, r, w] \textbf{ then } \textsc{done} \gets \ws$
\EndFor
\State $\textbf{if }\textsc{done} \textbf{ then break}$
\EndFor
\State \Return $M$ \Comment{Returns the completed Boolean parse chart.}
  \end{algorithmic}
\end{algorithm}\vspace{-0.2cm}

\noindent Note we may short-circuit for three reasons, if: $M_{i+1} = M_i$, when two states $q, q'$ are unreachable, or whenever a $\langle q, q', v\rangle$ is already present. Once we obtain $M_\infty$, we can immediately tell whether $\ell_\cap \neq \varnothing$ by checking whether $M_\infty[q_\alpha, q_\omega, S] = \bs$ for some $q_\omega: F$. Otherwise if no such $q_\omega$ exists, then $\ell_\cap$ must be empty and $d_\max$ should be enlarged before proceeding.

\noindent Now we can perform a second sweep over nonempty entries of the Boolean parse chart, reconstructing the provenance of each $\langle q, q', v\rangle$ constituent. For compactness it will be convenient to use a pointer-based representation of the regular expression instead of manipulating strings.

\begin{algorithm}[H]
\caption{\texttt{reg\_build} pseudocode}
\label{alg:reg_build}
\begin{algorithmic}[1]
  \Require Same as \texttt{cfl\_fixpt} (Alg.~\ref{alg:cfl_fixpt}), $M_{\mathbb{B}}[q_\alpha, q_\omega: F, S] = \bs$ for some $q_\omega$, and $M_{\mathbb{B}} = M_{\mathbb{B}} + M_{\mathbb{B}}^2$.
  \Procedure{\texttt{reg\_build}$\big(M_{\mathbb{B}}: \mathbb{B}^{|Q|\times |Q| \times |V|}, \langle \Sigma, V, P, S\rangle: \text{CFG}, \langle Q, \Sigma, \delta, q_\alpha, F\rangle: \text{NFA}\big)$}{}
  \State $P: \mathbb{B}^{|Q|\times |Q|} \gets \big[\bs \textbf{ if } \exists q: Q, v, v': V \mid M_{\mathbb{B}}[p, q, v] \land M_{\mathbb{B}}[q, r, v'] \textbf{ else } \ws\big]_{p,\,r\,:\,Q}$
  \State $M: \text{GRE}^{|Q|\times |Q| \times |V|} \gets \big[\{s: \Sigma \mid M[q, q', v] \land (q \overset{{\color{orange}\varphi}}{\rightarrow} q') \in \delta \land (v\rightarrow s) \in P \land {\color{orange}\varphi(}s{\color{orange})}\}\big]_{q,\,q'\,:\, Q,\,v\,:\,V}$
  \For{$i \textbf{ in } \big[0, \lceil\log_2(|Q||V|)\rceil\big]$}
  \State $M' \gets M$
  \For{$\langle p, r, w \rangle \textbf{ in } Q^2\times V$}
  \State $\textbf{if not } M_\mathbb{B}[p, r, w] \textbf{ then continue}$
  \State $Q_{pr} \gets \big\{q: Q \mid P[p, q] \land P[q, r]\big\}$ \Comment{Consider parseable states between p and r.}\vspace{0.2cm}
  \State \vspace{-0.42cm}\[\hspace{0.62cm}M'[p, r, w] \gets M[p, r, w] \vee \bigvee_{\mathclap{\substack{q\,:\,Q_{pr}\\x,\,z\,:\,V}}} \big\{M[p, q, x]\cdot M[q, r, z] \mid (w \rightarrow x z) \in P\big\}\]\vspace{-0.2cm}
  \EndFor
  \State $\textbf{if } M=M' \textbf{ then break else } M \gets M'$
  \EndFor \vspace{0.2cm}
  \State \vspace{-0.42cm}\[\hspace{-8.5cm}\textbf{return }\bigvee_{\mathclap{q_\omega\,:\,F}} M[q_\alpha, q_\omega, S]\vspace{-0.8cm}\] \Comment{Union regexes for all total parses yielding S.}\vspace{0.31cm}
\end{algorithmic}
\end{algorithm}

Finally, once we have the expression for $G_\cap$, we can decode it to extract a small set of candidates. Various strategies are possible here, and we opt for the simplest one. We use two priority queues to store partial and total trajectories, which are ranked by probability as estimated by a pretrained c-gram count tensor, $C$. Partial trajectories are greedily extended until termination, after which the trajectory it is diverted to the total queue, and the top-k total trajectories are returned.

\begin{algorithm}[H]
  \caption{\texttt{reg\_dcode} pseudocode}
  \label{alg:reg_dcode}
  \begin{algorithmic}[1]
  \Require We expect the shortest word to exceed the Markov order in length, $c < |\sigma|, \forall\sigma: \mathcal{L}(e)$.
  \Procedure{\texttt{reg\_dcode}$\big(e: \text{GRE}, C: \mathbb{N}^{|\Sigma|^{c\approx 3}}, k: \mathbb{N}\big)$}{}
    \State $\mathcal{T} \gets [], \mathcal{E} \gets \big[\langle \varepsilon^{c-1}, e\cdot \varepsilon^{c-1}, 0\rangle\big]$ \Comment{Initialize total and partial trajectories.}\vspace{0.5cm}
    \State \vspace{-0.55cm}\[\hspace{-4.58cm}\textbf{let } P(s: \Sigma \mid \sigma: \Sigma^{\geq c-1}) = \frac{C[\sigma_{|\sigma| - c + 1, |\sigma|}\cdot s]+ 1}{\sum_{s' \in \Sigma} C[\sigma_{|\sigma| - c + 1, |\sigma|}\cdot s']}\vspace{-0.7cm}\]\Comment{Define Markov transition probability.}\vspace{0.3cm}
    \Repeat
        \State $\langle\sigma, e, p\rangle \gets \textbf{pop } \mathcal{E}_0 \textbf{ off }\mathcal{E}$
        \State $\mathcal{E}' \gets \big[\langle\sigma\cdot a, \partial_a e, p + \ln P(a\mid \sigma) \rangle \mid a \in \texttt{follow}(e)\big]$
        \State $\mathcal{T}\hspace{0.05cm} \gets \mathcal{T} \texttt{++} \big[\langle \sigma, p \rangle \mid \langle \sigma, e, p\rangle \in \mathcal{E}' \land \varepsilon \in \mathcal{L}(e)\big]$
        \State $\mathcal{E}\phantom{'} \gets \big[\langle\sigma, e, p\rangle \in (\mathcal{E} \texttt{++} \mathcal{E}')\textbf{ sorted by } p\big]$
    \Until{interrupted or $\mathcal{E}$ is empty.}
    \State \Return $[\sigma \mid \langle\sigma, p\rangle \in \mathcal{T}_{0..k}\textbf{ sorted by } p]$ \Comment{Skim off top-k repairs by c-gram probability.}
  \end{algorithmic}
\end{algorithm}

\noindent Now, we have our shortlist of repairs and after cosmetic postprocessing, can present them to the user. With this approach, we can quickly generate a representative subset of $\ell_\cap$ within a fixed latency budget, e.g., \~100ms, or otherwise terminate early should we succeed in exhaustively generating it.

\clearpage\subsection{GPU translation}\label{sec:gpu_translation}

The foregoing architecture can be translated to series of high-performance GPU kernels. Our strategy will be to maximize GPU utilization by distributing the workload for each stage across as many independent threads as we can simultaneously dispatch. Each thread will be responsible for writing to a dedicated portion of a shared buffer without locking or external communication.

We will make the simplifying assumption that each GPU kernel is a pure function that takes as input a coordinate triple $r, c, v: \mathbb{N}$ and one or more flat buffers $b_1: \mathbb{N}^{d_1}, \ldots b_n: \mathbb{N}^{d_n}$ containing encoded data, does some arithmetic, and returns a single output buffer, $b_{\text{out}}: \mathbb{N}^{d}$. On a GPU, all memory must be sized ahead of time, as dynamic allocation is forbidden during a GPU kernel's execution. The main challenge of GPU programming then, becomes careful memory management and efficiently mapping aggregate datatypes to and from the integers. Conceptually, each $\langle r, c, v\rangle$ triple will be dispatched to a single GPU thread with global read access to the input buffers and exclusive write access to a contiguous region of the output buffer. Consistent with the PRAM model used in Theorem~\ref{thm:parallel_decision_complexity}, each thread will correspond to a single processor, with $|Q|^2|V|$ threads in total. Absent a GPU, this can be rewritten as a triply-nested loop, subject to additional latency.

For the CFG and NFA datatypes, we elect to use a dense representation $\mathbb{B}^{|V|\times|V|\times|V|}$ and $\mathbb{B}^{|Q|\times|Q|\times |\Sigma|}$ due to the tripartite coordinate structure and thread dispatching API. While these datatypes can be encoded sparsely as $\mathbb{N}^{3|P|}$ and $\mathbb{N}^{3|\delta|}$, for most repair instances and memory configurations representation size is not a bottleneck. It will be helpful to define characteristic functions $\texttt{nt\_enc}: \Sigma \rightarrow 2^V$, $\texttt{nt\_dec}: V \rightarrow 2^\Sigma$ for nonterminal encoding and decoding, and index sets $\Sigma \leftrightarrow \mathbb{N}$, $V \leftrightarrow \mathbb{N}$, $Q \leftrightarrow \mathbb{N}$ for getting in and out of the \texttt{uint} domain, with $|V|, |Q| \lesssim 10^3$ adjustable upward if memory permits.

The parse chart $M$ can be represented as a bit-packed integer matrix $\texttt{uint32}^{|Q|\times|Q|\times|V|}$, whose layout testifies to four properties of each $\langle q, q', v\rangle$ triple: (1) the first bit encodes the dis/equality predicate ${\color{orange}\varphi}}$, (2) the next 25 bits designate terminal participation $\big(\text{if }\exists s: \Sigma. \varphi(s) \land (q \overset{{\color{orange}\varphi}}{\rightarrow} q')\in \delta\big)$, (3)~the next five bits memoize the minimum $i_{\min}$ such that $M_{i_{\min}}[q, q', v] \& 1 = \bs$ for short-circuiting (see Line \#7 of Alg.~\ref{alg:reg_build}), and (4) the lowest-order bit denotes parsability, i.e., $q\rightsquigarrow q'\vdash v$. Note the decoder must acknowledge the possibility that $v$ can simultaneously parse (a) an arc $q\rightarrow q'$ and (b) a path $q\rightsquigarrow q'$, so each branch can be explored. This is depicted below in little-endian format:\vspace{-0.1cm}

\[
\big[\overset{\overset{{\color{orange}=/\neq}}{\Updownarrow}}{\bs}, \overbrace{\ws, \ws, \ldots, \bs, \ws, \bs}^{s:\:\Sigma \Leftrightarrow \mathbb{B}^{25}}, \overbrace{\ws, \ws, \ws, \ws, \bs}^{i_{\min}:\:\mathbb{N}_{\leq 32}\Leftrightarrow \mathbb{B}^{5}}, \overset{\overset{v:\:V}{\Updownarrow}}{\bs}\big]: \texttt{uint32}
\]

Once \texttt{cfl\_fixpt} (Alg.~\ref{alg:cfl_fixpt}) is complete, we can calculate the total amount of memory needed to allocate $G_\cap$ by counting constituents in the parse chart. Being an algebraic datatype, the GRE can be flattened according to a variety of allocation models. We will use the following memory layout,\vspace{-0.1cm}

\[
\big[\overbrace{\underset{\texttt{bp}_0}{2}, \underset{\texttt{bp}_1}{7}, \ldots, \underset{\texttt{bp}_{c-2}}{1}, \underset{\texttt{bp}_{c-1}}{3}}^{\texttt{bp\_counts}}, \overbrace{0, 4, \ldots, \underset{\texttt{bp}_{c-2}}{n-8}, \underset{\texttt{bp}_{c-1}}{n-6}}^{\texttt{bp\_offsets}}, \overbrace{\underbrace{\underline{59,83}, \underline{64, 152}}_{\texttt{bp}_0}, \ldots, \underbrace{\underline{34, 83}}_{\texttt{bp}_{c-2}}, \underbrace{\underline{22,74},  \underline{74, 90}, \underline{16, 66}}_{\texttt{bp}_{c-1}}}^{\texttt{bp\_storage}}\big]:  \texttt{uint32}^n
\]

\noindent where each $\texttt{bp}_i$ represents a nonempty $\langle q, q', v\rangle$ constituent with at least one back-pointer pair, $\texttt{bp\_count}(p, r, w) = \big|\big\{\langle q, x, z\rangle\mid  M[p, q, x] \land M[q, r, z] \land (w \rightarrow xz)\in P\big\}\big|$ counts the number of unique backpointers held by each nonterminal $w$ parseable from $p \rightsquigarrow r$, and $\texttt{bp\_storage}$ stores pointers to memory locations in the same data structure. These pointers should also be tied to locations in the parse chart $M[q, q', v]$ to recover the terminal subsets for unit productions.

In total, the GPU should have at least 4 GB of onboard memory to accommodate language intersections with up to $10^3$ states and nonterminals $\big(|Q|^2\times|V| \lesssim 10^6\times10^3\times 32 \text{ bits} \approx \text{4 GB}\big)$, however occupancy can be roughly halved by exploiting the upper-triangular structure of $M$.

\clearpage\subsection{Training the reranker}

After decoding, we have a list of repair candidates that are all valid, nearby and at least somewhat plausible, however it is possible this list may be quite long. No reasonable user would be expected to skim through more than a few dozen candidates to select their intended repair, especially since they could have presumably written it themselves in a few seconds. So, we will proceed to rerank the list. If we can guarantee the candidate repairs are sufficiently exhaustive, they should include with high probability the true repair, which then need only be surfaced into the user's field of view.

The ensuing method falls under the umbrella of the \textit{learning-to-rank} (LTR) problem in machine learning -- using their terminology, the broken code snippet would be called a \textit{query} and the list of repairs, \textit{documents}. To discuss the reranker, we must now overload some concepts, so the reader is trusted to contextually interpret $\mathcal{L}$ as denoting a \textit{loss} instead of a language, and the derivative~\footnote{There is a connection to Brzozowski's derivative, but to refrain from digression here we refer the reader to ~\cite{elliott2019generalized} for details.} as the directional rate of change of a differentiable manifold over the parameter space of a neural network. Matrix multiplication remains more or less the same, except now over the reals.

The reranker employs a transformer-encoder architecture to map both the query (broken code snippet, denoted $\err\sigma: \bar\ell$) and the document (candidate repair, denoted $\sigma: \ell_\cap$) to a $p$-dimensional real vector space $\mathbb{R}^p$. We elide the definition of a transformer-encoder (see Strobl et al.'s survey~\cite{strobl2024formal}), except to say that it is a function, $\text{E}_\theta$, which takes a string and a positional encoding, and returns an embedding, $\text{E}_\theta: (\Sigma\times \mathbb{N})^n \rightarrow \mathbb{R}^p$ where $\theta$ are learnable parameters. To these, we will introduce a Levenshtein alignment $(\text{LA}: \Sigma^n \times \Sigma^m \rightarrow \mathbb{N}^{[m, n]})$ as a third argument that, when applied to a query-document pair, will produce a vector tracking edit locations and types. Finally, a multilayer perceptron $(\text{MLP}_\theta: \mathbb{R}^p \times \mathbb{R}^p \rightarrow \mathbb{R}^+)$ processes the embedding to produce a relevance score:

\[
f_\theta(\err\sigma, \sigma): \Sigma^n \times \Sigma^m \rightarrow \mathbb{R}^+ = \text{MLP}_{\theta}\Big(\text{E}_\theta\big(\err\sigma, [i]_{i\in [0, n)}\big), \text{E}_\theta'\big(\sigma, [i]_{i \in [0, m)}, \text{LA}(\err\sigma, \sigma)\big)\Big)
\]

\noindent Our training objective will be to minimize the tempered \textit{softmax} or listwise cross-entropy loss,

\begin{equation}
\mathcal{L}(\theta) = -\sum_{q \in \mathcal{Q}} \log \left( \frac{\exp\big(f_\theta(q, d^*)\tau^{-1}\big)}{\sum_{d \in \mathcal{D}_q} \exp\big(f_\theta(q, d)\tau^{-1}\big)} \right)
\end{equation}

\noindent where $\mathcal{Q}$ is the set of training queries, $\mathcal{D}_q$ is the set of candidate repairs for query $q$, and $d^* \in \mathcal{D}_q$ is true repair. The temperature parameter, $\tau$ controls the sharpness of the softmax distribution, encouraging parameter settings that result in the true repair being assigned higher priority -- the closer to zero, the greater the loss will be for underestimating the relevance of the true repair.

\begin{wrapfigure}{r}{0.40\textwidth}
\resizebox{0.4\textwidth}{!}{%
\begin{equation*}
  \begin{array}{r@{\hskip 0.5em}c|ccccccc}
    \text{$q$:}         & & \texttt{NAME} & \texttt{=} & \texttt{NAME} & \texttt{)} & \texttt{NAME} & & \\
    \text{PE:} & & 0 & 1 & 2 & 3 & 4 & & \\\hline
    \text{$d_1$:}  & & \texttt{NAME} & \texttt{=} & \texttt{NAME} & \hlorange{\texttt{(}} & \texttt{NAME} &  \hlgreen{\texttt{)}} & \\
    \text{PE:} & & 0 & 1 & 2 & 3 & 4 & 5 &\\
    \text{LA:} & & 0 & 0 & 0 & 2 & 0 & 1 &\\\hline
    \text{$d_2$:}  & & \texttt{NAME} & \hlorange{\texttt{(}} & \texttt{NAME} & \texttt{)} & \hlred{\texttt{NAME}} & & \\
    \text{PE:} & & 0 & 1 & 2 & 3 & 4 & &  \\
    \text{LA:} & & 0 & 2 & 0 & 0 & 3 & & \\\hline
    \text{$d_3$:}  & & \texttt{NAME} & \texttt{=} & \texttt{NAME} & \hlorange{\texttt{.}} & \texttt{NAME} & \hlgreen{\texttt{(}} & \hlgreen{\texttt{)}} \\
    \text{PE:} & & 0 & 1 & 2 & 3 & 4 & 5 & 6 \\
    \text{LA:} & & 0 & 0 & 0 & 2 & 0 & 1 & 1 \\
  \end{array}
\end{equation*}
}
\caption{Transformer data encoding.}\label{fig:tx_data_encoding}
\end{wrapfigure}

More concretely, we depict a single instance of the training data in Fig.~\ref{fig:tx_data_encoding}. The reranking model sees a (1)~query, (2)~document, (3)~positional encoding and (4)~Levenshtein alignment, and returns a numerical score. Once the relevance scores are obtained, we calculate the cross-entropy loss across the top-$k$ scoring documents and backpropagate. In practice, this update is averaged across a batch $\langle q_i, [d_j]_{0\ldots k}\rangle_{i=0\ldots |B|^{\approx 16}}$ of repair instances to reduce noise. The batch update rule is a standard variant of stochastic gradient descent $\big(\theta' \leftarrow \theta - \alpha \nabla_{\theta} \mathcal{L}(\theta)\big)$ with momentum (AdamW), where the learning rate is in the range $\alpha\approx10^{-4}$. A more exhaustive description of the architectural details and hyperparameter settings used for training the reranker can be found in Appendix~\ref{sec:hyperparams}.

\clearpage\section{Evaluation}

We call our method Tidyparse and consider the following research questions:

\begin{itemize}
\item \textbf{RQ 1}: What statistical properties do human repairs exhibit? (e.g., length, edit distance)
\item \textbf{RQ 2}: How performant is Tidyparse at fixing syntax errors? (i.e., vs. Seq2Parse and BIFI)
\item \textbf{RQ 3}: Which design choices are most significant? (e.g., decoding, reranking, parallelism)
\end{itemize}

We address \textbf{RQ 1} in \S~\ref{sec:rq1} by analyzing the distribution of natural code snippet lengths and edit distances, \textbf{RQ 2} in \S~\ref{sec:rq2} by comparing Tidyparse against two existing syntax repair baselines, and \textbf{RQ 3} in \S~\ref{sec:rq3} by ablating various design choices and evaluating the impact on precision and latency.

\subsection{Experimental setup}

In the following set of experiments, we use syntax errors and fixes from the Python language. Python code snippets are abstracted as a sequence of lexical tokens using the official Python 3.8.11 parser, erasing alphanumeric identifiers and literals but retaining all other keywords. Accuracy is evaluated across a test set of pairwise errors and repairs by checking for lexical equivalence with the ground-truth repair, following the same methodology as Sakkas et al. (2022)~\cite{sakkas2022seq2parse}.

We use the Precision@k statistic, which measures the frequency of the true repair appearing in the top-k results, across a dataset of repair instances. Specifically, given a repair model, $R: \Sigma^* \rightarrow (\Sigma^*)^t$ and a test set, $\mathcal{D}_{\text{test}}$, containing pairwise aligned errors ($\err\sigma$) and fixes ($\sigma'$), we define Precision@k as:\vspace{-0.1cm}

\begin{equation}
\text{Precision@k}(R) = |\mathcal{D}_{\text{test}}|^{-1}\sum_{\mathcal{D}_{\text{test}}} \mathds{1}\left[\sigma' \in R(\err\sigma)_{0\ldots k}\right]
\end{equation}

Our full dataset~\cite{wong2019syntax} consists of $2\times 10^4$ naturally-occurring pairs of Python errors and human fixes from StackOverflow, which we use to evaluate the precision of each model at blind recovery of the ground truth repair. From the StackOverflow dataset, we filter for syntax errors shorter than 80 tokens and fewer than four lexical edits apart from the corresponding repair, then divide the remaining repairs into two disjoint sets: a training set $(\mathcal{D}_{\text{train}})$ of 4,586 repair instances and a test set $(\mathcal{D}_{\text{test}})$ of 2,238 repair instances, balanced across each length interval and edit distance, i.e., $\mathcal{D}_{\text{test}} = \{\langle\err\sigma, \sigma'\rangle \mid \lfloor |\err\sigma|/10 \rfloor \in [0, 8], \Delta(\err\sigma, \sigma') \in [1, 3]\}$, each test bin containing at least 50 instances.

To train the reranker, we augment each instance in the training and test set with a list of repair confounders. Each instance consists of a tuple, $\langle\err\sigma: \bar{\ell}, \sigma': \ell_\cap, \bm{\sigma}: \ell_\cap^{\leq 10^3}\rangle$, with a single syntax error $(\err\sigma)$, the ground truth repair, $(\sigma')$, and up to $10^3$ confounders $(\bm{\sigma})$ sampled without replacement from $\ell_\cap$ using our pretrained $4$-gram model. We then train the reranker on $\mathcal{D}_{\text{train}}$ for 13,000 steps which takes $\sim 4$ hours, and evaluate on $\mathcal{D}_{\text{test}}$. Hyperparameters are provided in Appendix~\ref{sec:hyperparams}.

For our final repair procedure, we use the CNF Python grammar, $G_{\text{Python}}$ and let $d_{\max}$ be the smallest value such that $\ell_\cap = \mathcal{L}(G)\cap\mathcal{L}\big(L(\err\sigma, d_{\max} - 1)\big)$ is nonempty. We decode $\ell_\cap$ with the same pretrained $4$-gram model used in reranker training, and pass the top $10^3$ results by 4-gram probability to the transformer encoder, then finally rerank the top $10^3$ by softmax probability and measure the Precision@k across repairs of differing length and edit distance in the test set.

We compare our method with two external baselines, Seq2Parse and Break-It-Fix-It (BIFI)~\cite{yasunaga2021break}, on the same test set. The Seq2Parse and BIFI experiments were conducted on a single Nvidia V100 GPU with 32 GB of RAM. For Seq2Parse, we use the default pretrained model provided in commit \texttt{7ae0681}.~\footnote{https://github.com/gsakkas/seq2parse/tree/7ae0681f1139cb873868727f035c1b7a369c3eb9} For BIFI, we use the Round 2 breaker and fixer from commit \texttt{ee2a68c},\footnote{https://github.com/michiyasunaga/BIFI/tree/ee2a68cff8dbe88d2a2b2b5feabc7311d5f8338b} the highest-performing model reported by the authors, with a variable-width beam search to control the number of predictions, and let the BIFI fixer model predict the top-$\{1, 2\times 10^4\}$ repairs. Finally, for Tidyparse, we use a standard Apple MacBook M4 Max with 128 GB of memory.

\clearpage\subsection{Dataset statistics}\label{sec:rq1}

In the following experiments, we use a dataset of Python snippets consisting of 20,500 pairwise-aligned human errors and fixes from StackOverflow~\cite{wong2019syntax}. We preprocess the dataset to lexicalize all code snippets, then filter by length and distance shorter than 80 lexical tokens and under four edits, i.e., with Levenshtein distance under four lexical edits $\big(|\Sigma| = 88, |\err{\sigma}| < 80, \Delta(\err{\sigma}, \sigma') < 4\big)$. We depict the length, edit distance, normalized edit locations and stability profile in Fig.~\ref{fig:patch_stats}.\vspace{-0.2cm}

\begin{figure}[h!]
\begin{tikzpicture}[scale=0.57]
  \begin{axis}[
ybar,
bar width=5pt,
xlabel={Snippet length, $|\err\sigma|$},
ylabel={Frequency},
title={Cumulative length distribution},
axis x line*=bottom,
axis y line*=left,
ymin=0,
ymax=65,
xtick=data,
xticklabels={,<20,,<40,,<60,,<80,,<100},
ymajorgrids=true,
grid style=dashed,
width=0.45\textwidth,
height=0.3\textwidth
]

\addplot[fill=black!30] table {
  X Y
  1 7.60
  2 14.52
  3 22.01
  4 30.54
  5 37.82
  6 44.30
  7 49.68
  8 55.21
  9 59.75
  10 63.59
};
\draw[red, dashed] (axis cs:8.5,0) -- (axis cs:8.5,65);
\end{axis}
\end{tikzpicture}
\begin{tikzpicture}[scale=0.57]
\begin{axis}[
ybar,
bar width=5pt,
title={Human repair distance},
xlabel={Edit distance, $\Delta(\err\sigma, \sigma')$},
ylabel={Frequency},
axis x line*=bottom,
axis y line*=left,
xtick=data,
ymajorgrids=true,
grid style=dashed,
xticklabels={,\leq 2,,\leq 4,,\leq 6,,\leq 8,,\leq 10},
ytick={0, 20, 40, 60, 80, 100},
ymin=0,
width=0.45\textwidth,
height=0.3\textwidth
]
\addplot[fill=black!30] table {
X Y
1  31.48
2  47.52
3  54.89
4  60.44
5  63.88
6  66.38
7  68.02
8  70.04
9  71.49
10 72.22
};
\draw[red, dashed] (axis cs:4.5,0) -- (axis cs:4.5,80);
\end{axis}
\end{tikzpicture}
\begin{tikzpicture}[scale=0.57]
\begin{axis}[
ybar,
bar width=5pt,
xlabel={Beginning $\longleftrightarrow$ End},
ylabel={Frequency},
title={Normalized edit locations},
axis x line*=bottom,
axis y line*=left,
ymin=0,
ymax=35,
xtick=data,
xticklabels={,20\%,,40\%,,60\%,,80\%,,100\%},
ymajorgrids=true,
grid style=dashed,
width=0.45\textwidth,
height=0.3\textwidth
]

\addplot[fill=black!30] table {
X Y
10 11.6539
20 5.7252
30 6.2087
40 5.9542
50 5.5980
60 7.9389
70 7.0738
80 6.9466
90 12.4173
100 30.4835
};
\end{axis}
\end{tikzpicture}
%    \begin{tikzpicture}
%      \begin{axis}[
%        ybar,
%        bar width=5pt,
%        title={Intra-patch edit distance},
%        xlabel={Caret distance},
%        ylabel={Frequency},
%        axis x line*=bottom,
%        axis y line*=left,
%        xtick=data,
%        ymajorgrids=true,
%        grid style=dashed,
%        xticklabels={1,2,3,4,5,6,7,8,9,10+},
%        width=0.45\textwidth,
%        height=0.3\textwidth
%      ]
%
%        \addplot table {
%          X Y
%          1 40.66
%          2 15.00
%          3 5.80
%          4 4.86
%          5 4.26
%          6 2.98
%          7 2.05
%          8 2.73
%          9 1.62
%          10 13.64
%        };
%      \end{axis}
%    \end{tikzpicture}
\begin{tikzpicture}[scale=0.57]
\begin{axis}[
legend cell align={left},
legend style={fill opacity=1, draw opacity=1, text opacity=1, draw=lightgray204, legend columns=-1, legend pos=south east},
xlabel={Snippet length, $|\err\sigma|$},
ylabel={Stable region},
title={Stability profile},
ybar,
axis lines*=left,
xtick={0, 10, 20, 30, 40, 50, 60, 70},
ytick={0, 0.2, 0.4, 0.6, 0.8, 1.0},
xticklabels={, {[}10{,}20{)}, , {[}30{,}40{)}, , {[}50{,}60{)}, , {[}70{,}80{)}},
yticklabels={0, 0.2, 0.4, 0.6, 0.8, 1.0},
x tick label style={font=\scriptsize},
ymax=1.0,
ymin=0.0,
bar width=3pt,
grid style=dashed,
ymajorgrids=true,
width=0.45\textwidth,
height=0.3\textwidth
]
\addlegendimage{empty legend}
\addlegendentry{$\Delta(\err\sigma, \sigma')=$}
\addlegendimage{ybar,ybar legend,draw=none,green,fill=green!50}
\addlegendentry{1,}
\addlegendimage{ybar,ybar legend,draw=none,blue,fill=blue!50}
\addlegendentry{2,}
\addlegendimage{ybar,ybar legend,draw=none,orange,fill=orange!50}
\addlegendentry{3}
\addplot[green, fill=green!50] coordinates {(0, 0.80) (10, 0.91) (20, 0.96) (30, 0.97) (40, 0.99) (50, 0.99) (60, 0.99) (70, 0.99)};
\addplot[blue, fill=blue!50] coordinates {(0, 0.35) (10, 0.59) (20, 0.69) (30, 0.73) (40, 0.79) (50, 0.82) (60, 0.84) (70, 0.86)};
\addplot[orange, fill=orange!50] coordinates {(0, 0.23) (10, 0.45) (20, 0.58) (30, 0.66) (40, 0.70) (50, 0.77) (60, 0.78) (70, 0.86)};
\end{axis}
\end{tikzpicture}
\vspace{-0.2cm}
\caption{Repair statistics across the StackOverflow dataset, of which Tidyparse can handle about half in under $\sim$3s and $\sim$4 GB. Larger repairs and edit distances are possible, albeit requiring additional time and memory.}\label{fig:patch_stats}\vspace{-0.2cm}
\end{figure}

We observe that slightly over 6,700 code snippet pairs in the StackOverflow dataset contain fewer than 80 tokens and four lexical edits, which are computational feasible to process in a few hundred milliseconds. We also note a slight primacy or recency bias in the edit locations, evidenced by a large fraction of human repairs which modify the boundaries of the broken code snippet.

For the stability profile, we enumerate repairs for each syntax error and estimate the average fraction of all edit locations that were never altered by any repair in the $L\big(\err\sigma, \Delta(\err\sigma, \sigma')\big)$-ball. For example, on average roughly half of the string is stable for 3-edit syntax repairs in the $[10-20)$ token range, whereas 1-edit repairs of the same length could modify only $\sim 10\%$ of all locations. For a fixed edit distance, we observe an overall decrease in the number of degrees of caret freedom with increasing length, which intuitively makes sense, as the repairs are more heavily constrained by the surrounding context and their locations grow more concentrated relative to the entire string.

\begin{wrapfigure}{r}{0.45\textwidth}
\vspace{-0.4cm}
\resizebox{.45\textwidth}{!}{% This file was created with matplot2tikz v0.3.3.
\tikzset{mark size=0.5}
\begin{tikzpicture}

\definecolor{darkgray176}{RGB}{176,176,176}
\definecolor{darkorange25512714}{RGB}{255,127,14}
\definecolor{forestgreen4416044}{RGB}{44,160,44}
\definecolor{lightgray204}{RGB}{204,204,204}
\definecolor{steelblue31119180}{RGB}{31,119,180}

\begin{axis}[
legend cell align={left},
legend style={fill opacity=0.8, draw opacity=1, text opacity=1, draw=lightgray204, legend columns=-1, legend pos=south east},
legend image post style={scale=2},
tick align=outside,
tick pos=left,
title={\textbf{Language intersection volume}},
x grid style={darkgray176},
xlabel={Snippet length, $|\err\sigma|$},
xmin=-0.75, xmax=103.75,
xtick style={color=black},
y grid style={darkgray176},
log basis y={2},
ymode=log,
axis lines*=left,
ylabel={Volume, \(\displaystyle |\ell_\cap|\)},
ymin=0.8, ymax=100000000,
ytick style={color=black}
]
\addlegendimage{empty legend}
\addlegendentry{$\Delta(\err\sigma, \sigma')= \text{LED}(\err\sigma) + \{$}
\addlegendentry{$0,$}
\addlegendentry{$1,$}
\addlegendentry{$2\}$}
\addplot [draw=forestgreen4416044, fill=forestgreen4416044, mark=*, only marks]
table{%
x  y
15 7
23 57
16 67
16 121
29 5
28 7
39 102
47 60
37 50
33 8
30 2
35 37
21 26
35 5
18 62
23 35
42 78
54 4
57 8
24 63
62 15
55 57
66 102
36 4
16 4
30 55
11 3
12 6
33 67
9 66
57 16
43 82
25 2
70 39
43 150
31 8
42 51
27 5
11 2
56 8
19 29
13 41
23 35
49 100
31 28
26 34
19 144
93 7
52 26
18 64
54 56
24 46
24 2
22 4
66 1795
98 32
34 9
21 48
16 48
45 78
79 5
15 81
14 14
34 1245
16 4
21 3
30 35
20 64
91 49
93 8
29 37
54 4
71 392
39 17
22 64
46 23
20 48
18 40
23 1716
42 3
27 14
96 29
93 94
26 2
17 58
21 86
28 15
29 70
9 549
89 58
52 30
47 14
51 102
26 95
34 2
16 23
22 42
32 5
33 8
60 254
32 37
21 1001
96 20
33 110
26 28
16 55
25 11
34 9290
16 2
91 36
20 65
36 8
13 56
20 2
19 18
78 23
53 207
67 152
59 176
15 20
59 4
78 28
31 6
16 25
47 3
24 48
71 22
48 57
54 2
32 3
26 43
25 58
18 46
72 72
16 7
16 15
9 226858
8 38
44 6
83 26
22 4
16 165
18 67
33 2
35 17
20 92
13 66
55 51
86 64
48 4
39 6
78 49
45 30
17 38
49 94
55 86
82 4
9 5
54 7
18 37
50 6
40 143380
34 2
11 8
25 16
63 102
25 2
15 80
60 13
49 16
81 56
84 49
73 4
20 13
92 4
39 28
74 8
39 4160
51 4
6 4
27 6
31 86
32 61
37 14
83 31
29 26
5 3
26 4
85 5
73 78
31 7
26 55
16 95
93 7
14 1285
29 28
20 13
14 111
8 59
42 22
43 2077
30 8
41 7
33 47
78 28
23 47
80 26
73 4
17 15
76 4
33 36
37 26
53 70
27 240
36 3
14 8
38 4
17 16
31 2
43 61
41 6
41 231
21 63
60 58
74 289
17 59
28 7
44 11
24 2660
34 4
27 117
50 1156
35 121
76 20
36 3
33 2
66 3
29 11
56 36
31 1067
52 15
33 20
63 841
36 27
35 88
5 35
49 80
13 15
63 2914
39 4
9 3
9 63
36 3
75 222
21 69
12 16
27 8
55 26
49 56
57 80
73 15
33 52
17 66
31 11
26 63
65 7
42 37
20 2
85 3
10 4095
56 6
8 47
41 28
92 38
29 7
8 32
9 90
16 112
40 39
82 14
67 4
78 20
27 3
40 1382
25 14
35 35
29 75
42 28
21 4351
59 49
25 146
28 15
7 11
18 2
64 38
18 3
28 33
62 64
29 33
27 14
19 64
83 237
45 28
36 78
94 7
56 154
39 2
41 13
23 82
24 26
76 29
41 90
9 2
27 20
83 4
32 10
58 33
34 4
9 46
22 18
65 88
27 53
36 58
16 14
97 90
89 19
23 98
56 335
35 3
62 29
42 13
74 47
47 28
9 64
25 64
37 101
34 112
72 71
29 26
76 120
35 85
42 1020
24 3
34 83
51 156
31 12
36 3
76 897
34 94
32 58
58 34
22 36
85 8
11 2
40 8
63 63
36 30
88 274
22 20
24 46
15 32
84 35
62 34
23 5
31 27
53 38
85 3
92 320
20 83
34 109
15 3
79 16
50 64
30 115
22 26
84 64
15 83
33 40
13 83
11 40
39 2
35 38
28 28
77 23
13 12
12 26
20 96
51 5
6 2
19 143
75 37
31 49
35 91
20 57
20 28
37 130
29 14
38 82
54 1223
36 58
25 11
24 76
95 2
97 70
60 62
72 26
58 4
13 78
28 6
33 20
42 53
20 37
44 41
35 29
12 12
57 28
78 56
50 7
35 40
29 9
16 10
79 29
12 102
17 41
68 17
29 36
72 80
30 2
23 311
29 19
63 8
36 81
13 68
22 85
11 17
36 79
86 121
86 9
25 94
25 31
77 58
97 13
39 28
25 34
36 2
16 10
43 49
37 13
50 76
95 80
98 49
27 3
4 46
32 11
18 31
10 58
29 15
35 4
33 222
17 58
11 32
24 30
44 628
22 19
57 8
30 102
69 32
63 7
26 47
24 12
77 1630
40 4
26 46
25 58
31 5
43 397
21 99
26 4
15 2
53 28
88 19
72 21
15 89
5 41
69 28
71 3
51 1156
66 69
57 35
33 38
48 555
12 61
40 82
31 9
93 65926
17 7
42 7
72 8
76 6395
43 38
25 58
34 9
54 52
45 7
19 86
35 3
30 19
48 36
68 49
15 262015
31 11547
82 9
28 101
37 15
34 18
32 35
39 8
36 35
33 5310
16 63
18 46
42 6
85 59
32 55
22 19
15 2
76 26
23 16
33 12
14 11
17 7
21 23
19 14
47 34
13 48
24 7
28 122
52 2
81 35
67 7
38 2
33 28
64 58
31 102
36 33
24 397
88 4
41 4
45 15
61 4
13 46
19 16
20 20
62 473
71 8
23 115
78 102
41 18
25 52
23 11
44 9
4 8
25 3
54 34
25 2
28 8
35 70
38 14
56 26
53 79
49 2
74 7
28 17
53 40
54 102
20 40
12 37
28 6648
28 64
88 7
27 27
95 50
33 81
30 4
27 41
45 2
46 34
50 35
82 35
52 289
48 8
25 113
79 47
19 66
89 5
45 37
40 3
16 28
24 43
64 78
27 35
17 88
15 79
24 27
32 80
16 37
10 41
39 63
17 72
55 102
26 102
35 86
96 43
40 8
30 33
25 11
26 102
21 91
26 7
14 48
47 58
14 58
34 165
14 23
23 10
24 47
15 19
96 8
43 65
29 29
38 68
20 13
18 10
34 3
48 96
84 4
28 139
43 38
15 5
60 46
19 62
11 14
66 58
22 19
18 37
74 5
61 41
50 11
76 7
16 6
39 1681
30 14
15 29
22 80
13 27
20 13
61 4
57 35
19 15
65 2
55 13
20 2
20 68
37 18
11 39
21 16
44 109
53 38
32 64
30 37
70 35
48 106
20 88
40 288
45 102
36 55
22 2
29 88
78 35
51 3069
69 6
33 4
52 4
71 3
52 40
27 2
53 11
58 38
11 13
11 7
28 2
35 41
16 68
37 29
30 52
38 15
38 6
36 12
27 47
21 3
25 28
19 38
53 7
21 3
12 35
35 40
12 2126
87 3
69 82
43 26
13 4
24 7
18 64
34 1836
76 117936
51 67
73 28
55 15
20 92
27 5
39 38
27 389
18 8
95 961
41 3845
12 36
31 2
39 36
11 8
23 27
14 894
76 3
12 39
31 5
19 102
17 45
18 15
16 2
52 82
43 35
22 166
32 63
50 2401
38 5
43 28
51 5
96 102
86 77
72 276
51 20
26 86
40 4
38 48
50 64
35 28
17 78
13 7
42 48
15 758
19 5
21 82
33 38
38 11
25 6
40 16
25 66
14 34
37 167
29 36
11 13
12 3
33 57
17 66
22 14
35 339
34 59
57 8
57 53
31 99
99 13
82 71952
18 8
17 58
9 16
53 10
18 15
26 64
49 7
47 559
17 8
41 4
23 2
35 11
10 56
19 1301
21 78
38 63
33 212
33 32
41 67
42 84
19 62
42 4
9 46
21 56
20 11
16 49
38 31
77 15
73 7
41 8
26 56
49 111
15 59
36 11
61 4
16 37
67 46
71 49
31 14
18 6
19 56
42 79
33 148
5 22
66 22
53 28
20 7
8 58
19 28
82 51
81 83
19 39
74 7
40 16
33 733
24 17
27 8
39 37
35 146
21 11
73 41
43 15
18 7
70 35
34 144
13 34
11 25
71 32
8 10
15 8
71 4
58 34
30 26
14 26
82 210
22 66
44 8
10 16
46 4
46 15
63 54
10 754
27 384
24 5
47 43
56 2
29 57
30 2078
88 185
16 34
29 3
21 841
35 211
13 81
48 10
37 2357
65 7
22 3
20 63
21 11
43 3
73 78
24 63
14 40
17 4896
24 70
99 28
25 81
65 4
63 35
22 47
59 21
36 79
34 18
65 2
34 3
28 13
61 2
15 32
11 20
15 398
12 108
46 42
28 3
48 34
53 80
27 27
24 2
34 2
63 15
42 41
30 58
13 80
27 32
21 1764
17 2
34 27
45 135
24 4
27 2
61 11
5 3
96 28
34 88
60 169
22 10404
80 3
29 22
25 398
63 27
70 102
67 17
10 37
69 10404
33 183
84 23
15 58
31 11
35 12
19 86
23 33
52 47
51 38
20 5
9 28
59 2442
39 7
78 58
75 3
55 5
57 26
42 14
22 5
60 2
31 2
41 2
29 6
50 41
24 26
9 12
89 6
15 17
43 30
29 82
11 72
46 10
14 4
26 7
28 37
30 3
22 3
20 16
71 5
20 86
16 10
19 31
30 9
29 47
24 17
60 2112
42 3782
34 15
23 890
20 7
18 4
34 80
29 37
60 88
29 44
16 30
21 188
22 34
13 57
37 86
48 4
34 101
90 26
17 6
32 57
47 157
36 26
92 16
79 80
13 18
22 35
50 30
79 5
30 82
52 13
10 15
35 79
17 31
27 7
38 43
15 3
40 29
45 15
54 8
35 4
45 2
18 30
27 37
83 28
43 5
15 65
50 38
74 81
13 6
37 29
15 5
70 58
24 18
38 58
69 14
55 13
29 3
8 102
18 48
21 784
29 150
46 9
8 15
8 15
13 1
10 35
20 115
35 8
11 64
18 9
28 58
45 36
26 43
55 37
50 1
65 3
23 2
15 37
62 68
27 5
56 17
71 1
57 55
37 869
94 65
35 5
75 27
62 26
39 4094
50 1
54 2
62 29
50 36
36 41
41 5
85 1
9 23
73 2
12 20
68 16
60 69
10 19830
30 2415
53 7
43 59
11 378
38 16
52 13
42 343
32 169
72 44
21 1
23 67
51 2
60 29
48 8
60 13
14 10
41 35
57 49
55 4
73 13
68 4
40 30
66 2
22 20
63 3
12 28
55 1
68 15
51 58
32 3
94 1
29 45
20 26
42 102
40 34
50 50
45 107
38 35
61 8
41 27
18 22
50 58
30 90
48 49
55 1
9 44
54 36
17 18
25 9
74 7
33 79
31 90
26 1
11 2
61 44
59 4
47 6
29 63
52 8
14 27
44 8
29 82
63 41
14 38
63 76
78 8
98 4
60 225
26 37
44 28
75 148
14 40
90 4
53 22
86 1
14 5
56 22
32 260
25 1
51 16
43 1
58 68
58 33
96 3136
27 14
18 99
21 7
61 26
30 111
15 83
11 5
42 4
8 4
29 38
78 9
24 28
8 100
24 18
40 4
27 8327
88 9
43 1
39 41
9 42
52 79
51 101
26 8
23 1
16 28
34 48
54 7
15 4
13 240
38 26
56 1113
49 1
50 1
69 1
46 5
59 13
21 1
64 1
64 77
31 14
19 3
47 1
49 1
37 75
41 31
54 64
37 442
34 1
39 22
78 4
67 16
72 112
13 853
38 1681
23 1
30 37
65 53
24 76
49 373
51 41
48 39
19 5
60 1
53 137
28 7
55 908
66 28
91 1
52 91
10 8
12 28
46 6
5 15
49 5
60 28
38 1
68 17
74 2
49 14
65 28
33 4
59 1
34 3
34 2
22 1
87 1
65 35
48 121
66 22
70 24
86 81
51 8
56 4
25 8
45 21
50 4
21 20
26 1
38 360
68 28
75 1
29 13
69 1
39 1
28 39
45 38
27 28
28 961
53 2
36 32
14 2
76 49
9 16
9 46
41 2
10 42
59 54
28 165
62 1
8 11
25 1
29 1
16 72
46 5040
69 7
30 5
93 38
67 6
33 31
92 29
70 2
26 222
89 22
16 31
22 256
49 32
7 13
44 6
55 4
37 46
10 11
38 110
17 56
26 322
28 51
15 48351
15 36
81 102
37 1
72 28
50 26
55 112
63 1281
30 378
44 676
35 8
43 1
77 1
35 3
13 58
25 18
47 76
87 8
54 28
18 37
45 1
30 69
52 1
28 1
26 20
84 27
34 18
55 3
39 9
7 5
25 3
20 1
52 5
15 59
37 1
63 18
84 33
16 78
96 27
16 46
59 165
87 1
14 3
79 104
98 4
28 2
60 14
59 1
26 34
83 64
19 14
58 14
32 19
76 82
86 66
14 64
52 12
23 28
93 87
9 16
77 6
49 4
73 78
14 113
99 28
55 17
23 28
53 102
50 69
54 1
25 49
10 5
18 64
33 34
41 15
71 1
35 19
50 4
59 1
32 30
52 16
79 121
41 5
54 381
12 29
69 1
68 1
44 33
32 29
25 3
71 11
50 1
62 1
94 46
66 9
93 43
36 37
75 1
77 37
81 35
47 63
26 51
12 1
31 32
16 15
40 112
62 6
71 2
50 54
15 44
38 56
41 1
42 1
27 58
40 1
72 34
17 22
29 137
65 14
4 8
19 1630
36 13
43 3
85 49
85 234
60 1
40 33
74 3360
25 7
29 2
57 1
25 2079
14 2
53 300
69 18
22 860
50 59976
60 40
9 48
28 8
57 5
45 240
44 170
13 47
45 4
41 1
46 53301
78 8
44 43
6 5
56 78
55 1
82 54
62 5
9 61
34 8
51 35
42 11
82 8
33 8
73 260
16 65
83 57
80 9
7 8
47 28
93 11
87 3
44 81
24 1
41 13
29 61
53 56
22 68921
8 5
62 12
13 165
12 46
37 11
48 30
41 8
8 2
44 4
63 343
42 1
51 2
52 27
77 43
30 49
29 95
22 175
22 1
41 16
22 14
36 7
81 7
56 8
50 197
57 1
29 8
14 41
43 12
94 24389
50 52
54 29
70 1
27 20
49 2
6 3
61 78
53 1
23 9
88 1
21 36
54 1
11 3
52 58
13 36
34 88
55 40
7 59
42 905
48 6
9 14
65 32
63 35
68 436
53 1
72 1
11 1
78 50
35 1
40 1
27 65
15 102
16 29
47 5
13 58
35 27
16 32
61 1
64 2
85 74
18 4
51 2
78 16
45 38
29 5
39 8
58 38
16 54
87 165
61 134
37 3
11 79
73 35
49 67
52 2
48 24
7 40
52 198
57 67
61 1
21 86
39 1
62 43
38 56
21 5
61 1
13 37
69 20
35 1
41 35
36 4
34 4
22 64
32 33
9 11
78 2650
54 9
56 841
64 4492
16 79
8 2
68 729
29 2
56 4
13 86
57 69
15 80
60 784
69 1
5 14
10 1
15 19
62 11
47 36
11 41
77 10
10 14
16 21
13 5
16 22
46 1
47 102
24 1
19 55
42 2866
8 17
41 5
15 318
15 35
30 77
35 102
48 5
33 1
65 1
50 1
16 59
13 11
77 1
73 17
56 783
97 1
69 58104
33 56
65 27
81 38
31 392
67 36
41 1
65 33
70 5
55 13
59 1
11 36
18 44
15 68921
40 8
18 64
61 85
63 1
51 35
81 200
8 11
23 4
35 1
35 1
41 66
86 46656
18 7
75 17
47 27
37 1
7 8
89 22
58 992
73 128
36 1
19 1381
78 1
74 31
23 49
81 2
9 46
40 1615
87 7
21 24
37 27
13 18
57 31
44 676
85 51
43 7
32 1
79 7
82 2
35 900
78 1
34 27
5 14
61 59
95 1156
30 15
27 1
59 222
48 7
64 2
76 30
29 5
7 8
23 89
48 41
13 66
34 11
57 1
15 698
89 49
68 4
43 1
82 56
49 27
66 8
20 13
75 6
58 130
57 61
55 4
6 32
59 5
58 7
25 1639
72 102
49 1
8 11
44 1
15 2
34 90
33 1
31 1
64 2
76 4
67 4
62 4
32 1180
29 7
68 143
64 64
21 48
30 64
64 14
32 5
43 1
61 1
47 1
44 14
55 1
10 421
32 783
41 12
30 1120
50 10
44 5
47 29
46 76
29 6
25 1
37 1
60 121
85 61
21 1047
64 46
65 14
5 5
63 1
52 78
86 2809
52 28
9 46
13 5
6 11
52 54
33 6
5 3
6 2
28 192
17 1
37 26
46 357
49 21
53 1
38 1
63 31
59 6
7 5
29 4
17 6
66 65
55 3
87 1
72 1
53 1
33 68
59 29
20 76
68 5
12 2
17 5
13 32
14 5
46 2
52 19
56 576
40 5
63 1
77 13
18 51
32 1
79 118
28 3612
96 208
59 1
17 16
70 4
43 67
13 7
17 41
63 35
36 1
57 16
41 2
54 544
5 1
41 31
15 102
8 11
18 16
50 80
71 64
31 7
69 1
43 1
22 1
72 2114
55 5
50 41
31 71
12 50
28 38
42 50
56 5
17 165
31 2
45 136
27 784
59 6
45 27
28 8
35 26
44 24476
6 3903
58 69
20 58
35 1
9 27
24 30
77 580
14 13
62 4606
64 2
58 1764
23 370
21 42
51 13
56 2
56 603432
12 3
35 1
36 1
19 13
74 819
52 2
52 1
38 4
50 5
26 4
37 1
9 46
39 1376
20 1
25 8327
80 11
63 62
21 8
33 15
31 76
55 63
28 1
42 63
54 13
25 29
13 6
14 1023
9 46
94 26
17 90
91 1
27 23
43 3
58 12
67 1061208
20 39
15 18
19 49
15 261
6 2
77 2
91 64
15 5
37 1
21 31
58 3
42 4
78 17
25 8
34 1
49 9
67 30
18 3
75 3711
85 79
14 39
24 5
17 165
76 19
84 82
12 801
83 102
65 1
61 28
93 47
44 13
31 4
15 3
71 58
62 2
20 13
30 46
23 64
45 1
73 36
11 80
50 1
18 68
60 1
79 3
30 1
11 35
62 10
20 93
21 64
57 5
59 17
97 45
19 57
47 1
91 24
70 265
16 81
73 1
8 35
5 41
40 16
86 71
33 70
30 5
20 58
31 5
47 102
47 1
44 8
51 29
28 4
41 26
67 1
92 230
47 26
37 9
12 90
5 20
97 6
31 46
73 18
24 32
23 1
59 2
75 9
5 11
71 1
16 1
31 9
7 6
50 47
10 34
24 4
94 80
52 37
30 222
44 26
31 20
61 79
65 1
21 34
24 11
51 37
40 1301
60 201
15 65
56 1
77 6
34 90
31 2
11 1
83 1
37 6
11 9
53 1
49 49
22 17
30 1
50 6
49 25
12 10404
42 68
32 841
20 63
70 55
11 34
54 47
45 8
36 203
25 41
20 35
80 1
33 1
57 1369
41 14
64 12
11 15
53 2
29 12
60 4
62 8
39 48
68 39
28 1
87 5
21 28
83 39
99 52
30 33
33 156
12 4
12 6
15 44
32 26
52 1
61 5
29 1
54 28
9 9
14 36
24 2
40 70
65 8
5 3
59 43
19 38
43 61
40 1
38 1
42 56
32 64
66 8
59 34
58 31
22 1
58 10
73 26
54 32
19 31
68 64
59 26
18 34
46 553
54 1
25 3
19 5
72 1
70 1532
5 41
54 44
72 24
42 1
39 70
87 62
61 11
43 112
24 2
24 48
9 12
91 1
6 4
91 1
65 1
79 102
38 28
77 1
24 1
20 21952
38 7
20 53
48 30
87 84
18 34
46 240
20 1
80 529
50 2
78 16
9 23
18 80
33 10
77 102
42 48
16 64
31 33
42 1
20 812
44 1
37 39
59 18
50 1
94 4704
61 64
82 54
69 3002
80 67
37 6
43 2
8 5
63 2
49 14
48 2
46 97
60 29
57 152
57 1
16 27
37 1
88 4
61 57
55 1
61 1
34 6
35 56
21 47
37 6
43 1
85 14
40 23
40 30
26 29
30 357
57 693
30 783
44 58
59 17
23 1
33 14
56 98
52 5
79 13
38 7
5 5
88 7
28 28
18 69
76 48
36 102
12 35
43 27
35 27
67 49
75 43
60 3
30 15
33 1639
18 64
50 49
10 1
79 1
23 45
66 59
27 15
40 4
52 783
45 1
79 1
60 23
73 64
18 1657
41 105
62 2
20 28
74 102
55 16
70 7
77 1
21 27
49 240
58 8
32 57
21 1510
67 26
40 3
45 28
11 2
16 583
37 1
41 64
54 46
72 1
63 1
50 1
40 7
95 11
12 1
5 59
59 1
27 10
85 1
32 49
55 1
23 97
93 1
35 102
91 792
38 72
76 26
31 29
16 22
32 1
75 8
65 1
50 1
10 34
84 1
19 81
46 4
27 6
47 1
9 3
67 1
32 1
28 1
29 210
76 6
82 22
44 16
45 1
32 1
6 3
26 1
8 14
95 1
24 844
21 84
41 1
6 4
30 189
65 1244
44 16
34 224
59 1
99 225
41 1
64 1
21 102
47 1
9 1
7 2
63 1
45 121
21 1638
59 17
39 78
66 200
12 58
31 36
71 7
86 35
46 8
26 360
81 102
35 1
38 16
28 37
73 156
73 35
76 1
38 117
55 76
20 5
30 1
47 90
95 16
32 38
59 61
37 9
54 7
9 79
23 1
26 38
60 28
34 2
18 335
68 26
15 2
20 47
36 66
56 2
73 26
44 1
28 1
46 53
55 39
41 5
7 1
22 96
9 14
82 3
29 1
24 32
58 11
5 1
20 35
31 1
13 215
56 22
75 6
49 4
64 726
36 7
59 84
34 29
38 1
63 1
46 36
14 11
30 5
83 18
37 2
59 1
62 6482
14 6
44 115
48 1
9 5
41 38
92 1
46 32
11 6
21 30
41 9010
61 28
39 2
48 3
79 16
85 29
78 47
48 80
68 102
18 2
10 38
87 403
88 1
87 15
85 17
60 512
19 78
38 58
35 1
32 1
69 2
94 185
55 57
33 1
63 20
74 1
10 19
15 1
29 38
44 15
38 237
70 1
19 129
82 848
20 8
71 58
69 24602
6 102
46 4
50 10
76 165
12 38
13 6
88 64
87 4
81 121
76 1
20 36
76 3
59 129
49 10026
21 64
43 1
56 1
35 1
32 123
77 1
65 1
64 67
34 22
54 31
15 3600
67 25
29 1
64 22
72 102
43 2
25 38
12 36
52 1
58 90
14 3
79 37
29 41
22 1
68 108
33 1
30 17
21 4
34 125
57 343
59 10
31 2
32 1
58 16
7 273
7 165
5 5
64 6
64 1
25 896
11 1
95 44
94 58
79 5
82 4
29 5
70 1
23 41
90 7
66 1
17 7
65 18
69 3
19 42
6 46
43 30
71 8
9 58
44 70
45 1
77 4096
24 29
30 16
19 23
41 220
23 7
44 26
77 26
84 13
42 3
69 1
48 28
23 1
8 76
74 19
41 64
16 16
81 1
65 99
30 16
21 6
33 11
68 1
38 102
33 1
80 1
34 76
30 11
38 1
26 222
19 26
33 13
56 8
47 1
57 64
8 14
42 8
15 67
26 16
60 1
58 59
18 35
61 28
13 31
33 7
31 34
18 49688
27 13
92 1
4 7
32 41
35 94
10 2352
60 82
54 783
99 5
32 108
29 14
66 8
76 1
31 30
49 1
56 1
74 4
35 4
19 2
58 1
50 1
53 97
65 176
31 1
29 27
27 1
21 8
15 62
36 27
51 63
33 28
81 18
7 29
27 28
24 5
9 64
12 64
46 2
58 46
42 48
67 1
19 1
81 105
9 6
16 8
62 26
33 20
49 67
15 3
5 41
21 1
24 1
23 50
42 1294
25 120
36 12
34 1
19 1
48 1
25 46
15 15
41 2
7 1
13 102
71 1
95 78
23 68921
88 1
9 5
70 102
55 34
58 21
48 78
82 68
41 5
23 8
33 34
4 7
21 12
91 131
71 39
70 121
4 15
84 13
58 1
40 66
89 91
27 34
28 47
22 66
16 7
85 5
20 212
54 41
66 40
81 14
28 69
42 74
82 4
27 195
46 28
8 5
34 783
29 26
29 49
30 30
47 21
30 57
36 2
24 18
70 37
35 860
27 1
58 80
8 15
38 33
59 29
32 13
58 144
63 48
44 69
78 68
26 1764
66 58
19 68
16 58
32 3136
23 6690
43 1
22 1
52 11
30 1188
27 28
13 7
41 170
48 34
93 12
75 186
12 58
87 1
19 7
14 1
49 39
9 5
52 88
44 49
19 8
40 18
41 2665
17 2
85 7
11 208
45 29
93 584
66 21
12 12
27 20
43 27
17 64
56 6
6 4
44 102
22 12
74 16
19 43
67 46
44 1
11 5
47 2
24 165
77 1
32 1
50 1
88 49
33 22
97 1
9 795
19 3
82 17
30 5040
44 41
8 59
60 39
27 109
77 36
68 35
58 1
52 102
34 4
74 26
21 35
18 18
19 1
30 34
46 27
67 532
43 35
78 28
27 99
14 5
98 16
52 110
39 1
14 61
78 11
94 35
48 31
13 14
84 1
40 1
47 64
67 3
40 49
88 59
61 3481
14 5
30 71
13 3
84 22
4 50
56 33
49 35
27 78
23 74472
36 3
71 1
31 1
24 5
48 26
76 55623
31 1
35 73
23 68
73 9
93 8
70 729
47 6
5 56
69 1
52 1
62 14
61 8
6 4
42 1
48 1
52 28
91 19
25 124
39 64
18 1
30 100
21 86
26 1
60 101
93 15
28 99
9 5
46 2
23 102
57 27
40 2
52 30653
41 16
20 26
91 58
42 41
12 257
52 3
6 15
7 39
48 3
44 1
75 2
86 4
85 4
92 176
51 27
78 2
30 1
28 32
79 1
35 4761
96 11
43 3
51 19
62 1
94 30
46 63
32 3
55 35
65 8249
20 20
52 40
29 8
15 34
19 271
18 238
75 13
38 104
95 2
79 26
43 784
21 27
17 6
9 395
93 33
68 1
97 64
60 16
70 91
50 102
57 1
54 784
97 14
20 1
55 31
22 29
47 3
61 90
77 15
42 1
43 95
52 30
82 8
42 73
60 1
61 61
80 3
16 2
5 12
48 35
40 1341
75 15
33 64
61 1
36 1630
32 8100
77 4
54 35
60 49
36 110
54 112
27 1
92 36
63 1
67 47
25 35
40 15
73 1
16 90
45 5
58 81
5 41
60 27
86 22
34 27
31 583
45 1
34 7
10 186
75 14
78 2
28 67
47 94
6 32
71 26
51 1090
51 61
45 208
54 7
7 2
25 165
70 151
30 1
88 2
28 77
18 69
84 4
95 22
25 1239
25 36
15 27
47 58
43 5
47 22
45 34
62 2
19 82
18 24
65 42
58 1
29 15
58 32
67 4
87 1
27 121
54 39304
30 78
25 1
12 32
46 1
28 7
71 41
8 36
67 1
20 68
45 1
25 30
24 6
39 38
42 3721
62 26
15 246
47 27
43 15
80 68
96 9592
56 153
82 22
27 480
56 26
15 74088
85 46
20 13
52 67
26 30
52 1
72 1
50 37
65 66
40 30
5 59
14 1
90 51
73 4
66 2
31 51
42 127
86 1178
28 3
16 45
34 1
32 841
57 1682
21 30
53 138
65 34
36 1
23 11
47 5451
93 1
82 314
9 35
23 40
69 1149
10 49
69 5
19 6
63 49
9 1
44 16
65 4
17 1
20 48
56 1
18 30
54 1
44 1
8 11
90 165
50 48
27 2
9 63
39 34
31 76
76 1
68 2
17 48
34 28
85 64
95 61
50 1
40 12
33 1
50 1
54 1
22 1
80 19
55 1
42 49
99 1061208
45 73
65 7
46 2
7 35
80 37
63 27
88 4
67 4
97 90
43 88
45 57
30 38
48 1
62 1
31 3
29 68
40 29
8 47
60 1
24 165
19 26
19 33
90 4
5 3
22 784
83 23
59 7
31 1
8 14
69 32
87 1
10 313
12 16
59 70
93 358
62 165
10 38
58 19
21 65
76 1
24 29
43 136
35 47
31 32
49 23
45 50
11 6325
35 15
18 6
27 1
17 37
85 49
42 65
55 47
50 8
11 66
44 1
32 27
18 1
40 80
10 33
62 58
39 1
36 44
55 69
44 1
41 24
12 9
8 33
41 1
71 1
86 71
26 216
9 76
86 64
23 1
75 26
88 4
73 2
41 16
8 1
13 1341
36 62405
18 39
91 7
66 2
9 63
52 32
16 1
50 5
37 9
43 11
30 1
16 30
69 58
11 35
35 66
14 38
11 8
56 33
31 3
43 39
27 1
10 11
5 14
76 922
39 17
7 53
36 64
79 102
58 3
44 165
27 8
48 33
62 1
54 153
47 185
52 5922
48 30
30 19
69 5040
31 17576
61 46
47 49
33 1
39 14
29 1
75 14
69 1
79 10
43 19
39 93
38 5
13 1
40 2
10 35
65 26
65 31
97 1
28 12
93 48
31 29
85 20
45 9
81 13
73 29
41 78
53 1
70 7
91 2
52 1
50 11
60 17
21 1
87 165
53 110
46 58
32 45
87 1
14 62
37 156
16 30
13 1
24 43
31 2
54 61
54 1
41 1
46 7
79 1
69 1
51 10
47 29
13 35
32 397
62 5
75 186
60 22
58 4
35 18
39 1
71 1
14 37
75 1
88 1
32 1
76 14722
75 1
57 8
66 102
19 86
41 2
32 7
13 29
44 2080
50 78
70 10
62 1
13 1
54 4
44 28
89 66
61 19683
73 1
74 144
32 7
29 59
55 38
45 38
45 1
63 49
23 35
30 40
29 14
28 172
21 45
69 11
48 195112
45 1
89 1
25 22
34 39
18 11
33 13
44 897
67 21
51 4
48 4
31 106
32 20
30 68
61 9
41 22
21 1
28 94
52 2
58 2
32 41
10 37
70 10404
36 39
43 41
83 143
90 1
76 1
65 358
28 8
6 15
36 1
69 35
99 225
67 1
85 1
82 1
45 177
29 27
32 1
59 1
50 18
19 27
35 4
42 9
40 68921
30 40
17 8
89 42
19 5
39 6
6 46
34 61
19 70
18 27
64 183661
14 3
78 28
20 6
29 1
73 24669
14 37
16 1
18 24
84 173
25 1
85 39
91 26
11 66
8 102
81 26
81 10404
86 46
96 15
51 4
55 85
48 3136
18 42
46 1924
67 49
6 8
86 74
38 1089
76 843
73 19
11 1
64 7
7 3
56 1
44 72
37 1
46 28
12 44334
41 448
31 1
13 1
63 64
45 108
18 44
33 17
73 9
54 1
55 1
10 16
84 34
44 8
18 1
18 10
83 1
61 14
61 67
50 783
40 408
32 8
18 12
52 15
59 58
38 102
73 10
40 21
27 1
7 65
10 35
73 10
50 34
45 39
8 35
72 59
71 1
49 517
66 27
28 36
26 3
34 8
13 18
50 4
61 26
19 165
52 8
70 1
49 15
60 1
75 19683
40 17
68 4401
25 1
57 1
74 26
13 481
66 1521
5 11
46 64
69 11
89 8
43 1
75 1
17 98
22 67
51 144
25 18
45 18
80 876
80 18
56 11
92 19
14 41
69 96
34 1
9 63
17 60
40 3175
4 50
19 10
14 6747
39 1
11 165
20 236
52 1
57 5
25 3
56 50
82 1
66 26
37 74
33 35
85 1
37 43
78 68
12 299
39 65
79 64
36 2
8 11
13 78
30 7
68 1764
33 5
52 354
23 1341
36 1
14 165
87 58
26 42
31 2
17 4
59 66
80 1
30 2
63 2
53 256
52 7
73 1
29 1
7 8
16 1
56 64
50 15
65 59
61 1
64 4
8 644
42 3166
27 5
90 8
26 8
69 33
51 22
30 1
18 4
22 31
62 1
41 1
53 1
56 65
46 30
84 153
95 116
22 55
26 30
60 26
42 106
40 9
59 39
9 27
53 28
43 169
80 11
38 4
79 1
46 7
15 305
9 202
68 55
27 59
94 1
21 61
66 1
44 24
54 65
70 10404
25 17
9 63
31 11
56 28
67 27
44 193
40 679
26 10973
77 1
26 1
26 242
17 29
36 34
68 27
86 57
33 17
16 195
13 5
5 23
76 1
83 1
77 15
19 3495
63 15
49 2069
49 110
80 24
59 222
45 83
51 58
91 1
36 35
58 1
12 64
88 4
6 1
46 1
59 34
34 54
32 144
24 1
29 146
9 63
36 48
47 100
17 1764
32 750
62 6
48 236
79 8
12 78
59 112
18 34
72 11
9 34
44 16
99 31
41 1
26 1
63 1
39 2
63 5
66 2
13 18
5 1
48 1
28 5
45 83
61 3
52 36
8 4
34 67
20 99
6 4
48 16
34 1
22 80
89 17
30 712
49 28
18 5
49 1
35 21
32 1
66 2
62 91
19 1
61 1
85 864
49 26
50 3
69 6
13 1
41 91
73 1
13 2
75 30
76 1
21 15
16 933
47 33
5 8
52 15
55 1
32 2
65 34
38 46
47 47
50 4569
8 4
25 10
50 1
36 1
62 163
41 7
13 79
23 1
35 2
50 2
28 5
67 7
81 1
7 16
17 35
56 21141
42 20
55 2
70 27
19 77
83 679
53 1
83 14
33 4
91 196
25 15
74 2177
63 15
72 222
93 2
49 1
44 8
57 5
49 1
25 1
17 1
85 1
9 46
81 37
44 28
80 6
73 1
30 1
26 1
58 104
10 21
65 9
19 38
24 1
20 79
17 567
33 56
74 5
16 17
8 15
54 2
95 37
66 1
71 1
41 4
18 30
73 1
10 38
80 16
15 14
65 184
17 29
63 1
8 47
34 20
52 42
55 1
67 5
26 6
79 33
43 1260
18 729
15 14
84 48
53 36
12 161
61 1
56 17
64 1
23 1
8 3
80 14
79 8
21 30
21 280
53 1253
22 20
24 1
19 113
62 1
16 4
38 28
42 3
70 7
58 33
89 22
85 377
66 1
74 35
69 1
5 59
43 155
8 4
15 58
89 14
40 82
95 4
54 1
56 1
90 61
83 85
33 1
16 7
77 17
49 9
12 193
29 1
39 38
55 864
76 283
53 34
40 33
28 2
76 26
15 5
37 105
37 37
42 43
50 8
26 222
40 14
89 1
28 141242
33 41
39 60
49 8
24 43
30 26
38 64
78 22
93 68
13 398
51 1
27 7
31 1
66 1
16 2
27 9
50 27
9 41
21 3
27 73
90 62
35 21952
80 1
4 9
20 102
25 5
48 844
32 1
82 13
60 1
89 23
35 15
36 8
47 1
45 102
74 38
24 7
39 1117
52 94
79 2
11 2
43 50
51 52
34 7
46 26
27 1
51 53
32 64
59 102
87 34
52 1
37 1
32 1
73 30
11 37
45 48
42 1
24 60
40 1
63 27
16 1
20 1
39 10
61 1
69 68
22 68921
82 8
32 122
41 1
94 5
83 37
25 8
50 69
51 36
13 3
17 5
53 100470
32 56
16 28
52 1
83 68
41 2357
30 2
89 27
12 1
31 63
86 1
47 102
14 16
12 77
50 17
35 176
73 2
14 11
49 250
51 11
54 114
25 58
78 242
16 30
28 1405
64 28
18 1
23 35
32 864
59 2
21 4
72 4
39 5
9 65
73 7
17 1
22 1
62 29
68 26
63 1
31 57
32 35
6 238
60 7
51 2
33 34
94 8
19 5
28 68
21 8
38 957
44 11
8 5
90 69
16 58
76 1
15 1
29 48276
69 1
53 71
82 38
10 37
25 42
52 1
60 4
32 1
23 11
53 3
33 5
42 253
75 23
25 9
52 1
70 1
5 41
24 1369
22 5
9 80
15 1
59 26
74 12
78 343
6 4
98 55
85 1
7 58
46 5
32 2
6 49
13 39
53 1
20 31
21 36
31 49
50 9
40 4
24 195
97 8
};
\addlegendentry{Lev Margin 0}
\addplot [draw=steelblue31119180, fill=steelblue31119180, mark=*, only marks]
table{%
x  y
6 809
24 12540
36 53800
59 34162
51 700
18 519
28 4184
65 83294
62 55274
15 15979
53 13841
41 23538
33 9101
51 1022
24 27172
36 8140
41 32883
51 14603
40 32882
52 8119
5 113
89 49372
55 4266
92 14839
29 26412
10 7799
24 23211
49 41879
51 31387
12 5880
65 127928
41 57332
38 23940
96 66192
13 3378
75 11146
88 84840
58 52570
30 3151
5 4661
13 13464
23 20487
36 21811
61 3086
32 756205
11 7145
9 1667
19 12210
49 58508
6 4189
75 969
36 26836
30 17095
82 2758945
5 6848
27 21866
37 23858
49 43970
50 27314
39 24705
16 14948
12 3118
44 33727
11 6340
26 5204
29 25506
21 21102
29 72055
36 8400
30 55417
75 2063
52 13905
37 30039
78 13146
71 9942
43 49896
46 46395
29 72055
30 120759
69 67463
36 2377
27 18917
30 53999
73 59251
8 14929
14 2787
19 18452
72 17668
39 6549
25 20529
32 29476
46 2868
26 28347
23 14589
15 2252
49 5456
33 32191
15 5070
70 60123
51 68503
17 17684
35 31956
37 44278
86 89315
68 57625
45 52339
55 38678
55 33058
65 248824
15 8486
51 27138
20 15102
24 19652
16 12073
36 54859
42 20691
8 5195
40 4948
54 3676
17 3972
27 31012
23 24794
41 5128
14 163
8 3686
70 4222
89 63703
8 7244
29 29379
78 64496
31 20061
59 78685
12 4715
36 27955
33 2762
27 39578
40 1504
32 12456
62 35826
46 29168
19 18998
14 13963
66 216106
31 24452
7 6192
12 3118
49 12514
16 887
18 2590
11 362
20 7361
69 85262
58 91098
48 82285
7 8302
29 2810
71 36386
10 472
19 24546
66 37107
71 63518
87 67891
78 58763
51 3301
45 180311
30 3564
51 3714
19 11440
22 14867
60 560567
87 3126013
95 62227
32 27569
29 4247
36 28036
42 32482
30 18479
65 43194
46 19563
19 13341
25 13490
20 2201
55 50842
37 27616
37 20564
63 8766
11 7027
15 6857
89 1781
88 5762
27 22314
94 81020
71 58140
69 54058
35 5309
37 69824
15 11283
46 37720
34 5863
45 30873
64 59089
8 5195
35 21865
28 6636
66 24850
47 12594
51 6811
18 2748
30 181234
34 12707
16 5591
61 78395
28 2801
53 107048
33 1422
28 13620
74 83536
21 3035
68 34167
29 72055
39 24754
31 178497
33 20442
13 6609
35 36728
83 46073
88 5772
26 47726
77 66291
67 71018
34 22259
9 1362
92 1819
44 7314
59 2051
27 26257
37 36644
50 6160
41 22695
22 884
89 4407
26 18958
33 15173
86 4522
64 33980
53 24449
94 4845
37 180646
30 28123
48 1730888
49 38387
62 55628
82 123529
39 15518
25 2986
11 28715
16 439
40 29746
28 10537
28 13457
97 26397
31 33002
7 6750
39 41304
29 27044
30 18263
10 1829
10 5881
23 14586
81 89843
50 182108
26 22749
88 47697
83 74637
11 8319
24 6398
17 7346
71 30657
32 2037
20 31128
26 54
29 17651
52 10489
17 11000
8 5195
40 24180
19 24475
41 21634
55 42852
80 14924
10 6271
60 56666
49 37249
26 31592
34 35360
12 2861
9 159
63 10695
34 14302
26 3696
68 85236
15 18468
52 44712
65 88978
16 1704
8 12442
39 32449
20 24317
49 40860
37 35876
77 6441
59 2192
86 4109
18 17031
45 2880
52 2012
36 21298
26 10492
28 16348
24 9035
21 13143
42 42868
26 11273
77 68403
82 79680
67 87097
55 49331
25 19289
16 2056
11 1460
10 1277
16 11308
16 16821
67 55129
25 2250
23 23640
29 3131
97 24091
41 50199
31 57199
4 525
28 13610
40 22984
29 4045
63 56345
34 53381
26 46098
27 10938
53 48066
43 116944
29 72055
72 50163
23 24321
23 1645
33 9560
70 31564
19 4523
88 107498
56 44182
41 4713
48 23195
46 20691
19 11026
27 31741
58 35024
25 19459
16 13859
50 60321
9 3550
91 19714
70 29635
56 4184
94 30367
63 9365
13 1950
40 19920
63 54393
43 24367
23 19076
69 689287
45 28936
49 1395
47 33402
43 131806
19 12507
62 15802
30 1262
32 548
11 1460
32 55669
27 27139
69 114891
23 2830
38 27623
82 206450
65 3804
24 91102
26 3152
14 7998
19 9522
53 54284
75 402305
39 770
9 6021
37 1788
45 29875
60 17834
12 279347
38 8193
64 1091495
18 947
28 4675
30 23798
25 462391
36 37644
84 1286
30 19989
27 22707
61 728118
46 29499
23 3013
72 36718
28 38279
97 12049
63 2114
26 3012
9 1601
45 7204
23 8614
33 4900
54 91740
18 15379
21 2517
89 24243
67 61284
29 28516
64 76746
29 17422
76 1088863
41 67784
17 923
36 22089
53 54052
14 7067
10 17759
49 42904
40 5548
43 99403
39 31780
81 273594
14 1562
21 500
9 1506
30 10866
52 2276
25 11030
40 32242
45 41063
13 10177
33 26227
50 33097
75 70150
60 4209
88 74322
32 7632
59 22156
35 342491
18 16178
15 15916
45 50958
16 19251
76 59356
56 433
44 6241
27 959
48 34744
91 67968
21 17680
72 60290
38 1392
8 141
33 25963
94 91958
47 9866
22 20389
20 21992
21 10804
13 4033
47 41793
33 7504
11 7027
8 2305
39 4246
49 43250
25 15682
84 83492
17 8946
22 24708
40 31387
81 168515
50 51788
19 18998
37 31533
77 79838
27 18230
42 51796
12 2093
33 3280
11 5558
53 6586
15 12860
87 4734
34 23730
61 6973
7 419
79 60996
91 73876
32 599960
33 21434
8 6589
16 1480
9 7422
9 11690
21 2783
16 41928
12 4530
30 16193
10 1414
25 12083
56 2137
57 30042
10 2758
66 10502
44 103935
33 2791
63 69558
44 88285
31 2701
7 24209
54 9296
43 12547
55 31438
52 29000
32 875
40 8225
5 606
11 3789
23 40584
39 230834
50 26407
58 68964
89 11769
38 60484
23 10930
55 4970
63 7023
20 896
93 22056
63 10266
8 286630
50 5009
58 23348
50 848
82 67868
39 9081
32 23260
56 6115
41 32498
26 27313
13 6675
12 2634
12 1613
12 8181
42 8528
17 4060
27 9616
78 68646
34 31219
10 1414
35 92245
43 5536
18 4145
67 169743
55 95970
46 34637
42 43109
51 52644
57 1379
26 31671
36 451
43 35074
49 210447
33 222461
22 2537
68 82799
65 7692
69 30729
56 39554
74 77560
39 30884
10 7799
27 2538
87 74999
30 17173
31 1867
43 617525
66 53864
47 53342
29 18930
99 15948
34 2622
8 7244
43 79965
23 2800
38 15813
8 5195
11 4037
57 1608
9 10111
46 6559
27 4690
52 27013
50 55344
82 64337
7 88794
59 17767
72 57266
60 39291
58 45213
5 4661
52 10447
26 19506
6 6216
13 211778
31 15282
13 11096
36 38628
38 769
56 8602
27 20332
21 5653
12 5191
12 48412
28 9778
64 3808
99 2975
60 34001
11 399
41 20533
14 13975
63 106889
19 851
20 13274
32 582
47 28466
25 1152
70 8404
44 20537
28 50112
73 27884
19 12146
54 9026
16 9202
38 35163
30 7053
96 103384
41 28981
11 1960
42 46184
60 29013
5 4121
35 21298
26 17842
44 7382
9 171
84 30301
46 23802
16 41928
20 3442
67 32093
24 10206
20 9446
42 3932
56 51353
23 578096
12 1204
41 63281
6 892
80 268585
5 3783
12 8305
67 63716
6 892
42 14659
31 18226
8 141
48 1085
38 18928
43 35780
57 7806
23 2825
19 18020
32 23859
74 3970
23 22511
80 85913
64 819
18 9800
4 525
56 21387
60 6560
75 78511
32 4972
69 1498
30 22837
43 4359
79 55028
16 17615
65 59428
54 148730
41 37604
12 1754
8 1148
28 6020
42 13693
48 1979
30 2869
20 3293
56 27963
37 54508
74 17542
14 1789
14 1838
50 6058
97 66878
42 43004
51 30009
18 8770
85 115534
19 17777
12 1787
85 2457
20 23874
38 4332
68 49110
73 56323
70 68466
30 50545
41 21883
44 55077
35 19383
61 40171
23 9074
33 1548212
60 47784
95 31686
21 10738
11 3090
87 846444
6 897
26 2684
80 126986
48 36701
19 12722
64 56694
32 22842
15 11609
19 12507
21 15858
22 5362
26 2040
53 55652
65 4947
46 35997
14 2357
11 3789
63 38798
25 16271
16 20132
49 1408
4 3390
31 33842
14 5375
83 105524
72 69083
68 38388
42 40515
8 5195
20 16306
52 2530
20 2580
28 15922
14 5837
46 30154
31 15441
73 11322
84 28933
11 6221
24 3286
23 5903
16 5306
12 646
6 342
50 158521
5 2084
31 54456
42 8290
30 22864
52 3304
35 25211
31 13583
36 65489
71 3337
61 42546
19 2767
91 937158
14 8479
35 8554
13 959
54 183182
24 8411
15 894
48 1494
30 26945
27 3640
58 49211
76 84038
88 60914
20 14649
82 3368
11 1649
94 82942
28 21746
31 182493
9 132096
76 38683
35 14678
9 9711
27 27876
14 1485
9 1263
55 33657
97 112085
46 10484
7 77263
25 24712
35 8549
48 24435
55 45822
40 3807
69 28493
66 11350
40 6536
84 10270
42 26052
9 16
37 21300
71 60196
7 724
19 10788
49 1707
72 64766
39 45156
7 516
19 5994
71 44659
29 14513
81 73910
12 2096
37 33890
30 66301
53 312877
14 6963
22 28260
30 27081
12 26967
31 29834
70 63795
24 7815
31 34468
28 1812
36 15581
18 351
14 8306
30 16942
88 9541
66 18826
47 15786
91 69945
10 9626
7 1089
49 38589
71 5413
37 33843
27 17496
8 14929
47 2132
18 8938
25 368056
12 6104
31 405972
39 40502
37 10135
25 31076
18 2393
23 13013
17 21721
30 21176
46 2896
22 2352
8 5195
25 235
38 13089
34 30344
28 19933
61 2950
11 8316
38 8085
40 1799
73 49868
27 17584
90 10189
66 67933
10 1991
82 58996
23 10686
70 47599
41 59356
30 10743
56 68972
73 218801
43 5442
11 1417
52 16901
51 9584
95 37673
71 2631484
91 85284
63 52531
5 606
75 203072
46 45129
46 43398
36 28883
74 22815
42 58037
30 361
15 6642
15 8261
25 2867
26 22873
39 1515528
82 66654
34 14688
14 11990
28 120651
72 12527
12 5191
13 12621
16 421035
89 185216
22 17482
5 606
53 11578
68 36463
47 839836
67 50177
64 202739
32 8666
53 2146872
53 57041
76 4753
19 16242
60 32719
73 2343
84 510600
31 2684
28 3856
11 680
5 82923
53 3914
15 4165
37 14652
25 41699
41 326052
65 1458
11 104184
9 1362
12 552940
49 29533
21 19280
62 46306
24 19161
17 370953
32 7242
21 15154
11 5558
42 30770
45 3254
70 3593
38 822003
70 47241
31 21817
16 41928
60 54341
14 1536
42 59269
17 22902
16 16772
37 17650
40 1907
12 371
31 152585
39 36567
28 16866
92 7240
18 10316
95 28808
60 38444
40 24820
7 12673
68 5016
22 2254
97 20482
43 45994
17 4877
40 15713
20 2942
40 19646
68 1861
44 530945
24 16273
32 37328
48 22688
33 31512
10 7799
65 5874
71 13132
34 376801
22 20389
49 5477
53 29635
37 19100
24 21348
27 25015
16 7245
75 1691
45 25077
66 57916
57 32520
37 7823
15 11616
76 2385
34 38973
41 12066
34 11353
92 8837994
13 3176
20 23960
67 13216
7 3128
29 20466
49 60176
55 27161
26 19890
47 783
32 4543
76 17438
66 60467
32 13731
43 378950
33 284
31 12022
94 12413
50 35868
52 6252
29 29136
18 14820
22 534
31 15207
54 2421
51 22017
44 24881
30 30692
30 2854
83 1818930
59 14889
22 25956
62 63009
59 110678
22 8562
35 24944
52 1395
72 13405
46 44439
67 2862
54 48193
6 1283
21 22937
50 6039
10 4656
88 72194
42 61543
34 5253
30 1519
32 23410
12 2311
9 1362
72 76308
9 4117
9 1369
92 74611
28 24363
33 16057
33 16946
69 72473
63 49804
17 3309
9 9259
90 81605
42 5428
7 4225
25 9603
18 17975
76 3618
22 4805
72 18557
16 385
17 8946
54 54712
20 20031
45 4331
12 3118
8 7120
57 72077
41 46613
77 1215290
19 7548
9 132096
89 13220
40 812
30 29897
12 7311
36 42404
5 113
28 6042
24 10868
80 22368
69 789
72 2453
91 197487
49 25657
28 26564
62 43744
7 77263
50 67361
50 38085
37 213674
15 5069
70 13593
26 1824
6 3138
61 69966
45 16699
68 55871
40 10668
23 29156
19 14302
43 35815
32 13560
5 4379
24 28714
25 15318
49 585306
92 3784
9 11184
16 9670
38 41369
50 1525
30 208913
55 14166
13 2620
27 8939
22 2440
42 2165
35 18729
10 4538
53 56243
57 343012
67 131246
19 1254
69 115892
76 49976
68 872
11 5502
32 12862
34 36068
13 9147
13 2739
10 10980
29 5431
57 35190
47 602
19 12463
19 17487
44 43059
40 30334
81 22886
43 34746
32 30244
23 1662
35 31123
12 3353
45 15529
45 3019
41 78129
17 2509
22 534
30 11714
77 95155
15 13938
25 8742
25 7175
6 7288
57 3779
29 35264
6 892
50 28631
20 14649
88 33029
59 46580
37 95239
98 3368
44 4644
61 2540
95 1207
11 19221
73 84755
76 29346
41 40561
17 16960
33 22177
47 11203
65 55791
72 52264
23 21185
48 20438
85 106784
26 416144
90 85176
26 7820
41 13494
38 5359
8 5195
37 26291
54 7891
67 15127
41 45914
82 35752
16 8631
26 2881
17 27326
49 6231
66 17078
20 20031
83 53233
84 1296
26 11963
13 7244
21 69876
7 53
30 2090
76 822
19 9808
47 43763
66 51750
46 68979
84 1107972
35 9811
58 25856
61 26812
12 123280
35 36324
14 4650
32 36004
79 76147
13 13300
35 45220
22 25317
55 43768
29 226621
51 67229
60 51186
37 42344
50 122136
55 51662
78 156580
35 6850
85 23300
70 46448
9 8125
41 33410
8 141
10 7554
41 14327
40 8254
36 35635
14 9729
49 88152
51 48676
16 13715
18 6656
77 180911
39 129300
15 13265
39 3930
4 525
20 31279
31 3735
47 37668
12 50404
14 2357
42 35432
40 37216
19 18360
94 1387
37 35620
73 18116
20 9446
34 14100
75 85119
28 1896
18 11454
32 25291
34 473
48 48672
33 34302
36 2564
31 2726
48 13227
73 2238840
21 6381
44 17153
33 19509
38 40552
73 70899
59 6400
59 25462
10 4656
36 168237
75 8243
34 31520
52 48240
64 26843
45 5227
70 742123
39 34129
31 15195
14 12657
70 20034
15 5726
28 4477
34 31220
74 27603
21 2230
19 8889
29 14571
12 11612
21 155906
15 13044
64 13289
11 1960
37 35685
42 4340
98 3679
51 3150
41 147329
10 7911
20 14559
15 20407
38 11874
14 12657
57 75486
26 627353
44 9812
23 10738
29 25892
94 3165
47 28981
69 1498
23 21558
59 5163
34 45032
6 158554
46 58228
90 100840
56 51948
18 203636
8 5195
13 424626
33 32480
10 7799
49 1314935
29 6849
10 5683
12 1927
27 8425
55 53604
22 6475
67 3582
6 1283
37 34764
70 84227
22 15026
48 10754
25 11153
40 3504
21 9640
20 23181
10 576578
15 17710
75 58739
61 23699
8 173202
39 341590
4 3327
37 775
44 30627
46 42117
99 3048
12 15185
97 6411
18 12150
67 12418
37 20914
32 5470
32 12093
43 27690
35 41307
11 1642
22 187060
62 393852
88 70139
19 8129
6 1283
20 19322
54 43570
14 12290
44 4068
14 11172
89 92708
35 12547
13 914
40 23437
53 45355
65 55627
35 36068
60 59507
51 43878
69 87204
29 3004
94 89167
23 9177
70 74528
59 787
57 14931
35 2098
76 73917
48 66450
46 23126
71 91589
43 3371
19 12722
22 26388
28 27925
17 3890
25 15623
91 72620
83 4977
22 28132
46 40690
31 56884
32 27952
54 4754
97 48370
8 90138
96 92208
26 49506
29 51593
31 32419
59 37403
50 18312
39 33715
17 1528
26 8608
24 27639
44 9881
18 30284
33 28945
42 4585
57 58201
40 28244
6 7901
92 94511
22 190269
42 724
49 40938
12 6459
29 72525
95 2240
41 385514
45 30984
27 40215
27 19040
17 16115
33 37518
45 5867
38 42751
65 27138
34 13238
75 42158
30 6091
26 10043
14 1789
22 596
60 28250
35 36007
58 8981
47 60901
57 1397
51 853
7 710
10 9626
69 44665
34 26697
47 6037
24 22263
39 1297
59 46356
7 296052
74 62538
22 17030
34 45714
88 180784
31 26353
14 6696
8 8403
77 1979704
50 39086
23 10751
16 41928
16 2499
94 13516
37 3071
40 4259
43 3749
78 4788
62 2641
11 3789
84 59215
92 25926
80 1679369
70 19966
48 35187
16 12158
56 67725
24 10422
41 50414
82 20492
5 4992
16 41928
73 132242
62 54473
94 125435
47 6350
11 7142
32 389769
29 10162
52 35010
86 21786
7 690
6 4406
13 16676
28 3336
48 26566
28 9036
26 13072
36 32051
19 709326
43 41012
35 7572
50 68008
56 386
59 622
24 3816
84 2625
38 63652
96 143215
85 47711
29 3479
11 12933
24 2923
15 1004
19 8889
7 7588
22 5598
51 25246
72 1269432
58 6947
44 970
90 3316
25 8877
15 14381
30 894596
53 57561
31 8492
42 734
20 13284
80 179023
82 52973
50 9408
25 13443
19 23460
35 31010
55 20360
31 47870
28 26033
84 1245491
63 14779
52 14366
84 1474956
67 5633
54 1248
76 71780
41 12712
47 37732
26 13362
27 15684
17 2153
69 16952
49 6328
65 5755960
54 7291
8 3600
8 374393
17 16028
15 12990
11 10391
10 6537
36 1547
57 87084
24 2867
69 48097
17 16978
24 54806
5 4661
77 37780
81 74672
32 6247
56 93097
41 10576
27 19719
93 26088
28 3729
67 102930
18 16811
31 24375
13 366279
22 10580
8 5195
38 46523
4 525
38 6454
6 4679
57 126081
38 37644
50 588
12 6384
48 38131
70 21409
47 33984
52 1007
71 2878
91 71353
36 149794
9 5860
35 5036
6 1005
49 1364783
33 3153
76 931535
11 7733
80 17154
66 74774
87 13425
21 589
60 88430
46 41961
30 15757
58 20175
24 98501
43 35448
77 195655
48 2322
8 5195
95 285565
36 18994
38 26930
22 792822
47 37164
62 25869
95 74011
73 82153
43 20978
54 33090
54 15348
8 5195
44 48868
14 8479
7 8678
22 1466
62 2096
22 3323
21 1096775
31 4284
62 844
9 135620
23 884
35 29491
30 5376
76 92554
82 260608
17 6450
29 23762
53 42338
24 11960
16 5761
22 3038
26 22116
41 34291
55 29230
51 51982
19 36597
8 5195
78 42141
31 4113
8 1102
64 71821
13 5585
46 8398
39 34159
27 10736
34 2613
46 80338
8 7961
33 36613
35 4778
13 9861
35 4932
13 7247
20 2498
69 23590
44 46707
9 132096
24 18179
53 46708
38 27845
40 635998
27 29721
19 18620
54 13337
65 3668
37 29573
35 3511
28 7101
12 25098
12 3118
12 323547
45 40140
39 3020
61 40016
96 81417
6 6126
18 2384
51 51863
21 23415
84 29724
49 46195
57 41978
47 78239
25 6425
72 53314
26 17829
11 299
17 76755
16 8631
72 67700
10 7799
89 283483
93 87469
21 196
32 389769
10 3545
50 27572
22 15140
5 1027
16 313420
15 9035
17 11959
26 34331
50 56347
17 13988
40 4740
39 4823
55 587
12 6373
57 6943
43 41004
39 142384
81 51928
20 1244
23 2524
60 2261
22 35524
30 29521
35 1339312
45 3644
24 8674
5 606
12 14497
21 12265
5 2307
12 4530
27 30817
10 3504
8 5111
44 47730
10 14822
44 10010
71 59673
78 95359
14 7404
27 5851
12 2561
49 8640
67 32730
26 12077
27 23867
14 2705
47 44852
22 2537
45 317674
71 40389
58 56051
62 60691
15 6152
28 52488
20 74795
69 57674
16 4887
33 6604
50 1141475
81 5722
15 21846
38 85244
41 4328
48 64893
27 3674
13 14659
48 67176
28 23169
69 10122
93 11893
76 22307
24 21028
94 120463
63 87784
6 6033
43 36010
49 61194
42 4350
67 67192
37 18935
30 316721
21 36406
93 2270961
11 3230
21 127862
17 10095
25 1147
19 1704
61 39496
35 18098
60 29196
90 903
43 71459
6 4277
42 4156
56 32890
46 42077
67 26011
95 820934
38 30380
9 17293
57 5853
52 44722
21 3815
35 30124
17 4582
10 1414
8 81701
15 10454
16 1581
15 13540
33 13277
29 72738
42 32501
30 31708
21 7729
27 20978
39 6972
88 109953
86 4160
22 2921
64 26843
29 3160
36 904809
51 9999
30 55394
54 56228
14 9495
79 9975
20 944
26 18986
46 7773
69 88740
42 5016
16 1844
47 20686
48 23466
36 840419
33 47831
16 1758
39 1184915
72 67891
49 4590
70 24131
20 2089
9 132096
32 30435
42 46903
33 26980
25 387
63 113157
40 33726
20 14326
61 11569
19 315895
53 352043
57 22898
57 52898
75 14361
44 10938
31 11308
43 38628
14 3583
50 4229
67 88290
79 60859
26 354
47 25762
73 43170
68 52193
15 19775
16 1248
30 24370
54 14679
44 34045
54 46327
43 32818
47 13244
72 64792
57 1583
27 18909
53 1857
9 7829
86 47435
61 592060
26 4963
37 1111287
33 16401
49 94229
54 14971
17 21348
17 3164
36 20082
84 99009
33 23732
12 12586
8 2421
74 48897
8 141
91 72740
88 1257
30 36702
45 67579
48 16417
13 1512
10 1628
47 42611
30 13015
14 12657
34 35380
54 263716
88 15985
75 117938
34 4643
32 23218
30 25011
59 6069
68 8138
17 17861
43 4655
18 4679
41 42327
69 5004
36 339767
21 21102
31 432
57 12437
20 2489
46 38043
34 1410
41 22130
56 16703
42 25138
29 49306
9 26421
14 6749
94 13436
24 18547
87 58919
22 42116
44 42252
41 7727
61 45555
59 44971
57 51188
21 96482
90 891315
8 5742
16 77578
59 15928
61 146793
34 28643
64 2078
28 701562
68 1005
94 48126
43 35871
34 24819
41 46156
8 1585
7 14
20 13274
40 8561
44 8077
18 2384
21 9495
44 1939
41 11416
89 23649
10 10980
47 56336
46 13525
77 88082
45 49462
5 5001
50 42943
39 29092
51 1347513
36 9242
38 144105
26 6437
17 17403
32 1971
13 2022
31 9508
12 12149
87 15806
47 19100
70 683580
19 11516
72 90919
17 22974
20 10990
10 7799
90 218996
43 25429
34 14159
27 40642
12 1241
38 8976
28 23425
37 1994
66 5466
19 62
56 30303
28 4496
6 892
39 2826
43 33907
26 13134
16 2386
18 15894
96 104407
26 3391
44 36906
37 4684
78 14633
44 1979
33 6355
59 37745
43 62899
26 7157
26 1055
25 58633
39 32306
46 38900
35 25339
26 4480
24 13503
37 19570
38 36584
59 68299
72 1280
20 1426
39 291345
63 5210
33 24481
41 67327
44 8142
26 375
16 14899
60 38993
24 17074
54 16500
41 4659
50 52082
76 93577
42 390
78 27199
81 50857
52 27579
18 132177
47 2518
28 27556
8 3456
66 70339
57 9185
71 44076
65 66419
14 2604
86 1539
62 140500
54 28498
85 74542
16 6098
26 59731
56 15099
21 3369
38 936620
65 14723
38 2830
12 5191
13 1957
68 28456
62 53897
55 6469
71 25465
26 820337
21 24868
52 28531
12 323547
12 4530
33 46101
86 81394
25 2167
82 23178
73 3443
27 21936
62 147468
33 60031
10 9652
22 5939
27 17270
22 15989
92 620206
8 3492
92 270249
59 41929
};
\addlegendentry{Lev Margin 1}
\addplot [draw=darkorange25512714, fill=darkorange25512714, mark=*, only marks]
table{%
x  y
38 1761664
32 3855870
20 852624
45 14663997
47 376033
58 18231184
56 170260
31 73726
53 1831132
95 57830719
43 863387
67 12696630
17 2098301
59 24778010
17 2240035
44 204177
37 5681724
18 4105974
77 22229933
28 9289358
63 20663115
34 274048
24 373626
41 6573601
34 879291
14 1024069
12 1349310
20 2155283
18 72053
88 16734099
31 3708284
36 13084204
25 6835180
93 2808110
16 3553168
56 11481040
12 98281
9 97953
30 57774
44 1157505
13 198316
11 1144472
33 1564362
18 450595
89 64028789
87 70370367
17 524593
31 7766430
70 314048
38 1567762
20 3909452
28 2062680
28 91010
70 6947754
73 7661113
25 4855540
26 552551
99 31844218
40 1895407
37 13333940
78 277626
29 1168007
32 7162021
21 825779
34 10010353
23 1583408
41 1036751
55 11137630
13 3663628
16 363587
32 2753832
44 9426605
18 473850
73 32731442
38 3692891
18 1147019
37 3735747
35 8479370
10 905170
10 243005
94 64472605
9 647530
11 1151495
78 26113548
46 10765001
77 5541437
10 1043338
47 19502165
27 2655775
46 2574607
23 427206
67 3640486
28 3242461
51 101926
22 3068263
15 2499221
64 6120523
9 465895
22 142208
69 15620809
14 1944834
23 308507
15 1707842
93 132267574
94 30772385
49 53767
51 1288510
31 4157788
36 2245258
75 45804139
38 4189522
29 118014
19 3979566
28 8234093
71 225754
89 14196814
43 8750889
81 36495261
20 426307
67 5846382
32 5680355
60 4539524
48 11022011
65 209419
53 12279975
45 4109575
29 4622928
84 21220569
25 13972
6 194921
28 794156
21 3607590
28 1826333
71 301476
15 1122566
6 62603
44 111138
39 1071361
42 4109376
27 3036119
70 1215027
51 2734288
83 44022366
10 44983
33 110107
30 2664473
17 56693
91 1364319
45 1319440
22 852847
43 1244376
17 818088
26 637762
72 8837767
77 25673602
43 6008438
27 304758
16 517958
77 2982644
26 2183197
52 15832812
57 2835179
52 15714564
63 18542731
35 169259
6 97330
17 1147370
71 18533479
38 697588
8 40212
49 195792
16 383632
82 7117191
48 4412856
93 7015936
29 8594103
54 14295824
32 12756906
67 3130051
52 7148833
28 666475
58 6405091
84 59392548
8 123221
15 967059
9 465999
7 188562
6 24605
58 3018070
21 2561596
10 1177361
58 3323839
44 8694774
75 20579660
61 4706380
27 667003
19 3745351
69 13741395
31 1862748
64 3634555
95 20580395
37 3660572
13 212715
40 7285659
21 1856976
76 17229884
81 7172206
9 324554
22 6504680
41 2115177
13 1449883
27 116383
37 6065307
20 3639302
26 1247232
70 21211953
62 18734022
30 805586
79 33243332
30 1710040
14 1834072
47 5819138
22 2398531
17 902462
19 1416971
40 3339799
26 1927936
60 45387739
43 7314223
25 6443213
87 29107942
20 939456
12 23357
64 1821499
25 1707639
60 265841
19 47951
33 981398
23 5107976
44 3997746
8 552839
49 13704422
97 810293
17 152871
28 2506013
36 4948476
39 8884677
29 6233750
45 586596
33 1018537
28 2626381
20 3227542
56 1742804
29 24845250
26 3019089
43 10768854
31 1839495
72 14184771
31 13614399
67 530695
48 879660
9 465895
35 14038742
15 187966
31 2833503
14 30761
79 5954153
55 8319688
18 296456
41 1407501
5 168382
34 4183717
10 292618
22 1511366
42 23198195
18 2165357
25 1763806
60 11093414
85 9994808
71 20033626
64 7381161
75 18302942
29 1532426
22 1716324
24 7026151
9 465895
17 1958460
59 4478990
15 346386
12 1120171
19 946285
47 17890688
45 432960
18 416429
31 3218201
32 318276
80 12028690
50 21320199
21 333593
36 4413871
14 706459
81 4605047
19 325104
57 4769493
12 202777
38 893577
41 1482981
51 2134222
40 1515244
20 7405699
69 401011
23 1347873
49 7383413
17 681713
82 1083184
90 2614276
15 1106943
16 2011956
7 86849
14 988075
26 3319892
9 465895
50 1590100
35 7494305
94 26750649
93 54460371
31 592646
71 3539684
19 668622
78 1468789
56 13430808
67 6507949
69 36953961
95 3956937
31 516381
9 98191
34 5685332
34 199649
79 2094422
8 71207
65 38140859
17 1747264
40 2103208
63 1842640
31 1710697
35 371360
20 626504
10 205379
50 7589275
92 43435016
18 157469
9 465895
90 58438334
13 1356414
22 313996
27 921149
20 135819
12 103803
42 2663809
13 82684
67 29251420
25 102672
24 4134615
10 247381
31 2231545
34 5085549
8 117863
71 13023623
38 4614217
34 1830874
12 132905
87 411989
57 20233643
76 931172
45 2101419
69 17198199
47 8210287
41 3279947
27 212117
22 1592700
22 6180583
19 173415
70 3360830
68 18396176
20 2124598
92 913981
21 298944
62 6856589
89 5477841
40 2498465
65 4420883
56 17824982
50 5831589
32 9388095
49 1322471
40 2765163
32 3138123
58 3681212
25 843354
14 80935
34 2447309
75 19555608
64 1339216
25 1763806
6 248157
82 8951454
26 1268632
52 5206968
89 5268507
63 178786
92 21230442
62 17961852
11 840450
65 20408776
21 1321530
27 6793334
36 5642902
7 507158
41 3302166
78 8686808
30 4070744
29 23930
45 5268370
39 51094
52 294886
12 567941
45 1596441
69 543839
98 54555994
83 4317561
41 3215203
31 5326205
25 2726523
34 9380874
58 198365
92 7572232
46 6437125
17 1245479
23 4593867
25 3584720
9 257743
44 318363
18 1001588
31 1405128
49 6009197
89 7417389
10 54671
85 38020200
59 26805586
56 2672193
61 11632730
51 1781993
75 6023792
55 27285531
67 37384626
14 888395
66 25924466
94 15843983
81 39292365
24 263184
31 5876139
69 7515243
43 2284542
59 16528827
31 2447515
23 2205211
89 16744354
25 1763806
56 10538801
41 159580
21 1014561
14 44983
25 1700293
69 333032
20 1377913
11 2751407
45 9099017
87 66798957
39 7381880
12 1777382
43 4310344
56 2657979
23 2540438
30 1720689
88 21333512
52 1077453
73 2618201
33 6200603
62 30414903
23 7607726
37 1995762
11 127838
17 1147370
69 44363444
59 3567728
35 1298229
64 3913902
16 382593
27 9884594
59 2083641
79 433257
60 29115490
42 6481432
52 1777430
64 13315278
81 50417390
39 2507633
44 5205210
13 7209
55 4248729
20 1082789
68 37798602
30 4779899
14 1046509
66 911667
66 28290350
46 12240615
58 17761165
27 1053551
96 48850214
78 2999069
17 1493715
14 413855
71 14616945
47 4448260
53 2442938
24 4917470
19 1066225
44 1108616
53 4510127
55 11153978
32 536649
23 1521489
78 25447202
38 2203828
46 140609
22 2435247
35 1470674
55 5119997
82 12484843
23 1355210
53 542694
52 5015660
50 15711253
57 5824750
57 17639849
53 339113
25 745334
35 4058742
35 419379
69 3513886
38 5052390
44 6886913
28 381027
40 508815
30 2152471
19 1605944
52 2677229
57 22688807
30 5931573
19 3939238
9 1183577
9 227049
45 4924114
22 6022528
87 25820739
38 184523
90 15149304
75 2451527
17 2765689
63 1240879
70 984094
29 9636833
15 1517358
19 671543
30 3876817
13 893298
6 95208
55 15555118
40 3626265
79 25929202
15 931207
48 8445678
10 427586
16 1972544
19 3980668
38 18709560
36 3967608
23 681845
56 291719
19 1751408
41 7603338
18 1276173
12 231397
14 143867
33 11463783
63 7452880
46 5632256
33 2553597
17 935329
66 61010085
10 229820
80 9286789
47 9629772
27 2371774
57 10355285
33 1765913
25 2413264
29 640318
67 307568
17 614532
29 4633303
7 419115
8 206051
7 240705
20 1698980
34 5005243
11 1540537
51 7671717
11 202293
58 35161545
15 262944
78 25939934
45 1851881
27 306287
65 1652167
24 1083911
50 4230285
44 4307935
21 751837
91 66900340
32 382047
72 3212922
59 15909735
37 989862
13 1480243
11 324975
13 4168516
42 6855146
34 8332051
11 89594
31 64078
86 28366716
71 25242607
71 5350924
7 491503
33 1055764
14 496526
22 890830
38 2089033
22 2739249
50 29572233
28 1936674
23 390479
65 49247702
39 329597
34 376469
30 1581071
7 33717
42 392941
89 18180261
11 454262
69 7865479
33 1771099
98 625711
25 387474
85 6186750
31 3252950
80 33894037
46 894220
32 2086446
33 4917669
12 1043348
5 9423
18 1190719
14 1905590
20 3227542
13 2385281
41 1054266
56 11652174
22 3543899
36 1548383
67 2467336
54 9993382
29 582824
73 5299434
27 903604
9 355239
23 3177346
37 2897492
26 2828988
82 20931997
58 2902750
69 18457899
22 830695
53 25826397
60 3010051
43 10797040
51 26364309
93 46571803
79 5472781
13 110028
75 118937726
59 20777431
13 245974
43 18146083
34 5544888
95 34889303
66 1592754
40 669602
65 1259786
20 148078
19 1186008
66 8298021
36 1136849
59 4450429
70 29306258
};
\addlegendentry{Lev Margin 2}
\end{axis}

\end{tikzpicture}
}
\vspace{-0.8cm}
\caption{Language volume versus snippet length and edit distance for Python repairs.}
\label{fig:volumetric_plot}
\vspace{-0.2cm}
\end{wrapfigure}

For an intuition about the size of the language intersections involved in syntax repair, volumetric analysis will be helpful, particularly in understanding the influence of snippet length and edit distance on language intersection volume. To measure the intersection volume we will form the $L\big(\err\sigma, \Delta(\err{\sigma}, \sigma')\big)$ automaton, intersect it with the Python grammar, then automatize the resulting regular expression and finally compute the DFA transfer matrix using the method described in \S~\ref{sec:transfer_method} to obtain the exact volume. For a given (error, fix) pair, this tells us how many repairs of equal or lesser distance exist in the Python language. Plotting intersection volume across the full dataset (Fig.~\ref{fig:volumetric_plot}), we observe a strong positive correlation with the Levenshtein margin and a mild correlation with snippet length. Fully materializing $\ell_\cap$ is typically only feasible if we extend the Levenshtein radius up to one edit beyond the language edit distance (LED) $\big(\text{i.e., } d_{\max} \leq \text{LED}(\err\sigma)$\footnote{Where $\text{LED}(\err\sigma)$ is shorthand for $\text{LED}(\err\sigma, \ell) = \min \big\{d_{\max}: \mathbb{N}\mid \mathcal{L}\big(L(\err\sigma, d_{\max})\big) \cap \ell \neq \varnothing\big\}$ with $\ell$ being the Python language.}$+ 1\big)$ after which it grows too large to exhaustively generate and must be sampled. Across all snippets where $\Delta(\err{\sigma}, \sigma') < 4$, approximately 54\% matched $\text{LED}(\err\sigma)$, 35\% had an edit distance of $\text{LED}(\err\sigma) + 1$ and 11\% had a distance of $\text{LED}(\err\sigma) + 2$.

\clearpage\subsection{StackOverflow evaluation}\label{sec:rq2}

For our first experiment, we measure the top-1 precision of our repair procedure at various lengths and Levenshtein distances, comparing Tidyparse against Seq2Parse, and BIFI on the same test set. Each bin in the test set contains at least 50 distinct repairs sampled uniformly at random from the StackOverflow dataset, none of which were present in the training set of any repair model.\vspace{-0.2cm}

\begin{figure}[h!]
\resizebox{.29\textwidth}{!}{\begin{tikzpicture}
  \begin{axis}[
    xlabel={$|\sigma|$},
    ylabel={Precision@1},
    title={Tidyparse Repair Precision},
    ybar,
    axis lines*=left,
    xtick={0, 10, 20, 30, 40, 50, 60, 70},
    ytick={0, 0.1, 0.2, 0.3, 0.4, 0.5, 0.6, 0.7, 0.8, 0.9, 1.0},
    ymax=1.0,
    ymin=0.0,
    bar width=4pt,
  ]

  \addplot[green, fill=green] coordinates {(0, 1.0) (10, 1.0) (20, 1.0) (30, 1.0) (40, 1.0) (50, 1.0) (60, 1.0) (70, 1.0)};
  \addplot[blue, fill=blue] coordinates {(0, 0.3) (10, 0.286) (20, 0.205) (30, 0.433) (40, 0.256) (50, 0.296) (60, 0.236) (70, 0.315)};
  \addplot[orange, fill=orange] coordinates {(0, 0.46875) (10, 0.321) (20, 0.366) (30, 0.24) (40, 0.407) (50, 0.454) (60, 0.574) (70, 0.526)};

%  \legend{Δ=1,Δ=2,Δ=3}
  \end{axis}
\end{tikzpicture}}\hspace{0.5cm}
\resizebox{.29\textwidth}{!}{\begin{tikzpicture}
  \begin{axis}[
    xlabel={Snippet length, $|\sigma|$},
    ylabel={Precision@1},
    title={\textbf{Seq2Parse Repair Precision@1}},
    ybar,
    axis lines*=left,
    xtick={0, 10, 20, 30, 40, 50, 60, 70},
    ytick={0, 0.1, 0.2, 0.3, 0.4, 0.5, 0.6, 0.7, 0.8, 0.9, 1.0},
    xticklabels={{(}0{,}10{)}, {[}10{,}20{)}, {[}20{,}30{)}, {[}30{,}40{)}, {[}40{,}50{)}, {[}50{,}60{)}, {[}60{,}70{)}, {[}70{,}80{)}},
    x tick label style={font=\scriptsize},
    ymax=1.0,
    ymin=0.0,
    bar width=4pt,
  ]

  \addplot[green, fill=green!50] coordinates {(0, 0.352631) (10, 0.413115) (20, 0.400502) (30, 0.378440) (40, 0.308869) (50, 0.287755) (60, 0.268817) (70, 0.210526)};
  \addplot[blue, fill=blue!50] coordinates {(0, 0.122529) (10, 0.126453) (20, 0.144192) (30, 0.118483) (40, 0.108007) (50, 0.106849) (60, 0.097403) (70, 0.122047)};
  \addplot[orange, fill=orange!50] coordinates {(0, 0.03125) (10, 0.070922) (20, 0.077348) (30, 0.087629) (40, 0.094675) (50, 0.02) (60, 0.066038) (70, 0.063291)};

  \end{axis}
\end{tikzpicture}}\hspace{0.5cm}
\resizebox{.29\textwidth}{!}{\begin{tikzpicture}
  \begin{axis}[
    legend cell align={left},
    legend style={fill opacity=0.8, draw opacity=1, text opacity=1, draw=lightgray204, legend columns=-1, legend pos=north east},
    xlabel={Snippet length, $|\sigma|$},
    ylabel={Precision@1},
    title={\Large\textbf{BIFI Repair Precision@1}},
    ybar,
    axis lines*=left,
    xtick={0, 10, 20, 30, 40, 50, 60, 70},
    ytick={0, 0.1, 0.2, 0.3, 0.4, 0.5, 0.6, 0.7, 0.8, 0.9, 1.0},
    xticklabels={{(}0{,}10{)}, {[}10{,}20{)}, {[}20{,}30{)}, {[}30{,}40{)}, {[}40{,}50{)}, {[}50{,}60{)}, {[}60{,}70{)}, {[}70{,}80{)}},
    x tick label style={font=\scriptsize},
    ymax=0.6,
    ymin=0.0,
    bar width=4pt,
  ]
    \addlegendimage{empty legend}
    \addlegendentry{$\Delta(\err\sigma, \sigma')=$}
    \addlegendimage{ybar,ybar legend,draw=none,green,fill=green!50}
    \addlegendentry{1,}
    \addlegendimage{ybar,ybar legend,draw=none,blue,fill=blue!50}
    \addlegendentry{2,}
    \addlegendimage{ybar,ybar legend,draw=none,orange,fill=orange!50}
    \addlegendentry{3}

    \addplot[green, fill=green!50] coordinates {(0, 0.196013) (10, 0.326401) (20, 0.318538) (30, 0.272843) (40, 0.213894) (50, 0.206651) (60, 0.247525) (70, 0.179245)};
    \addplot[blue, fill=blue!50] coordinates {(0, 0.174603) (10, 0.176651) (20, 0.209573) (30, 0.19195) (40, 0.18851) (50, 0.176166) (60, 0.110787) (70, 0.106383)};
    \addplot[orange, fill=orange!50] coordinates {(0, 0.015873) (10, 0.021858) (20, 0.030435) (30, 0.02439) (40, 0.032922) (50, 0.045) (60, 0.027397) (70, 0.017094)};
  \end{axis}
\end{tikzpicture}}
%\resizebox{.24\textwidth}{!}{\begin{tikzpicture}
  \begin{axis}[
  legend cell align={left},
  legend style={fill opacity=0.8, draw opacity=1, text opacity=1, draw=lightgray204, legend columns=-1, legend pos=north east},
  xlabel={Snippet length, $|\sigma|$},
  ylabel={Precision@10},
  title={\Large\textbf{Tidyparse Repair Precision@10}},
  ybar,
  axis lines*=left,
  xtick={0, 10, 20, 30, 40, 50, 60, 70},
  ytick={0, 0.1, 0.2, 0.3, 0.4, 0.5, 0.6, 0.7, 0.8, 0.9, 1.0},
  xticklabels={{(}0{,}10{)}, {[}10{,}20{)}, {[}20{,}30{)}, {[}30{,}40{)}, {[}40{,}50{)}, {[}50{,}60{)}, {[}60{,}70{)}, {[}70{,}80{)}},
  x tick label style={font=\scriptsize},
  ymax=1.0,
  ymin=0.0,
  bar width=4pt,
  ymajorgrids=true,
  grid style=dashed,
  tick align=outside,
  tick pos=left,
  ]
  \addlegendimage{empty legend}
  \addlegendentry{$\Delta(\err\sigma, \sigma')=$}
  \addlegendimage{ybar,ybar legend,draw=none,green,fill=green!50}
  \addlegendentry{1,}
  \addlegendimage{ybar,ybar legend,draw=none,blue,fill=blue!50}
  \addlegendentry{2,}
  \addlegendimage{ybar,ybar legend,draw=none,orange,fill=orange!50}
  \addlegendentry{3}
  \addplot[green, fill=green!50] coordinates { (0, 0.8421052631578947) (10, 0.8888888888888888) (20, 0.8622754491017964) (30, 0.8320610687022901) (40, 0.7777777777777778) (50, 0.7868852459016393) (60, 0.7659574468085106) (70, 0.7843137254901961) };
  \addplot[blue, fill=blue!50] coordinates { (0, 0.9354838709677419) (10, 0.8208955223880597) (20, 0.7272727272727273) (30, 0.6617647058823529) (40, 0.6) (50, 0.5961538461538461) (60, 0.7) (70, 0.56) };
  \addplot[orange, fill=orange!50] coordinates { (0, 0.3275862068965517) (10, 0.2411764705882353) (20, 0.23563218390804597) (30, 0.2360248447204969) (40, 0.2932330827067669) (50, 0.3548387096774194) (60, 0.32727272727272727) (70, 0.24358974358974358) };
  \end{axis}
\end{tikzpicture}}
%\resizebox{.19\textwidth}{!}{\begin{tikzpicture}
  \begin{axis}[
  xlabel={Snippet length, $|\sigma|$},
  ylabel={Precision@20k},
  title={\textbf{BIFI Repair Precision@20k}},
  legend cell align={left},
  legend style={fill opacity=0.8, draw opacity=1, text opacity=1, draw=lightgray204, legend columns=-1, legend pos=north east},
  ybar,
  axis lines*=left,
  xtick={0, 10, 20, 30, 40, 50, 60, 70},
  ytick={0, 0.1, 0.2, 0.3, 0.4, 0.5, 0.6, 0.7, 0.8, 0.9, 1.0},
  xticklabels={{(}0{,}10{)}, {[}10{,}20{)}, {[}20{,}30{)}, {[}30{,}40{)}, {[}40{,}50{)}, {[}50{,}60{)}, {[}60{,}70{)}, {[}70{,}80{)}},
  x tick label style={font=\scriptsize},
  ymax=1.0,
  ymin=0.0,
  bar width=4pt,
  ]

  \addlegendimage{empty legend}
  \addlegendentry{$\Delta(\err\sigma, \sigma')=$}
  \addlegendimage{ybar,ybar legend,draw=none,green,fill=green!50}
  \addlegendentry{1,}
  \addlegendimage{ybar,ybar legend,draw=none,blue,fill=blue!50}
  \addlegendentry{2,}
  \addlegendimage{ybar,ybar legend,draw=none,orange,fill=orange!50}
  \addlegendentry{3}

  \addplot[green, fill=green!50] coordinates   {(0, 0.65) (10, 0.67) (20, 0.71) (30, 0.63) (40, 0.60) (50, 0.62) (60, 0.59) (70, 0.64)};
  \addplot[blue, fill=blue!50] coordinates     {(0, 0.52) (10, 0.41) (20, 0.35) (30, 0.31) (40, 0.27) (50, 0.27) (60, 0.21) (70, 0.22)};
  \addplot[orange, fill=orange!50] coordinates {(0, 0.25) (10, 0.08) (20, 0.08) (30, 0.17) (40, 0.11) (50, 0.17) (60, 0.08) (70, 0.08)};

  \end{axis}
\end{tikzpicture}
}
\caption{Probability of the first recommendation matching the true repair for Tidyparse, Seq2Parse and BIFI repair precision at various lengths and Levenshtein distances.}\label{fig:len_dist_prec}
\end{figure}\vspace{-0.2cm}

Tidyparse attains state-of-the-art top-1 repair precision versus both models by a wide margin. Unexpectedly, precision does not monotonically decrease with edit distance, as Tidyparse's double-edit Precision@1 slightly outperforms single-edit Precision@1 across the test set. Although Seq2Parse outperforms BIFI by a lower margin, results are also mixed for the 2-edit repair case. Similar to Fig.~\ref{fig:volumetric_plot}, the nonlinear correlation between edit distance and repair precision holds across all three models, as does a slightly negative correlation between repair length and precision.

\begin{wrapfigure}{r}{0.50\textwidth}
\vspace{-0.1cm}
\resizebox{.24\textwidth}{!}{\begin{tikzpicture}
  \begin{axis}[
  legend cell align={left},
  legend style={fill opacity=0.8, draw opacity=1, text opacity=1, draw=lightgray204, legend columns=-1, legend pos=north east},
  xlabel={Snippet length, $|\sigma|$},
  ylabel={Precision@10},
  title={\Large\textbf{Tidyparse Repair Precision@10}},
  ybar,
  axis lines*=left,
  xtick={0, 10, 20, 30, 40, 50, 60, 70},
  ytick={0, 0.1, 0.2, 0.3, 0.4, 0.5, 0.6, 0.7, 0.8, 0.9, 1.0},
  xticklabels={{(}0{,}10{)}, {[}10{,}20{)}, {[}20{,}30{)}, {[}30{,}40{)}, {[}40{,}50{)}, {[}50{,}60{)}, {[}60{,}70{)}, {[}70{,}80{)}},
  x tick label style={font=\scriptsize},
  ymax=1.0,
  ymin=0.0,
  bar width=4pt,
  ymajorgrids=true,
  grid style=dashed,
  tick align=outside,
  tick pos=left,
  ]
  \addlegendimage{empty legend}
  \addlegendentry{$\Delta(\err\sigma, \sigma')=$}
  \addlegendimage{ybar,ybar legend,draw=none,green,fill=green!50}
  \addlegendentry{1,}
  \addlegendimage{ybar,ybar legend,draw=none,blue,fill=blue!50}
  \addlegendentry{2,}
  \addlegendimage{ybar,ybar legend,draw=none,orange,fill=orange!50}
  \addlegendentry{3}
  \addplot[green, fill=green!50] coordinates { (0, 0.8421052631578947) (10, 0.8888888888888888) (20, 0.8622754491017964) (30, 0.8320610687022901) (40, 0.7777777777777778) (50, 0.7868852459016393) (60, 0.7659574468085106) (70, 0.7843137254901961) };
  \addplot[blue, fill=blue!50] coordinates { (0, 0.9354838709677419) (10, 0.8208955223880597) (20, 0.7272727272727273) (30, 0.6617647058823529) (40, 0.6) (50, 0.5961538461538461) (60, 0.7) (70, 0.56) };
  \addplot[orange, fill=orange!50] coordinates { (0, 0.3275862068965517) (10, 0.2411764705882353) (20, 0.23563218390804597) (30, 0.2360248447204969) (40, 0.2932330827067669) (50, 0.3548387096774194) (60, 0.32727272727272727) (70, 0.24358974358974358) };
  \end{axis}
\end{tikzpicture}}
\resizebox{.24\textwidth}{!}{\begin{tikzpicture}
  \begin{axis}[
  xlabel={Snippet length, $|\sigma|$},
  ylabel={Precision@20k},
  title={\textbf{BIFI Repair Precision@20k}},
  legend cell align={left},
  legend style={fill opacity=0.8, draw opacity=1, text opacity=1, draw=lightgray204, legend columns=-1, legend pos=north east},
  ybar,
  axis lines*=left,
  xtick={0, 10, 20, 30, 40, 50, 60, 70},
  ytick={0, 0.1, 0.2, 0.3, 0.4, 0.5, 0.6, 0.7, 0.8, 0.9, 1.0},
  xticklabels={{(}0{,}10{)}, {[}10{,}20{)}, {[}20{,}30{)}, {[}30{,}40{)}, {[}40{,}50{)}, {[}50{,}60{)}, {[}60{,}70{)}, {[}70{,}80{)}},
  x tick label style={font=\scriptsize},
  ymax=1.0,
  ymin=0.0,
  bar width=4pt,
  ]

  \addlegendimage{empty legend}
  \addlegendentry{$\Delta(\err\sigma, \sigma')=$}
  \addlegendimage{ybar,ybar legend,draw=none,green,fill=green!50}
  \addlegendentry{1,}
  \addlegendimage{ybar,ybar legend,draw=none,blue,fill=blue!50}
  \addlegendentry{2,}
  \addlegendimage{ybar,ybar legend,draw=none,orange,fill=orange!50}
  \addlegendentry{3}

  \addplot[green, fill=green!50] coordinates   {(0, 0.65) (10, 0.67) (20, 0.71) (30, 0.63) (40, 0.60) (50, 0.62) (60, 0.59) (70, 0.64)};
  \addplot[blue, fill=blue!50] coordinates     {(0, 0.52) (10, 0.41) (20, 0.35) (30, 0.31) (40, 0.27) (50, 0.27) (60, 0.21) (70, 0.22)};
  \addplot[orange, fill=orange!50] coordinates {(0, 0.25) (10, 0.08) (20, 0.08) (30, 0.17) (40, 0.11) (50, 0.17) (60, 0.08) (70, 0.08)};

  \end{axis}
\end{tikzpicture}
}
\caption{Probability of the true repair being in the first ten Tidyparse repairs, and the first $2\times10^4$ BIFI repairs.}\label{fig:len_dist_prec_all}
\vspace{-0.3cm}
\end{wrapfigure}

For the next experiment, we evaluate the BIFI model, giving it a generous compute and latency advantage with an unlimited time budget to sample $2\times10^4$ repairs, and compare the Precision@10 of our approach with a 10s timeout. As Tidyparse uses a 4-gram model for decoding, it can sample a much larger candidate set in the time allotted, but must use a transformer-based reranker after decoding to sort the top-$10^3$ repairs. Since the Seq2Parse reference implementation does not support sampling more than one repair, we do not compare its Precision@k for higher k values. The raw data from these experiments can be found in Appendix~\ref{sec:raw_prec_data}.

\begin{wrapfigure}{l}{0.38\textwidth}
\vspace{-1.13cm}
\begin{center}\resizebox{.43\textwidth}{!}{\hspace{-0.6cm}% This file was created with matplot2tikz v0.3.3.
\begin{tikzpicture}

\definecolor{darkgray176}{RGB}{176,176,176}
\definecolor{teal9147104}{RGB}{9,147,104}

\begin{axis}[
hide x axis,
hide y axis,
tick align=outside,
tick pos=left,
x grid style={darkgray176},
xmin=-0.65, xmax=2.10714285714286,
xtick style={color=black},
y grid style={darkgray176},
ymin=-1.22685803692431, ymax=1.22218305326489,
ytick style={color=black}
]
\path [draw=teal9147104, fill=teal9147104]
(axis cs:-0.25,0.315857142857143)
--(axis cs:-0.125,0.315857142857143)
.. controls (axis cs:-0.03125,0.315857142857143) and (axis cs:0.03125,0.315857142857143) .. (axis cs:0.125,0.315857142857143)
--(axis cs:0.25,0.315857142857143)
--(axis cs:0.4,0.315857142857143)
.. controls (axis cs:0.4265114773,0.315857142857143) and (axis cs:0.4519642327,0.326400019357143) .. (axis cs:0.4707106781,0.345146464757143)
.. controls (axis cs:0.4894571235,0.363892910157143) and (axis cs:0.5,0.389345665557143) .. (axis cs:0.5,0.415857142857143)
--(axis cs:0.5,0.565857142857143)
--(axis cs:0.47,0.565857142857143)
--(axis cs:0.596714285714286,0.672183053264892)
--(axis cs:0.723428571428572,0.565857142857143)
--(axis cs:0.693428571428572,0.565857142857143)
--(axis cs:0.693428571428572,0.415857142857143)
.. controls (axis cs:0.693428571428572,0.338064893751143) and (axis cs:0.662492759527143,0.263379237191714) .. (axis cs:0.607485332596286,0.208371810260857)
.. controls (axis cs:0.552477905665429,0.15336438333) and (axis cs:0.477792249106,0.122428571428571) .. (axis cs:0.4,0.122428571428571)
--(axis cs:0.693428571428572,0.122428571428571)
--(axis cs:0.843428571428572,0.122428571428571)
.. controls (axis cs:0.869940048728572,0.122428571428571) and (axis cs:0.895392804128571,0.132971447928571) .. (axis cs:0.914139249528572,0.151717893328571)
.. controls (axis cs:0.932885694928572,0.170464338728571) and (axis cs:0.943428571428572,0.195917094128571) .. (axis cs:0.943428571428572,0.222428571428571)
--(axis cs:0.943428571428572,0.565857142857143)
--(axis cs:0.913428571428571,0.565857142857143)
--(axis cs:0.992571428571429,0.63226588509603)
--(axis cs:1.07171428571429,0.565857142857143)
--(axis cs:1.04171428571429,0.565857142857143)
--(axis cs:1.04171428571429,0.222428571428571)
.. controls (axis cs:1.04171428571429,0.169860099296571) and (axis cs:1.02080926774,0.119390921446286) .. (axis cs:0.983637744575429,0.0822193982817143)
.. controls (axis cs:0.946466221410857,0.0450478751171429) and (axis cs:0.895997043560572,0.0241428571428572) .. (axis cs:0.843428571428572,0.0241428571428572)
--(axis cs:1.04171428571429,0.0241428571428571)
--(axis cs:1.19171428571429,0.0241428571428571)
.. controls (axis cs:1.21822576301429,0.0241428571428571) and (axis cs:1.24367851841429,0.0346857336428571) .. (axis cs:1.26242496381429,0.0534321790428571)
.. controls (axis cs:1.28117140921429,0.0721786244428572) and (axis cs:1.29171428571429,0.0976313798428571) .. (axis cs:1.29171428571429,0.124142857142857)
--(axis cs:1.29171428571429,0.565857142857143)
--(axis cs:1.26171428571429,0.565857142857143)
--(axis cs:1.30557142857143,0.602657655253061)
--(axis cs:1.34942857142857,0.565857142857143)
--(axis cs:1.31942857142857,0.565857142857143)
--(axis cs:1.31942857142857,0.124142857142857)
.. controls (axis cs:1.31942857142857,0.0902839132768571) and (axis cs:1.30596381201286,0.0577771085231429) .. (axis cs:1.28202192317343,0.0338352196837143)
.. controls (axis cs:1.258080034334,0.00989333084428572) and (axis cs:1.22557322958029,-0.00357142857142857) .. (axis cs:1.19171428571429,-0.00357142857142857)
--(axis cs:1.31942857142857,-0.00357142857142857)
--(axis cs:1.46942857142857,-0.00357142857142857)
.. controls (axis cs:1.49594004872857,-0.00357142857142857) and (axis cs:1.52139280412857,0.00697144792857142) .. (axis cs:1.54013924952857,0.0257178933285714)
.. controls (axis cs:1.55888569492857,0.0444643387285714) and (axis cs:1.56942857142857,0.0699170941285714) .. (axis cs:1.56942857142857,0.0964285714285714)
--(axis cs:1.56942857142857,0.565857142857143)
--(axis cs:1.53942857142857,0.565857142857143)
--(axis cs:1.62328571428571,0.636221640500152)
--(axis cs:1.70714285714286,0.565857142857143)
--(axis cs:1.67714285714286,0.565857142857143)
--(axis cs:1.67714285714286,0.0964285714285714)
.. controls (axis cs:1.67714285714286,0.0413604457225714) and (axis cs:1.65524379652714,-0.0115085633511429) .. (axis cs:1.61630475136771,-0.0504476085105714)
.. controls (axis cs:1.57736570620829,-0.08938665367) and (axis cs:1.52449669713457,-0.111285714285714) .. (axis cs:1.46942857142857,-0.111285714285714)
--(axis cs:0.4,-0.111285714285714)
.. controls (axis cs:0.480746385148,-0.111285714285714) and (axis cs:0.558268205880571,-0.143396303854286) .. (axis cs:0.615364522441714,-0.200492620415429)
.. controls (axis cs:0.672460839002857,-0.257588936976571) and (axis cs:0.704571428571429,-0.335110757709143) .. (axis cs:0.704571428571429,-0.415857142857143)
--(axis cs:0.704571428571429,-0.565857142857143)
--(axis cs:0.734571428571429,-0.565857142857143)
--(axis cs:0.602285714285714,-0.676858036924309)
--(axis cs:0.47,-0.565857142857143)
--(axis cs:0.5,-0.565857142857143)
--(axis cs:0.5,-0.415857142857143)
.. controls (axis cs:0.5,-0.389345665557143) and (axis cs:0.4894571235,-0.363892910157143) .. (axis cs:0.4707106781,-0.345146464757143)
.. controls (axis cs:0.4519642327,-0.326400019357143) and (axis cs:0.4265114773,-0.315857142857143) .. (axis cs:0.4,-0.315857142857143)
--(axis cs:0.25,-0.315857142857143)
--(axis cs:0.25,-0.315857142857143)
--(axis cs:0.125,-0.315857142857143)
.. controls (axis cs:0.03125,-0.315857142857143) and (axis cs:-0.03125,-0.315857142857143) .. (axis cs:-0.125,-0.315857142857143)
--(axis cs:-0.25,-0.315857142857143)
--(axis cs:-0.25,-0.315857142857143)
--(axis cs:0.015035612076138,0)
--(axis cs:-0.25,0.315857142857143)
--(axis cs:0,0.315857142857143)
--(axis cs:-0.25,0.315857142857143)
--cycle;
\draw (axis cs:-0.134964387923862,0) node[
  scale=0.5,
  text=black,
  rotate=0.0,
  align=center
]{Total\\
2211};
\draw (axis cs:0.596714285714286,0.822183053264892) node[
  scale=0.5,
  text=black,
  rotate=0.0,
  align=center
]{Top-1\\677};
\draw (axis cs:0.992571428571429,0.78226588509603) node[
  scale=0.5,
  text=black,
  rotate=0.0,
  align=center
]{[2-10]\\344};
\draw (axis cs:1.30557142857143,0.752657655253061) node[
  scale=0.5,
  text=black,
  rotate=0.0,
  align=center
]{[11-99]\\97};
\draw (axis cs:1.62328571428571,0.786221640500152) node[
  scale=0.5,
  text=black,
  rotate=0.0,
  align=center
]{Top-100+\\377};
\draw (axis cs:0.602285714285714,-0.826858036924309) node[
  scale=0.5,
  text=black,
  rotate=0.0,
  align=center
]{NR\\716};
\end{axis}

\end{tikzpicture}
}\end{center}
\vspace{-1.1cm}
\caption{Outcomes in the repair pipeline.}
\label{fig:sankey}
\end{wrapfigure}

\noindent We present a Sankey diagram of the Tidyparse repair pipeline in Fig.~\ref{fig:sankey}. Across 2,238 test set repairs filtered by length and distance ($\lfloor|\err\sigma| / 10\rfloor \in [0, 8], \Delta(\err\sigma, \sigma') < 4$), we evaluated Tidyparse with a timeout of 10s and tracked individual repair outcomes. In 607 cases, the true repair was not contained in the language intersection and thus never sampled, in 1,631 cases the human repair was sampled, of which 675 cases the first prediction matched the human repair, in 1,242 cases, the true repair was in the top-10 results, and in the remaining 389 cases the true repair was drawn, but ranked lower than 10\textsuperscript{th} in the final results.

\clearpage\subsection{Internal evaluation}\label{sec:rq3}

The primary question of interest here is, to what extent does the neural reranker improve precision relative to a na\"ive decoding strategy? For comparison, we use an 4-gram based repair sans reranking. That is, we decode the language intersection with a 4-gram model, sort the repairs by their respective 4-gram probabilities, and without further processing, evaluate Precision@$10^{\{0, 1, 2, 3\}}$.\vspace{-0.2cm}
\begin{figure}[H]
\resizebox{.24\textwidth}{!}{\begin{tikzpicture}
  \begin{axis}[
  xlabel={Snippet length, $|\sigma|$},
  ylabel={Precision@1},
  title={\Large\textbf{4-gram Repair Precision@1}},
  ybar,
  axis lines*=left,
  xtick={0, 10, 20, 30, 40, 50, 60, 70},
  ytick={0, 0.1, 0.2, 0.3, 0.4, 0.5, 0.6, 0.7, 0.8, 0.9, 1.0},
  xticklabels={{(}0{,}10{)}, {[}10{,}20{)}, {[}20{,}30{)}, {[}30{,}40{)}, {[}40{,}50{)}, {[}50{,}60{)}, {[}60{,}70{)}, {[}70{,}80{)}},
  x tick label style={font=\scriptsize},
  ymax=1.0,
  ymin=0.0,
  bar width=4pt,
  ]
  \addplot[green, fill=green!50] coordinates { (0, 0.03289473684210526) (10, 0.033444816053511704) (20, 0.010344827586206896) (30, 0.013793103448275862) (40, 0.010380622837370242) (50, 0.010416666666666666) (60, 0.008333333333333333) (70, 0.0) };
  \addplot[blue, fill=blue!50] coordinates { (0, 0.18095238095238095) (10, 0.06397306397306397) (20, 0.058823529411764705) (30, 0.05190311418685121) (40, 0.05821917808219178) (50, 0.04455445544554455) (60, 0.027210884353741496) (70, 0.03636363636363636) };
  \addplot[orange, fill=orange!50] coordinates { (0, 0.07407407407407407) (10, 0.08904109589041095) (20, 0.08666666666666667) (30, 0.09090909090909091) (40, 0.07894736842105263) (50, 0.10833333333333334) (60, 0.06315789473684211) (70, 0.06349206349206349) };
  \end{axis}
\end{tikzpicture}}
\resizebox{.24\textwidth}{!}{\begin{tikzpicture}
  \begin{axis}[
  xlabel={Snippet length, $|\sigma|$},
  ylabel={Precision@10},
  title={\Large\textbf{4-gram Repair Precision@10}},
  ybar,
  axis lines*=left,
  xtick={0, 10, 20, 30, 40, 50, 60, 70},
  ytick={0, 0.1, 0.2, 0.3, 0.4, 0.5, 0.6, 0.7, 0.8, 0.9, 1.0},
  xticklabels={{(}0{,}10{)}, {[}10{,}20{)}, {[}20{,}30{)}, {[}30{,}40{)}, {[}40{,}50{)}, {[}50{,}60{)}, {[}60{,}70{)}, {[}70{,}80{)}},
  x tick label style={font=\scriptsize},
  ymax=1.0,
  ymin=0.0,
  bar width=4pt,
  ymajorgrids=true,
  grid style=dashed,
  tick align=outside,
  tick pos=left,
  ]
  \addplot[green, fill=green!50] coordinates { (0, 0.34210526315789475) (10, 0.28762541806020064) (20, 0.2150170648464164) (30, 0.27586206896551724) (40, 0.23529411764705882) (50, 0.2638888888888889) (60, 0.1875) (70, 0.23295454545454544) };
  \addplot[blue, fill=blue!50] coordinates { (0, 0.3904761904761905) (10, 0.27759197324414714) (20, 0.2560553633217993) (30, 0.2179930795847751) (40, 0.17465753424657535) (50, 0.18811881188118812) (60, 0.1292517006802721) (70, 0.1) };
  \addplot[orange, fill=orange!50] coordinates { (0, 0.18518518518518517) (10, 0.11643835616438356) (20, 0.14) (30, 0.12337662337662338) (40, 0.15789473684210525) (50, 0.175) (60, 0.15789473684210525) (70, 0.1111111111111111) };
  \end{axis}
\end{tikzpicture}}
\resizebox{.24\textwidth}{!}{\begin{tikzpicture}
  \begin{axis}[
  xlabel={Snippet length, $|\sigma|$},
  ylabel={Precision@100},
  title={\Large\textbf{4-gram Repair Precision@100}},
  ybar,
  axis lines*=left,
  xtick={0, 10, 20, 30, 40, 50, 60, 70},
  ytick={0, 0.1, 0.2, 0.3, 0.4, 0.5, 0.6, 0.7, 0.8, 0.9, 1.0},
  xticklabels={{(}0{,}10{)}, {[}10{,}20{)}, {[}20{,}30{)}, {[}30{,}40{)}, {[}40{,}50{)}, {[}50{,}60{)}, {[}60{,}70{)}, {[}70{,}80{)}},
  x tick label style={font=\scriptsize},
  ymax=1.0,
  ymin=0.0,
  bar width=4pt,
  ]
  \addplot[green, fill=green!50] coordinates { (0, 0.7105263157894737) (10, 0.6488294314381271) (20, 0.5324232081911263) (30, 0.593103448275862) (40, 0.5467128027681661) (50, 0.5694444444444444) (60, 0.6041666666666666) (70, 0.6079545454545454) };
  \addplot[blue, fill=blue!50] coordinates { (0, 0.7047619047619048) (10, 0.5551839464882943) (20, 0.5155709342560554) (30, 0.4602076124567474) (40, 0.4280821917808219) (50, 0.4306930693069307) (60, 0.3197278911564626) (70, 0.3090909090909091) };
  \addplot[orange, fill=orange!50] coordinates { (0, 0.2222222222222222) (10, 0.19863013698630136) (20, 0.19333333333333333) (30, 0.22077922077922077) (40, 0.23684210526315788) (50, 0.31666666666666665) (60, 0.24210526315789474) (70, 0.19047619047619047) };
  \end{axis}
\end{tikzpicture}}
\resizebox{.24\textwidth}{!}{\begin{tikzpicture}
  \begin{axis}[
  xlabel={Snippet length, $|\sigma|$},
  ylabel={Precision@1000},
  title={\Large\textbf{4-gram Repair Precision@1000}},
  ybar,
  axis lines*=left,
  xtick={0, 10, 20, 30, 40, 50, 60, 70},
  ytick={0, 0.1, 0.2, 0.3, 0.4, 0.5, 0.6, 0.7, 0.8, 0.9, 1.0},
  xticklabels={{(}0{,}10{)}, {[}10{,}20{)}, {[}20{,}30{)}, {[}30{,}40{)}, {[}40{,}50{)}, {[}50{,}60{)}, {[}60{,}70{)}, {[}70{,}80{)}},
  x tick label style={font=\scriptsize},
  ymax=1.0,
  ymin=0.0,
  bar width=4pt,
  ]
  \addplot[green, fill=green!50] coordinates { (0, 0.9276315789473685) (10, 0.9565217391304348) (20, 0.9010238907849829) (30, 0.8724137931034482) (40, 0.8961937716262975) (50, 0.8472222222222222) (60, 0.8875) (70, 0.8977272727272727) };
  \addplot[blue, fill=blue!50] coordinates { (0, 0.9523809523809523) (10, 0.8896321070234113) (20, 0.7889273356401384) (30, 0.754325259515571) (40, 0.702054794520548) (50, 0.7128712871287128) (60, 0.6870748299319728) (70, 0.6636363636363637) };
  \addplot[orange, fill=orange!50] coordinates { (0, 0.37037037037037035) (10, 0.2876712328767123) (20, 0.26666666666666666) (30, 0.33116883116883117) (40, 0.2982456140350877) (50, 0.38333333333333336) (60, 0.3157894736842105) (70, 0.30158730158730157) };
  \end{axis}
\end{tikzpicture}}
\caption{4-gram repairs. 4-gram Precision@1000 is an upper bound on Tidyparse Precision@k, since the latter only reranks the top-$10^3$ most probable 4-gram sampled repairs from the language intersection.}\label{fig:adaptive}
\end{figure}\vspace{-0.2cm}

\begin{wrapfigure}{r}{0.45\textwidth}
\vspace{-0.35cm}
%\begin{tikzpicture}[scale=0.9]
  \begin{axis}[
  ybar,
  xlabel={Edit distance, $\Delta(\err\sigma, \sigma)$},
  ylabel={Precision@1},
  title={Enumeration + PCFG (Alg. 1)},
  axis x line*=bottom,
  axis y line*=left,
  ymin=0,
  ymax=1,
  xtick=data,
  width=5cm,
  height=4cm,
  xticklabels={1, 2, 3, 4},
  enlarge x limits=0.15,
  legend style={at={(0.5,-0.15)},
  anchor=north,legend columns=-1},
  ]
  \addplot[fill=black!30] table[x=Lev, y=P@1] {
    Lev P@1
    1 0.41
    2 0.13
    3 0.02
    4 0.00
  };
  \end{axis}
\end{tikzpicture}
%\begin{tikzpicture}[scale=0.9]
  \begin{axis}[
  ybar,
  xlabel={Edit distance, $\Delta(\err\sigma, \sigma)$},
  ylabel={Precision@1},
  title={Enumeration + Markov (Alg. 2)},
  axis x line*=bottom,
  axis y line*=left,
  ymin=0,
  ymax=1,
  xtick=data,
  xticklabels={1, 2, 3, 4},
  width=5cm,
  height=4cm,
  enlarge x limits=0.15,
  legend style={at={(0.5,-0.15)},
  anchor=north,legend columns=-1},
  ]
  \addplot[fill=black!30] table[x=Lev, y=P@1] {
    Lev P@1
    1 0.52
    2 0.25
    3 0.14
    4 0.11
  };
  \end{axis}
\end{tikzpicture}
\resizebox{.45\textwidth}{!}{% This file was created with matplot2tikz v0.3.3.
\begin{tikzpicture}

\definecolor{darkgray176}{RGB}{176,176,176}
\definecolor{darkorange25512714}{RGB}{255,127,14}
\definecolor{lightgray204}{RGB}{204,204,204}
\definecolor{steelblue31119180}{RGB}{31,119,180}

\begin{axis}[
legend cell align={left},
legend style={
  fill opacity=0.8,
  draw opacity=1,
  text opacity=1,
  at={(0.97,0.05)},
  anchor=south east,
  draw=lightgray204
},
axis lines*=left,
log basis x={10},
width=0.8\textwidth,
height=0.3\textheight,
tick align=outside,
tick pos=left,
ymajorgrids=true,
grid style=dashed,
title={{\Large\textbf{Reranking improvement}}},
x grid style={darkgray176},
xlabel={Rank of true document (log scale)},
xmin=1, xmax=2000000,
xmode=log,
xtick style={color=black},
y grid style={darkgray176},
ylabel={Cumulative density},
ymin=0, ymax=1.04996934396076,
ytick style={color=black},
xticklabels={
  $\phantom{^1}1\phantom{^1}$,
  \leq$10\phantom{^2}$,
  \leq$10^2$,
  \leq$10^3$,
  \leq$10^4$,
  \leq$10^5$,
  \leq$10^6$,
},
]
\addlegendentry{Transformer}
\addlegendentry{4-gram \(P(\sigma)\)}
\addlegendentry{1k Threshold}
\addplot [semithick, steelblue31119180]
table {%
0 0.000613120784794605
0 0.00122624156958921
0 0.00183936235438381
0 0.00245248313917842
0 0.00306560392397302
0 0.00367872470876763
0 0.00429184549356223
0 0.00490496627835684
0 0.00551808706315144
0 0.00613120784794605
0 0.00674432863274065
0 0.00735744941753525
0 0.00797057020232986
0 0.00858369098712446
0 0.00919681177191907
0 0.00980993255671367
0 0.0104230533415083
0 0.0110361741263029
0 0.0116492949110975
0 0.0122624156958921
0 0.0128755364806867
0 0.0134886572654813
0 0.0141017780502759
0 0.0147148988350705
0 0.0153280196198651
0 0.0159411404046597
0 0.0165542611894543
0 0.0171673819742489
0 0.0177805027590435
0 0.0183936235438381
0 0.0190067443286327
0 0.0196198651134273
0 0.020232985898222
0 0.0208461066830166
0 0.0214592274678112
0 0.0220723482526058
0 0.0226854690374004
0 0.023298589822195
0 0.0239117106069896
0 0.0245248313917842
0 0.0251379521765788
0 0.0257510729613734
0 0.026364193746168
0 0.0269773145309626
0 0.0275904353157572
0 0.0282035561005518
0 0.0288166768853464
0 0.029429797670141
0 0.0300429184549356
0 0.0306560392397302
0 0.0312691600245248
0 0.0318822808093194
0 0.032495401594114
0 0.0331085223789086
0 0.0337216431637032
0 0.0343347639484979
0 0.0349478847332925
0 0.0355610055180871
0 0.0361741263028817
0 0.0367872470876763
0 0.0374003678724709
0 0.0380134886572655
0 0.0386266094420601
0 0.0392397302268547
0 0.0398528510116493
0 0.0404659717964439
0 0.0410790925812385
0 0.0416922133660331
0 0.0423053341508277
0 0.0429184549356223
0 0.0435315757204169
0 0.0441446965052115
0 0.0447578172900061
0 0.0453709380748007
0 0.0459840588595953
0 0.0465971796443899
0 0.0472103004291846
0 0.0478234212139792
0 0.0484365419987738
0 0.0490496627835684
0 0.049662783568363
0 0.0502759043531576
0 0.0508890251379522
0 0.0515021459227468
0 0.0521152667075414
0 0.052728387492336
0 0.0533415082771306
0 0.0539546290619252
0 0.0545677498467198
0 0.0551808706315144
0 0.055793991416309
0 0.0564071122011036
0 0.0570202329858982
0 0.0576333537706928
0 0.0582464745554874
0 0.058859595340282
0 0.0594727161250766
0 0.0600858369098712
0 0.0606989576946659
0 0.0613120784794605
0 0.0619251992642551
0 0.0625383200490497
0 0.0631514408338443
0 0.0637645616186389
0 0.0643776824034335
0 0.0649908031882281
0 0.0656039239730227
0 0.0662170447578173
0 0.0668301655426119
0 0.0674432863274065
0 0.0680564071122011
0 0.0686695278969957
0 0.0692826486817903
0 0.0698957694665849
0 0.0705088902513795
0 0.0711220110361741
0 0.0717351318209687
0 0.0723482526057633
0 0.0729613733905579
0 0.0735744941753525
0 0.0741876149601472
0 0.0748007357449418
0 0.0754138565297364
0 0.076026977314531
0 0.0766400980993256
0 0.0772532188841202
0 0.0778663396689148
0 0.0784794604537094
0 0.079092581238504
0 0.0797057020232986
0 0.0803188228080932
0 0.0809319435928878
0 0.0815450643776824
0 0.082158185162477
0 0.0827713059472716
0 0.0833844267320662
0 0.0839975475168608
0 0.0846106683016554
0 0.08522378908645
0 0.0858369098712446
0 0.0864500306560392
0 0.0870631514408338
0 0.0876762722256284
0 0.0882893930104231
0 0.0889025137952177
0 0.0895156345800123
0 0.0901287553648069
0 0.0907418761496015
0 0.0913549969343961
0 0.0919681177191907
0 0.0925812385039853
0 0.0931943592887799
0 0.0938074800735745
0 0.0944206008583691
0 0.0950337216431637
0 0.0956468424279583
0 0.0962599632127529
0 0.0968730839975475
0 0.0974862047823421
0 0.0980993255671367
0 0.0987124463519313
0 0.0993255671367259
0 0.0999386879215205
0 0.100551808706315
0 0.10116492949111
0 0.101778050275904
0 0.102391171060699
0 0.103004291845494
0 0.103617412630288
0 0.104230533415083
0 0.104843654199877
0 0.105456774984672
0 0.106069895769467
0 0.106683016554261
0 0.107296137339056
0 0.10790925812385
0 0.108522378908645
0 0.10913549969344
0 0.109748620478234
0 0.110361741263029
0 0.110974862047823
0 0.111587982832618
0 0.112201103617413
0 0.112814224402207
0 0.113427345187002
0 0.114040465971796
0 0.114653586756591
0 0.115266707541386
0 0.11587982832618
0 0.116492949110975
0 0.117106069895769
0 0.117719190680564
0 0.118332311465359
0 0.118945432250153
0 0.119558553034948
0 0.120171673819742
0 0.120784794604537
0 0.121397915389332
0 0.122011036174126
0 0.122624156958921
0 0.123237277743716
0 0.12385039852851
0 0.124463519313305
0 0.125076640098099
0 0.125689760882894
0 0.126302881667689
0 0.126916002452483
0 0.127529123237278
0 0.128142244022072
0 0.128755364806867
0 0.129368485591662
0 0.129981606376456
0 0.130594727161251
0 0.131207847946045
0 0.13182096873084
0 0.132434089515635
0 0.133047210300429
0 0.133660331085224
0 0.134273451870018
0 0.134886572654813
0 0.135499693439608
0 0.136112814224402
0 0.136725935009197
0 0.137339055793991
0 0.137952176578786
0 0.138565297363581
0 0.139178418148375
0 0.13979153893317
0 0.140404659717964
0 0.141017780502759
0 0.141630901287554
0 0.142244022072348
0 0.142857142857143
0 0.143470263641937
0 0.144083384426732
0 0.144696505211527
0 0.145309625996321
0 0.145922746781116
0 0.14653586756591
0 0.147148988350705
0 0.1477621091355
0 0.148375229920294
0 0.148988350705089
0 0.149601471489884
0 0.150214592274678
0 0.150827713059473
0 0.151440833844267
0 0.152053954629062
0 0.152667075413857
0 0.153280196198651
0 0.153893316983446
0 0.15450643776824
0 0.155119558553035
0 0.15573267933783
0 0.156345800122624
0 0.156958920907419
0 0.157572041692213
0 0.158185162477008
0 0.158798283261803
0 0.159411404046597
0 0.160024524831392
0 0.160637645616186
0 0.161250766400981
0 0.161863887185776
0 0.16247700797057
0 0.163090128755365
0 0.163703249540159
0 0.164316370324954
0 0.164929491109749
0 0.165542611894543
0 0.166155732679338
0 0.166768853464132
0 0.167381974248927
0 0.167995095033722
0 0.168608215818516
0 0.169221336603311
0 0.169834457388105
0 0.1704475781729
0 0.171060698957695
0 0.171673819742489
0 0.172286940527284
0 0.172900061312078
0 0.173513182096873
0 0.174126302881668
0 0.174739423666462
0 0.175352544451257
0 0.175965665236052
0 0.176578786020846
0 0.177191906805641
0 0.177805027590435
0 0.17841814837523
0 0.179031269160025
0 0.179644389944819
0 0.180257510729614
0 0.180870631514408
0 0.181483752299203
0 0.182096873083998
0 0.182709993868792
0 0.183323114653587
0 0.183936235438381
0 0.184549356223176
0 0.185162477007971
0 0.185775597792765
0 0.18638871857756
0 0.187001839362354
0 0.187614960147149
0 0.188228080931944
0 0.188841201716738
0 0.189454322501533
0 0.190067443286327
0 0.190680564071122
0 0.191293684855917
0 0.191906805640711
0 0.192519926425506
0 0.1931330472103
0 0.193746167995095
0 0.19435928877989
0 0.194972409564684
0 0.195585530349479
0 0.196198651134273
0 0.196811771919068
0 0.197424892703863
0 0.198038013488657
0 0.198651134273452
0 0.199264255058246
0 0.199877375843041
0 0.200490496627836
0 0.20110361741263
0 0.201716738197425
0 0.20232985898222
0 0.202942979767014
0 0.203556100551809
0 0.204169221336603
0 0.204782342121398
0 0.205395462906193
0 0.206008583690987
0 0.206621704475782
0 0.207234825260576
0 0.207847946045371
0 0.208461066830166
0 0.20907418761496
0 0.209687308399755
0 0.210300429184549
0 0.210913549969344
0 0.211526670754139
0 0.212139791538933
0 0.212752912323728
0 0.213366033108522
0 0.213979153893317
0 0.214592274678112
0 0.215205395462906
0 0.215818516247701
0 0.216431637032495
0 0.21704475781729
0 0.217657878602085
0 0.218270999386879
0 0.218884120171674
0 0.219497240956468
0 0.220110361741263
0 0.220723482526058
0 0.221336603310852
0 0.221949724095647
0 0.222562844880441
0 0.223175965665236
0 0.223789086450031
0 0.224402207234825
0 0.22501532801962
0 0.225628448804414
0 0.226241569589209
0 0.226854690374004
0 0.227467811158798
0 0.228080931943593
0 0.228694052728387
0 0.229307173513182
0 0.229920294297977
0 0.230533415082771
0 0.231146535867566
0 0.231759656652361
0 0.232372777437155
0 0.23298589822195
0 0.233599019006744
0 0.234212139791539
0 0.234825260576334
0 0.235438381361128
0 0.236051502145923
0 0.236664622930717
0 0.237277743715512
0 0.237890864500307
0 0.238503985285101
0 0.239117106069896
0 0.23973022685469
0 0.240343347639485
0 0.24095646842428
0 0.241569589209074
0 0.242182709993869
0 0.242795830778663
0 0.243408951563458
0 0.244022072348253
0 0.244635193133047
0 0.245248313917842
0 0.245861434702636
0 0.246474555487431
0 0.247087676272226
0 0.24770079705702
0 0.248313917841815
0 0.248927038626609
0 0.249540159411404
0 0.250153280196199
0 0.250766400980993
0 0.251379521765788
0 0.251992642550582
0 0.252605763335377
0 0.253218884120172
0 0.253832004904966
0 0.254445125689761
0 0.255058246474555
0 0.25567136725935
0 0.256284488044145
0 0.256897608828939
0 0.257510729613734
0 0.258123850398529
0 0.258736971183323
0 0.259350091968118
0 0.259963212752912
0 0.260576333537707
0 0.261189454322502
0 0.261802575107296
0 0.262415695892091
0 0.263028816676885
0 0.26364193746168
0 0.264255058246475
0 0.264868179031269
0 0.265481299816064
0 0.266094420600858
0 0.266707541385653
0 0.267320662170448
0 0.267933782955242
0 0.268546903740037
0 0.269160024524831
0 0.269773145309626
0 0.270386266094421
0 0.270999386879215
0 0.27161250766401
0 0.272225628448804
0 0.272838749233599
0 0.273451870018394
0 0.274064990803188
0 0.274678111587983
0 0.275291232372777
0 0.275904353157572
0 0.276517473942367
0 0.277130594727161
0 0.277743715511956
0 0.27835683629675
0 0.278969957081545
0 0.27958307786634
0 0.280196198651134
0 0.280809319435929
0 0.281422440220723
0 0.282035561005518
0 0.282648681790313
0 0.283261802575107
0 0.283874923359902
0 0.284488044144697
0 0.285101164929491
0 0.285714285714286
0 0.28632740649908
0 0.286940527283875
0 0.28755364806867
0 0.288166768853464
0 0.288779889638259
0 0.289393010423053
0 0.290006131207848
0 0.290619251992643
0 0.291232372777437
0 0.291845493562232
0 0.292458614347026
0 0.293071735131821
0 0.293684855916616
0 0.29429797670141
0 0.294911097486205
0 0.295524218270999
0 0.296137339055794
0 0.296750459840589
0 0.297363580625383
0 0.297976701410178
0 0.298589822194972
0 0.299202942979767
0 0.299816063764562
0 0.300429184549356
0 0.301042305334151
0 0.301655426118945
0 0.30226854690374
0 0.302881667688535
0 0.303494788473329
0 0.304107909258124
0 0.304721030042918
0 0.305334150827713
0 0.305947271612508
0 0.306560392397302
0 0.307173513182097
0 0.307786633966891
0 0.308399754751686
0 0.309012875536481
0 0.309625996321275
0 0.31023911710607
0 0.310852237890865
0 0.311465358675659
0 0.312078479460454
0 0.312691600245248
0 0.313304721030043
0 0.313917841814838
0 0.314530962599632
0 0.315144083384427
0 0.315757204169221
0 0.316370324954016
0 0.316983445738811
0 0.317596566523605
0 0.3182096873084
0 0.318822808093194
0 0.319435928877989
0 0.320049049662784
0 0.320662170447578
0 0.321275291232373
0 0.321888412017167
0 0.322501532801962
0 0.323114653586757
0 0.323727774371551
0 0.324340895156346
0 0.32495401594114
0 0.325567136725935
0 0.32618025751073
0 0.326793378295524
0 0.327406499080319
0 0.328019619865113
0 0.328632740649908
0 0.329245861434703
0 0.329858982219497
0 0.330472103004292
0 0.331085223789086
0 0.331698344573881
0 0.332311465358676
0 0.33292458614347
0 0.333537706928265
0 0.334150827713059
0 0.334763948497854
0 0.335377069282649
0 0.335990190067443
0 0.336603310852238
0 0.337216431637033
0 0.337829552421827
0 0.338442673206622
0 0.339055793991416
0 0.339668914776211
0 0.340282035561006
0 0.3408951563458
0 0.341508277130595
0 0.342121397915389
0 0.342734518700184
0 0.343347639484979
0 0.343960760269773
0 0.344573881054568
0 0.345187001839362
0 0.345800122624157
0 0.346413243408952
0 0.347026364193746
0 0.347639484978541
0 0.348252605763335
0 0.34886572654813
0 0.349478847332925
0 0.350091968117719
0 0.350705088902514
0 0.351318209687308
0 0.351931330472103
0 0.352544451256898
0 0.353157572041692
0 0.353770692826487
0 0.354383813611281
0 0.354996934396076
0 0.355610055180871
0 0.356223175965665
0 0.35683629675046
0 0.357449417535254
0 0.358062538320049
0 0.358675659104844
0 0.359288779889638
0 0.359901900674433
0 0.360515021459227
0 0.361128142244022
0 0.361741263028817
0 0.362354383813611
0 0.362967504598406
0 0.363580625383201
0 0.364193746167995
0 0.36480686695279
0 0.365419987737584
0 0.366033108522379
0 0.366646229307173
0 0.367259350091968
0 0.367872470876763
0 0.368485591661557
0 0.369098712446352
0 0.369711833231147
0 0.370324954015941
0 0.370938074800736
0 0.37155119558553
0 0.372164316370325
0 0.37277743715512
0 0.373390557939914
0 0.374003678724709
0 0.374616799509503
0 0.375229920294298
0 0.375843041079093
0 0.376456161863887
0 0.377069282648682
0 0.377682403433476
0 0.378295524218271
0 0.378908645003066
0 0.37952176578786
0 0.380134886572655
0 0.380748007357449
0 0.381361128142244
0 0.381974248927039
0 0.382587369711833
0 0.383200490496628
0 0.383813611281422
0 0.384426732066217
0 0.385039852851012
0 0.385652973635806
0 0.386266094420601
0 0.386879215205395
0 0.38749233599019
0 0.388105456774985
0 0.388718577559779
0 0.389331698344574
0 0.389944819129368
0 0.390557939914163
0 0.391171060698958
0 0.391784181483752
0 0.392397302268547
0 0.393010423053341
0 0.393623543838136
0 0.394236664622931
0 0.394849785407725
0 0.39546290619252
0 0.396076026977315
0 0.396689147762109
0 0.397302268546904
0 0.397915389331698
0 0.398528510116493
0 0.399141630901288
0 0.399754751686082
0 0.400367872470877
0 0.400980993255671
0 0.401594114040466
0 0.402207234825261
0 0.402820355610055
0 0.40343347639485
0 0.404046597179644
0 0.404659717964439
0 0.405272838749234
0 0.405885959534028
0 0.406499080318823
0 0.407112201103617
0 0.407725321888412
0 0.408338442673207
0 0.408951563458001
0 0.409564684242796
0 0.41017780502759
0 0.410790925812385
0 0.41140404659718
0 0.412017167381974
0 0.412630288166769
0 0.413243408951563
0 0.413856529736358
1 0.414469650521153
1 0.415082771305947
1 0.415695892090742
1 0.416309012875536
1 0.416922133660331
1 0.417535254445126
1 0.41814837522992
1 0.418761496014715
1 0.419374616799509
1 0.419987737584304
1 0.420600858369099
1 0.421213979153893
1 0.421827099938688
1 0.422440220723483
1 0.423053341508277
1 0.423666462293072
1 0.424279583077866
1 0.424892703862661
1 0.425505824647456
1 0.42611894543225
1 0.426732066217045
1 0.427345187001839
1 0.427958307786634
1 0.428571428571429
1 0.429184549356223
1 0.429797670141018
1 0.430410790925812
1 0.431023911710607
1 0.431637032495402
1 0.432250153280196
1 0.432863274064991
1 0.433476394849785
1 0.43408951563458
1 0.434702636419375
1 0.435315757204169
1 0.435928877988964
1 0.436541998773758
1 0.437155119558553
1 0.437768240343348
1 0.438381361128142
1 0.438994481912937
1 0.439607602697731
1 0.440220723482526
1 0.440833844267321
1 0.441446965052115
1 0.44206008583691
1 0.442673206621704
1 0.443286327406499
1 0.443899448191294
1 0.444512568976088
1 0.445125689760883
1 0.445738810545677
1 0.446351931330472
1 0.446965052115267
1 0.447578172900061
1 0.448191293684856
1 0.448804414469651
1 0.449417535254445
1 0.45003065603924
1 0.450643776824034
1 0.451256897608829
1 0.451870018393624
1 0.452483139178418
1 0.453096259963213
1 0.453709380748007
1 0.454322501532802
1 0.454935622317597
1 0.455548743102391
1 0.456161863887186
1 0.45677498467198
1 0.457388105456775
1 0.45800122624157
1 0.458614347026364
1 0.459227467811159
1 0.459840588595953
1 0.460453709380748
1 0.461066830165543
1 0.461679950950337
1 0.462293071735132
1 0.462906192519926
1 0.463519313304721
1 0.464132434089516
1 0.46474555487431
1 0.465358675659105
1 0.465971796443899
1 0.466584917228694
1 0.467198038013489
1 0.467811158798283
1 0.468424279583078
1 0.469037400367872
1 0.469650521152667
1 0.470263641937462
1 0.470876762722256
1 0.471489883507051
1 0.472103004291845
1 0.47271612507664
1 0.473329245861435
1 0.473942366646229
1 0.474555487431024
1 0.475168608215819
1 0.475781729000613
1 0.476394849785408
1 0.477007970570202
1 0.477621091354997
1 0.478234212139792
1 0.478847332924586
1 0.479460453709381
1 0.480073574494175
1 0.48068669527897
1 0.481299816063765
1 0.481912936848559
1 0.482526057633354
1 0.483139178418148
1 0.483752299202943
1 0.484365419987738
1 0.484978540772532
1 0.485591661557327
1 0.486204782342121
1 0.486817903126916
1 0.487431023911711
1 0.488044144696505
1 0.4886572654813
1 0.489270386266094
1 0.489883507050889
1 0.490496627835684
1 0.491109748620478
1 0.491722869405273
1 0.492335990190067
1 0.492949110974862
1 0.493562231759657
1 0.494175352544451
1 0.494788473329246
1 0.49540159411404
1 0.496014714898835
1 0.49662783568363
1 0.497240956468424
1 0.497854077253219
1 0.498467198038013
1 0.499080318822808
1 0.499693439607603
1 0.500306560392397
1 0.500919681177192
1 0.501532801961986
1 0.502145922746781
1 0.502759043531576
1 0.50337216431637
1 0.503985285101165
1 0.50459840588596
1 0.505211526670754
1 0.505824647455549
1 0.506437768240343
1 0.507050889025138
1 0.507664009809933
1 0.508277130594727
1 0.508890251379522
1 0.509503372164316
1 0.510116492949111
1 0.510729613733906
1 0.5113427345187
1 0.511955855303495
1 0.512568976088289
1 0.513182096873084
1 0.513795217657879
1 0.514408338442673
1 0.515021459227468
1 0.515634580012262
1 0.516247700797057
1 0.516860821581852
1 0.517473942366646
1 0.518087063151441
1 0.518700183936235
1 0.51931330472103
1 0.519926425505825
1 0.520539546290619
1 0.521152667075414
1 0.521765787860208
1 0.522378908645003
1 0.522992029429798
1 0.523605150214592
1 0.524218270999387
1 0.524831391784181
1 0.525444512568976
1 0.526057633353771
1 0.526670754138565
1 0.52728387492336
1 0.527896995708155
1 0.528510116492949
1 0.529123237277744
1 0.529736358062538
1 0.530349478847333
1 0.530962599632128
1 0.531575720416922
1 0.532188841201717
1 0.532801961986511
1 0.533415082771306
1 0.534028203556101
1 0.534641324340895
1 0.53525444512569
1 0.535867565910484
1 0.536480686695279
1 0.537093807480074
1 0.537706928264868
1 0.538320049049663
1 0.538933169834457
1 0.539546290619252
1 0.540159411404047
1 0.540772532188841
1 0.541385652973636
1 0.54199877375843
1 0.542611894543225
1 0.54322501532802
2 0.543838136112814
2 0.544451256897609
2 0.545064377682403
2 0.545677498467198
2 0.546290619251993
2 0.546903740036787
2 0.547516860821582
2 0.548129981606376
2 0.548743102391171
2 0.549356223175966
2 0.54996934396076
2 0.550582464745555
2 0.55119558553035
2 0.551808706315144
2 0.552421827099939
2 0.553034947884733
2 0.553648068669528
2 0.554261189454322
2 0.554874310239117
2 0.555487431023912
2 0.556100551808706
2 0.556713672593501
2 0.557326793378295
2 0.55793991416309
2 0.558553034947885
2 0.559166155732679
2 0.559779276517474
2 0.560392397302269
2 0.561005518087063
2 0.561618638871858
2 0.562231759656652
2 0.562844880441447
2 0.563458001226242
2 0.564071122011036
2 0.564684242795831
2 0.565297363580625
2 0.56591048436542
2 0.566523605150215
2 0.567136725935009
2 0.567749846719804
2 0.568362967504598
2 0.568976088289393
2 0.569589209074188
2 0.570202329858982
2 0.570815450643777
2 0.571428571428571
2 0.572041692213366
2 0.572654812998161
2 0.573267933782955
2 0.57388105456775
2 0.574494175352544
2 0.575107296137339
2 0.575720416922134
2 0.576333537706928
2 0.576946658491723
2 0.577559779276517
2 0.578172900061312
2 0.578786020846107
2 0.579399141630901
2 0.580012262415696
2 0.580625383200491
2 0.581238503985285
2 0.58185162477008
2 0.582464745554874
2 0.583077866339669
2 0.583690987124464
2 0.584304107909258
2 0.584917228694053
2 0.585530349478847
2 0.586143470263642
2 0.586756591048437
2 0.587369711833231
2 0.587982832618026
2 0.58859595340282
2 0.589209074187615
2 0.58982219497241
2 0.590435315757204
2 0.591048436541999
2 0.591661557326793
2 0.592274678111588
2 0.592887798896383
2 0.593500919681177
2 0.594114040465972
2 0.594727161250766
2 0.595340282035561
2 0.595953402820356
2 0.59656652360515
2 0.597179644389945
2 0.597792765174739
2 0.598405885959534
2 0.599019006744329
2 0.599632127529123
2 0.600245248313918
2 0.600858369098712
2 0.601471489883507
2 0.602084610668302
2 0.602697731453096
2 0.603310852237891
2 0.603923973022686
2 0.60453709380748
2 0.605150214592275
2 0.605763335377069
2 0.606376456161864
2 0.606989576946658
2 0.607602697731453
2 0.608215818516248
2 0.608828939301042
2 0.609442060085837
2 0.610055180870631
2 0.610668301655426
2 0.611281422440221
2 0.611894543225015
3 0.61250766400981
3 0.613120784794605
3 0.613733905579399
3 0.614347026364194
3 0.614960147148988
3 0.615573267933783
3 0.616186388718578
3 0.616799509503372
3 0.617412630288167
3 0.618025751072961
3 0.618638871857756
3 0.619251992642551
3 0.619865113427345
3 0.62047823421214
3 0.621091354996934
3 0.621704475781729
3 0.622317596566524
3 0.622930717351318
3 0.623543838136113
3 0.624156958920907
3 0.624770079705702
3 0.625383200490497
3 0.625996321275291
3 0.626609442060086
3 0.62722256284488
3 0.627835683629675
3 0.62844880441447
3 0.629061925199264
3 0.629675045984059
3 0.630288166768853
3 0.630901287553648
3 0.631514408338443
3 0.632127529123237
3 0.632740649908032
3 0.633353770692827
3 0.633966891477621
3 0.634580012262416
3 0.63519313304721
3 0.635806253832005
3 0.6364193746168
3 0.637032495401594
3 0.637645616186389
3 0.638258736971183
3 0.638871857755978
3 0.639484978540773
3 0.640098099325567
3 0.640711220110362
3 0.641324340895156
3 0.641937461679951
3 0.642550582464746
3 0.64316370324954
3 0.643776824034335
3 0.644389944819129
3 0.645003065603924
3 0.645616186388719
3 0.646229307173513
3 0.646842427958308
3 0.647455548743102
3 0.648068669527897
3 0.648681790312692
4 0.649294911097486
4 0.649908031882281
4 0.650521152667075
4 0.65113427345187
4 0.651747394236665
4 0.652360515021459
4 0.652973635806254
4 0.653586756591048
4 0.654199877375843
4 0.654812998160638
4 0.655426118945432
4 0.656039239730227
4 0.656652360515021
4 0.657265481299816
4 0.657878602084611
4 0.658491722869405
4 0.6591048436542
4 0.659717964438994
4 0.660331085223789
4 0.660944206008584
4 0.661557326793378
4 0.662170447578173
4 0.662783568362967
4 0.663396689147762
4 0.664009809932557
4 0.664622930717351
4 0.665236051502146
4 0.665849172286941
4 0.666462293071735
4 0.66707541385653
4 0.667688534641324
4 0.668301655426119
4 0.668914776210914
4 0.669527896995708
4 0.670141017780503
4 0.670754138565297
4 0.671367259350092
4 0.671980380134887
4 0.672593500919681
4 0.673206621704476
4 0.67381974248927
4 0.674432863274065
4 0.67504598405886
4 0.675659104843654
4 0.676272225628449
4 0.676885346413243
4 0.677498467198038
4 0.678111587982833
4 0.678724708767627
4 0.679337829552422
5 0.679950950337216
5 0.680564071122011
5 0.681177191906806
5 0.6817903126916
5 0.682403433476395
5 0.683016554261189
5 0.683629675045984
5 0.684242795830779
5 0.684855916615573
5 0.685469037400368
5 0.686082158185162
5 0.686695278969957
5 0.687308399754752
5 0.687921520539546
5 0.688534641324341
5 0.689147762109136
5 0.68976088289393
5 0.690374003678725
5 0.690987124463519
5 0.691600245248314
5 0.692213366033109
5 0.692826486817903
5 0.693439607602698
5 0.694052728387492
5 0.694665849172287
5 0.695278969957081
5 0.695892090741876
5 0.696505211526671
5 0.697118332311465
5 0.69773145309626
5 0.698344573881055
5 0.698957694665849
5 0.699570815450644
5 0.700183936235438
5 0.700797057020233
5 0.701410177805028
5 0.702023298589822
5 0.702636419374617
5 0.703249540159411
5 0.703862660944206
5 0.704475781729001
5 0.705088902513795
5 0.70570202329859
5 0.706315144083384
5 0.706928264868179
5 0.707541385652974
5 0.708154506437768
5 0.708767627222563
5 0.709380748007357
5 0.709993868792152
5 0.710606989576947
6 0.711220110361741
6 0.711833231146536
6 0.71244635193133
6 0.713059472716125
6 0.71367259350092
6 0.714285714285714
6 0.714898835070509
6 0.715511955855303
6 0.716125076640098
6 0.716738197424893
6 0.717351318209687
6 0.717964438994482
6 0.718577559779277
6 0.719190680564071
6 0.719803801348866
6 0.72041692213366
6 0.721030042918455
6 0.72164316370325
6 0.722256284488044
6 0.722869405272839
6 0.723482526057633
6 0.724095646842428
6 0.724708767627223
6 0.725321888412017
6 0.725935009196812
6 0.726548129981606
6 0.727161250766401
6 0.727774371551196
6 0.72838749233599
6 0.729000613120785
6 0.729613733905579
6 0.730226854690374
6 0.730839975475169
7 0.731453096259963
7 0.732066217044758
7 0.732679337829552
7 0.733292458614347
7 0.733905579399142
7 0.734518700183936
7 0.735131820968731
7 0.735744941753525
7 0.73635806253832
7 0.736971183323115
7 0.737584304107909
7 0.738197424892704
7 0.738810545677498
7 0.739423666462293
7 0.740036787247088
7 0.740649908031882
7 0.741263028816677
7 0.741876149601472
7 0.742489270386266
7 0.743102391171061
8 0.743715511955855
8 0.74432863274065
8 0.744941753525445
8 0.745554874310239
8 0.746167995095034
8 0.746781115879828
8 0.747394236664623
8 0.748007357449417
8 0.748620478234212
8 0.749233599019007
8 0.749846719803801
8 0.750459840588596
8 0.751072961373391
8 0.751686082158185
8 0.75229920294298
8 0.752912323727774
8 0.753525444512569
8 0.754138565297364
8 0.754751686082158
8 0.755364806866953
9 0.755977927651747
9 0.756591048436542
9 0.757204169221337
9 0.757817290006131
9 0.758430410790926
9 0.75904353157572
9 0.759656652360515
9 0.76026977314531
9 0.760882893930104
9 0.761496014714899
10 0.762109135499693
10 0.762722256284488
10 0.763335377069283
10 0.763948497854077
10 0.764561618638872
10 0.765174739423666
10 0.765787860208461
11 0.766400980993256
12 0.76701410177805
12 0.767627222562845
12 0.768240343347639
12 0.768853464132434
12 0.769466584917229
12 0.770079705702023
12 0.770692826486818
12 0.771305947271613
12 0.771919068056407
13 0.772532188841202
13 0.773145309625996
13 0.773758430410791
13 0.774371551195586
13 0.77498467198038
13 0.775597792765175
13 0.776210913549969
13 0.776824034334764
13 0.777437155119559
14 0.778050275904353
14 0.778663396689148
15 0.779276517473942
15 0.779889638258737
15 0.780502759043532
15 0.781115879828326
16 0.781729000613121
16 0.782342121397915
16 0.78295524218271
16 0.783568362967505
16 0.784181483752299
16 0.784794604537094
16 0.785407725321888
17 0.786020846106683
17 0.786633966891478
17 0.787247087676272
17 0.787860208461067
18 0.788473329245861
18 0.789086450030656
19 0.789699570815451
19 0.790312691600245
19 0.79092581238504
19 0.791538933169834
21 0.792152053954629
21 0.792765174739424
21 0.793378295524218
21 0.793991416309013
21 0.794604537093808
22 0.795217657878602
23 0.795830778663397
24 0.796443899448191
25 0.797057020232986
25 0.797670141017781
26 0.798283261802575
27 0.79889638258737
28 0.799509503372164
28 0.800122624156959
28 0.800735744941753
29 0.801348865726548
29 0.801961986511343
29 0.802575107296137
29 0.803188228080932
30 0.803801348865727
31 0.804414469650521
31 0.805027590435316
32 0.80564071122011
34 0.806253832004905
37 0.8068669527897
37 0.807480073574494
38 0.808093194359289
39 0.808706315144083
39 0.809319435928878
39 0.809932556713673
40 0.810545677498467
41 0.811158798283262
41 0.811771919068056
41 0.812385039852851
42 0.812998160637646
42 0.81361128142244
42 0.814224402207235
44 0.814837522992029
46 0.815450643776824
48 0.816063764561619
49 0.816676885346413
55 0.817290006131208
56 0.817903126916002
56 0.818516247700797
59 0.819129368485592
63 0.819742489270386
72 0.820355610055181
80 0.820968730839975
83 0.82158185162477
84 0.822194972409565
84 0.822808093194359
88 0.823421213979154
89 0.824034334763948
97 0.824647455548743
98 0.825260576333538
100 0.825873697118332
104 0.826486817903127
114 0.827099938687922
116 0.827713059472716
128 0.828326180257511
132 0.828939301042305
150 0.8295524218271
151 0.830165542611895
170 0.830778663396689
176 0.831391784181484
177 0.832004904966278
198 0.832618025751073
223 0.833231146535868
258 0.833844267320662
269 0.834457388105457
482 0.835070508890251
545 0.835683629675046
558 0.836296750459841
1150 0.836909871244635
1252 0.83752299202943
1262 0.838136112814224
1264 0.838749233599019
1297 0.839362354383814
1328 0.839975475168608
1336 0.840588595953403
1360 0.841201716738197
1374 0.841814837522992
1407 0.842427958307787
1408 0.843041079092581
1415 0.843654199877376
1419 0.84426732066217
1433 0.844880441446965
1438 0.84549356223176
1490 0.846106683016554
1512 0.846719803801349
1515 0.847332924586144
1526 0.847946045370938
1544 0.848559166155733
1560 0.849172286940527
1574 0.849785407725322
1590 0.850398528510117
1627 0.851011649294911
1628 0.851624770079706
1648 0.8522378908645
1660 0.852851011649295
1665 0.853464132434089
1669 0.854077253218884
1715 0.854690374003679
1717 0.855303494788473
1740 0.855916615573268
1762 0.856529736358063
1774 0.857142857142857
1792 0.857755977927652
1895 0.858369098712446
1925 0.858982219497241
1926 0.859595340282036
1963 0.86020846106683
1980 0.860821581851625
1983 0.861434702636419
2078 0.862047823421214
2094 0.862660944206009
2096 0.863274064990803
2126 0.863887185775598
2145 0.864500306560392
2145 0.865113427345187
2167 0.865726548129982
2171 0.866339668914776
2183 0.866952789699571
2190 0.867565910484365
2210 0.86817903126916
2222 0.868792152053955
2228 0.869405272838749
2229 0.870018393623544
2246 0.870631514408338
2248 0.871244635193133
2248 0.871857755977928
2276 0.872470876762722
2314 0.873083997547517
2318 0.873697118332311
2324 0.874310239117106
2335 0.874923359901901
2388 0.875536480686695
2394 0.87614960147149
2395 0.876762722256284
2415 0.877375843041079
2418 0.877988963825874
2469 0.878602084610668
2505 0.879215205395463
2508 0.879828326180258
2583 0.880441446965052
2608 0.881054567749847
2617 0.881667688534641
2742 0.882280809319436
2753 0.882893930104231
2776 0.883507050889025
2823 0.88412017167382
2841 0.884733292458614
2859 0.885346413243409
2878 0.885959534028204
2891 0.886572654812998
2899 0.887185775597793
2901 0.887798896382587
2920 0.888412017167382
2940 0.889025137952177
3016 0.889638258736971
3018 0.890251379521766
3103 0.89086450030656
3136 0.891477621091355
3153 0.89209074187615
3202 0.892703862660944
3214 0.893316983445739
3225 0.893930104230533
3240 0.894543225015328
3298 0.895156345800123
3365 0.895769466584917
3377 0.896382587369712
3451 0.896995708154506
3467 0.897608828939301
3474 0.898221949724096
3507 0.89883507050889
3557 0.899448191293685
3595 0.90006131207848
3642 0.900674432863274
3715 0.901287553648069
3774 0.901900674432863
3808 0.902513795217658
3829 0.903126916002452
3879 0.903740036787247
3947 0.904353157572042
3986 0.904966278356836
4042 0.905579399141631
4058 0.906192519926425
4071 0.90680564071122
4269 0.907418761496015
4271 0.908031882280809
4323 0.908645003065604
4353 0.909258123850399
4368 0.909871244635193
4376 0.910484365419988
4398 0.911097486204782
4437 0.911710606989577
4443 0.912323727774372
4443 0.912936848559166
4466 0.913549969343961
4581 0.914163090128755
4585 0.91477621091355
4646 0.915389331698345
4708 0.916002452483139
4711 0.916615573267934
4750 0.917228694052728
4757 0.917841814837523
4779 0.918454935622318
4792 0.919068056407112
4834 0.919681177191907
5003 0.920294297976701
5036 0.920907418761496
5055 0.921520539546291
5146 0.922133660331085
5160 0.92274678111588
5172 0.923359901900674
5199 0.923973022685469
5337 0.924586143470264
5372 0.925199264255058
5408 0.925812385039853
5441 0.926425505824647
5475 0.927038626609442
5571 0.927651747394237
5664 0.928264868179031
5665 0.928877988963826
5733 0.92949110974862
5759 0.930104230533415
5776 0.93071735131821
5904 0.931330472103004
5955 0.931943592887799
6148 0.932556713672594
6190 0.933169834457388
6405 0.933782955242183
6646 0.934396076026977
6731 0.935009196811772
6804 0.935622317596567
7057 0.936235438381361
7193 0.936848559166156
7227 0.93746167995095
7414 0.938074800735745
7472 0.93868792152054
7544 0.939301042305334
7600 0.939914163090129
7702 0.940527283874923
7735 0.941140404659718
7787 0.941753525444513
7969 0.942366646229307
8054 0.942979767014102
8064 0.943592887798896
8704 0.944206008583691
8704 0.944819129368486
8900 0.94543225015328
9085 0.946045370938075
9253 0.946658491722869
9615 0.947271612507664
9788 0.947884733292459
9947 0.948497854077253
10125 0.949110974862048
10370 0.949724095646842
10636 0.950337216431637
10646 0.950950337216432
10721 0.951563458001226
10894 0.952176578786021
11084 0.952789699570815
11140 0.95340282035561
11198 0.954015941140405
11352 0.954629061925199
12157 0.955242182709994
12253 0.955855303494788
13036 0.956468424279583
13063 0.957081545064378
13083 0.957694665849172
13454 0.958307786633967
13674 0.958920907418761
13941 0.959534028203556
13991 0.960147148988351
14833 0.960760269773145
15146 0.96137339055794
15157 0.961986511342734
15618 0.962599632127529
15899 0.963212752912324
16088 0.963825873697118
16448 0.964438994481913
16501 0.965052115266708
16579 0.965665236051502
16678 0.966278356836297
16794 0.966891477621091
16896 0.967504598405886
18219 0.968117719190681
19006 0.968730839975475
19142 0.96934396076027
19319 0.969957081545064
19389 0.970570202329859
20982 0.971183323114654
22081 0.971796443899448
22163 0.972409564684243
22750 0.973022685469037
22965 0.973635806253832
23038 0.974248927038627
23901 0.974862047823421
24727 0.975475168608216
25615 0.97608828939301
25736 0.976701410177805
26982 0.9773145309626
28980 0.977927651747394
30119 0.978540772532189
30493 0.979153893316983
32591 0.979767014101778
32892 0.980380134886573
33716 0.980993255671367
37101 0.981606376456162
37661 0.982219497240956
50307 0.982832618025751
55025 0.983445738810546
56567 0.98405885959534
62532 0.984671980380135
66172 0.98528510116493
69002 0.985898221949724
69281 0.986511342734519
76248 0.987124463519313
80645 0.987737584304108
81542 0.988350705088903
85430 0.988963825873697
107572 0.989576946658492
119507 0.990190067443286
127472 0.990803188228081
132721 0.991416309012876
136178 0.99202942979767
140808 0.992642550582465
146419 0.993255671367259
156109 0.993868792152054
166901 0.994481912936849
195682 0.995095033721643
216049 0.995708154506438
328097 0.996321275291232
337633 0.996934396076027
383598 0.997547516860822
394517 0.998160637645616
475152 0.998773758430411
475649 0.999386879215205
1076090 1
};
\addplot [semithick, darkorange25512714]
table {%
0 0.000613120784794605
0 0.00122624156958921
0 0.00183936235438381
0 0.00245248313917842
0 0.00306560392397302
0 0.00367872470876763
0 0.00429184549356223
0 0.00490496627835684
0 0.00551808706315144
0 0.00613120784794605
0 0.00674432863274065
0 0.00735744941753525
0 0.00797057020232986
0 0.00858369098712446
0 0.00919681177191907
0 0.00980993255671367
0 0.0104230533415083
0 0.0110361741263029
0 0.0116492949110975
0 0.0122624156958921
0 0.0128755364806867
0 0.0134886572654813
0 0.0141017780502759
0 0.0147148988350705
0 0.0153280196198651
0 0.0159411404046597
0 0.0165542611894543
0 0.0171673819742489
0 0.0177805027590435
0 0.0183936235438381
0 0.0190067443286327
0 0.0196198651134273
0 0.020232985898222
0 0.0208461066830166
0 0.0214592274678112
0 0.0220723482526058
0 0.0226854690374004
0 0.023298589822195
0 0.0239117106069896
0 0.0245248313917842
0 0.0251379521765788
0 0.0257510729613734
0 0.026364193746168
0 0.0269773145309626
0 0.0275904353157572
0 0.0282035561005518
0 0.0288166768853464
0 0.029429797670141
0 0.0300429184549356
0 0.0306560392397302
0 0.0312691600245248
0 0.0318822808093194
0 0.032495401594114
0 0.0331085223789086
0 0.0337216431637032
0 0.0343347639484979
0 0.0349478847332925
0 0.0355610055180871
0 0.0361741263028817
0 0.0367872470876763
0 0.0374003678724709
0 0.0380134886572655
0 0.0386266094420601
0 0.0392397302268547
0 0.0398528510116493
0 0.0404659717964439
0 0.0410790925812385
0 0.0416922133660331
0 0.0423053341508277
0 0.0429184549356223
0 0.0435315757204169
0 0.0441446965052115
0 0.0447578172900061
0 0.0453709380748007
0 0.0459840588595953
0 0.0465971796443899
0 0.0472103004291846
0 0.0478234212139792
0 0.0484365419987738
0 0.0490496627835684
0 0.049662783568363
0 0.0502759043531576
0 0.0508890251379522
0 0.0515021459227468
0 0.0521152667075414
0 0.052728387492336
0 0.0533415082771306
0 0.0539546290619252
0 0.0545677498467198
0 0.0551808706315144
0 0.055793991416309
0 0.0564071122011036
0 0.0570202329858982
0 0.0576333537706928
0 0.0582464745554874
0 0.058859595340282
0 0.0594727161250766
0 0.0600858369098712
0 0.0606989576946659
0 0.0613120784794605
0 0.0619251992642551
0 0.0625383200490497
0 0.0631514408338443
0 0.0637645616186389
0 0.0643776824034335
0 0.0649908031882281
0 0.0656039239730227
0 0.0662170447578173
1 0.0668301655426119
1 0.0674432863274065
1 0.0680564071122011
1 0.0686695278969957
1 0.0692826486817903
1 0.0698957694665849
1 0.0705088902513795
1 0.0711220110361741
1 0.0717351318209687
1 0.0723482526057633
1 0.0729613733905579
1 0.0735744941753525
1 0.0741876149601472
1 0.0748007357449418
1 0.0754138565297364
1 0.076026977314531
1 0.0766400980993256
1 0.0772532188841202
1 0.0778663396689148
1 0.0784794604537094
1 0.079092581238504
1 0.0797057020232986
1 0.0803188228080932
1 0.0809319435928878
1 0.0815450643776824
1 0.082158185162477
1 0.0827713059472716
1 0.0833844267320662
1 0.0839975475168608
1 0.0846106683016554
1 0.08522378908645
1 0.0858369098712446
1 0.0864500306560392
1 0.0870631514408338
1 0.0876762722256284
1 0.0882893930104231
1 0.0889025137952177
1 0.0895156345800123
1 0.0901287553648069
1 0.0907418761496015
1 0.0913549969343961
1 0.0919681177191907
1 0.0925812385039853
1 0.0931943592887799
1 0.0938074800735745
1 0.0944206008583691
1 0.0950337216431637
1 0.0956468424279583
1 0.0962599632127529
1 0.0968730839975475
1 0.0974862047823421
1 0.0980993255671367
1 0.0987124463519313
2 0.0993255671367259
2 0.0999386879215205
2 0.100551808706315
2 0.10116492949111
2 0.101778050275904
2 0.102391171060699
2 0.103004291845494
2 0.103617412630288
2 0.104230533415083
2 0.104843654199877
2 0.105456774984672
2 0.106069895769467
2 0.106683016554261
2 0.107296137339056
2 0.10790925812385
2 0.108522378908645
2 0.10913549969344
2 0.109748620478234
2 0.110361741263029
2 0.110974862047823
2 0.111587982832618
2 0.112201103617413
2 0.112814224402207
2 0.113427345187002
2 0.114040465971796
2 0.114653586756591
2 0.115266707541386
2 0.11587982832618
2 0.116492949110975
2 0.117106069895769
2 0.117719190680564
2 0.118332311465359
2 0.118945432250153
2 0.119558553034948
2 0.120171673819742
2 0.120784794604537
2 0.121397915389332
2 0.122011036174126
2 0.122624156958921
2 0.123237277743716
3 0.12385039852851
3 0.124463519313305
3 0.125076640098099
3 0.125689760882894
3 0.126302881667689
3 0.126916002452483
3 0.127529123237278
3 0.128142244022072
3 0.128755364806867
3 0.129368485591662
3 0.129981606376456
3 0.130594727161251
3 0.131207847946045
3 0.13182096873084
3 0.132434089515635
3 0.133047210300429
3 0.133660331085224
3 0.134273451870018
3 0.134886572654813
3 0.135499693439608
3 0.136112814224402
3 0.136725935009197
3 0.137339055793991
3 0.137952176578786
3 0.138565297363581
3 0.139178418148375
3 0.13979153893317
3 0.140404659717964
3 0.141017780502759
3 0.141630901287554
3 0.142244022072348
3 0.142857142857143
3 0.143470263641937
3 0.144083384426732
4 0.144696505211527
4 0.145309625996321
4 0.145922746781116
4 0.14653586756591
4 0.147148988350705
4 0.1477621091355
4 0.148375229920294
4 0.148988350705089
4 0.149601471489884
4 0.150214592274678
4 0.150827713059473
4 0.151440833844267
4 0.152053954629062
4 0.152667075413857
4 0.153280196198651
4 0.153893316983446
4 0.15450643776824
4 0.155119558553035
4 0.15573267933783
5 0.156345800122624
5 0.156958920907419
5 0.157572041692213
5 0.158185162477008
5 0.158798283261803
5 0.159411404046597
5 0.160024524831392
5 0.160637645616186
5 0.161250766400981
5 0.161863887185776
5 0.16247700797057
5 0.163090128755365
5 0.163703249540159
5 0.164316370324954
5 0.164929491109749
5 0.165542611894543
5 0.166155732679338
5 0.166768853464132
5 0.167381974248927
5 0.167995095033722
5 0.168608215818516
5 0.169221336603311
5 0.169834457388105
6 0.1704475781729
6 0.171060698957695
6 0.171673819742489
6 0.172286940527284
6 0.172900061312078
6 0.173513182096873
6 0.174126302881668
6 0.174739423666462
6 0.175352544451257
6 0.175965665236052
6 0.176578786020846
6 0.177191906805641
6 0.177805027590435
6 0.17841814837523
6 0.179031269160025
6 0.179644389944819
6 0.180257510729614
6 0.180870631514408
6 0.181483752299203
6 0.182096873083998
6 0.182709993868792
6 0.183323114653587
7 0.183936235438381
7 0.184549356223176
7 0.185162477007971
7 0.185775597792765
7 0.18638871857756
7 0.187001839362354
7 0.187614960147149
7 0.188228080931944
7 0.188841201716738
7 0.189454322501533
7 0.190067443286327
7 0.190680564071122
7 0.191293684855917
7 0.191906805640711
7 0.192519926425506
7 0.1931330472103
7 0.193746167995095
7 0.19435928877989
7 0.194972409564684
7 0.195585530349479
7 0.196198651134273
8 0.196811771919068
8 0.197424892703863
8 0.198038013488657
8 0.198651134273452
8 0.199264255058246
8 0.199877375843041
8 0.200490496627836
8 0.20110361741263
8 0.201716738197425
8 0.20232985898222
8 0.202942979767014
8 0.203556100551809
8 0.204169221336603
8 0.204782342121398
8 0.205395462906193
8 0.206008583690987
9 0.206621704475782
9 0.207234825260576
9 0.207847946045371
9 0.208461066830166
9 0.20907418761496
9 0.209687308399755
9 0.210300429184549
9 0.210913549969344
9 0.211526670754139
9 0.212139791538933
9 0.212752912323728
9 0.213366033108522
9 0.213979153893317
9 0.214592274678112
9 0.215205395462906
9 0.215818516247701
9 0.216431637032495
9 0.21704475781729
9 0.217657878602085
9 0.218270999386879
10 0.218884120171674
10 0.219497240956468
10 0.220110361741263
10 0.220723482526058
10 0.221336603310852
10 0.221949724095647
10 0.222562844880441
10 0.223175965665236
10 0.223789086450031
10 0.224402207234825
10 0.22501532801962
10 0.225628448804414
10 0.226241569589209
10 0.226854690374004
10 0.227467811158798
10 0.228080931943593
10 0.228694052728387
11 0.229307173513182
11 0.229920294297977
11 0.230533415082771
11 0.231146535867566
11 0.231759656652361
11 0.232372777437155
11 0.23298589822195
11 0.233599019006744
11 0.234212139791539
11 0.234825260576334
11 0.235438381361128
11 0.236051502145923
12 0.236664622930717
12 0.237277743715512
12 0.237890864500307
12 0.238503985285101
12 0.239117106069896
12 0.23973022685469
12 0.240343347639485
12 0.24095646842428
12 0.241569589209074
12 0.242182709993869
12 0.242795830778663
12 0.243408951563458
12 0.244022072348253
12 0.244635193133047
13 0.245248313917842
13 0.245861434702636
13 0.246474555487431
13 0.247087676272226
13 0.24770079705702
13 0.248313917841815
13 0.248927038626609
13 0.249540159411404
13 0.250153280196199
13 0.250766400980993
13 0.251379521765788
13 0.251992642550582
13 0.252605763335377
13 0.253218884120172
13 0.253832004904966
14 0.254445125689761
14 0.255058246474555
14 0.25567136725935
14 0.256284488044145
14 0.256897608828939
14 0.257510729613734
14 0.258123850398529
14 0.258736971183323
14 0.259350091968118
14 0.259963212752912
14 0.260576333537707
14 0.261189454322502
15 0.261802575107296
15 0.262415695892091
15 0.263028816676885
15 0.26364193746168
15 0.264255058246475
15 0.264868179031269
15 0.265481299816064
15 0.266094420600858
15 0.266707541385653
16 0.267320662170448
16 0.267933782955242
16 0.268546903740037
16 0.269160024524831
16 0.269773145309626
16 0.270386266094421
16 0.270999386879215
16 0.27161250766401
16 0.272225628448804
16 0.272838749233599
16 0.273451870018394
16 0.274064990803188
16 0.274678111587983
16 0.275291232372777
16 0.275904353157572
17 0.276517473942367
17 0.277130594727161
17 0.277743715511956
17 0.27835683629675
17 0.278969957081545
17 0.27958307786634
18 0.280196198651134
18 0.280809319435929
18 0.281422440220723
18 0.282035561005518
18 0.282648681790313
18 0.283261802575107
19 0.283874923359902
19 0.284488044144697
19 0.285101164929491
19 0.285714285714286
19 0.28632740649908
19 0.286940527283875
19 0.28755364806867
19 0.288166768853464
19 0.288779889638259
19 0.289393010423053
19 0.290006131207848
19 0.290619251992643
19 0.291232372777437
20 0.291845493562232
20 0.292458614347026
20 0.293071735131821
20 0.293684855916616
20 0.29429797670141
20 0.294911097486205
20 0.295524218270999
20 0.296137339055794
20 0.296750459840589
20 0.297363580625383
20 0.297976701410178
20 0.298589822194972
20 0.299202942979767
20 0.299816063764562
20 0.300429184549356
21 0.301042305334151
21 0.301655426118945
21 0.30226854690374
21 0.302881667688535
22 0.303494788473329
22 0.304107909258124
22 0.304721030042918
22 0.305334150827713
22 0.305947271612508
22 0.306560392397302
22 0.307173513182097
23 0.307786633966891
23 0.308399754751686
23 0.309012875536481
23 0.309625996321275
23 0.31023911710607
23 0.310852237890865
23 0.311465358675659
23 0.312078479460454
24 0.312691600245248
24 0.313304721030043
24 0.313917841814838
24 0.314530962599632
24 0.315144083384427
24 0.315757204169221
24 0.316370324954016
25 0.316983445738811
25 0.317596566523605
25 0.3182096873084
25 0.318822808093194
25 0.319435928877989
25 0.320049049662784
25 0.320662170447578
26 0.321275291232373
26 0.321888412017167
26 0.322501532801962
26 0.323114653586757
26 0.323727774371551
26 0.324340895156346
26 0.32495401594114
26 0.325567136725935
26 0.32618025751073
26 0.326793378295524
26 0.327406499080319
27 0.328019619865113
27 0.328632740649908
27 0.329245861434703
27 0.329858982219497
27 0.330472103004292
27 0.331085223789086
27 0.331698344573881
27 0.332311465358676
27 0.33292458614347
28 0.333537706928265
28 0.334150827713059
28 0.334763948497854
28 0.335377069282649
28 0.335990190067443
28 0.336603310852238
28 0.337216431637033
29 0.337829552421827
29 0.338442673206622
29 0.339055793991416
30 0.339668914776211
30 0.340282035561006
30 0.3408951563458
30 0.341508277130595
30 0.342121397915389
30 0.342734518700184
30 0.343347639484979
30 0.343960760269773
30 0.344573881054568
31 0.345187001839362
31 0.345800122624157
31 0.346413243408952
31 0.347026364193746
31 0.347639484978541
31 0.348252605763335
31 0.34886572654813
32 0.349478847332925
32 0.350091968117719
32 0.350705088902514
32 0.351318209687308
33 0.351931330472103
33 0.352544451256898
33 0.353157572041692
33 0.353770692826487
33 0.354383813611281
33 0.354996934396076
34 0.355610055180871
34 0.356223175965665
34 0.35683629675046
35 0.357449417535254
35 0.358062538320049
35 0.358675659104844
35 0.359288779889638
35 0.359901900674433
35 0.360515021459227
36 0.361128142244022
36 0.361741263028817
36 0.362354383813611
36 0.362967504598406
36 0.363580625383201
36 0.364193746167995
36 0.36480686695279
37 0.365419987737584
37 0.366033108522379
37 0.366646229307173
37 0.367259350091968
37 0.367872470876763
37 0.368485591661557
38 0.369098712446352
38 0.369711833231147
38 0.370324954015941
38 0.370938074800736
38 0.37155119558553
38 0.372164316370325
38 0.37277743715512
38 0.373390557939914
39 0.374003678724709
39 0.374616799509503
39 0.375229920294298
40 0.375843041079093
40 0.376456161863887
40 0.377069282648682
41 0.377682403433476
41 0.378295524218271
42 0.378908645003066
42 0.37952176578786
42 0.380134886572655
42 0.380748007357449
42 0.381361128142244
42 0.381974248927039
42 0.382587369711833
43 0.383200490496628
43 0.383813611281422
43 0.384426732066217
43 0.385039852851012
44 0.385652973635806
45 0.386266094420601
45 0.386879215205395
45 0.38749233599019
45 0.388105456774985
45 0.388718577559779
46 0.389331698344574
46 0.389944819129368
46 0.390557939914163
46 0.391171060698958
46 0.391784181483752
46 0.392397302268547
47 0.393010423053341
47 0.393623543838136
47 0.394236664622931
47 0.394849785407725
48 0.39546290619252
48 0.396076026977315
49 0.396689147762109
49 0.397302268546904
49 0.397915389331698
49 0.398528510116493
49 0.399141630901288
49 0.399754751686082
50 0.400367872470877
50 0.400980993255671
50 0.401594114040466
50 0.402207234825261
50 0.402820355610055
50 0.40343347639485
50 0.404046597179644
51 0.404659717964439
51 0.405272838749234
51 0.405885959534028
51 0.406499080318823
51 0.407112201103617
52 0.407725321888412
52 0.408338442673207
52 0.408951563458001
52 0.409564684242796
52 0.41017780502759
52 0.410790925812385
53 0.41140404659718
53 0.412017167381974
53 0.412630288166769
53 0.413243408951563
53 0.413856529736358
54 0.414469650521153
54 0.415082771305947
54 0.415695892090742
54 0.416309012875536
55 0.416922133660331
55 0.417535254445126
55 0.41814837522992
55 0.418761496014715
55 0.419374616799509
56 0.419987737584304
56 0.420600858369099
56 0.421213979153893
57 0.421827099938688
57 0.422440220723483
57 0.423053341508277
58 0.423666462293072
58 0.424279583077866
58 0.424892703862661
59 0.425505824647456
59 0.42611894543225
59 0.426732066217045
60 0.427345187001839
60 0.427958307786634
60 0.428571428571429
60 0.429184549356223
60 0.429797670141018
61 0.430410790925812
61 0.431023911710607
61 0.431637032495402
61 0.432250153280196
61 0.432863274064991
61 0.433476394849785
62 0.43408951563458
62 0.434702636419375
62 0.435315757204169
63 0.435928877988964
63 0.436541998773758
63 0.437155119558553
63 0.437768240343348
64 0.438381361128142
64 0.438994481912937
64 0.439607602697731
64 0.440220723482526
64 0.440833844267321
64 0.441446965052115
64 0.44206008583691
64 0.442673206621704
65 0.443286327406499
65 0.443899448191294
65 0.444512568976088
65 0.445125689760883
65 0.445738810545677
66 0.446351931330472
66 0.446965052115267
66 0.447578172900061
66 0.448191293684856
66 0.448804414469651
67 0.449417535254445
67 0.45003065603924
68 0.450643776824034
68 0.451256897608829
68 0.451870018393624
68 0.452483139178418
69 0.453096259963213
69 0.453709380748007
69 0.454322501532802
69 0.454935622317597
69 0.455548743102391
70 0.456161863887186
70 0.45677498467198
71 0.457388105456775
71 0.45800122624157
71 0.458614347026364
71 0.459227467811159
72 0.459840588595953
72 0.460453709380748
73 0.461066830165543
73 0.461679950950337
73 0.462293071735132
74 0.462906192519926
74 0.463519313304721
75 0.464132434089516
75 0.46474555487431
76 0.465358675659105
76 0.465971796443899
76 0.466584917228694
77 0.467198038013489
77 0.467811158798283
78 0.468424279583078
78 0.469037400367872
78 0.469650521152667
79 0.470263641937462
79 0.470876762722256
80 0.471489883507051
80 0.472103004291845
80 0.47271612507664
81 0.473329245861435
82 0.473942366646229
82 0.474555487431024
83 0.475168608215819
83 0.475781729000613
83 0.476394849785408
83 0.477007970570202
83 0.477621091354997
83 0.478234212139792
84 0.478847332924586
84 0.479460453709381
84 0.480073574494175
84 0.48068669527897
84 0.481299816063765
85 0.481912936848559
86 0.482526057633354
86 0.483139178418148
87 0.483752299202943
87 0.484365419987738
87 0.484978540772532
88 0.485591661557327
88 0.486204782342121
89 0.486817903126916
89 0.487431023911711
89 0.488044144696505
89 0.4886572654813
90 0.489270386266094
90 0.489883507050889
90 0.490496627835684
91 0.491109748620478
91 0.491722869405273
92 0.492335990190067
92 0.492949110974862
92 0.493562231759657
93 0.494175352544451
93 0.494788473329246
93 0.49540159411404
93 0.496014714898835
94 0.49662783568363
94 0.497240956468424
94 0.497854077253219
94 0.498467198038013
95 0.499080318822808
97 0.499693439607603
97 0.500306560392397
97 0.500919681177192
98 0.501532801961986
99 0.502145922746781
99 0.502759043531576
99 0.50337216431637
99 0.503985285101165
100 0.50459840588596
101 0.505211526670754
101 0.505824647455549
101 0.506437768240343
101 0.507050889025138
102 0.507664009809933
104 0.508277130594727
105 0.508890251379522
105 0.509503372164316
105 0.510116492949111
106 0.510729613733906
107 0.5113427345187
108 0.511955855303495
108 0.512568976088289
108 0.513182096873084
109 0.513795217657879
109 0.514408338442673
110 0.515021459227468
110 0.515634580012262
111 0.516247700797057
111 0.516860821581852
111 0.517473942366646
113 0.518087063151441
115 0.518700183936235
115 0.51931330472103
115 0.519926425505825
116 0.520539546290619
116 0.521152667075414
116 0.521765787860208
117 0.522378908645003
117 0.522992029429798
117 0.523605150214592
118 0.524218270999387
118 0.524831391784181
118 0.525444512568976
119 0.526057633353771
120 0.526670754138565
123 0.52728387492336
123 0.527896995708155
124 0.528510116492949
124 0.529123237277744
124 0.529736358062538
125 0.530349478847333
125 0.530962599632128
125 0.531575720416922
127 0.532188841201717
127 0.532801961986511
128 0.533415082771306
128 0.534028203556101
128 0.534641324340895
128 0.53525444512569
128 0.535867565910484
129 0.536480686695279
130 0.537093807480074
131 0.537706928264868
132 0.538320049049663
132 0.538933169834457
132 0.539546290619252
132 0.540159411404047
133 0.540772532188841
133 0.541385652973636
134 0.54199877375843
134 0.542611894543225
134 0.54322501532802
135 0.543838136112814
137 0.544451256897609
138 0.545064377682403
139 0.545677498467198
139 0.546290619251993
139 0.546903740036787
140 0.547516860821582
140 0.548129981606376
141 0.548743102391171
141 0.549356223175966
141 0.54996934396076
143 0.550582464745555
143 0.55119558553035
144 0.551808706315144
145 0.552421827099939
146 0.553034947884733
146 0.553648068669528
146 0.554261189454322
147 0.554874310239117
148 0.555487431023912
148 0.556100551808706
149 0.556713672593501
149 0.557326793378295
149 0.55793991416309
149 0.558553034947885
150 0.559166155732679
150 0.559779276517474
151 0.560392397302269
151 0.561005518087063
152 0.561618638871858
152 0.562231759656652
153 0.562844880441447
155 0.563458001226242
156 0.564071122011036
157 0.564684242795831
157 0.565297363580625
157 0.56591048436542
158 0.566523605150215
158 0.567136725935009
159 0.567749846719804
159 0.568362967504598
159 0.568976088289393
159 0.569589209074188
159 0.570202329858982
159 0.570815450643777
160 0.571428571428571
160 0.572041692213366
161 0.572654812998161
162 0.573267933782955
163 0.57388105456775
164 0.574494175352544
165 0.575107296137339
166 0.575720416922134
167 0.576333537706928
168 0.576946658491723
168 0.577559779276517
169 0.578172900061312
170 0.578786020846107
170 0.579399141630901
171 0.580012262415696
172 0.580625383200491
172 0.581238503985285
172 0.58185162477008
173 0.582464745554874
173 0.583077866339669
173 0.583690987124464
174 0.584304107909258
175 0.584917228694053
175 0.585530349478847
176 0.586143470263642
177 0.586756591048437
179 0.587369711833231
179 0.587982832618026
180 0.58859595340282
182 0.589209074187615
183 0.58982219497241
184 0.590435315757204
185 0.591048436541999
185 0.591661557326793
186 0.592274678111588
186 0.592887798896383
186 0.593500919681177
187 0.594114040465972
190 0.594727161250766
191 0.595340282035561
191 0.595953402820356
192 0.59656652360515
194 0.597179644389945
195 0.597792765174739
196 0.598405885959534
196 0.599019006744329
196 0.599632127529123
197 0.600245248313918
198 0.600858369098712
199 0.601471489883507
199 0.602084610668302
200 0.602697731453096
203 0.603310852237891
204 0.603923973022686
204 0.60453709380748
204 0.605150214592275
206 0.605763335377069
207 0.606376456161864
208 0.606989576946658
211 0.607602697731453
212 0.608215818516248
213 0.608828939301042
213 0.609442060085837
214 0.610055180870631
217 0.610668301655426
219 0.611281422440221
220 0.611894543225015
223 0.61250766400981
225 0.613120784794605
225 0.613733905579399
226 0.614347026364194
227 0.614960147148988
227 0.615573267933783
228 0.616186388718578
229 0.616799509503372
230 0.617412630288167
231 0.618025751072961
232 0.618638871857756
233 0.619251992642551
234 0.619865113427345
234 0.62047823421214
235 0.621091354996934
236 0.621704475781729
237 0.622317596566524
239 0.622930717351318
242 0.623543838136113
242 0.624156958920907
244 0.624770079705702
244 0.625383200490497
244 0.625996321275291
245 0.626609442060086
246 0.62722256284488
250 0.627835683629675
254 0.62844880441447
255 0.629061925199264
257 0.629675045984059
258 0.630288166768853
258 0.630901287553648
259 0.631514408338443
259 0.632127529123237
260 0.632740649908032
261 0.633353770692827
261 0.633966891477621
263 0.634580012262416
270 0.63519313304721
271 0.635806253832005
273 0.6364193746168
274 0.637032495401594
276 0.637645616186389
276 0.638258736971183
277 0.638871857755978
278 0.639484978540773
279 0.640098099325567
282 0.640711220110362
283 0.641324340895156
285 0.641937461679951
288 0.642550582464746
288 0.64316370324954
289 0.643776824034335
291 0.644389944819129
292 0.645003065603924
295 0.645616186388719
300 0.646229307173513
300 0.646842427958308
302 0.647455548743102
303 0.648068669527897
303 0.648681790312692
304 0.649294911097486
305 0.649908031882281
305 0.650521152667075
306 0.65113427345187
309 0.651747394236665
309 0.652360515021459
314 0.652973635806254
318 0.653586756591048
318 0.654199877375843
320 0.654812998160638
321 0.655426118945432
321 0.656039239730227
323 0.656652360515021
323 0.657265481299816
324 0.657878602084611
328 0.658491722869405
330 0.6591048436542
331 0.659717964438994
331 0.660331085223789
335 0.660944206008584
337 0.661557326793378
340 0.662170447578173
342 0.662783568362967
342 0.663396689147762
343 0.664009809932557
344 0.664622930717351
345 0.665236051502146
346 0.665849172286941
347 0.666462293071735
347 0.66707541385653
347 0.667688534641324
349 0.668301655426119
354 0.668914776210914
355 0.669527896995708
356 0.670141017780503
357 0.670754138565297
358 0.671367259350092
358 0.671980380134887
361 0.672593500919681
362 0.673206621704476
362 0.67381974248927
363 0.674432863274065
363 0.67504598405886
363 0.675659104843654
364 0.676272225628449
368 0.676885346413243
373 0.677498467198038
373 0.678111587982833
373 0.678724708767627
375 0.679337829552422
378 0.679950950337216
380 0.680564071122011
385 0.681177191906806
391 0.6817903126916
393 0.682403433476395
394 0.683016554261189
398 0.683629675045984
398 0.684242795830779
402 0.684855916615573
405 0.685469037400368
407 0.686082158185162
409 0.686695278969957
410 0.687308399754752
412 0.687921520539546
412 0.688534641324341
415 0.689147762109136
422 0.68976088289393
423 0.690374003678725
425 0.690987124463519
425 0.691600245248314
425 0.692213366033109
427 0.692826486817903
430 0.693439607602698
432 0.694052728387492
433 0.694665849172287
434 0.695278969957081
436 0.695892090741876
437 0.696505211526671
438 0.697118332311465
441 0.69773145309626
446 0.698344573881055
450 0.698957694665849
450 0.699570815450644
451 0.700183936235438
452 0.700797057020233
455 0.701410177805028
459 0.702023298589822
464 0.702636419374617
464 0.703249540159411
469 0.703862660944206
469 0.704475781729001
471 0.705088902513795
476 0.70570202329859
477 0.706315144083384
477 0.706928264868179
479 0.707541385652974
481 0.708154506437768
481 0.708767627222563
481 0.709380748007357
482 0.709993868792152
482 0.710606989576947
482 0.711220110361741
484 0.711833231146536
484 0.71244635193133
485 0.713059472716125
485 0.71367259350092
486 0.714285714285714
487 0.714898835070509
492 0.715511955855303
492 0.716125076640098
494 0.716738197424893
495 0.717351318209687
502 0.717964438994482
503 0.718577559779277
508 0.719190680564071
509 0.719803801348866
513 0.72041692213366
513 0.721030042918455
514 0.72164316370325
515 0.722256284488044
515 0.722869405272839
517 0.723482526057633
526 0.724095646842428
526 0.724708767627223
533 0.725321888412017
533 0.725935009196812
533 0.726548129981606
536 0.727161250766401
540 0.727774371551196
540 0.72838749233599
543 0.729000613120785
544 0.729613733905579
548 0.730226854690374
549 0.730839975475169
551 0.731453096259963
552 0.732066217044758
553 0.732679337829552
560 0.733292458614347
563 0.733905579399142
571 0.734518700183936
578 0.735131820968731
578 0.735744941753525
580 0.73635806253832
585 0.736971183323115
585 0.737584304107909
585 0.738197424892704
588 0.738810545677498
590 0.739423666462293
592 0.740036787247088
594 0.740649908031882
594 0.741263028816677
595 0.741876149601472
599 0.742489270386266
599 0.743102391171061
605 0.743715511955855
607 0.74432863274065
610 0.744941753525445
612 0.745554874310239
612 0.746167995095034
613 0.746781115879828
615 0.747394236664623
616 0.748007357449417
622 0.748620478234212
622 0.749233599019007
628 0.749846719803801
642 0.750459840588596
650 0.751072961373391
653 0.751686082158185
659 0.75229920294298
660 0.752912323727774
661 0.753525444512569
663 0.754138565297364
675 0.754751686082158
684 0.755364806866953
693 0.755977927651747
696 0.756591048436542
699 0.757204169221337
702 0.757817290006131
703 0.758430410790926
705 0.75904353157572
716 0.759656652360515
717 0.76026977314531
718 0.760882893930104
721 0.761496014714899
725 0.762109135499693
729 0.762722256284488
733 0.763335377069283
734 0.763948497854077
735 0.764561618638872
746 0.765174739423666
747 0.765787860208461
753 0.766400980993256
753 0.76701410177805
758 0.767627222562845
765 0.768240343347639
767 0.768853464132434
787 0.769466584917229
788 0.770079705702023
789 0.770692826486818
790 0.771305947271613
794 0.771919068056407
801 0.772532188841202
801 0.773145309625996
811 0.773758430410791
839 0.774371551195586
840 0.77498467198038
842 0.775597792765175
845 0.776210913549969
850 0.776824034334764
853 0.777437155119559
856 0.778050275904353
859 0.778663396689148
861 0.779276517473942
862 0.779889638258737
862 0.780502759043532
863 0.781115879828326
865 0.781729000613121
874 0.782342121397915
875 0.78295524218271
876 0.783568362967505
887 0.784181483752299
888 0.784794604537094
890 0.785407725321888
891 0.786020846106683
900 0.786633966891478
924 0.787247087676272
925 0.787860208461067
925 0.788473329245861
935 0.789086450030656
936 0.789699570815451
937 0.790312691600245
940 0.79092581238504
941 0.791538933169834
946 0.792152053954629
949 0.792765174739424
950 0.793378295524218
959 0.793991416309013
983 0.794604537093808
1014 0.795217657878602
1014 0.795830778663397
1015 0.796443899448191
1017 0.797057020232986
1034 0.797670141017781
1067 0.798283261802575
1088 0.79889638258737
1091 0.799509503372164
1097 0.800122624156959
1118 0.800735744941753
1125 0.801348865726548
1133 0.801961986511343
1146 0.802575107296137
1146 0.803188228080932
1148 0.803801348865727
1149 0.804414469650521
1150 0.805027590435316
1154 0.80564071122011
1168 0.806253832004905
1168 0.8068669527897
1169 0.807480073574494
1176 0.808093194359289
1189 0.808706315144083
1189 0.809319435928878
1189 0.809932556713673
1189 0.810545677498467
1195 0.811158798283262
1209 0.811771919068056
1216 0.812385039852851
1218 0.812998160637646
1221 0.81361128142244
1222 0.814224402207235
1252 0.814837522992029
1254 0.815450643776824
1262 0.816063764561619
1264 0.816676885346413
1264 0.817290006131208
1279 0.817903126916002
1297 0.818516247700797
1323 0.819129368485592
1328 0.819742489270386
1336 0.820355610055181
1347 0.820968730839975
1357 0.82158185162477
1360 0.822194972409565
1371 0.822808093194359
1372 0.823421213979154
1374 0.824034334763948
1392 0.824647455548743
1401 0.825260576333538
1407 0.825873697118332
1408 0.826486817903127
1408 0.827099938687922
1415 0.827713059472716
1417 0.828326180257511
1419 0.828939301042305
1433 0.8295524218271
1435 0.830165542611895
1438 0.830778663396689
1444 0.831391784181484
1453 0.832004904966278
1490 0.832618025751073
1512 0.833231146535868
1515 0.833844267320662
1526 0.834457388105457
1544 0.835070508890251
1546 0.835683629675046
1560 0.836296750459841
1574 0.836909871244635
1590 0.83752299202943
1627 0.838136112814224
1628 0.838749233599019
1648 0.839362354383814
1660 0.839975475168608
1665 0.840588595953403
1669 0.841201716738197
1715 0.841814837522992
1717 0.842427958307787
1740 0.843041079092581
1762 0.843654199877376
1774 0.84426732066217
1792 0.844880441446965
1797 0.84549356223176
1829 0.846106683016554
1895 0.846719803801349
1925 0.847332924586144
1926 0.847946045370938
1963 0.848559166155733
1980 0.849172286940527
1983 0.849785407725322
1985 0.850398528510117
1991 0.851011649294911
2078 0.851624770079706
2094 0.8522378908645
2096 0.852851011649295
2126 0.853464132434089
2145 0.854077253218884
2145 0.854690374003679
2167 0.855303494788473
2171 0.855916615573268
2183 0.856529736358063
2190 0.857142857142857
2210 0.857755977927652
2222 0.858369098712446
2228 0.858982219497241
2229 0.859595340282036
2246 0.86020846106683
2248 0.860821581851625
2248 0.861434702636419
2276 0.862047823421214
2314 0.862660944206009
2318 0.863274064990803
2324 0.863887185775598
2335 0.864500306560392
2365 0.865113427345187
2388 0.865726548129982
2394 0.866339668914776
2395 0.866952789699571
2415 0.867565910484365
2418 0.86817903126916
2469 0.868792152053955
2505 0.869405272838749
2508 0.870018393623544
2583 0.870631514408338
2608 0.871244635193133
2617 0.871857755977928
2742 0.872470876762722
2753 0.873083997547517
2771 0.873697118332311
2776 0.874310239117106
2817 0.874923359901901
2823 0.875536480686695
2841 0.87614960147149
2859 0.876762722256284
2878 0.877375843041079
2891 0.877988963825874
2899 0.878602084610668
2901 0.879215205395463
2920 0.879828326180258
2940 0.880441446965052
3006 0.881054567749847
3016 0.881667688534641
3018 0.882280809319436
3103 0.882893930104231
3136 0.883507050889025
3145 0.88412017167382
3153 0.884733292458614
3202 0.885346413243409
3214 0.885959534028204
3225 0.886572654812998
3240 0.887185775597793
3298 0.887798896382587
3357 0.888412017167382
3365 0.889025137952177
3377 0.889638258736971
3451 0.890251379521766
3467 0.89086450030656
3474 0.891477621091355
3507 0.89209074187615
3557 0.892703862660944
3595 0.893316983445739
3642 0.893930104230533
3715 0.894543225015328
3717 0.895156345800123
3774 0.895769466584917
3789 0.896382587369712
3808 0.896995708154506
3829 0.897608828939301
3879 0.898221949724096
3897 0.89883507050889
3947 0.899448191293685
3986 0.90006131207848
4042 0.900674432863274
4058 0.901287553648069
4071 0.901900674432863
4268 0.902513795217658
4269 0.903126916002452
4271 0.903740036787247
4323 0.904353157572042
4353 0.904966278356836
4368 0.905579399141631
4376 0.906192519926425
4398 0.90680564071122
4437 0.907418761496015
4443 0.908031882280809
4443 0.908645003065604
4466 0.909258123850399
4581 0.909871244635193
4585 0.910484365419988
4588 0.911097486204782
4593 0.911710606989577
4646 0.912323727774372
4708 0.912936848559166
4711 0.913549969343961
4750 0.914163090128755
4757 0.91477621091355
4779 0.915389331698345
4792 0.916002452483139
4834 0.916615573267934
4915 0.917228694052728
5003 0.917841814837523
5036 0.918454935622318
5055 0.919068056407112
5146 0.919681177191907
5160 0.920294297976701
5172 0.920907418761496
5199 0.921520539546291
5337 0.922133660331085
5372 0.92274678111588
5408 0.923359901900674
5441 0.923973022685469
5475 0.924586143470264
5571 0.925199264255058
5664 0.925812385039853
5665 0.926425505824647
5733 0.927038626609442
5759 0.927651747394237
5776 0.928264868179031
5828 0.928877988963826
5904 0.92949110974862
5955 0.930104230533415
6148 0.93071735131821
6190 0.931330472103004
6405 0.931943592887799
6646 0.932556713672594
6731 0.933169834457388
6804 0.933782955242183
7057 0.934396076026977
7193 0.935009196811772
7227 0.935622317596567
7414 0.936235438381361
7472 0.936848559166156
7544 0.93746167995095
7600 0.938074800735745
7702 0.93868792152054
7735 0.939301042305334
7787 0.939914163090129
7969 0.940527283874923
8054 0.941140404659718
8064 0.941753525444513
8375 0.942366646229307
8704 0.942979767014102
8704 0.943592887798896
8900 0.944206008583691
9085 0.944819129368486
9253 0.94543225015328
9615 0.946045370938075
9788 0.946658491722869
9900 0.947271612507664
9947 0.947884733292459
10125 0.948497854077253
10370 0.949110974862048
10568 0.949724095646842
10636 0.950337216431637
10646 0.950950337216432
10721 0.951563458001226
10894 0.952176578786021
11084 0.952789699570815
11140 0.95340282035561
11198 0.954015941140405
11352 0.954629061925199
12157 0.955242182709994
12253 0.955855303494788
13036 0.956468424279583
13063 0.957081545064378
13083 0.957694665849172
13454 0.958307786633967
13674 0.958920907418761
13941 0.959534028203556
13991 0.960147148988351
14833 0.960760269773145
15146 0.96137339055794
15157 0.961986511342734
15618 0.962599632127529
15899 0.963212752912324
16088 0.963825873697118
16448 0.964438994481913
16501 0.965052115266708
16579 0.965665236051502
16678 0.966278356836297
16794 0.966891477621091
16896 0.967504598405886
18219 0.968117719190681
19006 0.968730839975475
19142 0.96934396076027
19319 0.969957081545064
19389 0.970570202329859
20982 0.971183323114654
22081 0.971796443899448
22163 0.972409564684243
22750 0.973022685469037
22965 0.973635806253832
23038 0.974248927038627
23901 0.974862047823421
24727 0.975475168608216
25615 0.97608828939301
25736 0.976701410177805
26982 0.9773145309626
28980 0.977927651747394
30119 0.978540772532189
30493 0.979153893316983
32591 0.979767014101778
32892 0.980380134886573
33716 0.980993255671367
37101 0.981606376456162
37661 0.982219497240956
50307 0.982832618025751
55025 0.983445738810546
56567 0.98405885959534
62532 0.984671980380135
66172 0.98528510116493
69002 0.985898221949724
69281 0.986511342734519
76248 0.987124463519313
80645 0.987737584304108
81542 0.988350705088903
85430 0.988963825873697
107572 0.989576946658492
119507 0.990190067443286
127472 0.990803188228081
132721 0.991416309012876
136178 0.99202942979767
140808 0.992642550582465
146419 0.993255671367259
156109 0.993868792152054
166901 0.994481912936849
195682 0.995095033721643
216049 0.995708154506438
328097 0.996321275291232
337633 0.996934396076027
383598 0.997547516860822
394517 0.998160637645616
475152 0.998773758430411
475649 0.999386879215205
1076090 1
};
\addplot [semithick, red, dashed]
table {%
1000 -0.2
1000 2.54996934396076
};
\end{axis}

\end{tikzpicture}
}
\vspace{-0.8cm}
\caption{Observed improvement in repair rank with and without the transformer reranker.}
\label{fig:rank_cdf}
\vspace{-0.3cm}
\end{wrapfigure}

We can quantify the ranking improvement by comparing CDFs of the true repair's rank across the test set of human repairs, before and after reranking the top-$10^3$ sampled repairs (Fig.~\ref{fig:rank_cdf}). Since we decode but do not consider less probable repairs, reranking does not affect repairs originally ranked lower. Repairs initially ranked in the top-$10^3$ results by 4-gram probability tend to place between 1\textsuperscript{st} and 10\textsuperscript{th} in about 75\% of instances after reranking. It is possible the true repair can be ranked higher before reranking than after, which occurs in $\sim4\%$ of cases.

Finally, we investigate the impact of increased parallelism on repair throughput. To simulate a realistic editing scenario, we measure the end-to-end wallclock runtime required to construct the Levenshtein automaton, form the intersection regex $(S_\cap)$, decode and rank the entire intersection language $(\ell_\cap)$. Then, we swap in our GPU implementation of the algorithm described in \S~\ref{sec:implementation} as a replacement for the CPU version and compare the individual repair timings across the same test set.
\begin{wrapfigure}{l}{0.38\textwidth}
\vspace{-0.2cm}
\resizebox{.38\textwidth}{!}{% This file was created with matplot2tikz v0.3.3.
\tikzset{mark size=0.5}

\begin{tikzpicture}

\begin{axis}[
log basis y={10},
tick align=outside,
tick pos=left,
x grid style={darkdarkgray176},
xlabel={Snippet length},
xmin=0, xmax=80,
title={\textbf{Latency across repair size}},
xtick style={color=black},
y grid style={darkdarkgray176},
log basis y={10},
axis lines*=left,
ylabel={Repair latency (ms)},
ymin=6.99556128596625, ymax=147694.796423657,
ymode=log,
mark size=1.2pt,
ytick style={color=black},
yticklabels={
  \(\displaystyle {10^{0}}\),
  \(\displaystyle {10^{1}}\),
  \(\displaystyle {10^{2}}\),
  \(\displaystyle {10^{3}}\),
  \(\displaystyle {10^{4}}\),
  \(\displaystyle {10^{5}}\),
}
]
\addplot [draw=darkgray, fill=darkgray, mark=*, only marks]
table{%
x  y
9 1549
12 116
16 1976
18 288
39 1303
63 2641
11 56
73 4245
45 1416
34 706
27 1489
37 918
29 542
29 389
15 142
39 968
29 434
45 1260
19 879
61 2273
7 120
23 924
51 1570
49 1441
61 2300
42 1175
68 3944
60 2896
72 4049
53 2217
8 113
13 265
62 17742
33 3818
47 9889
41 1394
60 4760
77 4309
72 4271
27 484
75 3934
16 96
32 677
21 339
49 1836
45 1435
58 3188
18 120
30 816
15 463
19 714
49 1691
43 1476
41 831
14 192
21 182
19 10979
78 4436
52 1445
23 652
21 1557
33 974
17 118
30 933
30 374
24 541
66 25769
50 1829
47 1638
46 1360
35 4446
51 1482
59 2163
13 150
41 947
33 1189
26 312
17 182
17 178
20 253
27 1030
34 20964
26 315
8 25
20 152
53 2119
59 2414
41 26227
19 259
63 2519
63 10671
41 1815
32 423
65 4079
74 4863
36 986
9 37
46 1446
78 5210
61 4684
69 4292
14 79
45 1274
14 232
33 540
38 953
16 103
31 908
69 31228
33 709
31 13133
15 253
48 2581
66 19478
36 629
33 495
30 396
40 941
71 4825
10 983
28 426
69 3913
25 994
44 2016
7 21
50 2080
53 2054
29 382
34 763
63 2470
34 955
9 60
16 119
22 10030
40 1040
22 512
77 5203
30 616
25 242
39 958
69 5508
59 2896
21 1549
51 40442
10 134
30 722
54 29715
36 990
40 988
52 1745
10 340
48 1874
29 574
52 2219
39 794
34 732
33 516
23 470
50 15557
30 641
16 496
23 328
22 459
33 742
31 765
7 102
48 2084
45 1484
25 258
30 691
71 3900
27 548
26 353
66 3080
63 4977
40 1250
18 289
14 1780
27 2357
29 623
40 1008
22 213
64 4137
24 398
36 862
46 1424
13 113
68 4167
10 54
53 6662
40 1054
9 66
26 611
68 3685
50 4591
19 798
30 485
52 1583
12 74
34 644
77 6474
76 85265
21 429
11 170
23 287
25 496
34 1099
6 21
28 882
13 149
25 397
35 1044
23 4948
25 1657
15 243
36 848
23 237
27 1713
19 217
25 520
37 919
24 965
39 1717
15 792
49 1469
77 5034
38 3093
40 1097
51 1957
6 63
10 135
17 597
39 753
63 3249
25 447
41 2468
56 2072
74 11972
49 1742
25 494
34 93928
67 6025
64 3035
13 79
56 1714
67 3955
66 3282
63 3261
16 105
5 85
26 545
28 413
8 90
13 106
37 805
69 5137
28 369
29 673
10 44
42 1239
67 4648
27 792
51 7387
20 1096
30 585
37 790
30 462
28 986
9 44
50 1843
70 3327
52 1525
11 79
30 515
16 210
16 196
27 554
11 184
53 1947
8 67
37 929
26 971
26 367
33 648
32 616
14 134
47 1524
5 11
24 913
32 418
75 12513
35 1148
59 2608
11 157
50 2808
15 191
14 184
41 1019
10 157
71 4695
69 4455
34 540
69 4740
51 3258
44 1029
51 1498
50 1420
46 1850
40 1080
17 149
59 2862
21 14398
37 681
37 2124
11 222
23 463
51 2020
33 530
42 17922
4 13
39 698
52 2063
21 328
78 7145
28 429
19 250
7 97
43 1210
59 38172
22 362
42 2745
47 1206
28 379
62 2811
55 2179
75 4071
45 1063
37 1899
12 112
40 745
21 884
9 40
70 3981
36 888
28 392
56 5825
27 393
23 223
24 10408
18 174
23 2177
53 1662
39 1384
41 989
71 3464
60 2194
35 977
67 4303
50 9881
21 379
55 2905
58 2756
45 1769
20 609
43 38706
40 3024
32 664
23 217
18 135
65 5228
67 3550
51 1385
14 107
19 212
33 1746
75 3857
72 4455
44 1118
68 4834
8 55
39 29045
10 43
22 336
48 7343
72 19347
24 269
49 2791
19 785
39 1120
49 1322
21 760
28 521
63 62239
49 1629
23 6078
37 27981
41 1375
51 2274
5 65
17 191
62 3688
27 1016
41 1326
34 2739
33 2785
35 916
63 2554
40 1149
11 56
72 4190
18 145
62 3069
16 379
65 2780
46 1193
50 1621
76 4978
45 1264
16 382
23 655
20 292
27 705
58 2380
26 539
42 7325
15 264
55 2162
76 25267
49 2954
46 1565
55 2499
35 568
24 303
36 783
36 725
29 392
53 6744
19 185
52 1495
11 149
16 583
26 471
76 5789
51 1362
48 1514
54 2037
23 306
8 145
31 623
12 105
70 3810
21 167
11 186
43 11621
15 897
77 30152
8 219
14 2669
15 198
29 907
11 46
38 708
31 894
41 1293
30 24580
60 31750
64 3403
26 527
27 783
21 189
34 980
15 138
9 64
9 32
8 90
59 2109
15 94
33 695
17 14201
36 733
75 5222
36 1902
18 535
67 4110
29 950
35 1798
42 1463
33 766
41 1037
33 503
30 1323
21 252
27 757
32 916
37 1450
54 2323
49 1980
44 9744
26 226
40 11847
22 919
21 4868
42 1916
27 326
53 2114
50 1959
13 178
24 2361
8 92
19 298
29 454
56 3157
27 466
55 2120
71 4493
12 97
25 544
39 1089
38 1161
36 580
53 2653
16 203
10 3282
12 57
30 644
55 1814
34 17457
14 70
25 365
19 265
38 1074
16 203
23 626
46 1762
20 538
13 385
53 2187
37 1069
33 702
9 104
62 4072
32 632
47 1670
50 2049
65 3868
38 810
16 262
62 2518
51 1981
11 109
12 2936
28 520
76 11739
71 3836
49 2243
};
\addplot [draw=darkgray, fill=darkgray, mark=*, only marks]
table{%
x  y
37 967
13 576
31 725
13 82
62 3570
27 286
11 65
55 1667
29 342
18 467
27 300
16 272
57 2199
75 4119
20 205
17 366
16 125
63 4737
45 1024
70 3483
26 608
21 563
30 1598
38 1097
12 62
33 613
61 2243
19 280
27 902
35 556
63 3421
26 714
18 110
47 5579
47 1203
30 955
55 2276
36 1207
46 1206
58 2320
41 855
73 3738
30 646
73 4086
34 805
26 254
48 1192
33 562
14 286
50 1452
64 2511
31 576
28 290
73 4351
45 3090
28 614
36 667
52 2867
77 5575
31 489
50 1848
35 820
42 1054
28 397
76 4091
37 736
64 2432
52 1436
10 169
41 775
9 54
67 3335
27 451
28 303
31 427
57 1809
42 1335
18 216
45 978
39 846
12 157
42 872
20 163
17 398
19 2462
23 198
19 367
26 310
69 3317
15 249
77 4157
42 856
17 121
20 758
36 558
18 149
22 496
62 2332
24 569
29 376
17 939
40 993
24 297
8 70
18 285
57 2095
23 670
29 336
25 1004
30 396
72 3449
38 1492
58 2403
54 2802
43 1186
20 423
26 1536
15 76
5 19
78 5718
39 670
26 281
43 874
44 1863
13 67
17 471
43 1152
24 266
43 956
29 363
48 1478
39 992
16 153
36 693
57 1936
14 390
66 5587
52 2022
31 1252
30 384
20 248
20 150
42 1961
27 464
11 106
21 581
11 99
33 720
29 536
56 6622
61 2275
12 99
78 4706
9 158
78 5312
64 3852
43 935
36 573
33 1219
21 1063
13 59
19 243
23 199
12 185
29 1062
59 2021
25 290
34 585
5 19
39 974
20 186
25 260
74 4492
25 468
46 1182
28 310
44 913
38 735
23 361
31 388
25 252
77 4451
53 3401
10 67
10 110
55 2453
61 2455
71 3342
43 921
63 3298
78 4901
60 13008
30 599
68 2986
32 559
13 310
22 216
20 368
51 2238
29 691
37 608
28 320
35 797
53 3644
23 263
58 2283
24 1170
50 4202
11 130
60 2883
34 2120
71 3378
22 180
36 858
19 217
5 47
18 384
22 392
26 1020
30 430
9 75
62 2291
8 143
39 753
73 3607
57 2080
35 2589
47 1105
20 591
38 1445
38 718
31 728
60 11098
34 496
35 1681
41 781
14 208
65 5798
61 2345
18 116
17 124
21 178
8 93
39 736
49 1481
46 1083
25 554
24 347
63 2938
16 180
15 202
43 1072
11 54
22 216
36 572
27 451
18 292
27 333
37 923
54 1697
26 745
18 223
17 564
48 1236
13 102
24 313
47 1151
24 418
73 4691
22 186
50 1315
72 3757
32 846
29 851
43 1050
34 480
31 419
30 762
17 394
33 557
30 1017
32 422
20 162
16 94
33 690
70 4054
27 626
44 2331
52 1935
22 749
71 3332
20 1017
34 2256
14 107
29 559
18 146
38 761
39 1008
18 226
48 2323
60 2418
38 653
26 273
27 336
33 455
32 416
13 214
77 4543
49 2068
20 355
61 2236
55 4802
11 163
20 179
65 2701
48 6149
77 4137
51 4601
55 1865
26 273
35 4314
67 2978
25 303
24 699
76 4592
23 236
25 265
7 36
34 499
62 2235
22 344
27 787
70 3060
30 2236
20 457
36 1361
72 5525
23 1335
13 373
16 263
30 1738
33 901
9 61
71 3235
40 775
33 638
30 366
16 401
33 3066
40 807
69 3141
9 60
28 398
59 2206
72 5128
36 1173
38 755
30 396
59 9951
73 4086
18 244
19 287
31 629
58 1929
48 1854
29 543
15 574
76 4735
37 3785
28 3065
33 3925
20 315
13 128
21 779
35 586
27 461
38 728
49 2232
15 578
24 237
24 234
39 831
17 264
34 499
14 560
40 799
31 403
24 227
21 418
50 1515
61 2860
48 1200
62 2411
75 4749
35 511
39 687
51 2613
15 99
40 784
4 21
24 2146
37 665
46 1132
24 275
22 396
31 403
33 456
31 615
23 324
16 216
54 1671
34 1328
23 467
56 1767
9 181
36 560
20 191
63 2774
27 371
8 215
27 346
33 3668
41 6073
16 109
56 2557
42 854
29 923
66 7498
28 328
50 2110
17 183
73 4756
39 907
47 1142
74 4066
17 114
18 134
46 1133
24 243
45 1212
35 1622
19 718
50 1286
74 3944
52 1596
22 417
32 1180
24 807
20 175
45 1019
23 203
25 255
65 2716
37 612
40 799
9 75
36 668
16 181
39 3541
16 211
49 2661
25 495
59 1979
37 595
16 403
16 280
53 2048
62 2206
28 308
54 1583
47 1330
21 195
15 394
55 2754
37 1058
32 818
14 95
41 811
15 91
34 561
15 517
13 298
48 1138
28 458
44 1152
24 223
30 376
79 4556
35 814
43 2771
53 1486
21 202
42 848
45 995
18 115
12 170
37 616
19 546
57 1925
26 322
21 173
60 2626
51 1515
27 2046
19 366
18 332
71 3346
31 656
73 3772
44 944
70 3714
43 1159
68 3472
19 336
55 1778
24 251
45 1963
18 154
58 2220
10 220
45 1072
43 1188
33 517
40 876
16 646
48 1214
22 561
37 930
63 4137
33 1414
13 397
30 469
25 306
15 130
52 1676
51 1391
42 1326
29 596
65 2703
60 3376
71 3470
18 109
40 737
35 827
30 444
31 495
59 2695
79 4538
37 619
31 538
20 376
57 2979
23 410
29 908
28 376
36 955
13 225
43 1404
29 472
35 604
23 220
16 168
17 151
13 151
13 77
12 396
57 1895
41 860
57 2269
75 3797
22 176
21 658
27 286
19 172
54 1513
10 138
17 319
44 987
27 286
25 473
17 119
64 2514
28 447
42 1606
53 1810
14 160
33 1491
53 2038
58 2610
21 236
36 578
27 473
24 275
74 3879
79 5334
67 3619
47 2079
41 833
15 254
53 1662
26 265
20 142
22 415
52 1511
11 74
31 505
50 1314
9 219
20 178
53 1688
55 1727
25 714
21 380
45 1511
73 3864
20 169
14 178
35 2191
63 2616
29 603
49 2332
47 1248
26 793
7 19
28 335
18 416
36 1161
20 182
24 714
14 88
49 1390
37 637
66 3852
29 337
12 64
32 468
34 1461
17 104
42 912
17 406
28 464
20 661
34 1508
32 501
41 785
74 3767
17 125
40 1750
30 468
11 256
53 1588
35 716
28 523
20 132
38 1072
62 2652
45 1865
42 888
71 3634
55 1915
66 2838
39 826
66 3678
62 2941
33 862
32 1045
20 463
26 407
14 88
51 1526
32 1004
32 810
27 396
5 34
15 458
31 530
11 116
36 1260
21 204
46 1272
64 3805
53 1946
56 2260
44 981
29 394
34 764
29 503
24 267
20 357
70 5314
21 212
19 271
25 1602
36 668
54 1692
41 879
27 314
29 469
43 1248
26 288
78 7398
46 1522
42 1005
19 162
20 659
16 352
29 369
41 1444
35 584
29 493
72 4540
36 565
21 948
16 111
34 1065
59 2490
22 253
16 195
51 1527
49 1373
53 1549
36 677
64 2722
55 1864
25 269
24 230
37 748
32 447
16 151
15 122
16 872
65 3000
17 100
12 296
29 466
32 674
29 337
15 230
45 1084
35 960
79 5158
34 563
21 394
33 633
24 330
27 295
27 329
36 565
34 1367
41 802
71 4002
35 548
47 1487
18 198
29 448
57 3080
56 2030
39 706
37 669
11 104
41 877
27 318
36 1874
22 231
42 1185
8 40
42 894
32 855
64 3359
33 603
41 823
22 306
29 448
15 94
34 824
6 15
41 860
12 153
4 28
35 906
72 3895
64 2509
16 427
15 113
31 737
18 138
42 1059
11 63
12 149
5 37
65 2822
17 278
19 303
17 335
18 175
17 562
22 605
23 240
35 3819
26 259
45 997
52 1586
16 429
34 490
29 1283
22 555
33 991
34 520
70 4662
13 213
13 97
52 1514
22 210
50 1817
29 383
33 675
18 422
18 117
53 2349
24 2137
39 1467
18 157
78 5258
45 1250
29 342
10 82
28 322
24 292
71 4662
16 211
27 294
15 271
42 1225
33 532
57 1970
34 566
36 1764
34 592
32 499
9 81
72 5270
31 1433
31 725
61 2268
50 2300
19 925
32 446
8 73
71 4323
25 763
18 310
31 442
54 1689
22 248
17 298
19 148
39 715
46 1237
24 378
33 620
55 1928
33 448
54 2785
28 2556
16 114
17 143
29 414
6 15
25 820
76 4052
35 846
27 347
44 1173
73 3785
34 486
15 148
69 4745
31 1132
12 130
29 553
77 4168
47 1263
54 3108
40 2104
25 370
32 1279
52 12972
36 1281
29 417
20 184
38 1157
42 889
11 164
46 1070
47 2268
5 46
13 530
9 55
30 898
45 1442
20 179
34 2244
11 49
14 85
5 29
12 48
26 402
25 586
28 415
38 708
40 1013
24 470
26 771
28 380
36 1000
19 609
48 1182
52 1710
26 289
36 605
13 390
23 376
28 3022
20 985
26 405
42 853
20 159
11 89
15 99
60 2482
38 870
35 743
20 890
19 1640
14 92
25 295
25 913
5 41
32 472
25 233
22 446
15 232
68 3026
19 514
37 2185
78 4881
39 773
26 295
41 909
30 432
12 304
13 88
32 454
76 4302
26 819
74 6097
35 603
48 4513
21 378
40 834
29 380
66 2766
71 3677
23 1391
24 261
38 747
8 146
26 1750
39 732
22 588
9 155
33 485
23 1342
15 459
44 1375
76 4937
44 1381
15 372
34 1294
22 195
29 333
49 6540
18 163
57 2083
44 963
34 504
19 229
59 2052
34 624
35 551
59 2096
23 195
46 1312
74 3939
29 628
26 633
27 306
20 651
18 122
50 1552
43 2398
27 559
12 49
34 648
65 2751
27 443
33 518
56 1772
12 56
37 1870
36 600
29 2453
19 321
22 622
29 1666
62 2886
43 976
15 110
54 2059
53 1487
16 124
24 383
37 602
35 531
34 716
58 2505
14 164
22 323
35 664
23 400
78 5086
};
\addplot [draw=darkgray, fill=darkgray, mark=*, only marks]
table{%
x  y
19 524
42 1166
45 3390
69 18648
51 3254
32 10954
36 680
76 7230
40 2763
64 3123
35 970
75 26775
60 4154
37 3369
28 406
48 36005
70 5503
54 4443
43 4796
35 4535
43 2163
11 255
30 1756
31 701
61 13931
37 2268
76 30644
44 925
43 2764
15 1076
50 6609
60 12517
30 2429
29 1154
34 694
9 655
29 1101
31 3358
33 620
12 3279
32 5493
65 15497
26 507
29 1146
65 5376
24 858
36 1494
40 1621
26 787
71 4837
36 1460
39 1047
28 485
29 1114
25 3603
67 3174
};
\end{axis}

\end{tikzpicture}
}
\vspace{-0.7cm}
\caption{End-to-end repair timings.}
\label{fig:timings}
\vspace{-0.3cm}
\end{wrapfigure}

\noindent As shown in Fig.~\ref{fig:timings}, latency depends on various factors but supports the complexity analysis (\S~\ref{sec:method}), exhibiting a clearly superlinear but subexponential runtime profile. While real-world performance can vary based on LED, intersection volume and other load factors, the GPU runtime generally has lower variance and confers a 2-3x speedup across most common repair scenarios. Our GPU implementation is able to exhaustively decode the intersection for almost all instances before 10s, however the equivalent CPU version may struggle to meet the same latency target, especially on longer or multi-edit repairs. While the 10s timeout can be arbitrarily extended, we anticipate a much longer delay would begin to tax the patience of most users, and therefore consider it a reasonable upper bound for repair latency.

\clearpage\section{Discussion}\label{sec:discussion}

The main lesson we can draw is that it is feasible to significantly improve the precision of real-time syntax repair by incorporating syntactic constraints such as edit distance, then sampling and evaluating a large set of candidate repairs using a fast primary decoder and a more intensive secondary reranker. Though sample-efficient, transformer decoding comes at considerable cost to throughput, resulting in fewer repairs being discovered in a fixed amount of time.

Our approach uses a grammar and a high-throughput c-gram decoder to fetch the initial candidate repairs, then employs the transformer-encoder to rerank only the top-scoring repairs from the retrieved set. This allows us to repair errors in real-world programming languages and provides far more flexibility and controllability during the repair process, resulting in significantly higher precision on downstream repairs and ultimately, a smoother user experience.

Our primary insight leading to state-of-the-art precision is that repairs are typically concentrated near the center of a small Levenshtein ball, and by enumerating or sampling it carefully, then reranking repairs by naturalness, one can achieve significantly higher precision than one-shot neural repair. This is especially true for small-radii Levenshtein balls, where the admissible set is small enough to be completely enumerated and ranked. For larger radii, we can still achieve state-of-the-art precision by sampling a representative subset within a fixed timeout.

There is a clear tradeoff between latency and precision for any repair model. While existing neural syntax repair models scale poorly with additional time, Tidyparse is highly effective at exchanging more time for higher precision. We find that the Precision@10 of our method is competitive with BIFI's Precision@$2\times 10^4$, while requiring only a fraction of the inference time. Unlike neural syntax repair models, Tidyparse can sample directly from the language specification without any hallucination. The emphasis on completeness is especially useful for discovering small or contextually unlikely repairs, which are easily overlooked by neural models.

Although latency and precision are ultimately the deciding usability factors, repair throughput is a crucial intermediate factor to consider when evaluating the performance of a repair system. Even with a perfectly accurate reranker, if the correct repair is never retrieved, it will be for naught. By maximizing the total number of unique valid repairs, we increase the probability of retrieving natural repairs to give the reranker the best chance of surfacing them to the user.

One might be tempted to model syntax repair as a rejection sampling problem, but as Fig.~\ref{fig:volumetric_plot} portrays, this strategy would be mistaken. Even if checking a single repair for validity takes just 1 ms, complete enumeration could take 24+ hours, and we have mere seconds at most. While rejection sampling has lower latency to find admissible repairs, it wastes a tremendous amount of computation and scales poorly with edit distance. It is far better to spend more computation upfront by performing the intersection in a way that avoids rejection and returns natural repairs.

Likewise, methods that rely on decoding large language models appear to face a similar dilemma. As we show in Fig.~\ref{fig:len_dist_prec_all}, even if we sample thousands of repairs from BIFI, an LLM specifically trained on syntax repair, it is possible to miss natural valid repairs of a given distance that would be easily found by an extensive c-gram search of $\ell_\cap$, plus reranking. This suggests completeness may be equally, if not more important, than sample efficiency for the purposes of evaluating candidate repairs. Indeed, if we compare the 4-gram Precision@100 in Fig.~\ref{fig:adaptive} with BIFI's precision@$2\times 10^4$, 4-gram Precision@100 is highly competitive even without any post-decoder reranking.

%As shown in Fig.~\ref{fig:sankey}, about 27\% of repairs with fewer than 4 edits were outside the language intersection, which is comprised of all repairs within distance LED+1. This could be fixed by increasing the edit distance, but we find that LED+2 reduces overall repair precision.

Taken together, these results provide strong evidence to support the central claim made in the introduction (\S~\ref{sec:intro}): \textit{existing syntax repair methods simply generate far too few repairs to be effective}. By extensively generating and evaluating a large quantity of repairs within a fixed edit distance, we show it is possible to predict the author's intent far more reliably, with greater precision and lower latency than competing methods which rely solely on transformer-based neural networks.

\clearpage\subsection{Limitations and future work}

We identify four broad categories of limitations in Tidyparse and suggest directions for future work: naturalness, complexity, and toolchain integration.

\subsubsection{Naturalness}

Firstly, Tidyparse does not currently support intersections between weighted CFGs and weighted finite automata, a la Pasti et al.~\cite{pasti2023intersection}. This feature would allow us to put transition probabilities on the Levenshtein automaton corresponding to edit probability, then construct a weighted intersection grammar. With this, one could preemptively discard unlikely productions from $G_\cap$ to reduce the complexity of intersection in exchange for relaxed completeness. We also hope to explore alternate sampling strategies such as sequential Monte-Carlo~\cite{lew2023sequential} and denoising diffusion models~\cite{austin2021structured} with structured sampling priors for Levenshtein edits.

The reranker is currently evaluated over lexical tokens but we expect that a more precise ranking function could be constructed by using names and numbers from the original source code and then scoring plaintext. Furthermore, the decoder only considers each candidate repair $P_\theta(\sigma')$ in isolation, returning the most probable candidates independent of the original error. This could be improved by incorporating the broken sequence ($\err\sigma$), parser error message ($m$), original source ($s$), and possibly other contextual priors into the decoder.

\subsubsection{Complexity}

Latency can vary depending on several factors including string length, grammar size, and critically the Levenshtein edit distance. This can be an advantage because, without any contextual or statistical information, syntax and minimal Levenshtein edits are often sufficiently constrained to identify a small number of valid repairs. It is also a limitation because the admissible set expands rapidly with edit distance and the Levenshtein metric diminishes in usefulness without a very precise metric to discriminate natural solutions in the cosmos of equidistant repairs.

Space complexity increases sharply with edit distance and to a lesser extent with length. This can be partly alleviated with encoding tricks and a more efficient GPU implementation, but the memory overhead is still considerable. Memory pressure can be attributed to engineering factors such as the grammar encoding, but is also an inherent challenge of language intersection. Therefore, managing the size of the intersection grammar by preprocessing the syntax and automaton, then eliminating unnecessary productions, is a critical factor in scaling up our technique.

\subsubsection{Toolchain integration}

Program slicing is an important preprocessing consideration that has so far gone unmentioned. The current implementation expects pre-sliced code fragments, however in a more practical scenario, it would be necessary to leverage editor information to identify the boundaries of the repairable fragment. One solution would be to just use the current editor line, however a more complete solution requires careful editor integration.

Lastly and perhaps most significantly, Tidyparse does not incorporate semantic constraints, so its repairs whilst syntactically admissible, are not guaranteed to be type safe, and must be filtered by some form of compiler or incremental type checker before presenting them to the user. It may be possible to add a type-based semantic refinement to our language intersection, however this would require a more expressive grammatical formalism than CFGs naturally provide.

Extending this work to the problem of type error repair requires leaving the domain of syntax and entering the much more daunting world of semantics -- here one must contend with difficult questions in mathematical logic and finite model theory. One direction would be to collapse these problems down to automata theory using MSO over words via the B\"uchi-Elgot-Trakhtenbrot theorem. Another direction would be to increase the expressivity of the grammar, using something like conjunctive grammars~\cite{okhotin2001conjunctive}. A third approach would be to adopt the framework of contextual modal type theory, then study the behavior of Levenshtein edit distance on modal accessibility in weak substructural type systems like the Lambek calculus~\cite{pshenitsyn2025first}. We leave this for future work.

\clearpage\section{Related Work}\label{sec:related}

Our work draws a threefold correspondence between well-known techniques in (1) formal langauge theory, (2) program analysis and (3) incremental decoding. We will first survey these topics, then turn our attention to machine learning, with which we compare and partly use for reranking.

\subsection{Formal langauge theory}

Context-free language (CFL) parsing is the well-studied problem of how to turn a string into a unique tree, with many different algorithms and implementations (e.g., shift-reduce, recursive-descent, LR). Many of those algorithms expect grammars to be expressed in a certain form (e.g., left- or right- recursive) or are optimized for a narrow class of grammars (e.g., regular, linear).

General CFL parsing allows ambiguity (non-unique trees) and can be formulated as a dynamic programming problem, as shown by Cocke-Younger-Kasami (CYK)~\cite{sakai1961syntax}, Earley~\cite{earley1970efficient} and others. These parsers have roughly cubic complexity with respect to the length of the input string.

As shown by Valiant~\cite{valiant1975general}, Lee~\cite{lee2002fast} and others, general CFL recognition is in some sense equivalent to binary matrix multiplication, another well-studied combinatorial problem with broad applications, known to be at worst subcubic. This reduction opens the door to a range of complexity-theoretic speedups to CFL recognition; however large constants tend to limit their practical applicability.

Bar-Hillel~\cite{bar1961formal} proves the closure of CFLs under intersection with regular languages, but does not elaborate on how to construct the corresponding grammar. Salomaa~\cite{salomaa1973formal} and Pasti et al.~\cite{pasti2023intersection} provide helpful insights into constructing the intersection grammar, and Nederhof and Satta~\cite{nederhof2004language} specifically consider finite CFL intersections, but seem unaware of the connection to CFL reachability. Our work specializes Bar-Hillel intersections to Levenshtein automata in particular, and more generally acyclic automata using a refinement of Salomaa's construction~\cite{salomaa1973formal} via CFL reachability.

\subsection{CFL reachability}

Our contribution is closely related to the literature on CFL reachability. In brief, the CFL reachability problem seeks to determine, given an edge-labeled graph and distinguished vertex pair, $\langle v, v' \rangle$, whether there is a path, $v \rightsquigarrow v'$, whose concatenated edge labels are contained in the CFL. For a deeper overview, see Zhang and Su~\cite{zhang2017context}. This problem has been known~\cite{reps1998program} for some time~\cite{kodumal2004set} to have broad applications to program analysis and as our work finds, to syntactic program repair.

From a complexity-theoretic perspective, the CFL reachability problem is known to be at worst subcubic~\cite{chistikov2022subcubic} with polylogarithmic time factors. Koutrus and Deep~\cite{koutris2023fine} present a fine-grained complexity analysis, with concurrent work by Istomina et al.~\cite{istomina2023fine} expanding on fine-grained reductions. Muravev and Grigorev explore how to accelerate this technique on a GPU~\cite{muravev2025universal}.

Absent from the literature on CFL reachability is a discussion of the Bar-Hillel construction, regular expressions for witnessability, or the use of Brzozowski's derivative for incremental decoding. Nor does the literature specifically consider the parallel complexity of intersection nonemptiness between CFLs and acyclic automata such as the Levenshtein construction (\S~\ref{sec:repair_ex}). In Theorem~\ref{thm:parallel_decision_complexity} we give a constructive proof of intersection nonemptiness, borrowing the matrix multiplication technique from CFL reachability to build a star-free regular expression that we decode using the Brzozowski derivative. This technique sheds new light on the Bar-Hillel construction and naturally translates to a simple and efficient implementation which is fully compatible with left-to-right incremental decoding techniques used in machine learning and probabilistic language modeling.

\subsection{Language equations}

Language equations are a powerful tool for reasoning about formal languages and their inhabitants. First proposed by Ginsburg et al.~\cite{ginsburg1962two} for the ALGOL language, language equations are essentially systems of inequalities with variables representing \textit{holes}, i.e., unknown values, in the language or grammar. Solutions to these equations can be obtained using various fixpoint techniques, yielding members of the language. This insight reveals the true algebraic nature of CFLs and their cousins.

Being an algebraic formalism, language equations naturally give rise to a kind of calculus, vaguely reminiscent of Leibniz's and Newton's. First studied by Brzozowski~\cite{brzozowski1964derivatives, brzozowski1980equations} and Antimirov~\cite{antimirov1996partial}, one can take the derivative of a language equation, which can be interpreted as a kind of continuation or language quotient, revealing the suffixes that complete a given prefix. This technique leads to an elegant family of algorithms for incremental parsing~\cite{might2011parsing, adams2016complexity} and regular expression matching~\cite{stanford2021symbolic,varatalu2025re}.

%From a more applied perspective, parsers are ubiquitous in software engineering, but none are designed to handle arbitrary CFGs or recover from arbitrary errors. Parr and Quong introduce ANTLR~\cite{parr1995antlr} which can handle LL(k) grammars and offers an IDE plugin with limited support for error recovery. Scott and Johnstone~\cite{scott2010gll} introduce GLL parsing, which supports linear-time parsing for LL grammars and cubic for arbitrary CFGs, but has limited support for error recovery. Inspired by their work, we introduce a method for repairing small syntax errors in arbitrary CFLs.

More concretely, we restrict our attention to language equations over CFLs whose variables coincide with edit locations in the source code of a computer program, and solutions correspond to syntax repairs. While prior work has studied the use of language equations for parsing~\cite{might2011parsing}, to our knowledge, they were never specifically considered for code completion or syntax error correction.

\subsection{Syntax repair}

In finite languages, syntax repair corresponds to spelling correction, a more restrictive and largely solved problem. Schulz and Stoyan~\cite{schulz2002fast} construct a finite automaton that returns the nearest dictionary entry by Levenshtein edit distance. Though considerably simpler than syntax correction, their work shares similar challenges and offers insights for handling more general repair scenarios.

When a sentence is grammatically invalid, parsing grows more challenging. Like spelling, the problem is to find the minimum number of edits required to transform an arbitrary string into a syntactically valid one, where validity is defined as containment in a (typically) context-free language. Early work, including Irons~\cite{irons1963error} and Aho~\cite{aho1972minimum} propose a dynamic programming algorithm to compute the minimum number of edits required to fix an invalid string. Prior work on error correcting parsing only considers the nearest edit(s), and does not study edits of varying distance in the Levenshtein ball. Furthermore, the problem of repair is not generally well-posed, as there can be many valid solutions. We instead focus on maximum probability Levenshtein-CFL reachability, which attempts to find the most natural repair within a fixed Levenshtein distance.

Diekmann and Tratt~\cite{diekmann2018dont} present a rule-based syntax repair tool that retrieves the complete set of minimum-cost repairs, but only works for deterministic CFLs, a proper subset of the CFL family which admit a linear-time parser. Their cost model is based on insertion and deletion, and does not consider probability or non-minimal edit distance. Tidyparse can handle arbitrary CFLs and generate repairs within an arbitrary edit distance, using a Levenshtein cost model.

Zhang et al.~\cite{zhang2023ordinalfix} introduce OrdinalFix, which uses CFL reachability to repair compiler errors, however their method only returns admissible repairs and not necessarily probable ones. As they do not consider the problem of maximum-probability repairs, nor use any form of ranking to sort the results by naturalness or probability, we do not compare with their work.

%\subsection{String solving}
%
%There is related work on string constraints in the constraint programming literature, featuring solvers like CFGAnalyzer and HAMPI~\cite{kiezun2009hampi}, which consider bounded context free grammars and intersections thereof. Boja{\'n}czyk et al. (2014)~\cite{bojanczyk2014automata} introduce the theory of nominal automata. Around the same time, D'Antoni et al. (2014) introduce \textit{symbolic automata}~\cite{dantoni2014minimization}, a generalization of finite automata which allow infinite alphabets and symbolic expressions over them. Hague et al. (2024)~\cite{hague2024parikh} use Parikh's theorem in the context of symbolic automata to speed up string constraint solving. In none of the constraint programming literature we surveyed do any of the approaches specifically consider the problem of syntax error correction, which is the core focus of our work.

%\subsection{Error correcting codes}
%
%Our work focuses on errors arising from human factors in computer programming, in particular \textit{syntax error correction}, which is the problem of fixing partially corrupted programs. Modern research on error correction, however, can be traced back to the early days of coding theory when researchers designed \textit{error-correcting codes} (ECCs) to denoise transmission errors induced by external interference, e.g., collision with a high-energy proton, manipulation by an adversary or even typographical mistake. In this context, \textit{code} can be any logical representation for communicating information between two parties (such as a human and a computer), and an ECC is a carefully-designed scheme which ensures that even if some portion of the message should become corrupted, one can still recover the original message by solving a linear system of equations. When designing ECCs, one typically assumes a noise model over a certain sample space, such as the Hamming~\cite{titsias2017hamming} or Levenshtein~\cite{levenshtein1966binary, becerra2008learning, barlev2021levenshtein} balls, from which we draw inspiration for this work.

\subsection{Decoding}\label{sec:decoding}

Decoding is a key problem in machine translation, speech recognition, and other sequence-to-sequence tasks. Given a compressed encoding of some finite distribution, the goal is to find the maximum probability samples. A classic example is Viterbi decoding, which is used to find the most likely path through a hidden Markov model (HMM), a kind of weighted automaton.

In particular, we care about the problem of \textit{top-k decoding}, which attempts to find the exact or approximate $k$-most likely samples in order of decreasing likelihood. This is closely related to the $k$-best enumeration~\cite{eppstein2014k} problem, a carefully studied problem in graph theory and combinatorial optimization. An exact solution to this problem for large acyclic CFGs is often intractable, but we can approximate it using a beam search over c-grams, then rerank top scoring results.

A popular solution to k-best decoding in the NLP literature is a technique called cube-pruning~\cite{huang2005better, chiang2007hierarchical}, which samples maximum probability paths through a hypergraph. We take inspiration from this technique and adapt it to the setting of constrained decoding from finite CFGs.

An alternate line of work originates from combinatorics~\cite{hickey1983uniform} and Boltzmann sampling~\cite{duchon2004boltzmann}, which constructs a generating function for the language and samples it uniformly. This technique has applications to constraint satisfaction and model counting problems in formal languages.

A third approach would be to use some form of constrained decoding~\cite{willard2023efficient, ugare2024improving, loula2025syntactic} such as sequential Monte Carlo to steer an autoregressive LLM, as proposed by Lew et al.~\cite{lew2023sequential}. These techniques show promise for program repair, however, the question of whether to use left-to-right decoding or some other strategy is still unresolved in the large language modeling community. For example, there is an emerging class of flow-based or structured denoising diffusion models~\cite{austin2021structured} which starts from a noise distribution and iteratively denoises it by sampling one or more edits at random locations with each decoding step. Typical work focuses on audiovisual data, but very recent work by Havasi et al.~\cite{havasi2025edit} adapt this to the Levenshtein edit model for generating source code. Although these models do not yet use CFGs or consider language intersections, they are inherently more fault-tolerant than decoders which require expensive backtracking-style search.

\subsection{Gradient-based program repair}

The last decade has seen a surge of progress in programming with large language models. That work is primarily based on methods from differential calculus and continuous optimization, leading to the so-called \textit{naturalness hypothesis}~\cite{allamanis2018survey}, which suggests programming languages are not so different from natural ones. In contrast, PL theory takes the view that languages are essentially discrete sets governed by logical calculi. Programming, thus viewed, is more like a mathematical exercise in constraint satisfaction. These two approaches have more in common than would seem.

A number of approximate repair techniques have been introduced using neural models to predict the most likely repair~\cite{allamanis2021self, chirkova2021empirical, drain2021generating}. These approaches typically employ large language models (LLMs) and treat the problem as a sequence-to-sequence transformation. While capable of generating natural repairs, these models are susceptible to misgeneralization, costly to train, and challenging to customize thereafter. Furthermore, the generated repairs are not necessarily sound without additional filtering, and we observe the released models often hallucinate false positive repairs.

Two prior works specifically address syntax repair, Break-It-Fix-It (BIFI)~\cite{yasunaga2021break} and Seq2Parse~\cite{sakkas2022seq2parse}. BIFI adapts techniques from semi-supervised learning to generate synthetic errors in clean code and fixes them. This reduces the need for pairwise training data, but generalizes poorly to lengthy or out-of-distribution repairs. Seq2Parse combines a transformer-based model with an augmented version of the Early parser to suggest error rules, but only suggests a single repair.

Recent work by Merrill et al.~\cite{merrill2022saturated} and Chiang et al.~\cite{chiang2023tighter} suggest that the issue with generalization may be more foundational: transformer-based language models, a popular class of neural language models used in probabilistic program repair, are fundamentally less expressive than context-free grammars, which formally describe the syntax of most programming languages. This suggests such models, despite their useful approximation properties, are ill-suited for the task of end-to-end syntax repair. Yet, as our work demonstrates, they can be useful for resolving ambiguity between valid repairs of differing probability or reranking a set of repair candidates drawn from a CFL.

Using the RASP model from Weiss et al.~\cite{weiss2021thinking}, Yang et al.~\cite{yang2024masked} characterize the expressive power of hard-attention transformers in terms of star-free regular expressions. While their work uses formal language theory to investigate language learnability and structural priors in transformers, it shares fruitful connections to CFL reachability, which similar to RASP, treats matrix multiplication as a kind of programmable interface into which various state tracking problems and static analysis tasks can be compiled and analyzed in a common linear algebraic framework.

\clearpage\section{Conclusion}\label{sec:conclusion}

Our work, while a case study on syntax repair, is part of a broader line of inquiry in program synthesis that investigates how to weave formal language theory and machine learning into helpful programming tools for everyday developers. In some ways, syntax repair serves as a test bench for integrating learning and language theory, as it lacks the intricacies of type-checking and semantic analysis, but is still rich enough to be an interesting challenge. By starting with syntax repair, we hope to lay the foundation for more organic hybrid approaches to program synthesis.

Two high-level codesign patterns have emerged to combine the naturalness of neural language models with the precision of formal methods. One seeks to filter the outputs of a generative language model to satisfy a formal specification, typically by some form of rejection sampling. Alternatively, some attempt to use language models to steer an incremental search for valid programs via a reinforcement learning or hybrid neurosymbolic approach. However, implementing these strategies is often painstaking and their generalization behavior can be difficult to analyze.

In our work, we take a more pragmatic tack - by incorporating the distance metric into a formal language, we attempt to exhaustively enumerate repairs by increasing distance, then use the stochastic language model to sort the resulting solutions by naturalness. The more constraints we can incorporate into formal language, the more efficient sampling becomes, and the more precise control we have over the output. This reduces the need for training a large, expensive language model to relearn syntax, and allows us to leverage compute for more efficient search and ranking.

There is a delicate balance in formal methods between soundness and completeness. Often these two seem at odds because the target language is too expressive to achieve them both simultaneously. In syntax repair, we also care about \textit{naturalness}. Fortunately, syntax repair is tractable enough to achieve all three by modeling the problem using language intersection. Completeness helps us to avoid missing simple repairs that might be easily overlooked, soundness guarantees all repairs will be valid, and naturalness ensures the most likely repairs receive the highest priority.

We have implemented our approach and demonstrated its viability as a tool for syntax assistance in real-world programming languages. Tidyparse is capable of generating repairs for invalid source code in a range of practical languages. We plan to continue expanding the prototype's autocorrection functionality to cover an even broader range of real-world programming languages. We envision a few primary use cases for it: (1) helping novice programmers become more quickly familiar with a new programming language, (2) autocorrecting common typos among proficient but forgetful programmers, (3) as a prototyping tool for PL designers and educators, and (4) as a pluggable library or service for parser-generators and language servers.

\section*{Data-Availability Statement}

An artifact for Tidyparse is currently available as a browser application,~\footnote{\url{https://tidyparse.github.io/python}} supporting single-line syntax repairs in the Python language. The data, source code, and models necessary for reproducing the full experiments contained in this paper will be provided, contingent upon artifact review.

\clearpage\bibliography{../bib/acmart}\vspace{-1cm}

\pagebreak\appendix

\section{Levenshtein automata matrices}

These are useful for visually checking different implementations.

\begin{figure}[H]
\begin{center}
\adjustbox{valign=m}{\includegraphics[width=4cm]{figures/lev_nfa_6x1.pdf}}
\adjustbox{valign=m}{\resizebox{0.4\textwidth}{!}{\begin{tikzpicture}[x=0.3cm, y=0.3cm, draw=gray, very thin]
  \path[fill=white] (0,13) rectangle ++(1,1);
  \path[fill=black] (1,13) rectangle ++(1,1);
  \path[fill=black] (2,13) rectangle ++(1,1);
  \path[fill=black] (3,13) rectangle ++(1,1);
  \path[fill=white] (4,13) rectangle ++(1,1);
  \path[fill=black] (5,13) rectangle ++(1,1);
  \path[fill=white] (6,13) rectangle ++(1,1);
  \path[fill=white] (7,13) rectangle ++(1,1);
  \path[fill=white] (8,13) rectangle ++(1,1);
  \path[fill=white] (9,13) rectangle ++(1,1);
  \path[fill=white] (10,13) rectangle ++(1,1);
  \path[fill=white] (11,13) rectangle ++(1,1);
  \path[fill=white] (12,13) rectangle ++(1,1);
  \path[fill=white] (13,13) rectangle ++(1,1);
  \path[fill=white] (0,12) rectangle ++(1,1);
  \path[fill=white] (1,12) rectangle ++(1,1);
  \path[fill=white] (2,12) rectangle ++(1,1);
  \path[fill=black] (3,12) rectangle ++(1,1);
  \path[fill=white] (4,12) rectangle ++(1,1);
  \path[fill=white] (5,12) rectangle ++(1,1);
  \path[fill=white] (6,12) rectangle ++(1,1);
  \path[fill=white] (7,12) rectangle ++(1,1);
  \path[fill=white] (8,12) rectangle ++(1,1);
  \path[fill=white] (9,12) rectangle ++(1,1);
  \path[fill=white] (10,12) rectangle ++(1,1);
  \path[fill=white] (11,12) rectangle ++(1,1);
  \path[fill=white] (12,12) rectangle ++(1,1);
  \path[fill=white] (13,12) rectangle ++(1,1);
  \path[fill=white] (0,11) rectangle ++(1,1);
  \path[fill=white] (1,11) rectangle ++(1,1);
  \path[fill=white] (2,11) rectangle ++(1,1);
  \path[fill=black] (3,11) rectangle ++(1,1);
  \path[fill=black] (4,11) rectangle ++(1,1);
  \path[fill=black] (5,11) rectangle ++(1,1);
  \path[fill=white] (6,11) rectangle ++(1,1);
  \path[fill=black] (7,11) rectangle ++(1,1);
  \path[fill=white] (8,11) rectangle ++(1,1);
  \path[fill=white] (9,11) rectangle ++(1,1);
  \path[fill=white] (10,11) rectangle ++(1,1);
  \path[fill=white] (11,11) rectangle ++(1,1);
  \path[fill=white] (12,11) rectangle ++(1,1);
  \path[fill=white] (13,11) rectangle ++(1,1);
  \path[fill=white] (0,10) rectangle ++(1,1);
  \path[fill=white] (1,10) rectangle ++(1,1);
  \path[fill=white] (2,10) rectangle ++(1,1);
  \path[fill=white] (3,10) rectangle ++(1,1);
  \path[fill=white] (4,10) rectangle ++(1,1);
  \path[fill=black] (5,10) rectangle ++(1,1);
  \path[fill=white] (6,10) rectangle ++(1,1);
  \path[fill=white] (7,10) rectangle ++(1,1);
  \path[fill=white] (8,10) rectangle ++(1,1);
  \path[fill=white] (9,10) rectangle ++(1,1);
  \path[fill=white] (10,10) rectangle ++(1,1);
  \path[fill=white] (11,10) rectangle ++(1,1);
  \path[fill=white] (12,10) rectangle ++(1,1);
  \path[fill=white] (13,10) rectangle ++(1,1);
  \path[fill=white] (0,9) rectangle ++(1,1);
  \path[fill=white] (1,9) rectangle ++(1,1);
  \path[fill=white] (2,9) rectangle ++(1,1);
  \path[fill=white] (3,9) rectangle ++(1,1);
  \path[fill=white] (4,9) rectangle ++(1,1);
  \path[fill=black] (5,9) rectangle ++(1,1);
  \path[fill=black] (6,9) rectangle ++(1,1);
  \path[fill=black] (7,9) rectangle ++(1,1);
  \path[fill=white] (8,9) rectangle ++(1,1);
  \path[fill=black] (9,9) rectangle ++(1,1);
  \path[fill=white] (10,9) rectangle ++(1,1);
  \path[fill=white] (11,9) rectangle ++(1,1);
  \path[fill=white] (12,9) rectangle ++(1,1);
  \path[fill=white] (13,9) rectangle ++(1,1);
  \path[fill=white] (0,8) rectangle ++(1,1);
  \path[fill=white] (1,8) rectangle ++(1,1);
  \path[fill=white] (2,8) rectangle ++(1,1);
  \path[fill=white] (3,8) rectangle ++(1,1);
  \path[fill=white] (4,8) rectangle ++(1,1);
  \path[fill=white] (5,8) rectangle ++(1,1);
  \path[fill=white] (6,8) rectangle ++(1,1);
  \path[fill=black] (7,8) rectangle ++(1,1);
  \path[fill=white] (8,8) rectangle ++(1,1);
  \path[fill=white] (9,8) rectangle ++(1,1);
  \path[fill=white] (10,8) rectangle ++(1,1);
  \path[fill=white] (11,8) rectangle ++(1,1);
  \path[fill=white] (12,8) rectangle ++(1,1);
  \path[fill=white] (13,8) rectangle ++(1,1);
  \path[fill=white] (0,7) rectangle ++(1,1);
  \path[fill=white] (1,7) rectangle ++(1,1);
  \path[fill=white] (2,7) rectangle ++(1,1);
  \path[fill=white] (3,7) rectangle ++(1,1);
  \path[fill=white] (4,7) rectangle ++(1,1);
  \path[fill=white] (5,7) rectangle ++(1,1);
  \path[fill=white] (6,7) rectangle ++(1,1);
  \path[fill=black] (7,7) rectangle ++(1,1);
  \path[fill=black] (8,7) rectangle ++(1,1);
  \path[fill=black] (9,7) rectangle ++(1,1);
  \path[fill=white] (10,7) rectangle ++(1,1);
  \path[fill=black] (11,7) rectangle ++(1,1);
  \path[fill=white] (12,7) rectangle ++(1,1);
  \path[fill=white] (13,7) rectangle ++(1,1);
  \path[fill=white] (0,6) rectangle ++(1,1);
  \path[fill=white] (1,6) rectangle ++(1,1);
  \path[fill=white] (2,6) rectangle ++(1,1);
  \path[fill=white] (3,6) rectangle ++(1,1);
  \path[fill=white] (4,6) rectangle ++(1,1);
  \path[fill=white] (5,6) rectangle ++(1,1);
  \path[fill=white] (6,6) rectangle ++(1,1);
  \path[fill=white] (7,6) rectangle ++(1,1);
  \path[fill=white] (8,6) rectangle ++(1,1);
  \path[fill=black] (9,6) rectangle ++(1,1);
  \path[fill=white] (10,6) rectangle ++(1,1);
  \path[fill=white] (11,6) rectangle ++(1,1);
  \path[fill=white] (12,6) rectangle ++(1,1);
  \path[fill=white] (13,6) rectangle ++(1,1);
  \path[fill=white] (0,5) rectangle ++(1,1);
  \path[fill=white] (1,5) rectangle ++(1,1);
  \path[fill=white] (2,5) rectangle ++(1,1);
  \path[fill=white] (3,5) rectangle ++(1,1);
  \path[fill=white] (4,5) rectangle ++(1,1);
  \path[fill=white] (5,5) rectangle ++(1,1);
  \path[fill=white] (6,5) rectangle ++(1,1);
  \path[fill=white] (7,5) rectangle ++(1,1);
  \path[fill=white] (8,5) rectangle ++(1,1);
  \path[fill=black] (9,5) rectangle ++(1,1);
  \path[fill=black] (10,5) rectangle ++(1,1);
  \path[fill=black] (11,5) rectangle ++(1,1);
  \path[fill=white] (12,5) rectangle ++(1,1);
  \path[fill=black] (13,5) rectangle ++(1,1);
  \path[fill=white] (0,4) rectangle ++(1,1);
  \path[fill=white] (1,4) rectangle ++(1,1);
  \path[fill=white] (2,4) rectangle ++(1,1);
  \path[fill=white] (3,4) rectangle ++(1,1);
  \path[fill=white] (4,4) rectangle ++(1,1);
  \path[fill=white] (5,4) rectangle ++(1,1);
  \path[fill=white] (6,4) rectangle ++(1,1);
  \path[fill=white] (7,4) rectangle ++(1,1);
  \path[fill=white] (8,4) rectangle ++(1,1);
  \path[fill=white] (9,4) rectangle ++(1,1);
  \path[fill=white] (10,4) rectangle ++(1,1);
  \path[fill=black] (11,4) rectangle ++(1,1);
  \path[fill=white] (12,4) rectangle ++(1,1);
  \path[fill=white] (13,4) rectangle ++(1,1);
  \path[fill=white] (0,3) rectangle ++(1,1);
  \path[fill=white] (1,3) rectangle ++(1,1);
  \path[fill=white] (2,3) rectangle ++(1,1);
  \path[fill=white] (3,3) rectangle ++(1,1);
  \path[fill=white] (4,3) rectangle ++(1,1);
  \path[fill=white] (5,3) rectangle ++(1,1);
  \path[fill=white] (6,3) rectangle ++(1,1);
  \path[fill=white] (7,3) rectangle ++(1,1);
  \path[fill=white] (8,3) rectangle ++(1,1);
  \path[fill=white] (9,3) rectangle ++(1,1);
  \path[fill=white] (10,3) rectangle ++(1,1);
  \path[fill=black] (11,3) rectangle ++(1,1);
  \path[fill=black] (12,3) rectangle ++(1,1);
  \path[fill=black] (13,3) rectangle ++(1,1);
  \path[fill=white] (0,2) rectangle ++(1,1);
  \path[fill=white] (1,2) rectangle ++(1,1);
  \path[fill=white] (2,2) rectangle ++(1,1);
  \path[fill=white] (3,2) rectangle ++(1,1);
  \path[fill=white] (4,2) rectangle ++(1,1);
  \path[fill=white] (5,2) rectangle ++(1,1);
  \path[fill=white] (6,2) rectangle ++(1,1);
  \path[fill=white] (7,2) rectangle ++(1,1);
  \path[fill=white] (8,2) rectangle ++(1,1);
  \path[fill=white] (9,2) rectangle ++(1,1);
  \path[fill=white] (10,2) rectangle ++(1,1);
  \path[fill=white] (11,2) rectangle ++(1,1);
  \path[fill=white] (12,2) rectangle ++(1,1);
  \path[fill=black] (13,2) rectangle ++(1,1);
  \path[fill=white] (0,1) rectangle ++(1,1);
  \path[fill=white] (1,1) rectangle ++(1,1);
  \path[fill=white] (2,1) rectangle ++(1,1);
  \path[fill=white] (3,1) rectangle ++(1,1);
  \path[fill=white] (4,1) rectangle ++(1,1);
  \path[fill=white] (5,1) rectangle ++(1,1);
  \path[fill=white] (6,1) rectangle ++(1,1);
  \path[fill=white] (7,1) rectangle ++(1,1);
  \path[fill=white] (8,1) rectangle ++(1,1);
  \path[fill=white] (9,1) rectangle ++(1,1);
  \path[fill=white] (10,1) rectangle ++(1,1);
  \path[fill=white] (11,1) rectangle ++(1,1);
  \path[fill=white] (12,1) rectangle ++(1,1);
  \path[fill=black] (13,1) rectangle ++(1,1);
  \path[fill=white] (0,0) rectangle ++(1,1);
  \path[fill=white] (1,0) rectangle ++(1,1);
  \path[fill=white] (2,0) rectangle ++(1,1);
  \path[fill=white] (3,0) rectangle ++(1,1);
  \path[fill=white] (4,0) rectangle ++(1,1);
  \path[fill=white] (5,0) rectangle ++(1,1);
  \path[fill=white] (6,0) rectangle ++(1,1);
  \path[fill=white] (7,0) rectangle ++(1,1);
  \path[fill=white] (8,0) rectangle ++(1,1);
  \path[fill=white] (9,0) rectangle ++(1,1);
  \path[fill=white] (10,0) rectangle ++(1,1);
  \path[fill=white] (11,0) rectangle ++(1,1);
  \path[fill=white] (12,0) rectangle ++(1,1);
  \path[fill=white] (13,0) rectangle ++(1,1);
\end{tikzpicture}

\begin{tikzpicture}[x=0.3cm, y=0.3cm, draw=gray, very thin]
  \path[fill=white] (0,13) rectangle ++(1,1);
  \path[fill=black] (1,13) rectangle ++(1,1);
  \path[fill=black] (2,13) rectangle ++(1,1);
  \path[fill=black] (3,13) rectangle ++(1,1);
  \path[fill=black] (4,13) rectangle ++(1,1);
  \path[fill=black] (5,13) rectangle ++(1,1);
  \path[fill=black] (6,13) rectangle ++(1,1);
  \path[fill=black] (7,13) rectangle ++(1,1);
  \path[fill=black] (8,13) rectangle ++(1,1);
  \path[fill=black] (9,13) rectangle ++(1,1);
  \path[fill=black] (10,13) rectangle ++(1,1);
  \path[fill=black] (11,13) rectangle ++(1,1);
  \path[fill=black] (12,13) rectangle ++(1,1);
  \path[fill=black] (13,13) rectangle ++(1,1);
  \path[fill=white] (0,12) rectangle ++(1,1);
  \path[fill=white] (1,12) rectangle ++(1,1);
  \path[fill=white] (2,12) rectangle ++(1,1);
  \path[fill=black] (3,12) rectangle ++(1,1);
  \path[fill=white] (4,12) rectangle ++(1,1);
  \path[fill=black] (5,12) rectangle ++(1,1);
  \path[fill=white] (6,12) rectangle ++(1,1);
  \path[fill=black] (7,12) rectangle ++(1,1);
  \path[fill=white] (8,12) rectangle ++(1,1);
  \path[fill=black] (9,12) rectangle ++(1,1);
  \path[fill=white] (10,12) rectangle ++(1,1);
  \path[fill=black] (11,12) rectangle ++(1,1);
  \path[fill=white] (12,12) rectangle ++(1,1);
  \path[fill=black] (13,12) rectangle ++(1,1);
  \path[fill=white] (0,11) rectangle ++(1,1);
  \path[fill=white] (1,11) rectangle ++(1,1);
  \path[fill=white] (2,11) rectangle ++(1,1);
  \path[fill=black] (3,11) rectangle ++(1,1);
  \path[fill=black] (4,11) rectangle ++(1,1);
  \path[fill=black] (5,11) rectangle ++(1,1);
  \path[fill=black] (6,11) rectangle ++(1,1);
  \path[fill=black] (7,11) rectangle ++(1,1);
  \path[fill=black] (8,11) rectangle ++(1,1);
  \path[fill=black] (9,11) rectangle ++(1,1);
  \path[fill=black] (10,11) rectangle ++(1,1);
  \path[fill=black] (11,11) rectangle ++(1,1);
  \path[fill=black] (12,11) rectangle ++(1,1);
  \path[fill=black] (13,11) rectangle ++(1,1);
  \path[fill=white] (0,10) rectangle ++(1,1);
  \path[fill=white] (1,10) rectangle ++(1,1);
  \path[fill=white] (2,10) rectangle ++(1,1);
  \path[fill=white] (3,10) rectangle ++(1,1);
  \path[fill=white] (4,10) rectangle ++(1,1);
  \path[fill=black] (5,10) rectangle ++(1,1);
  \path[fill=white] (6,10) rectangle ++(1,1);
  \path[fill=black] (7,10) rectangle ++(1,1);
  \path[fill=white] (8,10) rectangle ++(1,1);
  \path[fill=black] (9,10) rectangle ++(1,1);
  \path[fill=white] (10,10) rectangle ++(1,1);
  \path[fill=black] (11,10) rectangle ++(1,1);
  \path[fill=white] (12,10) rectangle ++(1,1);
  \path[fill=black] (13,10) rectangle ++(1,1);
  \path[fill=white] (0,9) rectangle ++(1,1);
  \path[fill=white] (1,9) rectangle ++(1,1);
  \path[fill=white] (2,9) rectangle ++(1,1);
  \path[fill=white] (3,9) rectangle ++(1,1);
  \path[fill=white] (4,9) rectangle ++(1,1);
  \path[fill=black] (5,9) rectangle ++(1,1);
  \path[fill=black] (6,9) rectangle ++(1,1);
  \path[fill=black] (7,9) rectangle ++(1,1);
  \path[fill=black] (8,9) rectangle ++(1,1);
  \path[fill=black] (9,9) rectangle ++(1,1);
  \path[fill=black] (10,9) rectangle ++(1,1);
  \path[fill=black] (11,9) rectangle ++(1,1);
  \path[fill=black] (12,9) rectangle ++(1,1);
  \path[fill=black] (13,9) rectangle ++(1,1);
  \path[fill=white] (0,8) rectangle ++(1,1);
  \path[fill=white] (1,8) rectangle ++(1,1);
  \path[fill=white] (2,8) rectangle ++(1,1);
  \path[fill=white] (3,8) rectangle ++(1,1);
  \path[fill=white] (4,8) rectangle ++(1,1);
  \path[fill=white] (5,8) rectangle ++(1,1);
  \path[fill=white] (6,8) rectangle ++(1,1);
  \path[fill=black] (7,8) rectangle ++(1,1);
  \path[fill=white] (8,8) rectangle ++(1,1);
  \path[fill=black] (9,8) rectangle ++(1,1);
  \path[fill=white] (10,8) rectangle ++(1,1);
  \path[fill=black] (11,8) rectangle ++(1,1);
  \path[fill=white] (12,8) rectangle ++(1,1);
  \path[fill=black] (13,8) rectangle ++(1,1);
  \path[fill=white] (0,7) rectangle ++(1,1);
  \path[fill=white] (1,7) rectangle ++(1,1);
  \path[fill=white] (2,7) rectangle ++(1,1);
  \path[fill=white] (3,7) rectangle ++(1,1);
  \path[fill=white] (4,7) rectangle ++(1,1);
  \path[fill=white] (5,7) rectangle ++(1,1);
  \path[fill=white] (6,7) rectangle ++(1,1);
  \path[fill=black] (7,7) rectangle ++(1,1);
  \path[fill=black] (8,7) rectangle ++(1,1);
  \path[fill=black] (9,7) rectangle ++(1,1);
  \path[fill=black] (10,7) rectangle ++(1,1);
  \path[fill=black] (11,7) rectangle ++(1,1);
  \path[fill=black] (12,7) rectangle ++(1,1);
  \path[fill=black] (13,7) rectangle ++(1,1);
  \path[fill=white] (0,6) rectangle ++(1,1);
  \path[fill=white] (1,6) rectangle ++(1,1);
  \path[fill=white] (2,6) rectangle ++(1,1);
  \path[fill=white] (3,6) rectangle ++(1,1);
  \path[fill=white] (4,6) rectangle ++(1,1);
  \path[fill=white] (5,6) rectangle ++(1,1);
  \path[fill=white] (6,6) rectangle ++(1,1);
  \path[fill=white] (7,6) rectangle ++(1,1);
  \path[fill=white] (8,6) rectangle ++(1,1);
  \path[fill=black] (9,6) rectangle ++(1,1);
  \path[fill=white] (10,6) rectangle ++(1,1);
  \path[fill=black] (11,6) rectangle ++(1,1);
  \path[fill=white] (12,6) rectangle ++(1,1);
  \path[fill=black] (13,6) rectangle ++(1,1);
  \path[fill=white] (0,5) rectangle ++(1,1);
  \path[fill=white] (1,5) rectangle ++(1,1);
  \path[fill=white] (2,5) rectangle ++(1,1);
  \path[fill=white] (3,5) rectangle ++(1,1);
  \path[fill=white] (4,5) rectangle ++(1,1);
  \path[fill=white] (5,5) rectangle ++(1,1);
  \path[fill=white] (6,5) rectangle ++(1,1);
  \path[fill=white] (7,5) rectangle ++(1,1);
  \path[fill=white] (8,5) rectangle ++(1,1);
  \path[fill=black] (9,5) rectangle ++(1,1);
  \path[fill=black] (10,5) rectangle ++(1,1);
  \path[fill=black] (11,5) rectangle ++(1,1);
  \path[fill=black] (12,5) rectangle ++(1,1);
  \path[fill=black] (13,5) rectangle ++(1,1);
  \path[fill=white] (0,4) rectangle ++(1,1);
  \path[fill=white] (1,4) rectangle ++(1,1);
  \path[fill=white] (2,4) rectangle ++(1,1);
  \path[fill=white] (3,4) rectangle ++(1,1);
  \path[fill=white] (4,4) rectangle ++(1,1);
  \path[fill=white] (5,4) rectangle ++(1,1);
  \path[fill=white] (6,4) rectangle ++(1,1);
  \path[fill=white] (7,4) rectangle ++(1,1);
  \path[fill=white] (8,4) rectangle ++(1,1);
  \path[fill=white] (9,4) rectangle ++(1,1);
  \path[fill=white] (10,4) rectangle ++(1,1);
  \path[fill=black] (11,4) rectangle ++(1,1);
  \path[fill=white] (12,4) rectangle ++(1,1);
  \path[fill=black] (13,4) rectangle ++(1,1);
  \path[fill=white] (0,3) rectangle ++(1,1);
  \path[fill=white] (1,3) rectangle ++(1,1);
  \path[fill=white] (2,3) rectangle ++(1,1);
  \path[fill=white] (3,3) rectangle ++(1,1);
  \path[fill=white] (4,3) rectangle ++(1,1);
  \path[fill=white] (5,3) rectangle ++(1,1);
  \path[fill=white] (6,3) rectangle ++(1,1);
  \path[fill=white] (7,3) rectangle ++(1,1);
  \path[fill=white] (8,3) rectangle ++(1,1);
  \path[fill=white] (9,3) rectangle ++(1,1);
  \path[fill=white] (10,3) rectangle ++(1,1);
  \path[fill=black] (11,3) rectangle ++(1,1);
  \path[fill=black] (12,3) rectangle ++(1,1);
  \path[fill=black] (13,3) rectangle ++(1,1);
  \path[fill=white] (0,2) rectangle ++(1,1);
  \path[fill=white] (1,2) rectangle ++(1,1);
  \path[fill=white] (2,2) rectangle ++(1,1);
  \path[fill=white] (3,2) rectangle ++(1,1);
  \path[fill=white] (4,2) rectangle ++(1,1);
  \path[fill=white] (5,2) rectangle ++(1,1);
  \path[fill=white] (6,2) rectangle ++(1,1);
  \path[fill=white] (7,2) rectangle ++(1,1);
  \path[fill=white] (8,2) rectangle ++(1,1);
  \path[fill=white] (9,2) rectangle ++(1,1);
  \path[fill=white] (10,2) rectangle ++(1,1);
  \path[fill=white] (11,2) rectangle ++(1,1);
  \path[fill=white] (12,2) rectangle ++(1,1);
  \path[fill=black] (13,2) rectangle ++(1,1);
  \path[fill=white] (0,1) rectangle ++(1,1);
  \path[fill=white] (1,1) rectangle ++(1,1);
  \path[fill=white] (2,1) rectangle ++(1,1);
  \path[fill=white] (3,1) rectangle ++(1,1);
  \path[fill=white] (4,1) rectangle ++(1,1);
  \path[fill=white] (5,1) rectangle ++(1,1);
  \path[fill=white] (6,1) rectangle ++(1,1);
  \path[fill=white] (7,1) rectangle ++(1,1);
  \path[fill=white] (8,1) rectangle ++(1,1);
  \path[fill=white] (9,1) rectangle ++(1,1);
  \path[fill=white] (10,1) rectangle ++(1,1);
  \path[fill=white] (11,1) rectangle ++(1,1);
  \path[fill=white] (12,1) rectangle ++(1,1);
  \path[fill=black] (13,1) rectangle ++(1,1);
  \path[fill=white] (0,0) rectangle ++(1,1);
  \path[fill=white] (1,0) rectangle ++(1,1);
  \path[fill=white] (2,0) rectangle ++(1,1);
  \path[fill=white] (3,0) rectangle ++(1,1);
  \path[fill=white] (4,0) rectangle ++(1,1);
  \path[fill=white] (5,0) rectangle ++(1,1);
  \path[fill=white] (6,0) rectangle ++(1,1);
  \path[fill=white] (7,0) rectangle ++(1,1);
  \path[fill=white] (8,0) rectangle ++(1,1);
  \path[fill=white] (9,0) rectangle ++(1,1);
  \path[fill=white] (10,0) rectangle ++(1,1);
  \path[fill=white] (11,0) rectangle ++(1,1);
  \path[fill=white] (12,0) rectangle ++(1,1);
  \path[fill=white] (13,0) rectangle ++(1,1);
\end{tikzpicture}}}
\end{center}
\caption{Lev(|\sigma|=6, \Delta=1) automaton, adjacency and reachability matrix.}
\end{figure}

\begin{figure}[H]
\begin{center}
\adjustbox{valign=m}{\includegraphics[width=4cm]{figures/lev_nfa_6x2.pdf}}
\adjustbox{valign=m}{\resizebox{0.4\textwidth}{!}{\input{figures/lev_nfa_6x2}}}
\end{center}
\caption{Lev(|\sigma|=6, \Delta=2) automaton, adjacency and reachability matrix.}
\end{figure}

\begin{figure}[H]
\begin{center}
\adjustbox{valign=m}{\includegraphics[width=4cm]{figures/lev_nfa_6x3.pdf}}
\adjustbox{valign=m}{\resizebox{0.4\textwidth}{!}{\begin{tikzpicture}[x=0.3cm, y=0.3cm, draw=gray, very thin]
\path[fill=white] (0,27) rectangle ++(1,1);
\path[fill=black] (1,27) rectangle ++(1,1);
\path[fill=black] (2,27) rectangle ++(1,1);
\path[fill=white] (3,27) rectangle ++(1,1);
\path[fill=black] (4,27) rectangle ++(1,1);
\path[fill=white] (5,27) rectangle ++(1,1);
\path[fill=white] (6,27) rectangle ++(1,1);
\path[fill=white] (7,27) rectangle ++(1,1);
\path[fill=black] (8,27) rectangle ++(1,1);
\path[fill=white] (9,27) rectangle ++(1,1);
\path[fill=white] (10,27) rectangle ++(1,1);
\path[fill=white] (11,27) rectangle ++(1,1);
\path[fill=white] (12,27) rectangle ++(1,1);
\path[fill=white] (13,27) rectangle ++(1,1);
\path[fill=white] (14,27) rectangle ++(1,1);
\path[fill=black] (15,27) rectangle ++(1,1);
\path[fill=white] (16,27) rectangle ++(1,1);
\path[fill=white] (17,27) rectangle ++(1,1);
\path[fill=white] (18,27) rectangle ++(1,1);
\path[fill=white] (19,27) rectangle ++(1,1);
\path[fill=white] (20,27) rectangle ++(1,1);
\path[fill=white] (21,27) rectangle ++(1,1);
\path[fill=black] (22,27) rectangle ++(1,1);
\path[fill=white] (23,27) rectangle ++(1,1);
\path[fill=white] (24,27) rectangle ++(1,1);
\path[fill=white] (25,27) rectangle ++(1,1);
\path[fill=white] (26,27) rectangle ++(1,1);
\path[fill=white] (27,27) rectangle ++(1,1);
\path[fill=white] (0,26) rectangle ++(1,1);
\path[fill=white] (1,26) rectangle ++(1,1);
\path[fill=white] (2,26) rectangle ++(1,1);
\path[fill=black] (3,26) rectangle ++(1,1);
\path[fill=black] (4,26) rectangle ++(1,1);
\path[fill=white] (5,26) rectangle ++(1,1);
\path[fill=white] (6,26) rectangle ++(1,1);
\path[fill=black] (7,26) rectangle ++(1,1);
\path[fill=white] (8,26) rectangle ++(1,1);
\path[fill=white] (9,26) rectangle ++(1,1);
\path[fill=white] (10,26) rectangle ++(1,1);
\path[fill=black] (11,26) rectangle ++(1,1);
\path[fill=white] (12,26) rectangle ++(1,1);
\path[fill=white] (13,26) rectangle ++(1,1);
\path[fill=white] (14,26) rectangle ++(1,1);
\path[fill=white] (15,26) rectangle ++(1,1);
\path[fill=white] (16,26) rectangle ++(1,1);
\path[fill=white] (17,26) rectangle ++(1,1);
\path[fill=black] (18,26) rectangle ++(1,1);
\path[fill=white] (19,26) rectangle ++(1,1);
\path[fill=white] (20,26) rectangle ++(1,1);
\path[fill=white] (21,26) rectangle ++(1,1);
\path[fill=white] (22,26) rectangle ++(1,1);
\path[fill=white] (23,26) rectangle ++(1,1);
\path[fill=white] (24,26) rectangle ++(1,1);
\path[fill=white] (25,26) rectangle ++(1,1);
\path[fill=white] (26,26) rectangle ++(1,1);
\path[fill=white] (27,26) rectangle ++(1,1);
\path[fill=white] (0,25) rectangle ++(1,1);
\path[fill=white] (1,25) rectangle ++(1,1);
\path[fill=white] (2,25) rectangle ++(1,1);
\path[fill=white] (3,25) rectangle ++(1,1);
\path[fill=black] (4,25) rectangle ++(1,1);
\path[fill=black] (5,25) rectangle ++(1,1);
\path[fill=white] (6,25) rectangle ++(1,1);
\path[fill=white] (7,25) rectangle ++(1,1);
\path[fill=black] (8,25) rectangle ++(1,1);
\path[fill=white] (9,25) rectangle ++(1,1);
\path[fill=white] (10,25) rectangle ++(1,1);
\path[fill=white] (11,25) rectangle ++(1,1);
\path[fill=black] (12,25) rectangle ++(1,1);
\path[fill=white] (13,25) rectangle ++(1,1);
\path[fill=white] (14,25) rectangle ++(1,1);
\path[fill=white] (15,25) rectangle ++(1,1);
\path[fill=white] (16,25) rectangle ++(1,1);
\path[fill=white] (17,25) rectangle ++(1,1);
\path[fill=white] (18,25) rectangle ++(1,1);
\path[fill=black] (19,25) rectangle ++(1,1);
\path[fill=white] (20,25) rectangle ++(1,1);
\path[fill=white] (21,25) rectangle ++(1,1);
\path[fill=white] (22,25) rectangle ++(1,1);
\path[fill=white] (23,25) rectangle ++(1,1);
\path[fill=white] (24,25) rectangle ++(1,1);
\path[fill=black] (25,25) rectangle ++(1,1);
\path[fill=white] (26,25) rectangle ++(1,1);
\path[fill=white] (27,25) rectangle ++(1,1);
\path[fill=white] (0,24) rectangle ++(1,1);
\path[fill=white] (1,24) rectangle ++(1,1);
\path[fill=white] (2,24) rectangle ++(1,1);
\path[fill=white] (3,24) rectangle ++(1,1);
\path[fill=white] (4,24) rectangle ++(1,1);
\path[fill=white] (5,24) rectangle ++(1,1);
\path[fill=black] (6,24) rectangle ++(1,1);
\path[fill=black] (7,24) rectangle ++(1,1);
\path[fill=white] (8,24) rectangle ++(1,1);
\path[fill=white] (9,24) rectangle ++(1,1);
\path[fill=black] (10,24) rectangle ++(1,1);
\path[fill=white] (11,24) rectangle ++(1,1);
\path[fill=white] (12,24) rectangle ++(1,1);
\path[fill=white] (13,24) rectangle ++(1,1);
\path[fill=black] (14,24) rectangle ++(1,1);
\path[fill=white] (15,24) rectangle ++(1,1);
\path[fill=white] (16,24) rectangle ++(1,1);
\path[fill=white] (17,24) rectangle ++(1,1);
\path[fill=white] (18,24) rectangle ++(1,1);
\path[fill=white] (19,24) rectangle ++(1,1);
\path[fill=white] (20,24) rectangle ++(1,1);
\path[fill=white] (21,24) rectangle ++(1,1);
\path[fill=white] (22,24) rectangle ++(1,1);
\path[fill=white] (23,24) rectangle ++(1,1);
\path[fill=white] (24,24) rectangle ++(1,1);
\path[fill=white] (25,24) rectangle ++(1,1);
\path[fill=white] (26,24) rectangle ++(1,1);
\path[fill=white] (27,24) rectangle ++(1,1);
\path[fill=white] (0,23) rectangle ++(1,1);
\path[fill=white] (1,23) rectangle ++(1,1);
\path[fill=white] (2,23) rectangle ++(1,1);
\path[fill=white] (3,23) rectangle ++(1,1);
\path[fill=white] (4,23) rectangle ++(1,1);
\path[fill=white] (5,23) rectangle ++(1,1);
\path[fill=white] (6,23) rectangle ++(1,1);
\path[fill=black] (7,23) rectangle ++(1,1);
\path[fill=black] (8,23) rectangle ++(1,1);
\path[fill=white] (9,23) rectangle ++(1,1);
\path[fill=white] (10,23) rectangle ++(1,1);
\path[fill=black] (11,23) rectangle ++(1,1);
\path[fill=white] (12,23) rectangle ++(1,1);
\path[fill=white] (13,23) rectangle ++(1,1);
\path[fill=white] (14,23) rectangle ++(1,1);
\path[fill=black] (15,23) rectangle ++(1,1);
\path[fill=white] (16,23) rectangle ++(1,1);
\path[fill=white] (17,23) rectangle ++(1,1);
\path[fill=white] (18,23) rectangle ++(1,1);
\path[fill=white] (19,23) rectangle ++(1,1);
\path[fill=white] (20,23) rectangle ++(1,1);
\path[fill=white] (21,23) rectangle ++(1,1);
\path[fill=black] (22,23) rectangle ++(1,1);
\path[fill=white] (23,23) rectangle ++(1,1);
\path[fill=white] (24,23) rectangle ++(1,1);
\path[fill=white] (25,23) rectangle ++(1,1);
\path[fill=white] (26,23) rectangle ++(1,1);
\path[fill=white] (27,23) rectangle ++(1,1);
\path[fill=white] (0,22) rectangle ++(1,1);
\path[fill=white] (1,22) rectangle ++(1,1);
\path[fill=white] (2,22) rectangle ++(1,1);
\path[fill=white] (3,22) rectangle ++(1,1);
\path[fill=white] (4,22) rectangle ++(1,1);
\path[fill=white] (5,22) rectangle ++(1,1);
\path[fill=white] (6,22) rectangle ++(1,1);
\path[fill=white] (7,22) rectangle ++(1,1);
\path[fill=black] (8,22) rectangle ++(1,1);
\path[fill=black] (9,22) rectangle ++(1,1);
\path[fill=white] (10,22) rectangle ++(1,1);
\path[fill=white] (11,22) rectangle ++(1,1);
\path[fill=black] (12,22) rectangle ++(1,1);
\path[fill=white] (13,22) rectangle ++(1,1);
\path[fill=white] (14,22) rectangle ++(1,1);
\path[fill=white] (15,22) rectangle ++(1,1);
\path[fill=black] (16,22) rectangle ++(1,1);
\path[fill=white] (17,22) rectangle ++(1,1);
\path[fill=white] (18,22) rectangle ++(1,1);
\path[fill=white] (19,22) rectangle ++(1,1);
\path[fill=white] (20,22) rectangle ++(1,1);
\path[fill=white] (21,22) rectangle ++(1,1);
\path[fill=white] (22,22) rectangle ++(1,1);
\path[fill=black] (23,22) rectangle ++(1,1);
\path[fill=white] (24,22) rectangle ++(1,1);
\path[fill=white] (25,22) rectangle ++(1,1);
\path[fill=white] (26,22) rectangle ++(1,1);
\path[fill=black] (27,22) rectangle ++(1,1);
\path[fill=white] (0,21) rectangle ++(1,1);
\path[fill=white] (1,21) rectangle ++(1,1);
\path[fill=white] (2,21) rectangle ++(1,1);
\path[fill=white] (3,21) rectangle ++(1,1);
\path[fill=white] (4,21) rectangle ++(1,1);
\path[fill=white] (5,21) rectangle ++(1,1);
\path[fill=white] (6,21) rectangle ++(1,1);
\path[fill=white] (7,21) rectangle ++(1,1);
\path[fill=white] (8,21) rectangle ++(1,1);
\path[fill=white] (9,21) rectangle ++(1,1);
\path[fill=black] (10,21) rectangle ++(1,1);
\path[fill=white] (11,21) rectangle ++(1,1);
\path[fill=white] (12,21) rectangle ++(1,1);
\path[fill=white] (13,21) rectangle ++(1,1);
\path[fill=white] (14,21) rectangle ++(1,1);
\path[fill=white] (15,21) rectangle ++(1,1);
\path[fill=white] (16,21) rectangle ++(1,1);
\path[fill=white] (17,21) rectangle ++(1,1);
\path[fill=white] (18,21) rectangle ++(1,1);
\path[fill=white] (19,21) rectangle ++(1,1);
\path[fill=white] (20,21) rectangle ++(1,1);
\path[fill=white] (21,21) rectangle ++(1,1);
\path[fill=white] (22,21) rectangle ++(1,1);
\path[fill=white] (23,21) rectangle ++(1,1);
\path[fill=white] (24,21) rectangle ++(1,1);
\path[fill=white] (25,21) rectangle ++(1,1);
\path[fill=white] (26,21) rectangle ++(1,1);
\path[fill=white] (27,21) rectangle ++(1,1);
\path[fill=white] (0,20) rectangle ++(1,1);
\path[fill=white] (1,20) rectangle ++(1,1);
\path[fill=white] (2,20) rectangle ++(1,1);
\path[fill=white] (3,20) rectangle ++(1,1);
\path[fill=white] (4,20) rectangle ++(1,1);
\path[fill=white] (5,20) rectangle ++(1,1);
\path[fill=white] (6,20) rectangle ++(1,1);
\path[fill=white] (7,20) rectangle ++(1,1);
\path[fill=white] (8,20) rectangle ++(1,1);
\path[fill=white] (9,20) rectangle ++(1,1);
\path[fill=black] (10,20) rectangle ++(1,1);
\path[fill=black] (11,20) rectangle ++(1,1);
\path[fill=white] (12,20) rectangle ++(1,1);
\path[fill=white] (13,20) rectangle ++(1,1);
\path[fill=black] (14,20) rectangle ++(1,1);
\path[fill=white] (15,20) rectangle ++(1,1);
\path[fill=white] (16,20) rectangle ++(1,1);
\path[fill=white] (17,20) rectangle ++(1,1);
\path[fill=black] (18,20) rectangle ++(1,1);
\path[fill=white] (19,20) rectangle ++(1,1);
\path[fill=white] (20,20) rectangle ++(1,1);
\path[fill=white] (21,20) rectangle ++(1,1);
\path[fill=white] (22,20) rectangle ++(1,1);
\path[fill=white] (23,20) rectangle ++(1,1);
\path[fill=white] (24,20) rectangle ++(1,1);
\path[fill=white] (25,20) rectangle ++(1,1);
\path[fill=white] (26,20) rectangle ++(1,1);
\path[fill=white] (27,20) rectangle ++(1,1);
\path[fill=white] (0,19) rectangle ++(1,1);
\path[fill=white] (1,19) rectangle ++(1,1);
\path[fill=white] (2,19) rectangle ++(1,1);
\path[fill=white] (3,19) rectangle ++(1,1);
\path[fill=white] (4,19) rectangle ++(1,1);
\path[fill=white] (5,19) rectangle ++(1,1);
\path[fill=white] (6,19) rectangle ++(1,1);
\path[fill=white] (7,19) rectangle ++(1,1);
\path[fill=white] (8,19) rectangle ++(1,1);
\path[fill=white] (9,19) rectangle ++(1,1);
\path[fill=white] (10,19) rectangle ++(1,1);
\path[fill=black] (11,19) rectangle ++(1,1);
\path[fill=black] (12,19) rectangle ++(1,1);
\path[fill=white] (13,19) rectangle ++(1,1);
\path[fill=white] (14,19) rectangle ++(1,1);
\path[fill=black] (15,19) rectangle ++(1,1);
\path[fill=white] (16,19) rectangle ++(1,1);
\path[fill=white] (17,19) rectangle ++(1,1);
\path[fill=white] (18,19) rectangle ++(1,1);
\path[fill=black] (19,19) rectangle ++(1,1);
\path[fill=white] (20,19) rectangle ++(1,1);
\path[fill=white] (21,19) rectangle ++(1,1);
\path[fill=white] (22,19) rectangle ++(1,1);
\path[fill=white] (23,19) rectangle ++(1,1);
\path[fill=white] (24,19) rectangle ++(1,1);
\path[fill=black] (25,19) rectangle ++(1,1);
\path[fill=white] (26,19) rectangle ++(1,1);
\path[fill=white] (27,19) rectangle ++(1,1);
\path[fill=white] (0,18) rectangle ++(1,1);
\path[fill=white] (1,18) rectangle ++(1,1);
\path[fill=white] (2,18) rectangle ++(1,1);
\path[fill=white] (3,18) rectangle ++(1,1);
\path[fill=white] (4,18) rectangle ++(1,1);
\path[fill=white] (5,18) rectangle ++(1,1);
\path[fill=white] (6,18) rectangle ++(1,1);
\path[fill=white] (7,18) rectangle ++(1,1);
\path[fill=white] (8,18) rectangle ++(1,1);
\path[fill=white] (9,18) rectangle ++(1,1);
\path[fill=white] (10,18) rectangle ++(1,1);
\path[fill=white] (11,18) rectangle ++(1,1);
\path[fill=black] (12,18) rectangle ++(1,1);
\path[fill=black] (13,18) rectangle ++(1,1);
\path[fill=white] (14,18) rectangle ++(1,1);
\path[fill=white] (15,18) rectangle ++(1,1);
\path[fill=black] (16,18) rectangle ++(1,1);
\path[fill=white] (17,18) rectangle ++(1,1);
\path[fill=white] (18,18) rectangle ++(1,1);
\path[fill=white] (19,18) rectangle ++(1,1);
\path[fill=black] (20,18) rectangle ++(1,1);
\path[fill=white] (21,18) rectangle ++(1,1);
\path[fill=white] (22,18) rectangle ++(1,1);
\path[fill=white] (23,18) rectangle ++(1,1);
\path[fill=white] (24,18) rectangle ++(1,1);
\path[fill=white] (25,18) rectangle ++(1,1);
\path[fill=black] (26,18) rectangle ++(1,1);
\path[fill=white] (27,18) rectangle ++(1,1);
\path[fill=white] (0,17) rectangle ++(1,1);
\path[fill=white] (1,17) rectangle ++(1,1);
\path[fill=white] (2,17) rectangle ++(1,1);
\path[fill=white] (3,17) rectangle ++(1,1);
\path[fill=white] (4,17) rectangle ++(1,1);
\path[fill=white] (5,17) rectangle ++(1,1);
\path[fill=white] (6,17) rectangle ++(1,1);
\path[fill=white] (7,17) rectangle ++(1,1);
\path[fill=white] (8,17) rectangle ++(1,1);
\path[fill=white] (9,17) rectangle ++(1,1);
\path[fill=white] (10,17) rectangle ++(1,1);
\path[fill=white] (11,17) rectangle ++(1,1);
\path[fill=white] (12,17) rectangle ++(1,1);
\path[fill=white] (13,17) rectangle ++(1,1);
\path[fill=black] (14,17) rectangle ++(1,1);
\path[fill=white] (15,17) rectangle ++(1,1);
\path[fill=white] (16,17) rectangle ++(1,1);
\path[fill=white] (17,17) rectangle ++(1,1);
\path[fill=white] (18,17) rectangle ++(1,1);
\path[fill=white] (19,17) rectangle ++(1,1);
\path[fill=white] (20,17) rectangle ++(1,1);
\path[fill=white] (21,17) rectangle ++(1,1);
\path[fill=white] (22,17) rectangle ++(1,1);
\path[fill=white] (23,17) rectangle ++(1,1);
\path[fill=white] (24,17) rectangle ++(1,1);
\path[fill=white] (25,17) rectangle ++(1,1);
\path[fill=white] (26,17) rectangle ++(1,1);
\path[fill=white] (27,17) rectangle ++(1,1);
\path[fill=white] (0,16) rectangle ++(1,1);
\path[fill=white] (1,16) rectangle ++(1,1);
\path[fill=white] (2,16) rectangle ++(1,1);
\path[fill=white] (3,16) rectangle ++(1,1);
\path[fill=white] (4,16) rectangle ++(1,1);
\path[fill=white] (5,16) rectangle ++(1,1);
\path[fill=white] (6,16) rectangle ++(1,1);
\path[fill=white] (7,16) rectangle ++(1,1);
\path[fill=white] (8,16) rectangle ++(1,1);
\path[fill=white] (9,16) rectangle ++(1,1);
\path[fill=white] (10,16) rectangle ++(1,1);
\path[fill=white] (11,16) rectangle ++(1,1);
\path[fill=white] (12,16) rectangle ++(1,1);
\path[fill=white] (13,16) rectangle ++(1,1);
\path[fill=black] (14,16) rectangle ++(1,1);
\path[fill=black] (15,16) rectangle ++(1,1);
\path[fill=white] (16,16) rectangle ++(1,1);
\path[fill=white] (17,16) rectangle ++(1,1);
\path[fill=black] (18,16) rectangle ++(1,1);
\path[fill=white] (19,16) rectangle ++(1,1);
\path[fill=white] (20,16) rectangle ++(1,1);
\path[fill=white] (21,16) rectangle ++(1,1);
\path[fill=black] (22,16) rectangle ++(1,1);
\path[fill=white] (23,16) rectangle ++(1,1);
\path[fill=white] (24,16) rectangle ++(1,1);
\path[fill=white] (25,16) rectangle ++(1,1);
\path[fill=white] (26,16) rectangle ++(1,1);
\path[fill=white] (27,16) rectangle ++(1,1);
\path[fill=white] (0,15) rectangle ++(1,1);
\path[fill=white] (1,15) rectangle ++(1,1);
\path[fill=white] (2,15) rectangle ++(1,1);
\path[fill=white] (3,15) rectangle ++(1,1);
\path[fill=white] (4,15) rectangle ++(1,1);
\path[fill=white] (5,15) rectangle ++(1,1);
\path[fill=white] (6,15) rectangle ++(1,1);
\path[fill=white] (7,15) rectangle ++(1,1);
\path[fill=white] (8,15) rectangle ++(1,1);
\path[fill=white] (9,15) rectangle ++(1,1);
\path[fill=white] (10,15) rectangle ++(1,1);
\path[fill=white] (11,15) rectangle ++(1,1);
\path[fill=white] (12,15) rectangle ++(1,1);
\path[fill=white] (13,15) rectangle ++(1,1);
\path[fill=white] (14,15) rectangle ++(1,1);
\path[fill=black] (15,15) rectangle ++(1,1);
\path[fill=black] (16,15) rectangle ++(1,1);
\path[fill=white] (17,15) rectangle ++(1,1);
\path[fill=white] (18,15) rectangle ++(1,1);
\path[fill=black] (19,15) rectangle ++(1,1);
\path[fill=white] (20,15) rectangle ++(1,1);
\path[fill=white] (21,15) rectangle ++(1,1);
\path[fill=white] (22,15) rectangle ++(1,1);
\path[fill=black] (23,15) rectangle ++(1,1);
\path[fill=white] (24,15) rectangle ++(1,1);
\path[fill=white] (25,15) rectangle ++(1,1);
\path[fill=white] (26,15) rectangle ++(1,1);
\path[fill=black] (27,15) rectangle ++(1,1);
\path[fill=white] (0,14) rectangle ++(1,1);
\path[fill=white] (1,14) rectangle ++(1,1);
\path[fill=white] (2,14) rectangle ++(1,1);
\path[fill=white] (3,14) rectangle ++(1,1);
\path[fill=white] (4,14) rectangle ++(1,1);
\path[fill=white] (5,14) rectangle ++(1,1);
\path[fill=white] (6,14) rectangle ++(1,1);
\path[fill=white] (7,14) rectangle ++(1,1);
\path[fill=white] (8,14) rectangle ++(1,1);
\path[fill=white] (9,14) rectangle ++(1,1);
\path[fill=white] (10,14) rectangle ++(1,1);
\path[fill=white] (11,14) rectangle ++(1,1);
\path[fill=white] (12,14) rectangle ++(1,1);
\path[fill=white] (13,14) rectangle ++(1,1);
\path[fill=white] (14,14) rectangle ++(1,1);
\path[fill=white] (15,14) rectangle ++(1,1);
\path[fill=black] (16,14) rectangle ++(1,1);
\path[fill=black] (17,14) rectangle ++(1,1);
\path[fill=white] (18,14) rectangle ++(1,1);
\path[fill=white] (19,14) rectangle ++(1,1);
\path[fill=black] (20,14) rectangle ++(1,1);
\path[fill=white] (21,14) rectangle ++(1,1);
\path[fill=white] (22,14) rectangle ++(1,1);
\path[fill=white] (23,14) rectangle ++(1,1);
\path[fill=black] (24,14) rectangle ++(1,1);
\path[fill=white] (25,14) rectangle ++(1,1);
\path[fill=white] (26,14) rectangle ++(1,1);
\path[fill=white] (27,14) rectangle ++(1,1);
\path[fill=white] (0,13) rectangle ++(1,1);
\path[fill=white] (1,13) rectangle ++(1,1);
\path[fill=white] (2,13) rectangle ++(1,1);
\path[fill=white] (3,13) rectangle ++(1,1);
\path[fill=white] (4,13) rectangle ++(1,1);
\path[fill=white] (5,13) rectangle ++(1,1);
\path[fill=white] (6,13) rectangle ++(1,1);
\path[fill=white] (7,13) rectangle ++(1,1);
\path[fill=white] (8,13) rectangle ++(1,1);
\path[fill=white] (9,13) rectangle ++(1,1);
\path[fill=white] (10,13) rectangle ++(1,1);
\path[fill=white] (11,13) rectangle ++(1,1);
\path[fill=white] (12,13) rectangle ++(1,1);
\path[fill=white] (13,13) rectangle ++(1,1);
\path[fill=white] (14,13) rectangle ++(1,1);
\path[fill=white] (15,13) rectangle ++(1,1);
\path[fill=white] (16,13) rectangle ++(1,1);
\path[fill=white] (17,13) rectangle ++(1,1);
\path[fill=black] (18,13) rectangle ++(1,1);
\path[fill=white] (19,13) rectangle ++(1,1);
\path[fill=white] (20,13) rectangle ++(1,1);
\path[fill=white] (21,13) rectangle ++(1,1);
\path[fill=white] (22,13) rectangle ++(1,1);
\path[fill=white] (23,13) rectangle ++(1,1);
\path[fill=white] (24,13) rectangle ++(1,1);
\path[fill=white] (25,13) rectangle ++(1,1);
\path[fill=white] (26,13) rectangle ++(1,1);
\path[fill=white] (27,13) rectangle ++(1,1);
\path[fill=white] (0,12) rectangle ++(1,1);
\path[fill=white] (1,12) rectangle ++(1,1);
\path[fill=white] (2,12) rectangle ++(1,1);
\path[fill=white] (3,12) rectangle ++(1,1);
\path[fill=white] (4,12) rectangle ++(1,1);
\path[fill=white] (5,12) rectangle ++(1,1);
\path[fill=white] (6,12) rectangle ++(1,1);
\path[fill=white] (7,12) rectangle ++(1,1);
\path[fill=white] (8,12) rectangle ++(1,1);
\path[fill=white] (9,12) rectangle ++(1,1);
\path[fill=white] (10,12) rectangle ++(1,1);
\path[fill=white] (11,12) rectangle ++(1,1);
\path[fill=white] (12,12) rectangle ++(1,1);
\path[fill=white] (13,12) rectangle ++(1,1);
\path[fill=white] (14,12) rectangle ++(1,1);
\path[fill=white] (15,12) rectangle ++(1,1);
\path[fill=white] (16,12) rectangle ++(1,1);
\path[fill=white] (17,12) rectangle ++(1,1);
\path[fill=black] (18,12) rectangle ++(1,1);
\path[fill=black] (19,12) rectangle ++(1,1);
\path[fill=white] (20,12) rectangle ++(1,1);
\path[fill=white] (21,12) rectangle ++(1,1);
\path[fill=black] (22,12) rectangle ++(1,1);
\path[fill=white] (23,12) rectangle ++(1,1);
\path[fill=white] (24,12) rectangle ++(1,1);
\path[fill=black] (25,12) rectangle ++(1,1);
\path[fill=white] (26,12) rectangle ++(1,1);
\path[fill=white] (27,12) rectangle ++(1,1);
\path[fill=white] (0,11) rectangle ++(1,1);
\path[fill=white] (1,11) rectangle ++(1,1);
\path[fill=white] (2,11) rectangle ++(1,1);
\path[fill=white] (3,11) rectangle ++(1,1);
\path[fill=white] (4,11) rectangle ++(1,1);
\path[fill=white] (5,11) rectangle ++(1,1);
\path[fill=white] (6,11) rectangle ++(1,1);
\path[fill=white] (7,11) rectangle ++(1,1);
\path[fill=white] (8,11) rectangle ++(1,1);
\path[fill=white] (9,11) rectangle ++(1,1);
\path[fill=white] (10,11) rectangle ++(1,1);
\path[fill=white] (11,11) rectangle ++(1,1);
\path[fill=white] (12,11) rectangle ++(1,1);
\path[fill=white] (13,11) rectangle ++(1,1);
\path[fill=white] (14,11) rectangle ++(1,1);
\path[fill=white] (15,11) rectangle ++(1,1);
\path[fill=white] (16,11) rectangle ++(1,1);
\path[fill=white] (17,11) rectangle ++(1,1);
\path[fill=white] (18,11) rectangle ++(1,1);
\path[fill=black] (19,11) rectangle ++(1,1);
\path[fill=black] (20,11) rectangle ++(1,1);
\path[fill=white] (21,11) rectangle ++(1,1);
\path[fill=white] (22,11) rectangle ++(1,1);
\path[fill=black] (23,11) rectangle ++(1,1);
\path[fill=white] (24,11) rectangle ++(1,1);
\path[fill=white] (25,11) rectangle ++(1,1);
\path[fill=black] (26,11) rectangle ++(1,1);
\path[fill=white] (27,11) rectangle ++(1,1);
\path[fill=white] (0,10) rectangle ++(1,1);
\path[fill=white] (1,10) rectangle ++(1,1);
\path[fill=white] (2,10) rectangle ++(1,1);
\path[fill=white] (3,10) rectangle ++(1,1);
\path[fill=white] (4,10) rectangle ++(1,1);
\path[fill=white] (5,10) rectangle ++(1,1);
\path[fill=white] (6,10) rectangle ++(1,1);
\path[fill=white] (7,10) rectangle ++(1,1);
\path[fill=white] (8,10) rectangle ++(1,1);
\path[fill=white] (9,10) rectangle ++(1,1);
\path[fill=white] (10,10) rectangle ++(1,1);
\path[fill=white] (11,10) rectangle ++(1,1);
\path[fill=white] (12,10) rectangle ++(1,1);
\path[fill=white] (13,10) rectangle ++(1,1);
\path[fill=white] (14,10) rectangle ++(1,1);
\path[fill=white] (15,10) rectangle ++(1,1);
\path[fill=white] (16,10) rectangle ++(1,1);
\path[fill=white] (17,10) rectangle ++(1,1);
\path[fill=white] (18,10) rectangle ++(1,1);
\path[fill=white] (19,10) rectangle ++(1,1);
\path[fill=black] (20,10) rectangle ++(1,1);
\path[fill=black] (21,10) rectangle ++(1,1);
\path[fill=white] (22,10) rectangle ++(1,1);
\path[fill=white] (23,10) rectangle ++(1,1);
\path[fill=black] (24,10) rectangle ++(1,1);
\path[fill=white] (25,10) rectangle ++(1,1);
\path[fill=white] (26,10) rectangle ++(1,1);
\path[fill=white] (27,10) rectangle ++(1,1);
\path[fill=white] (0,9) rectangle ++(1,1);
\path[fill=white] (1,9) rectangle ++(1,1);
\path[fill=white] (2,9) rectangle ++(1,1);
\path[fill=white] (3,9) rectangle ++(1,1);
\path[fill=white] (4,9) rectangle ++(1,1);
\path[fill=white] (5,9) rectangle ++(1,1);
\path[fill=white] (6,9) rectangle ++(1,1);
\path[fill=white] (7,9) rectangle ++(1,1);
\path[fill=white] (8,9) rectangle ++(1,1);
\path[fill=white] (9,9) rectangle ++(1,1);
\path[fill=white] (10,9) rectangle ++(1,1);
\path[fill=white] (11,9) rectangle ++(1,1);
\path[fill=white] (12,9) rectangle ++(1,1);
\path[fill=white] (13,9) rectangle ++(1,1);
\path[fill=white] (14,9) rectangle ++(1,1);
\path[fill=white] (15,9) rectangle ++(1,1);
\path[fill=white] (16,9) rectangle ++(1,1);
\path[fill=white] (17,9) rectangle ++(1,1);
\path[fill=white] (18,9) rectangle ++(1,1);
\path[fill=white] (19,9) rectangle ++(1,1);
\path[fill=white] (20,9) rectangle ++(1,1);
\path[fill=white] (21,9) rectangle ++(1,1);
\path[fill=black] (22,9) rectangle ++(1,1);
\path[fill=white] (23,9) rectangle ++(1,1);
\path[fill=white] (24,9) rectangle ++(1,1);
\path[fill=white] (25,9) rectangle ++(1,1);
\path[fill=white] (26,9) rectangle ++(1,1);
\path[fill=white] (27,9) rectangle ++(1,1);
\path[fill=white] (0,8) rectangle ++(1,1);
\path[fill=white] (1,8) rectangle ++(1,1);
\path[fill=white] (2,8) rectangle ++(1,1);
\path[fill=white] (3,8) rectangle ++(1,1);
\path[fill=white] (4,8) rectangle ++(1,1);
\path[fill=white] (5,8) rectangle ++(1,1);
\path[fill=white] (6,8) rectangle ++(1,1);
\path[fill=white] (7,8) rectangle ++(1,1);
\path[fill=white] (8,8) rectangle ++(1,1);
\path[fill=white] (9,8) rectangle ++(1,1);
\path[fill=white] (10,8) rectangle ++(1,1);
\path[fill=white] (11,8) rectangle ++(1,1);
\path[fill=white] (12,8) rectangle ++(1,1);
\path[fill=white] (13,8) rectangle ++(1,1);
\path[fill=white] (14,8) rectangle ++(1,1);
\path[fill=white] (15,8) rectangle ++(1,1);
\path[fill=white] (16,8) rectangle ++(1,1);
\path[fill=white] (17,8) rectangle ++(1,1);
\path[fill=white] (18,8) rectangle ++(1,1);
\path[fill=white] (19,8) rectangle ++(1,1);
\path[fill=white] (20,8) rectangle ++(1,1);
\path[fill=white] (21,8) rectangle ++(1,1);
\path[fill=black] (22,8) rectangle ++(1,1);
\path[fill=black] (23,8) rectangle ++(1,1);
\path[fill=white] (24,8) rectangle ++(1,1);
\path[fill=black] (25,8) rectangle ++(1,1);
\path[fill=white] (26,8) rectangle ++(1,1);
\path[fill=black] (27,8) rectangle ++(1,1);
\path[fill=white] (0,7) rectangle ++(1,1);
\path[fill=white] (1,7) rectangle ++(1,1);
\path[fill=white] (2,7) rectangle ++(1,1);
\path[fill=white] (3,7) rectangle ++(1,1);
\path[fill=white] (4,7) rectangle ++(1,1);
\path[fill=white] (5,7) rectangle ++(1,1);
\path[fill=white] (6,7) rectangle ++(1,1);
\path[fill=white] (7,7) rectangle ++(1,1);
\path[fill=white] (8,7) rectangle ++(1,1);
\path[fill=white] (9,7) rectangle ++(1,1);
\path[fill=white] (10,7) rectangle ++(1,1);
\path[fill=white] (11,7) rectangle ++(1,1);
\path[fill=white] (12,7) rectangle ++(1,1);
\path[fill=white] (13,7) rectangle ++(1,1);
\path[fill=white] (14,7) rectangle ++(1,1);
\path[fill=white] (15,7) rectangle ++(1,1);
\path[fill=white] (16,7) rectangle ++(1,1);
\path[fill=white] (17,7) rectangle ++(1,1);
\path[fill=white] (18,7) rectangle ++(1,1);
\path[fill=white] (19,7) rectangle ++(1,1);
\path[fill=white] (20,7) rectangle ++(1,1);
\path[fill=white] (21,7) rectangle ++(1,1);
\path[fill=white] (22,7) rectangle ++(1,1);
\path[fill=black] (23,7) rectangle ++(1,1);
\path[fill=black] (24,7) rectangle ++(1,1);
\path[fill=white] (25,7) rectangle ++(1,1);
\path[fill=black] (26,7) rectangle ++(1,1);
\path[fill=white] (27,7) rectangle ++(1,1);
\path[fill=white] (0,6) rectangle ++(1,1);
\path[fill=white] (1,6) rectangle ++(1,1);
\path[fill=white] (2,6) rectangle ++(1,1);
\path[fill=white] (3,6) rectangle ++(1,1);
\path[fill=white] (4,6) rectangle ++(1,1);
\path[fill=white] (5,6) rectangle ++(1,1);
\path[fill=white] (6,6) rectangle ++(1,1);
\path[fill=white] (7,6) rectangle ++(1,1);
\path[fill=white] (8,6) rectangle ++(1,1);
\path[fill=white] (9,6) rectangle ++(1,1);
\path[fill=white] (10,6) rectangle ++(1,1);
\path[fill=white] (11,6) rectangle ++(1,1);
\path[fill=white] (12,6) rectangle ++(1,1);
\path[fill=white] (13,6) rectangle ++(1,1);
\path[fill=white] (14,6) rectangle ++(1,1);
\path[fill=white] (15,6) rectangle ++(1,1);
\path[fill=white] (16,6) rectangle ++(1,1);
\path[fill=white] (17,6) rectangle ++(1,1);
\path[fill=white] (18,6) rectangle ++(1,1);
\path[fill=white] (19,6) rectangle ++(1,1);
\path[fill=white] (20,6) rectangle ++(1,1);
\path[fill=white] (21,6) rectangle ++(1,1);
\path[fill=white] (22,6) rectangle ++(1,1);
\path[fill=white] (23,6) rectangle ++(1,1);
\path[fill=black] (24,6) rectangle ++(1,1);
\path[fill=white] (25,6) rectangle ++(1,1);
\path[fill=white] (26,6) rectangle ++(1,1);
\path[fill=white] (27,6) rectangle ++(1,1);
\path[fill=white] (0,5) rectangle ++(1,1);
\path[fill=white] (1,5) rectangle ++(1,1);
\path[fill=white] (2,5) rectangle ++(1,1);
\path[fill=white] (3,5) rectangle ++(1,1);
\path[fill=white] (4,5) rectangle ++(1,1);
\path[fill=white] (5,5) rectangle ++(1,1);
\path[fill=white] (6,5) rectangle ++(1,1);
\path[fill=white] (7,5) rectangle ++(1,1);
\path[fill=white] (8,5) rectangle ++(1,1);
\path[fill=white] (9,5) rectangle ++(1,1);
\path[fill=white] (10,5) rectangle ++(1,1);
\path[fill=white] (11,5) rectangle ++(1,1);
\path[fill=white] (12,5) rectangle ++(1,1);
\path[fill=white] (13,5) rectangle ++(1,1);
\path[fill=white] (14,5) rectangle ++(1,1);
\path[fill=white] (15,5) rectangle ++(1,1);
\path[fill=white] (16,5) rectangle ++(1,1);
\path[fill=white] (17,5) rectangle ++(1,1);
\path[fill=white] (18,5) rectangle ++(1,1);
\path[fill=white] (19,5) rectangle ++(1,1);
\path[fill=white] (20,5) rectangle ++(1,1);
\path[fill=white] (21,5) rectangle ++(1,1);
\path[fill=white] (22,5) rectangle ++(1,1);
\path[fill=white] (23,5) rectangle ++(1,1);
\path[fill=white] (24,5) rectangle ++(1,1);
\path[fill=black] (25,5) rectangle ++(1,1);
\path[fill=white] (26,5) rectangle ++(1,1);
\path[fill=white] (27,5) rectangle ++(1,1);
\path[fill=white] (0,4) rectangle ++(1,1);
\path[fill=white] (1,4) rectangle ++(1,1);
\path[fill=white] (2,4) rectangle ++(1,1);
\path[fill=white] (3,4) rectangle ++(1,1);
\path[fill=white] (4,4) rectangle ++(1,1);
\path[fill=white] (5,4) rectangle ++(1,1);
\path[fill=white] (6,4) rectangle ++(1,1);
\path[fill=white] (7,4) rectangle ++(1,1);
\path[fill=white] (8,4) rectangle ++(1,1);
\path[fill=white] (9,4) rectangle ++(1,1);
\path[fill=white] (10,4) rectangle ++(1,1);
\path[fill=white] (11,4) rectangle ++(1,1);
\path[fill=white] (12,4) rectangle ++(1,1);
\path[fill=white] (13,4) rectangle ++(1,1);
\path[fill=white] (14,4) rectangle ++(1,1);
\path[fill=white] (15,4) rectangle ++(1,1);
\path[fill=white] (16,4) rectangle ++(1,1);
\path[fill=white] (17,4) rectangle ++(1,1);
\path[fill=white] (18,4) rectangle ++(1,1);
\path[fill=white] (19,4) rectangle ++(1,1);
\path[fill=white] (20,4) rectangle ++(1,1);
\path[fill=white] (21,4) rectangle ++(1,1);
\path[fill=white] (22,4) rectangle ++(1,1);
\path[fill=white] (23,4) rectangle ++(1,1);
\path[fill=white] (24,4) rectangle ++(1,1);
\path[fill=black] (25,4) rectangle ++(1,1);
\path[fill=black] (26,4) rectangle ++(1,1);
\path[fill=black] (27,4) rectangle ++(1,1);
\path[fill=white] (0,3) rectangle ++(1,1);
\path[fill=white] (1,3) rectangle ++(1,1);
\path[fill=white] (2,3) rectangle ++(1,1);
\path[fill=white] (3,3) rectangle ++(1,1);
\path[fill=white] (4,3) rectangle ++(1,1);
\path[fill=white] (5,3) rectangle ++(1,1);
\path[fill=white] (6,3) rectangle ++(1,1);
\path[fill=white] (7,3) rectangle ++(1,1);
\path[fill=white] (8,3) rectangle ++(1,1);
\path[fill=white] (9,3) rectangle ++(1,1);
\path[fill=white] (10,3) rectangle ++(1,1);
\path[fill=white] (11,3) rectangle ++(1,1);
\path[fill=white] (12,3) rectangle ++(1,1);
\path[fill=white] (13,3) rectangle ++(1,1);
\path[fill=white] (14,3) rectangle ++(1,1);
\path[fill=white] (15,3) rectangle ++(1,1);
\path[fill=white] (16,3) rectangle ++(1,1);
\path[fill=white] (17,3) rectangle ++(1,1);
\path[fill=white] (18,3) rectangle ++(1,1);
\path[fill=white] (19,3) rectangle ++(1,1);
\path[fill=white] (20,3) rectangle ++(1,1);
\path[fill=white] (21,3) rectangle ++(1,1);
\path[fill=white] (22,3) rectangle ++(1,1);
\path[fill=white] (23,3) rectangle ++(1,1);
\path[fill=white] (24,3) rectangle ++(1,1);
\path[fill=white] (25,3) rectangle ++(1,1);
\path[fill=black] (26,3) rectangle ++(1,1);
\path[fill=white] (27,3) rectangle ++(1,1);
\path[fill=white] (0,2) rectangle ++(1,1);
\path[fill=white] (1,2) rectangle ++(1,1);
\path[fill=white] (2,2) rectangle ++(1,1);
\path[fill=white] (3,2) rectangle ++(1,1);
\path[fill=white] (4,2) rectangle ++(1,1);
\path[fill=white] (5,2) rectangle ++(1,1);
\path[fill=white] (6,2) rectangle ++(1,1);
\path[fill=white] (7,2) rectangle ++(1,1);
\path[fill=white] (8,2) rectangle ++(1,1);
\path[fill=white] (9,2) rectangle ++(1,1);
\path[fill=white] (10,2) rectangle ++(1,1);
\path[fill=white] (11,2) rectangle ++(1,1);
\path[fill=white] (12,2) rectangle ++(1,1);
\path[fill=white] (13,2) rectangle ++(1,1);
\path[fill=white] (14,2) rectangle ++(1,1);
\path[fill=white] (15,2) rectangle ++(1,1);
\path[fill=white] (16,2) rectangle ++(1,1);
\path[fill=white] (17,2) rectangle ++(1,1);
\path[fill=white] (18,2) rectangle ++(1,1);
\path[fill=white] (19,2) rectangle ++(1,1);
\path[fill=white] (20,2) rectangle ++(1,1);
\path[fill=white] (21,2) rectangle ++(1,1);
\path[fill=white] (22,2) rectangle ++(1,1);
\path[fill=white] (23,2) rectangle ++(1,1);
\path[fill=white] (24,2) rectangle ++(1,1);
\path[fill=white] (25,2) rectangle ++(1,1);
\path[fill=white] (26,2) rectangle ++(1,1);
\path[fill=black] (27,2) rectangle ++(1,1);
\path[fill=white] (0,1) rectangle ++(1,1);
\path[fill=white] (1,1) rectangle ++(1,1);
\path[fill=white] (2,1) rectangle ++(1,1);
\path[fill=white] (3,1) rectangle ++(1,1);
\path[fill=white] (4,1) rectangle ++(1,1);
\path[fill=white] (5,1) rectangle ++(1,1);
\path[fill=white] (6,1) rectangle ++(1,1);
\path[fill=white] (7,1) rectangle ++(1,1);
\path[fill=white] (8,1) rectangle ++(1,1);
\path[fill=white] (9,1) rectangle ++(1,1);
\path[fill=white] (10,1) rectangle ++(1,1);
\path[fill=white] (11,1) rectangle ++(1,1);
\path[fill=white] (12,1) rectangle ++(1,1);
\path[fill=white] (13,1) rectangle ++(1,1);
\path[fill=white] (14,1) rectangle ++(1,1);
\path[fill=white] (15,1) rectangle ++(1,1);
\path[fill=white] (16,1) rectangle ++(1,1);
\path[fill=white] (17,1) rectangle ++(1,1);
\path[fill=white] (18,1) rectangle ++(1,1);
\path[fill=white] (19,1) rectangle ++(1,1);
\path[fill=white] (20,1) rectangle ++(1,1);
\path[fill=white] (21,1) rectangle ++(1,1);
\path[fill=white] (22,1) rectangle ++(1,1);
\path[fill=white] (23,1) rectangle ++(1,1);
\path[fill=white] (24,1) rectangle ++(1,1);
\path[fill=white] (25,1) rectangle ++(1,1);
\path[fill=white] (26,1) rectangle ++(1,1);
\path[fill=black] (27,1) rectangle ++(1,1);
\path[fill=white] (0,0) rectangle ++(1,1);
\path[fill=white] (1,0) rectangle ++(1,1);
\path[fill=white] (2,0) rectangle ++(1,1);
\path[fill=white] (3,0) rectangle ++(1,1);
\path[fill=white] (4,0) rectangle ++(1,1);
\path[fill=white] (5,0) rectangle ++(1,1);
\path[fill=white] (6,0) rectangle ++(1,1);
\path[fill=white] (7,0) rectangle ++(1,1);
\path[fill=white] (8,0) rectangle ++(1,1);
\path[fill=white] (9,0) rectangle ++(1,1);
\path[fill=white] (10,0) rectangle ++(1,1);
\path[fill=white] (11,0) rectangle ++(1,1);
\path[fill=white] (12,0) rectangle ++(1,1);
\path[fill=white] (13,0) rectangle ++(1,1);
\path[fill=white] (14,0) rectangle ++(1,1);
\path[fill=white] (15,0) rectangle ++(1,1);
\path[fill=white] (16,0) rectangle ++(1,1);
\path[fill=white] (17,0) rectangle ++(1,1);
\path[fill=white] (18,0) rectangle ++(1,1);
\path[fill=white] (19,0) rectangle ++(1,1);
\path[fill=white] (20,0) rectangle ++(1,1);
\path[fill=white] (21,0) rectangle ++(1,1);
\path[fill=white] (22,0) rectangle ++(1,1);
\path[fill=white] (23,0) rectangle ++(1,1);
\path[fill=white] (24,0) rectangle ++(1,1);
\path[fill=white] (25,0) rectangle ++(1,1);
\path[fill=white] (26,0) rectangle ++(1,1);
\path[fill=white] (27,0) rectangle ++(1,1);
\end{tikzpicture}

\begin{tikzpicture}[x=0.3cm, y=0.3cm, draw=gray, very thin]
\path[fill=white] (0,27) rectangle ++(1,1);
\path[fill=black] (1,27) rectangle ++(1,1);
\path[fill=black] (2,27) rectangle ++(1,1);
\path[fill=black] (3,27) rectangle ++(1,1);
\path[fill=black] (4,27) rectangle ++(1,1);
\path[fill=black] (5,27) rectangle ++(1,1);
\path[fill=black] (6,27) rectangle ++(1,1);
\path[fill=black] (7,27) rectangle ++(1,1);
\path[fill=black] (8,27) rectangle ++(1,1);
\path[fill=black] (9,27) rectangle ++(1,1);
\path[fill=black] (10,27) rectangle ++(1,1);
\path[fill=black] (11,27) rectangle ++(1,1);
\path[fill=black] (12,27) rectangle ++(1,1);
\path[fill=black] (13,27) rectangle ++(1,1);
\path[fill=black] (14,27) rectangle ++(1,1);
\path[fill=black] (15,27) rectangle ++(1,1);
\path[fill=black] (16,27) rectangle ++(1,1);
\path[fill=black] (17,27) rectangle ++(1,1);
\path[fill=black] (18,27) rectangle ++(1,1);
\path[fill=black] (19,27) rectangle ++(1,1);
\path[fill=black] (20,27) rectangle ++(1,1);
\path[fill=black] (21,27) rectangle ++(1,1);
\path[fill=black] (22,27) rectangle ++(1,1);
\path[fill=black] (23,27) rectangle ++(1,1);
\path[fill=black] (24,27) rectangle ++(1,1);
\path[fill=black] (25,27) rectangle ++(1,1);
\path[fill=black] (26,27) rectangle ++(1,1);
\path[fill=black] (27,27) rectangle ++(1,1);
\path[fill=white] (0,26) rectangle ++(1,1);
\path[fill=white] (1,26) rectangle ++(1,1);
\path[fill=white] (2,26) rectangle ++(1,1);
\path[fill=black] (3,26) rectangle ++(1,1);
\path[fill=black] (4,26) rectangle ++(1,1);
\path[fill=white] (5,26) rectangle ++(1,1);
\path[fill=black] (6,26) rectangle ++(1,1);
\path[fill=black] (7,26) rectangle ++(1,1);
\path[fill=black] (8,26) rectangle ++(1,1);
\path[fill=white] (9,26) rectangle ++(1,1);
\path[fill=black] (10,26) rectangle ++(1,1);
\path[fill=black] (11,26) rectangle ++(1,1);
\path[fill=black] (12,26) rectangle ++(1,1);
\path[fill=white] (13,26) rectangle ++(1,1);
\path[fill=black] (14,26) rectangle ++(1,1);
\path[fill=black] (15,26) rectangle ++(1,1);
\path[fill=black] (16,26) rectangle ++(1,1);
\path[fill=white] (17,26) rectangle ++(1,1);
\path[fill=black] (18,26) rectangle ++(1,1);
\path[fill=black] (19,26) rectangle ++(1,1);
\path[fill=black] (20,26) rectangle ++(1,1);
\path[fill=white] (21,26) rectangle ++(1,1);
\path[fill=black] (22,26) rectangle ++(1,1);
\path[fill=black] (23,26) rectangle ++(1,1);
\path[fill=black] (24,26) rectangle ++(1,1);
\path[fill=black] (25,26) rectangle ++(1,1);
\path[fill=black] (26,26) rectangle ++(1,1);
\path[fill=black] (27,26) rectangle ++(1,1);
\path[fill=white] (0,25) rectangle ++(1,1);
\path[fill=white] (1,25) rectangle ++(1,1);
\path[fill=white] (2,25) rectangle ++(1,1);
\path[fill=white] (3,25) rectangle ++(1,1);
\path[fill=black] (4,25) rectangle ++(1,1);
\path[fill=black] (5,25) rectangle ++(1,1);
\path[fill=white] (6,25) rectangle ++(1,1);
\path[fill=black] (7,25) rectangle ++(1,1);
\path[fill=black] (8,25) rectangle ++(1,1);
\path[fill=black] (9,25) rectangle ++(1,1);
\path[fill=black] (10,25) rectangle ++(1,1);
\path[fill=black] (11,25) rectangle ++(1,1);
\path[fill=black] (12,25) rectangle ++(1,1);
\path[fill=black] (13,25) rectangle ++(1,1);
\path[fill=black] (14,25) rectangle ++(1,1);
\path[fill=black] (15,25) rectangle ++(1,1);
\path[fill=black] (16,25) rectangle ++(1,1);
\path[fill=black] (17,25) rectangle ++(1,1);
\path[fill=black] (18,25) rectangle ++(1,1);
\path[fill=black] (19,25) rectangle ++(1,1);
\path[fill=black] (20,25) rectangle ++(1,1);
\path[fill=black] (21,25) rectangle ++(1,1);
\path[fill=black] (22,25) rectangle ++(1,1);
\path[fill=black] (23,25) rectangle ++(1,1);
\path[fill=black] (24,25) rectangle ++(1,1);
\path[fill=black] (25,25) rectangle ++(1,1);
\path[fill=black] (26,25) rectangle ++(1,1);
\path[fill=black] (27,25) rectangle ++(1,1);
\path[fill=white] (0,24) rectangle ++(1,1);
\path[fill=white] (1,24) rectangle ++(1,1);
\path[fill=white] (2,24) rectangle ++(1,1);
\path[fill=white] (3,24) rectangle ++(1,1);
\path[fill=white] (4,24) rectangle ++(1,1);
\path[fill=white] (5,24) rectangle ++(1,1);
\path[fill=black] (6,24) rectangle ++(1,1);
\path[fill=black] (7,24) rectangle ++(1,1);
\path[fill=white] (8,24) rectangle ++(1,1);
\path[fill=white] (9,24) rectangle ++(1,1);
\path[fill=black] (10,24) rectangle ++(1,1);
\path[fill=black] (11,24) rectangle ++(1,1);
\path[fill=white] (12,24) rectangle ++(1,1);
\path[fill=white] (13,24) rectangle ++(1,1);
\path[fill=black] (14,24) rectangle ++(1,1);
\path[fill=black] (15,24) rectangle ++(1,1);
\path[fill=white] (16,24) rectangle ++(1,1);
\path[fill=white] (17,24) rectangle ++(1,1);
\path[fill=black] (18,24) rectangle ++(1,1);
\path[fill=black] (19,24) rectangle ++(1,1);
\path[fill=white] (20,24) rectangle ++(1,1);
\path[fill=white] (21,24) rectangle ++(1,1);
\path[fill=black] (22,24) rectangle ++(1,1);
\path[fill=black] (23,24) rectangle ++(1,1);
\path[fill=white] (24,24) rectangle ++(1,1);
\path[fill=black] (25,24) rectangle ++(1,1);
\path[fill=black] (26,24) rectangle ++(1,1);
\path[fill=black] (27,24) rectangle ++(1,1);
\path[fill=white] (0,23) rectangle ++(1,1);
\path[fill=white] (1,23) rectangle ++(1,1);
\path[fill=white] (2,23) rectangle ++(1,1);
\path[fill=white] (3,23) rectangle ++(1,1);
\path[fill=white] (4,23) rectangle ++(1,1);
\path[fill=white] (5,23) rectangle ++(1,1);
\path[fill=white] (6,23) rectangle ++(1,1);
\path[fill=black] (7,23) rectangle ++(1,1);
\path[fill=black] (8,23) rectangle ++(1,1);
\path[fill=white] (9,23) rectangle ++(1,1);
\path[fill=black] (10,23) rectangle ++(1,1);
\path[fill=black] (11,23) rectangle ++(1,1);
\path[fill=black] (12,23) rectangle ++(1,1);
\path[fill=white] (13,23) rectangle ++(1,1);
\path[fill=black] (14,23) rectangle ++(1,1);
\path[fill=black] (15,23) rectangle ++(1,1);
\path[fill=black] (16,23) rectangle ++(1,1);
\path[fill=white] (17,23) rectangle ++(1,1);
\path[fill=black] (18,23) rectangle ++(1,1);
\path[fill=black] (19,23) rectangle ++(1,1);
\path[fill=black] (20,23) rectangle ++(1,1);
\path[fill=white] (21,23) rectangle ++(1,1);
\path[fill=black] (22,23) rectangle ++(1,1);
\path[fill=black] (23,23) rectangle ++(1,1);
\path[fill=black] (24,23) rectangle ++(1,1);
\path[fill=black] (25,23) rectangle ++(1,1);
\path[fill=black] (26,23) rectangle ++(1,1);
\path[fill=black] (27,23) rectangle ++(1,1);
\path[fill=white] (0,22) rectangle ++(1,1);
\path[fill=white] (1,22) rectangle ++(1,1);
\path[fill=white] (2,22) rectangle ++(1,1);
\path[fill=white] (3,22) rectangle ++(1,1);
\path[fill=white] (4,22) rectangle ++(1,1);
\path[fill=white] (5,22) rectangle ++(1,1);
\path[fill=white] (6,22) rectangle ++(1,1);
\path[fill=white] (7,22) rectangle ++(1,1);
\path[fill=black] (8,22) rectangle ++(1,1);
\path[fill=black] (9,22) rectangle ++(1,1);
\path[fill=white] (10,22) rectangle ++(1,1);
\path[fill=black] (11,22) rectangle ++(1,1);
\path[fill=black] (12,22) rectangle ++(1,1);
\path[fill=black] (13,22) rectangle ++(1,1);
\path[fill=black] (14,22) rectangle ++(1,1);
\path[fill=black] (15,22) rectangle ++(1,1);
\path[fill=black] (16,22) rectangle ++(1,1);
\path[fill=black] (17,22) rectangle ++(1,1);
\path[fill=black] (18,22) rectangle ++(1,1);
\path[fill=black] (19,22) rectangle ++(1,1);
\path[fill=black] (20,22) rectangle ++(1,1);
\path[fill=black] (21,22) rectangle ++(1,1);
\path[fill=black] (22,22) rectangle ++(1,1);
\path[fill=black] (23,22) rectangle ++(1,1);
\path[fill=black] (24,22) rectangle ++(1,1);
\path[fill=black] (25,22) rectangle ++(1,1);
\path[fill=black] (26,22) rectangle ++(1,1);
\path[fill=black] (27,22) rectangle ++(1,1);
\path[fill=white] (0,21) rectangle ++(1,1);
\path[fill=white] (1,21) rectangle ++(1,1);
\path[fill=white] (2,21) rectangle ++(1,1);
\path[fill=white] (3,21) rectangle ++(1,1);
\path[fill=white] (4,21) rectangle ++(1,1);
\path[fill=white] (5,21) rectangle ++(1,1);
\path[fill=white] (6,21) rectangle ++(1,1);
\path[fill=white] (7,21) rectangle ++(1,1);
\path[fill=white] (8,21) rectangle ++(1,1);
\path[fill=white] (9,21) rectangle ++(1,1);
\path[fill=black] (10,21) rectangle ++(1,1);
\path[fill=white] (11,21) rectangle ++(1,1);
\path[fill=white] (12,21) rectangle ++(1,1);
\path[fill=white] (13,21) rectangle ++(1,1);
\path[fill=black] (14,21) rectangle ++(1,1);
\path[fill=white] (15,21) rectangle ++(1,1);
\path[fill=white] (16,21) rectangle ++(1,1);
\path[fill=white] (17,21) rectangle ++(1,1);
\path[fill=black] (18,21) rectangle ++(1,1);
\path[fill=white] (19,21) rectangle ++(1,1);
\path[fill=white] (20,21) rectangle ++(1,1);
\path[fill=white] (21,21) rectangle ++(1,1);
\path[fill=black] (22,21) rectangle ++(1,1);
\path[fill=white] (23,21) rectangle ++(1,1);
\path[fill=white] (24,21) rectangle ++(1,1);
\path[fill=black] (25,21) rectangle ++(1,1);
\path[fill=white] (26,21) rectangle ++(1,1);
\path[fill=black] (27,21) rectangle ++(1,1);
\path[fill=white] (0,20) rectangle ++(1,1);
\path[fill=white] (1,20) rectangle ++(1,1);
\path[fill=white] (2,20) rectangle ++(1,1);
\path[fill=white] (3,20) rectangle ++(1,1);
\path[fill=white] (4,20) rectangle ++(1,1);
\path[fill=white] (5,20) rectangle ++(1,1);
\path[fill=white] (6,20) rectangle ++(1,1);
\path[fill=white] (7,20) rectangle ++(1,1);
\path[fill=white] (8,20) rectangle ++(1,1);
\path[fill=white] (9,20) rectangle ++(1,1);
\path[fill=black] (10,20) rectangle ++(1,1);
\path[fill=black] (11,20) rectangle ++(1,1);
\path[fill=white] (12,20) rectangle ++(1,1);
\path[fill=white] (13,20) rectangle ++(1,1);
\path[fill=black] (14,20) rectangle ++(1,1);
\path[fill=black] (15,20) rectangle ++(1,1);
\path[fill=white] (16,20) rectangle ++(1,1);
\path[fill=white] (17,20) rectangle ++(1,1);
\path[fill=black] (18,20) rectangle ++(1,1);
\path[fill=black] (19,20) rectangle ++(1,1);
\path[fill=white] (20,20) rectangle ++(1,1);
\path[fill=white] (21,20) rectangle ++(1,1);
\path[fill=black] (22,20) rectangle ++(1,1);
\path[fill=black] (23,20) rectangle ++(1,1);
\path[fill=white] (24,20) rectangle ++(1,1);
\path[fill=black] (25,20) rectangle ++(1,1);
\path[fill=black] (26,20) rectangle ++(1,1);
\path[fill=black] (27,20) rectangle ++(1,1);
\path[fill=white] (0,19) rectangle ++(1,1);
\path[fill=white] (1,19) rectangle ++(1,1);
\path[fill=white] (2,19) rectangle ++(1,1);
\path[fill=white] (3,19) rectangle ++(1,1);
\path[fill=white] (4,19) rectangle ++(1,1);
\path[fill=white] (5,19) rectangle ++(1,1);
\path[fill=white] (6,19) rectangle ++(1,1);
\path[fill=white] (7,19) rectangle ++(1,1);
\path[fill=white] (8,19) rectangle ++(1,1);
\path[fill=white] (9,19) rectangle ++(1,1);
\path[fill=white] (10,19) rectangle ++(1,1);
\path[fill=black] (11,19) rectangle ++(1,1);
\path[fill=black] (12,19) rectangle ++(1,1);
\path[fill=white] (13,19) rectangle ++(1,1);
\path[fill=black] (14,19) rectangle ++(1,1);
\path[fill=black] (15,19) rectangle ++(1,1);
\path[fill=black] (16,19) rectangle ++(1,1);
\path[fill=white] (17,19) rectangle ++(1,1);
\path[fill=black] (18,19) rectangle ++(1,1);
\path[fill=black] (19,19) rectangle ++(1,1);
\path[fill=black] (20,19) rectangle ++(1,1);
\path[fill=white] (21,19) rectangle ++(1,1);
\path[fill=black] (22,19) rectangle ++(1,1);
\path[fill=black] (23,19) rectangle ++(1,1);
\path[fill=black] (24,19) rectangle ++(1,1);
\path[fill=black] (25,19) rectangle ++(1,1);
\path[fill=black] (26,19) rectangle ++(1,1);
\path[fill=black] (27,19) rectangle ++(1,1);
\path[fill=white] (0,18) rectangle ++(1,1);
\path[fill=white] (1,18) rectangle ++(1,1);
\path[fill=white] (2,18) rectangle ++(1,1);
\path[fill=white] (3,18) rectangle ++(1,1);
\path[fill=white] (4,18) rectangle ++(1,1);
\path[fill=white] (5,18) rectangle ++(1,1);
\path[fill=white] (6,18) rectangle ++(1,1);
\path[fill=white] (7,18) rectangle ++(1,1);
\path[fill=white] (8,18) rectangle ++(1,1);
\path[fill=white] (9,18) rectangle ++(1,1);
\path[fill=white] (10,18) rectangle ++(1,1);
\path[fill=white] (11,18) rectangle ++(1,1);
\path[fill=black] (12,18) rectangle ++(1,1);
\path[fill=black] (13,18) rectangle ++(1,1);
\path[fill=white] (14,18) rectangle ++(1,1);
\path[fill=black] (15,18) rectangle ++(1,1);
\path[fill=black] (16,18) rectangle ++(1,1);
\path[fill=black] (17,18) rectangle ++(1,1);
\path[fill=black] (18,18) rectangle ++(1,1);
\path[fill=black] (19,18) rectangle ++(1,1);
\path[fill=black] (20,18) rectangle ++(1,1);
\path[fill=black] (21,18) rectangle ++(1,1);
\path[fill=black] (22,18) rectangle ++(1,1);
\path[fill=black] (23,18) rectangle ++(1,1);
\path[fill=black] (24,18) rectangle ++(1,1);
\path[fill=black] (25,18) rectangle ++(1,1);
\path[fill=black] (26,18) rectangle ++(1,1);
\path[fill=black] (27,18) rectangle ++(1,1);
\path[fill=white] (0,17) rectangle ++(1,1);
\path[fill=white] (1,17) rectangle ++(1,1);
\path[fill=white] (2,17) rectangle ++(1,1);
\path[fill=white] (3,17) rectangle ++(1,1);
\path[fill=white] (4,17) rectangle ++(1,1);
\path[fill=white] (5,17) rectangle ++(1,1);
\path[fill=white] (6,17) rectangle ++(1,1);
\path[fill=white] (7,17) rectangle ++(1,1);
\path[fill=white] (8,17) rectangle ++(1,1);
\path[fill=white] (9,17) rectangle ++(1,1);
\path[fill=white] (10,17) rectangle ++(1,1);
\path[fill=white] (11,17) rectangle ++(1,1);
\path[fill=white] (12,17) rectangle ++(1,1);
\path[fill=white] (13,17) rectangle ++(1,1);
\path[fill=black] (14,17) rectangle ++(1,1);
\path[fill=white] (15,17) rectangle ++(1,1);
\path[fill=white] (16,17) rectangle ++(1,1);
\path[fill=white] (17,17) rectangle ++(1,1);
\path[fill=black] (18,17) rectangle ++(1,1);
\path[fill=white] (19,17) rectangle ++(1,1);
\path[fill=white] (20,17) rectangle ++(1,1);
\path[fill=white] (21,17) rectangle ++(1,1);
\path[fill=black] (22,17) rectangle ++(1,1);
\path[fill=white] (23,17) rectangle ++(1,1);
\path[fill=white] (24,17) rectangle ++(1,1);
\path[fill=black] (25,17) rectangle ++(1,1);
\path[fill=white] (26,17) rectangle ++(1,1);
\path[fill=black] (27,17) rectangle ++(1,1);
\path[fill=white] (0,16) rectangle ++(1,1);
\path[fill=white] (1,16) rectangle ++(1,1);
\path[fill=white] (2,16) rectangle ++(1,1);
\path[fill=white] (3,16) rectangle ++(1,1);
\path[fill=white] (4,16) rectangle ++(1,1);
\path[fill=white] (5,16) rectangle ++(1,1);
\path[fill=white] (6,16) rectangle ++(1,1);
\path[fill=white] (7,16) rectangle ++(1,1);
\path[fill=white] (8,16) rectangle ++(1,1);
\path[fill=white] (9,16) rectangle ++(1,1);
\path[fill=white] (10,16) rectangle ++(1,1);
\path[fill=white] (11,16) rectangle ++(1,1);
\path[fill=white] (12,16) rectangle ++(1,1);
\path[fill=white] (13,16) rectangle ++(1,1);
\path[fill=black] (14,16) rectangle ++(1,1);
\path[fill=black] (15,16) rectangle ++(1,1);
\path[fill=white] (16,16) rectangle ++(1,1);
\path[fill=white] (17,16) rectangle ++(1,1);
\path[fill=black] (18,16) rectangle ++(1,1);
\path[fill=black] (19,16) rectangle ++(1,1);
\path[fill=white] (20,16) rectangle ++(1,1);
\path[fill=white] (21,16) rectangle ++(1,1);
\path[fill=black] (22,16) rectangle ++(1,1);
\path[fill=black] (23,16) rectangle ++(1,1);
\path[fill=white] (24,16) rectangle ++(1,1);
\path[fill=black] (25,16) rectangle ++(1,1);
\path[fill=black] (26,16) rectangle ++(1,1);
\path[fill=black] (27,16) rectangle ++(1,1);
\path[fill=white] (0,15) rectangle ++(1,1);
\path[fill=white] (1,15) rectangle ++(1,1);
\path[fill=white] (2,15) rectangle ++(1,1);
\path[fill=white] (3,15) rectangle ++(1,1);
\path[fill=white] (4,15) rectangle ++(1,1);
\path[fill=white] (5,15) rectangle ++(1,1);
\path[fill=white] (6,15) rectangle ++(1,1);
\path[fill=white] (7,15) rectangle ++(1,1);
\path[fill=white] (8,15) rectangle ++(1,1);
\path[fill=white] (9,15) rectangle ++(1,1);
\path[fill=white] (10,15) rectangle ++(1,1);
\path[fill=white] (11,15) rectangle ++(1,1);
\path[fill=white] (12,15) rectangle ++(1,1);
\path[fill=white] (13,15) rectangle ++(1,1);
\path[fill=white] (14,15) rectangle ++(1,1);
\path[fill=black] (15,15) rectangle ++(1,1);
\path[fill=black] (16,15) rectangle ++(1,1);
\path[fill=white] (17,15) rectangle ++(1,1);
\path[fill=black] (18,15) rectangle ++(1,1);
\path[fill=black] (19,15) rectangle ++(1,1);
\path[fill=black] (20,15) rectangle ++(1,1);
\path[fill=white] (21,15) rectangle ++(1,1);
\path[fill=black] (22,15) rectangle ++(1,1);
\path[fill=black] (23,15) rectangle ++(1,1);
\path[fill=black] (24,15) rectangle ++(1,1);
\path[fill=black] (25,15) rectangle ++(1,1);
\path[fill=black] (26,15) rectangle ++(1,1);
\path[fill=black] (27,15) rectangle ++(1,1);
\path[fill=white] (0,14) rectangle ++(1,1);
\path[fill=white] (1,14) rectangle ++(1,1);
\path[fill=white] (2,14) rectangle ++(1,1);
\path[fill=white] (3,14) rectangle ++(1,1);
\path[fill=white] (4,14) rectangle ++(1,1);
\path[fill=white] (5,14) rectangle ++(1,1);
\path[fill=white] (6,14) rectangle ++(1,1);
\path[fill=white] (7,14) rectangle ++(1,1);
\path[fill=white] (8,14) rectangle ++(1,1);
\path[fill=white] (9,14) rectangle ++(1,1);
\path[fill=white] (10,14) rectangle ++(1,1);
\path[fill=white] (11,14) rectangle ++(1,1);
\path[fill=white] (12,14) rectangle ++(1,1);
\path[fill=white] (13,14) rectangle ++(1,1);
\path[fill=white] (14,14) rectangle ++(1,1);
\path[fill=white] (15,14) rectangle ++(1,1);
\path[fill=black] (16,14) rectangle ++(1,1);
\path[fill=black] (17,14) rectangle ++(1,1);
\path[fill=white] (18,14) rectangle ++(1,1);
\path[fill=black] (19,14) rectangle ++(1,1);
\path[fill=black] (20,14) rectangle ++(1,1);
\path[fill=black] (21,14) rectangle ++(1,1);
\path[fill=black] (22,14) rectangle ++(1,1);
\path[fill=black] (23,14) rectangle ++(1,1);
\path[fill=black] (24,14) rectangle ++(1,1);
\path[fill=black] (25,14) rectangle ++(1,1);
\path[fill=black] (26,14) rectangle ++(1,1);
\path[fill=black] (27,14) rectangle ++(1,1);
\path[fill=white] (0,13) rectangle ++(1,1);
\path[fill=white] (1,13) rectangle ++(1,1);
\path[fill=white] (2,13) rectangle ++(1,1);
\path[fill=white] (3,13) rectangle ++(1,1);
\path[fill=white] (4,13) rectangle ++(1,1);
\path[fill=white] (5,13) rectangle ++(1,1);
\path[fill=white] (6,13) rectangle ++(1,1);
\path[fill=white] (7,13) rectangle ++(1,1);
\path[fill=white] (8,13) rectangle ++(1,1);
\path[fill=white] (9,13) rectangle ++(1,1);
\path[fill=white] (10,13) rectangle ++(1,1);
\path[fill=white] (11,13) rectangle ++(1,1);
\path[fill=white] (12,13) rectangle ++(1,1);
\path[fill=white] (13,13) rectangle ++(1,1);
\path[fill=white] (14,13) rectangle ++(1,1);
\path[fill=white] (15,13) rectangle ++(1,1);
\path[fill=white] (16,13) rectangle ++(1,1);
\path[fill=white] (17,13) rectangle ++(1,1);
\path[fill=black] (18,13) rectangle ++(1,1);
\path[fill=white] (19,13) rectangle ++(1,1);
\path[fill=white] (20,13) rectangle ++(1,1);
\path[fill=white] (21,13) rectangle ++(1,1);
\path[fill=black] (22,13) rectangle ++(1,1);
\path[fill=white] (23,13) rectangle ++(1,1);
\path[fill=white] (24,13) rectangle ++(1,1);
\path[fill=black] (25,13) rectangle ++(1,1);
\path[fill=white] (26,13) rectangle ++(1,1);
\path[fill=black] (27,13) rectangle ++(1,1);
\path[fill=white] (0,12) rectangle ++(1,1);
\path[fill=white] (1,12) rectangle ++(1,1);
\path[fill=white] (2,12) rectangle ++(1,1);
\path[fill=white] (3,12) rectangle ++(1,1);
\path[fill=white] (4,12) rectangle ++(1,1);
\path[fill=white] (5,12) rectangle ++(1,1);
\path[fill=white] (6,12) rectangle ++(1,1);
\path[fill=white] (7,12) rectangle ++(1,1);
\path[fill=white] (8,12) rectangle ++(1,1);
\path[fill=white] (9,12) rectangle ++(1,1);
\path[fill=white] (10,12) rectangle ++(1,1);
\path[fill=white] (11,12) rectangle ++(1,1);
\path[fill=white] (12,12) rectangle ++(1,1);
\path[fill=white] (13,12) rectangle ++(1,1);
\path[fill=white] (14,12) rectangle ++(1,1);
\path[fill=white] (15,12) rectangle ++(1,1);
\path[fill=white] (16,12) rectangle ++(1,1);
\path[fill=white] (17,12) rectangle ++(1,1);
\path[fill=black] (18,12) rectangle ++(1,1);
\path[fill=black] (19,12) rectangle ++(1,1);
\path[fill=white] (20,12) rectangle ++(1,1);
\path[fill=white] (21,12) rectangle ++(1,1);
\path[fill=black] (22,12) rectangle ++(1,1);
\path[fill=black] (23,12) rectangle ++(1,1);
\path[fill=white] (24,12) rectangle ++(1,1);
\path[fill=black] (25,12) rectangle ++(1,1);
\path[fill=black] (26,12) rectangle ++(1,1);
\path[fill=black] (27,12) rectangle ++(1,1);
\path[fill=white] (0,11) rectangle ++(1,1);
\path[fill=white] (1,11) rectangle ++(1,1);
\path[fill=white] (2,11) rectangle ++(1,1);
\path[fill=white] (3,11) rectangle ++(1,1);
\path[fill=white] (4,11) rectangle ++(1,1);
\path[fill=white] (5,11) rectangle ++(1,1);
\path[fill=white] (6,11) rectangle ++(1,1);
\path[fill=white] (7,11) rectangle ++(1,1);
\path[fill=white] (8,11) rectangle ++(1,1);
\path[fill=white] (9,11) rectangle ++(1,1);
\path[fill=white] (10,11) rectangle ++(1,1);
\path[fill=white] (11,11) rectangle ++(1,1);
\path[fill=white] (12,11) rectangle ++(1,1);
\path[fill=white] (13,11) rectangle ++(1,1);
\path[fill=white] (14,11) rectangle ++(1,1);
\path[fill=white] (15,11) rectangle ++(1,1);
\path[fill=white] (16,11) rectangle ++(1,1);
\path[fill=white] (17,11) rectangle ++(1,1);
\path[fill=white] (18,11) rectangle ++(1,1);
\path[fill=black] (19,11) rectangle ++(1,1);
\path[fill=black] (20,11) rectangle ++(1,1);
\path[fill=white] (21,11) rectangle ++(1,1);
\path[fill=black] (22,11) rectangle ++(1,1);
\path[fill=black] (23,11) rectangle ++(1,1);
\path[fill=black] (24,11) rectangle ++(1,1);
\path[fill=black] (25,11) rectangle ++(1,1);
\path[fill=black] (26,11) rectangle ++(1,1);
\path[fill=black] (27,11) rectangle ++(1,1);
\path[fill=white] (0,10) rectangle ++(1,1);
\path[fill=white] (1,10) rectangle ++(1,1);
\path[fill=white] (2,10) rectangle ++(1,1);
\path[fill=white] (3,10) rectangle ++(1,1);
\path[fill=white] (4,10) rectangle ++(1,1);
\path[fill=white] (5,10) rectangle ++(1,1);
\path[fill=white] (6,10) rectangle ++(1,1);
\path[fill=white] (7,10) rectangle ++(1,1);
\path[fill=white] (8,10) rectangle ++(1,1);
\path[fill=white] (9,10) rectangle ++(1,1);
\path[fill=white] (10,10) rectangle ++(1,1);
\path[fill=white] (11,10) rectangle ++(1,1);
\path[fill=white] (12,10) rectangle ++(1,1);
\path[fill=white] (13,10) rectangle ++(1,1);
\path[fill=white] (14,10) rectangle ++(1,1);
\path[fill=white] (15,10) rectangle ++(1,1);
\path[fill=white] (16,10) rectangle ++(1,1);
\path[fill=white] (17,10) rectangle ++(1,1);
\path[fill=white] (18,10) rectangle ++(1,1);
\path[fill=white] (19,10) rectangle ++(1,1);
\path[fill=black] (20,10) rectangle ++(1,1);
\path[fill=black] (21,10) rectangle ++(1,1);
\path[fill=white] (22,10) rectangle ++(1,1);
\path[fill=black] (23,10) rectangle ++(1,1);
\path[fill=black] (24,10) rectangle ++(1,1);
\path[fill=black] (25,10) rectangle ++(1,1);
\path[fill=black] (26,10) rectangle ++(1,1);
\path[fill=black] (27,10) rectangle ++(1,1);
\path[fill=white] (0,9) rectangle ++(1,1);
\path[fill=white] (1,9) rectangle ++(1,1);
\path[fill=white] (2,9) rectangle ++(1,1);
\path[fill=white] (3,9) rectangle ++(1,1);
\path[fill=white] (4,9) rectangle ++(1,1);
\path[fill=white] (5,9) rectangle ++(1,1);
\path[fill=white] (6,9) rectangle ++(1,1);
\path[fill=white] (7,9) rectangle ++(1,1);
\path[fill=white] (8,9) rectangle ++(1,1);
\path[fill=white] (9,9) rectangle ++(1,1);
\path[fill=white] (10,9) rectangle ++(1,1);
\path[fill=white] (11,9) rectangle ++(1,1);
\path[fill=white] (12,9) rectangle ++(1,1);
\path[fill=white] (13,9) rectangle ++(1,1);
\path[fill=white] (14,9) rectangle ++(1,1);
\path[fill=white] (15,9) rectangle ++(1,1);
\path[fill=white] (16,9) rectangle ++(1,1);
\path[fill=white] (17,9) rectangle ++(1,1);
\path[fill=white] (18,9) rectangle ++(1,1);
\path[fill=white] (19,9) rectangle ++(1,1);
\path[fill=white] (20,9) rectangle ++(1,1);
\path[fill=white] (21,9) rectangle ++(1,1);
\path[fill=black] (22,9) rectangle ++(1,1);
\path[fill=white] (23,9) rectangle ++(1,1);
\path[fill=white] (24,9) rectangle ++(1,1);
\path[fill=black] (25,9) rectangle ++(1,1);
\path[fill=white] (26,9) rectangle ++(1,1);
\path[fill=black] (27,9) rectangle ++(1,1);
\path[fill=white] (0,8) rectangle ++(1,1);
\path[fill=white] (1,8) rectangle ++(1,1);
\path[fill=white] (2,8) rectangle ++(1,1);
\path[fill=white] (3,8) rectangle ++(1,1);
\path[fill=white] (4,8) rectangle ++(1,1);
\path[fill=white] (5,8) rectangle ++(1,1);
\path[fill=white] (6,8) rectangle ++(1,1);
\path[fill=white] (7,8) rectangle ++(1,1);
\path[fill=white] (8,8) rectangle ++(1,1);
\path[fill=white] (9,8) rectangle ++(1,1);
\path[fill=white] (10,8) rectangle ++(1,1);
\path[fill=white] (11,8) rectangle ++(1,1);
\path[fill=white] (12,8) rectangle ++(1,1);
\path[fill=white] (13,8) rectangle ++(1,1);
\path[fill=white] (14,8) rectangle ++(1,1);
\path[fill=white] (15,8) rectangle ++(1,1);
\path[fill=white] (16,8) rectangle ++(1,1);
\path[fill=white] (17,8) rectangle ++(1,1);
\path[fill=white] (18,8) rectangle ++(1,1);
\path[fill=white] (19,8) rectangle ++(1,1);
\path[fill=white] (20,8) rectangle ++(1,1);
\path[fill=white] (21,8) rectangle ++(1,1);
\path[fill=black] (22,8) rectangle ++(1,1);
\path[fill=black] (23,8) rectangle ++(1,1);
\path[fill=white] (24,8) rectangle ++(1,1);
\path[fill=black] (25,8) rectangle ++(1,1);
\path[fill=black] (26,8) rectangle ++(1,1);
\path[fill=black] (27,8) rectangle ++(1,1);
\path[fill=white] (0,7) rectangle ++(1,1);
\path[fill=white] (1,7) rectangle ++(1,1);
\path[fill=white] (2,7) rectangle ++(1,1);
\path[fill=white] (3,7) rectangle ++(1,1);
\path[fill=white] (4,7) rectangle ++(1,1);
\path[fill=white] (5,7) rectangle ++(1,1);
\path[fill=white] (6,7) rectangle ++(1,1);
\path[fill=white] (7,7) rectangle ++(1,1);
\path[fill=white] (8,7) rectangle ++(1,1);
\path[fill=white] (9,7) rectangle ++(1,1);
\path[fill=white] (10,7) rectangle ++(1,1);
\path[fill=white] (11,7) rectangle ++(1,1);
\path[fill=white] (12,7) rectangle ++(1,1);
\path[fill=white] (13,7) rectangle ++(1,1);
\path[fill=white] (14,7) rectangle ++(1,1);
\path[fill=white] (15,7) rectangle ++(1,1);
\path[fill=white] (16,7) rectangle ++(1,1);
\path[fill=white] (17,7) rectangle ++(1,1);
\path[fill=white] (18,7) rectangle ++(1,1);
\path[fill=white] (19,7) rectangle ++(1,1);
\path[fill=white] (20,7) rectangle ++(1,1);
\path[fill=white] (21,7) rectangle ++(1,1);
\path[fill=white] (22,7) rectangle ++(1,1);
\path[fill=black] (23,7) rectangle ++(1,1);
\path[fill=black] (24,7) rectangle ++(1,1);
\path[fill=black] (25,7) rectangle ++(1,1);
\path[fill=black] (26,7) rectangle ++(1,1);
\path[fill=black] (27,7) rectangle ++(1,1);
\path[fill=white] (0,6) rectangle ++(1,1);
\path[fill=white] (1,6) rectangle ++(1,1);
\path[fill=white] (2,6) rectangle ++(1,1);
\path[fill=white] (3,6) rectangle ++(1,1);
\path[fill=white] (4,6) rectangle ++(1,1);
\path[fill=white] (5,6) rectangle ++(1,1);
\path[fill=white] (6,6) rectangle ++(1,1);
\path[fill=white] (7,6) rectangle ++(1,1);
\path[fill=white] (8,6) rectangle ++(1,1);
\path[fill=white] (9,6) rectangle ++(1,1);
\path[fill=white] (10,6) rectangle ++(1,1);
\path[fill=white] (11,6) rectangle ++(1,1);
\path[fill=white] (12,6) rectangle ++(1,1);
\path[fill=white] (13,6) rectangle ++(1,1);
\path[fill=white] (14,6) rectangle ++(1,1);
\path[fill=white] (15,6) rectangle ++(1,1);
\path[fill=white] (16,6) rectangle ++(1,1);
\path[fill=white] (17,6) rectangle ++(1,1);
\path[fill=white] (18,6) rectangle ++(1,1);
\path[fill=white] (19,6) rectangle ++(1,1);
\path[fill=white] (20,6) rectangle ++(1,1);
\path[fill=white] (21,6) rectangle ++(1,1);
\path[fill=white] (22,6) rectangle ++(1,1);
\path[fill=white] (23,6) rectangle ++(1,1);
\path[fill=black] (24,6) rectangle ++(1,1);
\path[fill=white] (25,6) rectangle ++(1,1);
\path[fill=black] (26,6) rectangle ++(1,1);
\path[fill=black] (27,6) rectangle ++(1,1);
\path[fill=white] (0,5) rectangle ++(1,1);
\path[fill=white] (1,5) rectangle ++(1,1);
\path[fill=white] (2,5) rectangle ++(1,1);
\path[fill=white] (3,5) rectangle ++(1,1);
\path[fill=white] (4,5) rectangle ++(1,1);
\path[fill=white] (5,5) rectangle ++(1,1);
\path[fill=white] (6,5) rectangle ++(1,1);
\path[fill=white] (7,5) rectangle ++(1,1);
\path[fill=white] (8,5) rectangle ++(1,1);
\path[fill=white] (9,5) rectangle ++(1,1);
\path[fill=white] (10,5) rectangle ++(1,1);
\path[fill=white] (11,5) rectangle ++(1,1);
\path[fill=white] (12,5) rectangle ++(1,1);
\path[fill=white] (13,5) rectangle ++(1,1);
\path[fill=white] (14,5) rectangle ++(1,1);
\path[fill=white] (15,5) rectangle ++(1,1);
\path[fill=white] (16,5) rectangle ++(1,1);
\path[fill=white] (17,5) rectangle ++(1,1);
\path[fill=white] (18,5) rectangle ++(1,1);
\path[fill=white] (19,5) rectangle ++(1,1);
\path[fill=white] (20,5) rectangle ++(1,1);
\path[fill=white] (21,5) rectangle ++(1,1);
\path[fill=white] (22,5) rectangle ++(1,1);
\path[fill=white] (23,5) rectangle ++(1,1);
\path[fill=white] (24,5) rectangle ++(1,1);
\path[fill=black] (25,5) rectangle ++(1,1);
\path[fill=white] (26,5) rectangle ++(1,1);
\path[fill=black] (27,5) rectangle ++(1,1);
\path[fill=white] (0,4) rectangle ++(1,1);
\path[fill=white] (1,4) rectangle ++(1,1);
\path[fill=white] (2,4) rectangle ++(1,1);
\path[fill=white] (3,4) rectangle ++(1,1);
\path[fill=white] (4,4) rectangle ++(1,1);
\path[fill=white] (5,4) rectangle ++(1,1);
\path[fill=white] (6,4) rectangle ++(1,1);
\path[fill=white] (7,4) rectangle ++(1,1);
\path[fill=white] (8,4) rectangle ++(1,1);
\path[fill=white] (9,4) rectangle ++(1,1);
\path[fill=white] (10,4) rectangle ++(1,1);
\path[fill=white] (11,4) rectangle ++(1,1);
\path[fill=white] (12,4) rectangle ++(1,1);
\path[fill=white] (13,4) rectangle ++(1,1);
\path[fill=white] (14,4) rectangle ++(1,1);
\path[fill=white] (15,4) rectangle ++(1,1);
\path[fill=white] (16,4) rectangle ++(1,1);
\path[fill=white] (17,4) rectangle ++(1,1);
\path[fill=white] (18,4) rectangle ++(1,1);
\path[fill=white] (19,4) rectangle ++(1,1);
\path[fill=white] (20,4) rectangle ++(1,1);
\path[fill=white] (21,4) rectangle ++(1,1);
\path[fill=white] (22,4) rectangle ++(1,1);
\path[fill=white] (23,4) rectangle ++(1,1);
\path[fill=white] (24,4) rectangle ++(1,1);
\path[fill=black] (25,4) rectangle ++(1,1);
\path[fill=black] (26,4) rectangle ++(1,1);
\path[fill=black] (27,4) rectangle ++(1,1);
\path[fill=white] (0,3) rectangle ++(1,1);
\path[fill=white] (1,3) rectangle ++(1,1);
\path[fill=white] (2,3) rectangle ++(1,1);
\path[fill=white] (3,3) rectangle ++(1,1);
\path[fill=white] (4,3) rectangle ++(1,1);
\path[fill=white] (5,3) rectangle ++(1,1);
\path[fill=white] (6,3) rectangle ++(1,1);
\path[fill=white] (7,3) rectangle ++(1,1);
\path[fill=white] (8,3) rectangle ++(1,1);
\path[fill=white] (9,3) rectangle ++(1,1);
\path[fill=white] (10,3) rectangle ++(1,1);
\path[fill=white] (11,3) rectangle ++(1,1);
\path[fill=white] (12,3) rectangle ++(1,1);
\path[fill=white] (13,3) rectangle ++(1,1);
\path[fill=white] (14,3) rectangle ++(1,1);
\path[fill=white] (15,3) rectangle ++(1,1);
\path[fill=white] (16,3) rectangle ++(1,1);
\path[fill=white] (17,3) rectangle ++(1,1);
\path[fill=white] (18,3) rectangle ++(1,1);
\path[fill=white] (19,3) rectangle ++(1,1);
\path[fill=white] (20,3) rectangle ++(1,1);
\path[fill=white] (21,3) rectangle ++(1,1);
\path[fill=white] (22,3) rectangle ++(1,1);
\path[fill=white] (23,3) rectangle ++(1,1);
\path[fill=white] (24,3) rectangle ++(1,1);
\path[fill=white] (25,3) rectangle ++(1,1);
\path[fill=black] (26,3) rectangle ++(1,1);
\path[fill=black] (27,3) rectangle ++(1,1);
\path[fill=white] (0,2) rectangle ++(1,1);
\path[fill=white] (1,2) rectangle ++(1,1);
\path[fill=white] (2,2) rectangle ++(1,1);
\path[fill=white] (3,2) rectangle ++(1,1);
\path[fill=white] (4,2) rectangle ++(1,1);
\path[fill=white] (5,2) rectangle ++(1,1);
\path[fill=white] (6,2) rectangle ++(1,1);
\path[fill=white] (7,2) rectangle ++(1,1);
\path[fill=white] (8,2) rectangle ++(1,1);
\path[fill=white] (9,2) rectangle ++(1,1);
\path[fill=white] (10,2) rectangle ++(1,1);
\path[fill=white] (11,2) rectangle ++(1,1);
\path[fill=white] (12,2) rectangle ++(1,1);
\path[fill=white] (13,2) rectangle ++(1,1);
\path[fill=white] (14,2) rectangle ++(1,1);
\path[fill=white] (15,2) rectangle ++(1,1);
\path[fill=white] (16,2) rectangle ++(1,1);
\path[fill=white] (17,2) rectangle ++(1,1);
\path[fill=white] (18,2) rectangle ++(1,1);
\path[fill=white] (19,2) rectangle ++(1,1);
\path[fill=white] (20,2) rectangle ++(1,1);
\path[fill=white] (21,2) rectangle ++(1,1);
\path[fill=white] (22,2) rectangle ++(1,1);
\path[fill=white] (23,2) rectangle ++(1,1);
\path[fill=white] (24,2) rectangle ++(1,1);
\path[fill=white] (25,2) rectangle ++(1,1);
\path[fill=white] (26,2) rectangle ++(1,1);
\path[fill=black] (27,2) rectangle ++(1,1);
\path[fill=white] (0,1) rectangle ++(1,1);
\path[fill=white] (1,1) rectangle ++(1,1);
\path[fill=white] (2,1) rectangle ++(1,1);
\path[fill=white] (3,1) rectangle ++(1,1);
\path[fill=white] (4,1) rectangle ++(1,1);
\path[fill=white] (5,1) rectangle ++(1,1);
\path[fill=white] (6,1) rectangle ++(1,1);
\path[fill=white] (7,1) rectangle ++(1,1);
\path[fill=white] (8,1) rectangle ++(1,1);
\path[fill=white] (9,1) rectangle ++(1,1);
\path[fill=white] (10,1) rectangle ++(1,1);
\path[fill=white] (11,1) rectangle ++(1,1);
\path[fill=white] (12,1) rectangle ++(1,1);
\path[fill=white] (13,1) rectangle ++(1,1);
\path[fill=white] (14,1) rectangle ++(1,1);
\path[fill=white] (15,1) rectangle ++(1,1);
\path[fill=white] (16,1) rectangle ++(1,1);
\path[fill=white] (17,1) rectangle ++(1,1);
\path[fill=white] (18,1) rectangle ++(1,1);
\path[fill=white] (19,1) rectangle ++(1,1);
\path[fill=white] (20,1) rectangle ++(1,1);
\path[fill=white] (21,1) rectangle ++(1,1);
\path[fill=white] (22,1) rectangle ++(1,1);
\path[fill=white] (23,1) rectangle ++(1,1);
\path[fill=white] (24,1) rectangle ++(1,1);
\path[fill=white] (25,1) rectangle ++(1,1);
\path[fill=white] (26,1) rectangle ++(1,1);
\path[fill=black] (27,1) rectangle ++(1,1);
\path[fill=white] (0,0) rectangle ++(1,1);
\path[fill=white] (1,0) rectangle ++(1,1);
\path[fill=white] (2,0) rectangle ++(1,1);
\path[fill=white] (3,0) rectangle ++(1,1);
\path[fill=white] (4,0) rectangle ++(1,1);
\path[fill=white] (5,0) rectangle ++(1,1);
\path[fill=white] (6,0) rectangle ++(1,1);
\path[fill=white] (7,0) rectangle ++(1,1);
\path[fill=white] (8,0) rectangle ++(1,1);
\path[fill=white] (9,0) rectangle ++(1,1);
\path[fill=white] (10,0) rectangle ++(1,1);
\path[fill=white] (11,0) rectangle ++(1,1);
\path[fill=white] (12,0) rectangle ++(1,1);
\path[fill=white] (13,0) rectangle ++(1,1);
\path[fill=white] (14,0) rectangle ++(1,1);
\path[fill=white] (15,0) rectangle ++(1,1);
\path[fill=white] (16,0) rectangle ++(1,1);
\path[fill=white] (17,0) rectangle ++(1,1);
\path[fill=white] (18,0) rectangle ++(1,1);
\path[fill=white] (19,0) rectangle ++(1,1);
\path[fill=white] (20,0) rectangle ++(1,1);
\path[fill=white] (21,0) rectangle ++(1,1);
\path[fill=white] (22,0) rectangle ++(1,1);
\path[fill=white] (23,0) rectangle ++(1,1);
\path[fill=white] (24,0) rectangle ++(1,1);
\path[fill=white] (25,0) rectangle ++(1,1);
\path[fill=white] (26,0) rectangle ++(1,1);
\path[fill=white] (27,0) rectangle ++(1,1);
\end{tikzpicture}}}
\end{center}
\caption{Lev(|\sigma|=6, \Delta=3) automaton, adjacency and reachability matrix.}
\end{figure}

\begin{figure}[H]
\begin{center}
\adjustbox{valign=m}{\includegraphics[height=4cm]{figures/lev_nfa_6x4.pdf}}
\adjustbox{valign=m}{\resizebox{0.4\textwidth}{!}{\begin{tikzpicture}[x=0.3cm, y=0.3cm, draw=gray, very thin]
\path[fill=white] (0,34) rectangle ++(1,1);
\path[fill=black] (1,34) rectangle ++(1,1);
\path[fill=black] (2,34) rectangle ++(1,1);
\path[fill=white] (3,34) rectangle ++(1,1);
\path[fill=black] (4,34) rectangle ++(1,1);
\path[fill=white] (5,34) rectangle ++(1,1);
\path[fill=white] (6,34) rectangle ++(1,1);
\path[fill=white] (7,34) rectangle ++(1,1);
\path[fill=black] (8,34) rectangle ++(1,1);
\path[fill=white] (9,34) rectangle ++(1,1);
\path[fill=white] (10,34) rectangle ++(1,1);
\path[fill=white] (11,34) rectangle ++(1,1);
\path[fill=white] (12,34) rectangle ++(1,1);
\path[fill=white] (13,34) rectangle ++(1,1);
\path[fill=white] (14,34) rectangle ++(1,1);
\path[fill=white] (15,34) rectangle ++(1,1);
\path[fill=white] (16,34) rectangle ++(1,1);
\path[fill=black] (17,34) rectangle ++(1,1);
\path[fill=white] (18,34) rectangle ++(1,1);
\path[fill=white] (19,34) rectangle ++(1,1);
\path[fill=white] (20,34) rectangle ++(1,1);
\path[fill=white] (21,34) rectangle ++(1,1);
\path[fill=white] (22,34) rectangle ++(1,1);
\path[fill=white] (23,34) rectangle ++(1,1);
\path[fill=white] (24,34) rectangle ++(1,1);
\path[fill=white] (25,34) rectangle ++(1,1);
\path[fill=black] (26,34) rectangle ++(1,1);
\path[fill=white] (27,34) rectangle ++(1,1);
\path[fill=white] (28,34) rectangle ++(1,1);
\path[fill=white] (29,34) rectangle ++(1,1);
\path[fill=white] (30,34) rectangle ++(1,1);
\path[fill=white] (31,34) rectangle ++(1,1);
\path[fill=black] (32,34) rectangle ++(1,1);
\path[fill=white] (33,34) rectangle ++(1,1);
\path[fill=white] (34,34) rectangle ++(1,1);
\path[fill=white] (0,33) rectangle ++(1,1);
\path[fill=white] (1,33) rectangle ++(1,1);
\path[fill=white] (2,33) rectangle ++(1,1);
\path[fill=black] (3,33) rectangle ++(1,1);
\path[fill=black] (4,33) rectangle ++(1,1);
\path[fill=white] (5,33) rectangle ++(1,1);
\path[fill=white] (6,33) rectangle ++(1,1);
\path[fill=black] (7,33) rectangle ++(1,1);
\path[fill=white] (8,33) rectangle ++(1,1);
\path[fill=white] (9,33) rectangle ++(1,1);
\path[fill=white] (10,33) rectangle ++(1,1);
\path[fill=white] (11,33) rectangle ++(1,1);
\path[fill=black] (12,33) rectangle ++(1,1);
\path[fill=white] (13,33) rectangle ++(1,1);
\path[fill=white] (14,33) rectangle ++(1,1);
\path[fill=white] (15,33) rectangle ++(1,1);
\path[fill=white] (16,33) rectangle ++(1,1);
\path[fill=white] (17,33) rectangle ++(1,1);
\path[fill=white] (18,33) rectangle ++(1,1);
\path[fill=white] (19,33) rectangle ++(1,1);
\path[fill=white] (20,33) rectangle ++(1,1);
\path[fill=black] (21,33) rectangle ++(1,1);
\path[fill=white] (22,33) rectangle ++(1,1);
\path[fill=white] (23,33) rectangle ++(1,1);
\path[fill=white] (24,33) rectangle ++(1,1);
\path[fill=white] (25,33) rectangle ++(1,1);
\path[fill=white] (26,33) rectangle ++(1,1);
\path[fill=white] (27,33) rectangle ++(1,1);
\path[fill=white] (28,33) rectangle ++(1,1);
\path[fill=black] (29,33) rectangle ++(1,1);
\path[fill=white] (30,33) rectangle ++(1,1);
\path[fill=white] (31,33) rectangle ++(1,1);
\path[fill=white] (32,33) rectangle ++(1,1);
\path[fill=white] (33,33) rectangle ++(1,1);
\path[fill=white] (34,33) rectangle ++(1,1);
\path[fill=white] (0,32) rectangle ++(1,1);
\path[fill=white] (1,32) rectangle ++(1,1);
\path[fill=white] (2,32) rectangle ++(1,1);
\path[fill=white] (3,32) rectangle ++(1,1);
\path[fill=black] (4,32) rectangle ++(1,1);
\path[fill=black] (5,32) rectangle ++(1,1);
\path[fill=white] (6,32) rectangle ++(1,1);
\path[fill=white] (7,32) rectangle ++(1,1);
\path[fill=black] (8,32) rectangle ++(1,1);
\path[fill=white] (9,32) rectangle ++(1,1);
\path[fill=white] (10,32) rectangle ++(1,1);
\path[fill=white] (11,32) rectangle ++(1,1);
\path[fill=white] (12,32) rectangle ++(1,1);
\path[fill=black] (13,32) rectangle ++(1,1);
\path[fill=white] (14,32) rectangle ++(1,1);
\path[fill=white] (15,32) rectangle ++(1,1);
\path[fill=white] (16,32) rectangle ++(1,1);
\path[fill=white] (17,32) rectangle ++(1,1);
\path[fill=white] (18,32) rectangle ++(1,1);
\path[fill=white] (19,32) rectangle ++(1,1);
\path[fill=white] (20,32) rectangle ++(1,1);
\path[fill=white] (21,32) rectangle ++(1,1);
\path[fill=black] (22,32) rectangle ++(1,1);
\path[fill=white] (23,32) rectangle ++(1,1);
\path[fill=white] (24,32) rectangle ++(1,1);
\path[fill=white] (25,32) rectangle ++(1,1);
\path[fill=white] (26,32) rectangle ++(1,1);
\path[fill=white] (27,32) rectangle ++(1,1);
\path[fill=white] (28,32) rectangle ++(1,1);
\path[fill=white] (29,32) rectangle ++(1,1);
\path[fill=black] (30,32) rectangle ++(1,1);
\path[fill=white] (31,32) rectangle ++(1,1);
\path[fill=white] (32,32) rectangle ++(1,1);
\path[fill=white] (33,32) rectangle ++(1,1);
\path[fill=black] (34,32) rectangle ++(1,1);
\path[fill=white] (0,31) rectangle ++(1,1);
\path[fill=white] (1,31) rectangle ++(1,1);
\path[fill=white] (2,31) rectangle ++(1,1);
\path[fill=white] (3,31) rectangle ++(1,1);
\path[fill=white] (4,31) rectangle ++(1,1);
\path[fill=white] (5,31) rectangle ++(1,1);
\path[fill=black] (6,31) rectangle ++(1,1);
\path[fill=black] (7,31) rectangle ++(1,1);
\path[fill=white] (8,31) rectangle ++(1,1);
\path[fill=white] (9,31) rectangle ++(1,1);
\path[fill=white] (10,31) rectangle ++(1,1);
\path[fill=black] (11,31) rectangle ++(1,1);
\path[fill=white] (12,31) rectangle ++(1,1);
\path[fill=white] (13,31) rectangle ++(1,1);
\path[fill=white] (14,31) rectangle ++(1,1);
\path[fill=white] (15,31) rectangle ++(1,1);
\path[fill=black] (16,31) rectangle ++(1,1);
\path[fill=white] (17,31) rectangle ++(1,1);
\path[fill=white] (18,31) rectangle ++(1,1);
\path[fill=white] (19,31) rectangle ++(1,1);
\path[fill=white] (20,31) rectangle ++(1,1);
\path[fill=white] (21,31) rectangle ++(1,1);
\path[fill=white] (22,31) rectangle ++(1,1);
\path[fill=white] (23,31) rectangle ++(1,1);
\path[fill=white] (24,31) rectangle ++(1,1);
\path[fill=black] (25,31) rectangle ++(1,1);
\path[fill=white] (26,31) rectangle ++(1,1);
\path[fill=white] (27,31) rectangle ++(1,1);
\path[fill=white] (28,31) rectangle ++(1,1);
\path[fill=white] (29,31) rectangle ++(1,1);
\path[fill=white] (30,31) rectangle ++(1,1);
\path[fill=white] (31,31) rectangle ++(1,1);
\path[fill=white] (32,31) rectangle ++(1,1);
\path[fill=white] (33,31) rectangle ++(1,1);
\path[fill=white] (34,31) rectangle ++(1,1);
\path[fill=white] (0,30) rectangle ++(1,1);
\path[fill=white] (1,30) rectangle ++(1,1);
\path[fill=white] (2,30) rectangle ++(1,1);
\path[fill=white] (3,30) rectangle ++(1,1);
\path[fill=white] (4,30) rectangle ++(1,1);
\path[fill=white] (5,30) rectangle ++(1,1);
\path[fill=white] (6,30) rectangle ++(1,1);
\path[fill=black] (7,30) rectangle ++(1,1);
\path[fill=black] (8,30) rectangle ++(1,1);
\path[fill=white] (9,30) rectangle ++(1,1);
\path[fill=white] (10,30) rectangle ++(1,1);
\path[fill=white] (11,30) rectangle ++(1,1);
\path[fill=black] (12,30) rectangle ++(1,1);
\path[fill=white] (13,30) rectangle ++(1,1);
\path[fill=white] (14,30) rectangle ++(1,1);
\path[fill=white] (15,30) rectangle ++(1,1);
\path[fill=white] (16,30) rectangle ++(1,1);
\path[fill=black] (17,30) rectangle ++(1,1);
\path[fill=white] (18,30) rectangle ++(1,1);
\path[fill=white] (19,30) rectangle ++(1,1);
\path[fill=white] (20,30) rectangle ++(1,1);
\path[fill=white] (21,30) rectangle ++(1,1);
\path[fill=white] (22,30) rectangle ++(1,1);
\path[fill=white] (23,30) rectangle ++(1,1);
\path[fill=white] (24,30) rectangle ++(1,1);
\path[fill=white] (25,30) rectangle ++(1,1);
\path[fill=black] (26,30) rectangle ++(1,1);
\path[fill=white] (27,30) rectangle ++(1,1);
\path[fill=white] (28,30) rectangle ++(1,1);
\path[fill=white] (29,30) rectangle ++(1,1);
\path[fill=white] (30,30) rectangle ++(1,1);
\path[fill=white] (31,30) rectangle ++(1,1);
\path[fill=black] (32,30) rectangle ++(1,1);
\path[fill=white] (33,30) rectangle ++(1,1);
\path[fill=white] (34,30) rectangle ++(1,1);
\path[fill=white] (0,29) rectangle ++(1,1);
\path[fill=white] (1,29) rectangle ++(1,1);
\path[fill=white] (2,29) rectangle ++(1,1);
\path[fill=white] (3,29) rectangle ++(1,1);
\path[fill=white] (4,29) rectangle ++(1,1);
\path[fill=white] (5,29) rectangle ++(1,1);
\path[fill=white] (6,29) rectangle ++(1,1);
\path[fill=white] (7,29) rectangle ++(1,1);
\path[fill=black] (8,29) rectangle ++(1,1);
\path[fill=black] (9,29) rectangle ++(1,1);
\path[fill=white] (10,29) rectangle ++(1,1);
\path[fill=white] (11,29) rectangle ++(1,1);
\path[fill=white] (12,29) rectangle ++(1,1);
\path[fill=black] (13,29) rectangle ++(1,1);
\path[fill=white] (14,29) rectangle ++(1,1);
\path[fill=white] (15,29) rectangle ++(1,1);
\path[fill=white] (16,29) rectangle ++(1,1);
\path[fill=white] (17,29) rectangle ++(1,1);
\path[fill=black] (18,29) rectangle ++(1,1);
\path[fill=white] (19,29) rectangle ++(1,1);
\path[fill=white] (20,29) rectangle ++(1,1);
\path[fill=white] (21,29) rectangle ++(1,1);
\path[fill=white] (22,29) rectangle ++(1,1);
\path[fill=white] (23,29) rectangle ++(1,1);
\path[fill=white] (24,29) rectangle ++(1,1);
\path[fill=white] (25,29) rectangle ++(1,1);
\path[fill=white] (26,29) rectangle ++(1,1);
\path[fill=black] (27,29) rectangle ++(1,1);
\path[fill=white] (28,29) rectangle ++(1,1);
\path[fill=white] (29,29) rectangle ++(1,1);
\path[fill=white] (30,29) rectangle ++(1,1);
\path[fill=white] (31,29) rectangle ++(1,1);
\path[fill=white] (32,29) rectangle ++(1,1);
\path[fill=black] (33,29) rectangle ++(1,1);
\path[fill=white] (34,29) rectangle ++(1,1);
\path[fill=white] (0,28) rectangle ++(1,1);
\path[fill=white] (1,28) rectangle ++(1,1);
\path[fill=white] (2,28) rectangle ++(1,1);
\path[fill=white] (3,28) rectangle ++(1,1);
\path[fill=white] (4,28) rectangle ++(1,1);
\path[fill=white] (5,28) rectangle ++(1,1);
\path[fill=white] (6,28) rectangle ++(1,1);
\path[fill=white] (7,28) rectangle ++(1,1);
\path[fill=white] (8,28) rectangle ++(1,1);
\path[fill=white] (9,28) rectangle ++(1,1);
\path[fill=black] (10,28) rectangle ++(1,1);
\path[fill=black] (11,28) rectangle ++(1,1);
\path[fill=white] (12,28) rectangle ++(1,1);
\path[fill=white] (13,28) rectangle ++(1,1);
\path[fill=white] (14,28) rectangle ++(1,1);
\path[fill=black] (15,28) rectangle ++(1,1);
\path[fill=white] (16,28) rectangle ++(1,1);
\path[fill=white] (17,28) rectangle ++(1,1);
\path[fill=white] (18,28) rectangle ++(1,1);
\path[fill=white] (19,28) rectangle ++(1,1);
\path[fill=black] (20,28) rectangle ++(1,1);
\path[fill=white] (21,28) rectangle ++(1,1);
\path[fill=white] (22,28) rectangle ++(1,1);
\path[fill=white] (23,28) rectangle ++(1,1);
\path[fill=white] (24,28) rectangle ++(1,1);
\path[fill=white] (25,28) rectangle ++(1,1);
\path[fill=white] (26,28) rectangle ++(1,1);
\path[fill=white] (27,28) rectangle ++(1,1);
\path[fill=white] (28,28) rectangle ++(1,1);
\path[fill=white] (29,28) rectangle ++(1,1);
\path[fill=white] (30,28) rectangle ++(1,1);
\path[fill=white] (31,28) rectangle ++(1,1);
\path[fill=white] (32,28) rectangle ++(1,1);
\path[fill=white] (33,28) rectangle ++(1,1);
\path[fill=white] (34,28) rectangle ++(1,1);
\path[fill=white] (0,27) rectangle ++(1,1);
\path[fill=white] (1,27) rectangle ++(1,1);
\path[fill=white] (2,27) rectangle ++(1,1);
\path[fill=white] (3,27) rectangle ++(1,1);
\path[fill=white] (4,27) rectangle ++(1,1);
\path[fill=white] (5,27) rectangle ++(1,1);
\path[fill=white] (6,27) rectangle ++(1,1);
\path[fill=white] (7,27) rectangle ++(1,1);
\path[fill=white] (8,27) rectangle ++(1,1);
\path[fill=white] (9,27) rectangle ++(1,1);
\path[fill=white] (10,27) rectangle ++(1,1);
\path[fill=black] (11,27) rectangle ++(1,1);
\path[fill=black] (12,27) rectangle ++(1,1);
\path[fill=white] (13,27) rectangle ++(1,1);
\path[fill=white] (14,27) rectangle ++(1,1);
\path[fill=white] (15,27) rectangle ++(1,1);
\path[fill=black] (16,27) rectangle ++(1,1);
\path[fill=white] (17,27) rectangle ++(1,1);
\path[fill=white] (18,27) rectangle ++(1,1);
\path[fill=white] (19,27) rectangle ++(1,1);
\path[fill=white] (20,27) rectangle ++(1,1);
\path[fill=black] (21,27) rectangle ++(1,1);
\path[fill=white] (22,27) rectangle ++(1,1);
\path[fill=white] (23,27) rectangle ++(1,1);
\path[fill=white] (24,27) rectangle ++(1,1);
\path[fill=white] (25,27) rectangle ++(1,1);
\path[fill=white] (26,27) rectangle ++(1,1);
\path[fill=white] (27,27) rectangle ++(1,1);
\path[fill=white] (28,27) rectangle ++(1,1);
\path[fill=black] (29,27) rectangle ++(1,1);
\path[fill=white] (30,27) rectangle ++(1,1);
\path[fill=white] (31,27) rectangle ++(1,1);
\path[fill=white] (32,27) rectangle ++(1,1);
\path[fill=white] (33,27) rectangle ++(1,1);
\path[fill=white] (34,27) rectangle ++(1,1);
\path[fill=white] (0,26) rectangle ++(1,1);
\path[fill=white] (1,26) rectangle ++(1,1);
\path[fill=white] (2,26) rectangle ++(1,1);
\path[fill=white] (3,26) rectangle ++(1,1);
\path[fill=white] (4,26) rectangle ++(1,1);
\path[fill=white] (5,26) rectangle ++(1,1);
\path[fill=white] (6,26) rectangle ++(1,1);
\path[fill=white] (7,26) rectangle ++(1,1);
\path[fill=white] (8,26) rectangle ++(1,1);
\path[fill=white] (9,26) rectangle ++(1,1);
\path[fill=white] (10,26) rectangle ++(1,1);
\path[fill=white] (11,26) rectangle ++(1,1);
\path[fill=black] (12,26) rectangle ++(1,1);
\path[fill=black] (13,26) rectangle ++(1,1);
\path[fill=white] (14,26) rectangle ++(1,1);
\path[fill=white] (15,26) rectangle ++(1,1);
\path[fill=white] (16,26) rectangle ++(1,1);
\path[fill=black] (17,26) rectangle ++(1,1);
\path[fill=white] (18,26) rectangle ++(1,1);
\path[fill=white] (19,26) rectangle ++(1,1);
\path[fill=white] (20,26) rectangle ++(1,1);
\path[fill=white] (21,26) rectangle ++(1,1);
\path[fill=black] (22,26) rectangle ++(1,1);
\path[fill=white] (23,26) rectangle ++(1,1);
\path[fill=white] (24,26) rectangle ++(1,1);
\path[fill=white] (25,26) rectangle ++(1,1);
\path[fill=white] (26,26) rectangle ++(1,1);
\path[fill=white] (27,26) rectangle ++(1,1);
\path[fill=white] (28,26) rectangle ++(1,1);
\path[fill=white] (29,26) rectangle ++(1,1);
\path[fill=black] (30,26) rectangle ++(1,1);
\path[fill=white] (31,26) rectangle ++(1,1);
\path[fill=white] (32,26) rectangle ++(1,1);
\path[fill=white] (33,26) rectangle ++(1,1);
\path[fill=black] (34,26) rectangle ++(1,1);
\path[fill=white] (0,25) rectangle ++(1,1);
\path[fill=white] (1,25) rectangle ++(1,1);
\path[fill=white] (2,25) rectangle ++(1,1);
\path[fill=white] (3,25) rectangle ++(1,1);
\path[fill=white] (4,25) rectangle ++(1,1);
\path[fill=white] (5,25) rectangle ++(1,1);
\path[fill=white] (6,25) rectangle ++(1,1);
\path[fill=white] (7,25) rectangle ++(1,1);
\path[fill=white] (8,25) rectangle ++(1,1);
\path[fill=white] (9,25) rectangle ++(1,1);
\path[fill=white] (10,25) rectangle ++(1,1);
\path[fill=white] (11,25) rectangle ++(1,1);
\path[fill=white] (12,25) rectangle ++(1,1);
\path[fill=black] (13,25) rectangle ++(1,1);
\path[fill=black] (14,25) rectangle ++(1,1);
\path[fill=white] (15,25) rectangle ++(1,1);
\path[fill=white] (16,25) rectangle ++(1,1);
\path[fill=white] (17,25) rectangle ++(1,1);
\path[fill=black] (18,25) rectangle ++(1,1);
\path[fill=white] (19,25) rectangle ++(1,1);
\path[fill=white] (20,25) rectangle ++(1,1);
\path[fill=white] (21,25) rectangle ++(1,1);
\path[fill=white] (22,25) rectangle ++(1,1);
\path[fill=black] (23,25) rectangle ++(1,1);
\path[fill=white] (24,25) rectangle ++(1,1);
\path[fill=white] (25,25) rectangle ++(1,1);
\path[fill=white] (26,25) rectangle ++(1,1);
\path[fill=white] (27,25) rectangle ++(1,1);
\path[fill=white] (28,25) rectangle ++(1,1);
\path[fill=white] (29,25) rectangle ++(1,1);
\path[fill=white] (30,25) rectangle ++(1,1);
\path[fill=black] (31,25) rectangle ++(1,1);
\path[fill=white] (32,25) rectangle ++(1,1);
\path[fill=white] (33,25) rectangle ++(1,1);
\path[fill=white] (34,25) rectangle ++(1,1);
\path[fill=white] (0,24) rectangle ++(1,1);
\path[fill=white] (1,24) rectangle ++(1,1);
\path[fill=white] (2,24) rectangle ++(1,1);
\path[fill=white] (3,24) rectangle ++(1,1);
\path[fill=white] (4,24) rectangle ++(1,1);
\path[fill=white] (5,24) rectangle ++(1,1);
\path[fill=white] (6,24) rectangle ++(1,1);
\path[fill=white] (7,24) rectangle ++(1,1);
\path[fill=white] (8,24) rectangle ++(1,1);
\path[fill=white] (9,24) rectangle ++(1,1);
\path[fill=white] (10,24) rectangle ++(1,1);
\path[fill=white] (11,24) rectangle ++(1,1);
\path[fill=white] (12,24) rectangle ++(1,1);
\path[fill=white] (13,24) rectangle ++(1,1);
\path[fill=white] (14,24) rectangle ++(1,1);
\path[fill=black] (15,24) rectangle ++(1,1);
\path[fill=white] (16,24) rectangle ++(1,1);
\path[fill=white] (17,24) rectangle ++(1,1);
\path[fill=white] (18,24) rectangle ++(1,1);
\path[fill=white] (19,24) rectangle ++(1,1);
\path[fill=white] (20,24) rectangle ++(1,1);
\path[fill=white] (21,24) rectangle ++(1,1);
\path[fill=white] (22,24) rectangle ++(1,1);
\path[fill=white] (23,24) rectangle ++(1,1);
\path[fill=white] (24,24) rectangle ++(1,1);
\path[fill=white] (25,24) rectangle ++(1,1);
\path[fill=white] (26,24) rectangle ++(1,1);
\path[fill=white] (27,24) rectangle ++(1,1);
\path[fill=white] (28,24) rectangle ++(1,1);
\path[fill=white] (29,24) rectangle ++(1,1);
\path[fill=white] (30,24) rectangle ++(1,1);
\path[fill=white] (31,24) rectangle ++(1,1);
\path[fill=white] (32,24) rectangle ++(1,1);
\path[fill=white] (33,24) rectangle ++(1,1);
\path[fill=white] (34,24) rectangle ++(1,1);
\path[fill=white] (0,23) rectangle ++(1,1);
\path[fill=white] (1,23) rectangle ++(1,1);
\path[fill=white] (2,23) rectangle ++(1,1);
\path[fill=white] (3,23) rectangle ++(1,1);
\path[fill=white] (4,23) rectangle ++(1,1);
\path[fill=white] (5,23) rectangle ++(1,1);
\path[fill=white] (6,23) rectangle ++(1,1);
\path[fill=white] (7,23) rectangle ++(1,1);
\path[fill=white] (8,23) rectangle ++(1,1);
\path[fill=white] (9,23) rectangle ++(1,1);
\path[fill=white] (10,23) rectangle ++(1,1);
\path[fill=white] (11,23) rectangle ++(1,1);
\path[fill=white] (12,23) rectangle ++(1,1);
\path[fill=white] (13,23) rectangle ++(1,1);
\path[fill=white] (14,23) rectangle ++(1,1);
\path[fill=black] (15,23) rectangle ++(1,1);
\path[fill=black] (16,23) rectangle ++(1,1);
\path[fill=white] (17,23) rectangle ++(1,1);
\path[fill=white] (18,23) rectangle ++(1,1);
\path[fill=white] (19,23) rectangle ++(1,1);
\path[fill=black] (20,23) rectangle ++(1,1);
\path[fill=white] (21,23) rectangle ++(1,1);
\path[fill=white] (22,23) rectangle ++(1,1);
\path[fill=white] (23,23) rectangle ++(1,1);
\path[fill=white] (24,23) rectangle ++(1,1);
\path[fill=black] (25,23) rectangle ++(1,1);
\path[fill=white] (26,23) rectangle ++(1,1);
\path[fill=white] (27,23) rectangle ++(1,1);
\path[fill=white] (28,23) rectangle ++(1,1);
\path[fill=white] (29,23) rectangle ++(1,1);
\path[fill=white] (30,23) rectangle ++(1,1);
\path[fill=white] (31,23) rectangle ++(1,1);
\path[fill=white] (32,23) rectangle ++(1,1);
\path[fill=white] (33,23) rectangle ++(1,1);
\path[fill=white] (34,23) rectangle ++(1,1);
\path[fill=white] (0,22) rectangle ++(1,1);
\path[fill=white] (1,22) rectangle ++(1,1);
\path[fill=white] (2,22) rectangle ++(1,1);
\path[fill=white] (3,22) rectangle ++(1,1);
\path[fill=white] (4,22) rectangle ++(1,1);
\path[fill=white] (5,22) rectangle ++(1,1);
\path[fill=white] (6,22) rectangle ++(1,1);
\path[fill=white] (7,22) rectangle ++(1,1);
\path[fill=white] (8,22) rectangle ++(1,1);
\path[fill=white] (9,22) rectangle ++(1,1);
\path[fill=white] (10,22) rectangle ++(1,1);
\path[fill=white] (11,22) rectangle ++(1,1);
\path[fill=white] (12,22) rectangle ++(1,1);
\path[fill=white] (13,22) rectangle ++(1,1);
\path[fill=white] (14,22) rectangle ++(1,1);
\path[fill=white] (15,22) rectangle ++(1,1);
\path[fill=black] (16,22) rectangle ++(1,1);
\path[fill=black] (17,22) rectangle ++(1,1);
\path[fill=white] (18,22) rectangle ++(1,1);
\path[fill=white] (19,22) rectangle ++(1,1);
\path[fill=white] (20,22) rectangle ++(1,1);
\path[fill=black] (21,22) rectangle ++(1,1);
\path[fill=white] (22,22) rectangle ++(1,1);
\path[fill=white] (23,22) rectangle ++(1,1);
\path[fill=white] (24,22) rectangle ++(1,1);
\path[fill=white] (25,22) rectangle ++(1,1);
\path[fill=black] (26,22) rectangle ++(1,1);
\path[fill=white] (27,22) rectangle ++(1,1);
\path[fill=white] (28,22) rectangle ++(1,1);
\path[fill=white] (29,22) rectangle ++(1,1);
\path[fill=white] (30,22) rectangle ++(1,1);
\path[fill=white] (31,22) rectangle ++(1,1);
\path[fill=black] (32,22) rectangle ++(1,1);
\path[fill=white] (33,22) rectangle ++(1,1);
\path[fill=white] (34,22) rectangle ++(1,1);
\path[fill=white] (0,21) rectangle ++(1,1);
\path[fill=white] (1,21) rectangle ++(1,1);
\path[fill=white] (2,21) rectangle ++(1,1);
\path[fill=white] (3,21) rectangle ++(1,1);
\path[fill=white] (4,21) rectangle ++(1,1);
\path[fill=white] (5,21) rectangle ++(1,1);
\path[fill=white] (6,21) rectangle ++(1,1);
\path[fill=white] (7,21) rectangle ++(1,1);
\path[fill=white] (8,21) rectangle ++(1,1);
\path[fill=white] (9,21) rectangle ++(1,1);
\path[fill=white] (10,21) rectangle ++(1,1);
\path[fill=white] (11,21) rectangle ++(1,1);
\path[fill=white] (12,21) rectangle ++(1,1);
\path[fill=white] (13,21) rectangle ++(1,1);
\path[fill=white] (14,21) rectangle ++(1,1);
\path[fill=white] (15,21) rectangle ++(1,1);
\path[fill=white] (16,21) rectangle ++(1,1);
\path[fill=black] (17,21) rectangle ++(1,1);
\path[fill=black] (18,21) rectangle ++(1,1);
\path[fill=white] (19,21) rectangle ++(1,1);
\path[fill=white] (20,21) rectangle ++(1,1);
\path[fill=white] (21,21) rectangle ++(1,1);
\path[fill=black] (22,21) rectangle ++(1,1);
\path[fill=white] (23,21) rectangle ++(1,1);
\path[fill=white] (24,21) rectangle ++(1,1);
\path[fill=white] (25,21) rectangle ++(1,1);
\path[fill=white] (26,21) rectangle ++(1,1);
\path[fill=black] (27,21) rectangle ++(1,1);
\path[fill=white] (28,21) rectangle ++(1,1);
\path[fill=white] (29,21) rectangle ++(1,1);
\path[fill=white] (30,21) rectangle ++(1,1);
\path[fill=white] (31,21) rectangle ++(1,1);
\path[fill=white] (32,21) rectangle ++(1,1);
\path[fill=black] (33,21) rectangle ++(1,1);
\path[fill=white] (34,21) rectangle ++(1,1);
\path[fill=white] (0,20) rectangle ++(1,1);
\path[fill=white] (1,20) rectangle ++(1,1);
\path[fill=white] (2,20) rectangle ++(1,1);
\path[fill=white] (3,20) rectangle ++(1,1);
\path[fill=white] (4,20) rectangle ++(1,1);
\path[fill=white] (5,20) rectangle ++(1,1);
\path[fill=white] (6,20) rectangle ++(1,1);
\path[fill=white] (7,20) rectangle ++(1,1);
\path[fill=white] (8,20) rectangle ++(1,1);
\path[fill=white] (9,20) rectangle ++(1,1);
\path[fill=white] (10,20) rectangle ++(1,1);
\path[fill=white] (11,20) rectangle ++(1,1);
\path[fill=white] (12,20) rectangle ++(1,1);
\path[fill=white] (13,20) rectangle ++(1,1);
\path[fill=white] (14,20) rectangle ++(1,1);
\path[fill=white] (15,20) rectangle ++(1,1);
\path[fill=white] (16,20) rectangle ++(1,1);
\path[fill=white] (17,20) rectangle ++(1,1);
\path[fill=black] (18,20) rectangle ++(1,1);
\path[fill=black] (19,20) rectangle ++(1,1);
\path[fill=white] (20,20) rectangle ++(1,1);
\path[fill=white] (21,20) rectangle ++(1,1);
\path[fill=white] (22,20) rectangle ++(1,1);
\path[fill=black] (23,20) rectangle ++(1,1);
\path[fill=white] (24,20) rectangle ++(1,1);
\path[fill=white] (25,20) rectangle ++(1,1);
\path[fill=white] (26,20) rectangle ++(1,1);
\path[fill=white] (27,20) rectangle ++(1,1);
\path[fill=black] (28,20) rectangle ++(1,1);
\path[fill=white] (29,20) rectangle ++(1,1);
\path[fill=white] (30,20) rectangle ++(1,1);
\path[fill=white] (31,20) rectangle ++(1,1);
\path[fill=white] (32,20) rectangle ++(1,1);
\path[fill=white] (33,20) rectangle ++(1,1);
\path[fill=white] (34,20) rectangle ++(1,1);
\path[fill=white] (0,19) rectangle ++(1,1);
\path[fill=white] (1,19) rectangle ++(1,1);
\path[fill=white] (2,19) rectangle ++(1,1);
\path[fill=white] (3,19) rectangle ++(1,1);
\path[fill=white] (4,19) rectangle ++(1,1);
\path[fill=white] (5,19) rectangle ++(1,1);
\path[fill=white] (6,19) rectangle ++(1,1);
\path[fill=white] (7,19) rectangle ++(1,1);
\path[fill=white] (8,19) rectangle ++(1,1);
\path[fill=white] (9,19) rectangle ++(1,1);
\path[fill=white] (10,19) rectangle ++(1,1);
\path[fill=white] (11,19) rectangle ++(1,1);
\path[fill=white] (12,19) rectangle ++(1,1);
\path[fill=white] (13,19) rectangle ++(1,1);
\path[fill=white] (14,19) rectangle ++(1,1);
\path[fill=white] (15,19) rectangle ++(1,1);
\path[fill=white] (16,19) rectangle ++(1,1);
\path[fill=white] (17,19) rectangle ++(1,1);
\path[fill=white] (18,19) rectangle ++(1,1);
\path[fill=white] (19,19) rectangle ++(1,1);
\path[fill=black] (20,19) rectangle ++(1,1);
\path[fill=white] (21,19) rectangle ++(1,1);
\path[fill=white] (22,19) rectangle ++(1,1);
\path[fill=white] (23,19) rectangle ++(1,1);
\path[fill=white] (24,19) rectangle ++(1,1);
\path[fill=white] (25,19) rectangle ++(1,1);
\path[fill=white] (26,19) rectangle ++(1,1);
\path[fill=white] (27,19) rectangle ++(1,1);
\path[fill=white] (28,19) rectangle ++(1,1);
\path[fill=white] (29,19) rectangle ++(1,1);
\path[fill=white] (30,19) rectangle ++(1,1);
\path[fill=white] (31,19) rectangle ++(1,1);
\path[fill=white] (32,19) rectangle ++(1,1);
\path[fill=white] (33,19) rectangle ++(1,1);
\path[fill=white] (34,19) rectangle ++(1,1);
\path[fill=white] (0,18) rectangle ++(1,1);
\path[fill=white] (1,18) rectangle ++(1,1);
\path[fill=white] (2,18) rectangle ++(1,1);
\path[fill=white] (3,18) rectangle ++(1,1);
\path[fill=white] (4,18) rectangle ++(1,1);
\path[fill=white] (5,18) rectangle ++(1,1);
\path[fill=white] (6,18) rectangle ++(1,1);
\path[fill=white] (7,18) rectangle ++(1,1);
\path[fill=white] (8,18) rectangle ++(1,1);
\path[fill=white] (9,18) rectangle ++(1,1);
\path[fill=white] (10,18) rectangle ++(1,1);
\path[fill=white] (11,18) rectangle ++(1,1);
\path[fill=white] (12,18) rectangle ++(1,1);
\path[fill=white] (13,18) rectangle ++(1,1);
\path[fill=white] (14,18) rectangle ++(1,1);
\path[fill=white] (15,18) rectangle ++(1,1);
\path[fill=white] (16,18) rectangle ++(1,1);
\path[fill=white] (17,18) rectangle ++(1,1);
\path[fill=white] (18,18) rectangle ++(1,1);
\path[fill=white] (19,18) rectangle ++(1,1);
\path[fill=black] (20,18) rectangle ++(1,1);
\path[fill=black] (21,18) rectangle ++(1,1);
\path[fill=white] (22,18) rectangle ++(1,1);
\path[fill=white] (23,18) rectangle ++(1,1);
\path[fill=white] (24,18) rectangle ++(1,1);
\path[fill=black] (25,18) rectangle ++(1,1);
\path[fill=white] (26,18) rectangle ++(1,1);
\path[fill=white] (27,18) rectangle ++(1,1);
\path[fill=white] (28,18) rectangle ++(1,1);
\path[fill=black] (29,18) rectangle ++(1,1);
\path[fill=white] (30,18) rectangle ++(1,1);
\path[fill=white] (31,18) rectangle ++(1,1);
\path[fill=white] (32,18) rectangle ++(1,1);
\path[fill=white] (33,18) rectangle ++(1,1);
\path[fill=white] (34,18) rectangle ++(1,1);
\path[fill=white] (0,17) rectangle ++(1,1);
\path[fill=white] (1,17) rectangle ++(1,1);
\path[fill=white] (2,17) rectangle ++(1,1);
\path[fill=white] (3,17) rectangle ++(1,1);
\path[fill=white] (4,17) rectangle ++(1,1);
\path[fill=white] (5,17) rectangle ++(1,1);
\path[fill=white] (6,17) rectangle ++(1,1);
\path[fill=white] (7,17) rectangle ++(1,1);
\path[fill=white] (8,17) rectangle ++(1,1);
\path[fill=white] (9,17) rectangle ++(1,1);
\path[fill=white] (10,17) rectangle ++(1,1);
\path[fill=white] (11,17) rectangle ++(1,1);
\path[fill=white] (12,17) rectangle ++(1,1);
\path[fill=white] (13,17) rectangle ++(1,1);
\path[fill=white] (14,17) rectangle ++(1,1);
\path[fill=white] (15,17) rectangle ++(1,1);
\path[fill=white] (16,17) rectangle ++(1,1);
\path[fill=white] (17,17) rectangle ++(1,1);
\path[fill=white] (18,17) rectangle ++(1,1);
\path[fill=white] (19,17) rectangle ++(1,1);
\path[fill=white] (20,17) rectangle ++(1,1);
\path[fill=black] (21,17) rectangle ++(1,1);
\path[fill=black] (22,17) rectangle ++(1,1);
\path[fill=white] (23,17) rectangle ++(1,1);
\path[fill=white] (24,17) rectangle ++(1,1);
\path[fill=white] (25,17) rectangle ++(1,1);
\path[fill=black] (26,17) rectangle ++(1,1);
\path[fill=white] (27,17) rectangle ++(1,1);
\path[fill=white] (28,17) rectangle ++(1,1);
\path[fill=white] (29,17) rectangle ++(1,1);
\path[fill=black] (30,17) rectangle ++(1,1);
\path[fill=white] (31,17) rectangle ++(1,1);
\path[fill=white] (32,17) rectangle ++(1,1);
\path[fill=white] (33,17) rectangle ++(1,1);
\path[fill=black] (34,17) rectangle ++(1,1);
\path[fill=white] (0,16) rectangle ++(1,1);
\path[fill=white] (1,16) rectangle ++(1,1);
\path[fill=white] (2,16) rectangle ++(1,1);
\path[fill=white] (3,16) rectangle ++(1,1);
\path[fill=white] (4,16) rectangle ++(1,1);
\path[fill=white] (5,16) rectangle ++(1,1);
\path[fill=white] (6,16) rectangle ++(1,1);
\path[fill=white] (7,16) rectangle ++(1,1);
\path[fill=white] (8,16) rectangle ++(1,1);
\path[fill=white] (9,16) rectangle ++(1,1);
\path[fill=white] (10,16) rectangle ++(1,1);
\path[fill=white] (11,16) rectangle ++(1,1);
\path[fill=white] (12,16) rectangle ++(1,1);
\path[fill=white] (13,16) rectangle ++(1,1);
\path[fill=white] (14,16) rectangle ++(1,1);
\path[fill=white] (15,16) rectangle ++(1,1);
\path[fill=white] (16,16) rectangle ++(1,1);
\path[fill=white] (17,16) rectangle ++(1,1);
\path[fill=white] (18,16) rectangle ++(1,1);
\path[fill=white] (19,16) rectangle ++(1,1);
\path[fill=white] (20,16) rectangle ++(1,1);
\path[fill=white] (21,16) rectangle ++(1,1);
\path[fill=black] (22,16) rectangle ++(1,1);
\path[fill=black] (23,16) rectangle ++(1,1);
\path[fill=white] (24,16) rectangle ++(1,1);
\path[fill=white] (25,16) rectangle ++(1,1);
\path[fill=white] (26,16) rectangle ++(1,1);
\path[fill=black] (27,16) rectangle ++(1,1);
\path[fill=white] (28,16) rectangle ++(1,1);
\path[fill=white] (29,16) rectangle ++(1,1);
\path[fill=white] (30,16) rectangle ++(1,1);
\path[fill=black] (31,16) rectangle ++(1,1);
\path[fill=white] (32,16) rectangle ++(1,1);
\path[fill=white] (33,16) rectangle ++(1,1);
\path[fill=white] (34,16) rectangle ++(1,1);
\path[fill=white] (0,15) rectangle ++(1,1);
\path[fill=white] (1,15) rectangle ++(1,1);
\path[fill=white] (2,15) rectangle ++(1,1);
\path[fill=white] (3,15) rectangle ++(1,1);
\path[fill=white] (4,15) rectangle ++(1,1);
\path[fill=white] (5,15) rectangle ++(1,1);
\path[fill=white] (6,15) rectangle ++(1,1);
\path[fill=white] (7,15) rectangle ++(1,1);
\path[fill=white] (8,15) rectangle ++(1,1);
\path[fill=white] (9,15) rectangle ++(1,1);
\path[fill=white] (10,15) rectangle ++(1,1);
\path[fill=white] (11,15) rectangle ++(1,1);
\path[fill=white] (12,15) rectangle ++(1,1);
\path[fill=white] (13,15) rectangle ++(1,1);
\path[fill=white] (14,15) rectangle ++(1,1);
\path[fill=white] (15,15) rectangle ++(1,1);
\path[fill=white] (16,15) rectangle ++(1,1);
\path[fill=white] (17,15) rectangle ++(1,1);
\path[fill=white] (18,15) rectangle ++(1,1);
\path[fill=white] (19,15) rectangle ++(1,1);
\path[fill=white] (20,15) rectangle ++(1,1);
\path[fill=white] (21,15) rectangle ++(1,1);
\path[fill=white] (22,15) rectangle ++(1,1);
\path[fill=black] (23,15) rectangle ++(1,1);
\path[fill=black] (24,15) rectangle ++(1,1);
\path[fill=white] (25,15) rectangle ++(1,1);
\path[fill=white] (26,15) rectangle ++(1,1);
\path[fill=white] (27,15) rectangle ++(1,1);
\path[fill=black] (28,15) rectangle ++(1,1);
\path[fill=white] (29,15) rectangle ++(1,1);
\path[fill=white] (30,15) rectangle ++(1,1);
\path[fill=white] (31,15) rectangle ++(1,1);
\path[fill=white] (32,15) rectangle ++(1,1);
\path[fill=white] (33,15) rectangle ++(1,1);
\path[fill=white] (34,15) rectangle ++(1,1);
\path[fill=white] (0,14) rectangle ++(1,1);
\path[fill=white] (1,14) rectangle ++(1,1);
\path[fill=white] (2,14) rectangle ++(1,1);
\path[fill=white] (3,14) rectangle ++(1,1);
\path[fill=white] (4,14) rectangle ++(1,1);
\path[fill=white] (5,14) rectangle ++(1,1);
\path[fill=white] (6,14) rectangle ++(1,1);
\path[fill=white] (7,14) rectangle ++(1,1);
\path[fill=white] (8,14) rectangle ++(1,1);
\path[fill=white] (9,14) rectangle ++(1,1);
\path[fill=white] (10,14) rectangle ++(1,1);
\path[fill=white] (11,14) rectangle ++(1,1);
\path[fill=white] (12,14) rectangle ++(1,1);
\path[fill=white] (13,14) rectangle ++(1,1);
\path[fill=white] (14,14) rectangle ++(1,1);
\path[fill=white] (15,14) rectangle ++(1,1);
\path[fill=white] (16,14) rectangle ++(1,1);
\path[fill=white] (17,14) rectangle ++(1,1);
\path[fill=white] (18,14) rectangle ++(1,1);
\path[fill=white] (19,14) rectangle ++(1,1);
\path[fill=white] (20,14) rectangle ++(1,1);
\path[fill=white] (21,14) rectangle ++(1,1);
\path[fill=white] (22,14) rectangle ++(1,1);
\path[fill=white] (23,14) rectangle ++(1,1);
\path[fill=white] (24,14) rectangle ++(1,1);
\path[fill=black] (25,14) rectangle ++(1,1);
\path[fill=white] (26,14) rectangle ++(1,1);
\path[fill=white] (27,14) rectangle ++(1,1);
\path[fill=white] (28,14) rectangle ++(1,1);
\path[fill=white] (29,14) rectangle ++(1,1);
\path[fill=white] (30,14) rectangle ++(1,1);
\path[fill=white] (31,14) rectangle ++(1,1);
\path[fill=white] (32,14) rectangle ++(1,1);
\path[fill=white] (33,14) rectangle ++(1,1);
\path[fill=white] (34,14) rectangle ++(1,1);
\path[fill=white] (0,13) rectangle ++(1,1);
\path[fill=white] (1,13) rectangle ++(1,1);
\path[fill=white] (2,13) rectangle ++(1,1);
\path[fill=white] (3,13) rectangle ++(1,1);
\path[fill=white] (4,13) rectangle ++(1,1);
\path[fill=white] (5,13) rectangle ++(1,1);
\path[fill=white] (6,13) rectangle ++(1,1);
\path[fill=white] (7,13) rectangle ++(1,1);
\path[fill=white] (8,13) rectangle ++(1,1);
\path[fill=white] (9,13) rectangle ++(1,1);
\path[fill=white] (10,13) rectangle ++(1,1);
\path[fill=white] (11,13) rectangle ++(1,1);
\path[fill=white] (12,13) rectangle ++(1,1);
\path[fill=white] (13,13) rectangle ++(1,1);
\path[fill=white] (14,13) rectangle ++(1,1);
\path[fill=white] (15,13) rectangle ++(1,1);
\path[fill=white] (16,13) rectangle ++(1,1);
\path[fill=white] (17,13) rectangle ++(1,1);
\path[fill=white] (18,13) rectangle ++(1,1);
\path[fill=white] (19,13) rectangle ++(1,1);
\path[fill=white] (20,13) rectangle ++(1,1);
\path[fill=white] (21,13) rectangle ++(1,1);
\path[fill=white] (22,13) rectangle ++(1,1);
\path[fill=white] (23,13) rectangle ++(1,1);
\path[fill=white] (24,13) rectangle ++(1,1);
\path[fill=black] (25,13) rectangle ++(1,1);
\path[fill=black] (26,13) rectangle ++(1,1);
\path[fill=white] (27,13) rectangle ++(1,1);
\path[fill=white] (28,13) rectangle ++(1,1);
\path[fill=black] (29,13) rectangle ++(1,1);
\path[fill=white] (30,13) rectangle ++(1,1);
\path[fill=white] (31,13) rectangle ++(1,1);
\path[fill=black] (32,13) rectangle ++(1,1);
\path[fill=white] (33,13) rectangle ++(1,1);
\path[fill=white] (34,13) rectangle ++(1,1);
\path[fill=white] (0,12) rectangle ++(1,1);
\path[fill=white] (1,12) rectangle ++(1,1);
\path[fill=white] (2,12) rectangle ++(1,1);
\path[fill=white] (3,12) rectangle ++(1,1);
\path[fill=white] (4,12) rectangle ++(1,1);
\path[fill=white] (5,12) rectangle ++(1,1);
\path[fill=white] (6,12) rectangle ++(1,1);
\path[fill=white] (7,12) rectangle ++(1,1);
\path[fill=white] (8,12) rectangle ++(1,1);
\path[fill=white] (9,12) rectangle ++(1,1);
\path[fill=white] (10,12) rectangle ++(1,1);
\path[fill=white] (11,12) rectangle ++(1,1);
\path[fill=white] (12,12) rectangle ++(1,1);
\path[fill=white] (13,12) rectangle ++(1,1);
\path[fill=white] (14,12) rectangle ++(1,1);
\path[fill=white] (15,12) rectangle ++(1,1);
\path[fill=white] (16,12) rectangle ++(1,1);
\path[fill=white] (17,12) rectangle ++(1,1);
\path[fill=white] (18,12) rectangle ++(1,1);
\path[fill=white] (19,12) rectangle ++(1,1);
\path[fill=white] (20,12) rectangle ++(1,1);
\path[fill=white] (21,12) rectangle ++(1,1);
\path[fill=white] (22,12) rectangle ++(1,1);
\path[fill=white] (23,12) rectangle ++(1,1);
\path[fill=white] (24,12) rectangle ++(1,1);
\path[fill=white] (25,12) rectangle ++(1,1);
\path[fill=black] (26,12) rectangle ++(1,1);
\path[fill=black] (27,12) rectangle ++(1,1);
\path[fill=white] (28,12) rectangle ++(1,1);
\path[fill=white] (29,12) rectangle ++(1,1);
\path[fill=black] (30,12) rectangle ++(1,1);
\path[fill=white] (31,12) rectangle ++(1,1);
\path[fill=white] (32,12) rectangle ++(1,1);
\path[fill=black] (33,12) rectangle ++(1,1);
\path[fill=white] (34,12) rectangle ++(1,1);
\path[fill=white] (0,11) rectangle ++(1,1);
\path[fill=white] (1,11) rectangle ++(1,1);
\path[fill=white] (2,11) rectangle ++(1,1);
\path[fill=white] (3,11) rectangle ++(1,1);
\path[fill=white] (4,11) rectangle ++(1,1);
\path[fill=white] (5,11) rectangle ++(1,1);
\path[fill=white] (6,11) rectangle ++(1,1);
\path[fill=white] (7,11) rectangle ++(1,1);
\path[fill=white] (8,11) rectangle ++(1,1);
\path[fill=white] (9,11) rectangle ++(1,1);
\path[fill=white] (10,11) rectangle ++(1,1);
\path[fill=white] (11,11) rectangle ++(1,1);
\path[fill=white] (12,11) rectangle ++(1,1);
\path[fill=white] (13,11) rectangle ++(1,1);
\path[fill=white] (14,11) rectangle ++(1,1);
\path[fill=white] (15,11) rectangle ++(1,1);
\path[fill=white] (16,11) rectangle ++(1,1);
\path[fill=white] (17,11) rectangle ++(1,1);
\path[fill=white] (18,11) rectangle ++(1,1);
\path[fill=white] (19,11) rectangle ++(1,1);
\path[fill=white] (20,11) rectangle ++(1,1);
\path[fill=white] (21,11) rectangle ++(1,1);
\path[fill=white] (22,11) rectangle ++(1,1);
\path[fill=white] (23,11) rectangle ++(1,1);
\path[fill=white] (24,11) rectangle ++(1,1);
\path[fill=white] (25,11) rectangle ++(1,1);
\path[fill=white] (26,11) rectangle ++(1,1);
\path[fill=black] (27,11) rectangle ++(1,1);
\path[fill=black] (28,11) rectangle ++(1,1);
\path[fill=white] (29,11) rectangle ++(1,1);
\path[fill=white] (30,11) rectangle ++(1,1);
\path[fill=black] (31,11) rectangle ++(1,1);
\path[fill=white] (32,11) rectangle ++(1,1);
\path[fill=white] (33,11) rectangle ++(1,1);
\path[fill=white] (34,11) rectangle ++(1,1);
\path[fill=white] (0,10) rectangle ++(1,1);
\path[fill=white] (1,10) rectangle ++(1,1);
\path[fill=white] (2,10) rectangle ++(1,1);
\path[fill=white] (3,10) rectangle ++(1,1);
\path[fill=white] (4,10) rectangle ++(1,1);
\path[fill=white] (5,10) rectangle ++(1,1);
\path[fill=white] (6,10) rectangle ++(1,1);
\path[fill=white] (7,10) rectangle ++(1,1);
\path[fill=white] (8,10) rectangle ++(1,1);
\path[fill=white] (9,10) rectangle ++(1,1);
\path[fill=white] (10,10) rectangle ++(1,1);
\path[fill=white] (11,10) rectangle ++(1,1);
\path[fill=white] (12,10) rectangle ++(1,1);
\path[fill=white] (13,10) rectangle ++(1,1);
\path[fill=white] (14,10) rectangle ++(1,1);
\path[fill=white] (15,10) rectangle ++(1,1);
\path[fill=white] (16,10) rectangle ++(1,1);
\path[fill=white] (17,10) rectangle ++(1,1);
\path[fill=white] (18,10) rectangle ++(1,1);
\path[fill=white] (19,10) rectangle ++(1,1);
\path[fill=white] (20,10) rectangle ++(1,1);
\path[fill=white] (21,10) rectangle ++(1,1);
\path[fill=white] (22,10) rectangle ++(1,1);
\path[fill=white] (23,10) rectangle ++(1,1);
\path[fill=white] (24,10) rectangle ++(1,1);
\path[fill=white] (25,10) rectangle ++(1,1);
\path[fill=white] (26,10) rectangle ++(1,1);
\path[fill=white] (27,10) rectangle ++(1,1);
\path[fill=black] (28,10) rectangle ++(1,1);
\path[fill=white] (29,10) rectangle ++(1,1);
\path[fill=white] (30,10) rectangle ++(1,1);
\path[fill=white] (31,10) rectangle ++(1,1);
\path[fill=white] (32,10) rectangle ++(1,1);
\path[fill=white] (33,10) rectangle ++(1,1);
\path[fill=white] (34,10) rectangle ++(1,1);
\path[fill=white] (0,9) rectangle ++(1,1);
\path[fill=white] (1,9) rectangle ++(1,1);
\path[fill=white] (2,9) rectangle ++(1,1);
\path[fill=white] (3,9) rectangle ++(1,1);
\path[fill=white] (4,9) rectangle ++(1,1);
\path[fill=white] (5,9) rectangle ++(1,1);
\path[fill=white] (6,9) rectangle ++(1,1);
\path[fill=white] (7,9) rectangle ++(1,1);
\path[fill=white] (8,9) rectangle ++(1,1);
\path[fill=white] (9,9) rectangle ++(1,1);
\path[fill=white] (10,9) rectangle ++(1,1);
\path[fill=white] (11,9) rectangle ++(1,1);
\path[fill=white] (12,9) rectangle ++(1,1);
\path[fill=white] (13,9) rectangle ++(1,1);
\path[fill=white] (14,9) rectangle ++(1,1);
\path[fill=white] (15,9) rectangle ++(1,1);
\path[fill=white] (16,9) rectangle ++(1,1);
\path[fill=white] (17,9) rectangle ++(1,1);
\path[fill=white] (18,9) rectangle ++(1,1);
\path[fill=white] (19,9) rectangle ++(1,1);
\path[fill=white] (20,9) rectangle ++(1,1);
\path[fill=white] (21,9) rectangle ++(1,1);
\path[fill=white] (22,9) rectangle ++(1,1);
\path[fill=white] (23,9) rectangle ++(1,1);
\path[fill=white] (24,9) rectangle ++(1,1);
\path[fill=white] (25,9) rectangle ++(1,1);
\path[fill=white] (26,9) rectangle ++(1,1);
\path[fill=white] (27,9) rectangle ++(1,1);
\path[fill=white] (28,9) rectangle ++(1,1);
\path[fill=black] (29,9) rectangle ++(1,1);
\path[fill=white] (30,9) rectangle ++(1,1);
\path[fill=white] (31,9) rectangle ++(1,1);
\path[fill=white] (32,9) rectangle ++(1,1);
\path[fill=white] (33,9) rectangle ++(1,1);
\path[fill=white] (34,9) rectangle ++(1,1);
\path[fill=white] (0,8) rectangle ++(1,1);
\path[fill=white] (1,8) rectangle ++(1,1);
\path[fill=white] (2,8) rectangle ++(1,1);
\path[fill=white] (3,8) rectangle ++(1,1);
\path[fill=white] (4,8) rectangle ++(1,1);
\path[fill=white] (5,8) rectangle ++(1,1);
\path[fill=white] (6,8) rectangle ++(1,1);
\path[fill=white] (7,8) rectangle ++(1,1);
\path[fill=white] (8,8) rectangle ++(1,1);
\path[fill=white] (9,8) rectangle ++(1,1);
\path[fill=white] (10,8) rectangle ++(1,1);
\path[fill=white] (11,8) rectangle ++(1,1);
\path[fill=white] (12,8) rectangle ++(1,1);
\path[fill=white] (13,8) rectangle ++(1,1);
\path[fill=white] (14,8) rectangle ++(1,1);
\path[fill=white] (15,8) rectangle ++(1,1);
\path[fill=white] (16,8) rectangle ++(1,1);
\path[fill=white] (17,8) rectangle ++(1,1);
\path[fill=white] (18,8) rectangle ++(1,1);
\path[fill=white] (19,8) rectangle ++(1,1);
\path[fill=white] (20,8) rectangle ++(1,1);
\path[fill=white] (21,8) rectangle ++(1,1);
\path[fill=white] (22,8) rectangle ++(1,1);
\path[fill=white] (23,8) rectangle ++(1,1);
\path[fill=white] (24,8) rectangle ++(1,1);
\path[fill=white] (25,8) rectangle ++(1,1);
\path[fill=white] (26,8) rectangle ++(1,1);
\path[fill=white] (27,8) rectangle ++(1,1);
\path[fill=white] (28,8) rectangle ++(1,1);
\path[fill=black] (29,8) rectangle ++(1,1);
\path[fill=black] (30,8) rectangle ++(1,1);
\path[fill=white] (31,8) rectangle ++(1,1);
\path[fill=black] (32,8) rectangle ++(1,1);
\path[fill=white] (33,8) rectangle ++(1,1);
\path[fill=black] (34,8) rectangle ++(1,1);
\path[fill=white] (0,7) rectangle ++(1,1);
\path[fill=white] (1,7) rectangle ++(1,1);
\path[fill=white] (2,7) rectangle ++(1,1);
\path[fill=white] (3,7) rectangle ++(1,1);
\path[fill=white] (4,7) rectangle ++(1,1);
\path[fill=white] (5,7) rectangle ++(1,1);
\path[fill=white] (6,7) rectangle ++(1,1);
\path[fill=white] (7,7) rectangle ++(1,1);
\path[fill=white] (8,7) rectangle ++(1,1);
\path[fill=white] (9,7) rectangle ++(1,1);
\path[fill=white] (10,7) rectangle ++(1,1);
\path[fill=white] (11,7) rectangle ++(1,1);
\path[fill=white] (12,7) rectangle ++(1,1);
\path[fill=white] (13,7) rectangle ++(1,1);
\path[fill=white] (14,7) rectangle ++(1,1);
\path[fill=white] (15,7) rectangle ++(1,1);
\path[fill=white] (16,7) rectangle ++(1,1);
\path[fill=white] (17,7) rectangle ++(1,1);
\path[fill=white] (18,7) rectangle ++(1,1);
\path[fill=white] (19,7) rectangle ++(1,1);
\path[fill=white] (20,7) rectangle ++(1,1);
\path[fill=white] (21,7) rectangle ++(1,1);
\path[fill=white] (22,7) rectangle ++(1,1);
\path[fill=white] (23,7) rectangle ++(1,1);
\path[fill=white] (24,7) rectangle ++(1,1);
\path[fill=white] (25,7) rectangle ++(1,1);
\path[fill=white] (26,7) rectangle ++(1,1);
\path[fill=white] (27,7) rectangle ++(1,1);
\path[fill=white] (28,7) rectangle ++(1,1);
\path[fill=white] (29,7) rectangle ++(1,1);
\path[fill=black] (30,7) rectangle ++(1,1);
\path[fill=black] (31,7) rectangle ++(1,1);
\path[fill=white] (32,7) rectangle ++(1,1);
\path[fill=black] (33,7) rectangle ++(1,1);
\path[fill=white] (34,7) rectangle ++(1,1);
\path[fill=white] (0,6) rectangle ++(1,1);
\path[fill=white] (1,6) rectangle ++(1,1);
\path[fill=white] (2,6) rectangle ++(1,1);
\path[fill=white] (3,6) rectangle ++(1,1);
\path[fill=white] (4,6) rectangle ++(1,1);
\path[fill=white] (5,6) rectangle ++(1,1);
\path[fill=white] (6,6) rectangle ++(1,1);
\path[fill=white] (7,6) rectangle ++(1,1);
\path[fill=white] (8,6) rectangle ++(1,1);
\path[fill=white] (9,6) rectangle ++(1,1);
\path[fill=white] (10,6) rectangle ++(1,1);
\path[fill=white] (11,6) rectangle ++(1,1);
\path[fill=white] (12,6) rectangle ++(1,1);
\path[fill=white] (13,6) rectangle ++(1,1);
\path[fill=white] (14,6) rectangle ++(1,1);
\path[fill=white] (15,6) rectangle ++(1,1);
\path[fill=white] (16,6) rectangle ++(1,1);
\path[fill=white] (17,6) rectangle ++(1,1);
\path[fill=white] (18,6) rectangle ++(1,1);
\path[fill=white] (19,6) rectangle ++(1,1);
\path[fill=white] (20,6) rectangle ++(1,1);
\path[fill=white] (21,6) rectangle ++(1,1);
\path[fill=white] (22,6) rectangle ++(1,1);
\path[fill=white] (23,6) rectangle ++(1,1);
\path[fill=white] (24,6) rectangle ++(1,1);
\path[fill=white] (25,6) rectangle ++(1,1);
\path[fill=white] (26,6) rectangle ++(1,1);
\path[fill=white] (27,6) rectangle ++(1,1);
\path[fill=white] (28,6) rectangle ++(1,1);
\path[fill=white] (29,6) rectangle ++(1,1);
\path[fill=white] (30,6) rectangle ++(1,1);
\path[fill=black] (31,6) rectangle ++(1,1);
\path[fill=white] (32,6) rectangle ++(1,1);
\path[fill=white] (33,6) rectangle ++(1,1);
\path[fill=white] (34,6) rectangle ++(1,1);
\path[fill=white] (0,5) rectangle ++(1,1);
\path[fill=white] (1,5) rectangle ++(1,1);
\path[fill=white] (2,5) rectangle ++(1,1);
\path[fill=white] (3,5) rectangle ++(1,1);
\path[fill=white] (4,5) rectangle ++(1,1);
\path[fill=white] (5,5) rectangle ++(1,1);
\path[fill=white] (6,5) rectangle ++(1,1);
\path[fill=white] (7,5) rectangle ++(1,1);
\path[fill=white] (8,5) rectangle ++(1,1);
\path[fill=white] (9,5) rectangle ++(1,1);
\path[fill=white] (10,5) rectangle ++(1,1);
\path[fill=white] (11,5) rectangle ++(1,1);
\path[fill=white] (12,5) rectangle ++(1,1);
\path[fill=white] (13,5) rectangle ++(1,1);
\path[fill=white] (14,5) rectangle ++(1,1);
\path[fill=white] (15,5) rectangle ++(1,1);
\path[fill=white] (16,5) rectangle ++(1,1);
\path[fill=white] (17,5) rectangle ++(1,1);
\path[fill=white] (18,5) rectangle ++(1,1);
\path[fill=white] (19,5) rectangle ++(1,1);
\path[fill=white] (20,5) rectangle ++(1,1);
\path[fill=white] (21,5) rectangle ++(1,1);
\path[fill=white] (22,5) rectangle ++(1,1);
\path[fill=white] (23,5) rectangle ++(1,1);
\path[fill=white] (24,5) rectangle ++(1,1);
\path[fill=white] (25,5) rectangle ++(1,1);
\path[fill=white] (26,5) rectangle ++(1,1);
\path[fill=white] (27,5) rectangle ++(1,1);
\path[fill=white] (28,5) rectangle ++(1,1);
\path[fill=white] (29,5) rectangle ++(1,1);
\path[fill=white] (30,5) rectangle ++(1,1);
\path[fill=white] (31,5) rectangle ++(1,1);
\path[fill=black] (32,5) rectangle ++(1,1);
\path[fill=white] (33,5) rectangle ++(1,1);
\path[fill=white] (34,5) rectangle ++(1,1);
\path[fill=white] (0,4) rectangle ++(1,1);
\path[fill=white] (1,4) rectangle ++(1,1);
\path[fill=white] (2,4) rectangle ++(1,1);
\path[fill=white] (3,4) rectangle ++(1,1);
\path[fill=white] (4,4) rectangle ++(1,1);
\path[fill=white] (5,4) rectangle ++(1,1);
\path[fill=white] (6,4) rectangle ++(1,1);
\path[fill=white] (7,4) rectangle ++(1,1);
\path[fill=white] (8,4) rectangle ++(1,1);
\path[fill=white] (9,4) rectangle ++(1,1);
\path[fill=white] (10,4) rectangle ++(1,1);
\path[fill=white] (11,4) rectangle ++(1,1);
\path[fill=white] (12,4) rectangle ++(1,1);
\path[fill=white] (13,4) rectangle ++(1,1);
\path[fill=white] (14,4) rectangle ++(1,1);
\path[fill=white] (15,4) rectangle ++(1,1);
\path[fill=white] (16,4) rectangle ++(1,1);
\path[fill=white] (17,4) rectangle ++(1,1);
\path[fill=white] (18,4) rectangle ++(1,1);
\path[fill=white] (19,4) rectangle ++(1,1);
\path[fill=white] (20,4) rectangle ++(1,1);
\path[fill=white] (21,4) rectangle ++(1,1);
\path[fill=white] (22,4) rectangle ++(1,1);
\path[fill=white] (23,4) rectangle ++(1,1);
\path[fill=white] (24,4) rectangle ++(1,1);
\path[fill=white] (25,4) rectangle ++(1,1);
\path[fill=white] (26,4) rectangle ++(1,1);
\path[fill=white] (27,4) rectangle ++(1,1);
\path[fill=white] (28,4) rectangle ++(1,1);
\path[fill=white] (29,4) rectangle ++(1,1);
\path[fill=white] (30,4) rectangle ++(1,1);
\path[fill=white] (31,4) rectangle ++(1,1);
\path[fill=black] (32,4) rectangle ++(1,1);
\path[fill=black] (33,4) rectangle ++(1,1);
\path[fill=black] (34,4) rectangle ++(1,1);
\path[fill=white] (0,3) rectangle ++(1,1);
\path[fill=white] (1,3) rectangle ++(1,1);
\path[fill=white] (2,3) rectangle ++(1,1);
\path[fill=white] (3,3) rectangle ++(1,1);
\path[fill=white] (4,3) rectangle ++(1,1);
\path[fill=white] (5,3) rectangle ++(1,1);
\path[fill=white] (6,3) rectangle ++(1,1);
\path[fill=white] (7,3) rectangle ++(1,1);
\path[fill=white] (8,3) rectangle ++(1,1);
\path[fill=white] (9,3) rectangle ++(1,1);
\path[fill=white] (10,3) rectangle ++(1,1);
\path[fill=white] (11,3) rectangle ++(1,1);
\path[fill=white] (12,3) rectangle ++(1,1);
\path[fill=white] (13,3) rectangle ++(1,1);
\path[fill=white] (14,3) rectangle ++(1,1);
\path[fill=white] (15,3) rectangle ++(1,1);
\path[fill=white] (16,3) rectangle ++(1,1);
\path[fill=white] (17,3) rectangle ++(1,1);
\path[fill=white] (18,3) rectangle ++(1,1);
\path[fill=white] (19,3) rectangle ++(1,1);
\path[fill=white] (20,3) rectangle ++(1,1);
\path[fill=white] (21,3) rectangle ++(1,1);
\path[fill=white] (22,3) rectangle ++(1,1);
\path[fill=white] (23,3) rectangle ++(1,1);
\path[fill=white] (24,3) rectangle ++(1,1);
\path[fill=white] (25,3) rectangle ++(1,1);
\path[fill=white] (26,3) rectangle ++(1,1);
\path[fill=white] (27,3) rectangle ++(1,1);
\path[fill=white] (28,3) rectangle ++(1,1);
\path[fill=white] (29,3) rectangle ++(1,1);
\path[fill=white] (30,3) rectangle ++(1,1);
\path[fill=white] (31,3) rectangle ++(1,1);
\path[fill=white] (32,3) rectangle ++(1,1);
\path[fill=black] (33,3) rectangle ++(1,1);
\path[fill=white] (34,3) rectangle ++(1,1);
\path[fill=white] (0,2) rectangle ++(1,1);
\path[fill=white] (1,2) rectangle ++(1,1);
\path[fill=white] (2,2) rectangle ++(1,1);
\path[fill=white] (3,2) rectangle ++(1,1);
\path[fill=white] (4,2) rectangle ++(1,1);
\path[fill=white] (5,2) rectangle ++(1,1);
\path[fill=white] (6,2) rectangle ++(1,1);
\path[fill=white] (7,2) rectangle ++(1,1);
\path[fill=white] (8,2) rectangle ++(1,1);
\path[fill=white] (9,2) rectangle ++(1,1);
\path[fill=white] (10,2) rectangle ++(1,1);
\path[fill=white] (11,2) rectangle ++(1,1);
\path[fill=white] (12,2) rectangle ++(1,1);
\path[fill=white] (13,2) rectangle ++(1,1);
\path[fill=white] (14,2) rectangle ++(1,1);
\path[fill=white] (15,2) rectangle ++(1,1);
\path[fill=white] (16,2) rectangle ++(1,1);
\path[fill=white] (17,2) rectangle ++(1,1);
\path[fill=white] (18,2) rectangle ++(1,1);
\path[fill=white] (19,2) rectangle ++(1,1);
\path[fill=white] (20,2) rectangle ++(1,1);
\path[fill=white] (21,2) rectangle ++(1,1);
\path[fill=white] (22,2) rectangle ++(1,1);
\path[fill=white] (23,2) rectangle ++(1,1);
\path[fill=white] (24,2) rectangle ++(1,1);
\path[fill=white] (25,2) rectangle ++(1,1);
\path[fill=white] (26,2) rectangle ++(1,1);
\path[fill=white] (27,2) rectangle ++(1,1);
\path[fill=white] (28,2) rectangle ++(1,1);
\path[fill=white] (29,2) rectangle ++(1,1);
\path[fill=white] (30,2) rectangle ++(1,1);
\path[fill=white] (31,2) rectangle ++(1,1);
\path[fill=white] (32,2) rectangle ++(1,1);
\path[fill=white] (33,2) rectangle ++(1,1);
\path[fill=black] (34,2) rectangle ++(1,1);
\path[fill=white] (0,1) rectangle ++(1,1);
\path[fill=white] (1,1) rectangle ++(1,1);
\path[fill=white] (2,1) rectangle ++(1,1);
\path[fill=white] (3,1) rectangle ++(1,1);
\path[fill=white] (4,1) rectangle ++(1,1);
\path[fill=white] (5,1) rectangle ++(1,1);
\path[fill=white] (6,1) rectangle ++(1,1);
\path[fill=white] (7,1) rectangle ++(1,1);
\path[fill=white] (8,1) rectangle ++(1,1);
\path[fill=white] (9,1) rectangle ++(1,1);
\path[fill=white] (10,1) rectangle ++(1,1);
\path[fill=white] (11,1) rectangle ++(1,1);
\path[fill=white] (12,1) rectangle ++(1,1);
\path[fill=white] (13,1) rectangle ++(1,1);
\path[fill=white] (14,1) rectangle ++(1,1);
\path[fill=white] (15,1) rectangle ++(1,1);
\path[fill=white] (16,1) rectangle ++(1,1);
\path[fill=white] (17,1) rectangle ++(1,1);
\path[fill=white] (18,1) rectangle ++(1,1);
\path[fill=white] (19,1) rectangle ++(1,1);
\path[fill=white] (20,1) rectangle ++(1,1);
\path[fill=white] (21,1) rectangle ++(1,1);
\path[fill=white] (22,1) rectangle ++(1,1);
\path[fill=white] (23,1) rectangle ++(1,1);
\path[fill=white] (24,1) rectangle ++(1,1);
\path[fill=white] (25,1) rectangle ++(1,1);
\path[fill=white] (26,1) rectangle ++(1,1);
\path[fill=white] (27,1) rectangle ++(1,1);
\path[fill=white] (28,1) rectangle ++(1,1);
\path[fill=white] (29,1) rectangle ++(1,1);
\path[fill=white] (30,1) rectangle ++(1,1);
\path[fill=white] (31,1) rectangle ++(1,1);
\path[fill=white] (32,1) rectangle ++(1,1);
\path[fill=white] (33,1) rectangle ++(1,1);
\path[fill=black] (34,1) rectangle ++(1,1);
\path[fill=white] (0,0) rectangle ++(1,1);
\path[fill=white] (1,0) rectangle ++(1,1);
\path[fill=white] (2,0) rectangle ++(1,1);
\path[fill=white] (3,0) rectangle ++(1,1);
\path[fill=white] (4,0) rectangle ++(1,1);
\path[fill=white] (5,0) rectangle ++(1,1);
\path[fill=white] (6,0) rectangle ++(1,1);
\path[fill=white] (7,0) rectangle ++(1,1);
\path[fill=white] (8,0) rectangle ++(1,1);
\path[fill=white] (9,0) rectangle ++(1,1);
\path[fill=white] (10,0) rectangle ++(1,1);
\path[fill=white] (11,0) rectangle ++(1,1);
\path[fill=white] (12,0) rectangle ++(1,1);
\path[fill=white] (13,0) rectangle ++(1,1);
\path[fill=white] (14,0) rectangle ++(1,1);
\path[fill=white] (15,0) rectangle ++(1,1);
\path[fill=white] (16,0) rectangle ++(1,1);
\path[fill=white] (17,0) rectangle ++(1,1);
\path[fill=white] (18,0) rectangle ++(1,1);
\path[fill=white] (19,0) rectangle ++(1,1);
\path[fill=white] (20,0) rectangle ++(1,1);
\path[fill=white] (21,0) rectangle ++(1,1);
\path[fill=white] (22,0) rectangle ++(1,1);
\path[fill=white] (23,0) rectangle ++(1,1);
\path[fill=white] (24,0) rectangle ++(1,1);
\path[fill=white] (25,0) rectangle ++(1,1);
\path[fill=white] (26,0) rectangle ++(1,1);
\path[fill=white] (27,0) rectangle ++(1,1);
\path[fill=white] (28,0) rectangle ++(1,1);
\path[fill=white] (29,0) rectangle ++(1,1);
\path[fill=white] (30,0) rectangle ++(1,1);
\path[fill=white] (31,0) rectangle ++(1,1);
\path[fill=white] (32,0) rectangle ++(1,1);
\path[fill=white] (33,0) rectangle ++(1,1);
\path[fill=white] (34,0) rectangle ++(1,1);
\end{tikzpicture}

\begin{tikzpicture}[x=0.3cm, y=0.3cm, draw=gray, very thin]
\path[fill=white] (0,34) rectangle ++(1,1);
\path[fill=black] (1,34) rectangle ++(1,1);
\path[fill=black] (2,34) rectangle ++(1,1);
\path[fill=black] (3,34) rectangle ++(1,1);
\path[fill=black] (4,34) rectangle ++(1,1);
\path[fill=black] (5,34) rectangle ++(1,1);
\path[fill=black] (6,34) rectangle ++(1,1);
\path[fill=black] (7,34) rectangle ++(1,1);
\path[fill=black] (8,34) rectangle ++(1,1);
\path[fill=black] (9,34) rectangle ++(1,1);
\path[fill=black] (10,34) rectangle ++(1,1);
\path[fill=black] (11,34) rectangle ++(1,1);
\path[fill=black] (12,34) rectangle ++(1,1);
\path[fill=black] (13,34) rectangle ++(1,1);
\path[fill=black] (14,34) rectangle ++(1,1);
\path[fill=black] (15,34) rectangle ++(1,1);
\path[fill=black] (16,34) rectangle ++(1,1);
\path[fill=black] (17,34) rectangle ++(1,1);
\path[fill=black] (18,34) rectangle ++(1,1);
\path[fill=black] (19,34) rectangle ++(1,1);
\path[fill=black] (20,34) rectangle ++(1,1);
\path[fill=black] (21,34) rectangle ++(1,1);
\path[fill=black] (22,34) rectangle ++(1,1);
\path[fill=black] (23,34) rectangle ++(1,1);
\path[fill=black] (24,34) rectangle ++(1,1);
\path[fill=black] (25,34) rectangle ++(1,1);
\path[fill=black] (26,34) rectangle ++(1,1);
\path[fill=black] (27,34) rectangle ++(1,1);
\path[fill=black] (28,34) rectangle ++(1,1);
\path[fill=black] (29,34) rectangle ++(1,1);
\path[fill=black] (30,34) rectangle ++(1,1);
\path[fill=black] (31,34) rectangle ++(1,1);
\path[fill=black] (32,34) rectangle ++(1,1);
\path[fill=black] (33,34) rectangle ++(1,1);
\path[fill=black] (34,34) rectangle ++(1,1);
\path[fill=white] (0,33) rectangle ++(1,1);
\path[fill=white] (1,33) rectangle ++(1,1);
\path[fill=white] (2,33) rectangle ++(1,1);
\path[fill=black] (3,33) rectangle ++(1,1);
\path[fill=black] (4,33) rectangle ++(1,1);
\path[fill=white] (5,33) rectangle ++(1,1);
\path[fill=black] (6,33) rectangle ++(1,1);
\path[fill=black] (7,33) rectangle ++(1,1);
\path[fill=black] (8,33) rectangle ++(1,1);
\path[fill=white] (9,33) rectangle ++(1,1);
\path[fill=black] (10,33) rectangle ++(1,1);
\path[fill=black] (11,33) rectangle ++(1,1);
\path[fill=black] (12,33) rectangle ++(1,1);
\path[fill=black] (13,33) rectangle ++(1,1);
\path[fill=white] (14,33) rectangle ++(1,1);
\path[fill=black] (15,33) rectangle ++(1,1);
\path[fill=black] (16,33) rectangle ++(1,1);
\path[fill=black] (17,33) rectangle ++(1,1);
\path[fill=black] (18,33) rectangle ++(1,1);
\path[fill=white] (19,33) rectangle ++(1,1);
\path[fill=black] (20,33) rectangle ++(1,1);
\path[fill=black] (21,33) rectangle ++(1,1);
\path[fill=black] (22,33) rectangle ++(1,1);
\path[fill=black] (23,33) rectangle ++(1,1);
\path[fill=white] (24,33) rectangle ++(1,1);
\path[fill=black] (25,33) rectangle ++(1,1);
\path[fill=black] (26,33) rectangle ++(1,1);
\path[fill=black] (27,33) rectangle ++(1,1);
\path[fill=black] (28,33) rectangle ++(1,1);
\path[fill=black] (29,33) rectangle ++(1,1);
\path[fill=black] (30,33) rectangle ++(1,1);
\path[fill=black] (31,33) rectangle ++(1,1);
\path[fill=black] (32,33) rectangle ++(1,1);
\path[fill=black] (33,33) rectangle ++(1,1);
\path[fill=black] (34,33) rectangle ++(1,1);
\path[fill=white] (0,32) rectangle ++(1,1);
\path[fill=white] (1,32) rectangle ++(1,1);
\path[fill=white] (2,32) rectangle ++(1,1);
\path[fill=white] (3,32) rectangle ++(1,1);
\path[fill=black] (4,32) rectangle ++(1,1);
\path[fill=black] (5,32) rectangle ++(1,1);
\path[fill=white] (6,32) rectangle ++(1,1);
\path[fill=black] (7,32) rectangle ++(1,1);
\path[fill=black] (8,32) rectangle ++(1,1);
\path[fill=black] (9,32) rectangle ++(1,1);
\path[fill=white] (10,32) rectangle ++(1,1);
\path[fill=black] (11,32) rectangle ++(1,1);
\path[fill=black] (12,32) rectangle ++(1,1);
\path[fill=black] (13,32) rectangle ++(1,1);
\path[fill=black] (14,32) rectangle ++(1,1);
\path[fill=black] (15,32) rectangle ++(1,1);
\path[fill=black] (16,32) rectangle ++(1,1);
\path[fill=black] (17,32) rectangle ++(1,1);
\path[fill=black] (18,32) rectangle ++(1,1);
\path[fill=black] (19,32) rectangle ++(1,1);
\path[fill=black] (20,32) rectangle ++(1,1);
\path[fill=black] (21,32) rectangle ++(1,1);
\path[fill=black] (22,32) rectangle ++(1,1);
\path[fill=black] (23,32) rectangle ++(1,1);
\path[fill=black] (24,32) rectangle ++(1,1);
\path[fill=black] (25,32) rectangle ++(1,1);
\path[fill=black] (26,32) rectangle ++(1,1);
\path[fill=black] (27,32) rectangle ++(1,1);
\path[fill=black] (28,32) rectangle ++(1,1);
\path[fill=black] (29,32) rectangle ++(1,1);
\path[fill=black] (30,32) rectangle ++(1,1);
\path[fill=black] (31,32) rectangle ++(1,1);
\path[fill=black] (32,32) rectangle ++(1,1);
\path[fill=black] (33,32) rectangle ++(1,1);
\path[fill=black] (34,32) rectangle ++(1,1);
\path[fill=white] (0,31) rectangle ++(1,1);
\path[fill=white] (1,31) rectangle ++(1,1);
\path[fill=white] (2,31) rectangle ++(1,1);
\path[fill=white] (3,31) rectangle ++(1,1);
\path[fill=white] (4,31) rectangle ++(1,1);
\path[fill=white] (5,31) rectangle ++(1,1);
\path[fill=black] (6,31) rectangle ++(1,1);
\path[fill=black] (7,31) rectangle ++(1,1);
\path[fill=white] (8,31) rectangle ++(1,1);
\path[fill=white] (9,31) rectangle ++(1,1);
\path[fill=black] (10,31) rectangle ++(1,1);
\path[fill=black] (11,31) rectangle ++(1,1);
\path[fill=black] (12,31) rectangle ++(1,1);
\path[fill=white] (13,31) rectangle ++(1,1);
\path[fill=white] (14,31) rectangle ++(1,1);
\path[fill=black] (15,31) rectangle ++(1,1);
\path[fill=black] (16,31) rectangle ++(1,1);
\path[fill=black] (17,31) rectangle ++(1,1);
\path[fill=white] (18,31) rectangle ++(1,1);
\path[fill=white] (19,31) rectangle ++(1,1);
\path[fill=black] (20,31) rectangle ++(1,1);
\path[fill=black] (21,31) rectangle ++(1,1);
\path[fill=black] (22,31) rectangle ++(1,1);
\path[fill=white] (23,31) rectangle ++(1,1);
\path[fill=white] (24,31) rectangle ++(1,1);
\path[fill=black] (25,31) rectangle ++(1,1);
\path[fill=black] (26,31) rectangle ++(1,1);
\path[fill=black] (27,31) rectangle ++(1,1);
\path[fill=white] (28,31) rectangle ++(1,1);
\path[fill=black] (29,31) rectangle ++(1,1);
\path[fill=black] (30,31) rectangle ++(1,1);
\path[fill=black] (31,31) rectangle ++(1,1);
\path[fill=black] (32,31) rectangle ++(1,1);
\path[fill=black] (33,31) rectangle ++(1,1);
\path[fill=black] (34,31) rectangle ++(1,1);
\path[fill=white] (0,30) rectangle ++(1,1);
\path[fill=white] (1,30) rectangle ++(1,1);
\path[fill=white] (2,30) rectangle ++(1,1);
\path[fill=white] (3,30) rectangle ++(1,1);
\path[fill=white] (4,30) rectangle ++(1,1);
\path[fill=white] (5,30) rectangle ++(1,1);
\path[fill=white] (6,30) rectangle ++(1,1);
\path[fill=black] (7,30) rectangle ++(1,1);
\path[fill=black] (8,30) rectangle ++(1,1);
\path[fill=white] (9,30) rectangle ++(1,1);
\path[fill=white] (10,30) rectangle ++(1,1);
\path[fill=black] (11,30) rectangle ++(1,1);
\path[fill=black] (12,30) rectangle ++(1,1);
\path[fill=black] (13,30) rectangle ++(1,1);
\path[fill=white] (14,30) rectangle ++(1,1);
\path[fill=black] (15,30) rectangle ++(1,1);
\path[fill=black] (16,30) rectangle ++(1,1);
\path[fill=black] (17,30) rectangle ++(1,1);
\path[fill=black] (18,30) rectangle ++(1,1);
\path[fill=white] (19,30) rectangle ++(1,1);
\path[fill=black] (20,30) rectangle ++(1,1);
\path[fill=black] (21,30) rectangle ++(1,1);
\path[fill=black] (22,30) rectangle ++(1,1);
\path[fill=black] (23,30) rectangle ++(1,1);
\path[fill=white] (24,30) rectangle ++(1,1);
\path[fill=black] (25,30) rectangle ++(1,1);
\path[fill=black] (26,30) rectangle ++(1,1);
\path[fill=black] (27,30) rectangle ++(1,1);
\path[fill=black] (28,30) rectangle ++(1,1);
\path[fill=black] (29,30) rectangle ++(1,1);
\path[fill=black] (30,30) rectangle ++(1,1);
\path[fill=black] (31,30) rectangle ++(1,1);
\path[fill=black] (32,30) rectangle ++(1,1);
\path[fill=black] (33,30) rectangle ++(1,1);
\path[fill=black] (34,30) rectangle ++(1,1);
\path[fill=white] (0,29) rectangle ++(1,1);
\path[fill=white] (1,29) rectangle ++(1,1);
\path[fill=white] (2,29) rectangle ++(1,1);
\path[fill=white] (3,29) rectangle ++(1,1);
\path[fill=white] (4,29) rectangle ++(1,1);
\path[fill=white] (5,29) rectangle ++(1,1);
\path[fill=white] (6,29) rectangle ++(1,1);
\path[fill=white] (7,29) rectangle ++(1,1);
\path[fill=black] (8,29) rectangle ++(1,1);
\path[fill=black] (9,29) rectangle ++(1,1);
\path[fill=white] (10,29) rectangle ++(1,1);
\path[fill=white] (11,29) rectangle ++(1,1);
\path[fill=black] (12,29) rectangle ++(1,1);
\path[fill=black] (13,29) rectangle ++(1,1);
\path[fill=black] (14,29) rectangle ++(1,1);
\path[fill=white] (15,29) rectangle ++(1,1);
\path[fill=black] (16,29) rectangle ++(1,1);
\path[fill=black] (17,29) rectangle ++(1,1);
\path[fill=black] (18,29) rectangle ++(1,1);
\path[fill=black] (19,29) rectangle ++(1,1);
\path[fill=black] (20,29) rectangle ++(1,1);
\path[fill=black] (21,29) rectangle ++(1,1);
\path[fill=black] (22,29) rectangle ++(1,1);
\path[fill=black] (23,29) rectangle ++(1,1);
\path[fill=black] (24,29) rectangle ++(1,1);
\path[fill=black] (25,29) rectangle ++(1,1);
\path[fill=black] (26,29) rectangle ++(1,1);
\path[fill=black] (27,29) rectangle ++(1,1);
\path[fill=black] (28,29) rectangle ++(1,1);
\path[fill=black] (29,29) rectangle ++(1,1);
\path[fill=black] (30,29) rectangle ++(1,1);
\path[fill=black] (31,29) rectangle ++(1,1);
\path[fill=black] (32,29) rectangle ++(1,1);
\path[fill=black] (33,29) rectangle ++(1,1);
\path[fill=black] (34,29) rectangle ++(1,1);
\path[fill=white] (0,28) rectangle ++(1,1);
\path[fill=white] (1,28) rectangle ++(1,1);
\path[fill=white] (2,28) rectangle ++(1,1);
\path[fill=white] (3,28) rectangle ++(1,1);
\path[fill=white] (4,28) rectangle ++(1,1);
\path[fill=white] (5,28) rectangle ++(1,1);
\path[fill=white] (6,28) rectangle ++(1,1);
\path[fill=white] (7,28) rectangle ++(1,1);
\path[fill=white] (8,28) rectangle ++(1,1);
\path[fill=white] (9,28) rectangle ++(1,1);
\path[fill=black] (10,28) rectangle ++(1,1);
\path[fill=black] (11,28) rectangle ++(1,1);
\path[fill=white] (12,28) rectangle ++(1,1);
\path[fill=white] (13,28) rectangle ++(1,1);
\path[fill=white] (14,28) rectangle ++(1,1);
\path[fill=black] (15,28) rectangle ++(1,1);
\path[fill=black] (16,28) rectangle ++(1,1);
\path[fill=white] (17,28) rectangle ++(1,1);
\path[fill=white] (18,28) rectangle ++(1,1);
\path[fill=white] (19,28) rectangle ++(1,1);
\path[fill=black] (20,28) rectangle ++(1,1);
\path[fill=black] (21,28) rectangle ++(1,1);
\path[fill=white] (22,28) rectangle ++(1,1);
\path[fill=white] (23,28) rectangle ++(1,1);
\path[fill=white] (24,28) rectangle ++(1,1);
\path[fill=black] (25,28) rectangle ++(1,1);
\path[fill=black] (26,28) rectangle ++(1,1);
\path[fill=white] (27,28) rectangle ++(1,1);
\path[fill=white] (28,28) rectangle ++(1,1);
\path[fill=black] (29,28) rectangle ++(1,1);
\path[fill=black] (30,28) rectangle ++(1,1);
\path[fill=white] (31,28) rectangle ++(1,1);
\path[fill=black] (32,28) rectangle ++(1,1);
\path[fill=black] (33,28) rectangle ++(1,1);
\path[fill=black] (34,28) rectangle ++(1,1);
\path[fill=white] (0,27) rectangle ++(1,1);
\path[fill=white] (1,27) rectangle ++(1,1);
\path[fill=white] (2,27) rectangle ++(1,1);
\path[fill=white] (3,27) rectangle ++(1,1);
\path[fill=white] (4,27) rectangle ++(1,1);
\path[fill=white] (5,27) rectangle ++(1,1);
\path[fill=white] (6,27) rectangle ++(1,1);
\path[fill=white] (7,27) rectangle ++(1,1);
\path[fill=white] (8,27) rectangle ++(1,1);
\path[fill=white] (9,27) rectangle ++(1,1);
\path[fill=white] (10,27) rectangle ++(1,1);
\path[fill=black] (11,27) rectangle ++(1,1);
\path[fill=black] (12,27) rectangle ++(1,1);
\path[fill=white] (13,27) rectangle ++(1,1);
\path[fill=white] (14,27) rectangle ++(1,1);
\path[fill=black] (15,27) rectangle ++(1,1);
\path[fill=black] (16,27) rectangle ++(1,1);
\path[fill=black] (17,27) rectangle ++(1,1);
\path[fill=white] (18,27) rectangle ++(1,1);
\path[fill=white] (19,27) rectangle ++(1,1);
\path[fill=black] (20,27) rectangle ++(1,1);
\path[fill=black] (21,27) rectangle ++(1,1);
\path[fill=black] (22,27) rectangle ++(1,1);
\path[fill=white] (23,27) rectangle ++(1,1);
\path[fill=white] (24,27) rectangle ++(1,1);
\path[fill=black] (25,27) rectangle ++(1,1);
\path[fill=black] (26,27) rectangle ++(1,1);
\path[fill=black] (27,27) rectangle ++(1,1);
\path[fill=white] (28,27) rectangle ++(1,1);
\path[fill=black] (29,27) rectangle ++(1,1);
\path[fill=black] (30,27) rectangle ++(1,1);
\path[fill=black] (31,27) rectangle ++(1,1);
\path[fill=black] (32,27) rectangle ++(1,1);
\path[fill=black] (33,27) rectangle ++(1,1);
\path[fill=black] (34,27) rectangle ++(1,1);
\path[fill=white] (0,26) rectangle ++(1,1);
\path[fill=white] (1,26) rectangle ++(1,1);
\path[fill=white] (2,26) rectangle ++(1,1);
\path[fill=white] (3,26) rectangle ++(1,1);
\path[fill=white] (4,26) rectangle ++(1,1);
\path[fill=white] (5,26) rectangle ++(1,1);
\path[fill=white] (6,26) rectangle ++(1,1);
\path[fill=white] (7,26) rectangle ++(1,1);
\path[fill=white] (8,26) rectangle ++(1,1);
\path[fill=white] (9,26) rectangle ++(1,1);
\path[fill=white] (10,26) rectangle ++(1,1);
\path[fill=white] (11,26) rectangle ++(1,1);
\path[fill=black] (12,26) rectangle ++(1,1);
\path[fill=black] (13,26) rectangle ++(1,1);
\path[fill=white] (14,26) rectangle ++(1,1);
\path[fill=white] (15,26) rectangle ++(1,1);
\path[fill=black] (16,26) rectangle ++(1,1);
\path[fill=black] (17,26) rectangle ++(1,1);
\path[fill=black] (18,26) rectangle ++(1,1);
\path[fill=white] (19,26) rectangle ++(1,1);
\path[fill=black] (20,26) rectangle ++(1,1);
\path[fill=black] (21,26) rectangle ++(1,1);
\path[fill=black] (22,26) rectangle ++(1,1);
\path[fill=black] (23,26) rectangle ++(1,1);
\path[fill=white] (24,26) rectangle ++(1,1);
\path[fill=black] (25,26) rectangle ++(1,1);
\path[fill=black] (26,26) rectangle ++(1,1);
\path[fill=black] (27,26) rectangle ++(1,1);
\path[fill=black] (28,26) rectangle ++(1,1);
\path[fill=black] (29,26) rectangle ++(1,1);
\path[fill=black] (30,26) rectangle ++(1,1);
\path[fill=black] (31,26) rectangle ++(1,1);
\path[fill=black] (32,26) rectangle ++(1,1);
\path[fill=black] (33,26) rectangle ++(1,1);
\path[fill=black] (34,26) rectangle ++(1,1);
\path[fill=white] (0,25) rectangle ++(1,1);
\path[fill=white] (1,25) rectangle ++(1,1);
\path[fill=white] (2,25) rectangle ++(1,1);
\path[fill=white] (3,25) rectangle ++(1,1);
\path[fill=white] (4,25) rectangle ++(1,1);
\path[fill=white] (5,25) rectangle ++(1,1);
\path[fill=white] (6,25) rectangle ++(1,1);
\path[fill=white] (7,25) rectangle ++(1,1);
\path[fill=white] (8,25) rectangle ++(1,1);
\path[fill=white] (9,25) rectangle ++(1,1);
\path[fill=white] (10,25) rectangle ++(1,1);
\path[fill=white] (11,25) rectangle ++(1,1);
\path[fill=white] (12,25) rectangle ++(1,1);
\path[fill=black] (13,25) rectangle ++(1,1);
\path[fill=black] (14,25) rectangle ++(1,1);
\path[fill=white] (15,25) rectangle ++(1,1);
\path[fill=white] (16,25) rectangle ++(1,1);
\path[fill=black] (17,25) rectangle ++(1,1);
\path[fill=black] (18,25) rectangle ++(1,1);
\path[fill=black] (19,25) rectangle ++(1,1);
\path[fill=white] (20,25) rectangle ++(1,1);
\path[fill=black] (21,25) rectangle ++(1,1);
\path[fill=black] (22,25) rectangle ++(1,1);
\path[fill=black] (23,25) rectangle ++(1,1);
\path[fill=black] (24,25) rectangle ++(1,1);
\path[fill=black] (25,25) rectangle ++(1,1);
\path[fill=black] (26,25) rectangle ++(1,1);
\path[fill=black] (27,25) rectangle ++(1,1);
\path[fill=black] (28,25) rectangle ++(1,1);
\path[fill=black] (29,25) rectangle ++(1,1);
\path[fill=black] (30,25) rectangle ++(1,1);
\path[fill=black] (31,25) rectangle ++(1,1);
\path[fill=black] (32,25) rectangle ++(1,1);
\path[fill=black] (33,25) rectangle ++(1,1);
\path[fill=black] (34,25) rectangle ++(1,1);
\path[fill=white] (0,24) rectangle ++(1,1);
\path[fill=white] (1,24) rectangle ++(1,1);
\path[fill=white] (2,24) rectangle ++(1,1);
\path[fill=white] (3,24) rectangle ++(1,1);
\path[fill=white] (4,24) rectangle ++(1,1);
\path[fill=white] (5,24) rectangle ++(1,1);
\path[fill=white] (6,24) rectangle ++(1,1);
\path[fill=white] (7,24) rectangle ++(1,1);
\path[fill=white] (8,24) rectangle ++(1,1);
\path[fill=white] (9,24) rectangle ++(1,1);
\path[fill=white] (10,24) rectangle ++(1,1);
\path[fill=white] (11,24) rectangle ++(1,1);
\path[fill=white] (12,24) rectangle ++(1,1);
\path[fill=white] (13,24) rectangle ++(1,1);
\path[fill=white] (14,24) rectangle ++(1,1);
\path[fill=black] (15,24) rectangle ++(1,1);
\path[fill=white] (16,24) rectangle ++(1,1);
\path[fill=white] (17,24) rectangle ++(1,1);
\path[fill=white] (18,24) rectangle ++(1,1);
\path[fill=white] (19,24) rectangle ++(1,1);
\path[fill=black] (20,24) rectangle ++(1,1);
\path[fill=white] (21,24) rectangle ++(1,1);
\path[fill=white] (22,24) rectangle ++(1,1);
\path[fill=white] (23,24) rectangle ++(1,1);
\path[fill=white] (24,24) rectangle ++(1,1);
\path[fill=black] (25,24) rectangle ++(1,1);
\path[fill=white] (26,24) rectangle ++(1,1);
\path[fill=white] (27,24) rectangle ++(1,1);
\path[fill=white] (28,24) rectangle ++(1,1);
\path[fill=black] (29,24) rectangle ++(1,1);
\path[fill=white] (30,24) rectangle ++(1,1);
\path[fill=white] (31,24) rectangle ++(1,1);
\path[fill=black] (32,24) rectangle ++(1,1);
\path[fill=white] (33,24) rectangle ++(1,1);
\path[fill=black] (34,24) rectangle ++(1,1);
\path[fill=white] (0,23) rectangle ++(1,1);
\path[fill=white] (1,23) rectangle ++(1,1);
\path[fill=white] (2,23) rectangle ++(1,1);
\path[fill=white] (3,23) rectangle ++(1,1);
\path[fill=white] (4,23) rectangle ++(1,1);
\path[fill=white] (5,23) rectangle ++(1,1);
\path[fill=white] (6,23) rectangle ++(1,1);
\path[fill=white] (7,23) rectangle ++(1,1);
\path[fill=white] (8,23) rectangle ++(1,1);
\path[fill=white] (9,23) rectangle ++(1,1);
\path[fill=white] (10,23) rectangle ++(1,1);
\path[fill=white] (11,23) rectangle ++(1,1);
\path[fill=white] (12,23) rectangle ++(1,1);
\path[fill=white] (13,23) rectangle ++(1,1);
\path[fill=white] (14,23) rectangle ++(1,1);
\path[fill=black] (15,23) rectangle ++(1,1);
\path[fill=black] (16,23) rectangle ++(1,1);
\path[fill=white] (17,23) rectangle ++(1,1);
\path[fill=white] (18,23) rectangle ++(1,1);
\path[fill=white] (19,23) rectangle ++(1,1);
\path[fill=black] (20,23) rectangle ++(1,1);
\path[fill=black] (21,23) rectangle ++(1,1);
\path[fill=white] (22,23) rectangle ++(1,1);
\path[fill=white] (23,23) rectangle ++(1,1);
\path[fill=white] (24,23) rectangle ++(1,1);
\path[fill=black] (25,23) rectangle ++(1,1);
\path[fill=black] (26,23) rectangle ++(1,1);
\path[fill=white] (27,23) rectangle ++(1,1);
\path[fill=white] (28,23) rectangle ++(1,1);
\path[fill=black] (29,23) rectangle ++(1,1);
\path[fill=black] (30,23) rectangle ++(1,1);
\path[fill=white] (31,23) rectangle ++(1,1);
\path[fill=black] (32,23) rectangle ++(1,1);
\path[fill=black] (33,23) rectangle ++(1,1);
\path[fill=black] (34,23) rectangle ++(1,1);
\path[fill=white] (0,22) rectangle ++(1,1);
\path[fill=white] (1,22) rectangle ++(1,1);
\path[fill=white] (2,22) rectangle ++(1,1);
\path[fill=white] (3,22) rectangle ++(1,1);
\path[fill=white] (4,22) rectangle ++(1,1);
\path[fill=white] (5,22) rectangle ++(1,1);
\path[fill=white] (6,22) rectangle ++(1,1);
\path[fill=white] (7,22) rectangle ++(1,1);
\path[fill=white] (8,22) rectangle ++(1,1);
\path[fill=white] (9,22) rectangle ++(1,1);
\path[fill=white] (10,22) rectangle ++(1,1);
\path[fill=white] (11,22) rectangle ++(1,1);
\path[fill=white] (12,22) rectangle ++(1,1);
\path[fill=white] (13,22) rectangle ++(1,1);
\path[fill=white] (14,22) rectangle ++(1,1);
\path[fill=white] (15,22) rectangle ++(1,1);
\path[fill=black] (16,22) rectangle ++(1,1);
\path[fill=black] (17,22) rectangle ++(1,1);
\path[fill=white] (18,22) rectangle ++(1,1);
\path[fill=white] (19,22) rectangle ++(1,1);
\path[fill=black] (20,22) rectangle ++(1,1);
\path[fill=black] (21,22) rectangle ++(1,1);
\path[fill=black] (22,22) rectangle ++(1,1);
\path[fill=white] (23,22) rectangle ++(1,1);
\path[fill=white] (24,22) rectangle ++(1,1);
\path[fill=black] (25,22) rectangle ++(1,1);
\path[fill=black] (26,22) rectangle ++(1,1);
\path[fill=black] (27,22) rectangle ++(1,1);
\path[fill=white] (28,22) rectangle ++(1,1);
\path[fill=black] (29,22) rectangle ++(1,1);
\path[fill=black] (30,22) rectangle ++(1,1);
\path[fill=black] (31,22) rectangle ++(1,1);
\path[fill=black] (32,22) rectangle ++(1,1);
\path[fill=black] (33,22) rectangle ++(1,1);
\path[fill=black] (34,22) rectangle ++(1,1);
\path[fill=white] (0,21) rectangle ++(1,1);
\path[fill=white] (1,21) rectangle ++(1,1);
\path[fill=white] (2,21) rectangle ++(1,1);
\path[fill=white] (3,21) rectangle ++(1,1);
\path[fill=white] (4,21) rectangle ++(1,1);
\path[fill=white] (5,21) rectangle ++(1,1);
\path[fill=white] (6,21) rectangle ++(1,1);
\path[fill=white] (7,21) rectangle ++(1,1);
\path[fill=white] (8,21) rectangle ++(1,1);
\path[fill=white] (9,21) rectangle ++(1,1);
\path[fill=white] (10,21) rectangle ++(1,1);
\path[fill=white] (11,21) rectangle ++(1,1);
\path[fill=white] (12,21) rectangle ++(1,1);
\path[fill=white] (13,21) rectangle ++(1,1);
\path[fill=white] (14,21) rectangle ++(1,1);
\path[fill=white] (15,21) rectangle ++(1,1);
\path[fill=white] (16,21) rectangle ++(1,1);
\path[fill=black] (17,21) rectangle ++(1,1);
\path[fill=black] (18,21) rectangle ++(1,1);
\path[fill=white] (19,21) rectangle ++(1,1);
\path[fill=white] (20,21) rectangle ++(1,1);
\path[fill=black] (21,21) rectangle ++(1,1);
\path[fill=black] (22,21) rectangle ++(1,1);
\path[fill=black] (23,21) rectangle ++(1,1);
\path[fill=white] (24,21) rectangle ++(1,1);
\path[fill=black] (25,21) rectangle ++(1,1);
\path[fill=black] (26,21) rectangle ++(1,1);
\path[fill=black] (27,21) rectangle ++(1,1);
\path[fill=black] (28,21) rectangle ++(1,1);
\path[fill=black] (29,21) rectangle ++(1,1);
\path[fill=black] (30,21) rectangle ++(1,1);
\path[fill=black] (31,21) rectangle ++(1,1);
\path[fill=black] (32,21) rectangle ++(1,1);
\path[fill=black] (33,21) rectangle ++(1,1);
\path[fill=black] (34,21) rectangle ++(1,1);
\path[fill=white] (0,20) rectangle ++(1,1);
\path[fill=white] (1,20) rectangle ++(1,1);
\path[fill=white] (2,20) rectangle ++(1,1);
\path[fill=white] (3,20) rectangle ++(1,1);
\path[fill=white] (4,20) rectangle ++(1,1);
\path[fill=white] (5,20) rectangle ++(1,1);
\path[fill=white] (6,20) rectangle ++(1,1);
\path[fill=white] (7,20) rectangle ++(1,1);
\path[fill=white] (8,20) rectangle ++(1,1);
\path[fill=white] (9,20) rectangle ++(1,1);
\path[fill=white] (10,20) rectangle ++(1,1);
\path[fill=white] (11,20) rectangle ++(1,1);
\path[fill=white] (12,20) rectangle ++(1,1);
\path[fill=white] (13,20) rectangle ++(1,1);
\path[fill=white] (14,20) rectangle ++(1,1);
\path[fill=white] (15,20) rectangle ++(1,1);
\path[fill=white] (16,20) rectangle ++(1,1);
\path[fill=white] (17,20) rectangle ++(1,1);
\path[fill=black] (18,20) rectangle ++(1,1);
\path[fill=black] (19,20) rectangle ++(1,1);
\path[fill=white] (20,20) rectangle ++(1,1);
\path[fill=white] (21,20) rectangle ++(1,1);
\path[fill=black] (22,20) rectangle ++(1,1);
\path[fill=black] (23,20) rectangle ++(1,1);
\path[fill=black] (24,20) rectangle ++(1,1);
\path[fill=white] (25,20) rectangle ++(1,1);
\path[fill=black] (26,20) rectangle ++(1,1);
\path[fill=black] (27,20) rectangle ++(1,1);
\path[fill=black] (28,20) rectangle ++(1,1);
\path[fill=black] (29,20) rectangle ++(1,1);
\path[fill=black] (30,20) rectangle ++(1,1);
\path[fill=black] (31,20) rectangle ++(1,1);
\path[fill=black] (32,20) rectangle ++(1,1);
\path[fill=black] (33,20) rectangle ++(1,1);
\path[fill=black] (34,20) rectangle ++(1,1);
\path[fill=white] (0,19) rectangle ++(1,1);
\path[fill=white] (1,19) rectangle ++(1,1);
\path[fill=white] (2,19) rectangle ++(1,1);
\path[fill=white] (3,19) rectangle ++(1,1);
\path[fill=white] (4,19) rectangle ++(1,1);
\path[fill=white] (5,19) rectangle ++(1,1);
\path[fill=white] (6,19) rectangle ++(1,1);
\path[fill=white] (7,19) rectangle ++(1,1);
\path[fill=white] (8,19) rectangle ++(1,1);
\path[fill=white] (9,19) rectangle ++(1,1);
\path[fill=white] (10,19) rectangle ++(1,1);
\path[fill=white] (11,19) rectangle ++(1,1);
\path[fill=white] (12,19) rectangle ++(1,1);
\path[fill=white] (13,19) rectangle ++(1,1);
\path[fill=white] (14,19) rectangle ++(1,1);
\path[fill=white] (15,19) rectangle ++(1,1);
\path[fill=white] (16,19) rectangle ++(1,1);
\path[fill=white] (17,19) rectangle ++(1,1);
\path[fill=white] (18,19) rectangle ++(1,1);
\path[fill=white] (19,19) rectangle ++(1,1);
\path[fill=black] (20,19) rectangle ++(1,1);
\path[fill=white] (21,19) rectangle ++(1,1);
\path[fill=white] (22,19) rectangle ++(1,1);
\path[fill=white] (23,19) rectangle ++(1,1);
\path[fill=white] (24,19) rectangle ++(1,1);
\path[fill=black] (25,19) rectangle ++(1,1);
\path[fill=white] (26,19) rectangle ++(1,1);
\path[fill=white] (27,19) rectangle ++(1,1);
\path[fill=white] (28,19) rectangle ++(1,1);
\path[fill=black] (29,19) rectangle ++(1,1);
\path[fill=white] (30,19) rectangle ++(1,1);
\path[fill=white] (31,19) rectangle ++(1,1);
\path[fill=black] (32,19) rectangle ++(1,1);
\path[fill=white] (33,19) rectangle ++(1,1);
\path[fill=black] (34,19) rectangle ++(1,1);
\path[fill=white] (0,18) rectangle ++(1,1);
\path[fill=white] (1,18) rectangle ++(1,1);
\path[fill=white] (2,18) rectangle ++(1,1);
\path[fill=white] (3,18) rectangle ++(1,1);
\path[fill=white] (4,18) rectangle ++(1,1);
\path[fill=white] (5,18) rectangle ++(1,1);
\path[fill=white] (6,18) rectangle ++(1,1);
\path[fill=white] (7,18) rectangle ++(1,1);
\path[fill=white] (8,18) rectangle ++(1,1);
\path[fill=white] (9,18) rectangle ++(1,1);
\path[fill=white] (10,18) rectangle ++(1,1);
\path[fill=white] (11,18) rectangle ++(1,1);
\path[fill=white] (12,18) rectangle ++(1,1);
\path[fill=white] (13,18) rectangle ++(1,1);
\path[fill=white] (14,18) rectangle ++(1,1);
\path[fill=white] (15,18) rectangle ++(1,1);
\path[fill=white] (16,18) rectangle ++(1,1);
\path[fill=white] (17,18) rectangle ++(1,1);
\path[fill=white] (18,18) rectangle ++(1,1);
\path[fill=white] (19,18) rectangle ++(1,1);
\path[fill=black] (20,18) rectangle ++(1,1);
\path[fill=black] (21,18) rectangle ++(1,1);
\path[fill=white] (22,18) rectangle ++(1,1);
\path[fill=white] (23,18) rectangle ++(1,1);
\path[fill=white] (24,18) rectangle ++(1,1);
\path[fill=black] (25,18) rectangle ++(1,1);
\path[fill=black] (26,18) rectangle ++(1,1);
\path[fill=white] (27,18) rectangle ++(1,1);
\path[fill=white] (28,18) rectangle ++(1,1);
\path[fill=black] (29,18) rectangle ++(1,1);
\path[fill=black] (30,18) rectangle ++(1,1);
\path[fill=white] (31,18) rectangle ++(1,1);
\path[fill=black] (32,18) rectangle ++(1,1);
\path[fill=black] (33,18) rectangle ++(1,1);
\path[fill=black] (34,18) rectangle ++(1,1);
\path[fill=white] (0,17) rectangle ++(1,1);
\path[fill=white] (1,17) rectangle ++(1,1);
\path[fill=white] (2,17) rectangle ++(1,1);
\path[fill=white] (3,17) rectangle ++(1,1);
\path[fill=white] (4,17) rectangle ++(1,1);
\path[fill=white] (5,17) rectangle ++(1,1);
\path[fill=white] (6,17) rectangle ++(1,1);
\path[fill=white] (7,17) rectangle ++(1,1);
\path[fill=white] (8,17) rectangle ++(1,1);
\path[fill=white] (9,17) rectangle ++(1,1);
\path[fill=white] (10,17) rectangle ++(1,1);
\path[fill=white] (11,17) rectangle ++(1,1);
\path[fill=white] (12,17) rectangle ++(1,1);
\path[fill=white] (13,17) rectangle ++(1,1);
\path[fill=white] (14,17) rectangle ++(1,1);
\path[fill=white] (15,17) rectangle ++(1,1);
\path[fill=white] (16,17) rectangle ++(1,1);
\path[fill=white] (17,17) rectangle ++(1,1);
\path[fill=white] (18,17) rectangle ++(1,1);
\path[fill=white] (19,17) rectangle ++(1,1);
\path[fill=white] (20,17) rectangle ++(1,1);
\path[fill=black] (21,17) rectangle ++(1,1);
\path[fill=black] (22,17) rectangle ++(1,1);
\path[fill=white] (23,17) rectangle ++(1,1);
\path[fill=white] (24,17) rectangle ++(1,1);
\path[fill=black] (25,17) rectangle ++(1,1);
\path[fill=black] (26,17) rectangle ++(1,1);
\path[fill=black] (27,17) rectangle ++(1,1);
\path[fill=white] (28,17) rectangle ++(1,1);
\path[fill=black] (29,17) rectangle ++(1,1);
\path[fill=black] (30,17) rectangle ++(1,1);
\path[fill=black] (31,17) rectangle ++(1,1);
\path[fill=black] (32,17) rectangle ++(1,1);
\path[fill=black] (33,17) rectangle ++(1,1);
\path[fill=black] (34,17) rectangle ++(1,1);
\path[fill=white] (0,16) rectangle ++(1,1);
\path[fill=white] (1,16) rectangle ++(1,1);
\path[fill=white] (2,16) rectangle ++(1,1);
\path[fill=white] (3,16) rectangle ++(1,1);
\path[fill=white] (4,16) rectangle ++(1,1);
\path[fill=white] (5,16) rectangle ++(1,1);
\path[fill=white] (6,16) rectangle ++(1,1);
\path[fill=white] (7,16) rectangle ++(1,1);
\path[fill=white] (8,16) rectangle ++(1,1);
\path[fill=white] (9,16) rectangle ++(1,1);
\path[fill=white] (10,16) rectangle ++(1,1);
\path[fill=white] (11,16) rectangle ++(1,1);
\path[fill=white] (12,16) rectangle ++(1,1);
\path[fill=white] (13,16) rectangle ++(1,1);
\path[fill=white] (14,16) rectangle ++(1,1);
\path[fill=white] (15,16) rectangle ++(1,1);
\path[fill=white] (16,16) rectangle ++(1,1);
\path[fill=white] (17,16) rectangle ++(1,1);
\path[fill=white] (18,16) rectangle ++(1,1);
\path[fill=white] (19,16) rectangle ++(1,1);
\path[fill=white] (20,16) rectangle ++(1,1);
\path[fill=white] (21,16) rectangle ++(1,1);
\path[fill=black] (22,16) rectangle ++(1,1);
\path[fill=black] (23,16) rectangle ++(1,1);
\path[fill=white] (24,16) rectangle ++(1,1);
\path[fill=white] (25,16) rectangle ++(1,1);
\path[fill=black] (26,16) rectangle ++(1,1);
\path[fill=black] (27,16) rectangle ++(1,1);
\path[fill=black] (28,16) rectangle ++(1,1);
\path[fill=black] (29,16) rectangle ++(1,1);
\path[fill=black] (30,16) rectangle ++(1,1);
\path[fill=black] (31,16) rectangle ++(1,1);
\path[fill=black] (32,16) rectangle ++(1,1);
\path[fill=black] (33,16) rectangle ++(1,1);
\path[fill=black] (34,16) rectangle ++(1,1);
\path[fill=white] (0,15) rectangle ++(1,1);
\path[fill=white] (1,15) rectangle ++(1,1);
\path[fill=white] (2,15) rectangle ++(1,1);
\path[fill=white] (3,15) rectangle ++(1,1);
\path[fill=white] (4,15) rectangle ++(1,1);
\path[fill=white] (5,15) rectangle ++(1,1);
\path[fill=white] (6,15) rectangle ++(1,1);
\path[fill=white] (7,15) rectangle ++(1,1);
\path[fill=white] (8,15) rectangle ++(1,1);
\path[fill=white] (9,15) rectangle ++(1,1);
\path[fill=white] (10,15) rectangle ++(1,1);
\path[fill=white] (11,15) rectangle ++(1,1);
\path[fill=white] (12,15) rectangle ++(1,1);
\path[fill=white] (13,15) rectangle ++(1,1);
\path[fill=white] (14,15) rectangle ++(1,1);
\path[fill=white] (15,15) rectangle ++(1,1);
\path[fill=white] (16,15) rectangle ++(1,1);
\path[fill=white] (17,15) rectangle ++(1,1);
\path[fill=white] (18,15) rectangle ++(1,1);
\path[fill=white] (19,15) rectangle ++(1,1);
\path[fill=white] (20,15) rectangle ++(1,1);
\path[fill=white] (21,15) rectangle ++(1,1);
\path[fill=white] (22,15) rectangle ++(1,1);
\path[fill=black] (23,15) rectangle ++(1,1);
\path[fill=black] (24,15) rectangle ++(1,1);
\path[fill=white] (25,15) rectangle ++(1,1);
\path[fill=white] (26,15) rectangle ++(1,1);
\path[fill=black] (27,15) rectangle ++(1,1);
\path[fill=black] (28,15) rectangle ++(1,1);
\path[fill=white] (29,15) rectangle ++(1,1);
\path[fill=black] (30,15) rectangle ++(1,1);
\path[fill=black] (31,15) rectangle ++(1,1);
\path[fill=black] (32,15) rectangle ++(1,1);
\path[fill=black] (33,15) rectangle ++(1,1);
\path[fill=black] (34,15) rectangle ++(1,1);
\path[fill=white] (0,14) rectangle ++(1,1);
\path[fill=white] (1,14) rectangle ++(1,1);
\path[fill=white] (2,14) rectangle ++(1,1);
\path[fill=white] (3,14) rectangle ++(1,1);
\path[fill=white] (4,14) rectangle ++(1,1);
\path[fill=white] (5,14) rectangle ++(1,1);
\path[fill=white] (6,14) rectangle ++(1,1);
\path[fill=white] (7,14) rectangle ++(1,1);
\path[fill=white] (8,14) rectangle ++(1,1);
\path[fill=white] (9,14) rectangle ++(1,1);
\path[fill=white] (10,14) rectangle ++(1,1);
\path[fill=white] (11,14) rectangle ++(1,1);
\path[fill=white] (12,14) rectangle ++(1,1);
\path[fill=white] (13,14) rectangle ++(1,1);
\path[fill=white] (14,14) rectangle ++(1,1);
\path[fill=white] (15,14) rectangle ++(1,1);
\path[fill=white] (16,14) rectangle ++(1,1);
\path[fill=white] (17,14) rectangle ++(1,1);
\path[fill=white] (18,14) rectangle ++(1,1);
\path[fill=white] (19,14) rectangle ++(1,1);
\path[fill=white] (20,14) rectangle ++(1,1);
\path[fill=white] (21,14) rectangle ++(1,1);
\path[fill=white] (22,14) rectangle ++(1,1);
\path[fill=white] (23,14) rectangle ++(1,1);
\path[fill=white] (24,14) rectangle ++(1,1);
\path[fill=black] (25,14) rectangle ++(1,1);
\path[fill=white] (26,14) rectangle ++(1,1);
\path[fill=white] (27,14) rectangle ++(1,1);
\path[fill=white] (28,14) rectangle ++(1,1);
\path[fill=black] (29,14) rectangle ++(1,1);
\path[fill=white] (30,14) rectangle ++(1,1);
\path[fill=white] (31,14) rectangle ++(1,1);
\path[fill=black] (32,14) rectangle ++(1,1);
\path[fill=white] (33,14) rectangle ++(1,1);
\path[fill=black] (34,14) rectangle ++(1,1);
\path[fill=white] (0,13) rectangle ++(1,1);
\path[fill=white] (1,13) rectangle ++(1,1);
\path[fill=white] (2,13) rectangle ++(1,1);
\path[fill=white] (3,13) rectangle ++(1,1);
\path[fill=white] (4,13) rectangle ++(1,1);
\path[fill=white] (5,13) rectangle ++(1,1);
\path[fill=white] (6,13) rectangle ++(1,1);
\path[fill=white] (7,13) rectangle ++(1,1);
\path[fill=white] (8,13) rectangle ++(1,1);
\path[fill=white] (9,13) rectangle ++(1,1);
\path[fill=white] (10,13) rectangle ++(1,1);
\path[fill=white] (11,13) rectangle ++(1,1);
\path[fill=white] (12,13) rectangle ++(1,1);
\path[fill=white] (13,13) rectangle ++(1,1);
\path[fill=white] (14,13) rectangle ++(1,1);
\path[fill=white] (15,13) rectangle ++(1,1);
\path[fill=white] (16,13) rectangle ++(1,1);
\path[fill=white] (17,13) rectangle ++(1,1);
\path[fill=white] (18,13) rectangle ++(1,1);
\path[fill=white] (19,13) rectangle ++(1,1);
\path[fill=white] (20,13) rectangle ++(1,1);
\path[fill=white] (21,13) rectangle ++(1,1);
\path[fill=white] (22,13) rectangle ++(1,1);
\path[fill=white] (23,13) rectangle ++(1,1);
\path[fill=white] (24,13) rectangle ++(1,1);
\path[fill=black] (25,13) rectangle ++(1,1);
\path[fill=black] (26,13) rectangle ++(1,1);
\path[fill=white] (27,13) rectangle ++(1,1);
\path[fill=white] (28,13) rectangle ++(1,1);
\path[fill=black] (29,13) rectangle ++(1,1);
\path[fill=black] (30,13) rectangle ++(1,1);
\path[fill=white] (31,13) rectangle ++(1,1);
\path[fill=black] (32,13) rectangle ++(1,1);
\path[fill=black] (33,13) rectangle ++(1,1);
\path[fill=black] (34,13) rectangle ++(1,1);
\path[fill=white] (0,12) rectangle ++(1,1);
\path[fill=white] (1,12) rectangle ++(1,1);
\path[fill=white] (2,12) rectangle ++(1,1);
\path[fill=white] (3,12) rectangle ++(1,1);
\path[fill=white] (4,12) rectangle ++(1,1);
\path[fill=white] (5,12) rectangle ++(1,1);
\path[fill=white] (6,12) rectangle ++(1,1);
\path[fill=white] (7,12) rectangle ++(1,1);
\path[fill=white] (8,12) rectangle ++(1,1);
\path[fill=white] (9,12) rectangle ++(1,1);
\path[fill=white] (10,12) rectangle ++(1,1);
\path[fill=white] (11,12) rectangle ++(1,1);
\path[fill=white] (12,12) rectangle ++(1,1);
\path[fill=white] (13,12) rectangle ++(1,1);
\path[fill=white] (14,12) rectangle ++(1,1);
\path[fill=white] (15,12) rectangle ++(1,1);
\path[fill=white] (16,12) rectangle ++(1,1);
\path[fill=white] (17,12) rectangle ++(1,1);
\path[fill=white] (18,12) rectangle ++(1,1);
\path[fill=white] (19,12) rectangle ++(1,1);
\path[fill=white] (20,12) rectangle ++(1,1);
\path[fill=white] (21,12) rectangle ++(1,1);
\path[fill=white] (22,12) rectangle ++(1,1);
\path[fill=white] (23,12) rectangle ++(1,1);
\path[fill=white] (24,12) rectangle ++(1,1);
\path[fill=white] (25,12) rectangle ++(1,1);
\path[fill=black] (26,12) rectangle ++(1,1);
\path[fill=black] (27,12) rectangle ++(1,1);
\path[fill=white] (28,12) rectangle ++(1,1);
\path[fill=black] (29,12) rectangle ++(1,1);
\path[fill=black] (30,12) rectangle ++(1,1);
\path[fill=black] (31,12) rectangle ++(1,1);
\path[fill=black] (32,12) rectangle ++(1,1);
\path[fill=black] (33,12) rectangle ++(1,1);
\path[fill=black] (34,12) rectangle ++(1,1);
\path[fill=white] (0,11) rectangle ++(1,1);
\path[fill=white] (1,11) rectangle ++(1,1);
\path[fill=white] (2,11) rectangle ++(1,1);
\path[fill=white] (3,11) rectangle ++(1,1);
\path[fill=white] (4,11) rectangle ++(1,1);
\path[fill=white] (5,11) rectangle ++(1,1);
\path[fill=white] (6,11) rectangle ++(1,1);
\path[fill=white] (7,11) rectangle ++(1,1);
\path[fill=white] (8,11) rectangle ++(1,1);
\path[fill=white] (9,11) rectangle ++(1,1);
\path[fill=white] (10,11) rectangle ++(1,1);
\path[fill=white] (11,11) rectangle ++(1,1);
\path[fill=white] (12,11) rectangle ++(1,1);
\path[fill=white] (13,11) rectangle ++(1,1);
\path[fill=white] (14,11) rectangle ++(1,1);
\path[fill=white] (15,11) rectangle ++(1,1);
\path[fill=white] (16,11) rectangle ++(1,1);
\path[fill=white] (17,11) rectangle ++(1,1);
\path[fill=white] (18,11) rectangle ++(1,1);
\path[fill=white] (19,11) rectangle ++(1,1);
\path[fill=white] (20,11) rectangle ++(1,1);
\path[fill=white] (21,11) rectangle ++(1,1);
\path[fill=white] (22,11) rectangle ++(1,1);
\path[fill=white] (23,11) rectangle ++(1,1);
\path[fill=white] (24,11) rectangle ++(1,1);
\path[fill=white] (25,11) rectangle ++(1,1);
\path[fill=white] (26,11) rectangle ++(1,1);
\path[fill=black] (27,11) rectangle ++(1,1);
\path[fill=black] (28,11) rectangle ++(1,1);
\path[fill=white] (29,11) rectangle ++(1,1);
\path[fill=black] (30,11) rectangle ++(1,1);
\path[fill=black] (31,11) rectangle ++(1,1);
\path[fill=black] (32,11) rectangle ++(1,1);
\path[fill=black] (33,11) rectangle ++(1,1);
\path[fill=black] (34,11) rectangle ++(1,1);
\path[fill=white] (0,10) rectangle ++(1,1);
\path[fill=white] (1,10) rectangle ++(1,1);
\path[fill=white] (2,10) rectangle ++(1,1);
\path[fill=white] (3,10) rectangle ++(1,1);
\path[fill=white] (4,10) rectangle ++(1,1);
\path[fill=white] (5,10) rectangle ++(1,1);
\path[fill=white] (6,10) rectangle ++(1,1);
\path[fill=white] (7,10) rectangle ++(1,1);
\path[fill=white] (8,10) rectangle ++(1,1);
\path[fill=white] (9,10) rectangle ++(1,1);
\path[fill=white] (10,10) rectangle ++(1,1);
\path[fill=white] (11,10) rectangle ++(1,1);
\path[fill=white] (12,10) rectangle ++(1,1);
\path[fill=white] (13,10) rectangle ++(1,1);
\path[fill=white] (14,10) rectangle ++(1,1);
\path[fill=white] (15,10) rectangle ++(1,1);
\path[fill=white] (16,10) rectangle ++(1,1);
\path[fill=white] (17,10) rectangle ++(1,1);
\path[fill=white] (18,10) rectangle ++(1,1);
\path[fill=white] (19,10) rectangle ++(1,1);
\path[fill=white] (20,10) rectangle ++(1,1);
\path[fill=white] (21,10) rectangle ++(1,1);
\path[fill=white] (22,10) rectangle ++(1,1);
\path[fill=white] (23,10) rectangle ++(1,1);
\path[fill=white] (24,10) rectangle ++(1,1);
\path[fill=white] (25,10) rectangle ++(1,1);
\path[fill=white] (26,10) rectangle ++(1,1);
\path[fill=white] (27,10) rectangle ++(1,1);
\path[fill=black] (28,10) rectangle ++(1,1);
\path[fill=white] (29,10) rectangle ++(1,1);
\path[fill=white] (30,10) rectangle ++(1,1);
\path[fill=black] (31,10) rectangle ++(1,1);
\path[fill=white] (32,10) rectangle ++(1,1);
\path[fill=black] (33,10) rectangle ++(1,1);
\path[fill=black] (34,10) rectangle ++(1,1);
\path[fill=white] (0,9) rectangle ++(1,1);
\path[fill=white] (1,9) rectangle ++(1,1);
\path[fill=white] (2,9) rectangle ++(1,1);
\path[fill=white] (3,9) rectangle ++(1,1);
\path[fill=white] (4,9) rectangle ++(1,1);
\path[fill=white] (5,9) rectangle ++(1,1);
\path[fill=white] (6,9) rectangle ++(1,1);
\path[fill=white] (7,9) rectangle ++(1,1);
\path[fill=white] (8,9) rectangle ++(1,1);
\path[fill=white] (9,9) rectangle ++(1,1);
\path[fill=white] (10,9) rectangle ++(1,1);
\path[fill=white] (11,9) rectangle ++(1,1);
\path[fill=white] (12,9) rectangle ++(1,1);
\path[fill=white] (13,9) rectangle ++(1,1);
\path[fill=white] (14,9) rectangle ++(1,1);
\path[fill=white] (15,9) rectangle ++(1,1);
\path[fill=white] (16,9) rectangle ++(1,1);
\path[fill=white] (17,9) rectangle ++(1,1);
\path[fill=white] (18,9) rectangle ++(1,1);
\path[fill=white] (19,9) rectangle ++(1,1);
\path[fill=white] (20,9) rectangle ++(1,1);
\path[fill=white] (21,9) rectangle ++(1,1);
\path[fill=white] (22,9) rectangle ++(1,1);
\path[fill=white] (23,9) rectangle ++(1,1);
\path[fill=white] (24,9) rectangle ++(1,1);
\path[fill=white] (25,9) rectangle ++(1,1);
\path[fill=white] (26,9) rectangle ++(1,1);
\path[fill=white] (27,9) rectangle ++(1,1);
\path[fill=white] (28,9) rectangle ++(1,1);
\path[fill=black] (29,9) rectangle ++(1,1);
\path[fill=white] (30,9) rectangle ++(1,1);
\path[fill=white] (31,9) rectangle ++(1,1);
\path[fill=black] (32,9) rectangle ++(1,1);
\path[fill=white] (33,9) rectangle ++(1,1);
\path[fill=black] (34,9) rectangle ++(1,1);
\path[fill=white] (0,8) rectangle ++(1,1);
\path[fill=white] (1,8) rectangle ++(1,1);
\path[fill=white] (2,8) rectangle ++(1,1);
\path[fill=white] (3,8) rectangle ++(1,1);
\path[fill=white] (4,8) rectangle ++(1,1);
\path[fill=white] (5,8) rectangle ++(1,1);
\path[fill=white] (6,8) rectangle ++(1,1);
\path[fill=white] (7,8) rectangle ++(1,1);
\path[fill=white] (8,8) rectangle ++(1,1);
\path[fill=white] (9,8) rectangle ++(1,1);
\path[fill=white] (10,8) rectangle ++(1,1);
\path[fill=white] (11,8) rectangle ++(1,1);
\path[fill=white] (12,8) rectangle ++(1,1);
\path[fill=white] (13,8) rectangle ++(1,1);
\path[fill=white] (14,8) rectangle ++(1,1);
\path[fill=white] (15,8) rectangle ++(1,1);
\path[fill=white] (16,8) rectangle ++(1,1);
\path[fill=white] (17,8) rectangle ++(1,1);
\path[fill=white] (18,8) rectangle ++(1,1);
\path[fill=white] (19,8) rectangle ++(1,1);
\path[fill=white] (20,8) rectangle ++(1,1);
\path[fill=white] (21,8) rectangle ++(1,1);
\path[fill=white] (22,8) rectangle ++(1,1);
\path[fill=white] (23,8) rectangle ++(1,1);
\path[fill=white] (24,8) rectangle ++(1,1);
\path[fill=white] (25,8) rectangle ++(1,1);
\path[fill=white] (26,8) rectangle ++(1,1);
\path[fill=white] (27,8) rectangle ++(1,1);
\path[fill=white] (28,8) rectangle ++(1,1);
\path[fill=black] (29,8) rectangle ++(1,1);
\path[fill=black] (30,8) rectangle ++(1,1);
\path[fill=white] (31,8) rectangle ++(1,1);
\path[fill=black] (32,8) rectangle ++(1,1);
\path[fill=black] (33,8) rectangle ++(1,1);
\path[fill=black] (34,8) rectangle ++(1,1);
\path[fill=white] (0,7) rectangle ++(1,1);
\path[fill=white] (1,7) rectangle ++(1,1);
\path[fill=white] (2,7) rectangle ++(1,1);
\path[fill=white] (3,7) rectangle ++(1,1);
\path[fill=white] (4,7) rectangle ++(1,1);
\path[fill=white] (5,7) rectangle ++(1,1);
\path[fill=white] (6,7) rectangle ++(1,1);
\path[fill=white] (7,7) rectangle ++(1,1);
\path[fill=white] (8,7) rectangle ++(1,1);
\path[fill=white] (9,7) rectangle ++(1,1);
\path[fill=white] (10,7) rectangle ++(1,1);
\path[fill=white] (11,7) rectangle ++(1,1);
\path[fill=white] (12,7) rectangle ++(1,1);
\path[fill=white] (13,7) rectangle ++(1,1);
\path[fill=white] (14,7) rectangle ++(1,1);
\path[fill=white] (15,7) rectangle ++(1,1);
\path[fill=white] (16,7) rectangle ++(1,1);
\path[fill=white] (17,7) rectangle ++(1,1);
\path[fill=white] (18,7) rectangle ++(1,1);
\path[fill=white] (19,7) rectangle ++(1,1);
\path[fill=white] (20,7) rectangle ++(1,1);
\path[fill=white] (21,7) rectangle ++(1,1);
\path[fill=white] (22,7) rectangle ++(1,1);
\path[fill=white] (23,7) rectangle ++(1,1);
\path[fill=white] (24,7) rectangle ++(1,1);
\path[fill=white] (25,7) rectangle ++(1,1);
\path[fill=white] (26,7) rectangle ++(1,1);
\path[fill=white] (27,7) rectangle ++(1,1);
\path[fill=white] (28,7) rectangle ++(1,1);
\path[fill=white] (29,7) rectangle ++(1,1);
\path[fill=black] (30,7) rectangle ++(1,1);
\path[fill=black] (31,7) rectangle ++(1,1);
\path[fill=black] (32,7) rectangle ++(1,1);
\path[fill=black] (33,7) rectangle ++(1,1);
\path[fill=black] (34,7) rectangle ++(1,1);
\path[fill=white] (0,6) rectangle ++(1,1);
\path[fill=white] (1,6) rectangle ++(1,1);
\path[fill=white] (2,6) rectangle ++(1,1);
\path[fill=white] (3,6) rectangle ++(1,1);
\path[fill=white] (4,6) rectangle ++(1,1);
\path[fill=white] (5,6) rectangle ++(1,1);
\path[fill=white] (6,6) rectangle ++(1,1);
\path[fill=white] (7,6) rectangle ++(1,1);
\path[fill=white] (8,6) rectangle ++(1,1);
\path[fill=white] (9,6) rectangle ++(1,1);
\path[fill=white] (10,6) rectangle ++(1,1);
\path[fill=white] (11,6) rectangle ++(1,1);
\path[fill=white] (12,6) rectangle ++(1,1);
\path[fill=white] (13,6) rectangle ++(1,1);
\path[fill=white] (14,6) rectangle ++(1,1);
\path[fill=white] (15,6) rectangle ++(1,1);
\path[fill=white] (16,6) rectangle ++(1,1);
\path[fill=white] (17,6) rectangle ++(1,1);
\path[fill=white] (18,6) rectangle ++(1,1);
\path[fill=white] (19,6) rectangle ++(1,1);
\path[fill=white] (20,6) rectangle ++(1,1);
\path[fill=white] (21,6) rectangle ++(1,1);
\path[fill=white] (22,6) rectangle ++(1,1);
\path[fill=white] (23,6) rectangle ++(1,1);
\path[fill=white] (24,6) rectangle ++(1,1);
\path[fill=white] (25,6) rectangle ++(1,1);
\path[fill=white] (26,6) rectangle ++(1,1);
\path[fill=white] (27,6) rectangle ++(1,1);
\path[fill=white] (28,6) rectangle ++(1,1);
\path[fill=white] (29,6) rectangle ++(1,1);
\path[fill=white] (30,6) rectangle ++(1,1);
\path[fill=black] (31,6) rectangle ++(1,1);
\path[fill=white] (32,6) rectangle ++(1,1);
\path[fill=black] (33,6) rectangle ++(1,1);
\path[fill=black] (34,6) rectangle ++(1,1);
\path[fill=white] (0,5) rectangle ++(1,1);
\path[fill=white] (1,5) rectangle ++(1,1);
\path[fill=white] (2,5) rectangle ++(1,1);
\path[fill=white] (3,5) rectangle ++(1,1);
\path[fill=white] (4,5) rectangle ++(1,1);
\path[fill=white] (5,5) rectangle ++(1,1);
\path[fill=white] (6,5) rectangle ++(1,1);
\path[fill=white] (7,5) rectangle ++(1,1);
\path[fill=white] (8,5) rectangle ++(1,1);
\path[fill=white] (9,5) rectangle ++(1,1);
\path[fill=white] (10,5) rectangle ++(1,1);
\path[fill=white] (11,5) rectangle ++(1,1);
\path[fill=white] (12,5) rectangle ++(1,1);
\path[fill=white] (13,5) rectangle ++(1,1);
\path[fill=white] (14,5) rectangle ++(1,1);
\path[fill=white] (15,5) rectangle ++(1,1);
\path[fill=white] (16,5) rectangle ++(1,1);
\path[fill=white] (17,5) rectangle ++(1,1);
\path[fill=white] (18,5) rectangle ++(1,1);
\path[fill=white] (19,5) rectangle ++(1,1);
\path[fill=white] (20,5) rectangle ++(1,1);
\path[fill=white] (21,5) rectangle ++(1,1);
\path[fill=white] (22,5) rectangle ++(1,1);
\path[fill=white] (23,5) rectangle ++(1,1);
\path[fill=white] (24,5) rectangle ++(1,1);
\path[fill=white] (25,5) rectangle ++(1,1);
\path[fill=white] (26,5) rectangle ++(1,1);
\path[fill=white] (27,5) rectangle ++(1,1);
\path[fill=white] (28,5) rectangle ++(1,1);
\path[fill=white] (29,5) rectangle ++(1,1);
\path[fill=white] (30,5) rectangle ++(1,1);
\path[fill=white] (31,5) rectangle ++(1,1);
\path[fill=black] (32,5) rectangle ++(1,1);
\path[fill=white] (33,5) rectangle ++(1,1);
\path[fill=black] (34,5) rectangle ++(1,1);
\path[fill=white] (0,4) rectangle ++(1,1);
\path[fill=white] (1,4) rectangle ++(1,1);
\path[fill=white] (2,4) rectangle ++(1,1);
\path[fill=white] (3,4) rectangle ++(1,1);
\path[fill=white] (4,4) rectangle ++(1,1);
\path[fill=white] (5,4) rectangle ++(1,1);
\path[fill=white] (6,4) rectangle ++(1,1);
\path[fill=white] (7,4) rectangle ++(1,1);
\path[fill=white] (8,4) rectangle ++(1,1);
\path[fill=white] (9,4) rectangle ++(1,1);
\path[fill=white] (10,4) rectangle ++(1,1);
\path[fill=white] (11,4) rectangle ++(1,1);
\path[fill=white] (12,4) rectangle ++(1,1);
\path[fill=white] (13,4) rectangle ++(1,1);
\path[fill=white] (14,4) rectangle ++(1,1);
\path[fill=white] (15,4) rectangle ++(1,1);
\path[fill=white] (16,4) rectangle ++(1,1);
\path[fill=white] (17,4) rectangle ++(1,1);
\path[fill=white] (18,4) rectangle ++(1,1);
\path[fill=white] (19,4) rectangle ++(1,1);
\path[fill=white] (20,4) rectangle ++(1,1);
\path[fill=white] (21,4) rectangle ++(1,1);
\path[fill=white] (22,4) rectangle ++(1,1);
\path[fill=white] (23,4) rectangle ++(1,1);
\path[fill=white] (24,4) rectangle ++(1,1);
\path[fill=white] (25,4) rectangle ++(1,1);
\path[fill=white] (26,4) rectangle ++(1,1);
\path[fill=white] (27,4) rectangle ++(1,1);
\path[fill=white] (28,4) rectangle ++(1,1);
\path[fill=white] (29,4) rectangle ++(1,1);
\path[fill=white] (30,4) rectangle ++(1,1);
\path[fill=white] (31,4) rectangle ++(1,1);
\path[fill=black] (32,4) rectangle ++(1,1);
\path[fill=black] (33,4) rectangle ++(1,1);
\path[fill=black] (34,4) rectangle ++(1,1);
\path[fill=white] (0,3) rectangle ++(1,1);
\path[fill=white] (1,3) rectangle ++(1,1);
\path[fill=white] (2,3) rectangle ++(1,1);
\path[fill=white] (3,3) rectangle ++(1,1);
\path[fill=white] (4,3) rectangle ++(1,1);
\path[fill=white] (5,3) rectangle ++(1,1);
\path[fill=white] (6,3) rectangle ++(1,1);
\path[fill=white] (7,3) rectangle ++(1,1);
\path[fill=white] (8,3) rectangle ++(1,1);
\path[fill=white] (9,3) rectangle ++(1,1);
\path[fill=white] (10,3) rectangle ++(1,1);
\path[fill=white] (11,3) rectangle ++(1,1);
\path[fill=white] (12,3) rectangle ++(1,1);
\path[fill=white] (13,3) rectangle ++(1,1);
\path[fill=white] (14,3) rectangle ++(1,1);
\path[fill=white] (15,3) rectangle ++(1,1);
\path[fill=white] (16,3) rectangle ++(1,1);
\path[fill=white] (17,3) rectangle ++(1,1);
\path[fill=white] (18,3) rectangle ++(1,1);
\path[fill=white] (19,3) rectangle ++(1,1);
\path[fill=white] (20,3) rectangle ++(1,1);
\path[fill=white] (21,3) rectangle ++(1,1);
\path[fill=white] (22,3) rectangle ++(1,1);
\path[fill=white] (23,3) rectangle ++(1,1);
\path[fill=white] (24,3) rectangle ++(1,1);
\path[fill=white] (25,3) rectangle ++(1,1);
\path[fill=white] (26,3) rectangle ++(1,1);
\path[fill=white] (27,3) rectangle ++(1,1);
\path[fill=white] (28,3) rectangle ++(1,1);
\path[fill=white] (29,3) rectangle ++(1,1);
\path[fill=white] (30,3) rectangle ++(1,1);
\path[fill=white] (31,3) rectangle ++(1,1);
\path[fill=white] (32,3) rectangle ++(1,1);
\path[fill=black] (33,3) rectangle ++(1,1);
\path[fill=black] (34,3) rectangle ++(1,1);
\path[fill=white] (0,2) rectangle ++(1,1);
\path[fill=white] (1,2) rectangle ++(1,1);
\path[fill=white] (2,2) rectangle ++(1,1);
\path[fill=white] (3,2) rectangle ++(1,1);
\path[fill=white] (4,2) rectangle ++(1,1);
\path[fill=white] (5,2) rectangle ++(1,1);
\path[fill=white] (6,2) rectangle ++(1,1);
\path[fill=white] (7,2) rectangle ++(1,1);
\path[fill=white] (8,2) rectangle ++(1,1);
\path[fill=white] (9,2) rectangle ++(1,1);
\path[fill=white] (10,2) rectangle ++(1,1);
\path[fill=white] (11,2) rectangle ++(1,1);
\path[fill=white] (12,2) rectangle ++(1,1);
\path[fill=white] (13,2) rectangle ++(1,1);
\path[fill=white] (14,2) rectangle ++(1,1);
\path[fill=white] (15,2) rectangle ++(1,1);
\path[fill=white] (16,2) rectangle ++(1,1);
\path[fill=white] (17,2) rectangle ++(1,1);
\path[fill=white] (18,2) rectangle ++(1,1);
\path[fill=white] (19,2) rectangle ++(1,1);
\path[fill=white] (20,2) rectangle ++(1,1);
\path[fill=white] (21,2) rectangle ++(1,1);
\path[fill=white] (22,2) rectangle ++(1,1);
\path[fill=white] (23,2) rectangle ++(1,1);
\path[fill=white] (24,2) rectangle ++(1,1);
\path[fill=white] (25,2) rectangle ++(1,1);
\path[fill=white] (26,2) rectangle ++(1,1);
\path[fill=white] (27,2) rectangle ++(1,1);
\path[fill=white] (28,2) rectangle ++(1,1);
\path[fill=white] (29,2) rectangle ++(1,1);
\path[fill=white] (30,2) rectangle ++(1,1);
\path[fill=white] (31,2) rectangle ++(1,1);
\path[fill=white] (32,2) rectangle ++(1,1);
\path[fill=white] (33,2) rectangle ++(1,1);
\path[fill=black] (34,2) rectangle ++(1,1);
\path[fill=white] (0,1) rectangle ++(1,1);
\path[fill=white] (1,1) rectangle ++(1,1);
\path[fill=white] (2,1) rectangle ++(1,1);
\path[fill=white] (3,1) rectangle ++(1,1);
\path[fill=white] (4,1) rectangle ++(1,1);
\path[fill=white] (5,1) rectangle ++(1,1);
\path[fill=white] (6,1) rectangle ++(1,1);
\path[fill=white] (7,1) rectangle ++(1,1);
\path[fill=white] (8,1) rectangle ++(1,1);
\path[fill=white] (9,1) rectangle ++(1,1);
\path[fill=white] (10,1) rectangle ++(1,1);
\path[fill=white] (11,1) rectangle ++(1,1);
\path[fill=white] (12,1) rectangle ++(1,1);
\path[fill=white] (13,1) rectangle ++(1,1);
\path[fill=white] (14,1) rectangle ++(1,1);
\path[fill=white] (15,1) rectangle ++(1,1);
\path[fill=white] (16,1) rectangle ++(1,1);
\path[fill=white] (17,1) rectangle ++(1,1);
\path[fill=white] (18,1) rectangle ++(1,1);
\path[fill=white] (19,1) rectangle ++(1,1);
\path[fill=white] (20,1) rectangle ++(1,1);
\path[fill=white] (21,1) rectangle ++(1,1);
\path[fill=white] (22,1) rectangle ++(1,1);
\path[fill=white] (23,1) rectangle ++(1,1);
\path[fill=white] (24,1) rectangle ++(1,1);
\path[fill=white] (25,1) rectangle ++(1,1);
\path[fill=white] (26,1) rectangle ++(1,1);
\path[fill=white] (27,1) rectangle ++(1,1);
\path[fill=white] (28,1) rectangle ++(1,1);
\path[fill=white] (29,1) rectangle ++(1,1);
\path[fill=white] (30,1) rectangle ++(1,1);
\path[fill=white] (31,1) rectangle ++(1,1);
\path[fill=white] (32,1) rectangle ++(1,1);
\path[fill=white] (33,1) rectangle ++(1,1);
\path[fill=black] (34,1) rectangle ++(1,1);
\path[fill=white] (0,0) rectangle ++(1,1);
\path[fill=white] (1,0) rectangle ++(1,1);
\path[fill=white] (2,0) rectangle ++(1,1);
\path[fill=white] (3,0) rectangle ++(1,1);
\path[fill=white] (4,0) rectangle ++(1,1);
\path[fill=white] (5,0) rectangle ++(1,1);
\path[fill=white] (6,0) rectangle ++(1,1);
\path[fill=white] (7,0) rectangle ++(1,1);
\path[fill=white] (8,0) rectangle ++(1,1);
\path[fill=white] (9,0) rectangle ++(1,1);
\path[fill=white] (10,0) rectangle ++(1,1);
\path[fill=white] (11,0) rectangle ++(1,1);
\path[fill=white] (12,0) rectangle ++(1,1);
\path[fill=white] (13,0) rectangle ++(1,1);
\path[fill=white] (14,0) rectangle ++(1,1);
\path[fill=white] (15,0) rectangle ++(1,1);
\path[fill=white] (16,0) rectangle ++(1,1);
\path[fill=white] (17,0) rectangle ++(1,1);
\path[fill=white] (18,0) rectangle ++(1,1);
\path[fill=white] (19,0) rectangle ++(1,1);
\path[fill=white] (20,0) rectangle ++(1,1);
\path[fill=white] (21,0) rectangle ++(1,1);
\path[fill=white] (22,0) rectangle ++(1,1);
\path[fill=white] (23,0) rectangle ++(1,1);
\path[fill=white] (24,0) rectangle ++(1,1);
\path[fill=white] (25,0) rectangle ++(1,1);
\path[fill=white] (26,0) rectangle ++(1,1);
\path[fill=white] (27,0) rectangle ++(1,1);
\path[fill=white] (28,0) rectangle ++(1,1);
\path[fill=white] (29,0) rectangle ++(1,1);
\path[fill=white] (30,0) rectangle ++(1,1);
\path[fill=white] (31,0) rectangle ++(1,1);
\path[fill=white] (32,0) rectangle ++(1,1);
\path[fill=white] (33,0) rectangle ++(1,1);
\path[fill=white] (34,0) rectangle ++(1,1);
\end{tikzpicture}}}
\end{center}
\caption{Lev(|\sigma|=6, \Delta=4) automaton, adjacency and reachability matrix.}
\end{figure}

\section{Levenshtein automata minimality}

It is reasonable to ask whether the Levenshtein automaton defined in \S~\ref{sec:repair_ex} is minimal, in the sense of whether there exists an automaton with fewer states than $A$ yet still generates $\mathcal{L}(G_\cap)$ when intersected with $\mathcal{L}(G)$. In other words, given $G$ and $\err\sigma$, is there an $A'$ such that $|Q_{A'}| < |Q_{A}|$ yet $\mathcal{L}(G) \cap \mathcal{L}(A') = \mathcal{L}(G) \cap \mathcal{L}(A)$ still holds? In fact, there is a trivial example:

\begin{theorem}
  Let $Q_{A'}$ be defined as $Q_A \setminus \{q_{n, 0}\}$.
\end{theorem}

Since $q_{n, 0}$ accepts the original string $\err\sigma: \bar\ell \cap \Sigma^n$ which is by definition outside $\mathcal{L}(G)$, we can immediately rule out this state. Moreover, we can define a family of automata with strictly fewer states than the full LBH construction by making the following observation: if we can prove one edit must occur before the last $s$ tokens, we can rule out the last $s$ states absorbing editless trajectories.

\begin{theorem}
  $\varnothing = \mathcal{L}(\err\sigma_{1 \ldots (n-s)}\cdot\Sigma^s)\cap \mathcal{L}(G)$ implies the states $[q_{n-i, 0}]_{i \in 1\ldots s}$ are unnecessary.
\end{theorem}

Likewise, if we expend our entire edit budget in the first $p$ tokens, we will be unable to recover in a string where at least one repair must occur after the first $p$ tokens.

\begin{theorem}
  $\varnothing = \mathcal{L}(\Sigma^p\cdot\err\sigma_{p\ldots n})\cap \mathcal{L}(G)$ implies the states $[q_{i, d_{\max}}]_{i \in 0\ldots p}$ are unnecessary.
\end{theorem}

\noindent Therefore, we can eliminate $p+s$ states from $A$ by proving emptiness of $\mathcal{L}(\Sigma^p\cdot\err\sigma_{p\ldots (n-s)}\cdot\Sigma^s) \cap \mathcal{L}(G)$, without affecting $\mathcal{L}(G_\cap)$. For example, let us consider the pruned L-NFA for the broken string $\err\sigma = \texttt{[ ( + ) ]}$ with $G = \{S \rightarrow ( S ) \mid [ S ] \mid S + S \mid 1\}$. Its longest parseable suffix and prefix are:\\

\noindent(B.1)\phantom{..}\texttt{\_ \_ + ) ]}\phantom{.}$\not\in \mathcal{L}(G)$\phantom{...}\emoji{cross-mark}\phantom{...} $\land$ \phantom{...}\texttt{\_ \_ \_ ) ]}\phantom{...}$\in \mathcal{L}(G)$\phantom{...}\emoji{check-mark-button}\phantom{...}$\Longrightarrow [q_{n-i, 0}]_{i \in 1\ldots s}$ are unnecessary.\\
\noindent(B.2)\phantom{..}\texttt{[ ( + \_ \_}$\hspace{2pt}\not\in \mathcal{L}(G)$\phantom{...}\emoji{cross-mark}\phantom{...} $\land$ \phantom{...}\texttt{[ ( \_ \_ \_}\phantom{...}$\in \mathcal{L}(G)$\phantom{...}\emoji{check-mark-button}\phantom{...}$\Longrightarrow [q_{i, d_{\max}}]_{i \in 0\ldots p}$ are unnecessary.\\

\noindent Now we can prune the top leftmost and bottom rightmost states. Pictorially, this looks as follows:

\begin{figure}[H]
  \resizebox{0.47\textwidth}{!}{
    \input{figures/original_nfa}
  }
  \resizebox{0.47\textwidth}{!}{
    \begin{tikzpicture}[
%->, % makes the edges directed
>=stealth',
node distance=2.5cm, % specifies the minimum distance between two nodes. Change if necessary.
%  every state/.style={thick, fill=gray!10}, % sets the properties for each ’state’ node
initial text=$ $, % sets the text that appears on the start arrow
]
\node[state, initial]                (00) {$q_{0,0}$};
\node[state, right of=00]            (10) {$q_{1,0}$};
\node[accepting, state, right of=10] (20) {$q_{2,0}$};
\phantom{\node[accepting, state, right of=20] (30) {$q_{3,0}$};}
\phantom{\node[accepting, state, right of=30] (40) {$q_{4,0}$};}
\phantom{\node[accepting, state, right of=40] (50) {$q_{5,0}$};}

\node[state, above of=00, shift={(-2cm,0cm)}] (01) {$q_{0,1}$};
\node[state, right of=01]                          (11) {$q_{1,1}$};
\node[state, right of=11]                          (21) {$q_{2,1}$};
\node[accepting, state, right of=21]               (31) {$q_{3,1}$};
\node[accepting, state, right of=31]               (41) {$q_{4,1}$};
\node[accepting, state, right of=41]               (51) {$q_{5,1}$};

\node[state, above of=01, shift={(-2cm,0cm)}] (0j) {$q_{0,2}$};
\node[state, right of=0j]                          (1j) {$q_{1,2}$};
\node[state, right of=1j]                          (2j) {$q_{2,2}$};
\node[state, right of=2j]                          (3j) {$q_{3,2}$};
\node[accepting, state, right of=3j]               (4j) {$q_{4,2}$};
\node[accepting, state, right of=4j]               (5j) {$q_{5,2}$};

\phantom{\node[state, above of=0j, shift={(-2cm,0cm)}] (0k) {$q_{0,3}$};}
\phantom{\node[state, right of=0k]                         (1k) {$q_{1,3}$};}
\phantom{\node[state, right of=1k]                         (2k) {$q_{2,3}$};}
\node[state, right of=2k]                         (3k) {$q_{3,3}$};
\node[state, right of=3k]                         (4k) {$q_{4,3}$};
\node[accepting, state, right of=4k]              (5k) {$q_{5,3}$};

\draw [->] (00) edge[below] node{$\sigma_1$} (10);
\draw [->] (10) edge[below] node{$\sigma_2$} (20);
%        \draw [->] (20) edge[below] node{$\sigma_3$} (30);
%        \draw [->] (30) edge[below] node{$\sigma_4$} (40);
%        \draw [->] (40) edge[below] node{$\sigma_5$} (50);

\draw [->] (01) edge[below] node{$\sigma_1$} (11);
\draw [->] (11) edge[below] node[shift={(-0.2cm,0cm)}]{$\sigma_2$} (21);
\draw [->] (21) edge[below] node[shift={(-0.2cm,0cm)}]{$\sigma_3$} (31);
\draw [->] (31) edge[below] node[shift={(-0.2cm,0cm)}]{$\sigma_4$} (41);
\draw [->] (41) edge[below] node{$\sigma_5$} (51);

\draw [->] (0j) edge[below] node{$\sigma_1$} (1j);
\draw [->] (1j) edge[below] node{$\sigma_2$} (2j);
\draw [->] (2j) edge[below] node{$\sigma_3$} (3j);
\draw [->] (3j) edge[below] node{$\sigma_4$} (4j);
\draw [->] (4j) edge[below] node{$\sigma_5$} (5j);

%        \draw [->] (0k) edge[below] node{$\sigma_1$} (1k);
%        \draw [->] (1k) edge[below] node{$\sigma_2$} (2k);
%        \draw [->] (2k) edge[below] node{$\sigma_3$} (3k);
\draw [->] (3k) edge[below] node{$\sigma_4$} (4k);
\draw [->] (4k) edge[below] node{$\sigma_5$} (5k);

\draw [->] (00) edge[left] node{$\phantom{\cdot}$} (11);
\draw [->] (10) edge[left] node{$\phantom{\cdot}$} (21);
\draw [->] (20) edge[left] node{$\phantom{\cdot}$} (31);
%        \draw [->] (30) edge[left] node{$\phantom{\cdot}$} (41);
%        \draw [->] (40) edge[left] node{$\phantom{\cdot}$} (51);

% Super-knight arcs
\draw [->, red] (00) edge[bend right=8] node[east, shift={(-0.2cm,-0.7cm)}]{$\color{red}\sigma_3$}         (3j);
\draw [->, red] (10) edge[bend right=8] node[east, shift={(-0.2cm,-0.7cm)}]{$\color{red}\sigma_4$}         (4j);
\draw [->, red] (20) edge[bend right=8] node[east, shift={(-0.2cm,-0.7cm)}]{$\color{red}\sigma_5$}         (5j);

\draw [->, red] (01) edge[bend left=8] node[east, shift={(-0.2cm,-0.7cm)}]{$\color{red}\sigma_3$}         (3k);
\draw [->, red] (11) edge[bend left=8] node[east, shift={(-0.2cm,-0.7cm)}]{$\color{red}\sigma_4$}         (4k);
\draw [->, red] (21) edge[bend left=8] node[east, shift={(-0.2cm,-0.7cm)}]{$\color{red}\sigma_5$}         (5k);

\draw [->, violet] (00) edge node[east, shift={(-0.1cm,-0.8cm)}]{$\color{violet}\sigma_4$}  (4k);
\draw [->, violet] (10) edge node[east, shift={(-0.1cm,-0.8cm)}]{$\color{violet}\sigma_5$}  (5k);

\draw [->] (01) edge[left] node{$\phantom{\cdot}$} (1j);
\draw [->] (11) edge[left] node{$\phantom{\cdot}$} (2j);
\draw [->] (21) edge[left] node{$\phantom{\cdot}$} (3j);
\draw [->] (31) edge[left] node{$\phantom{\cdot}$} (4j);
\draw [->] (41) edge[left] node{$\phantom{\cdot}$} (5j);

%        \draw [->] (0j) edge[left] node{$\phantom{\cdot}$} (1k);
%        \draw [->] (1j) edge[left] node{$\phantom{\cdot}$} (2k);
\draw [->] (2j) edge[left] node{$\phantom{\cdot}$} (3k);
\draw [->] (3j) edge[left] node{$\phantom{\cdot}$} (4k);
\draw [->] (4j) edge[left] node{$\phantom{\cdot}$} (5k);

\draw [->] (00) edge[bend left=10, left] node{$\phantom{\cdot}$} (01);
\draw [->] (10) edge[bend left=10, left] node{$\phantom{\cdot}$} (11);
\draw [->] (20) edge[bend left=10, left] node{$\phantom{\cdot}$} (21);
%        \draw [->] (30) edge[bend left=10, left] node{$\phantom{\cdot}$} (31);
%        \draw [->] (40) edge[bend left=10, left] node{$\phantom{\cdot}$} (41);
%        \draw [->] (50) edge[bend left=10, left] node{$\phantom{\cdot}$} (51);

\draw [->] (01) edge[bend left=10, left] node{$\phantom{\cdot}$} (0j);
\draw [->] (11) edge[bend left=10, left] node{$\phantom{\cdot}$} (1j);
\draw [->] (21) edge[bend left=10, left] node{$\phantom{\cdot}$} (2j);
\draw [->] (31) edge[bend left=10, left] node{$\phantom{\cdot}$} (3j);
\draw [->] (41) edge[bend left=10, left] node{$\phantom{\cdot}$} (4j);
\draw [->] (51) edge[bend left=10, left] node{$\phantom{\cdot}$} (5j);

%        \draw [->] (0j) edge[bend left=10, left] node{$\phantom{\cdot}$} (0k);
%        \draw [->] (1j) edge[bend left=10, left] node{$\phantom{\cdot}$} (1k);
%        \draw [->] (2j) edge[bend left=10, left] node{$\phantom{\cdot}$} (2k);
\draw [->] (3j) edge[bend left=10, left] node{$\phantom{\cdot}$} (3k);
\draw [->] (4j) edge[bend left=10, left] node{$\phantom{\cdot}$} (4k);
\draw [->] (5j) edge[bend left=10, left] node{$\phantom{\cdot}$} (5k);

\draw [->, blue] (00) edge[bend right=11,below] node[shift={(0.5cm,0.3cm)}]{$\color{blue}\sigma_2$}    (21);
\draw [->, blue] (10) edge[bend right=11,below] node[shift={(0.5cm,0.3cm)}]{$\color{blue}\sigma_3$}    (31);
\draw [->, blue] (20) edge[bend right=11,below] node[shift={(0.5cm,0.3cm)}]{$\color{blue}\sigma_4$}    (41);
%        \draw [->, blue] (30) edge[bend right=11,below] node[shift={(0.5cm,0.3cm)}]{$\color{blue}\sigma_5$}    (51);

\draw [->, blue] (01) edge[bend right=3,below] node[shift={(0.3cm,0.2cm)}]{$\color{blue}\sigma_2$}    (2j);
\draw [->, blue] (11) edge[bend right=3,below] node[shift={(0.3cm,0.2cm)}]{$\color{blue}\sigma_3$}    (3j);
\draw [->, blue] (21) edge[bend right=3,below] node[shift={(0.3cm,0.2cm)}]{$\color{blue}\sigma_4$}    (4j);
\draw [->, blue] (31) edge[bend right=3,below] node[shift={(0.3cm,0.2cm)}]{$\color{blue}\sigma_4$}    (5j);

%        \draw [->, blue] (0j) edge[bend left=8,below] node[shift={(-0.45cm,-0.55cm)}]{$\color{blue}\sigma_2$}    (2k);
\draw [->, blue] (1j) edge[bend left=8,below] node[shift={(-0.45cm,-0.55cm)}]{$\color{blue}\sigma_3$}    (3k);
\draw [->, blue] (2j) edge[bend left=8,below] node[shift={(-0.45cm,-0.55cm)}]{$\color{blue}\sigma_4$}    (4k);
\draw [->, blue] (3j) edge[bend left=8,below] node[shift={(-0.45cm,-0.55cm)}]{$\color{blue}\sigma_5$}    (5k);

%https://tex.stackexchange.com/a/20986/139648
%        \draw [decorate,decoration={brace,amplitude=10pt,raise=10pt,mirror}] (00.south west) -- (50.south east) node[midway,yshift=-3em]{\textbf{String length}};
%        \draw [decorate,decoration={brace,amplitude=10pt,raise=20pt}] (00.south west) -- (0k.north west) node[midway,xshift=-1cm,yshift=-1cm,rotate=-54]{\textbf{Edit distance}};
\end{tikzpicture}
  }
  \caption{Levenshtein NFA before and after prefix and suffix pruning.}
\end{figure}\vspace{-0.175cm}

\section{Hyperparameter settings}\label{sec:hyperparams}

Below is a listing of the hyperparameter settings used for training the reranking model:

\begin{multicols}{3}
\begin{itemize}
\item Input dimension: 100
\item Encoder dimension: 512
\item Attention heads: 4
\item Encoder layers: 4
\item Vocab size, $|\Sigma|= 94$
\item Learning rate, $\alpha= 10^{-3}$
\item Temperature, $\tau= 10^{-1}$
\item Optimizer: AdamW
\item Negative rate: $10^{-2}$
\item Batch size: 8
\item Dropout: $10^{-1}$
\item Activation: GELU
\end{itemize}
\end{multicols}

The full parameters $\theta$ are partitioned into two sets, $\theta_e, \theta_r$, for the encoder and reranker layers. The encoder is pretrained on next-token prediction, then fine-tuned on the reranking task. During optimization, we use a smaller learning rate ($\alpha = 10^{-5}$) so as not to disturb the pretrained encoder parameters and a larger learning rate ($\alpha = 10^{-4}$) for the reranker parameters. We train the encoder for $2\times 10^4$ steps and the reranker for $1.3 \times 10^4$ steps, taking $\sim 4$ hours on an Nvidia H100 GPU.

\clearpage\section{Example Repairs}\label{sec:exaple_repairs}

Below, we provide a few representative examples of broken code snippets and the corresponding human repairs that were successfully ranked first by our method. On the left is a complete snippet fed to the model, and on the right, the corresponding human repair that was correctly predicted.

\begin{figure}[H]
\begin{tabular}{|m{6.6cm}|m{6.6cm}|}
\hline \rule{0pt}{2.5ex}\textbf{Original broken code}\rule[-1ex]{0pt}{2ex} &  \rule{0pt}{2.5ex}\textbf{First predicted repair}\rule[-1ex]{0pt}{2ex} \\\hline
\begin{smallpy}

(*@\hlorange{form}@*) sympy import *
x = Symbol('x', real=True)
x, re(x), im(x)

\end{smallpy} & \begin{smallpy}

(*@\hlorange{\textbf{from}}@*) sympy import *
x = Symbol('x', real=True)
x, re(x), im(x)

\end{smallpy} \\\hline
\begin{smallpy}

result = (*@\hlorange{yeald}@*) From(item.create())
raise Return(result)

\end{smallpy} & \begin{smallpy}

result = (*@\hlorange{\textbf{yield}}@*) From(item.create())
raise Return(result)

\end{smallpy} \\\hline
\begin{smallpy}

df.apply(lambda row: list(set(row['ids'(*@\hlorange{)}@*))))

\end{smallpy} & \begin{smallpy}

df.apply(lambda row: list(set(row['ids'(*@\hlorange{]}@*))))

\end{smallpy} \\\hline
%        \begin{lstlisting}[basicstyle=\ttfamily\lst@ifdisplaystyle\footnotesize\fi, language=python]
%
%  import numpy (*@\hlorange{ad}@*) np
%  A_concate = np.array([a_0, a_1, a_2,..., a_n])
%
%        \end{lstlisting} & \begin{lstlisting}[basicstyle=\ttfamily\lst@ifdisplaystyle\footnotesize\fi, language=python]
%
%  import numpy (*@\hlorange{\textbf{as}}@*) np
%  A_concate = np.array([a_0, a_1, a_2,..., a_n])
%
%        \end{lstlisting} \\\hline
\begin{smallpy}

sum(len(v) for v items.values())(*@\hlred{)}@*)

\end{smallpy} & \begin{smallpy}

sum(len(v) for v (*@\hlgreen{\textbf{in}}@*) items.values())

\end{smallpy} \\\hline
\begin{smallpy}

def average(values):
  if values == (1,2,3):
    return (1+2+3)/3
  (*@\hlorange{else}@*) (*@\hlred{if}@*) values == (-3,2,8,-1):
    return (-3+2+8-1)/4

\end{smallpy} & \begin{smallpy}

def average(values):
  if values == (1,2,3):
    return (1+2+3)/3
  (*@\hlorange{elif}@*) values == (-3,2,8,-1):
    return (-3+2+8-1)/4

\end{smallpy} \\\hline
\begin{smallpy}

dict = {
  "Jan": 1
  "January": 1
  "Feb": 2 # and so on
}

\end{smallpy} & \begin{smallpy}

dict = {
  "Jan": 1(*@\hlgreen{,}@*)
  "January": 1(*@\hlgreen{,}@*)
  "Feb": 2 # and so on
}

\end{smallpy} \\\hline
\begin{smallpy}

class MixIn(object)
  def m():
    pass

class classA(MixIn):

class classB(MixIn):

\end{smallpy} & \begin{smallpy}

class MixIn(object)(*@\hlgreen{:}@*)
  def m():
    pass

class classA(MixIn): (*@\hlgreen{\textbf{pass}}@*)

class classB(MixIn): (*@\hlgreen{\textbf{pass}}@*)

\end{smallpy} \\\hline
\end{tabular}
\end{figure}

\clearpage\section{Raw data}\label{sec:raw_prec_data}

Raw data from Precision@k experiments across snippet length and Levenshtein distance from \S~\ref{sec:rq2}. $|\err\sigma|$ indicates the snippet length and $\Delta$ indicates the Levenshtein distance between the broken and code and human fix computed over lexical tokens. For Tidyparse, we sample until exhausting the admissible set or a 10 second timeout is reached, whichever happens first, then rank the results. For the other models Precision@1, we sample one repair and report the percentage of repairs matching the human repair. For Precision@All, we report the percentage of repairs matching the human repair within the top 20,000 samples. Each entry in the following table represents a pairwise disjoint subset of $D_{\text{test}}$, with at least 50 distinct Python syntax errors and repairs matching the length and distance criteria, sampled uniformly from the full StackOverflow dataset~\cite{wong2019syntax}.

\begin{table}[!h]
\centering
\begin{tabular}{c|c|cccccccc}
\hline\hline
& $\Delta$ & \multicolumn{8}{c}{Precision@1} \\ \hline
$|\err\sigma|$ &  & $(0,10)$ & $[10,20)$ & $[20,30)$ & $[30, 40)$ & $[40,50)$ & $[50, 60)$ & $[60,70)$ & $[70, 80)$ \\ \hline
Tidyparse
& 1 & 0.37 & 0.52 & 0.44 & 0.40 & 0.38 & 0.34 & 0.43 & 0.27 \\
& 2 & 0.65 & 0.64 & 0.56 & 0.50 & 0.42 & 0.48 & 0.30 & 0.32 \\
& 3 & 0.21 & 0.15 & 0.12 & 0.13 & 0.13 & 0.18 & 0.15 & 0.10 \\ \hline
Seq2Parse
& 1 & 0.35 & 0.41 & 0.40 & 0.37 & 0.31 & 0.29 & 0.27 & 0.21 \\
& 2 & 0.12 & 0.13 & 0.14 & 0.12 & 0.11 & 0.11 & 0.10 & 0.12 \\
& 3 & 0.03 & 0.07 & 0.08 & 0.09 & 0.09 & 0.02 & 0.07 & 0.06 \\ \hline
BIFI
& 1 & 0.20 & 0.33 & 0.32 & 0.27 & 0.21 & 0.21 & 0.25 & 0.18 \\
& 2 & 0.18 & 0.18 & 0.21 & 0.19 & 0.19 & 0.18 & 0.11 & 0.11 \\
& 3 & 0.02 & 0.02 & 0.03 & 0.02 & 0.03 & 0.05 & 0.03 & 0.02 \\ \hline
& & \multicolumn{8}{c}{Precision@All} \\ \hline
Tidyparse
& 1 & 1.00 & 1.00 & 1.00 & 0.99 & 0.99 & 1.00 & 0.97 & 0.97 \\
& 2 & 1.00 & 0.99 & 0.98 & 1.00 & 1.00 & 1.00 & 0.94 & 0.90 \\
& 3 & 1.00 & 0.98 & 0.80 & 0.70 & 0.55 & 0.42 & 0.42 & 0.31 \\ \hline
BIFI
& 1 & 0.65 & 0.67 & 0.70 & 0.65 & 0.60 & 0.62 & 0.60 & 0.64 \\
& 2 & 0.52 & 0.41 & 0.37 & 0.32 & 0.27 & 0.27 & 0.21 & 0.24 \\
& 3 & 0.20 & 0.13 & 0.08 & 0.17 & 0.15 & 0.18 & 0.17 & 0.07 \\ \hline\hline
\end{tabular}
\end{table}

\section{Symbols at a glance}

Below we provide an inexhaustive listing of some common notation used throughout this paper.

\begin{table}[!h]
\centering
\begin{tabular}{c|l}
\hline
Notation & Meaning \\ \hline
$G=\langle \Sigma, V, P,S \rangle$ & CFG with terminals, $\Sigma$, nonterminals, $V$, productions, $P$, and start symbol $S$. \\
$A=\langle Q,\Sigma, \delta, q_\alpha, F \rangle$ & Automaton with states, $Q$, transitions, $\delta$, start state, $q_\alpha$ and final states, $F$. \\
$\err\sigma$ & Syntactically invalid input string with a known target language. \\
$|\sigma|$ & Length (number of terminals) of string, $\sigma$. \\
$G^*$ & Chomsky Normal Form (CNF) grammar. \\
$G_\cap$ & Intersection grammar formed by intersecting an automaton with a CFG. \\
$\ell_\cap, \mathcal{L}(G_\cap)$ & Intersection language generated by some $G_\cap$. \\
$L(\err\sigma, k)$ & Levenshtein automaton of radius $k$ for a broken string, $\err\sigma$. \\
${\color{orange}[\ldots]}$ & Orange text is related to the nominal predicate in the Levenshtein automaton.\\
$d_{\max}$ & Maximum permitted Levenshtein edit distance (repair radius). \\
$M$ & Matrix encoding the product construction $\mathcal{L}(G)\cap\mathcal{L}\big(L(\err\sigma, k)\big)$. \\
$P@k$ & Precision at rank $k$ evaluation metric.\\
\end{tabular}\vspace{-2cm}
\end{table}


\end{document}